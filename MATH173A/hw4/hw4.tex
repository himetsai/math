\documentclass{article}

\usepackage{fancyhdr}
\usepackage{extramarks}
\usepackage{amsmath}
\usepackage{amsthm}
\usepackage{amsfonts}
\usepackage{tikz}
\usepackage[plain]{algorithm}
\usepackage{algpseudocode}
\usepackage{enumerate}
\usepackage{amssymb}
\usepackage{mathtools}

\usetikzlibrary{automata,positioning}

%
% Basic Document Settings
%

\topmargin=-0.45in
\evensidemargin=0in
\oddsidemargin=0in
\textwidth=6.5in
\textheight=9.0in
\headsep=0.25in

\linespread{1.1}

\pagestyle{fancy}
\lhead{\hmwkAuthorName}
\chead{\hmwkClass:\ \hmwkTitle}
\rhead{\firstxmark}
\lfoot{\lastxmark}
\cfoot{\thepage}

\renewcommand\headrulewidth{0.4pt}
\renewcommand\footrulewidth{0.4pt}

\setlength\parindent{0pt}
\setlength{\parskip}{5pt}

%
% Create Problem Sections
%

\newcommand{\enterProblemHeader}[1]{
    \nobreak\extramarks{}{Problem \arabic{#1} continued on next page\ldots}\nobreak{}
    \nobreak\extramarks{Problem \arabic{#1} (continued)}{Problem \arabic{#1} continued on next page\ldots}\nobreak{}
}

\newcommand{\exitProblemHeader}[1]{
    \nobreak\extramarks{Problem \arabic{#1} (continued)}{Problem \arabic{#1} continued on next page\ldots}\nobreak{}
    \stepcounter{#1}
    \nobreak\extramarks{Problem \arabic{#1}}{}\nobreak{}
}

\setcounter{secnumdepth}{0}
\newcounter{partCounter}
\newcounter{homeworkProblemCounter}
\setcounter{homeworkProblemCounter}{1}
\nobreak\extramarks{Problem \arabic{homeworkProblemCounter}}{}\nobreak{}

%
% Homework Problem Environment
%
% This environment takes an optional argument. When given, it will adjust the
% problem counter. This is useful for when the problems given for your
% assignment aren't sequential. See the last 3 problems of this template for an
% example.
%
\newenvironment{homeworkProblem}[1][-1]{
    \ifnum#1>0
        \setcounter{homeworkProblemCounter}{#1}
    \fi
    \section{Problem \arabic{homeworkProblemCounter}}
    \setcounter{partCounter}{1}
    \enterProblemHeader{homeworkProblemCounter}
}{
    \exitProblemHeader{homeworkProblemCounter}
}

%
% Homework Details
%   - Title
%   - Due date
%   - Class
%   - Section/Time
%   - Instructor
%   - Author
%

\newcommand{\hmwkTitle}{Homework\ \#4}
\newcommand{\hmwkDueDate}{Nov 10, 2024}
\newcommand{\hmwkClass}{MATH 173A}
\newcommand{\hmwkClassInstructor}{Professor Cloninger}
\newcommand{\hmwkAuthorName}{\textbf{Ray Tsai}}
\newcommand{\hmwkPID}{A16848188}

%
% Title Page
%

\title{
    \vspace{2in}
    \textmd{\textbf{\hmwkClass:\ \hmwkTitle}}\\
    \normalsize\vspace{0.1in}\small{Due\ on\ \hmwkDueDate\ at 23:59pm}\\
    \vspace{0.1in}\large{\textit{\hmwkClassInstructor}} \\
    \vspace{3in}
}

\author{
  \hmwkAuthorName \\
  \vspace{0.1in}\small\hmwkPID
}
\date{}

\renewcommand{\part}[1]{\textbf{\large Part \Alph{partCounter}}\stepcounter{partCounter}\\}

%
% Various Helper Commands
%

% Useful for algorithms
\newcommand{\alg}[1]{\textsc{\bfseries \footnotesize #1}}

% For derivatives
\newcommand{\deriv}[1]{\frac{\mathrm{d}}{\mathrm{d}x} (#1)}

% For partial derivatives
\newcommand{\pderiv}[2]{\frac{\partial}{\partial #1} (#2)}

% Integral dx
\newcommand{\dx}{\mathrm{d}x}

% Probability commands: Expectation, Variance, Covariance, Bias
\newcommand{\Var}{\mathrm{Var}}
\newcommand{\Cov}{\mathrm{Cov}}
\newcommand{\Bias}{\mathrm{Bias}}
\newcommand*{\Z}{\mathbb{Z}}
\newcommand*{\Q}{\mathbb{Q}}
\newcommand*{\R}{\mathbb{R}}
\newcommand*{\C}{\mathbb{C}}
\newcommand*{\N}{\mathbb{N}}
\newcommand*{\prob}{\mathds{P}}
\newcommand*{\E}{\mathds{E}}

\begin{document}

\maketitle

\pagebreak

\begin{homeworkProblem}
	\begin{enumerate}[(a)]
		\item Find an expression for the orthogonal projection of a point $x\in\mathbb{R}^n$ onto the convex set
		\begin{align*}
			B = \{z\in\mathbb{R}^n : 0\le z_i \le 1 \textnormal{ for
			each } i=1, ..., n\}.
		\end{align*}
		You need to show your work, and justify your answer. The expression can be written piecewise, and per dimension if it's easier / more compact. {\bf Hint:} It might be helpful to sketch $B$, when $n = 2$ (i.e., in 2 dimensions), and use the sketch to help you figure out what the projection should be.

		\begin{proof}
			For $x \in \R^n$, we need to find $\Pi_B(x) = \underset{z \in B}{\arg\min} \|z - x\| = \underset{z \in B}{\arg\min} \sum_i (z_i - x_i)^2$. Notice that we may decouple this minimization problem across $n$ dimension by minimizing each $z_i$ independently. That is, for all $i$
			\[
				z_i = \underset{a \in [0, 1]}{\arg\min} (a - x_i)^2 = \begin{cases}
					0 & \textnormal{if } x_i < 0, \\
					x_i & \textnormal{if } 0 \le x_i \le 1, \\
					1 & \textnormal{if } x_i > 1.
				\end{cases} = \min(\max(0, x_i), 1).
			\]
		\end{proof}

		\item Let $f:\mathbb{R}^n\rightarrow\mathbb{R}$ be given by
		\begin{align*}
		f(x) = \|Ax\|_2^2 + a^T x
		\end{align*}
		where $A\in\mathbb{R}^{n\times n}$ is a positive definite matrix, and $a\in\mathbb{R}^n$. Write a projected gradient descent algorithm to solve
		\begin{align*}
		\min_{x\in\Omega} f(x)
		\end{align*}
		for $\Omega = B$, with $B$ from part (a). You do not need to specify the step size for this problem.

		\begin{proof}
			Note that
			\[
				\nabla f(x) = 2A^TAx + a,
			\]
			and thus the projected gradient descent algorithm is
			\[
				x^{(k+1)} = \Pi_\Omega\left(x^{(k)} - \mu \nabla f(x^{(k)})\right) = \Pi_B\left(x^{(k)} - \mu (2A^TAx^{(k)} + a)\right).
			\]
			More explicitly, for all $i$,
			\[
				x^{(k+1)}_i = \min\left(\max\left(0, x^{(k)}_i - 2\mu (A^TAx^{(k)} + a)_i\right), 1\right).
			\]
		\end{proof}

		\item Repeat part (b) but for $\Omega = B_2^n = \{z\in \mathbb{R}^n : \|z\|_2 \le 1\}$.
		\begin{proof}
			Notice
			\[
				\Pi_\Omega(x) = \begin{cases}
					\frac{x}{\|x\|_2} & \textnormal{if } \|x\| > 1, \\
					x & \textnormal{if } \|x\| \le 1.
				\end{cases}
			\]
			Hence, the projected gradient descent algorithm is
			\[
				x^{(k+1)} = \Pi_\Omega\left(x^{(k)} - \mu \nabla f(x^{(k)})\right) = \Pi_B\left((I - 2\mu A^TA)x^{(k)} - \mu a\right),
			\]
			which is $\frac{(I - 2\mu A^TA)x^{(k)} - \mu a}{\|(I - 2\mu A^TA)x^{(k)} - \mu a\|_2}$ if $\|x\| > 1$ and $(I - 2\mu A^TA)x^{(k)} - \mu a$ otherwise.
		\end{proof}
		\end{enumerate}
\end{homeworkProblem}

\newpage

\begin{homeworkProblem}
	Consider the \emph{hollow} sphere $S$ in $\mathbb{R}^n$, i.e., the set $S := \{x \in \mathbb{R}^n : \|x\|_2^2 = 1\}$. Consider the function $f:\mathbb{R}^n \rightarrow \mathbb{R}$ given by
	\begin{align*}
	f(x) = x^T Qx
	\end{align*}
	where $Q$ is an $n\times n$ symmetric matrix. For this problem you may use the fact that $\nabla f(x) = 2Qx$.

	\begin{enumerate}[(a)]
		\item For an arbitrary point $y\in\mathbb{R}^n$, $\Pi(y)$ be the projection of $y$ onto $S$. Find an expression for $\Pi(y)$ and give a short argument (i.e., proof) for why this is the correct expression. Make sure to handle the case $y = 0$ (i.e., the zero vector).

		\begin{proof}
			I claim that $\Pi(y) = \frac{y}{\|y\|_2}$ if $y \neq 0$ and $\Pi(0)$ can be any point in $S$. Note that the reverse triangle-inequality yields a lower bound
			\[
				\|x - y\|_2 \ge \left|\|x\|_2 - \|y\|_2\right| = \left|1 - \|y\|_2\right|,
			\]
			for $x \in \Omega$. Obvisouly, any $x \in \Omega$ achieves the lower bound when $y = 0$. Suppose $y \neq 0$. Obviously $\frac{y}{\|y\|_2} \in \Omega$. Since 
			\[
				\left\|\frac{y}{\|y\|_2} - y\right\| = \left\|\left(\frac{1}{\|y\|_2} - 1\right)y\right\| = \|y\|_2\left|\frac{1}{\|y\|_2} - 1\right| = |1 - \|y\|_2|
			\]
			achieves the lower bound, $\Pi(y) = \frac{y}{\|y\|_2}$.
		\end{proof}

		\item Is $S$ a convex set?

		\begin{proof}
			$S$ is not a convex set. Consider $x = (1, 0)$ and $y = (-1, 0)$. Then $0 = \frac{1}{2}(1, 0) + \frac{1}{2}(-1, 0) \notin S$.
		\end{proof}

		\item Write a projected gradient descent algorithm, with constant step size $\mu$, for
		\begin{align*}
		\min_{x\in\mathbb{R}^n} x^T Q x \qquad \textnormal{ subject
		to } \qquad \|x\|^2_2 = 1.
		\end{align*}

		\begin{proof}
			Note that $\nabla f(x) = 2Qx$, and thus the projected gradient descent algorithm is
			\[
				x^{(k+1)} = \Pi_S\left((I - 2\mu Q)x^{(k)}\right),
			\]
			which is equal to $\frac{(I - 2\mu Q)x^{(k)}}{\|(I - 2\mu Q)x^{(k)}\|}$ if $x^{(k)} \neq 0$ and any point in $S$ if $x^{(k)} = 0$.
		\end{proof}

		\item Is the projected gradient descent algorithm guaranteed to converge to the solution for small enough $\mu$? If not, can you give an example of $Q$ and an initialization $x^{(0)}$ where the algorithm won't converge?

		\begin{proof}
			Fix $\mu \in (0, 0.5)$. Consider $Q = \text{diag}(1, 0)$ and $x^{(0)} = (1, 0)$. Then
			\[
				x^{(k + 1)} = \Pi_S\left(\begin{bmatrix} 1 - 2\mu & 0 \\ 
						0 & 1 
				\end{bmatrix}x^{(k)}\right) = \frac{1}{\sqrt{(1 - 2\mu)^2(x^{(k)}_1)^2 + (x^{(k)}_2)^2}}\begin{bmatrix} (1 - 2\mu)x^{(k)}_1 \\ x^{(k)}_2 \end{bmatrix}.
			\] 
			
			Since $x^{(0)}$ only have the first entry non-zero, $x_2^{(k)} = 0$ for all $k$ by induction and thus
			\[
				x^{(k + 1)} = \frac{1}{(1 - 2\mu)x^{(k)}_1}\begin{bmatrix} (1 - 2\mu)x^{(k)}_1 \\ 0 \end{bmatrix} = \begin{bmatrix} 1 \\ 0 \end{bmatrix}.
			\]
			But then $f\left(\begin{bmatrix} 1 \\ 0 \end{bmatrix}\right) = 1$ and $f\left(\begin{bmatrix} 0 \\ 1 \end{bmatrix}\right) = 0$, so the algorithm fails to converge to a minimum.
		\end{proof}
	\end{enumerate}
\end{homeworkProblem}
\end{document}