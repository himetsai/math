\documentclass[addpoints, 11pt]{exam}
\setlength{\headsep}{0.25in}
\setlength{\unitlength}{1in}
%
\pagestyle{head}
%
\usepackage[utf8]{inputenc} %use Unicode
\usepackage[T1]{fontenc} %European fonts

\usepackage{%
	amsmath,       %some math tools
	amssymb,       %math symbols
	graphicx,      %enhanced graphics options
	mathtools,     %extension of amsmath
	microtype,     %small typographic effects
	bm,            %bold math symbols
	% todonotes,   %adds the option \todo{...} (use fixme instead)
	stmaryrd,      %some more math symbolswork
	% nicematrix,    %nicer matrix controls
	mathrsfs,      %more math fonts
        dsfont,
}
\usepackage{url}
\usepackage{hyperref}
%
\usepackage{amsthm}
% \newtheorem*{thm11.1.7}{Theorem 11.1.7}
%
\usepackage[shortlabels]{enumitem}
\usepackage{nicematrix}
\usepackage{multicol}
%
\usepackage[normalem]{ulem}
%
\newcommand{\myCourseNumber}{Math 100A}
\newcommand{\mySectionTime}{A02 5:00PM - 5:50PM}
\newcommand{\mySectionLeader}{Castellano-Macias}
\newcommand{\myName}{Ray Tsai}
\newcommand{\myID}{A16848188}
\newcommand{\myProfessor}{Professor McKernan}
\newcommand{\mySource}{Source Consulted: Textbook, Lecture, Discussion}
\newcommand{\myHmwkNumber}{0}
% \newcommand{\myExamVersion}{}

\newcommand*{\Z}{\mathbb{Z}}
\newcommand*{\Q}{\mathbb{Q}}
\newcommand*{\R}{\mathbb{R}}
\newcommand*{\C}{\mathbb{C}}
\newcommand*{\N}{\mathbb{N}}
\newcommand*{\prob}{\mathds{P}}
\newcommand*{\E}{\mathds{E}}

\definecolor{crimson}{rgb}{0.86,0.08,0.24}

\setlength{\parindent}{0pt}
\setlength{\parskip}{5pt}

\newenvironment{question}[1]{\smallskip\noindent\color{crimson}{\bf Question #1.}}{}
\allowdisplaybreaks[1]

%% define absolute value \abs{...}:
\DeclarePairedDelimiterX\abs[1]\lvert\rvert{%
  \ifblank{#1}{\:\cdot\:}{#1}
}
% define norm \norm{...}:
\DeclarePairedDelimiterX\norm[1]\lVert\rVert{%
  \ifblank{#1}{\:\cdot\:}{#1}
}
% define inner product \inner{...}{...}:
\DeclarePairedDelimiterX{\inner}[2]{\langle}{\rangle}{%
  \ifblank{#1}{\:\cdot\:}{#1},\ifblank{#2}{\:\cdot\:}{#2}
}

% define \set{...} to write sets and \given to write \set{... \given ...} for {...|...}
\newcommand*\setSymbol[1][]{
  \nonscript\:#1\vert\allowbreak\nonscript\:\mathopen{}
}
\providecommand\given{}
\DeclarePairedDelimiterX\set[1]{\lbrace}{\rbrace}{
  \renewcommand*\given{\setSymbol[\delimsize]}
  #1
}

% free group geneated by ... \free{...} or \free{... \given ...}
\DeclarePairedDelimiterX\free[1]{\langle}{\rangle}{
  \renewcommand\given{\nonscript\:\delimsize\vert\nonscript\:
    \mathopen{}}
  #1}

% define \lopen{...}{...}, \ropen{...}{...}, \open{...}{...}, \closed{...}{...} for intervals
\DeclarePairedDelimiterX\open[2](){#1,#2}
\DeclarePairedDelimiterX\lopen[2](]{#1,#2}
\DeclarePairedDelimiterX\ropen[2][){#1,#2}
\DeclarePairedDelimiterX\closed[2][]{#1,#2}

\NiceMatrixOptions{cell-space-limits = 1pt}
\newcommand*{\pmat}[1]{\begin{pNiceMatrix} #1 \end{pNiceMatrix}}
\newcommand*{\dfdx}[2]{\frac{\partial #1}{\partial #2}}

\DeclareMathOperator{\vol}{vol}
%
\pointsinmargin
\pointpoints{\thinspace point}{points}
\marginpointname{ \points}
%
\begin{document}
%
\firstpageheader{\bfseries \myCourseNumber \\ \smallskip \scriptsize \mySectionTime \\ \mySectionLeader}{\bfseries Homework \myHmwkNumber \\ \smallskip \scriptsize \mySource}{\bfseries \myName \\ \smallskip \scriptsize \myID \\ \myProfessor}
%
\runningheader{}{(page \textit{\thepage}\ of \textit{\numpages})}{}
%
%

\begin{question}{1.1.1}
    Let $S$ be a set having an operation $*$ which assigns an element $a * b$ of $S$ for any $A,B \in S$. Let us assume that the following two rules hold:
    \begin{enumerate}
        \item If $a,b$ are any objects in $S$, then $a * b = a$.
        \item If $a,b$ are any objects in $S$, then $a * b = b * a$.
    \end{enumerate}
    Show that $S$ can only have at most one object.
\end{question}

\begin{proof}
    Suppose for the sake of contradiction that $S$ has more than one object. Let $a, b \in S$ be two distinct objects. By rule one, $a * b = a$ and $b * a = b$, contradicting rule 2's statement that $a * b = b * a$. Therefore, $S$ has at most one object.
\end{proof}

\newpage

\begin{question}{1.1.2}
    Let $S$ be the set of all integers $0, \pm 1, \pm 2, \dots, \pm n, \dots$. For $a, b$ in $S$ define $*$ by $a * b = a - b$. Verify the following:
\end{question}

\begin{enumerate}[(a)]
    \color{crimson}
    \item  $a * b \neq b * a$ unless $a = b$.
    \normalcolor
    
    \begin{proof}
        True. Let $a, b \in S$ such that $a \neq b$. Then $a + a \neq b + b$, and so $a * b = a - b \neq b - a = b * a$.
    \end{proof}

    \color{crimson}
    \item  $(a * b) * c \neq a * (b * c)$ in general. Under what condition on $a, b, c$ is $(a * b) * c = a * (b * c)$?
    \normalcolor
    
    \begin{proof}
        True. Let $a, b, c \in S$.
        \[
            (a * b) * c = (a - b) - c = a - b - c \neq a - b + c = a - (b - c) = a * (b * c).
        \]
        Suppose that $(a * b) * c = a * (b * c)$. 
        \begin{align*}
            (a * b) * c &= a * (b * c) \\
            a - b - c &= a - b + c \\
            c &= 0.
        \end{align*}
        Only when $c = 0$ is $(a * b) * c = a * (b * c)$.
    \end{proof}

    \color{crimson}
    \item  The integer $0$ has the property that $a * 0 = a$ for every $a$ in $S$.
    \normalcolor
    
    \begin{proof}
        True. $a * 0 = a - 0 = a$.
    \end{proof}

    \color{crimson}
    \item  For $a$ in $S$, $a * a = 0$.
    \normalcolor
    
    \begin{proof}
        True. $a * a = a - a = 0$.
    \end{proof}
\end{enumerate}

\newpage

\begin{question}{2.1.1}
    Determine if the following sets $G$ with the operation indicated form a group. If not, point out which of the group axioms fail.
\end{question}

\begin{enumerate}[(a)]
    \color{crimson}
    \item  $G = $ the set of all integers, $a * b = a - b$.
    \normalcolor
    
    \begin{proof}
        Fails the associative property. Let $a,b,c \in \Z$.
        \[
            a * (b * c) = a - (b - c) = a - b + c \neq (a - b) - c = (a * b) * c
        \]
    \end{proof}

    \color{crimson}
    \item  $G = $ the set of all integers, $a * b = a + b + ab$.
    \normalcolor
    
    \begin{proof}
        Fails the inverse property. Let $a, b \in \Z$. Since $a * 0 = 0 * a = a$, we know the identity element of $G$ is $0$. Let $a = 1$. Since $a * b = b * a = 1 + b + b = 1 + 2b = 0$ has no integer solutions, $(G, *)$ does not fulfill the inverse property.
    \end{proof}

    \color{crimson}
    \item  $G = $ the set of non-negative integers, $a * b = a + b$.
    \normalcolor
    
    \begin{proof}
        Fails the inverse property. We know the identity element $e \in G$ is $0$, as $s + 0 = 0 + s = s$ for any $s \in G$. For $a, b \in G$ such that $a \neq 0$, since $a + b > 0$, any positive element in $G$ has no inverse.
    \end{proof}

    \color{crimson}
    \item  $G = $ the set of all rational numbers $\neq -1$, $a * b = a + b + ab$.
    \normalcolor
    
    \begin{proof}
        $(G, *)$ forms a group. Let $a, b, c \in G$.

        We first prove the closed property. We know $a + b + ab \in \Q$. Suppose for the sake of contradiction that $a + b + ab = -1$. Rearranged, we get $(a + 1)b = -(a + 1)$. Since $a \neq -1$, we cancel $(a + 1)$ from each side and get $b = -1$, contradiction. Therefore, $a + b + ab \in G$.

        The associative property is met, as
        \begin{align*}
            (a * b) * c
            &= (a + b + ab) * c \\
            &= a + b + c + ab + ac + bc + abc \\
            &= a + (b + c + bc) + a(b + c + bc) \\
            &= a * (b * c).
        \end{align*}

        Since $a * 0 = 0 * a = a$, $e = 0 \in G$ is the identity element.

        Finally, we show the inverse property. Let $b = \frac{-a}{a + 1}$. Since 
        \[
            a * b = b * a = a + \frac{-a}{a + 1} + a \cdot \frac{-a}{a + 1} = \frac{a^2 + a - a - a^2}{a + 1} = 0,
        \]
        for all $a \in G$, $a$ has an inverse $b = \frac{-a}{a + 1} \in G$.

        Since all four properties are met, $G$ with $*$ form a group.
    \end{proof}

    \color{crimson}
    \item  $G = $ the set of all rational numbers with denominator divisible by $5$ (written so that numerators and denominator are relatively prime), $a * b = a + b$.
    \normalcolor
    
    \begin{proof}
        Fails the identity property. Suppose for the sake of contradiction that there exists $e \in G$ such that $a * e = a + e = a$. Then $e = 0$. However, $0 \notin G$, as the numerator $0$ is not relatively prime to any integer denominators divisible by $5$, contradiction.
    \end{proof}

    \color{crimson}
    \item  $G$ is the set having more than one element, $a * b = a$ for all $a, b \in G$.
    \normalcolor
    
    \begin{proof}
        Fails the identity property. Let $a, e \in G$ be two distinct elements. Suppose for the sake of contradiction that $e$ is the identity element. We then have $e * a = e \neq a$, contradiction.
    \end{proof}
\end{enumerate}

\newpage

\begin{question}{2.1.2}
    In the group $G$ defined in Example 6, show that the set $H = \{T_{a,b} \, | \, a = \pm 1, b \text{ any real }\}$ forms a group under the $*$ of $G$.
\end{question}

\begin{proof}
    We prove all four properties of a group.
    
    \textbf{Closed property:} Let $T_{a, b}, T_{c, d} \in H$. We then have
    \[
        T_{a, b} * T_{c, d} = T_{ac, ad + b}.
    \]
    Since $ac = \pm 1$ and $ad + b \in \R$, $f * g \in H$.

    \textbf{Associative property:} Let $f,g,h \in H$. Since all three functions are $\R \rightarrow \R$, we get $(f * g) * h = f * (g * h)$ by lemma 1.3.1 in Herstein.

    \textbf{Identity property:} For all $T_{a, b} \in H$, we have $T_{1, 0}$ such that
    \begin{gather*}
        T_{a, b} * T_{1, 0} = T_{a, b}, \\
        T_{1, 0} * T_{a, b} = T_{a, b},
    \end{gather*}
    and thus $G$ has an identity element $T_{1, 0}$ under $*$.

    \textbf{Inverse property:} For all $T_{a, b} \in H$, we have $T_{a, -a^{-1}b} \in H$, such that 

    \begin{gather*}
        T_{a, b} * T_{a, a^{-1}b} = T_{a^2, a \cdot a^{-1}b + b} = T_{1, 0} \\
       T_{a, a^{-1}b * T_{a, b}}  = T_{a^2, a^{-1}b \cdot a + b} = T_{1, 0}.
    \end{gather*}

    Since all four properties are fulfilled, $H$ forms a group under the $*$ of $G$.
\end{proof}

\newpage

\begin{question}{2.1.5}
    in Example 9, prove that $g * f = f * g^{-1}$, and that $G$ is a group, is non-abelian, and is order of $8$.
\end{question}

\begin{proof}
    We restate that $S = \{(x,y) \in \R^2 \}$, $f, g \in A(S)$ such that $f(x, y) = (-x, y)$ and $g(x, y) = (-y, x)$, and $G = \{f^ig^j \, | \, i = 0, 1; j = 0, 1, 2, 3 \}$. Note that since $f$ is a reflection and $g$ is a $90^{\circ}$ rotation, both $f^k = f^{(k \mod 2)}$ and $g^l = g^{(l \mod 4)}$ are in $G$, for $k,l \in \Z$. And also note that $g^4 = f^2 = $ identity mapping $e$.
    
    We first prove that $g * f = f * g^{-1}$. We first note that since $e = g^4$, $g^{-1} = g^3 = (y, -x)$. On the left-hand side of the statement, we have
    \[
        (g * f)(x, y) = g(f(x, y)) = g(-x, y) = (-y, -x).
    \]
    On the right-hand side, we have
    \[
        (f * g^{-1})(x, y) = f(g^3(x,y)) = f(y, -x) = (-y, -x).
    \]
    Thus, we have $g * f = f * g^{-1} = (-y, -x)$.

    We now show that $G$ fulfills the 4 properties of a group. Let $a, b, c \in G$.

    \textbf{Associative property:} Since $a, b, c$ are all $S \rightarrow S$, we get $(a * b) * c = a * (b * c)$ by lemma 1.3.1 in Herstein.

    

    \textbf{Closed property:} We first show $g^nf = fg^{-n}$ by induction. The base case $gf = fg^{-1}$ is done above. For $n > 1$, $g^nf = gfg^{-(n - 1)}$. By the associative property, we get
    \begin{gather}
        g^nf = (gf)g^{-(n - 1)} = fg^{-n}.
    \end{gather}

    We now show that for $i,j,k,l \in \Z$, $f^ig^jf^kg^l = f^{i + k}g^{(-1)^kj + l}$. If $k$ is even, then $f^ig^jf^kg^l = f^ig^{j + l}$. If $k$ is odd, then $f^ig^jf^kg^l = f^i(g^jf)g^l = f^{i + 1}g^{-j + l}$, by $(1)$. Combining two cases, we get a generalized equality
    \begin{gather}
        f^ig^jf^kg^l = f^{i + k}g^{(-1)^kj + l}.
    \end{gather}
    
    Finally, we show that $G$ is closed under $*$. Let $a = f^ig^j, b = f^kg^l \in G$. Then,
    \begin{align*}
        (a * b)(x, y)
        &= (f^ig^jf^kg^l)(x, y) \\
        &= (f^{i + k}g^{(-1)^kj + l})(x, y) && \text{by } (2) \\
        &= f^{i + k}(g^{(-1)^kj + l \mod 4}(x, y)) \\
        &= (f^{i + k \mod 2}g^{(-1)^kj + l \mod 4})(x, y) \in G
    \end{align*}

    \textbf{Identity property:} Let $a = f^ig^j, e = g^4 = f^2 \in G$. Then, we have
    \begin{gather*}
        a * e = f^ig^{j + 4} = f^ig^j = a, \\
        e * a = f^{i + 2}g^j = f^ig^j = a.
    \end{gather*}
    Thus, $G$ has $e$ as the identity element under $*$.

    \textbf{Inverse property:} For all $a = f^ig^j \in G$, we have $b = f^{i}g^{(-1)^ij}$ such that
    \begin{gather*}
        a * b = f^ig^j * f^{i}g^{(-1)^{i + 1}j} = f^{2i}g^{(-1)^ij + (-1)^{i + 1}j} = f^0g^0 = e, \\
        b * a = f^{i}g^{(-1)^{i + 1}j} * f^ig^j = f^{2i}g^{(-1)^{2i + 1}j + j} = f^0g^0 = e
    \end{gather*}
    by $(2)$. Thus, the inverse property holds. Since all four properties hold, $G$ is a group under $*$.

    We will prove that $G$ is a non-abelian group. Since 
    \[
        (f * g)(x, y) = f(g(x, y)) = f(-y, x) = (y, x),
    \]
    but
    \[
        (g * f)(x, y) = g(f(x, y)) = g(-x, y) = (-y, -x),
    \]
    we get that $f * g \neq g * f$. Thus, $G$ is a non-abelian group.

    Finally, we prove that $G$ is order of $8$. Since there are $2$ possible values for $i$ and $4$ possible values for $j$, $G$ has at most $8$ elements. We will show that each combination of $i, j$ leads to a distinct $f^ig^j$. Let $a = f^ig^j$, $b = f^kg^l$, for $i, k = 0 ,1$, $j, l = 0, 1, 2, 3$, and $i \neq k$ or $j \neq l$. Suppose for the sake of contradiction that $a = b$. Then
    \begin{align*}
        a &= b \\
        f^ig^j &= f^kg^l \\
        f^{-k}f^ig^jg^{-l} &= e \\
        f^{i - k}g^{j - l} &= e.
    \end{align*}
    However, since $i \neq k$ or $j \neq l$, $f^{i - k}g^{j - l} \neq e$, contradiction. Therefore, $G$ has an order of $8$.
\end{proof}

\newpage

\begin{question}{2.1.21}
    Show that a group of order $5$ must be abelian.
\end{question}

\begin{proof}
    Suppose for sake of contradiction that there exists a non-abelian group $G = \{e, f, g, h, j\}$ of order $5$ with $e$ as the identity element. Since $G$ is non-abelian, there exists a pair of elements, say $f, g$, such that $fg \neq gf$, where $fg, gf \in G$. $fg, gf \neq e$ as otherwise it would contradict the rule of inverse. And since $fg, gf \neq f, g$, we know $fg, gf$ must be the rest of the $2$ elements, namely $h, j$. Let $h = fg$ and $j = gf$. Thus, we can represent any non-abelian group of order $5$ in the form of $G = \{e, f, g, fg, gf\}$. Note that any non-abelian group of order $5$ can be represented in this form. Then,
    \begin{align*}
        f^2 &\neq f && \text{otherwise } f \neq e \\
        f^2 &\neq g && \text{otherwise } fg = fff = gf \\
        f^2 &\neq fg, gf && \text{otherwise } f = g \\
        fgf &\neq e && \text{otherwise } f(gf) = f(fg) = e \rightarrow gf = fg \\
        fgf &\neq f,fg,gf && \text{otherwise } e \text{ is not unique}
    \end{align*}
    Thus, $f^2 = e$ and $fgf = g$. However, $ffg = g = fgf$, and so $fg = gf$, contradiction. Therefore, a group of order $5$ must be abelian.
\end{proof}

\newpage

\begin{question}{2.1.23}
    In the group $G$ of Example 6, find all elements $U \in G$ such that $U * T_{a, b} = T_{a, b} * U$ for every $T_{a, b} \in G$.
\end{question}

\begin{proof}
    We will show that $U = T_{1,0}$ is the only solution. Let $m, n, a, b, c, d \in \R$. Suppose that $T_{m, n} * T_{a, b} = T_{a, b} * T_{m, n}$, and $T_{m, n} * T_{c, d} = T_{c, d} * T_{m, n}$. Then, for $r \in \R$, we have $mar + mb + n = amr + an + b$ and $mcr + md + n = cmr + cn + d$,  and thus we get the system of equations
    \[
        \begin{cases}
            bm + (1 - a)n = b \\
            dm + (1 - c)n = d.
        \end{cases}
    \]
    Suppose that $b = 0$, we get $n = 0$ from the first equation. Plugging $n = 0$ into the equation, we get $m = 1$. Suppose that $b \neq 0$, we solve the system and get $(\frac{b - cb - d + da}{b})n = 0$, and thus, in general, $n = 0$. Plugging $n = 0$ into equation $1$, we get $m = 1$. Therefore, $U = T_{1, 0}$.

\end{proof}

\end{document}
