\documentclass{article}

\usepackage{fancyhdr}
\usepackage{extramarks}
\usepackage{amsmath}
\usepackage{amsthm}
\usepackage{amsfonts}
\usepackage{tikz}
\usepackage[plain]{algorithm}
\usepackage{algpseudocode}

\usetikzlibrary{automata,positioning}

%
% Basic Document Settings
%

\topmargin=-0.45in
\evensidemargin=0in
\oddsidemargin=0in
\textwidth=6.5in
\textheight=9.0in
\headsep=0.25in

\linespread{1.1}

\pagestyle{fancy}
\lhead{\hmwkAuthorName}
\chead{\hmwkClass:\ \hmwkTitle}
\rhead{\firstxmark}
\lfoot{\lastxmark}
\cfoot{\thepage}

\renewcommand\headrulewidth{0.4pt}
\renewcommand\footrulewidth{0.4pt}

\setlength\parindent{0pt}
\setlength{\parskip}{5pt}

%
% Create Problem Sections
%

\newcommand{\enterProblemHeader}[1]{
    \nobreak\extramarks{}{Problem \arabic{#1} continued on next page\ldots}\nobreak{}
    \nobreak\extramarks{Problem \arabic{#1} (continued)}{Problem \arabic{#1} continued on next page\ldots}\nobreak{}
}

\newcommand{\exitProblemHeader}[1]{
    \nobreak\extramarks{Problem \arabic{#1} (continued)}{Problem \arabic{#1} continued on next page\ldots}\nobreak{}
    \stepcounter{#1}
    \nobreak\extramarks{Problem \arabic{#1}}{}\nobreak{}
}

\setcounter{secnumdepth}{0}
\newcounter{partCounter}
\newcounter{homeworkProblemCounter}
\setcounter{homeworkProblemCounter}{1}
\nobreak\extramarks{Problem \arabic{homeworkProblemCounter}}{}\nobreak{}

%
% Homework Problem Environment
%
% This environment takes an optional argument. When given, it will adjust the
% problem counter. This is useful for when the problems given for your
% assignment aren't sequential. See the last 3 problems of this template for an
% example.
%
\newenvironment{homeworkProblem}[1][-1]{
    \ifnum#1>0
        \setcounter{homeworkProblemCounter}{#1}
    \fi
    \section{Problem \arabic{homeworkProblemCounter}}
    \setcounter{partCounter}{1}
    \enterProblemHeader{homeworkProblemCounter}
}{
    \exitProblemHeader{homeworkProblemCounter}
}

%
% Homework Details
%   - Title
%   - Due date
%   - Class
%   - Section/Time
%   - Instructor
%   - Author
%

\newcommand{\hmwkTitle}{Homework\ \#3}
\newcommand{\hmwkDueDate}{October 19, 2023}
\newcommand{\hmwkClass}{MATH 100}
\newcommand{\hmwkClassTime}{Section A02 5:00PM - 5:50PM}
\newcommand{\hmwkSectionLeader}{Castellano}
\newcommand{\hmwkClassInstructor}{Professor McKernan}
\newcommand{\hmwkSource}{Source Consulted: Textbook, Lecture, Discussion}
\newcommand{\hmwkAuthorName}{\textbf{Ray Tsai}}
\newcommand{\hmwkPID}{A16848188}

%
% Title Page
%

\title{
    \vspace{2in}
    \textmd{\textbf{\hmwkClass:\ \hmwkTitle}}\\
    \normalsize\vspace{0.1in}\small{Due\ on\ \hmwkDueDate\ at 12:00pm}\\
    \vspace{0.1in}\large{\textit{\hmwkClassInstructor}} \\
    \vspace{0.1in}\small\hmwkClassTime \\
    \small Section Leader: \hmwkSectionLeader \\
    \vspace{0.1in}\small\hmwkSource \\
    \vspace{3in}
}

\author{
  \hmwkAuthorName \\
  \vspace{0.1in}\small\hmwkPID
}
\date{}

\renewcommand{\part}[1]{\textbf{\large Part \Alph{partCounter}}\stepcounter{partCounter}\\}

%
% Various Helper Commands
%

% Useful for algorithms
\newcommand{\alg}[1]{\textsc{\bfseries \footnotesize #1}}

% For derivatives
\newcommand{\deriv}[1]{\frac{\mathrm{d}}{\mathrm{d}x} (#1)}

% For partial derivatives
\newcommand{\pderiv}[2]{\frac{\partial}{\partial #1} (#2)}

% Integral dx
\newcommand{\dx}{\mathrm{d}x}

% Probability commands: Expectation, Variance, Covariance, Bias
\newcommand{\Var}{\mathrm{Var}}
\newcommand{\Cov}{\mathrm{Cov}}
\newcommand{\Bias}{\mathrm{Bias}}
\newcommand*{\Z}{\mathbb{Z}}
\newcommand*{\Q}{\mathbb{Q}}
\newcommand*{\R}{\mathbb{R}}
\newcommand*{\C}{\mathbb{C}}
\newcommand*{\N}{\mathbb{N}}
\newcommand*{\prob}{\mathds{P}}
\newcommand*{\E}{\mathds{E}}

\begin{document}

\maketitle

\pagebreak

\begin{homeworkProblem}
    “The union of two subgroups of a group $G$ is a subgroup of $G$.”
    True or False? If true then give a proof and if false then give a counterexample.

    \begin{proof}
      The statement is false. Consider the $D_3$, the groups of symmetries of a triangle, and its subgroups $\{I, F_1\}$, $\{I, F_2\}$,
      two cyclic subgroups of distinct flips. 
      Since $F_1F_2 = R$, their union $\{I, F_1, F_2\}$ is not closed under the operation of $G$, and thus it's not a subgroup.
    \end{proof}
\end{homeworkProblem}

\newpage

\begin{homeworkProblem}

    Verify that the relation $\sim$ is an equivalence relation on the set $S$ given.

    \begin{enumerate}
        \item[(b)] $S = \C$, the complex numbers, $a \sim b$ if $|a| = |b|$.
        
        \begin{proof}
            We check each property of a equivalence relation.

            \textbf{Reflexivity:} $|a| = |a|$, and so $a \sim a$, trivial.

            \textbf{Symmetry:} Suppose that $|a| = |b|$, then $|b| = |a|$. Thus, $a \sim b$ imples $b \sim a$.

            \textbf{Transitivity:} Suppose that $a \sim b$ and $b \sim c$. Then $|a| = |b| = |c|$, and so $a \sim c$.

            Thus, $\sim$ is an equivalence relation. Its equivalence classes are sets of complex numbers of the same distance to the origin,
            namely, circles of different radius centering the origin on the complex plane.
        \end{proof}

        \item[(c)] $S =$ straight lines in the plane, $a \sim b$ if $a, b$ are parallel.
        
        \begin{proof}
            We again check each property of a equivalence relation.

            \textbf{Reflexivity:} $a$ is paralled to itself so $a \sim a$.

            \textbf{Symmetry:} Suppose that $a \sim b$. Then $a, b$ are parallel to each other, and so $b \sim a$. 

            \textbf{Transitivity:} Suppose that $a \sim b$ and $b \sim c$. Let $s$ be the slope of $b$.
            Since $a \sim b$ and $b \sim c$, the slope of $a, b, c$ are all $s$, and so $a \sim c$.

            Thus, $\sim$ is an equivalence relation. Its equivalence classes are sets of straight lines with the same slope.
        \end{proof}
    \end{enumerate}
\end{homeworkProblem}

\newpage

\begin{homeworkProblem}
    For each subgroup of $D_4$, list all the left and right cosets.  (Since $D_4$
    has many subgroups, it is only necessary to do this up to the obvious
    symmetries)

    \begin{proof}
        The left and right cosets of $\{I\}$ are all the sets that only contain a non-identity element in $D_4$.

        The left and right cosets of $D_4$ is $D_4$ itself.

        Since $\{I, R_1, R_2, R_3\}$ contains $4$ elements, by the Lagrange Theorem, 
        the only possible left/right cosets of it is $\{I, R_1, R_2, R_3\}$ itself and the rest of the elements $\{F_1, F_2, F_3, F_3\}$, namely, all of the flips.

        For $\{I, F_1\}$, its left cosets are $\{I, F_1\}, \{F_2, R^2\}, \{F_3, R\}, \{F_4, R^3\}$, while
        while the right cosets are \\ $\{I, F_1\}, \{F_2, R^2\}, \{F_3, R^3\}, \{F_4, R\}$.

        For $\{I, F_2\}$, its left cosets are $\{I, F_2\}, \{F_1, R^2\}, \{F_3, R^3\}, \{F_4, R\}$, while
        while the right cosets are \\ $\{I, F_2\}, \{F_1, R^2\}, \{F_3, R\}, \{F_4, R^3\}$.

        For $\{I, F_3\}$, its left cosets are $\{I, F_3\}, \{F_1, R^3\}, \{F_2, R\}, \{F_4, R^2\}$, while
        while the right cosets are \\ $\{I, F_3\}, \{F_1 R\}, \{F_2, R^3\}, \{F_4, R^2\}$.

        For $\{I, F_4\}$, its left cosets are $\{I, F_4\}, \{F_1, R\}, \{F_2, R^3\}, \{F_4, R^2\}$, while
        while the right cosets are \\ $\{I, F_4\}, \{F_1 R^3\}, \{F_2, R\}, \{F_4, R^2\}$.

        For $\{I, R^2\}$, its left and right cosets are both $\{I, R^2\}, \{R, R^3\}, \{F_1, F_2\}, \{F_3, F_4\}$.
    \end{proof}
\end{homeworkProblem}

\newpage

\begin{homeworkProblem}
    In $\Z_{16}$, write down all the cosets of the subgroup $H = \{[0], [4], [8], [12]\}$.

    \begin{proof}  
        \begin{align*}
            [0] + H &= H \\
            [1] + H &= \{[1], [5], [9], [13]\} \\
            [2] + H &= \{[2], [6], [10], [14]\} \\
            [3] + H &= \{[3], [7], [11], [15]\}
        \end{align*}
        Since $[4] + H = [0] + H = H$, $[a] + H$ repeats the above listed cosets, for all $a \geq 4$.

        Thus, we have obtained all cosets of $H$.
    \end{proof}
\end{homeworkProblem}

\newpage

\begin{homeworkProblem}
    In problem $4$, what is the index of $H$ in $\Z_{16}$?

    \begin{proof}
        As listed in above question, there are $4$ left/right cosets of $H$, and thus $[\Z_{16};H] = 4$.
    \end{proof}
\end{homeworkProblem}

\newpage

\begin{homeworkProblem}
    If $aH$ and $bH$ are distinct left cosets of $H$ in $G$,
    are $Ha$ and $Hb$ distinct right cosets of $H$ in $G$?

    \begin{proof}
        No. Consider $D_4$'s subgroup $H = \{I, F_1\}$. 
        From problem 3, we know $F_3H = \{F_3, R\}$ and $R^3H = \{F_4, R_3\}$ are distinct cosets.
        However $HF_3 = \{F_3, R^3\} = HR^3$ are the same. Thus the statement is disproved.
    \end{proof}
\end{homeworkProblem}

\newpage

\begin{homeworkProblem}
    If $G$ is a finite abelian group and $a_1, \dots, a_n$ are all elements,
    show that $x = a_1a_2 \dots a_n$ must satisfy $x^2 = e$.

    \begin{proof}
        We first prove that for all $k \geq 1$, $\Pi_{1 \leq j \leq k} a_j = a_k\Pi_{1 \leq j < k} a_j$.
        Let $y = \Pi_{1 \leq j \leq k} a_j \in G$. Since $G$ is abelian, 
        \begin{gather}
            \Pi_{1 \leq j \leq k} a_j = ya_k = a_ky = a_k\Pi_{1 \leq j < k} a_j.
        \end{gather}

        We now prove that we can rearrange $x = a_1a_2 \dots a_n$ into any ordering by induction on $n$.
        The base case is trivial. For $n > 1$, suppose we aim to rearrange $x = a_1a_2 \dots a_n$ into some ordering such that $a_n$ is the $l$-th element in the order.
        We can first take the last $n - l + 1$ elements and apply (1) to move $a_n$ to the $l$-th position.
        Then, by induction, we can rearrange the first $l - 1$ elements and the last $n - l$ elements into the desired ordering, and thus the statement is proven.

        Since each element has one unique inverse, we can rearrange $x = a_1a_2 \dots a_n$ into $x = a_{m_n}a_{m_{n-1}} \dots a_{m_1}$, such that $a_ia_{m_i} = e$ for all $1 \leq i \leq n$.
        Therefore, 
        \begin{align*}
            x^2
            &= a_1a_2 \dots a_{n-1}(a_na_{m_n})a_{m_{n-1}} \dots a_{m_1} \\
            &= a_1a_2 \dots a_{n-2}(a_{n-1}a_{m_{n-1}})a_{m_{n-2}} \dots a_{m_1} \\
            &= a_1a_{m_1} \\
            &= e.
        \end{align*}
    \end{proof}
\end{homeworkProblem}

\newpage

\begin{homeworkProblem}
    If $G$ is of odd order, what can you say about the $x$ in problem $16$?
    
    \begin{proof}
        Since $G$ is of odd order, $G$ cannot have subgroups of order $2$,
        and thus for all non indentity $a \in G$, $a^2 \neq e$, otherwise $\{e, a\}$ would a a subgroup of order $2$ in $G$.
        This implies that each non-identity element can be paired with an unique inverse distinct to itself.
        By the result we obtained in the previous question, we can rearrange $x$ such that each non-identity element in the sequence is next to its inverse. 
        By associativity, each non-identity element in the new ordering would pair up with its neighboring inverse and resolve to $e$, and thus we get $x = e$.
    \end{proof}
\end{homeworkProblem}

\newpage

\begin{homeworkProblem}
    Let $G$ be a group, $H$ a subgroup of $G$, and let $S$ be the set of all distinct right cosets of $H$ in $G$,
    $T$ the set of all left cosets of $H$ in $G$. Prove that there is a 1-1 mapping of $S$ onto $T$.    

    \begin{proof}
        Consider the function $f: S \rightarrow T$, $f(Hx) = x^{-1}H$, for $x \in G$. We want to show $f$ is injective.
        Let $a, b \in G$, such that $f(Ha) = f(Hb)$. Then, we know $a^{-1}H = b^{-1}H$, and so $ba^{-1}H = H$, which implies $ba^{-1} \in H$.
        Let $h = ba^{-1} \in H$. We then get $ha = b \in Ha$, and thus $Ha = Hb$. Therefore, $f$ is a 1-1 mapping of $S$ onto $T$.
    \end{proof}
\end{homeworkProblem}

\newpage

\begin{homeworkProblem}
    If $aH = bH$ forces $Ha = Hb$ in $G$, show that $aHa^{-1} = H$ for every $a \in G$.
    
    \begin{proof}
        Let $b \in aH$. Then, $aH = bH$, which forces $b \in Hb = Ha$. Thus, $aH \subseteq Ha$, so $aHa^{-1} \subseteq H$.
        We now show that $|aHa^{-1}| \geq |H|$. Define $f: aHa^{-1} \rightarrow H$ as $f(x) = a^{-1}xa$.
        For each $y \in H$, we have $x = aya^{-1}$, such that $f(x) = a^{-1}(aya^{-1})a = y$. 
        Thus, $f$ is surjective, and so $|aHa^{-1}| \geq |H|$. 
        Since $aHa^{-1} \subseteq H$ and $|aHa^{-1}| \geq |H|$, we have $aHa^{-1} = H$.
    \end{proof}
\end{homeworkProblem}

\newpage

\begin{homeworkProblem}
    If in a group $G$, $aba^{-1} = b^i$, show that $a^rba^{-r} = b^{i^r}$ for all positive integers $r$.
    \begin{proof}
        We proceed by induction on $r$. The base case $aba^{-1} = b^i$ is already given. 
        For $r > 1$, we get $a^rba^{-r} = a \cdot a^{r - 1}ba^{-(r - 1)} \cdot a^{-1}$. 
        By induction, $a \cdot a^{r - 1}ba^{-(r - 1)} \cdot a^{-1} = ab^{i^{r - 1}}a^{-1} = b^{i^{r - 1} \cdot i} = b^{i^r}$, and we are done.
    \end{proof}
\end{homeworkProblem}

\newpage

\begin{homeworkProblem}
    If in $G$, $a^5 = e$ and $aba^{-1} = b^2$, find $o(b)$ if $b \neq e$.

    \begin{proof}
        Since $aba^{-1} = b^2$, by the result we obtained from the previous question, we know $a^5ba^{-5} = b = b^{2^5}$, and thus we get $b^{2^5 - 1} = e$.
        Since $2^5 - 1 = 31$ is a prime number and $b \neq e$, there are no positive $r < 31$ such that $b^r = e$, and so $o(b) = 31$.
    \end{proof}
\end{homeworkProblem}

\end{document}