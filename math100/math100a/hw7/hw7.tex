\documentclass{article}

\usepackage{fancyhdr}
\usepackage{extramarks}
\usepackage{amsmath}
\usepackage{amsthm}
\usepackage{amsfonts}
\usepackage{tikz}
\usepackage[plain]{algorithm}
\usepackage{algpseudocode}
\usepackage{enumitem}
\usepackage{amssymb}

\usetikzlibrary{automata,positioning}

%
% Basic Document Settings
%

\topmargin=-0.45in
\evensidemargin=0in
\oddsidemargin=0in
\textwidth=6.5in
\textheight=9.0in
\headsep=0.25in

\linespread{1.1}

\pagestyle{fancy}
\lhead{\hmwkAuthorName}
\chead{\hmwkClass:\ \hmwkTitle}
\rhead{\firstxmark}
\lfoot{\lastxmark}
\cfoot{\thepage}

\renewcommand\headrulewidth{0.4pt}
\renewcommand\footrulewidth{0.4pt}

\setlength\parindent{0pt}
\setlength{\parskip}{5pt}

%
% Create Problem Sections
%

\newcommand{\enterProblemHeader}[1]{
    \nobreak\extramarks{}{Problem \arabic{#1} continued on next page\ldots}\nobreak{}
    \nobreak\extramarks{Problem \arabic{#1} (continued)}{Problem \arabic{#1} continued on next page\ldots}\nobreak{}
}

\newcommand{\exitProblemHeader}[1]{
    \nobreak\extramarks{Problem \arabic{#1} (continued)}{Problem \arabic{#1} continued on next page\ldots}\nobreak{}
    \stepcounter{#1}
    \nobreak\extramarks{Problem \arabic{#1}}{}\nobreak{}
}

\setcounter{secnumdepth}{0}
\newcounter{partCounter}
\newcounter{homeworkProblemCounter}
\setcounter{homeworkProblemCounter}{1}
\nobreak\extramarks{Problem \arabic{homeworkProblemCounter}}{}\nobreak{}

%
% Homework Problem Environment
%
% This environment takes an optional argument. When given, it will adjust the
% problem counter. This is useful for when the problems given for your
% assignment aren't sequential. See the last 3 problems of this template for an
% example.
%
\newenvironment{homeworkProblem}[1][-1]{
    \ifnum#1>0
        \setcounter{homeworkProblemCounter}{#1}
    \fi
    \section{Problem \arabic{homeworkProblemCounter}}
    \setcounter{partCounter}{1}
    \enterProblemHeader{homeworkProblemCounter}
}{
    \exitProblemHeader{homeworkProblemCounter}
}

%
% Homework Details
%   - Title
%   - Due date
%   - Class
%   - Section/Time
%   - Instructor
%   - Author
%

\newcommand{\hmwkTitle}{Homework\ \#7}
\newcommand{\hmwkDueDate}{November 21, 2023}
\newcommand{\hmwkClass}{MATH 100A}
\newcommand{\hmwkClassTime}{Section A02 5:00PM - 5:50PM}
\newcommand{\hmwkSectionLeader}{Castellano}
\newcommand{\hmwkClassInstructor}{Professor McKernan}
\newcommand{\hmwkSource}{Source Consulted: Textbook, Lecture, Discussion}
\newcommand{\hmwkAuthorName}{\textbf{Ray Tsai}}
\newcommand{\hmwkPID}{A16848188}

%
% Title Page
%

\title{
    \vspace{2in}
    \textmd{\textbf{\hmwkClass:\ \hmwkTitle}}\\
    \normalsize\vspace{0.1in}\small{Due\ on\ \hmwkDueDate\ at 12:00pm}\\
    \vspace{0.1in}\large{\textit{\hmwkClassInstructor}} \\
    \vspace{0.1in}\small\hmwkClassTime \\
    \small Section Leader: \hmwkSectionLeader \\
    \vspace{0.1in}\small\hmwkSource \\
    \vspace{3in}
}

\author{
  \hmwkAuthorName \\
  \vspace{0.1in}\small\hmwkPID
}
\date{}

\renewcommand{\part}[1]{\textbf{\large Part \Alph{partCounter}}\stepcounter{partCounter}\\}

%
% Various Helper Commands
%

% Useful for algorithms
\newcommand{\alg}[1]{\textsc{\bfseries \footnotesize #1}}

% For derivatives
\newcommand{\deriv}[1]{\frac{\mathrm{d}}{\mathrm{d}x} (#1)}

% For partial derivatives
\newcommand{\pderiv}[2]{\frac{\partial}{\partial #1} (#2)}

% Integral dx
\newcommand{\dx}{\mathrm{d}x}

% Probability commands: Expectation, Variance, Covariance, Bias
\newcommand{\Var}{\mathrm{Var}}
\newcommand{\Cov}{\mathrm{Cov}}
\newcommand{\Bias}{\mathrm{Bias}}
\newcommand*{\Z}{\mathbb{Z}}
\newcommand*{\Q}{\mathbb{Q}}
\newcommand*{\R}{\mathbb{R}}
\newcommand*{\C}{\mathbb{C}}
\newcommand*{\N}{\mathbb{N}}
\newcommand*{\prob}{\mathds{P}}
\newcommand*{\E}{\mathds{E}}

\begin{document}

\maketitle

\pagebreak

\begin{homeworkProblem}
    If $G_1$ and $G_2$ are groups, prove that $G_1 \times G_2 \simeq G_2 \times G_1$.

    \begin{proof}
        Define $\phi: G_1 \times G_2 \rightarrow G_2 \times G_1$ as $\phi(a, b) = (b, a)$.
        $\phi$ is obviously a well-defined.
        Define $\psi: G_2 \times G_1 \rightarrow G_1 \times G_2$ as $\psi(b, a) = (a, b)$.
        Since $\phi(\psi(b, a)) = \phi(a, b) = (b, a)$ and $\psi(\phi(a, b)) = \psi(b, a) = (a, b)$, $\psi$ is an inverse of $\phi$, so $\phi$ is bijective.
        Since $\phi(a, b)\phi(a', b') = (bb', aa') = \phi(aa', bb')$, $\phi$ is an isomorphism, and thus $G_1 \times G_2 \simeq G_2 \times G_1$.
    \end{proof}
\end{homeworkProblem}

\pagebreak

\begin{homeworkProblem}
    If $G_1$ and $G_2$ are cyclic groups of orders $m$ and $n$, respectively, prove that $G_1 \times G_2$ is cyclic if and only if $m$ and $n$ are relatively prime.

    \begin{proof}
        Suppose that $G_1 \times G_2$ is cyclic.
        Then $G_1 \times G_2 = \{(a^i, b^i) \, | \, i \in \Z\}$, for some $a \in G_1$, $b \in G_2$.
        Since $G_1 \times G_2$ is of order $mn$, we know $m, n$ is relatively prime, otherwise we can find $k < mn$ such that $(a^k, b^k) = (e_1, e_2)$, which contradicts that $G_1 \times G_2$ is of order $mn$.
        Suppose that $m, n$ are relatively prime.
        Let $c \in G_1, d \in G_2$ each be the generator of their respective group.
        Let $(x, y) = (c^j, d^l) \in G_1 \times G_2$, and let $d = l - j$.
        Since $m, n$ are relatively prime, there exists $m\alpha + n\beta = 1$.
        Multiplying both sides by $d$, we get $md\alpha + nd\beta = l - j$, and so there exists $x = (d\alpha) m + j = (-d\beta) n + l$.
        Thus, $(x, y) = (c^j, d^l) = (c^x, d^x)$, and so $G_1 \times G_2$ is cyclic.
    \end{proof}
\end{homeworkProblem}

\pagebreak

\begin{homeworkProblem}
    Let $G$ be a group, $A = G \times G$.
    In $A$ let $T = \{(g, g) \, | \, g \in G\}$.

    \begin{enumerate}[label=(\alph*)]
        \item Prove that $T \simeq G$.
        
        \begin{proof}
            Let $\phi: T \rightarrow G$ be the natural projection.
            Then, $\phi$ is well-defined and surjective.
            Since $\phi(g, g)\phi(g', g') = gg' = \phi(gg', gg')$, $\phi$ is a homomorphism.
            Let $(a, a) \in \text{Ker }\phi$.
            $\phi(a, a) = a = e$, and so Ker $\phi$ is trivial.
            Therefore, $\phi$ is an isomorphism, and thus $T \simeq G$.
        \end{proof}

        \item Prove that $T \vartriangleleft A$ if and only if $G$ is abelian.

        \begin{proof}
            Suppose that $T \vartriangleleft A$.
            For $(g, h) \in A$, $(g, h)(g, g)(g^{-1}, h^{-1}) = (g, hgh^{-1}) \in T$.
            This implies that for all $g, h \in G$, $g = hgh^{-1}$.
            Rearranged, we get $gh = hg$, which makes $G$ abelian.
            Suppose that $G$ is abelian. 
            Let $(g, g) \in T$, $(a, b) \in A$.
            Since $(a, b)(g, g)(a^{-1}, b^{-1}) = (aga^{-1}, bgb^{-1}) = (g, g) \in T$, $T$ is normal in $A$.
        \end{proof}
    \end{enumerate}
\end{homeworkProblem}

\pagebreak

\begin{homeworkProblem}
    Let $H$ and $K$ be two normal subgroups of a group $G$, whose intersection is the trivial subgroup. 
    Prove that every element of $H$ commutes with every element of $K$.

    \begin{proof}
        Let $h \in H$, $k \in K$.
        Since $H$ is normal, $h^{-1}k^{-1}hk = h^{-1}h'k^{-1}k = h^{-1}h' \in H$.
        By symmetry, $h^{-1}k^{-1}hk \in K$, which makes $h^{-1}k^{-1}hk \in H \cap K = \{e\}$.
        Thus, we know $h^{-1}k^{-1}hk$ must be the identity element, and thus $hk = kh$.
    \end{proof}
\end{homeworkProblem}

\pagebreak

\begin{homeworkProblem}
    Prove that a group $G$ is isomorphic to the product of two groups $H'$ and $K'$ if and only if $G$ contains two normal subgroups $H$ and $K$,
    such that
    \begin{enumerate}
        \item $H$ is isomorphic to $H'$ and $K$ is isomorphic to $K'$.
        \item $H \cap K = \{e\}$.
        \item $G = H \vee K$.
    \end{enumerate}

    \begin{proof}
        Suppose that $G \simeq H' \times K'$.
        Let $\phi: H' \times K' \rightarrow G$ be an isomorphism, $G_{H'} = \{(h, e_{k'}) \, | \, h \in H'\}$, and $G_{K'} = \{(e_{h'}, k) \, | \, k \in K'\}$, where $e_{h'} \in H', e_{k'} \in K'$ are the identity element of their corresponding groups.
        Let $H = \phi(G_{H'})$ and $K = \phi(G_{K'})$.
        From Homework 6 question 2.7.4, we have shown that $H' \simeq G_{H'}$ and $K' \simeq G_{K'}$, and $G_{H'}, G_{K'}$ are normal subgroups of $H' \times K'$.
        Thus, we know $H \simeq G_{H'} \simeq H'$ and $K \simeq G_{K'} \simeq K'$ are both normal subgroups of $G$.
        Let $\psi: G \rightarrow H' \times K'$ be the inverse of $\phi$.
        Then, $\psi(H \cap K) = G_{H'} \cap G_{K'} = \{(e_{h'}, e_{k'})\}$, which contains only the identity element of $H' \times K'$.
        Since $\psi$ is an isomorphism, $H \cap K  = \{e\}$.
        Note that for all $x \in H' \times K'$, $x = ab$, for some $a \in G_{H'}$, $b \in G_{K'}$.
        Thus, $\phi(x) = \phi(ab) = \phi(a)\phi(b) = hk$, where $h \in H$ and $k \in K$.
        This implies that $G = HK$, and so $G = H \vee K$, by the Second Isomorphism Theorem.
        
        We now suppose that conditions 1-3 hold.
        Since $H, K$ are normal, by the Second Isomorphism Theorem, $G = H \vee K = HK$.
        Let $\alpha: H \rightarrow H'$ and $\beta: K \rightarrow K'$ be isomorphisms.
        Define $\varphi: G \rightarrow H' \times K'$ as $\varphi(hk) = (\alpha(h), \beta(k))$, for $h \in H$, $k \in K$.
        Suppose $hk = h_0k_0 \in G$, for $h, h_0 \in H$ and $k, k_0 \in K$.
        Then, $\varphi(hk) = (\alpha(h), \beta(k)) = (\alpha(h_0), \beta(k_0)) = \varphi(h'k')$, so $\varphi$ is well-defined.
        Define $\theta: H' \times K' \rightarrow G$ as $\theta(h', k') = \alpha^{-1}(h')\beta^{-1}(k')$, where $\alpha^{-1}, \beta^{-1}$ are the inverses of $\alpha, \beta$, respectively.
        We then get $\varphi(\theta(h', k')) = \varphi(\alpha^{-1}(h')\beta^{-1}(k')) = (\alpha(\alpha^{-1}(h')), \beta(\beta^{-1}(k'))) = (h', k')$ and $\theta(\varphi(hk)) = \theta(\alpha(h), \beta(k)) = \alpha^{-1}(\alpha(h))\beta^{-1}(\beta(k)) = hk$.
        Thus, $\theta$ is the inverse of $\varphi$, so $\varphi$ is a bijective mapping.
        Finally, we check that $\varphi$ is a homomorphism.
        Let $m = hk, n = h_1k_1 \in G$, where $h, h_1 \in H$ and $k, k_1 \in K$.
        Note that since $H, K$ are both normal and $H \cap K = \{e\}$, every element of $H$ commutes with every element of $K$, by result we obtained in the previous problem.
        Thus, 
        \begin{align*}
            \varphi(mn) 
            &= \varphi(hkh_1k_1) \\
            &= \varphi(hh_1kk_1) \\
            &= (\alpha(hh_1), \beta(kk_1)) \\
            &= (\alpha(h)\alpha(h_1), \beta(k)\beta(k_1)) \\
            &= (\alpha(h), \beta(k))(\alpha(h_1), \beta(k_1)) \\
            &= \varphi(hk)\varphi(h_1k_1) \\
            &= \varphi(m)\varphi(n).
        \end{align*}
        Therefore, $\varphi$ is an isomorphism, and so $G \simeq H' \times K'$.
    \end{proof}
\end{homeworkProblem}
\end{document}