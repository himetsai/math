\documentclass{article}

\usepackage{fancyhdr}
\usepackage{extramarks}
\usepackage{amsmath}
\usepackage{amsthm}
\usepackage{amsfonts}
\usepackage{tikz}
\usepackage[plain]{algorithm}
\usepackage{algpseudocode}
\usepackage{ifmtarg}

\usetikzlibrary{automata,positioning}

%
% Basic Document Settings
%

\topmargin=-0.45in
\evensidemargin=0in
\oddsidemargin=0in
\textwidth=6.5in
\textheight=9.0in
\headsep=0.25in

\linespread{1.1}

\pagestyle{fancy}
\lhead{\hmwkAuthorName}
\chead{\hmwkClass:\ \hmwkTitle}
\rhead{\firstxmark}
\lfoot{\lastxmark}
\cfoot{\thepage}

\renewcommand\headrulewidth{0.4pt}
\renewcommand\footrulewidth{0.4pt}

\setlength\parindent{0pt}
\setlength{\parskip}{5pt}

%
% Create Problem Sections
%

\newcommand{\enterProblemHeader}[1]{
    \nobreak\extramarks{}{Problem \arabic{#1} continued on next page\ldots}\nobreak{}
    \nobreak\extramarks{Problem \arabic{#1} (continued)}{Problem \arabic{#1} continued on next page\ldots}\nobreak{}
}

\newcommand{\exitProblemHeader}[1]{
    \nobreak\extramarks{Problem \arabic{#1} (continued)}{Problem \arabic{#1} continued on next page\ldots}\nobreak{}
    \stepcounter{#1}
    \nobreak\extramarks{Problem \arabic{#1}}{}\nobreak{}
}

\setcounter{secnumdepth}{0}
\newcounter{partCounter}
\newcounter{homeworkProblemCounter}
\setcounter{homeworkProblemCounter}{1}
\nobreak\extramarks{Problem \arabic{homeworkProblemCounter}}{}\nobreak{}

%
% Homework Problem Environment
%
% This environment takes an optional argument. When given, it will adjust the
% problem counter. This is useful for when the problems given for your
% assignment aren't sequential. See the last 3 problems of this template for an
% example.
%
\newenvironment{homeworkProblem}[1][-1]{
    \ifnum#1>0
        \setcounter{homeworkProblemCounter}{#1}
    \fi
    \section{Problem \arabic{homeworkProblemCounter}}
    \setcounter{partCounter}{1}
    \enterProblemHeader{homeworkProblemCounter}
}{
    \exitProblemHeader{homeworkProblemCounter}
}

%
% Homework Details
%   - Title
%   - Due date
%   - Class
%   - Section/Time
%   - Instructor
%   - Author
%

\newcommand{\hmwkTitle}{Homework\ \#2}
\newcommand{\hmwkDueDate}{October 12, 2023}
\newcommand{\hmwkClass}{MATH 100A}
\newcommand{\hmwkClassTime}{Section A02 5:00PM - 5:50PM}
\newcommand{\hmwkSectionLeader}{Castellano}
\newcommand{\hmwkClassInstructor}{Professor McKernan}
\newcommand{\hmwkSource}{Source Consulted: Textbook, Lecture, Discussion, Office Hour, ChatGPT}
\newcommand{\hmwkAuthorName}{\textbf{Ray Tsai}}
\newcommand{\hmwkPID}{A16848188}

%
% Title Page
%

\title{
    \vspace{2in}
    \textmd{\textbf{\hmwkClass:\ \hmwkTitle}}\\
    \normalsize\vspace{0.1in}\small{Due\ on\ \hmwkDueDate\ at 12:00pm}\\
    \vspace{0.1in}\large{\textit{\hmwkClassInstructor}} \\
    \vspace{0.1in}\small\hmwkClassTime \\
    \small Section Leader: \hmwkSectionLeader \\
    \vspace{0.1in}\small\hmwkSource \\
    \vspace{3in}
}

\author{
  \hmwkAuthorName \\
  \vspace{0.1in}\small\hmwkPID
}
\date{}

\renewcommand{\part}[1]{\textbf{\large Part \Alph{partCounter}}\stepcounter{partCounter}\\}

%
% Various Helper Commands
%

% Useful for algorithms
\newcommand{\alg}[1]{\textsc{\bfseries \footnotesize #1}}

% For derivatives
\newcommand{\deriv}[1]{\frac{\mathrm{d}}{\mathrm{d}x} (#1)}

% For partial derivatives
\newcommand{\pderiv}[2]{\frac{\partial}{\partial #1} (#2)}

% Integral dx
\newcommand{\dx}{\mathrm{d}x}

% Probability commands: Expectation, Variance, Covariance, Bias
\newcommand{\Var}{\mathrm{Var}}
\newcommand{\Cov}{\mathrm{Cov}}
\newcommand{\Bias}{\mathrm{Bias}}
\newcommand*{\Z}{\mathbb{Z}}
\newcommand*{\Q}{\mathbb{Q}}
\newcommand*{\R}{\mathbb{R}}
\newcommand*{\C}{\mathbb{C}}
\newcommand*{\N}{\mathbb{N}}
\newcommand*{\prob}{\mathds{P}}
\newcommand*{\E}{\mathds{E}}

\begin{document}

\maketitle

\pagebreak

\begin{homeworkProblem}
  Give a description of $D_4$, the group of symmetries of the square,
  similar to the one given in class, and find all of its subgroups.
  
  \begin{proof}
    Let there be a square $S$ with vertices $A, B, C, D$ labeled clockwise, as shown below.
    
    \begin{figure}[H]
      \centering
      \begin{tikzpicture}
        \coordinate (A) at (0,2);
        \coordinate (B) at (2,2);
        \coordinate (C) at (2,0);
        \coordinate (D) at (0,0);
        \draw (A) -- (B) -- (C) -- (D) -- cycle;
        \node[above left] at (A) {A};
        \node[above right] at (B) {B};
        \node[below right] at (C) {C};
        \node[below left] at (D) {D};
      \end{tikzpicture}
    \end{figure}

    $D_4$ contains all the symmetries of $S$, including rotations, flips, and the identity $I$.
    There are three types of rotations, $R, R^2, R^3$, which rotates $S$ counter-clockwise by 
    $90^{\circ}, 180^{\circ}, 270^{\circ}$, respectively. Note that the $360^{\circ}$ rotation $R^4 = I$. 
    Additionally, there are four types of flips, $F_1, F_2, F_3, F_4$. $F_1$ flips $S$ vertically through its center.
    $F_2$ flips $S$ horizontally through its center. $F_3$ flips $S$ diagonally through vertices $A, C$. $F_4$ flips $S$ diagonally
    through vertices $B, D$. Note that each flip is its own inverse. 
    
    We now show that there can be at most $8$ configurations 
    of the positions of vertices. Since $A, C$ must be at the opposite position, $A$'s position determines $C$'s. We pick $A$'s 
    position first, then pick $B$ and $D$'s, so there are at most ${4 \choose 1}{2 \choose 1} = 8$ symmetries.
    Since the identity, rotations, and flips we mentioned are all distinct, they account for all $8$ symmetries. Thus, we conclude
    that $D_4 = \{I, R, R^2, R^3, F_1, F_2, F_3, F_4\}$.

    We now find all the subgroups of $D_4$. We first note $\{I\}$ and $D_4$ itself are subgroups of $D_4$. There are also the cyclic subgroups 
    $\langle R \rangle = \langle R^3 \rangle, \langle R^2 \rangle, \langle F_1 \rangle, \langle F_2 \rangle, \langle F_3 \rangle, \langle F_4 \rangle$. Note that $\langle R \rangle = \langle R^3 \rangle = \{I, R, R^2, R^3\}$ is the subgroup that contains all rotations. 
    Suppose that we include some flip, say $F_1$ to $\langle R \rangle$. Then, we would also have $F_2 = R^2F_1, F_3 = RF_1, F_4 = R^3F_1$ in the group, which 
    becomes $D_4$. 

    We also observe that $F_1F_3 = F_2F_4 = R$, and $F_2F_3 = F_1F_4 = R^3$. This implies that if we include any of those pairs of flips in the same group, 
    the group untimately becomes $D_4$, by the result we obtained above. 
    
    Note that $F_1F_2 = F_2F_1 = F_3F_4 = F_4F_3 = R^2$, so we attempt to construct subgroups with $\langle R^2 \rangle$. Suppose we include $F_1$ to $\langle R^2 \rangle$, then 
    we get $F_2 = F_1R^2$. Since each of $\{I, R^2, F_1, F_2\}$ is its own inverse and any two elements' product is still in the group, it is a subgroup. Suppose we
    include $F_3$ to $\langle R^2 \rangle$, then we get $F_4 = F_3R^2$. Since each of $\{I, R^2, F_3, F_4\}$ is its own inverse and any two elements' product is still in the group,
    it is a subgroup. 
    
    Since no more combination of elements in $D_4$ can be use to generate a new group, we have exausted all subgroups of $D_4$, namely 
    \[
      \langle I \rangle, \langle R \rangle, \langle R^2 \rangle, \langle F_1 \rangle, \langle F_2 \rangle, \langle F_3 \rangle, \langle F_4\rangle , \{I, R^2, F_1, F_2\}, \{I, R^2, F_3, F_4\}, D_4.
    \]
  \end{proof}
\end{homeworkProblem}

\newpage

\begin{homeworkProblem}
  Suppose that $G$ is a set closed under an associative operation such that
  \begin{enumerate}
    \item given $a, y \in G$, there is an $x \in G$ such that $ax = y$, and
    \item given $a, w \in G$, there is a $u \in G$ such that $ua = w$.
  \end{enumerate}
  Show that $G$ is a group.

  \begin{proof}
    Let $b, c \in G$. We know that there exists $a, d \in G$ such that $ab = b$ 
    and $bd = c$. Then, we get $abd = bd = ac = c$, and so $a$ is a left identity
    element. Similarly, we can also find a right identity element $f$ using the above approach.
    This follows that sicne $af = a = f$, the left and right inverse are the same element, and so 
    $G$ contains an identity element $e = a = f$.

    Let $\alpha \in G$. We know that there exists $\beta, \gamma \in G$ such that $\alpha\beta = e$ and
    $\beta\gamma = e$. This follows that since $\alpha\beta\gamma = \alpha = \gamma$, $\alpha\beta = e$ 
    and $\beta\alpha = e$, and thus all elements in $G$ has an inverse. Therefore, $G$ is a group.
  \end{proof}
\end{homeworkProblem}

\newpage

\begin{homeworkProblem}
  If $G$ is a finite set closed under an associative operation such that $ax = ay$ 
  forces $x = y$ and $ua = wa$ forces $u = w$, for every $a,x,y,u,w \in G$, prove
  that $G$ is a group.

  \begin{proof}
    Let $a \in G$. Define $f: G \rightarrow G$ to be $f(g) = ag$. Since $ax = ay$ implies $x = y$, 
    we know $f$ is injective. This follows that $f$ is also surjective since $G$ is a finite set, and 
    so for each $c \in G$, there exists $b \in G$ such that $ab = c$. Similarly, we can define $h: G \rightarrow G$ 
    to be $h(g) = ga$ and show that there exists $x \in G$ such that $xa = c$, and thus the rest of the proof follows
    the previous problem.
  \end{proof}
\end{homeworkProblem}

\newpage

\begin{homeworkProblem}
  Verify that $Z(G)$, the center of $G$, is a subgroup of $G$.

  \begin{proof}
    We first verify that $Z(G)$ is closed under the operation of $G$. Let $a, b \in Z(G)$, and let $x \in G$. 
    Since $abg = agb = gab$, $ab \in Z(G)$, and thus $Z(G)$ fulfills the closed property.

    We now check the inverse property. Let $e \in G$ be the identity element, and let $c \in Z(G)$. Then, for all $x \in G$,
    \begin{align*}
      cx &= xc \\
      x &= c^{-1}xc \\
      xc^{-1} &= c^{-1}x,
    \end{align*}
    and thus $c^{-1} \in Z(G)$. Therefore, $Z(G)$ is a subgroup of $G$.
  \end{proof}
\end{homeworkProblem}

\newpage

\begin{homeworkProblem}
  If $C(a)$ is the centralizer of $a$ in $G$, prove that $Z(G) = \bigcap_{a \in G}C(a)$.

  \begin{proof}
    Let $z \in Z(G)$, and let $a \in G$. Since $za = az$, $z \in C(a)$, and so $z \in \bigcap_{a \in G}C(a)$. Therefore,
    $Z(G) \subseteq \bigcap_{a \in G}C(a)$.

    Let $c \in \bigcap_{a \in G}C(a)$. Since for all $a \in G$, $c \in C(a)$, and so $ca = ac$. Therefore, $c \in Z(G)$, which
    means that $\bigcap_{a \in G}C(a) \subseteq Z(G)$. Combining two results, we conclude that $Z(G) = \bigcap_{a \in G}C(a)$.
  \end{proof}
\end{homeworkProblem}

\newpage

\begin{homeworkProblem}
  If $G$ is an abelian group and if $H = \{a \in G \, | \, a^2 = e\}$, show that $H$ is a subgroup of $G$.

  \begin{proof}
    Let $a, b \in H$. Since $(ab)^2= abab = a^2b^2 = e$, we get $ab \in H$, and thus $H$ is closed under the operation of $G$.
    Since $a^2a^{-2} = e = (a^{-1})^2$, $a^{-1} \in H$ for all $a \in H$. Therefore, $H$ is a subgroup of $G$.
  \end{proof}
\end{homeworkProblem}

\newpage

\begin{homeworkProblem}
  Prove that a cyclic group is abelian.

  \begin{proof}
    Let $G = \langle a \rangle$ be a cyclic group. Let $b = a^k, c = a^j \in G$.
    Since $bc = a^{k + j} = cb$, $G$ is abelian.
  \end{proof}
\end{homeworkProblem}

\newpage

\begin{homeworkProblem}
  If $A, B$ are subgroups of an abelian group $G$, let $AB = \{ab \, | \, a \in A, b \in B\}$. Prove that $AB$ is a subgroup of $G$.

  \begin{proof}
    Let $c = a_1b_1, d = a_2b_2 \in AB$, for $a_1, a_2, b_1, b_2 \in G$. Since $a_1a_2 \in A$ and $b_1b_2 \in B$, we get 
    $cd = a_1b_1a_2b_2 = (a_1a_2)(b_1b_2) \in AB$, and so $AB$ is closed under the operation of $G$. 
    Note that since $A, B$ are subgroups of $G$, we know $a^{-1} \in A$ and $b^{-1} \in B$.
    Since $c^{-1} = (a_1b_1)^{-1} = b_1^{-1}a_1^{-1} = a^{-1}b^{-1}$, we get $c^{-1} \in AB$. Therefore, $AB$ is a subgroup of $G$.
  \end{proof}
\end{homeworkProblem}

\newpage

\begin{homeworkProblem}
  Give an example of a group $G$ and two subgroups $A, B$ of $G$ such that $AB$ is not a subgroup of $G$.

  \begin{proof}
    Consider $D_3$, the group of symmetries of a triangle. Let $F_1, F_2, F_3$ be the flips through vertex $A, B, C$ respectively.

    \begin{figure}[h]
    \centering
    \begin{tikzpicture}
        \draw (0,0) -- (4,0) -- (2,3.5) -- cycle;
        \node[anchor=east] at (0,0) {C};
        \node[anchor=west] at (4,0) {B};
        \node[anchor=south] at (2,3.5) {A};
    \end{tikzpicture}
    \end{figure}

    We take subgroups $A = \langle F_1 \rangle$ and
    $B = \langle R \rangle$. Consider $F_3$ and $R$. Since $F_3 = F_1R$ and $R = IR$, we know $F_3, R \in AB$.
    However, $F_2 = F_3R \neq ab$ for all $a \in A, b \in B$. This implies that $AB$ does not have the closed property,
    and thus it is not a subgroup of $G$.
  \end{proof}
\end{homeworkProblem}

\end{document}