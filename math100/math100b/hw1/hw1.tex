\documentclass{article}

\usepackage{fancyhdr}
\usepackage{extramarks}
\usepackage{amsmath}
\usepackage{amsthm}
\usepackage{amsfonts}
\usepackage{tikz}
\usepackage[plain]{algorithm}
\usepackage{algpseudocode}
\usepackage{enumerate}
\usepackage{amssymb}

\usetikzlibrary{automata,positioning}

%
% Basic Document Settings
%

\topmargin=-0.45in
\evensidemargin=0in
\oddsidemargin=0in
\textwidth=6.5in
\textheight=9.0in
\headsep=0.25in

\linespread{1.1}

\pagestyle{fancy}
\lhead{\hmwkAuthorName}
\chead{\hmwkClass:\ \hmwkTitle}
\rhead{\firstxmark}
\lfoot{\lastxmark}
\cfoot{\thepage}

\renewcommand\headrulewidth{0.4pt}
\renewcommand\footrulewidth{0.4pt}

\setlength\parindent{0pt}
\setlength{\parskip}{5pt}

%
% Create Problem Sections
%

\newcommand{\enterProblemHeader}[1]{
    \nobreak\extramarks{}{Problem \arabic{#1} continued on next page\ldots}\nobreak{}
    \nobreak\extramarks{Problem \arabic{#1} (continued)}{Problem \arabic{#1} continued on next page\ldots}\nobreak{}
}

\newcommand{\exitProblemHeader}[1]{
    \nobreak\extramarks{Problem \arabic{#1} (continued)}{Problem \arabic{#1} continued on next page\ldots}\nobreak{}
    \stepcounter{#1}
    \nobreak\extramarks{Problem \arabic{#1}}{}\nobreak{}
}

\setcounter{secnumdepth}{0}
\newcounter{partCounter}
\newcounter{homeworkProblemCounter}
\setcounter{homeworkProblemCounter}{1}
\nobreak\extramarks{Problem \arabic{homeworkProblemCounter}}{}\nobreak{}

%
% Homework Problem Environment
%
% This environment takes an optional argument. When given, it will adjust the
% problem counter. This is useful for when the problems given for your
% assignment aren't sequential. See the last 3 problems of this template for an
% example.
%
\newenvironment{homeworkProblem}[1][-1]{
    \ifnum#1>0
        \setcounter{homeworkProblemCounter}{#1}
    \fi
    \section{Problem \arabic{homeworkProblemCounter}}
    \setcounter{partCounter}{1}
    \enterProblemHeader{homeworkProblemCounter}
}{
    \exitProblemHeader{homeworkProblemCounter}
}

%
% Homework Details
%   - Title
%   - Due date
%   - Class
%   - Section/Time
%   - Instructor
%   - Author
%

\newcommand{\hmwkTitle}{Homework\ \#1}
\newcommand{\hmwkDueDate}{Janurary 18, 2024}
\newcommand{\hmwkClass}{MATH 100B}
\newcommand{\hmwkClassTime}{Section A02 6:00PM - 6:50PM}
\newcommand{\hmwkSectionLeader}{Castellano-Macías}
\newcommand{\hmwkClassInstructor}{Professor McKernan}
\newcommand{\hmwkSource}{Source Consulted: Textbook, Lecture, Discussion, Office Hour}
\newcommand{\hmwkAuthorName}{\textbf{Ray Tsai}}
\newcommand{\hmwkPID}{A16848188} 

%
% Title Page
%

\title{
    \vspace{2in}
    \textmd{\textbf{\hmwkClass:\ \hmwkTitle}}\\
    \normalsize\vspace{0.1in}\small{Due\ on\ \hmwkDueDate\ at 12:00pm}\\
    \vspace{0.1in}\large{\textit{\hmwkClassInstructor}} \\
    \vspace{0.1in}\small\hmwkClassTime \\
    \small Section Leader: \hmwkSectionLeader \\
    \vspace{0.1in}\small\hmwkSource \\
    \vspace{3in}
}

\author{
  \hmwkAuthorName \\
  \vspace{0.1in}\small\hmwkPID
}
\date{}

\renewcommand{\part}[1]{\textbf{\large Part \Alph{partCounter}}\stepcounter{partCounter}\\}

%
% Various Helper Commands
%

% Useful for algorithms
\newcommand{\alg}[1]{\textsc{\bfseries \footnotesize #1}}

% For derivatives
\newcommand{\deriv}[1]{\frac{\mathrm{d}}{\mathrm{d}x} (#1)}

% For partial derivatives
\newcommand{\pderiv}[2]{\frac{\partial}{\partial #1} (#2)}

% Integral dx
\newcommand{\dx}{\mathrm{d}x}

\newcommand{\Var}{\mathrm{Var}}
\newcommand{\Cov}{\mathrm{Cov}}
\newcommand{\Bias}{\mathrm{Bias}}
\newcommand*{\Z}{\mathbb{Z}}
\newcommand*{\Q}{\mathbb{Q}}
\newcommand*{\R}{\mathbb{R}}
\newcommand*{\C}{\mathbb{C}}
\newcommand*{\N}{\mathbb{N}}
\newcommand*{\prob}{\mathds{P}}
\newcommand*{\E}{\mathds{E}}

\begin{document}

\maketitle

\pagebreak

\begin{homeworkProblem}
    Show that any field is an integral domain.

    \begin{proof}
        Let $F$ be a field, and let $a, b \in F$, such that $ab = 0$. Suppose for the sake of
        contradiction that $a, b \neq 0$. Since $F$ is a division ring, there exists $a^{-1} \in F$.
        But this implies $a^{-1}ab = b = 0$, contradiction. Thus, $F$ is an integral domain.
    \end{proof}
\end{homeworkProblem}

\newpage

\begin{homeworkProblem}
    Fine all matrices $\begin{pmatrix}
        a & b \\
        c & d
    \end{pmatrix}$ such that $\begin{pmatrix}
        a & b \\
        c & d
    \end{pmatrix}\begin{pmatrix}
        1 & 0 \\
        0 & 0
    \end{pmatrix} = \begin{pmatrix}
        1 & 0 \\
        0 & 0
    \end{pmatrix}\begin{pmatrix}
        a & b \\
        c & d
    \end{pmatrix}$.

    \begin{proof}
        $\begin{pmatrix}
            a & b \\
            c & d
        \end{pmatrix}\begin{pmatrix}
            1 & 0 \\
            0 & 0
        \end{pmatrix} = \begin{pmatrix}
            1 & 0 \\
            0 & 0
        \end{pmatrix}\begin{pmatrix}
            a & b \\
            c & d
        \end{pmatrix}$ if and only if $\begin{pmatrix}
            a & 0 \\
            c & 0
        \end{pmatrix} = \begin{pmatrix}
            a & b \\
            0 & 0 
        \end{pmatrix}$ if and only if $b = c = 0$. Thus, only diagonal $2 \times 2$ matrices meet 
        the requirement.
    \end{proof}
\end{homeworkProblem}

\newpage

\begin{homeworkProblem}
    Let $R$ be any ring with unit, $S$ the ring of $2 \times 2$ matrices over $R$.
    \begin{enumerate}[(a)]
        \item Check the associative law of multiplication in $S$.
        \begin{proof}
            Let $\begin{pmatrix}
                a & b \\
                c & d
            \end{pmatrix},\begin{pmatrix}
                g & h \\
                k & l 
            \end{pmatrix}, \begin{pmatrix}
                w & x \\
                y & z 
            \end{pmatrix} \in S$.
            Since $$\left[\begin{pmatrix}
                a & b \\
                c & d
            \end{pmatrix}\begin{pmatrix}
                g & h \\
                k & l 
            \end{pmatrix}\right]\begin{pmatrix}
                w & x \\
                y & z 
            \end{pmatrix} = \begin{pmatrix}
                ag + bk & ah + bl \\
                cg + dk & ch + dl 
            \end{pmatrix}\begin{pmatrix}
                w & x \\
                y & z 
            \end{pmatrix} = \begin{pmatrix}
                agw + bkw + ahy + bly & agx + bkx + ahz + blz \\
                cgw + dkw + chy + dly & cgx + dkx + chz + dlz 
            \end{pmatrix},$$
            $$\begin{pmatrix}
                a & b \\
                c & d
            \end{pmatrix}\left[\begin{pmatrix}
                g & h \\
                k & l 
            \end{pmatrix}\begin{pmatrix}
                w & x \\
                y & z 
            \end{pmatrix}\right] = \begin{pmatrix}
                a & b \\
                c & d
            \end{pmatrix}\begin{pmatrix}
                gw + hy & gx + hz \\
                kw + ly & kx + lz
            \end{pmatrix} = \begin{pmatrix}
                agw + bkw + ahy + bly & agx + bkx + ahz + blz \\
                cgw + dkw + chy + dly & cgx + dkx + chz + dlz 
            \end{pmatrix},$$ the associative law is met.
        \end{proof}

        \item Show that $\left\{\left.\begin{pmatrix}
            a & b \\
            0 & c
        \end{pmatrix} \, \right| \, a, b, c \in R\right\}$ is a subring of $S$.
        \begin{proof}
            We name the set $L$. $L$ contains the unit, namely the identity matrix. If suffices to
            check that $L$ is closed under addition, additive inverses, and multiplication. Let 
            $\begin{pmatrix}
                x & y \\
                0 & z
            \end{pmatrix}, \begin{pmatrix}
                g & h \\
                0 & k
            \end{pmatrix} \in L$. Since $\begin{pmatrix}
                x & y \\
                0 & z
            \end{pmatrix} + \begin{pmatrix}
                g & h \\
                0 & k
            \end{pmatrix} = \begin{pmatrix}
                x + g & y + h \\
                0 & z + k
            \end{pmatrix} \in L$, $L$ is closed under addition. Since there exists 
            $\begin{pmatrix}
                -x & -y \\
                0 & -z
            \end{pmatrix} \in L$ such that $\begin{pmatrix}
                x & y \\
                0 & z
            \end{pmatrix} + \begin{pmatrix}
                -x & -y \\
                0 & -z
            \end{pmatrix} = \begin{pmatrix}
                -x & -y \\
                0 & -z
            \end{pmatrix} + \begin{pmatrix}
                x & y \\
                0 & z
            \end{pmatrix} = \begin{pmatrix}
                0 & 0 \\
                0 & 0
            \end{pmatrix}$, $L$ is closed under taking additive inverse. Since $\begin{pmatrix}
                x & y \\
                0 & z
            \end{pmatrix}\begin{pmatrix}
                g & h \\
                0 & k
            \end{pmatrix} = \begin{pmatrix}
                xg & xh + yk \\
                0 & zk
            \end{pmatrix} \in L$, $L$ is closed under multiplication. Therefore, $L$ is a subring.
        \end{proof}

        \item Show that $\begin{pmatrix}
            a & b \\
            0 & c
        \end{pmatrix}$ has an inverse in $S$ if and only if $a$ and $c$ have inverses in $R$. In
        that case write down $\begin{pmatrix}
            a & b \\
            0 & c
        \end{pmatrix}^{-1}$ explicitly.

        \begin{proof}
            Suppose that there exists $\begin{pmatrix}
                a & b \\
                0 & c
            \end{pmatrix}^{-1} = \begin{pmatrix}
                x & y \\
                w & z
            \end{pmatrix} \in S$, such that $\begin{pmatrix}
                a & b \\
                0 & c
            \end{pmatrix}\begin{pmatrix}
                x & y \\
                w & z
            \end{pmatrix} = \begin{pmatrix}
                x & y \\
                w & z
            \end{pmatrix}\begin{pmatrix}
                a & b \\
                0 & c
            \end{pmatrix} = \begin{pmatrix}
                1 & 0 \\
                0 & 1
            \end{pmatrix}$. Then, $\begin{pmatrix}
                ax + bw & ay + bz \\
                cw & cz
            \end{pmatrix} = \begin{pmatrix}
                xa & xb + yc \\
                wa & wb + zc
            \end{pmatrix} = \begin{pmatrix}
                1 & 0 \\
                0 & 1 \end{pmatrix}$. Notice that $w = 0$, otherwise $a = c = 0$ and $xa = 0 \neq
            1$. Thus, we have $xa = ax + bw = ax = 1$ and $cz = wb + zc = zc = 1$, so $a, c$ have
            inverse $x, z \in R$, respectively. Since $ay + bc^{-1} = a^{-1}b + yc = 0$, we know $y
            = -a^{-1}bc^{-1}$, and so $\begin{pmatrix}
                a & b \\
                0 & c
            \end{pmatrix}^{-1} = \begin{pmatrix}
                a^{-1} & -a^{-1}bc^{-1} \\
                0 & c^{-1}
            \end{pmatrix}$.

            We now suppose that $a^{-1}, c^{-1} \in R$. Then, there exists $\begin{pmatrix}
                a^{-1} & -a^{-1}bc^{-1} \\
                0 & c^{-1}
            \end{pmatrix} \in S$, such that $$\begin{pmatrix}
                a^{-1} & -a^{-1}bc^{-1} \\
                0 & c^{-1}
            \end{pmatrix}\begin{pmatrix}
                a & b \\
                0 & c
            \end{pmatrix} = \begin{pmatrix}
                a & b \\
                0 & c
            \end{pmatrix}\begin{pmatrix}
                a^{-1} & -a^{-1}bc^{-1} \\
                0 & c^{-1}
            \end{pmatrix} = \begin{pmatrix}
                1 & 0 \\
                0 & 1
            \end{pmatrix},$$ and we are done.
        \end{proof}
    \end{enumerate}
\end{homeworkProblem}

\newpage

\begin{homeworkProblem}
    Let $F: \C \rightarrow \C$ be defined by $F(a + bi) = a - bi$. Show that:
    \begin{enumerate}[(a)]
        \item $F(xy) = F(x)F(y)$ for $x, y \in \C$.
        \begin{proof}
            Let $x = a + bi, y = c + di \in \C$.
            \begin{align*}
                F(xy)
                &= F[(a + bi)(c + di)] \\
                &= F(ac - bd + (ad + bc)i) \\
                &= ac - bd - (ad + bc)i \\
                &= (a - bi)(c - di) = F(x)F(y).
            \end{align*}
        \end{proof}
        
        \item $F(x\bar{x}) = |x|^2$.
        \begin{proof}
            \[
                F(x\bar{x}) = F((a + bi)(a - bi)) = F(a^2 + b^2) = |x|^2.
            \]
        \end{proof}

        \item Using Parts (a) and (b), show that
        \[
            (a^2 + b^2)(c^2 + d^2) = (ac - bd)^2 + (ad + bc)^2.
        \]
        \begin{proof}
            \begin{align*}
                (a^2 + b^2)(c^2 + d^2)
                &= F(x\bar{x})F(y\bar{y}) \\
                &= F(x\bar{x}y\bar{y}) \\
                &= F(xy\bar{x}\bar{y}) \\
                &= F(xy\overline{xy}) \\
                &= |xy|^2 \\
                &= (ac - bd)^2 + (ad + bc)^2.
            \end{align*}
        \end{proof}
    \end{enumerate}
\end{homeworkProblem}

\newpage

\begin{homeworkProblem}
    Show that the only quaternions commuting with $i$ are of the form $\alpha + \beta i$.

    \begin{proof}
        Let $q = ai + bj + ck + d$ be a quaternion that commutes with $i$. This means that $qi = -a
        -bk + cj + di = -a + bk - cj + di = iq$, so $b = -b$ and $c = -c$. Thus, $b = c = 0$, so $q
        = d + ai$ is of the form $\alpha + \beta i$.
    \end{proof}
\end{homeworkProblem}

\newpage

\begin{homeworkProblem}
    Find the quaternions that commute with both $i$ and $j$.

    \begin{proof}
        Let $q = ai + bj + ck + d$ be a quaternion that commutes with both $i$ and $j$. This means
        that $qi = -a -bk + cj + di = -a + bk - cj + di = iq$ and $qj = ak -b - ci + dj = -ak - b +
        ci + dj = jq$, so $b = -b, c = -c,$ and $a = -a$. Thus, $a = b = c = 0$, so $q$ is a real
        number.
    \end{proof}
\end{homeworkProblem}

\newpage

\begin{homeworkProblem}
    Show that there is an \textit{inifnite} number of solutions to $x^2 = -1$ in the quaternions.

    \begin{proof}
        Consider $x = bi + cj + dk$. Then, $x^2 = -(b^2 + c^2 + d^2) = -1$, but $b^2 + c^2 + d^2 =
        1$ has infinitly many real solutions. Therefore, there is an inifnite number of solutions to
        $x^2 = -1$ in the quaternions.
    \end{proof}
\end{homeworkProblem}

\newpage

\begin{homeworkProblem}
    In the quaternions, consider the following set $G$ having eight elements: $G = \{\pm 1, \pm i,
    \pm j, \pm k\}$.
    \begin{enumerate}[(a)]
        \item Prove that $G$ is a group under multiplication.
        \begin{proof}
            Since the quaternions form a division ring and $G$ is a subset of the quaternions ring,
            it suffices to show that $G$ is closed under multiplication and taking inverses. By the
            quaternions multiplication rule carved on the Brougham Bridge in Dublin, $G$ is closed
            under multiplication. Since the inverse of each element in $G$ is just the conjugate of
            itself, which is also in $G$, $G$ is closed under taking inverses, and this completes
            the proof.
        \end{proof}

        \item List all subgroups of $G$.
        \begin{proof}
            $G$ itself and the trivial subgroup $\{1\}$ are subgroups of $G$. By Lagrange's Theorem,
            the remaining subgroups are of sizes either $2$ or $4$. We first consider subgroups
            generated by a single element. We know $\langle -1 \rangle = \{\pm 1\}$. Consider the
            subgroup generated by $i$ or $-i$. We get $\langle i \rangle = \langle -i \rangle =
            \{\pm 1, \pm i\}$. By symmetry, we also have $\{\pm 1, \pm j\}$ and $\{\pm 1, \pm k\}$.
            Since any pair of elements $\neq \pm 1$ and not from the same subgroup listed above
            would generate $G$, we have listed all the subgroups of $G$.
        \end{proof}

        \item What is the center of $G$.
        \begin{proof}
            Since only $\pm 1$ commute with all elements in $G$, $\{\pm 1\}$ is the center of $G$.
        \end{proof}

        \item Show that $G$ is a nonabelian group all of whose subgroups are normal.
        \begin{proof}
            Since $ij \neq ji$, $G$ is nonabelian. Since subgroups of order 4 is half the size
            of $G$, all subgroups of order 4 are normal. However, the remaining subgroups of $G$ are
            the trivial subgroup, the center, and $G$ itself, so all subgroups of $G$ are normal. 
        \end{proof}
    \end{enumerate}
\end{homeworkProblem}

\newpage

\begin{homeworkProblem}
    Define the map * in the quaternions by
    \[
        (\alpha_0 + \alpha_1i + \alpha_2j + \alpha_3k)^* = (\alpha_0 - \alpha_1i - \alpha_2j - \alpha_3k).
    \]
    Show that 
    \begin{enumerate}[(a)]
        \item $x^{**} = (x^*)^* = x$.
        \item $(x + y)^* = x^* + y^*$.
        \item $xx^* = x^*x$ is real and nonnegative.
        \item $(xy)^* = y^*x^*$.
    \end{enumerate}
    \begin{proof}
        Let $x = a + bi + cj + dk$, $y = m + yi + wj + zk$. 
        \begin{enumerate}[(a)]
            \item $x^{**} = (a - bi - cj - dk)^* = a + bi + cj + dk = x$
            \item \begin{align*}
                (x + y)^*
                &= ((a + m) + (b + y)i + (c + w)j + (d + z)k)^* \\
                &= (a + m) - (b + y)i - (c + w)j - (d + z)k \\
                &= (a - bi - cj - dk) + (m - yi - wj - zk) = x^* + y^*.
            \end{align*}
            \item $xx^* = (a + bi + cj + dk)(a - bi - cj - dk) = a^2 + b^2 + c^2 + d^2 = (a - bi -
            cj - dk)(a + bi + cj + dk) = x^*x$, which is a sum of squares.

            \item \begin{align*}
                (xy)^*
                &= ((a + bi + cj + dk)(m + yi + wj + zk))^* \\
                &= ((am - by - cw - dz) + (ay + bm - cz + dw)i + (az - bx + cm + dy)j + (aw + bx -
                cy + dm)k)^* \\
                &= (am - by - cw - dz) - (ay + bm - cz + dw)i - (az - bx + cm + dy)j - (aw + bx - cy
                + dm)k, \\
                y^*x^*
                &= am+aw-ayi+azi-bmi-bw-by+bz+cm+cw-cyi+czi+dmi+dw+dy-dz \\
                &= (am+bw+cz+dy)-(ay+bm-cz+dw)i-(az+bx-cm-dy)j-(aw-bx+cy-dm)k,
            \end{align*}
            so $(xy)^* = y^*x^*$.
        \end{enumerate}
    \end{proof}
\end{homeworkProblem}

\newpage

\begin{homeworkProblem}
    If $R$ is an integral domain and $ab = ac$ for $a \neq 0, b, c \in R$, show that $b = c$.

    \begin{proof}
        $ab = ac$ implies $ab - ac = a(b - c) = 0$. Since $R$ is an integral domain and $a \neq 0$,
        we know $b - c = 0$, and so $b = c$.
    \end{proof}
\end{homeworkProblem}

\newpage

\begin{homeworkProblem}
    If $R$ is a finite integral domain, show that $R$ is a field.

    \begin{proof}
        Since $R$ is a ring, $R - \{0\}$ is closed under multiplication, so it suffices to show that
        $R$ is closed under taking inverse. Suppose for the sake of contradiction that $a \neq 0$
        does not have an multiplicative inverse in $R - \{0\}$. Since $R$ is an integral domain, we
        know $a^i \neq 0$ for all positive $i$. Then, $a^i \neq 1$ for finite $i$, which makes $R$
        an infinite group, contradiction. Therefore, $R - \{0\}$ is closed under taking inverse.
    \end{proof}
\end{homeworkProblem}

\newpage

\begin{homeworkProblem}
    If $F$ is a finite field, show that:
    \begin{enumerate}[(a)]
        \item There exists a prime $p$ such that $pa = 0$ for all $a \in F$.
        \begin{proof}
            Denote $[k]$ as $1$ added to itself $k \in \N$ times. Note that $[i][j] = [ij]$, for $i,
            j \in \N$. Then, $$ka = \underbrace{a + a + \dots + a}_{k \text{ times}} =
            (\underbrace{1 + 1 + \dots + 1}_{k \text{ times}})a = [k]a.$$ Since $F$ is finite, there
            exists $k$ such that $[k]a = 0$. Since $F$ is an integral domain, $[k]a = 0$ implies
            $[k] = 0$. Suppose that $k$ is a composite number, say $k = xy$. Then, $[k] = [x][y] =
            0$, so one of $[x], [y]$ is equal to 0. Suppose that $[x] = 0$. We may recursively take
            $[x]$ as our current $[k]$ and decompose it to eventually end up with a prime number $p$
            such that $[p] = 0$, and thus $pa = [p]a = 0$, for all $a \in F$.
        \end{proof}

        \item If $F$ has $q$ elements, then $q = p^n$ for some integer $n$. 
        \begin{proof}
            Since $pa = 0$ for all $a \in F$, all non-identity elements in $F$ are of order $p$
            under addition. Therefore, there does not exists prime number $m \neq p$ that divides
            $q$, otherwise there exists an element of order $m$, by Sylow's Theorem.
        \end{proof}
    \end{enumerate}
\end{homeworkProblem}
\end{document}