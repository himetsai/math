\documentclass{article}

\usepackage{fancyhdr}
\usepackage{extramarks}
\usepackage{amsmath}
\usepackage{amsthm}
\usepackage{amsfonts}
\usepackage{tikz}
\usepackage[plain]{algorithm}
\usepackage{algpseudocode}
\usepackage{enumerate}
\usepackage{amssymb}
\usepackage{dsfont}

\usetikzlibrary{automata,positioning}

%
% Basic Document Settings
%

\topmargin=-0.45in
\evensidemargin=0in
\oddsidemargin=0in
\textwidth=6.5in
\textheight=9.0in
\headsep=0.25in

\linespread{1.1}

\pagestyle{fancy}
\lhead{\hmwkAuthorName}
\chead{\hmwkClass:\ \hmwkTitle}
\rhead{\firstxmark}
\lfoot{\lastxmark}
\cfoot{\thepage}

\renewcommand\headrulewidth{0.4pt}
\renewcommand\footrulewidth{0.4pt}

\setlength\parindent{0pt}
\setlength{\parskip}{5pt}

%
% Create Problem Sections
%

\newcommand{\enterProblemHeader}[1]{
    \nobreak\extramarks{}{Problem \arabic{#1} continued on next page\ldots}\nobreak{}
    \nobreak\extramarks{Problem \arabic{#1} (continued)}{Problem \arabic{#1} continued on next page\ldots}\nobreak{}
}

\newcommand{\exitProblemHeader}[1]{
    \nobreak\extramarks{Problem \arabic{#1} (continued)}{Problem \arabic{#1} continued on next page\ldots}\nobreak{}
    \stepcounter{#1}
    \nobreak\extramarks{Problem \arabic{#1}}{}\nobreak{}
}

\setcounter{secnumdepth}{0}
\newcounter{partCounter}
\newcounter{homeworkProblemCounter}
\setcounter{homeworkProblemCounter}{1}
\nobreak\extramarks{Problem \arabic{homeworkProblemCounter}}{}\nobreak{}

%
% Homework Problem Environment
%
% This environment takes an optional argument. When given, it will adjust the
% problem counter. This is useful for when the problems given for your
% assignment aren't sequential. See the last 3 problems of this template for an
% example.
%
\newenvironment{homeworkProblem}[1][-1]{
    \ifnum#1>0
        \setcounter{homeworkProblemCounter}{#1}
    \fi
    \section{Problem \arabic{homeworkProblemCounter}}
    \setcounter{partCounter}{1}
    \enterProblemHeader{homeworkProblemCounter}
}{
    \exitProblemHeader{homeworkProblemCounter}
}

%
% Homework Details
%   - Title
%   - Due date
%   - Class
%   - Section/Time
%   - Instructor
%   - Author
%

\newcommand{\hmwkTitle}{Homework\ \#4}
\newcommand{\hmwkDueDate}{Feb 3, 2025}
\newcommand{\hmwkClass}{MATH 190A}
\newcommand{\hmwkClassTime}{Section A02 8:00AM - 8:50AM}
\newcommand{\hmwkSectionLeader}{Zhiyuan Jiang}
\newcommand{\hmwkClassInstructor}{Professor McKernan}
\newcommand{\hmwkSource}{Source Consulted: Textbook, Lecture, Discussion}
\newcommand{\hmwkAuthorName}{\textbf{Ray Tsai}}
\newcommand{\hmwkPID}{A16848188}

%
% Title Page
%

\title{
    \vspace{2in}
    \textmd{\textbf{\hmwkClass:\ \hmwkTitle}}\\
    \normalsize\vspace{0.1in}\small{Due\ on\ \hmwkDueDate\ at 12:00pm}\\
    \vspace{0.1in}\large{\textit{\hmwkClassInstructor}} \\
    \vspace{0.1in}\small\hmwkClassTime \\
    \small Section Leader: \hmwkSectionLeader \\
    \vspace{0.1in}\small\hmwkSource \\
    \vspace{3in}
}

\author{
  \hmwkAuthorName \\
  \vspace{0.1in}\small\hmwkPID
}
\date{}

\renewcommand{\part}[1]{\textbf{\large Part \Alph{partCounter}}\stepcounter{partCounter}\\}

%
% Various Helper Commands
%

% Useful for algorithms
\newcommand{\alg}[1]{\textsc{\bfseries \footnotesize #1}}

% For derivatives
\newcommand{\deriv}[1]{\frac{\mathrm{d}}{\mathrm{d}x} (#1)}

% For partial derivatives
\newcommand{\pderiv}[2]{\frac{\partial}{\partial #1} (#2)}

% Integral dx
\newcommand{\dx}{\mathrm{d}x}

% Probability commands: Expectation, Variance, Covariance, Bias
\newcommand{\Var}{\mathrm{Var}}
\newcommand{\Cov}{\mathrm{Cov}}
\newcommand{\Bias}{\mathrm{Bias}}
\newcommand*{\Z}{\mathbb{Z}}
\newcommand*{\Q}{\mathbb{Q}}
\newcommand*{\R}{\mathbb{R}}
\newcommand*{\C}{\mathbb{C}}
\newcommand*{\N}{\mathbb{N}}
\newcommand*{\prob}{\mathds{P}}
\newcommand*{\E}{\mathds{E}}
\newcommand*{\T}{\mathcal{T}}

\begin{document}

\maketitle

\pagebreak

\begin{homeworkProblem}
	Find all topologies on the set
	\[
		X = \{ a, b, c, d \}
	\]
	up to homeomorphism. (This means, give a list of topologies on $X$, such that every other topology on $X$ is homeomorphic to exactly one topology in your list).

	\begin{proof}
		bruh.
		\begin{enumerate}
			\item $\{\emptyset, X\}$
			\item $\{\emptyset, X, \{a, b\}\}$
			\item $\{\emptyset, X, \{a, b, c\}\}$
			\item $\{\emptyset, X, \{a, b\}, \{c, d\}\}$
			\item $\{\emptyset, X, \{a, b\}, \{a, b, c\}\}$
			\item $\{\emptyset, X, \{a, b\}, \{a, b, c\}, \{a, b, d\}\}$
			\item $\{\emptyset, X, \{a\}\}$
			\item $\{\emptyset, X, \{a\}, \{a, b\}\}$
			\item $\{\emptyset, X, \{a\}, \{a, b, c\}\}$
			\item $\{\emptyset, X, \{a\}, \{b, c, d\}\}$
			\item $\{\emptyset, X, \{a\}, \{a, b\}, \{a, b, c\}\}$
			\item $\{\emptyset, X, \{a\}, \{a, d\}, \{a, b, c\}\}$
			\item $\{\emptyset, X, \{a\}, \{b, c\}, \{a, b, c\}\}$
			\item $\{\emptyset, X, \{a\}, \{a, b, c\}, \{a, c, d\}\}$
			\item $\{\emptyset, X, \{a\}, \{b, c\}, \{a, d\}, \{a, b, c\}\}$
			\item $\{\emptyset, X, \{a\}, \{a, b\}, \{a, c\}, \{a, b, c\}\}$
			\item $\{\emptyset, X, \{a\}, \{b, c\}, \{a, b, c\}, \{b, c, d\}\}$
			\item $\{\emptyset, X, \{a\}, \{a, b\}, \{a, b, c\}, \{a, b, d\}\}$
			\item $\{\emptyset, X, \{a\}, \{a, b\}, \{a, c\}, \{a, b, c\}, \{a, b, d\}\}$
			\item $\{\emptyset, X, \{a\}, \{a, b\}, \{a, c\}, \{a, d\}, \{a, b, c\}, \{a, b, d\}, \{a, c, d\}\}$
			\item $\{\emptyset, X, \{a\}, \{b\}, \{a, b\}\}$
			\item $\{\emptyset, X, \{a\}, \{b\}, \{a, b\}, \{a, b, c\}\}$
			\item $\{\emptyset, X, \{a\}, \{b\}, \{a, b\}, \{b, c, d\}\}$
			\item $\{\emptyset, X, \{a\}, \{b\}, \{a, b\}, \{a, c\}, \{a, b, c\}\}$
			\item $\{\emptyset, X, \{a\}, \{b\}, \{a, b\}, \{a, b, c\}, \{a, b, d\}\}$
			\item $\{\emptyset, X, \{a\}, \{b\}, \{b, c\}, \{a, b, c\}, \{b, c, d\}\}$
			\item $\{\emptyset, X, \{a\}, \{b\}, \{a, b\}, \{c, d\}, \{a, c, d\}, \{b, c, d\}\}$
			\item $\{\emptyset, X, \{a\}, \{b\}, \{a, b\}, \{a, c\}, \{a, b, c\}, \{a, b, d\}\}$
			\item $\{\emptyset, X, \{a\}, \{b\}, \{c\}, \{a, b\}, \{a, c\}, \{b, c\}, \{a, b, c\}\}$
			\item $\{\emptyset, X, \{a\}, \{b\}, \{a, b\}, \{b, c\}, \{b, d\}, \{a, b, c\}, \{a, b, d\}, \{b, c, d\}\}$
			\item $\{\emptyset, X, \{a\}, \{b\}, \{c\}, \{a, b\}, \{a, c\}, \{b, c\}, \{a, b, c\}, \{b, c, d\}\}$
			\item $\{\emptyset, X, \{a\}, \{b\}, \{c\}, \{c, d\}, \{a, b\}, \{a, c\}, \{b, c\}, \{a, b, c\}, \{a, c, d\}, \{b, c, d\}\}$
			\item $\wp(X)$
		\end{enumerate}
	\end{proof}
\end{homeworkProblem}

\newpage

\begin{homeworkProblem}
	Let $X$ and $Y$ be two sets. Give both sets the topology where the closed sets are the finite sets, plus the whole set. Under what conditions are $X$ and $Y$ homeomorphic?

	\begin{proof}
		$X$ and $Y$ are homeomorphic if $X$ and $Y$ have the same cardinality. When $X, Y$ are finite, the given topology is just the discrete topology, so they are homeomorphic if and only if $|X| = |Y|$. If $X, Y$ are infinite, let $f: X \to Y$ be a bijection. Then $f$ is continuous since the preimage of any closed set in $Y$ is closed in $X$. The inverse function $f^{-1}: Y \to X$ is also continuous since the preimage of any closed set in $X$ is closed in $Y$. Thus, $f$ is a homeomorphism if $|X| = |Y|$.
	\end{proof}
\end{homeworkProblem}

\newpage

\begin{homeworkProblem}
	Show that any two closed and bounded intervals in $\mathbb{R}$ are homeomorphic.

	\begin{proof}
		By lemma 7.3, it suffices to show that any any closed interval $[a, b]$ is homeomorphic to $[0, 1]$. Let $f: [0, 1] \to [a, b]$ be defined as
		\[
			f(x) = a + (b - a)x.
		\]
		A basis for the subspace topology is given by intervals $(\alpha, \beta), (\alpha, 1]$, $[0, \beta)$, and $[0, 1]$, for $\alpha, \beta \in (0, 1)$ and $\alpha < \beta$, and $f$ sends these to intervals $(a + (b - a)\alpha, a + (b - a)\beta), (a + (b - a)\alpha, b]$, $[a, a + (b - a)\beta)$, and $[a, b]$, respectively. The inverse of $f$ is the function $g: [a, b] \to [0, 1]$ which is defined as
		\[
			g(x) = \frac{x - a}{b - a}.
		\]
		A basis for the subspace topology for $[a, b]$ is given by intervals $(r, s), (r, d]$, $[c, s)$, and $[c, d]$, for $r, s \in (a, b)$, and $g$ sends these to intervals $(\frac{r - a}{b - a}, \frac{s - a}{b - a}), (\frac{r - a}{b - a}, 1]$, $[0, \frac{s - a}{b - a})$, and $[0, 1]$, respectively. Thus, $f$ and $g$ are homeomorphisms by lemma 7.4.
	\end{proof}
\end{homeworkProblem}

\newpage

\begin{homeworkProblem}
	Complete the proof of Theorem 7.2.

	\begin{proof}
		Let $a \in \R$. 
		
		Define $f: (0, \infty) \to (a, \infty)$ to be $f(x) = a + x$. $f$ sends interval $(\alpha, \beta) \subseteq (0, \infty)$ to $(\alpha + a, \beta + a) \subseteq (a, \infty)$. $f$'s inverse $f^{-1}: (a, \infty) \to (0, \infty)$ defined as $f^{-1}(x) = x - a$ sends $(\gamma, \eta) \subseteq (a, \infty)$ to $(\gamma - a, \eta - a) \subseteq (0, \infty)$. Thus, $f$ and $f^{-1}$ are continuous by lemma 7.4, and thus $(0, \infty), (a, \infty)$ are homeomorphic.

		Define $g: (a, \infty) \to (-\infty, a)$ to be $g(x) = -x$. $g$ sends interval $(\alpha, \beta) \subseteq (a, \infty)$ to $(-\beta, -\alpha) \subseteq (-\infty, a)$. $g$'s inverse $g^{-1}: (-\infty, a) \to (a, \infty)$ defined as $g^{-1}(x) = -x$ sends $(\gamma, \eta) \subseteq (-\infty, a)$ to $(-\eta, \gamma) \subseteq (a, \infty)$. Thus, $g$ and $g^{-1}$ are continuous by lemma 7.4, and thus $(a, \infty), (-\infty, a)$ are homeomorphic.

		Define $h: (0, \infty) \to \R$ to be $h(x) = \ln x$. $h$ sends interval $(\alpha, \beta) \subseteq (0, \infty)$ to $(\ln \alpha, \ln \beta) \subseteq \R$. $h$'s inverse $h^{-1}: \R \to (0, \infty)$ defined as $h^{-1}(x) = e^x$ sends $(\gamma, \eta) \subseteq \R$ to $(e^\gamma, e^\eta) \subseteq (0, \infty)$. Thus, $h$ and $h^{-1}$ are continuous by lemma 7.4, and thus $(0, \infty), \R$ are homeomorphic.
	\end{proof}
\end{homeworkProblem}

\newpage

\begin{homeworkProblem}
	True or false? If true then give a proof and if false then give a counterexample.
   \begin{enumerate}[(i)]
			\item If $(X, \mathcal{T})$ and $(Y, \mathcal{S})$ are topological spaces then $X \times Y$ and $Y \times X$, both with the product topology, are homeomorphic.
			\begin{proof}
				True. Consider the map $f: X \times Y \to Y \times X$ that sends $(x, y)$ to $(y, x)$. This map is a bijection. Given any open set $U \times V \subseteq X \times Y$, we have $f(U \times V) = V \times U$ is open in $Y \times X$. Similarly, given any open set $V \times U \subseteq Y \times X$, we have $f^{-1}(V \times U) = U \times V$ is open in $X \times Y$. Thus, $X \times Y$ and $Y \times X$ are homeomorphic.
			\end{proof}
			\item Let $\mathcal{T}$ be the Euclidean topology on $\mathbb{R}$ and let $\mathcal{S}$ be the topology where the open sets are half open intervals of the form $(a, \infty)$. Then $(\mathbb{R}, \mathcal{T})$ and $(\mathbb{R}, \mathcal{S})$ are homeomorphic.
			\begin{proof}
				False. Since $\mathcal{S} \subseteq \mathcal{T}$ but $(0, 1) \in \mathcal{T} \backslash \mathcal{S}$, $\T$ is strictly finer than $\mathcal{S}$.
			\end{proof}
			\item Let $\mathcal{T}$ be the topology where the open sets are half open intervals of the form $(a, \infty)$ and let $\mathcal{S}$ be the topology where the open sets are half open intervals of the form $(-\infty, a)$. Then $(\mathbb{R}, \mathcal{T})$ and $(\mathbb{R}, \mathcal{S})$ are homeomorphic.
			\begin{proof}
				True. Consider the bijection $f: \R \to \R$ that sends $x$ to $-x$. Then the induced map $F: \T \to \mathcal{S}$ is a bijection that sends $(a, \infty)$ to $(-\infty, a)$. Thus, $F$ is a homeomorphism.
			\end{proof}
			\item If $(X, \mathcal{T})$ and $(Y, \mathcal{S})$ are homeomorphic topological spaces then $(X, \mathcal{T})$ is Hausdorff if and only if $(Y, \mathcal{S})$ is Hausdorff.
			\begin{proof}
				True. Let $f: X \to Y$ be a homeomorphism. Let $x_1, x_2 \in X$ be distinct points. Then $f(x_1), f(x_2) \in Y$ are distinct points. Since $(X, \mathcal{T})$ is Hausdorff, there exists disjoint open sets $U_1, U_2 \in \mathcal{T}$ such that $x_1 \in U_1$ and $x_2 \in U_2$. Then $f(U_1), f(U_2) \in \mathcal{S}$ are open sets such that $f(x_1) \in f(U_1)$, $f(x_2) \in f(U_2)$. Note that $f(U_1), f(U_2)$ are disjoint, otherwise $f^{-1}(f(U_1) \cap f(U_2)) = U_1 \cap U_2$ is nonempty. By syymmetry, the converse is also true.
			\end{proof}
			\item If we are given four topological spaces, $X, Y, Z$ and $W$ and $X \times Y$ is homeomorphic to $Z \times W$ then $X$ is homeomorphic to $Z$ or $X$ is homeomorphic to $W$.
			\begin{proof}
				False. Consider $X = Y = \R$, $Z = \R^2$, and $W = \{0\}$. Obviously $X \times Y = \R^2$ is homeomorphic to $Z \times W = \R^2 \times \{0\}$. However, $X$ is not homeomorphic to $Z$ or $W$.
			\end{proof}
   \end{enumerate}
\end{homeworkProblem}
\end{document}