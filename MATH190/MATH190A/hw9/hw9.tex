\documentclass{article}

\usepackage{fancyhdr}
\usepackage{extramarks}
\usepackage{amsmath}
\usepackage{amsthm}
\usepackage{amsfonts}
\usepackage{tikz}
\usepackage[plain]{algorithm}
\usepackage{algpseudocode}
\usepackage{enumerate}
\usepackage{amssymb}
\usepackage{dsfont}

\usetikzlibrary{automata,positioning}

%
% Basic Document Settings
%

\topmargin=-0.45in
\evensidemargin=0in
\oddsidemargin=0in
\textwidth=6.5in
\textheight=9.0in
\headsep=0.25in

\linespread{1.1}

\pagestyle{fancy}
\lhead{\hmwkAuthorName}
\chead{\hmwkClass:\ \hmwkTitle}
\rhead{\firstxmark}
\lfoot{\lastxmark}
\cfoot{\thepage}

\renewcommand\headrulewidth{0.4pt}
\renewcommand\footrulewidth{0.4pt}

\setlength\parindent{0pt}
\setlength{\parskip}{5pt}

%
% Create Problem Sections
%

\newcommand{\enterProblemHeader}[1]{
    \nobreak\extramarks{}{Problem \arabic{#1} continued on next page\ldots}\nobreak{}
    \nobreak\extramarks{Problem \arabic{#1} (continued)}{Problem \arabic{#1} continued on next page\ldots}\nobreak{}
}

\newcommand{\exitProblemHeader}[1]{
    \nobreak\extramarks{Problem \arabic{#1} (continued)}{Problem \arabic{#1} continued on next page\ldots}\nobreak{}
    \stepcounter{#1}
    \nobreak\extramarks{Problem \arabic{#1}}{}\nobreak{}
}

\setcounter{secnumdepth}{0}
\newcounter{partCounter}
\newcounter{homeworkProblemCounter}
\setcounter{homeworkProblemCounter}{1}
\nobreak\extramarks{Problem \arabic{homeworkProblemCounter}}{}\nobreak{}

%
% Homework Problem Environment
%
% This environment takes an optional argument. When given, it will adjust the
% problem counter. This is useful for when the problems given for your
% assignment aren't sequential. See the last 3 problems of this template for an
% example.
%
\newenvironment{homeworkProblem}[1][-1]{
    \ifnum#1>0
        \setcounter{homeworkProblemCounter}{#1}
    \fi
    \section{Problem \arabic{homeworkProblemCounter}}
    \setcounter{partCounter}{1}
    \enterProblemHeader{homeworkProblemCounter}
}{
    \exitProblemHeader{homeworkProblemCounter}
}

%
% Homework Details
%   - Title
%   - Due date
%   - Class
%   - Section/Time
%   - Instructor
%   - Author
%

\newcommand{\hmwkTitle}{Homework\ \#9}
\newcommand{\hmwkDueDate}{Mar 12, 2025}
\newcommand{\hmwkClass}{MATH 190A}
\newcommand{\hmwkClassTime}{Section A02 8:00AM - 8:50AM}
\newcommand{\hmwkSectionLeader}{Zhiyuan Jiang}
\newcommand{\hmwkClassInstructor}{Professor McKernan}
\newcommand{\hmwkSource}{Source Consulted: Textbook, Lecture, Discussion}
\newcommand{\hmwkAuthorName}{\textbf{Ray Tsai}}
\newcommand{\hmwkPID}{A16848188}

%
% Title Page
%

\title{
    \vspace{2in}
    \textmd{\textbf{\hmwkClass:\ \hmwkTitle}}\\
    \normalsize\vspace{0.1in}\small{Due\ on\ \hmwkDueDate\ at 12:00pm}\\
    \vspace{0.1in}\large{\textit{\hmwkClassInstructor}} \\
    \vspace{0.1in}\small\hmwkClassTime \\
    \small Section Leader: \hmwkSectionLeader \\
    \vspace{0.1in}\small\hmwkSource \\
    \vspace{3in}
}

\author{
  \hmwkAuthorName \\
  \vspace{0.1in}\small\hmwkPID
}
\date{}

\renewcommand{\part}[1]{\textbf{\large Part \Alph{partCounter}}\stepcounter{partCounter}\\}

%
% Various Helper Commands
%

% Useful for algorithms
\newcommand{\alg}[1]{\textsc{\bfseries \footnotesize #1}}

% For derivatives
\newcommand{\deriv}[1]{\frac{\mathrm{d}}{\mathrm{d}x} (#1)}

% For partial derivatives
\newcommand{\pderiv}[2]{\frac{\partial}{\partial #1} (#2)}

% Integral dx
\newcommand{\dx}{\mathrm{d}x}

% Probability commands: Expectation, Variance, Covariance, Bias
\newcommand{\Var}{\mathrm{Var}}
\newcommand{\Cov}{\mathrm{Cov}}
\newcommand{\Bias}{\mathrm{Bias}}
\newcommand*{\Z}{\mathbb{Z}}
\newcommand*{\Q}{\mathbb{Q}}
\newcommand*{\R}{\mathbb{R}}
\newcommand*{\C}{\mathbb{C}}
\newcommand*{\N}{\mathbb{N}}
\newcommand*{\prob}{\mathds{P}}
\newcommand*{\E}{\mathds{E}}
\newcommand*{\T}{\mathcal{T}}

\begin{document}

\maketitle

\pagebreak

\begin{homeworkProblem}
	Show that an arbitrary product of Hausdorff spaces is Hausdorff.

	\begin{proof}
		Let $\{X\alpha\}$ be a collection of Hausdorff spaces and consider 
		\[
			X = \prod_{\alpha} X_\alpha.
		\]
		Let $x, y \in X$ such that $x \neq y$. Then there exists some index $\beta$ such that $x_\beta \neq y_\beta$. Since $X_\beta$ is Hausdorff, there exists disjoint open sets $U, V \subset X_\beta$ such that $x_\beta \in U$ and $y_\beta \in V$. Let $p_\beta: X \to X_\beta$ be the natural projection map. Then $p_\beta^{-1}(U)$ and $p_\beta^{-1}(V)$ are open in $X$ and $x \in p_\beta^{-1}(U)$. $y \in p_\beta^{-1}(V)$. Moreover, $p_\beta^{-1}(U)$ and $p_\beta^{-1}(V)$ are disjoint and $U, V$ are joint, so $X$ is Hausdorff.
	\end{proof}
\end{homeworkProblem}

\newpage

\begin{homeworkProblem}
	Show that an arbitrary product of connected sets is connected.

	\begin{proof}
		Let $\{X_\alpha\}$ be a collection of connected spaces and consider 
		\[
			X = \prod_{\alpha \in \Lambda} X_\alpha.
		\]
		First note that since $X_\alpha$ is connected, $\prod_{\alpha \in F} X_\alpha$ is connected for all finite $F \subseteq \Lambda$. Suppose for sake of contradiction that $X$ may be partitioned into two disjoint nonempty open sets $U, V$. Fix $a \in X$. For $\alpha \in \Lambda$, let 
		\[
			Z_\alpha = \{x \in X \mid x_\beta = a_\beta, \beta \neq \alpha\},
		\]
		the set of points in $X$ that agree with $a$ in all coordinates except $\alpha$. Then $f_\alpha: X_\alpha \to Z_\alpha$ which sends $k$ to $(x_\beta)$ where $x_\beta = a_\beta$ for $\beta \neq \alpha$ and $x_\alpha = k$ is a homeomorphism. Since $X_\alpha$ is connected, $Z_\alpha$ is connected. Hence, for each $\alpha \in \Lambda$, $Z_\alpha$ is contained in either $U$ or $V$. Since $U, V$ are nonempty, we may find $Z_{\alpha_1}, Z_{\alpha_2}$ such that $Z_{\alpha_1} \subseteq U$ and $Z_{\alpha_2} \subseteq V$. Now consider $W = Z_{\alpha_1} \cup Z_{\alpha_2} \subset X$. Then the function $g: X_{\alpha_1} \times X_{\alpha_2} \to W$ which sends $(k, m)$ to $(x_\beta)$ where $x_\beta = a_\beta$ for $\beta \neq \alpha_1, \alpha_2$ and $(x_{\alpha_1}, x_{\alpha_2}) = (k, m)$ is a homeomorphism. But then this implies that $W = (W \cap U) \cup (W \cap V)$ is connected, contradiction.
	\end{proof}
\end{homeworkProblem}

\newpage

\begin{homeworkProblem}
	Let $f: X \to Y$ be a continuous and surjective function.

\begin{enumerate}[(i)]
    \item Show that $f$ is a quotient map if and only if $Y$ has the finest topology such that $f$ is continuous (that is, $V \subset Y$ is open if and only if $U = f^{-1}(V)$ is open).
    \begin{proof}
			Suppose $f$ is a quotient map. 
		\end{proof}
    
    \item Show that if $f$ is open then $f$ is a quotient map.
    
    \item Show that if $f$ is closed then $f$ is a quotient map.
    
    \item Show that if $X$ is compact and $Y$ is Hausdorff then $f$ is a quotient map.
    
    \item Show that if $f$ has a right inverse, that is, there is a continuous function $g: Y \to X$ such that $f \circ g: Y \to Y$ is the identity, then $f$ is a quotient map.
    
    \item Show that if $p: X \times Y \to Y$ is the projection then $p$ is a quotient map.
    
    \item Let $X$ be a topological space and let $Y \subset X$ be a subspace. We say that $r$ is a \textit{retraction} of $X$ onto $Y$ if $r: X \to Y$ is a continuous map whose restriction to $Y$ is the identity, $r|_Y = i_Y$.\\
    If $r: X \to Y$ is a retraction then show that $r$ is a quotient map.
    
    \item Show that the composition of quotient maps is a quotient map.
\end{enumerate}
\end{homeworkProblem}

\newpage

\begin{homeworkProblem}
	Let
	\[
	X = \{(x,y) \mid \text{either } x \geq 0 \text{ or } y = 0 \} \subset \mathbb{R}^2.
	\]
	Let $p: \mathbb{R}^2 \to \mathbb{R}$ denote projection onto the first factor and let
	\[
	f: X \to \mathbb{R}
	\]
	be the restriction of $p$ to $X$.

	Show that $f$ is a quotient map, even though $f$ is neither open nor closed and $X$.

\end{homeworkProblem}

\newpage

\begin{homeworkProblem}
	Consider the Möbius strip $X_1$ and the closed unit ball $X_2$ in $\mathbb{R}^2$. The boundary of both is homeomorphic to $S^1$ (in fact the boundary of the closed unit ball is $S^1$). Define a topological space $X$ by taking the quotient of the disjoint union of $X_1$ and $X_2$, modulo an equivalence relation that identifies the two boundaries (in other words, pick a homeomorphism between the two boundaries and identify corresponding points).

	Show that $X$ is homeomorphic to the real projective plane.
\end{homeworkProblem}
\end{document}