\documentclass{article}

\usepackage{fancyhdr}
\usepackage{extramarks}
\usepackage{amsmath}
\usepackage{amsthm}
\usepackage{amsfonts}
\usepackage{tikz}
\usepackage[plain]{algorithm}
\usepackage{algpseudocode}
\usepackage{enumerate}
\usepackage{amssymb}
\usepackage{dsfont}

\usetikzlibrary{automata,positioning}

%
% Basic Document Settings
%

\topmargin=-0.45in
\evensidemargin=0in
\oddsidemargin=0in
\textwidth=6.5in
\textheight=9.0in
\headsep=0.25in

\linespread{1.1}

\pagestyle{fancy}
\lhead{\hmwkAuthorName}
\chead{\hmwkClass:\ \hmwkTitle}
\rhead{\firstxmark}
\lfoot{\lastxmark}
\cfoot{\thepage}

\renewcommand\headrulewidth{0.4pt}
\renewcommand\footrulewidth{0.4pt}

\setlength\parindent{0pt}
\setlength{\parskip}{5pt}

%
% Create Problem Sections
%

\newcommand{\enterProblemHeader}[1]{
    \nobreak\extramarks{}{Problem \arabic{#1} continued on next page\ldots}\nobreak{}
    \nobreak\extramarks{Problem \arabic{#1} (continued)}{Problem \arabic{#1} continued on next page\ldots}\nobreak{}
}

\newcommand{\exitProblemHeader}[1]{
    \nobreak\extramarks{Problem \arabic{#1} (continued)}{Problem \arabic{#1} continued on next page\ldots}\nobreak{}
    \stepcounter{#1}
    \nobreak\extramarks{Problem \arabic{#1}}{}\nobreak{}
}

\setcounter{secnumdepth}{0}
\newcounter{partCounter}
\newcounter{homeworkProblemCounter}
\setcounter{homeworkProblemCounter}{1}
\nobreak\extramarks{Problem \arabic{homeworkProblemCounter}}{}\nobreak{}

%
% Homework Problem Environment
%
% This environment takes an optional argument. When given, it will adjust the
% problem counter. This is useful for when the problems given for your
% assignment aren't sequential. See the last 3 problems of this template for an
% example.
%
\newenvironment{homeworkProblem}[1][-1]{
    \ifnum#1>0
        \setcounter{homeworkProblemCounter}{#1}
    \fi
    \section{Problem \arabic{homeworkProblemCounter}}
    \setcounter{partCounter}{1}
    \enterProblemHeader{homeworkProblemCounter}
}{
    \exitProblemHeader{homeworkProblemCounter}
}

%
% Homework Details
%   - Title
%   - Due date
%   - Class
%   - Section/Time
%   - Instructor
%   - Author
%

\newcommand{\hmwkTitle}{Homework\ \#5}
\newcommand{\hmwkDueDate}{Feb 12, 2025}
\newcommand{\hmwkClass}{MATH 190A}
\newcommand{\hmwkClassTime}{Section A02 8:00AM - 8:50AM}
\newcommand{\hmwkSectionLeader}{Zhiyuan Jiang}
\newcommand{\hmwkClassInstructor}{Professor McKernan}
\newcommand{\hmwkSource}{Source Consulted: Textbook, Lecture, Discussion}
\newcommand{\hmwkAuthorName}{\textbf{Ray Tsai}}
\newcommand{\hmwkPID}{A16848188}

%
% Title Page
%

\title{
    \vspace{2in}
    \textmd{\textbf{\hmwkClass:\ \hmwkTitle}}\\
    \normalsize\vspace{0.1in}\small{Due\ on\ \hmwkDueDate\ at 12:00pm}\\
    \vspace{0.1in}\large{\textit{\hmwkClassInstructor}} \\
    \vspace{0.1in}\small\hmwkClassTime \\
    \small Section Leader: \hmwkSectionLeader \\
    \vspace{0.1in}\small\hmwkSource \\
    \vspace{3in}
}

\author{
  \hmwkAuthorName \\
  \vspace{0.1in}\small\hmwkPID
}
\date{}

\renewcommand{\part}[1]{\textbf{\large Part \Alph{partCounter}}\stepcounter{partCounter}\\}

%
% Various Helper Commands
%

% Useful for algorithms
\newcommand{\alg}[1]{\textsc{\bfseries \footnotesize #1}}

% For derivatives
\newcommand{\deriv}[1]{\frac{\mathrm{d}}{\mathrm{d}x} (#1)}

% For partial derivatives
\newcommand{\pderiv}[2]{\frac{\partial}{\partial #1} (#2)}

% Integral dx
\newcommand{\dx}{\mathrm{d}x}

% Probability commands: Expectation, Variance, Covariance, Bias
\newcommand{\Var}{\mathrm{Var}}
\newcommand{\Cov}{\mathrm{Cov}}
\newcommand{\Bias}{\mathrm{Bias}}
\newcommand*{\Z}{\mathbb{Z}}
\newcommand*{\Q}{\mathbb{Q}}
\newcommand*{\R}{\mathbb{R}}
\newcommand*{\C}{\mathbb{C}}
\newcommand*{\N}{\mathbb{N}}
\newcommand*{\prob}{\mathds{P}}
\newcommand*{\E}{\mathds{E}}
\newcommand*{\T}{\mathcal{T}}

\begin{document}

\maketitle

\pagebreak

\begin{homeworkProblem}
  The purpose of this exercise is to build up as many of the familiar functions from analysis as we can, using just the definitions and some results from the course.  

	Let $X$ and $Y$ be topological spaces.  

	\begin{enumerate}[(i)]
    \item If $b \in Y$ then show that the constant function  
    \[
    f: X \to Y \quad \text{given by} \quad f(x) = b,
    \]
    is continuous.

		\begin{proof}
			Let $V$ be an open set in $Y$. If $b \in V$, then $f^{-1}(V) = X$. Otherwise, $f^{-1}(V) = \emptyset$. In either case, $f^{-1}(V)$ is open in $X$, and thus $f$ is continuous.
		\end{proof}

    \item If $m \in \mathbb{R}$ is a real then show that the function  
    \[
    f: \mathbb{R} \to \mathbb{R} \quad \text{given by} \quad f(x) = mx
    \]
    is continuous.

		\begin{proof}
			Let $V = (a, b) \subseteq \R$. Then $f^{-1}(V)$ is either $(a/m, b/m)$ or $(b/m, a/m)$, both of which are open in $\R$. Thus, $f$ is continuous.
		\end{proof}

    \item Show that the function  
    \[
    a: \mathbb{R}^2 \to \mathbb{R} \quad \text{given by} \quad a(x,y) = x + y,
    \]
    is continuous.

		\begin{proof}
			Write $a(x, y) = p_1(x, y) + p_2(x, y)$, where $p_1, p_2$ are coordinate projections. Then $p_1, p_2$ are continuous. Since addition is continuous in $\R$, $a$ is continuous.
		\end{proof}

    \item If $f: X \to \mathbb{R}$ and $g: Y \to \mathbb{R}$ are continuous functions then show that the function  
    \[
    f + g: X \to \mathbb{R} \quad \text{given by} \quad (f+g)(x) = f(x) + g(x)
    \]
    is continuous.

		\begin{proof}
			Define $h: X \to \R^2$ as $h(x) = (f(x), g(x))$, and write $f + g = a \circ h$, where $a$ is the addition function from part (iii). Since $h$ is continuous by the continuity of $f, g$, and $a$ is continuous, $f + g$ is continuous by the composition of continuous functions.
		\end{proof}

    \item Show that the function  
    \[
    m: \mathbb{R}^2 \to \mathbb{R} \quad \text{given by} \quad m(x,y) = xy,
    \]
    is continuous.

		\begin{proof}
			Write $m(x, y) = p_1(x, y)p_2(x, y)$, where $p_1, p_2$ are coordinate projections. Then $p_1, p_2$ are continuous. Since multiplication is continuous in $\R$, $m$ is continuous.
		\end{proof}

    \item If $f: X \to \mathbb{R}$ and $g: X \to \mathbb{R}$ are continuous functions then show that the function  
    \[
    fg: X \to \mathbb{R} \quad \text{given by} \quad (fg)(x) = f(x)g(x)
    \]
    is continuous.

		\begin{proof}
			Define $h: X \to \R^2$ as $h(x) = (f(x), g(x))$, and write $fg = m \circ h$, where $m$ is the addition function from the last part. Since $h$ is continuous by the continuity of $f, g$, and $m$ is continuous, $fg$ is continuous by the composition of continuous functions.
		\end{proof}

    \item Show that the function  
    \[
    f: \mathbb{R} \to \mathbb{R} \quad \text{given by} \quad x \mapsto x^n
    \]
    is continuous.

		\begin{proof}
			The last part establishes that $x^n = x^{n - 1} \cdot x$ is continuous if $x^{n - 1}$ is continuous. Since $x^0 = 1$ is continuous, $x^n$ is continuous for all $n \in \N$ by induction.
		\end{proof}

    \item Any polynomial function  
    \[
    f: \mathbb{R} \to \mathbb{R} \quad \text{given by} \quad f(x) = a_n x^n + a_{n-1} x^{n-1} + \dots + a_0,
    \]
    where $a_0, a_1, \dots, a_n$ are real numbers, is continuous.

		\begin{proof}
			By the previous part, $x^n$ is continuous for all $n \in \N$. Since addition and linear functions are continuous, the polynomial function is continuous.
		\end{proof}

    \item The function  
    \[
    f: \mathbb{R}^* \to \mathbb{R} \quad \text{given by} \quad f(x) = \frac{1}{x}
    \]
    is continuous, where  
    \[
    \mathbb{R}^* = \mathbb{R} \setminus \{0\}.
    \]

		\begin{proof}
			Let $V = (a, b) \subseteq \R$. Since $0$ is not in the range of $f$, we may assume $(a, b) = (a^-, b^-) \cup (a^+, b^+)$, where $a^-, b^- < 0$ and $a^+, b^+ > 0$. Then $f^{-1}(V) = (1/b^-, 1/a^-) \cup (1/b^+, 1/a^+)$, which is open in $\R^*$. Thus, $f$ is continuous.
		\end{proof}

    \item Let  
    \[
    g(x) = \sum b_m x^m + b_{m-1} x^{m-1} + \dots + b_0,
    \]
    be a polynomial function, where $b_0, b_1, \dots, b_m$ are real numbers. Show that the set of points where $g$ is non-zero is an open subset $U$ of $\mathbb{R}$. Show that the rational function  
    \[
    \phi: U \to \mathbb{R} \quad \text{given by} \quad \phi(x) = \frac{a_n x^n + a_{n-1} x^{n-1} + \dots + a_0}{b_m x^m + b_{m-1} x^{m-1} + \dots + b_0}
    \]
    is continuous.

		\begin{proof}
			Since $g$ is a polynomial function, it is continuous, and so $U = g^{-1}(\R^*)$ is open in $\R$. Let $f$ be the inversion function defined in the last part and $k(x) = a_nx^n + \cdots + a_0$. Then $\phi = k(\phi \circ g)$ is continuous by the results of the previous parts.
		\end{proof}
  \end{enumerate}
\end{homeworkProblem}

\newpage

\begin{homeworkProblem}
	Let $f: X \to Y$ be any map of sets whose image lies in a subset $B$ of $Y$. Note that there is a unique function $g: X \to B$ with the property that $f = j \circ g$, where $j: B \to Y$ is the natural inclusion.  

	Now suppose that $X$ and $Y$ are topological spaces and give $B$ the subspace topology.  

	Show that $f$ is continuous if and only if $g$ is continuous.

	\begin{proof}
		First note that $j$ is continuous. Let $V$ be open in $Y$. Then $f$ is continuous if and only if $f^{-1}(V) = g^{-1}(j^{-1}(V)) = U$ is open in $X$ if and only if $g^{-1}(V \cap B) = U$ is open in $X$ if and only if $g$ is continuous.
	\end{proof}
\end{homeworkProblem}

\newpage

\begin{homeworkProblem}
	True or false? If true then give a proof and if false then give a counterexample.

	\begin{enumerate}[(i)]
    \item If $f: X \to Y$ is a continuous map of topological spaces and $f$ is a bijection then $f$ is a homeomorphism.
    \begin{proof}
			Let $X = \R$ be the topological space with discrete topology and $Y = \R$ be the topological space with the Euclidean topology. Consider the identity function $i: X \to Y$. Since the discrete topology is strictly finer than the Euclidean topology, $i$ is continuous but $i^{-1}$ is not.
		\end{proof}

    \item If $f: X \to Y$ is a continuous function between two topological spaces then the inverse image of every closed set is closed.
    
		\begin{proof}
			True. Let $V$ be closed in $Y$. Then $f^{-1}(Y \backslash V) = X \backslash f^{-1}(V)$ is open in $X$, so $f^{-1}(V)$ is closed in $X$.
		\end{proof}
    
    \item If $f: X \to Y$ is a function between two topological spaces and the inverse image of every closed set is closed then $f$ is continuous.
    
		\begin{proof}
			True. Let $V$ be open in $Y$. Then $f^{-1}(Y \backslash V) = X \backslash f^{-1}(V)$ is closed in $X$. That is, $f^{-1}(V)$ is open in $X$. Thus, $f$ is continuous.
		\end{proof}

    \item The function
    \[
    	f: \mathbb{R} \to \mathbb{R}
    \]
    defined as
    \[
			f(x) =
			\begin{cases} 
			0 & \text{if } x \leq 0 \\
			1 & \text{otherwise}
			\end{cases}
    \]
    is continuous.

		\begin{proof}
			False. Consider $V = (-1, 1)$. Then $f^{-1}(V) = (-\infty, 0]$, which is not open in $\R$.
		\end{proof}

    \item If $f: X \to Y$ is a continuous function between two topological spaces and $A \subset X$ then
    \[
    	f(\overline{A}) = \overline{f(A)}.
    \]
		\begin{proof}
			False. Consider $f(x) = e^x$ and let $A = (-\infty, 0)$. Then $\overline{A} = (-\infty, 0]$ so $f(\overline{A})=(0, 1]$. But then $\overline{f(A)} = [0, 1]$.
		\end{proof}

    \item If $f: X \to Y$ is a continuous function between two topological spaces and $A \subset X$ then
    \[
    	f(\overline{A}) \subset \overline{f(A)}.
    \]

		\begin{proof}
			True. Let $y \in f(\overline{A})$. Then there exists $x \in \overline{A}$ such that $f(x) = y$. Let $V$ be an open set containing $y$. Then $f^{-1}(V)$ is open and contains $x$. But then $x \in \overline{A}$, so $f^{-1}(V)$ intersects $A$. Thus, every open set containing $y$ intersects $f(A)$, so $y \in \overline{f(A)}$.
		\end{proof}
	\end{enumerate}
\end{homeworkProblem}

\newpage

\begin{homeworkProblem}
	Let  
	\[
		S^n = \{ x \in \mathbb{R}^{n+1} \mid ||x||^2 = 1 \}
	\]
	be the unit sphere.  

	\begin{enumerate}[(i)]
    \item Show that $S^n$ is a closed subset of $\mathbb{R}^{n+1}$. Let  
    \[
    N = (0, 0, 0, \dots, 0, 1) \in \mathbb{R}^{n+1}.
    \]
    Show that $N \in S^n$ ($N$ is sometimes called the North pole). Show that $U = S^n - \{N\}$ is an open subset of $S^n$. Let $H$ be the hyperplane determined by the equation $x_{n+1} = 0$,  
    \[
    H = \{ (a_1, a_2, \dots, a_{n+1}) \mid a_{n+1} = 0 \}.
    \]
    Show that $H$ is homeomorphic to $\mathbb{R}^n$.

		\begin{proof}
			Let $x \in \R^{n + 1} \backslash S^n$. Then $||x||^2 \neq 1$, so $||x||^2 > 1$ or $||x||^2 < 1$. In the former case, let $r = ||x||^2 - 1$. Then $B_r(x) \subseteq \R^{n + 1} \backslash S^n$, so $\R^{n + 1} \backslash S^n$ is open. In the latter case, let $r = 1 - ||x||^2$. Then $B_r(x) \subseteq \R^{n + 1} \backslash S^n$, so $\R^{n + 1} \backslash S^n$ is open. Thus, $S^n$ is closed.

			Let $P = \{x \in \R^{n + 1} \mid x_{n + 1} < 1\}$. If $x \in P$, then there exists $r \in (0, 1 - x_{n + 1})$ such that $B_{r}(x) \subset P$. Thus, $P$ is open. But then $U = S^n \cap P$, so $U$ is open in $S^n$.

			Define $f: \R^{n + 1} \to \R^n$ as $f(a_1, \dots, a_n, a_{n + 1}) = (a_1, \dots, a_n)$. $f$ is a projection, so it is continuous. Now consider $f|_H$. $f|_H$ is a bijection, and its inverse is $f|_H^{-1}(a_1, \dots, a_n) = (a_1, \dots, a_n, 0)$ is obviously continuous. Thus, $f|_H$ is a homeomorphism.
		\end{proof}

    \item Define a function  
    \[
    f: U \to H
    \]
    as follows. Consider the straight line connecting the north pole $N$ to the point $P \in \mathbb{R}^{n+1} \setminus \{N\}$. This line intersects $H$ in a unique point $Q$. If $P \in U$ then we send $P$ to this unique point $Q$. $f$ is called \textbf{Stereographic projection}.  

    Write down a formula for $f$ and show that $f$ is continuous.

		\begin{proof}
			For $P = (x_1, x_2, \dots, x_n, x_{n+1}) \in U$, define
			\[
			f(P) = \left( \frac{x_1}{1-x_{n+1}},\, \frac{x_2}{1-x_{n+1}},\, \dots,\, \frac{x_n}{1-x_{n+1}} \right).
			\]
			Note that for every point in $U$, we have $x_{n+1} \neq 1$, which ensures $1 - x_{n + 1} \neq 0$. 

			For continuity, note that each component of $f$ is given by
			\[
				x_i \mapsto \frac{x_i}{1-x_{n+1}},
			\]
			which is a composition of continuous functions. Hence, $f$ is continuous.
		\end{proof}

    \item Show that $f$ is a homeomorphism of $U$ with $H$, so that $U$ is homeomorphic to $\mathbb{R}^n$.
    
		\begin{proof}
		For $y = (y_1,y_2,\dots,y_n) \in H$, define
		\[
		f^{-1}(y_1,y_2,\dots,y_n)=\left(\frac{2y_1}{1+\|y\|^2},\frac{2y_2}{1+\|y\|^2},\dots,\frac{2y_n}{1+\|y\|^2},\frac{\|y\|^2-1}{1+\|y\|^2}\right).
		\]
		A direct but uninformative calculation shows that for $x \in U$, $f^{-1}(f(x)) = x$, and for $y \in H$, $f(f^{-1}(y))=y$. Each components of $f$ and $f^{-1}$ are compositions of continuous functions, so both maps are continuous. The result now follows.
		\end{proof}
	\end{enumerate}
\end{homeworkProblem}

\newpage

\begin{homeworkProblem}
	Let  
	\[
		Y = \overline{B}_1(0) = \{ (x,y) \in \mathbb{R}^2 \mid x^2 + y^2 \leq 1 \}
	\]
	be the closed unit ball, centred at the origin. Let $a$ and $b$ be any two points on the unit circle, the boundary of $Y$. Let $C_{a,b}$ be the part of the boundary going from $a$ to $b$ (going anti-clockwise), so that $C_{a,b}$ is an arc of the unit circle including $a$ and $b$. Let  
	\[
		U_{a,b} = Y \setminus C_{a,b}.
	\]
	Show that the homeomorphism type of $U_{a,b}$ does not depend on $a$ and $b$ (even allowing $a = b$). (Hint: you may use the fact that $\cos$ and $\sin$ are continuous functions).

	\begin{proof}
		It suffices to show that $U_{a,b}$ is homeomorphic to $T \backslash \{(1, 0)\}$ for all $a, b \in \partial Y$. Note that since $\sin$ and $\cos$ are homeomorphic, the rotation linear transformation on $U_{a,b}$ is homeomorphic. Hence, we may fix $b = (1, 0)$. Suppose $a = (\cos \theta_a, \sin \theta_a)$ for some $\theta_a \in (0, 2\pi]$. Define $g_a: [0, \theta_a] \to [0, 2\pi]$ as $g_a(\theta) = \frac{2\pi}{\theta_a}\theta$. Note that $g_a(0) = 0$ and $g_a(\theta_a) = 2\pi$. Since $g_a$ is a linear function, $g_a$ is a homeomorphism. Now consider the function $f_a: U_{a, (1, 0)} \to Y \backslash \{(1, 0)\}$ that sends $(r\cos\theta, r\sin \theta)$ to $(r \cos g_a(\theta), r \sin g_a(\theta))$. $f_a$ has an inverse $f_a^{-1}: Y \backslash \{(1, 0)\} \to U_{a, (1, 0)}$ that sends $(r\cos\theta, r\sin \theta)$ to $(r\cos g_a^{-1}(\theta), r\sin g_a^{-1}(\theta))$. Since $r, \sin. g_a, g_a^{-1}$ are continuous, $f_a$ and $f_a^{-1}$ are continuous. Thus, $U_{a, (1, 0)}$ is homeomorphic to $Y \backslash \{(1, 0)\}$.
	\end{proof}
\end{homeworkProblem}
\end{document}