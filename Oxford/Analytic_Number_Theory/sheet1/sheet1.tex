\documentclass{article}

% Packages
\usepackage{fancyhdr}
\usepackage{extramarks}
\usepackage{amsmath}
\usepackage{amsthm}
\usepackage{amsfonts}
\usepackage{tikz}
\usepackage[plain]{algorithm}
\usepackage{algpseudocode}
\usepackage{enumerate}

\usetikzlibrary{automata,positioning}

% Document Layout
\topmargin=-0.45in
\evensidemargin=0in
\oddsidemargin=0in
\textwidth=6.5in
\textheight=9.0in
\headsep=0.25in
\linespread{1.1}

% Page Style
\pagestyle{fancy}
\lhead{\hmwkAuthorName}
\chead{\hmwkClass:\ \hmwkTitle}
\rhead{Section \hmwkSection, \firstxmark}
\lfoot{\lastxmark}
\cfoot{\thepage}
\renewcommand\headrulewidth{0.4pt}
\renewcommand\footrulewidth{0.4pt}

% Paragraph Settings
\setlength\parindent{0pt}
\setlength{\parskip}{5pt}

% Section Management
\newcommand{\hmwkSection}{A} % Current section (A, B, or C) - update manually

% Problem Header Management
\newcommand{\enterProblemHeader}[1]{
  \nobreak\extramarks{}{Problem \arabic{#1} continued on next page\ldots}\nobreak{}
  \nobreak\extramarks{Problem \arabic{#1} (continued)}{Problem \arabic{#1} continued on next page\ldots}\nobreak{}
}

\newcommand{\exitProblemHeader}[1]{
  \nobreak\extramarks{Problem \arabic{#1} (continued)}{Problem \arabic{#1} continued on next page\ldots}\nobreak{}
  \stepcounter{#1}
  \nobreak\extramarks{Problem \arabic{#1}}{}\nobreak{}
}

% Counters
\setcounter{secnumdepth}{0}
\newcounter{partCounter}
\newcounter{homeworkProblemCounter}
\setcounter{homeworkProblemCounter}{1}
\nobreak\extramarks{Problem \arabic{homeworkProblemCounter}}{}\nobreak{}

% Homework Problem Environment
% Optional argument adjusts problem counter for non-sequential problems
\newenvironment{homeworkProblem}[1][-1]{
  \ifnum#1>0
    \setcounter{homeworkProblemCounter}{#1}
  \fi
  \section{Problem \arabic{homeworkProblemCounter}}
  \setcounter{partCounter}{1}
  \enterProblemHeader{homeworkProblemCounter}
}{
  \exitProblemHeader{homeworkProblemCounter}
}

% Assignment Details
\newcommand{\hmwkTitle}{Sheet\ \#1}
\newcommand{\hmwkDueDate}{October 15, 2025}
\newcommand{\hmwkClass}{C3.8 Analytic Number Theory}
\newcommand{\hmwkClassInstructor}{Professor B. Green}
\newcommand{\hmwkAuthorName}{\textbf{Ray Tsai}}

% Title Page
\title{
  \vspace{2in}
  \textmd{\textbf{\hmwkClass:\ \hmwkTitle}}\\
  \normalsize\vspace{0.1in}\small{Due\ on\ \hmwkDueDate\ at 12:00pm}\\
  \vspace{0.1in}\large{\textit{\hmwkClassInstructor}} \\
  \vspace{3in}
}

\author{\hmwkAuthorName}
\date{}

% Part Command
\renewcommand{\part}[1]{\textbf{\large Part \Alph{partCounter}}\stepcounter{partCounter}\\}

% Mathematical Commands
% Algorithms
\newcommand{\alg}[1]{\textsc{\bfseries \footnotesize #1}}

% Calculus
\newcommand{\deriv}[1]{\frac{\mathrm{d}}{\mathrm{d}x} (#1)}
\newcommand{\pderiv}[2]{\frac{\partial}{\partial #1} (#2)}
\newcommand{\dx}{\mathrm{d}x}

% Probability and Statistics
\newcommand{\Var}{\mathrm{Var}}
\newcommand{\Cov}{\mathrm{Cov}}
\newcommand{\Bias}{\mathrm{Bias}}
\newcommand*{\prob}{\mathds{P}}
\newcommand*{\E}{\mathds{E}}

% Number Sets
\newcommand*{\Z}{\mathbb{Z}}
\newcommand*{\Q}{\mathbb{Q}}
\newcommand*{\R}{\mathbb{R}}
\newcommand*{\C}{\mathbb{C}}
\newcommand*{\N}{\mathbb{N}}

\begin{document}

\maketitle
\pagebreak

\begin{homeworkProblem}
  Prove the following.
  
  \begin{enumerate}[(i)]
    \item $(\log X)^4 < X^{1/10}$ for all sufficiently large $X$.
    \begin{proof}
      By L'Hopital's rule,
      \[
        \lim_{X \to \infty} \frac{X}{e^{X^{1/40}}} = \lim_{X \to \infty} \frac{40}{X^{-39/40} e^{X^{1/40}}} = 0.
      \]
      Thus, $X < e^{X^{1/40}}$ for all sufficiently large $X$. The result now follows from taking logarithms on both sides.
    \end{proof}
    
    \item $e^{\sqrt{\log X}} = O_\varepsilon(X^\varepsilon)$ for all $\varepsilon > 0$ and $X \geq 1$.
    
    \begin{proof}
      Fix $\varepsilon > 0$. Since
      \[
        \lim_{X \to \infty} \frac{X^\varepsilon}{e^{\sqrt{\log X}}} = \lim_{X \to \infty} \frac{e^{\varepsilon \log X}}{e^{\sqrt{\log X}}} = \lim_{Y \to \infty} e^{Y(\varepsilon Y - 1)}.
      \]
      Since $Y(\varepsilon Y - 1) \to \infty$ as $Y \to \infty$, the result now follows.
    \end{proof}
    
    \item $X(1 + e^{-\sqrt{\log X}}) + X^{3/4} \sin X \sim X$.
    \begin{proof}
      First note that $|X^{3/4}\sin X| \leq X^{3/4} = o(X)$, and $e^{-\sqrt{\log X}} = o(1)$. Hence, $X(1 + e^{-\sqrt{\log X}}) + X^{3/4} \sin X = (1 + o(1))X$.
    \end{proof}
  \end{enumerate}
\end{homeworkProblem}

\newpage

\begin{homeworkProblem}
  In the following exercise, $a(X)$, $b(X)$ are positive functions tending to $\infty$ as $X \to \infty$. Say whether each of the following is true or false.
  
  \begin{enumerate}[(i)]
    \item If $a(X) - b(X) \to 0$ then $a(X) \sim b(X)$.
    
    \begin{proof}
      True, as
      \[
        \left|\frac{a(X)}{b(X)} - 1\right| = \left|\frac{a(X) - b(X)}{b(X)}\right| \to 0.
      \]
    \end{proof}
    
    \item If $a(X) \sim b(X)$ then $a(X) - b(X) \to 0$.
    
    \begin{proof}
      False. Consider $a(X) = X^2 + X$ and $b(X) = X^2$. Then $a(X) \sim b(X)$ but $a(X) - b(X) \to \infty$.
    \end{proof}
    
    \item If $a(X) \sim b(X)$ and $a'(X) := \sum_{y \leq X} a(y)$, $b'(X) := \sum_{y \leq X} b(y)$ then $a'(X) \sim b'(X)$.
    
    \begin{proof}
      True. Fix $\varepsilon > 0$. By definition, there exists $X_0 = X_0(\varepsilon)$ such that $a(y) \geq (1 - \varepsilon)b(y)$ for $y \geq X_0$. But then 
      \[
        a'(X) = \sum_{y < X_0} a(y) + \sum_{X_0 \leq y \leq X} a(y) \geq \sum_{y < X_0} a(y) + \sum_{X_0 \leq y \leq X} (1 - \varepsilon)b(y) \geq (1 - \varepsilon)b'(X) - \sum_{y < X_0} b(y)
      \]
      Since $X_0$ only depends on $\varepsilon$, $\sum_{y < X_0} b(y) < \varepsilon b'(X)$ for large enough $X$. Thus, $a'(X) \geq (1 - 2\varepsilon)b'(X)$. The reverse inequality follows similarly.
    \end{proof}
    
    \item The converse to (iii).
    
    \begin{proof}
      False. Consider $a(X) = X$ whereas $b(X) = \begin{cases} 0 & \text{if } X = 2^k, k \in \Z \\ X & \text{otherwise} \end{cases}$.
    \end{proof}
  \end{enumerate}
\end{homeworkProblem}

\newpage

% To change sections, use: \renewcommand{\hmwkSection}{B}
% Example: Uncomment the line below when you start Section B
\renewcommand{\hmwkSection}{B}

\begin{homeworkProblem}
  Prove the following.
  
  \begin{enumerate}[(i)]
    \item There are infinitely many primes of the form $4k + 3$.
    
    \begin{proof}
      Suppose not. Let $p_1, \dots, p_n$ be the list of all such primes and consider $N = 4p_1 \dots p_n - 1$. Since $N$ is odd, it can only have prime factors of the form $4k + 1$ or $4k + 3$. But then $N \equiv 3 \pmod 4$, so it must have a prime factor of the form $4k + 3$. Thus $p_i | N$ for some $i$. But then $4p_1 \dots p_n - N = 1$ is divisible by $p_i$, contradiction.
    \end{proof}
    
    \item There are infinitely many primes of the form $4k+1$. (Hint: you may wish to prove that $-1$ is not a quadratic residue modulo any prime $p \equiv 3 \pmod 4$.)

    \begin{proof}
      Suppose not. Let $p_1, \dots, p_n$ be the list of all such primes and consider $N = (2p_1 \dots p_n)^2 + 1$. Let $q$ be a prime factor of $N$. Since $N$ is odd, $q \equiv 1, 3 \pmod 4$. Notice that $(2p_1 \dots p_n)^2 \equiv -1 \pmod q$, so we must have $q \equiv 3 \pmod 4$. But then $(q - 1)/2$ is odd, and so $(-1)^{(q - 1)/2} \equiv -1 \pmod q$. By Euler's criterion, $-1$ is not a quadratic residue modulo $q$, contradiction.
    \end{proof}
  \end{enumerate}
\end{homeworkProblem}

\newpage

\begin{homeworkProblem}
  We say that an arithmetic function is \emph{multiplicative} if $f(ab) = f(a)f(b)$ whenever $(a, b) = 1$, and \emph{completely multiplicative} if this holds without the coprimality restriction. For each of the functions $\Lambda, \mu, \phi, \tau, \sigma$, say with proof whether or not it is (a) multiplicative or (b) completely multiplicative.

  \begin{enumerate}[(i)]
    \item $\Lambda$ is not multiplicative.
    \begin{proof}
      Consider $a = 2$ and $b = 3$. Then $\Lambda(ab) = \Lambda(6) = 0$ whereas $\Lambda(a)\Lambda(b) = (\log 2)(\log 3) \neq 0$.
    \end{proof}
    \item $\mu$ is multiplicative but not completely multiplicative.
    \begin{proof}
      Suppose $(a, b) = 1$. Without loss of generality, assume that $p^2 | a$ for some prime $p$. Then $p^2 | ab$ and so $\mu(ab) = \mu(a)\mu(b) = 0$. Now assume $a = p_1\dots p_k$ and $b = q_1\dots q_l$, where $p_i$ and $q_j$ are distinct primes. Since $(a, b) = 1$, $p_i \neq q_j$ for all $i, j$. Thus $ab = p_1\dots p_k q_1\dots q_l$ is a product of distinct prime. It now follows that $\mu(ab) = (-1)^{k + l} = (-1)^{k}(-1)^{l} = \mu(a)\mu(b)$.

      To see that $\mu$ is not completely multiplicative, consider $a = 2$ and $b = 4$. Then $\mu(ab) = \mu(8) = 0$ whereas $\mu(a)\mu(b) = (-1)(-1) = 1 \neq 0$.
    \end{proof}
    \item $\phi$ is multiplicative but not completely multiplicative.
    \begin{proof}
      Suppose $(a, b) = 1$. The Chinese Remainder Theorem yields a ring isomorphism $f: \Z/ab\Z \rightarrow \Z/a\Z \times \Z/b\Z$ that sends $k \in \Z/ab\Z$ to $(k \pmod a, k \pmod b)$. But then $(k, ab) = 1$ if and only if $(k, a) = 1$ and $(k, b) = 1$. Hence, $f$ may be restricted to a group isomorphism $(\Z/ab\Z)^{\times} \rightarrow (\Z/a\Z)^{\times} \times (\Z/b\Z)^{\times}$. It now follows from the bijectivity of $f$ that $\phi(ab) = \phi(a)\phi(b)$.

      Consider $a = 2$ and $b = 6$. Then $\phi(ab) = \phi(12) = 4$ whereas $\phi(a)\phi(b) = 1 \times 2 = 2 \neq 4$. Thus $\phi$ is not completely multiplicative.
    \end{proof}
    \item $\tau$ is multiplicative but not completely multiplicative.
    \begin{proof}
      Suppose $(a, b) = 1$. Let $S, A, B$ be the sets of divisors of $a, b, ab$ respectively. Define $f: S \to A \times B$ as $f(d) = ((d, a), (d, b))$. $f$ is well-defined as $(\cdot , \cdot)$ is well-defined. We now show that $f$ has an inverse $g: A \times B \to S$ defined by $g(m, n) = mn$. Since $m | a$ and $n | b$, we have $mn | ab$ and so $g$ is well-defined. Let $m \in A$ and $n \in B$, Since $(a, b) = 1$, we have $m \not| b$ and $n \not| a$. But then $(mn, a) = m$ and $(mn, b) = n$, so $f(g(m, n)) = f(mn) = ((mn, a), (mn, b)) = (m, n)$. For $d \in S$, let $d_1 = (d, a)$ and $d_2 = (d, b)$. Then $g(f(d)) = g(d_1, d_2) = d_1d_2$. Note that $(d_1, d_2) = 1$ as $(a, b) = 1$, so $d_1d_2 | d$. Since $(a, b) = 1$ and $d | ab$, the prime powers of $d$ cannot exceed the prime powers of $a$ and $b$, respectively. But then $d | d_1d_2$ and so $d = g(f(d))$. This shows that $f$ is a bijection, so $|S| = |A| |B|$. It now follows that $\tau(ab) = \tau(a)\tau(b)$.

      To see that $\tau$ is not completely multiplicative, consider $a = 2$ and $b = 4$. Then $\tau(ab) = \tau(8) = 4$ whereas $\tau(a)\tau(b) = 2 \cdot 3 = 6 \neq 4$.
    \end{proof}
    \item $\sigma$ is multiplicative but not completely multiplicative.
    \begin{proof}
      Suppose $(a, b) = 1$. By the bijection $g$ defined in (iv), 
      \[
        \sigma(a)\sigma(b) = \left(\sum_{m | a} m\right)\left(\sum_{n | b} n\right) = \sum_{m|a}\sum_{n|b} g(m, n) = \sum_{d | ab} d = \sigma(ab).
      \]
      To see that $\sigma$ is not completely multiplicative, consider $a = 2$ and $b = 2$. Then $\sigma(ab) = \sigma(4) = 7$ whereas $\sigma(a)\sigma(b) = 3 \cdot 3 = 9 \neq 7$.
    \end{proof}
  \end{enumerate}
\end{homeworkProblem}

\newpage

\begin{homeworkProblem}
  Show that there are arbitrarily large gaps between consecutive primes by
  
  \begin{enumerate}[(i)]
    \item utilizing the bounds on $\pi(x)$ shown in the course;
    \begin{proof}
      Suppose not. Then for all $n$, there exists $M$ such that $p_{n+1} - p_n \leq M$, where $p_n$ is the $n$-th prime. Since $p_1 = 2$, by induction we have $p_n \leq 2 + (n - 1)M$ for all $n$. Hence we have $\pi(p_n) \geq p_n/M + o(1)$. But then by Theorem 1.2, $\pi(p_n) \leq cp_n/\log p_n$ for some constant $0 < c < 1$. Combining the inequalities yields $cM \geq \log p_n + o(1)$, contradiction.
    \end{proof}
    
    \item considering the numbers $n! + 2, \dots, n! + n$.
    \begin{proof}
      Let $n$ be a positive integer. Consider the numbers $n! + 2, \dots, n! + n$. For $2 \leq k \leq n$, we have $k | n! + k$, so none of these numbers is prime. That is, $n! + 2, \dots, n! + n$ are $n - 1$ consecutive composite numbers. Thus we may find arbitrarily large gaps between consecutive primes.
    \end{proof}
  \end{enumerate}
  
  Which of the two approaches gives the better bound?

  (i) yields a better bound. For any given $M$, (i) guarantees the existence of a prime gap of size at least $M$ for $p_n > e^{cM}$, whereas (ii) requires $p_n > n!$.
\end{homeworkProblem}

\newpage

\begin{homeworkProblem}
  Assuming the prime number theorem, show that $p_n \sim n \log n$, where $p_n$ denotes the $n^{th}$ prime.

  \begin{proof}
    By the prime number theorem $\pi(p_n) = (1 + o(1))p_n/\log p_n$. But $\pi(p_n) = n$ by definition, so $n = (1 + o(1))p_n/\log p_n$. Rearranging gives $p_n = (1 + o(1))n \log p_n$. Taking logarithms on both sides yields $\log p_n = \log n + \log \log p_n + o(1) = \log n + o(\log n) + o(1) = (1 + o(1))\log n$. Substituting this back gives $p_n = (1 + o(1))n \log n$.
  \end{proof}
\end{homeworkProblem}

\newpage

\begin{homeworkProblem}
  Denote by $\tau$ the divisor function.
  
  \begin{enumerate}[(i)]
    \item Show that $\tau(n) \leq 2\sqrt{n}$.
    \begin{proof}
      Let $n \in \N$. Let $D$ be the set of divisors of $n$. Then for $d \in D$ we have $\min(d, n/d) \leq \sqrt{n}$. Consider $f:D \to D$ defined by $f(d) = n/d$. Then $f$ is an involution that pairs up divisors $\leq \sqrt{n}$ with divisors $\geq \sqrt{n}$. Thus, $\tau(n) = |D| \leq 2\sqrt{n}$.
    \end{proof}
    
    \item Find a formula for $\tau$ in terms of the prime factorisation of $n$.
    \begin{proof}
      Let $n = p_1^{\alpha_1}p_2^{\alpha_2} \cdots p_k^{\alpha_k}$ be he prime factorisation of $n$. Then any divisor $d$ of $n$ is of the form $d = p_1^{\beta_1}p_2^{\beta_2} \cdots p_k^{\beta_k}$, where $0 \leq \beta_i \leq \alpha_i$ for all $1 \leq i \leq k$. Thus the number of choices for each $\beta_i$ is $\alpha_i + 1$, and so there are 
      \[
      \tau(n) = (\alpha_1 + 1)(\alpha_2 + 1) \cdots (\alpha_k + 1)
      \]
      divisors of $n$.
    \end{proof}
    
    \item Using your formula from (ii), show that for any $\varepsilon > 0$ we have $\tau(n) < n^\varepsilon$ for sufficiently large $n$.
    \begin{proof}
      Fix $\varepsilon > 0$. Let $n = p_1^{\alpha_1}p_2^{\alpha_2} \cdots p_k^{\alpha_k}$ be the prime factorisation of $n$. Consider the ratio $\tau(n)/n^{\varepsilon}$. By (ii),
      \[
        \frac{\tau(n)}{n^\varepsilon} = \prod_{i = 1}^k \frac{\alpha_i + 1}{p_i^{\varepsilon\alpha_i}}.
      \]
      Put $\varepsilon' = \epsilon/2$. If $p_i > 2^{1/\varepsilon'}$, then $p_i^{\varepsilon'} > 2$ and so
      \[
        \frac{\alpha_i + 1}{p_i^{\varepsilon'\alpha_i}} < \frac{\alpha_i + 1}{2^{\alpha_i}} < 1.
      \]
      Now suppose $p_i \leq 2^{1/\varepsilon'}$. Since $p_i^{\varepsilon} > 1$, we have
      \[
        \frac{\alpha_i + 1}{p_i^{\varepsilon'\alpha_i}} \leq \frac{\alpha_i + 1}{2^{\varepsilon'\alpha_i}} \to 0,
      \]
      as $\alpha \to \infty$. Hence $\frac{\alpha_i + 1}{p_i^{\varepsilon'\alpha_i}} < C_i$ for some constant $C_i$. Since there are only finitely many such $p_i$, 
      \[
        C = \prod_{p_i \leq 2^{1/\varepsilon'}} C_i < \infty.
      \]
      Combining both cases, we have
      \[
        \frac{\tau(n)}{n^{\varepsilon'}} < C\prod_{p_i > 2^{1/\varepsilon'}} 1 = C.
      \]
      Thus we now have
      \[
        \frac{\tau(n)}{n^{\varepsilon}} = \frac{\tau(n)}{n^{\varepsilon'}} \cdot \frac{1}{n^{\varepsilon'}} < \frac{C}{n^{\varepsilon'}} \to 0,
      \]
      as $n \to \infty$. This completes the proof.
    \end{proof}
  \end{enumerate}
\end{homeworkProblem}

\newpage

\begin{homeworkProblem}
  \begin{enumerate}[(i)]
    \item Let $X$ be an integer. Show that
    $$ \sum_{n \leq X} \log n = X \log X - X + O(\log X). $$

    \begin{proof}
      Since $\log n$ is increasing, 
      \[
        X \log X - X \leq \int_1^X \log t \, dt \leq \sum_{n \leq X} \log n \leq \int_1^{X} \log (t + 1) \, dt = X \log X - X + O(\log X).
      \]
      The result now follows.
    \end{proof}
    
    \item Show that if $X$ is an integer then
    $$ \sum_{p \leq X} \log p \left( \left\lfloor \frac{X}{p} \right\rfloor + \left\lfloor \frac{X}{p^2} \right\rfloor + \dots \right) = X \log X - X + O(\log X). $$

    \begin{proof}
      By Legendre's formula, $\alpha(p) = \sum_{k=1}^{\infty} \left\lfloor \frac{X}{p^k} \right\rfloor$ is the largest power of $p$ dividing $X!$. Thus
      \[
        \sum_{p \leq X} \log p \left( \left\lfloor \frac{X}{p} \right\rfloor + \left\lfloor \frac{X}{p^2} \right\rfloor + \dots \right) = \sum_{p \leq X} \log p^{\alpha(p)} = \log \prod_{p \leq X} p^{\alpha(p)} = \log X! = \sum_{n \leq X} \log n.
      \]
      The result now follows from (i).
    \end{proof}
    
    \item Show that the contribution from the terms $\left\lfloor \frac{X}{p^k} \right\rfloor$ with $k \geq 2$ is $O(X)$.
    
    \begin{proof}
      Let $L = \sum_{p \leq X} \log p \sum_{k = 2}^{\infty} \left\lfloor \frac{X}{p^k} \right\rfloor$. Then
      \[
        L \leq X\sum_{p \leq X} \log p \sum_{k = 2}^{\infty}\frac{1}{p^k} = X\sum_{p \leq X} \frac{\log p}{p(p - 1)}.
      \]
      Since $\log p \leq p^{1/2}$ for all prime $p$,
      \[
        \sum_{p \leq X} \frac{\log p}{p(p - 1)} \leq \sum_{p \leq X} \frac{p^{1/2}}{p(p - 1)} = \sum_{p \leq X} \frac{1}{p^{1/2}(p - 1)} \leq \sum_{p \leq X} \frac{1}{p^{1 + \varepsilon}} \leq \sum_{n \leq X} \frac{1}{n^{1 + \varepsilon}} < \infty,
      \]
      for some $\varepsilon > 0$. Thus $L = O(X)$.
    \end{proof}
    
    \item Deduce Mertens' estimate
    $$ \sum_{p \leq X} \frac{\log p}{p} = \log X + O(1). $$
    Explain why this remains valid even if $X$ is not necessarily an integer.
    \begin{proof}
      Since $|\left\lfloor\frac{X}{p} \right\rfloor\log p - \frac{X\log p}{p}| \leq \log p$, by (ii) and (iii)
      \[
        X\sum_{p \leq X} \frac{\log p}{p} + O(X) = \sum_{p \leq X} \log p\left\lfloor\frac{X}{p} \right\rfloor = X\log X + O(X).
      \]
      Dividing both sides by $X$ gives the result.
    \end{proof}
  \end{enumerate}
\end{homeworkProblem}

\newpage

% To change sections, use: \renewcommand{\hmwkSection}{C}
\renewcommand{\hmwkSection}{C}

\begin{homeworkProblem}
  Prove the second Mertens estimate:
  $$\sum_{p \le X} \frac{1}{p} = \log \log X + O(1).$$
  
  (Hint: Write $F(y) = \sum_{p \le y} \frac{\log p}{p}$ and consider $\int_2^x F(y)w(y)dy$ for an appropriate weight function $w$.)
  
  Deduce that there are constants $c_1, c_2 > 0$ such that
  $$\frac{c_1}{\log X} \le \prod_{p \le X} \left(1 - \frac{1}{p}\right) \le \frac{c_2}{\log X}.$$

  \begin{proof}
    % Your proof here
  \end{proof}
\end{homeworkProblem}

\newpage

\begin{homeworkProblem}
  Let $p_n$ denote the $n^{th}$ prime.
  
  \begin{enumerate}[(i)]
    \item Is it the case that, for sufficiently large $n$, the sequence $p_{n+1} - p_n$ is strictly increasing?
    
    \item Is it the case that, for sufficiently large $n$, the sequence $p_{n+1} - p_n$ is nondecreasing?
  \end{enumerate}

  \begin{proof}
    % Your proof here
  \end{proof}
\end{homeworkProblem}

\end{document}