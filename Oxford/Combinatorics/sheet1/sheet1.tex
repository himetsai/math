\documentclass{article}

% Packages
\usepackage{fancyhdr}
\usepackage{extramarks}
\usepackage{amsmath}
\usepackage{amsthm}
\usepackage{amsfonts}
\usepackage{tikz}
\usepackage[plain]{algorithm}
\usepackage{algpseudocode}
\usepackage{enumerate}

\usetikzlibrary{automata,positioning}

% Document Layout
\topmargin=-0.45in
\evensidemargin=0in
\oddsidemargin=0in
\textwidth=6.5in
\textheight=9.0in
\headsep=0.25in
\linespread{1.1}

% Page Style
\pagestyle{fancy}
\lhead{\hmwkAuthorName}
\chead{\hmwkClass:\ \hmwkTitle}
\rhead{Section \hmwkSection, \firstxmark}
\lfoot{\lastxmark}
\cfoot{\thepage}
\renewcommand\headrulewidth{0.4pt}
\renewcommand\footrulewidth{0.4pt}

% Paragraph Settings
\setlength\parindent{0pt}
\setlength{\parskip}{5pt}

% Section Management
\newcommand{\hmwkSection}{A} % Current section (A, B, or C) - update manually

% Problem Header Management
\newcommand{\enterProblemHeader}[1]{
  \nobreak\extramarks{}{Problem \arabic{#1} continued on next page\ldots}\nobreak{}
  \nobreak\extramarks{Problem \arabic{#1} (continued)}{Problem \arabic{#1} continued on next page\ldots}\nobreak{}
}

\newcommand{\exitProblemHeader}[1]{
  \nobreak\extramarks{Problem \arabic{#1} (continued)}{Problem \arabic{#1} continued on next page\ldots}\nobreak{}
  \stepcounter{#1}
  \nobreak\extramarks{Problem \arabic{#1}}{}\nobreak{}
}

% Counters
\setcounter{secnumdepth}{0}
\newcounter{partCounter}
\newcounter{homeworkProblemCounter}
\setcounter{homeworkProblemCounter}{1}
\nobreak\extramarks{Problem \arabic{homeworkProblemCounter}}{}\nobreak{}

% Homework Problem Environment
% Optional argument adjusts problem counter for non-sequential problems
\newenvironment{homeworkProblem}[1][-1]{
  \ifnum#1>0
    \setcounter{homeworkProblemCounter}{#1}
  \fi
  \section{Problem \arabic{homeworkProblemCounter}}
  \setcounter{partCounter}{1}
  \enterProblemHeader{homeworkProblemCounter}
}{
  \exitProblemHeader{homeworkProblemCounter}
}

% Assignment Details
\newcommand{\hmwkTitle}{Sheet\ \#1}
\newcommand{\hmwkDueDate}{October 28, 2025}
\newcommand{\hmwkClass}{C8.3 Combinatorics}
\newcommand{\hmwkClassInstructor}{Professor A. Scott}
\newcommand{\hmwkAuthorName}{\textbf{Ray Tsai}}

% Title Page
\title{
  \vspace{2in}
  \textmd{\textbf{\hmwkClass:\ \hmwkTitle}}\\
  \normalsize\vspace{0.1in}\small{Due\ on\ \hmwkDueDate\ at 12:00pm}\\
  \vspace{0.1in}\large{\textit{\hmwkClassInstructor}} \\
  \vspace{3in}
}

\author{\hmwkAuthorName}
\date{}

% Part Command
\renewcommand{\part}[1]{\textbf{\large Part \Alph{partCounter}}\stepcounter{partCounter}\\}

% Mathematical Commands
% Algorithms
\newcommand{\alg}[1]{\textsc{\bfseries \footnotesize #1}}

% Calculus
\newcommand{\deriv}[1]{\frac{\mathrm{d}}{\mathrm{d}x} (#1)}
\newcommand{\pderiv}[2]{\frac{\partial}{\partial #1} (#2)}
\newcommand{\dx}{\mathrm{d}x}

% Probability and Statistics
\newcommand{\Var}{\mathrm{Var}}
\newcommand{\Cov}{\mathrm{Cov}}
\newcommand{\Bias}{\mathrm{Bias}}
\newcommand*{\prob}{\mathds{P}}
\newcommand*{\E}{\mathds{E}}

% Number Sets
\newcommand*{\Z}{\mathbb{Z}}
\newcommand*{\Q}{\mathbb{Q}}
\newcommand*{\R}{\mathbb{R}}
\newcommand*{\C}{\mathbb{C}}
\newcommand*{\N}{\mathbb{N}}

\begin{document}

\maketitle
\pagebreak

\begin{homeworkProblem}
  Write down all antichains contained in $\mathcal{P}(1)$ and $\mathcal{P}(2)$. How many different antichains are there in $\mathcal{P}(3)$?

  \begin{proof}
    The antichains in $\mathcal{P}(1)$ are $\{\emptyset\}$ and $\{1\}$. The antichains in $\mathcal{P}(2)$ are $\{\emptyset\}$, $\{1\}$, $\{2\}$, $\{1, 2\}$, and $\{12\}$. There are 20 antichains in $\mathcal{P}(3)$.
  \end{proof}
\end{homeworkProblem}

\newpage

\begin{homeworkProblem}
  \begin{enumerate}[(a)]
    \item Look up Stirling's Formula. Use it to find an asymptotic estimate for $\binom{n}{n/2}$ of the form $(1+o(1))f(n)$ when $n$ is even.
    
    \begin{proof}
      By Stirling's Formula,
      \[
        \binom{n}{n/2} = \frac{n!}{(n/2)!(n/2)!} = \frac{(1 + o(1))\sqrt{2\pi n}(n/e)^n}{(1 + o(1))\pi n(n/2e)^{n}} = (1 + o(1))2^n\sqrt{\frac{2}{\pi n}}.
      \]
    \end{proof}
    
    \item Now do the same for $\binom{n}{pn}$ where $p \in (0, 1)$ is a constant and $pn$ is an integer. Write your answer in terms of the binary entropy function
    \[
    H(p) = -p \log p - (1 - p) \log(1 - p)
    \]
    \begin{proof}
      By Stirling's Formula,
      \begin{align*}
        \binom{n}{pn} &= \frac{n!}{(pn)!((1 - p)n)!} \\
        &= \frac{(1 + o(1))\sqrt{2\pi n}(n/e)^n}{(1 + o(1))\sqrt{2\pi pn}(pn/e)^{pn}\sqrt{2\pi (1 - p)n}((1 - p)n/e)^{(1 - p)n}} \\
        &= (1 + o(1))\frac{1}{\sqrt{2\pi p(1 - p)n}} \cdot \frac{(n/e)^n}{(pn/e)^{pn}(n(1 - p)/e)^{(1 - p)n}} \\
        &= (1 + o(1))\frac{1}{\sqrt{2\pi p(1 - p)n}} \cdot \frac{n^n}{(pn)^{pn}(n(1 - p))^{(1 - p)n}} \\
        &= (1 + o(1))\frac{1}{\sqrt{2\pi p(1 - p)n}} \cdot \frac{2^{n\log n}}{2^{pn\log(pn)}2^{(1 - p)n \log(n(1 - p))}} \\
        &= (1 + o(1))\frac{2^{n\log n - pn(\log n + \log p) - (1 - p)n(\log n + \log(1 - p))}}{\sqrt{2\pi p(1 - p)n}} \\
        &= (1 + o(1))\frac{2^{nH(p)}}{\sqrt{2\pi p(1 - p)n}}.
      \end{align*}
    \end{proof}
  \end{enumerate}
\end{homeworkProblem}

\newpage

% To change sections, use: \renewcommand{\hmwkSection}{B}
% Example: Uncomment the line below when you start Section B
\renewcommand{\hmwkSection}{B}

\begin{homeworkProblem}
  Let $k \le n/2$, and suppose that $\mathcal{F}$ is an antichain in $\mathcal{P}[n]$ such that every $A \in \mathcal{F}$ has $|A| \le k$. Prove that $|\mathcal{F}| \le \binom{n}{k}$.

  \begin{proof}
    Let $\mathcal{P}_k[n]$ be the set of all subsets of $[n]$ of size $k \leq n$. For $1 \leq k \leq n/2$, consider the bipartite subgraph $G_k$ of the discrete cube $Q_n$ induced by $[n]^{(k - 1)} \sqcup [n]^{(k)}$. Note that there is edge between $A \in [n]^{(k - 1)}$ and $B \in [n]^{(k)}$ if and only if $A \subseteq B$. 
    
    We now verify the conditions of Hall's Theorem to show that there is a matching saturating $[n]^{(k - 1)}$. Let $S \subseteq [n]^{(k - 1)}$ and let $T = \Gamma(S)$. Notice that each $A \in S$ has $n - k + 1$ neighbors in $T$, whereas each $B \in T$ has $k - 1$ neighbors in $[n]^{(k - 1)}$. But then 
    \[
    |S| \cdot (n - k + 1) = e(S, T) \leq |T| \cdot k.
    \]
    Since $k \leq n/2$, we have $|S| \leq |T| \cdot k/(n - k + 1) \leq |T|$. Hall's Theorem now furnishes a matching in $G_k$ saturating $[n]^{(k - 1)}$, for any $1\leq k \leq n/2$. By connecting the matchings between $G_k$ for $1 \leq k \leq n/2$, we get $\binom{n}{k}$ chains that partition $\mathcal{P}_k[n]$. It now follows that $\mathcal{F}$ intersects with any of these chains in at most one element, and so $|\mathcal{F}| \leq \binom{n}{k}$.
  \end{proof}
\end{homeworkProblem}

\newpage

\begin{homeworkProblem}
  Let $(P, \le)$ be a poset. Suppose that every chain in $P$ has at most $k$ elements. Prove that $P$ can be written as the union of $k$ antichains.

  \begin{proof}
    For $x \in P$, define $h(x)$ as the length of the longest chain containing $x$ as the maximal element. Notice that if $x > y$ then $h(x) > h(y)$, as we may append $x$ to the end of any chain containing $y$. This implies $x$ and $y$ are incomparable if $h(x) = h(y)$. But then for any $x \in P$ we have $h(x) \leq k$. Thus for $1 \leq n \leq k$, $A_n = \{x \in P \mid h(x) = n\}$ is an antichain. The result now follows. 
  \end{proof}
\end{homeworkProblem}

\newpage

\begin{homeworkProblem}
  Suppose $\mathcal{F} \subset \mathcal{P}[n]$ is a set system containing no chain with $k+1$ sets.
  
  \begin{enumerate}[(a)]
    \item Prove that $\sum_{i=0}^{n} \frac{|\mathcal{F}_i|}{\binom{n}{i}} \le k$, where $\mathcal{F}_i = \mathcal{F} \cap [n]^{(i)}$ for each $i$.

    \begin{proof}
      Since every chain in $\mathcal{F}$ has at most $k$ elements, the proof of Problem 4 furnishes a partition of $\mathcal{F}$ into $k$ antichains $A_1, \ldots, A_k$. By the LYM inequality, for $1 \leq j \leq k$ we have
      \[
        \sum_{i = 0}^n \frac{|A_j \cap [n]^{(i)}|}{\binom{n}{i}} \leq 1. \tag{1}
      \]
      But then
      \[
        \sum_{i=0}^{n} \frac{|\mathcal{F}_i|}{\binom{n}{i}} = \sum_{i=0}^{n} \sum_{j = 1}^k \frac{|A_j \cap [n]^{(i)}|}{\binom{n}{i}} = \sum_{j = 1}^k \sum_{i=0}^{n}  \frac{|A_j \cap [n]^{(i)}|}{\binom{n}{i}} \leq k. \tag{2}
      \]
    \end{proof}

    \item What is the maximum possible size of such a system?

    \begin{proof}
      By the LYM inequality, equality holds in (1) if and only if $A_j = [n]^{(i)}$ for some $i$. Thus, equality can be achieved when $\mathcal{F} = \bigsqcup_{i \in I} [n]^{(i)}$ for some $I \subseteq [n]$ of size $k$.
    \end{proof}
  \end{enumerate}
\end{homeworkProblem}

\newpage

\begin{homeworkProblem}
  Let $\mathcal{A}$ be an antichain in $\mathcal{P}[n]$ that is not of the form $[n]^{(r)}$. Must there exist a maximal chain disjoint from $\mathcal{A}$?

  \begin{proof}
    For $A \in \mathcal{A}$, the fraction of chains in $\mathcal{P}[n]$ that contain $A$ is
    \[
      \frac{|A|!(n - |A|)!}{n!} = \frac{1}{\binom{n}{|A|}}.
    \]
    Since each chain intersects with at most one element of $\mathcal{A}$, the fraction of chains that intersect with $\mathcal{A}$ is
    \[
      \sum_{A \in \mathcal{A}} \frac{1}{\binom{n}{|A|}} = \sum_{i = 0}^n \sum_{A \in \mathcal{A} \cap [n]^{(i)}} \frac{1}{\binom{n}{i}} = \sum_{i = 0}^n \frac{|\mathcal{A} \cap [n]^{(i)}|}{\binom{n}{i}}.
    \]
    But then $\mathcal{A}$ is not of the form $[n]^{(r)}$, so by the LYM inequality, the above sum is strictly less than 1. This completes the proof.
  \end{proof}
\end{homeworkProblem}

\newpage

\begin{homeworkProblem}
  Let $(P, \le)$ be an infinite poset. Must $P$ contain an infinite chain or antichain?

  \begin{proof}
    Suppose $P$ contains no infinite antichain and no infinite chain. Define $h(x)$ as the length of the longest chain containing $x$ as the maximal element. Notice that if $x > y$ then $h(x) > h(y)$, as we may append $x$ to the end of any chain containing $y$. This implies $x$ and $y$ are incomparable if $h(x) = h(y)$. Thus $A_n = \{x \in P \mid h(x) = n\}$ is an antichain for $n \in \N$ and $P = \bigsqcup_{n \in \N} A_n$. Since each $A_n$ is finite, there must be infinitely many $n$ such that $A_n$ is non-empty. But then $h(x)$ is unbounded on $P$, so there must exists an infinite chain in $P$, contradiction.
  \end{proof}
\end{homeworkProblem}


\end{document}