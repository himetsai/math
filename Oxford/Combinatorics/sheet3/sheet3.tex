\documentclass{article}

% Packages
\usepackage{fancyhdr}
\usepackage{extramarks}
\usepackage{amsmath}
\usepackage{amsthm}
\usepackage{amsfonts}
\usepackage{tikz}
\usepackage[plain]{algorithm}
\usepackage{algpseudocode}
\usepackage{enumerate}
\usepackage{dsfont}

\usetikzlibrary{automata,positioning}

% Document Layout
\topmargin=-0.45in
\evensidemargin=0in
\oddsidemargin=0in
\textwidth=6.5in
\textheight=9.0in
\headsep=0.25in
\linespread{1.1}

% Page Style
\pagestyle{fancy}
\lhead{\hmwkAuthorName}
\chead{\hmwkClass:\ \hmwkTitle}
\rhead{Section \hmwkSection, \firstxmark}
\lfoot{\lastxmark}
\cfoot{\thepage}
\renewcommand\headrulewidth{0.4pt}
\renewcommand\footrulewidth{0.4pt}

% Paragraph Settings
\setlength\parindent{0pt}
\setlength{\parskip}{5pt}

% Section Management
\newcommand{\hmwkSection}{A} % Current section (A, B, or C) - update manually

% Problem Header Management
\newcommand{\enterProblemHeader}[1]{
  \nobreak\extramarks{}{Problem \arabic{#1} continued on next page\ldots}\nobreak{}
  \nobreak\extramarks{Problem \arabic{#1} (continued)}{Problem \arabic{#1} continued on next page\ldots}\nobreak{}
}

\newcommand{\exitProblemHeader}[1]{
  \nobreak\extramarks{Problem \arabic{#1} (continued)}{Problem \arabic{#1} continued on next page\ldots}\nobreak{}
  \stepcounter{#1}
  \nobreak\extramarks{Problem \arabic{#1}}{}\nobreak{}
}

% Counters
\setcounter{secnumdepth}{0}
\newcounter{partCounter}
\newcounter{homeworkProblemCounter}
\setcounter{homeworkProblemCounter}{1}
\nobreak\extramarks{Problem \arabic{homeworkProblemCounter}}{}\nobreak{}

% Homework Problem Environment
% Optional argument adjusts problem counter for non-sequential problems
\newenvironment{homeworkProblem}[1][-1]{
  \ifnum#1>0
    \setcounter{homeworkProblemCounter}{#1}
  \fi
  \section{Problem \arabic{homeworkProblemCounter}}
  \setcounter{partCounter}{1}
  \enterProblemHeader{homeworkProblemCounter}
}{
  \exitProblemHeader{homeworkProblemCounter}
}

% Assignment Details
\newcommand{\hmwkTitle}{Sheet\ \#3}
\newcommand{\hmwkDueDate}{November 27, 2025}
\newcommand{\hmwkClass}{C8.3 Combinatorics}
\newcommand{\hmwkClassInstructor}{Professor A. Scott}
\newcommand{\hmwkAuthorName}{\textbf{Ray Tsai}}

% Title Page
\title{
  \vspace{2in}
  \textmd{\textbf{\hmwkClass:\ \hmwkTitle}}\\
  \normalsize\vspace{0.1in}\small{Due\ on\ \hmwkDueDate\ at 12:00pm}\\
  \vspace{0.1in}\large{\textit{\hmwkClassInstructor}} \\
  \vspace{3in}
}

\author{\hmwkAuthorName}
\date{}

% Part Command
\renewcommand{\part}[1]{\textbf{\large Part \Alph{partCounter}}\stepcounter{partCounter}\\}

% Mathematical Commands
% Algorithms
\newcommand{\alg}[1]{\textsc{\bfseries \footnotesize #1}}

% Calculus
\newcommand{\deriv}[1]{\frac{\mathrm{d}}{\mathrm{d}x} (#1)}
\newcommand{\pderiv}[2]{\frac{\partial}{\partial #1} (#2)}
\newcommand{\dx}{\mathrm{d}x}

% Probability and Statistics
\newcommand{\Var}{\mathrm{Var}}
\newcommand{\Cov}{\mathrm{Cov}}
\newcommand{\Bias}{\mathrm{Bias}}
\newcommand*{\prob}{\mathds{P}}
\newcommand*{\E}{\mathds{E}}

% Number Sets
\newcommand*{\Z}{\mathbb{Z}}
\newcommand*{\Q}{\mathbb{Q}}
\newcommand*{\R}{\mathbb{R}}
\newcommand*{\C}{\mathbb{C}}
\newcommand*{\N}{\mathbb{N}}

\begin{document}

\maketitle
\pagebreak

\begin{homeworkProblem}
  Let $\mathcal{A} \subset \mathcal{P}[n]$ be an upset and $\mathcal{B} \subset \mathcal{P}[n]$ be a downset. Prove that $|\mathcal{A} \cap \mathcal{B}| \leq 2^{-n}|\mathcal{A}| \cdot |\mathcal{B}|$.

  \begin{proof}
    Note that $\mathcal{A}^C$ is a downset. By Kleitman's Theorem,
    \[
      |\mathcal{B}| - |\mathcal{A} \cap \mathcal{B}| = |\mathcal{A}^C \cap \mathcal{B}| \geq \frac{|\mathcal{A}^C||\mathcal{B}|}{2^n} = \frac{(2^n - |\mathcal{A}|)|\mathcal{B}|}{2^n} = |\mathcal{B}| - \frac{|\mathcal{A}||\mathcal{B}|}{2^n}.
    \]
    The result now follows.
  \end{proof}
\end{homeworkProblem}

\newpage

\begin{homeworkProblem}
  The $i$-compression operator $\pi_i$ is defined by $\pi_i(A) = A \setminus \{i\}$ and, for a set system $\mathcal{A}$,
  \[
  \pi_i(\mathcal{A}) = \{\pi_i(A) : A \in \mathcal{A}\} \cup \{A \in \mathcal{A} : \pi_i(A) \in \mathcal{A}\}.
  \]
  Let $\mathcal{F} \subset \mathcal{P}[n]$ be a set system and $\mathcal{A} = \pi_i(\mathcal{F})$ for some $i \in [n]$. Show that $\operatorname{tr}_{\mathcal{A}}(S) \leq \operatorname{tr}_{\mathcal{F}}(S)$ for every $S \subset [n]$.

  \begin{proof}
    Suppose $B \subseteq S$ such that $B = A \cap S$ for some $A \in \mathcal{A}$. Let $F \in \mathcal{F}$ such that $\pi_i(F) = A$. We may assume $F = A \cup \{i\}$ otherwise $F \cap S = F \cap A = B$ and we are done. If $i \notin S$, then $F \cap S = A \cap S = B$. If $i \in S$, then $F \cap S = B \cup \{i\} \notin \mathcal{A} \mid S$. Thus, regardless of whether $B \in \mathcal{F} \mid S$ or not,
    \[
      \operatorname{tr}_{\mathcal{A}}(S) \leq \operatorname{tr}_{\mathcal{F}}(S).
    \]
  \end{proof}
\end{homeworkProblem}

\newpage

\begin{homeworkProblem}
  \begin{enumerate}[(a)]
    \item Let $X = \mathbb{R}$ and let $\mathcal{F} = \{[a, b] : a < b\}$. What is the VC-dimension of $\mathcal{F}$?
    \begin{proof}
      Note that the VC-dimension of $\mathcal{F}$ is at least 2: the set $\{0, 2\}$ is shattered by the intervals $[3, 4], [0, 1], [1, 2], [0, 2]$. 
      
      It has VC-dimension less than 3: consider any set of three points $a < b < c$. Then there are no intervals that contain both $a$ and $c$ while excluding $b$. 

      Hence, the VC-dimension of $\mathcal{F}$ is 2.
    \end{proof}
    \item What if $X = \mathbb{R}^2$ and $\mathcal{F} = \{[a, b] \times [c, d] : a < b \text{ and } c < d\}$?
    
    \begin{proof}
      The VC-dimension of $\mathcal{F}$ is at least 4, as it can shatter the set $\{(-1, -1), (1, 1), (-1, 1), (1, -1)\}$. 

      It has VC-dimension less than 4: Let $S$ be a set of 5 elements. Let $x_M, x_m, y_M, y_m$ be the maximum and minimum $x$- and $y$-coordinates of the points in $S$, respectively. Then the box that contains $\{x_M, x_m, y_M, y_m\}$ must contain the rest of $S$. 
    \end{proof}
  \end{enumerate}
\end{homeworkProblem}

\newpage

% To change sections, use: \renewcommand{\hmwkSection}{B}
% Example: Uncomment the line below when you start Section B
\renewcommand{\hmwkSection}{B}

\begin{homeworkProblem}
  Let $\mathcal{F}$ be the collection of all convex sets in $\mathbb{R}^2$. Show that $\mathcal{F}$ does not have bounded VC-dimension.

  \begin{proof}
    For any $n \in \N$, consider a set $S$ of $n$ points lying on the unit circle. Then for any subset $T \subseteq S$, the polygon formed by the points in $T$ is convex and only contains points in $T$. Hence, the VC-dimension of $\mathcal{F}$ is at least $n$. This completes the proof.
  \end{proof}
\end{homeworkProblem}

\newpage

\begin{homeworkProblem}
  A \emph{sunflower} is a sequence $F_1, \dots, F_k$ of sets such that for some set $S$, and all $i < j$,
  \[
  F_i \cap F_j = S.
  \]
  Let $r, s \geq 1$. Prove that there is $m = m(r, s)$ such that every sequence of $m$ sets from $\mathbb{N}^{(r)}$ has a subsequence of length $s$ that forms a sunflower.

  [Bonus question: explain the term \emph{sunflower} by means of a nice picture.]

  \begin{proof}
    Fix $s \geq 1$. We proceed by induction on $r$ to show that $m(r, s)$ is bounded. Note that if $m(1, s) \geq s^2 + 1$, then either there are $s$ distinct singletons or there exists subsequence $F_{i_1} = F_{i_2} = \cdots = F_{i_s}$, by the pigeonhole principle. But then either case yields a sunflower, so the base case is done. Suppose $r \geq 2$. By induction, $m(r - 1, s) < \infty$. Let $F_1, \ldots, F_k$ be a sequence of $k \geq (m(r - 1, s) + 1)^2 + 1$ sets from $\N^{(r)}$. Just as the base case, there exists a subsequence $F_{i_1}, \ldots F_{i_{m(r - 1, s) + 1}}$, such that either there exists $f_i \in F_i$ with $f_i \neq f_j$ for $i \neq j$, or there exists $f \in \N$ with $f \in \bigcap_{i = 1}^{m(r - 1, s) + 1} F_i$. In either case, consider the sequence $F_1 \setminus \{f_1\}, \ldots, F_{m(r - 1, s) + 1} \setminus \{f_{m(r - 1, s) + 1}\}$ or $F_1 \setminus \{f\}, \ldots, F_{m(r - 1, s) + 1} \setminus \{f\}$. Since it is a sequence from $\N^{(r - 1)}$ of length $> m(r - 1, s)$, there is a subsequence $F_{j_1}, \ldots, F_{j_{s}}$ such that $F_{j_1} \setminus \{f_{j_1}\}, \ldots, F_{j_s} \setminus \{f_{j_s}\}$ or $F_{j_1} \setminus \{f\}, \ldots, F_{j_s} \setminus \{f\}$ is a sunflower. But then $F_{j_1}, \ldots, F_{j_{s}}$ is a sunflower of length $s$ in either case. This completes the proof.
  \end{proof}
\end{homeworkProblem}

\newpage

\begin{homeworkProblem}
  Let $\mathcal{F} \subset \mathcal{P}[n]$ be a set system. The \emph{dual set system} $\mathcal{F}^*$ has vertex set $\mathcal{F}$, and for each $i \in [n]$, there is an edge $\{F \in \mathcal{F} : i \in F\}$ (we ignore duplicate edges). Prove that for every positive integer $d$ there is a constant $f(d)$ such that if $\mathcal{F}$ has VC-dimension at most $d$ then $\mathcal{F}^*$ has VC-dimension at most $f(d)$.

  \begin{proof}
    Suppose $\mathcal{F}$ has VC-dimension $d$. Suppose for the sake of contradiction that $\mathcal{F}^*$ has VC-dimension $2^{d + 1}$. Then there exists $\mathcal{S} \subseteq \mathcal{F}$ of size $2^{d + 1}$ that is shattered by $\mathcal{F}^*$. Consider the indicent matrix $M$, whose rows are indexed by the elements of $[n]$, columns are indexed by the elements of $\mathcal{S}$, and $M_{i, F} = \mathds{1}_{i \in F}$ for $F \in \mathcal{S}, i \in [n]$. Since $\mathcal{S}$ is shattered by $\mathcal{F}^{*}$, there are $2^{2^{d + 1}}$ unique binary vectors of length $2^{d + 1}$ among the rows of $M$. Omit all duplicate rows of $M$ so that $M$ turns into a $2^{2^{d + 1}} \times 2^{d + 1}$ matrix with unique rows. Let $M'$ be the $(d + 1) \times 2^{d + 1}$ matrix whose columns are the binary expansions of the numbers $0, \ldots, 2^{d + 1} - 1$ in order. Note that the columns of $M'$ are distinct. Since the rows of $M$ contain all possible binary vectors of length $2^{d + 1}$, each row of $M'$ corresponds to a unique row of $M$. Let $X \subseteq [n]$ be the set of $d + 1$ rows in $M$ that $M'$ corresponds to. Then $X$ is shattered by $\mathcal{F}$, contradicting that $\mathcal{F}$ has VC-dimension $d$. Thus $F^*$ has VC-dimension at most $f(d) = 2^{{d + 1}} - 1$.
  \end{proof}
\end{homeworkProblem}

\newpage

\begin{homeworkProblem}
  Suppose that $\mathcal{F}_1, \dots, \mathcal{F}_s \subset \mathcal{P}(n)$ are intersecting families. Prove that $|\mathcal{F}_1 \cup \dots \cup \mathcal{F}_s| \leq 2^n - 2^{n-s}$.

  \begin{proof}
    For $1 \leq i \leq s$ and $\mathcal{S} \subseteq \mathcal{F}_i$, Define
    \[
      \mathcal{D}_i = \{D \in \mathcal{P}(n) : F \subseteq D \text{ for some } F \in \mathcal{F}_i\}. 
    \]
    Note that $\mathcal{D}_i$ is intersecting and $|\mathcal{F}_i| = |\mathcal{D}_i|$. Since
    \[
      |\mathcal{F}_1 \cup \dots \cup \mathcal{F}_s| \leq |\mathcal{D}_1 \cup \dots \cup \mathcal{D}_s| = 2^n - |\mathcal{D}_1^C \cap \dots \cap \mathcal{D}_s^C|,
    \]
    it suffices to show that $|\mathcal{D}_1^C \cap \dots \cap \mathcal{D}_s^C| \geq 2^{n-s}$. Since $|\mathcal{D}_i| \leq 2^{n - 1}$, we have $|\mathcal{D}_i^C| \geq 2^{n - 1}$. For $D \in \mathcal{D}_i$, notice that if $D' \supseteq D$ then $D' \in \mathcal{D}_i$, so $\mathcal{D}_i$ is an upset. But then $\mathcal{D}_i^C$ is a downset. It now follows from the Kleitman's Theorem that
    \[
      |\mathcal{D}_1^C \cap \dots \cap \mathcal{D}_s^C| \geq \frac{|\mathcal{D}_1^C||\mathcal{D}_2^C| \cdots |\mathcal{D}_s^C|}{(2^n)^{s - 1}} \geq \frac{2^{s(n - 1)}}{2^{n(s - 1)}} = 2^{n - s}.
    \]
  \end{proof}
\end{homeworkProblem}

\end{document}