\documentclass{article}

% Packages
\usepackage{fancyhdr}
\usepackage{extramarks}
\usepackage{amsmath}
\usepackage{amsthm}
\usepackage{amsfonts}
\usepackage{tikz}
\usepackage[plain]{algorithm}
\usepackage{algpseudocode}
\usepackage{enumerate}

\usetikzlibrary{automata,positioning}

% Document Layout
\topmargin=-0.45in
\evensidemargin=0in
\oddsidemargin=0in
\textwidth=6.5in
\textheight=9.0in
\headsep=0.25in
\linespread{1.1}

% Page Style
\pagestyle{fancy}
\lhead{\hmwkAuthorName}
\chead{\hmwkClass:\ \hmwkTitle}
\rhead{Section \hmwkSection, \firstxmark}
\lfoot{\lastxmark}
\cfoot{\thepage}
\renewcommand\headrulewidth{0.4pt}
\renewcommand\footrulewidth{0.4pt}

% Paragraph Settings
\setlength\parindent{0pt}
\setlength{\parskip}{5pt}

% Section Management
\newcommand{\hmwkSection}{A} % Current section (A, B, or C) - update manually

% Problem Header Management
\newcommand{\enterProblemHeader}[1]{
  \nobreak\extramarks{}{Problem \arabic{#1} continued on next page\ldots}\nobreak{}
  \nobreak\extramarks{Problem \arabic{#1} (continued)}{Problem \arabic{#1} continued on next page\ldots}\nobreak{}
}

\newcommand{\exitProblemHeader}[1]{
  \nobreak\extramarks{Problem \arabic{#1} (continued)}{Problem \arabic{#1} continued on next page\ldots}\nobreak{}
  \stepcounter{#1}
  \nobreak\extramarks{Problem \arabic{#1}}{}\nobreak{}
}

% Counters
\setcounter{secnumdepth}{0}
\newcounter{partCounter}
\newcounter{homeworkProblemCounter}
\setcounter{homeworkProblemCounter}{1}
\nobreak\extramarks{Problem \arabic{homeworkProblemCounter}}{}\nobreak{}

% Homework Problem Environment
% Optional argument adjusts problem counter for non-sequential problems
\newenvironment{homeworkProblem}[1][-1]{
  \ifnum#1>0
    \setcounter{homeworkProblemCounter}{#1}
  \fi
  \section{Problem \arabic{homeworkProblemCounter}}
  \setcounter{partCounter}{1}
  \enterProblemHeader{homeworkProblemCounter}
}{
  \exitProblemHeader{homeworkProblemCounter}
}

% Assignment Details
\newcommand{\hmwkTitle}{Sheet\ \#2}
\newcommand{\hmwkDueDate}{November12, 2025}
\newcommand{\hmwkClass}{C8.3 Combinatorics}
\newcommand{\hmwkClassInstructor}{Professor A. Scott}
\newcommand{\hmwkAuthorName}{\textbf{Ray Tsai}}

% Title Page
\title{
  \vspace{2in}
  \textmd{\textbf{\hmwkClass:\ \hmwkTitle}}\\
  \normalsize\vspace{0.1in}\small{Due\ on\ \hmwkDueDate\ at 12:00pm}\\
  \vspace{0.1in}\large{\textit{\hmwkClassInstructor}} \\
  \vspace{3in}
}

\author{\hmwkAuthorName}
\date{}

% Part Command
\renewcommand{\part}[1]{\textbf{\large Part \Alph{partCounter}}\stepcounter{partCounter}\\}

% Mathematical Commands
% Algorithms
\newcommand{\alg}[1]{\textsc{\bfseries \footnotesize #1}}

% Calculus
\newcommand{\deriv}[1]{\frac{\mathrm{d}}{\mathrm{d}x} (#1)}
\newcommand{\pderiv}[2]{\frac{\partial}{\partial #1} (#2)}
\newcommand{\dx}{\mathrm{d}x}

% Probability and Statistics
\newcommand{\Var}{\mathrm{Var}}
\newcommand{\Cov}{\mathrm{Cov}}
\newcommand{\Bias}{\mathrm{Bias}}
\newcommand*{\prob}{\mathds{P}}
\newcommand*{\E}{\mathds{E}}

% Number Sets
\newcommand*{\Z}{\mathbb{Z}}
\newcommand*{\Q}{\mathbb{Q}}
\newcommand*{\R}{\mathbb{R}}
\newcommand*{\C}{\mathbb{C}}
\newcommand*{\N}{\mathbb{N}}

\begin{document}

\maketitle
\pagebreak

\begin{homeworkProblem}
  What are the 50th, 51st and 52nd elements of $\mathbb{N}^{(3)}$ in the colex order? What about in lex?

  \begin{proof}
    Notice that are $\binom{k - 1}{2}$ elements in $\mathbb{N}^{(3)}$ with $k + 2$ as the largest number. Thus, there are
    \[
      \sum_{k = 1}^n \binom{k - 1}{2}
    \]
    elements $x \in \mathbb{N}^{(3)}$ such that $x \leq_{colex} (k, k + 1, k + 2)$. Note that
    \[
      \sum_{k = 1}^6 \binom{k - 1}{2} = 35, \quad \sum_{k = 1}^7 \binom{k - 1}{2} = 56.
    \]
    Thus the 50th, 51st and 52nd elements have $8$ as their largest number. Counting down from $(6, 7, 8)$ now gives us the 50th, 51st and 52nd elements in colex order as $(5, 6, 8)$, $(1, 7, 8)$ and $(2, 7, 8)$.
  \end{proof}
\end{homeworkProblem}

\newpage

\begin{homeworkProblem}
  Let $\mathcal{F} \subset [10]^{(3)}$, and suppose $|\mathcal{F}| = 29$.
    \begin{enumerate}[(a)]
        \item What is the minimum possible size of $\partial \mathcal{F}$?
        \begin{proof}
          By Kruskal-Katona Theorem, the family $\mathcal{F} \subset [10]^{(3)}$ with the minimum possible shadow is the family consisting of the first $29$ elements in colex order, which is the family
          \[
            [7]^{(3)} \backslash \{765, 764, 763, 762, 761, 754\} = [6]^{(3)} \cup \{ab7 : ab \in [4]^{(2)}\} \cup \{753, 752, 751\}.
          \]
          The shadow of this family is 
          \[
            [6]^{(2)} \cup \{a7 : a \in [5]\},
          \]
          which has size $20$.
        \end{proof}
        \item Find a family that achieves this minimum.
        \begin{proof}
          See part (a).
        \end{proof}
    \end{enumerate}
\end{homeworkProblem}

\newpage

\begin{homeworkProblem}
  Suppose that $\mathcal{F} \subset [n]^{(r)}$, and let $\mathcal{A}$ denote the first $|\mathcal{F}|$ elements of $[n]^{(r)}$ in colex order. If $|\partial \mathcal{F}| = |\partial \mathcal{A}|$ must we have $\mathcal{F} = \mathcal{A}$ (possibly after relabelling elements)?

  \begin{proof}
    No. Consider the case $r = 2$, where $\mathcal{F} = \{13, 23, 14, 24\}$. Then $\mathcal{A} = \{13, 23, 14, 24\}$ and $|\partial \mathcal{F}| = |\partial \mathcal{A}| = 4$. But $\mathcal{F} \neq \mathcal{A}$ for any relabelling of elements.
  \end{proof}
\end{homeworkProblem}

\newpage

% To change sections, use: \renewcommand{\hmwkSection}{B}
% Example: Uncomment the line below when you start Section B
\renewcommand{\hmwkSection}{B}

\begin{homeworkProblem}
  
  The \textit{upper shadow} $\partial^+(\mathcal{F})$ of a set $\mathcal{F} \subset [n]^{(r)}$ is the set
  \[
      \partial^+(\mathcal{F}) := \{ A \in [n]^{(r+1)} : A \supset B \text{ for some } B \in \mathcal{F} \}.
  \]
  Give a version of the Kruskal-Katona Theorem for the upper shadow.

  \begin{proof}
    Let $\mathcal{F} \subseteq [n]^{(r)}$ and let $\mathcal{A}$ be the family consisting of the last $|\mathcal{F}|$ elements of $[n]^{(r)}$ in colex order. We will show that $|\partial^+ \mathcal{F}| \geq |\partial^+ \mathcal{A}|$.

    For any family of subsets $\mathcal{S}$ of $[n]$, let $\mathcal{S}^C := \{[n] \backslash S : S \in \mathcal{S}\}$, and note that $|\mathcal{S}| = |\mathcal{S}^C|$. Since $\partial \mathcal{F}^C \subseteq [n]^{(n - r - 1)}$ consists of all the $(n - r - 1)$-element subsets of $[n]$ that are disjoint from $\mathcal{F}$, its complement are all the $(r + 1)$-element subsets that contains $\mathcal{F}$. In other words,
    $\partial^+ \mathcal{F} = (\partial \mathcal{F}^C)^C$. Let $\mathcal{A}$ be the family consisting of the last $|\mathcal{F}|$ elements of $[n]^{(r)}$ in colex order. Then $\mathcal{A}^C$ is the first $|\mathcal{F}|$ elements of $[n]^{(n - r)}$. It now follows from Kruskal-Katona Theorem that
    \[
      |\partial^+ \mathcal{F}| = |(\partial \mathcal{F}^C)^C| = |\partial \mathcal{F}^C| \geq |\partial \mathcal{A}^C| = |(\partial^+ \mathcal{A}^C)^C| = |\partial^+ \mathcal{A}|.
    \]
  \end{proof}
\end{homeworkProblem}

\newpage

\begin{homeworkProblem}
  Give a proof of Hall's Theorem using Dilworth's Theorem.

  \begin{proof}
    Let $G$ be a bipartite graph with parts $A$ and $B$ such that $|A| \leq |B|$. If there is a complete matching in $G$ from $A$ to $B$, then $\Gamma(S) \geq |S|$ for all $S \subseteq A$. 

    Now suppose that $G$ satisfies the Hall's Condition. Define poset $(P, \leq)$, where $P = V(G)$ and $x < y$ if $x \in A$, $y \in B$, and $\{x, y\} \in E(G)$. Note that each chain has lengh at most 2. For any antichain $\mathcal{X}$ in $P$ with $S = A \cap \mathcal{X}$, we have
    \[
      |\mathcal{X}| \leq |S| + |B| - |\Gamma(S)| \leq |B|,
    \]
    by Hall's Condition. Hence, $B$ is the maximum antichain in $P$. Dilworth's Theorem now furnishes a set $\mathcal{C}$ of $|B|$ chains that cover $P$, and note that each $b \in B$ is in exactly one chain in $\mathcal{C}$. Since $\mathcal{C}$ covers $A$, there is a chain $C_a = \{a, b\} \in \mathcal{C}$ for some $b \in B$. Then $\mathcal{M} = \{C_a\}_{a \in A}$ is a set of disjoint edges that saturates $A$.
  \end{proof}
\end{homeworkProblem}

\newpage

\begin{homeworkProblem}
  Prove that in any sequence of $n^2 + 1$ real numbers there is an increasing subsequence of length $n + 1$ or a decreasing subsequence of length $n + 1$.

  \begin{proof}
    Let $x_1, x_2, \ldots, x_{n^2 + 1}$ be the sequence of distinct real numbers. Define a poset $(P, \leq)$ by letting $P$ be the set of numbers in the sequence and $x_i < x_j$ in the poset if $x_i < x_j$ and $i < j$. Suppose that there is no decreasing subsequence of length $n + 1$. Then the maximum antichain in $P$ has size at most $n$. By Dilworth's Theorem, at most $n$ chains are needed to cover $P$. But then there must be a chain of length $n + 1$, otherwise $n$ chains would not be enough to cover $n^2 + 1$ elements.
  \end{proof}
\end{homeworkProblem}

\newpage

\begin{homeworkProblem}
  We say that $\mathcal{A} \subset \mathcal{P}[n]$ is a \textit{downset} if, for every $A \in \mathcal{A}$, every subset of $A$ belongs to $\mathcal{A}$. Prove that if $\mathcal{A}$ is a downset then the average size of sets in $\mathcal{A}$ is at most $n/2$.

  \begin{proof}
    By proposition 10, $\mathcal{P}[n]$ may be partitioned into symmetric chains, which also gives a partition of $\mathcal{A}$ into chains. Let $\mathcal{C}$ be a symmetric chain that contains some element $A \in \mathcal{A}$. Since $\mathcal{A}$ is a downset, every element below $A$ in $\mathcal{C}$ is also in $\mathcal{A}$. Thus the average size of $\mathcal{C}$ does not increase when restricted to $\mathcal{A}$. But then the average size of a symmetric chain is $n/2$.
  \end{proof}
\end{homeworkProblem}

\newpage

\begin{homeworkProblem}
  Prove that every intersecting family $\mathcal{F} \subset \mathcal{P}[n]$ is contained in an intersecting family of size $2^{n-1}$.

  \begin{proof}
    We proceed by induction on $n$. The base case is trivial. Suppose $n \geq 2$. Let $\mathcal{F}_{<n} = \mathcal{F} \cap \mathcal{P}[n - 1]$ and $\mathcal{F}_{n} = \mathcal{F} \setminus \mathcal{P}[n - 1]$. By induction, $\mathcal{F}_{<n}$ is contained in an intersecting family $\mathcal{S} \subset \mathcal{P}[n - 1]$ of size $2^{n - 2}$. Note that any element of $\mathcal{P}[n - 1] \setminus \mathcal{S}$ is disjoint from some element of $\mathcal{S}$. Consider the family $\mathcal{S}' = \{A \cup \{n\} : A \in \mathcal{S}\}$. By defnition $\mathcal{S}'$ intersects with any element of $\mathcal{S}$. Notice that we also have $\mathcal{F}_n \subseteq \mathcal{S}'$, otherwise there exists $A \cup \{n\} \in \mathcal{F}_n$ such that $A \notin \mathcal{S}$, which implies that $A \cup \{n\}$ is disjoint from some element of $\mathcal{S}$, contradiction. It now follows that $\mathcal{S} \cup \mathcal{S'}$ is an intersecting family of size $2^{n - 1}$ that contains $\mathcal{F}$.
  \end{proof}
\end{homeworkProblem}

\end{document}