\documentclass{article}

% Packages
\usepackage{fancyhdr}
\usepackage{extramarks}
\usepackage{amsmath}
\usepackage{amsthm}
\usepackage{amsfonts}
\usepackage{tikz}
\usepackage[plain]{algorithm}
\usepackage{algpseudocode}
\usepackage{enumerate}
\usepackage{amssymb}

\usetikzlibrary{automata,positioning}

% Document Layout
\topmargin=-0.45in
\evensidemargin=0in
\oddsidemargin=0in
\textwidth=6.5in
\textheight=9.0in
\headsep=0.25in
\linespread{1.1}

% Page Style
\pagestyle{fancy}
\lhead{\hmwkAuthorName}
\chead{\hmwkClass:\ \hmwkTitle}
\rhead{Section \hmwkSection, \firstxmark}
\lfoot{\lastxmark}
\cfoot{\thepage}
\renewcommand\headrulewidth{0.4pt}
\renewcommand\footrulewidth{0.4pt}

% Paragraph Settings
\setlength\parindent{0pt}
\setlength{\parskip}{5pt}

% Section Management
\newcommand{\hmwkSection}{A} % Current section (A, B, or C) - update manually

% Problem Header Management
\newcommand{\enterProblemHeader}[1]{
  \nobreak\extramarks{}{Problem \arabic{#1} continued on next page\ldots}\nobreak{}
  \nobreak\extramarks{Problem \arabic{#1} (continued)}{Problem \arabic{#1} continued on next page\ldots}\nobreak{}
}

\newcommand{\exitProblemHeader}[1]{
  \nobreak\extramarks{Problem \arabic{#1} (continued)}{Problem \arabic{#1} continued on next page\ldots}\nobreak{}
  \stepcounter{#1}
  \nobreak\extramarks{Problem \arabic{#1}}{}\nobreak{}
}

% Counters
\setcounter{secnumdepth}{0}
\newcounter{partCounter}
\newcounter{homeworkProblemCounter}
\setcounter{homeworkProblemCounter}{1}
\nobreak\extramarks{Problem \arabic{homeworkProblemCounter}}{}\nobreak{}

% Homework Problem Environment
% Optional argument adjusts problem counter for non-sequential problems
\newenvironment{homeworkProblem}[1][-1]{
  \ifnum#1>0
    \setcounter{homeworkProblemCounter}{#1}
  \fi
  \section{Problem \arabic{homeworkProblemCounter}}
  \setcounter{partCounter}{1}
  \enterProblemHeader{homeworkProblemCounter}
}{
  \exitProblemHeader{homeworkProblemCounter}
}

% Assignment Details
\newcommand{\hmwkTitle}{Sheet\ \#0}
\newcommand{\hmwkDueDate}{January 26, 2026}
\newcommand{\hmwkClass}{C8.4 Probabilistic Combinatorics}
\newcommand{\hmwkClassInstructor}{Professor P. Balister}
\newcommand{\hmwkAuthorName}{\textbf{Ray Tsai}}

% Title Page
\title{
  \vspace{2in}
  \textmd{\textbf{\hmwkClass:\ \hmwkTitle}}\\
  \normalsize\vspace{0.1in}\small{Due\ on\ \hmwkDueDate\ at 12:00pm}\\
  \vspace{0.1in}\large{\textit{\hmwkClassInstructor}} \\
  \vspace{3in}
}

\author{\hmwkAuthorName}
\date{}

% Part Command
\renewcommand{\part}[1]{\textbf{\large Part \Alph{partCounter}}\stepcounter{partCounter}\\}

% Mathematical Commands
% Algorithms
\newcommand{\alg}[1]{\textsc{\bfseries \footnotesize #1}}

% Calculus
\newcommand{\deriv}[1]{\frac{\mathrm{d}}{\mathrm{d}x} (#1)}
\newcommand{\pderiv}[2]{\frac{\partial}{\partial #1} (#2)}
\newcommand{\dx}{\mathrm{d}x}

% Probability and Statistics
\newcommand{\Var}{\mathrm{Var}}
\newcommand{\Cov}{\mathrm{Cov}}
\newcommand{\Bias}{\mathrm{Bias}}
\newcommand*{\prob}{\mathbb{P}}
\newcommand*{\E}{\mathbb{E}}

% Number Sets
\newcommand*{\Z}{\mathbb{Z}}
\newcommand*{\Q}{\mathbb{Q}}
\newcommand*{\R}{\mathbb{R}}
\newcommand*{\C}{\mathbb{C}}
\newcommand*{\N}{\mathbb{N}}

\begin{document}

\maketitle
\pagebreak

\begin{homeworkProblem}
  Prove the following inequalities:
  \begin{enumerate}[(a)]
    \item $1+x \leqslant e^x$ for all real $x$.
    \begin{proof}
      Note that $y = x + 1$ is the tangent line to $y = e^x$ at $x = 0$. Since $e^x$ is convex, $e^x \geq x + 1$ for all real $x$.
    \end{proof}
    \item $e^{nx/(1+x)} \leqslant (1+x)^n \leqslant e^{nx}$ for $x > -1, n \geqslant 0$.
    \begin{proof}
      Since $x > -1$, by (a) we have $e^{x} \geq 1 + x > 0$. Hence, $(1 + x)^n \leq e^{nx}$. Let $z = -x/(1 + x)$. By (a),
      \[
        \frac{1}{1 + x} =1 + z \leq e^z = \frac{1}{e^{x/(1 + x)}} \implies 0 < e^{x/(1 + x)} \leq 1 + x.
      \] 
      The result now follows.
    \end{proof}
    \item $k! \geqslant (k/e)^k$ for $k \geqslant 1$.
    \begin{proof}
      Note that
      \[
        e^k = \sum_{i = 0}^{\infty} \frac{k^i}{i!} \geq 1 + k + \frac{k^2}{2!} + \cdots + \frac{k^k}{k!} \geq \frac{k^k}{k!}.
      \]
      The result now follows.
    \end{proof}
    \item $\left(\frac{n}{k}\right)^k \leqslant \binom{n}{k} \leqslant \frac{n^k}{k!} \leqslant \left(\frac{en}{k}\right)^k$ for $1 \leqslant k \leqslant n$.
    \begin{proof}
      It is obvious that $\binom{n}{k} \leq \frac{n^k}{k!}$. By (c), $\frac{n^k}{k!} \leq \left(\frac{ne}{k}\right)^k$. Since $n \geq k$, we have $n/k \geq (n - i)/(k - i)$ for all $1 \leq i \leq k$. Hence,
      \[
        \binom{n}{k} = \frac{n}{k} \cdot \frac{n - 1}{k - 1} \cdots \frac{n - k + 1}{1} \geq \left(\frac{n}{k}\right)^k.
      \]
    \end{proof}
  \end{enumerate}
\end{homeworkProblem}

\newpage

\begin{homeworkProblem}
  For the following functions $f(n)$ and $g(n)$, decide whether $f = o(g)$ or $g = o(f)$ or $f = \Theta(g)$ as $n \to \infty$:
  \begin{enumerate}[(a)]
    \item $f(n) = \binom{n}{k}, g(n) = n^k$, first for $k$ fixed and then for the case where $k = k(n) \to \infty$ as $n \to \infty$;
    \begin{proof}
      Since $\left(\frac{n}{k}\right)^k \leqslant \binom{n}{k} \leqslant \frac{n^k}{k!}$,
      \[
        \frac{1}{k^k} \leq \frac{f(n)}{g(n)} \leq \frac{1}{k!}.
      \]
      Thus if $k$ is fixed, then $f(n) = \Theta(g)$. If $k = k(n) \to \infty$ as $n \to \infty$, then $f(n) = o(g)$.
    \end{proof}
    \item $f(n) = (\log n)^{1000}, g(n) = n^{1/1000}$.
    \begin{proof}
      Put $x = \log n$. By applying L'Hopital's rule 1000 times, 
      \[
      \lim_{n \to \infty} \frac{f(n)}{g(n)} = \lim_{x \to \infty} \frac{x^{1000}}{e^{x/1000}} = \lim_{x \to \infty} \frac{1000!}{(0.001)^{1000}e^{x/1000}} = 0.
      \]
    \end{proof}
  \end{enumerate}
\end{homeworkProblem}

\newpage

\begin{homeworkProblem}
  Find the simplest function $f(n)$ you can such that $(n-2)^{n+2}/n^n \sim f(n)$ as $n \to \infty$.

  \begin{proof}
    Note that $(n-2)^{n+2}/n^n = (n - 2)^2 \cdot (1 - 2/n)^{n}$. But then
    \[
      \lim_{n \to \infty} \left(1 - \frac{2}{n}\right)^{n} = e^{-2}.
    \]
    Thus,
    \[
      (n-2)^{n+2}/n^n \sim n^2 e^{-2}.
    \]
  \end{proof}
\end{homeworkProblem}

\newpage

\begin{homeworkProblem}
  Show that if $n, k, \ell \geqslant 1$ are integers and $0 < p < 1$, then
  \[
  R(k, \ell) > n - \binom{n}{k} p^{\binom{k}{2}} - \binom{n}{\ell} (1-p)^{\binom{\ell}{2}}.
  \]

  \begin{proof}
    Let $G \sim G(n, p)$. Let $X$ be the number of cliques of size $k$ and independent sets of size $\ell$ in $G$. Then $\mu = \binom{n}{k} p^{\binom{k}{2}} + \binom{n}{\ell} (1-p)^{\binom{\ell}{2}}$ is the expected number of $k$-cliques and $\ell$-independent sets in $G$. Since $\prob(X \leq \mu) > 0$, there exists a configuration such that $G$ has at most $\mu$ $k$-cliques and $\ell$-independent sets. Removing a vertex from each $k$-clique and $\ell$-independent set in $G$ yields a ramsey graph with at least $n - \mu$ vertices. The result now follows.
  \end{proof}
\end{homeworkProblem}

\newpage

\begin{homeworkProblem}
  Let $H$ be an $r$-uniform hypergraph with fewer than $\frac{3^{r-1}}{2^r}$ edges. Prove that the vertices of $H$ can be coloured using three colours in such a way that in each edge, all three colours are represented.

  \begin{proof}
    Randomly the vertices of $H$ with 3 colors uniformly and independently. Let $A_e$ be the event that not all three colors are represented in edge $e$. Then,
    \[
      \prob(A_e) = \frac{2^{r} - 1}{3^{r - 1}}.
    \]
    Then the probability that there exists an edge without all three colors is
    \[
      \prob\left(\bigcup_{e \in E(H)} A_e\right) \leq \sum_{e \in E(H)} \prob(A_e) < \frac{3^{r-1}}{2^r} \cdot \frac{2^{r} - 1}{3^{r - 1}} < 1.
    \]
    Thus, there exists a coloring of $H$ such that no edge contains all three colors. 
  \end{proof}
\end{homeworkProblem}

\newpage

\begin{homeworkProblem}
  Let $F$ be a collection of binary strings (``codewords'') of finite length, where the $i$th codeword has length $c_i$. Suppose that no member of $F$ is an initial segment of another member (so you can decode any string made up by concatenating codewords as you go along, without looking ahead). Show that $\sum_i 2^{-c_i} \leqslant 1$ (the \textit{Kraft inequality} for prefix-free codes).

  \begin{proof}
    Let $s$ be a random string of infinite length. Let $X$ be the number of codewords in $F$ that are prefixes of $s$. Since no member of $F$ is an initial segment of another member, $X \leq 1$. But then
    \[
      1 \geq \E[X] = \sum_{s_i \in F} \prob(s_i \text{ is a prefix of } s) = \sum_{s_i \in F} 2^{-c_i}.
    \]
  \end{proof}
\end{homeworkProblem}

\end{document}