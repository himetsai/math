\documentclass{article}

% Packages
\usepackage{fancyhdr}
\usepackage{extramarks}
\usepackage{amsmath}
\usepackage{amsthm}
\usepackage{amsfonts}
\usepackage{tikz}
\usepackage[plain]{algorithm}
\usepackage{algpseudocode}
\usepackage{enumerate}
\usepackage{bbm}

\usetikzlibrary{automata,positioning}

% Document Layout
\topmargin=-0.45in
\evensidemargin=0in
\oddsidemargin=0in
\textwidth=6.5in
\textheight=9.0in
\headsep=0.25in
\linespread{1.1}

% Page Style
\pagestyle{fancy}
\lhead{\hmwkAuthorName}
\chead{\hmwkClass:\ \hmwkTitle}
\rhead{Section \hmwkSection, \firstxmark}
\lfoot{\lastxmark}
\cfoot{\thepage}
\renewcommand\headrulewidth{0.4pt}
\renewcommand\footrulewidth{0.4pt}

% Paragraph Settings
\setlength\parindent{0pt}
\setlength{\parskip}{5pt}

% Section Management
\newcommand{\hmwkSection}{A} % Current section (A, B, or C) - update manually

% Problem Header Management
\newcommand{\enterProblemHeader}[1]{
  \nobreak\extramarks{}{Problem \arabic{#1} continued on next page\ldots}\nobreak{}
  \nobreak\extramarks{Problem \arabic{#1} (continued)}{Problem \arabic{#1} continued on next page\ldots}\nobreak{}
}

\newcommand{\exitProblemHeader}[1]{
  \nobreak\extramarks{Problem \arabic{#1} (continued)}{Problem \arabic{#1} continued on next page\ldots}\nobreak{}
  \stepcounter{#1}
  \nobreak\extramarks{Problem \arabic{#1}}{}\nobreak{}
}

% Counters
\setcounter{secnumdepth}{0}
\newcounter{partCounter}
\newcounter{homeworkProblemCounter}
\setcounter{homeworkProblemCounter}{1}
\nobreak\extramarks{Problem \arabic{homeworkProblemCounter}}{}\nobreak{}

% Homework Problem Environment
% Optional argument adjusts problem counter for non-sequential problems
\newenvironment{homeworkProblem}[1][-1]{
  \ifnum#1>0
    \setcounter{homeworkProblemCounter}{#1}
  \fi
  \section{Problem \arabic{homeworkProblemCounter}}
  \setcounter{partCounter}{1}
  \enterProblemHeader{homeworkProblemCounter}
}{
  \exitProblemHeader{homeworkProblemCounter}
}

% Assignment Details
\newcommand{\hmwkTitle}{Sheet\ \#1}
\newcommand{\hmwkDueDate}{January 29, 2026}
\newcommand{\hmwkClass}{C8.4 Probabilistic Combinatorics}
\newcommand{\hmwkClassInstructor}{Professor P. Balister}
\newcommand{\hmwkAuthorName}{\textbf{Ray Tsai}}

% Title Page
\title{
  \vspace{2in}
  \textmd{\textbf{\hmwkClass:\ \hmwkTitle}}\\
  \normalsize\vspace{0.1in}\small{Due\ on\ \hmwkDueDate\ at 12:00pm}\\
  \vspace{0.1in}\large{\textit{\hmwkClassInstructor}} \\
  \vspace{3in}
}

\author{\hmwkAuthorName}
\date{}

% Part Command
\renewcommand{\part}[1]{\textbf{\large Part \Alph{partCounter}}\stepcounter{partCounter}\\}

% Mathematical Commands
% Algorithms
\newcommand{\alg}[1]{\textsc{\bfseries \footnotesize #1}}

% Calculus
\newcommand{\deriv}[1]{\frac{\mathrm{d}}{\mathrm{d}x} (#1)}
\newcommand{\pderiv}[2]{\frac{\partial}{\partial #1} (#2)}
\newcommand{\dx}{\mathrm{d}x}

% Probability and Statistics
\newcommand{\Var}{\mathrm{Var}}
\newcommand{\Cov}{\mathrm{Cov}}
\newcommand{\Bias}{\mathrm{Bias}}
\newcommand*{\prob}{\mathbb{P}}
\newcommand*{\E}{\mathbb{E}}

% Number Sets
\newcommand*{\Z}{\mathbb{Z}}
\newcommand*{\Q}{\mathbb{Q}}
\newcommand*{\R}{\mathbb{R}}
\newcommand*{\C}{\mathbb{C}}
\newcommand*{\N}{\mathbb{N}}
\newcommand*{\F}{\mathbb{F}}

\begin{document}

\maketitle
\pagebreak

\begin{homeworkProblem}
  \begin{enumerate}[(a)]
    \item Show that for $1 \leq k \leq n$
    \[
    \exp\left(-\frac{k(k-1)}{2(n-k+1)}\right) \leq \prod_{i=0}^{k-1} \left(1 - \frac{i}{n}\right) \leq \exp\left(-\frac{k(k-1)}{2n}\right).
    \]
    \begin{proof}
      Since $e^{-x/(1 - x)} \leq 1 - x \leq e^{-x}$ for $x < 1$,
      \[
        -\frac{i}{n - i} \leq \ln \left(1 - \frac{i}{n}\right) \leq -\frac{i}{n}.
      \]
      It now follows that
      \[
        -\frac{k(k-1)}{2(n-k+1)} \leq \sum_{i = 0}^{k - 1} -\frac{i}{n - i} \leq \ln \prod_{i=0}^{k-1} \left(1 - \frac{i}{n}\right) = \sum_{i = 0}^{k - 1} \ln \left(1 - \frac{i}{n}\right) \leq \sum_{i = 0}^{k - 1} - \frac{i}{n} = -\frac{k(k-1)}{2n}.
      \]
    \end{proof}

    \item Deduce, using Stirling's formula, that for $n, k \to \infty$ and $k = o(n^{2/3})$,
    \[
    \binom{n}{k} \sim \frac{1}{\sqrt{2\pi k}} \left(\frac{en}{k}\right)^k e^{-k^2/2n}.
    \]
    \begin{proof}
      Since $\prod_{i=0}^{k - 1} \left(1 - \frac{i}{n}\right) = \frac{n!}{k!n^{k}}$,
      \[
        \frac{n^{k}}{k!}\exp\left(-\frac{k(k-1)}{2(n - k + 1)}\right) \leq \binom{n}{k} = \frac{n^{k}}{k!}\prod_{i=0}^{k-1} \left(1 - \frac{i}{n}\right) \leq \frac{n^{k}}{k!}\exp\left(-\frac{k(k-1)}{2n}\right).
      \]
      By Stirling's Formula,
      \[
        \frac{n^{k}}{k!}\sim \frac{1}{\sqrt{2\pi k}}\left(\frac{en}{k}\right)^k.
      \]
      Since $k = o(n^{2/3})$, we have $k(k-1)/2n \sim k^2/2n$ and $k(k - 1)/2(n - k + 1) = k^2/2(1 + o(1))n + o(1) \sim k^2/2n$. The result now follows from combining all of the above.
    \end{proof}
  \end{enumerate}
\end{homeworkProblem}

\newpage

\begin{homeworkProblem}
  Show that for $r \ge 2$, any graph $G$ contains an $r$-partite subgraph $H$ with $e(H) \ge \frac{r-1}{r}e(G)$. [\textit{Hint: randomly assign each vertex to a part.}]

  \begin{proof}
    Randomly color the vertices of $G$ with $r$ colors uniformly and independently. Let $H$ be the $r$-partite subgraph with no edges between vertices with the same color. Then
    \[
      \E[e(H)] = \sum_{\{u, v\} \in E(G)} \prob(u \text{ and } v \text{ have different colors}) = \frac{r-1}{r}e(G).
    \]
    Thus there is a coloring such that $e(H) \geq \frac{r-1}{r}e(G)$.
  \end{proof}
\end{homeworkProblem}

\newpage

\begin{homeworkProblem}
  Prove the Paley--Zygmund inequality: for any non-negative random variable $X$ with finite variance,
  \[
  \mathbb{P}(X > 0) \ge \frac{\mathbb{E}[X]^2}{\mathbb{E}[X^2]}.
  \]
  [\textit{Hint: Cauchy--Schwarz.}] How does this compare with the bound from Chebyshev?

  \begin{proof}
    Write $X = X \cdot \mathbbm{1}_{\{X > 0\}}$. Then by the Cauchy--Schwarz inequality,
    \[
      \E[X]^2 \leq \E[X^2] \cdot \E[\mathbbm{1}_{\{X > 0\}}^2] = \E[X^2] \cdot \mathbb{P}(X > 0).
    \]
    The result now follows from rearranging the above.
  \end{proof}
\end{homeworkProblem}

\newpage

% To change sections, use: \renewcommand{\hmwkSection}{B}
% Example: Uncomment the line below when you start Section B
\renewcommand{\hmwkSection}{B}

\begin{homeworkProblem}
  A \textit{dominating set} in a graph $G = (V, E)$ is a set $U \subseteq V$ such that every vertex $v \in V$ is either in $U$ or has a neighbour in $U$.

  Suppose that $|V| = n$ and that $G$ has minimum degree $\delta \geq 2$. Choose a subset $X$ of $V$ by including each vertex independently with probability $p$. Let $Y$ be the set of all vertices which are not in $X$ and which have no neighbour in $X$.

  Show that $\mathbb{E}[|X \cup Y|] \leq np + n e^{-p(\delta+1)}$. What can you say about the set $X \cup Y$?

  By optimizing over $p$, show that the graph $G$ has a dominating set which contains at most $n \frac{1+\log(\delta+1)}{\delta+1}$ vertices.

  \begin{proof}
    Note that
    \begin{align*}
      \E[|X \cup Y|] 
      &= \sum_{v \in V} \prob(v \in X \cup Y) \\
      &\leq \sum_{v \in V} \prob(v \in X) + \sum_{v \in V} \prob(v \in Y) \\
      &= np + \sum_{v \in V} \prob(v \notin X) \cdot \prob(N(v) \cap X = \emptyset) \\
      &\leq np + \sum_{v \in V} (1 - p)^{\delta + 1} \\
      &\leq np + n e^{-p(\delta + 1)}.
    \end{align*}
    Notice that for $v \in V$, if $v \notin X$, then either $v \in Y$ or $v$ has a neighbour in $X$. Thus, $X \cup Y$ is a dominating set. Set $p = \log(\delta + 1)/(\delta + 1)$. Then
    \[
      \E[|X \cup Y|] \leq n\left(\frac{\log(\delta + 1)}{\delta + 1} + \frac{1}{\delta + 1}\right) = n \cdot \frac{1+\log(\delta+1)}{\delta+1}.
    \]
  \end{proof}
\end{homeworkProblem}

\newpage

\begin{homeworkProblem}
  For $n, r \in \mathbb{N}$, $1 < r < n$, let $z(r, n)$ be the largest possible number of 0 entries in an $n \times n$ matrix which has no $r \times r$ submatrix whose entries are all 0. (Here a submatrix is obtained by selecting any $r$ rows and any $r$ columns; the rows or columns need not be consecutive.)

  Consider a random matrix in which each entry is 0 with probability $p$ and 1 with probability $1 - p$, independently. What is the expected number of entries which are 0? What is the expected number of $r \times r$ submatrices which are ``all 0''?

  Deduce that $z(r, n) \ge p n^2 - p^{r^2} n^{2r} / r!^2$.

  Optimize over $p$ to find the best lower bound on $z(r, n)$ that you can, for fixed $r$ and large $n$.

  \begin{proof}
    Let $M$ be the random matrix. Let $X$ be the number of $0$ entries in $M$. Then $\E[X] = pn^2$. Let $R$ be the number of $r \times r$ submatrices which are ``all 0''. Then
    \[
      \E[R] = \sum_{A \text{ is a } r \times r \text{ submatrix of } M} \prob(A \text{ is all 0}) = \binom{n}{r}^2 p^{r^2}\leq \frac{n^{2r}p^{r^2}}{r!^2}.
    \]
    Obtain matrix $M'$ by flipping an entry of each $r \times r$ submatrix of $M$ which is ``all 0''. Note that $M'$ has no $r \times r$ submatrix which is ``all 0'' and the number of $0$ entries in $M'$ is $Z = X - R$. Thus,
    \[
      \E[Z] = \E[X] - \E[R] = pn^2 - \frac{n^{2r}p^{r^2}}{r!^2}.
    \]
    But then there is an $n \times n$ matrix $M'$ without any $r \times r$ submatrix which is ``all 0'' and the number of $0$ entries in $M'$ is at least $pn^2 - n^{2r}p^{r^2}/r!^2$. Thus, $z(r, n) \geq pn^2 - n^{2r}p^{r^2}/r!^2$. By calculus, the bound is optimized when $p = c_rn^{-2/(r + 1)}$, where $c_r = (r - 1)!^{2/(r^2 - 1)}$.
  \end{proof}
\end{homeworkProblem}

\newpage

\begin{homeworkProblem}
  Let $G$ be a graph with $n$ vertices, and let $d_v$ denote the degree of vertex $v$.
  \begin{enumerate}[(a)]
      \item Consider a random ordering of $V = V(G)$ (chosen uniformly from all $n!$ possibilities). What is the probability that $v$ precedes all its neighbours in the ordering?
      \begin{proof}
        The probability that $v$ precedes all its neighbours in the ordering is equals the probability that it is the first vertex in the ordering among its neighbours, which is $\frac{1}{d_v+1}$.
      \end{proof}
      \item Show that $G$ has an independent set of size at least $\sum_{v \in V} \frac{1}{d_v+1}$.
      \begin{proof}
        Consider a random ordering of $V$, and let $I$ be the set of all vertices that precede all their neighbours in the ordering. Notice that for $u, v \in I$, $u$ and $v$ are not neighbours, otherwise one would precede the other in the ordering. Thus, $I$ is an independent set. Since
        \[
          \E[|I|] = \sum_{v \in V} \prob(v \in I) = \sum_{v \in V} \frac{1}{d_v+1},
        \]
        there exists an independent set of size at least $\sum_{v \in V} \frac{1}{d_v+1}$.
      \end{proof}
      \item Deduce that any graph with $n$ vertices and $m$ edges has an independent set of size at least $\frac{n^2}{2m+n}$.
      \begin{proof}
        Note that the function $f(x) = 1/(x + 1)$ is convex on $[0, \infty)$. Thus by Jensen's inequality, 
        \[
          \sum_{v \in V} \frac{1}{d_v + 1} \geq \frac{n}{1 + \frac{1}{n}\sum_{v \in V} d_v} = \frac{n^2}{2m + n}.
        \]
        The result now follows from (b).
      \end{proof}
  \end{enumerate}
\end{homeworkProblem}

\newpage

\begin{homeworkProblem}
  Let $G$ be a bipartite graph with $n$ vertices. Suppose each vertex $v$ has a list $S(v)$ of more than $\log_2 n$ colours associated to it. Show that there is a proper colouring of $G$ in which each vertex $v$ receives a colour from its list $S(v)$.

  \begin{proof}
    Let $A$ and $B$ be the two parts of $G$. Let $C = \bigcup_{v \in V(G)} S(v)$ and let $C_A$ be a random subset of $C$ such that each element of $C$ is included in $C_A$ with probability $1/2$. Then for $v \in V(G)$,
    \[
      \prob(S(v) \cap C_A = \emptyset) = \prob(S(v) \cap C_B = \emptyset) < 2^{-\log_2 n} = \frac{1}{n}.
    \]
    
    Let $C_B = C \setminus C_A$. For $v \in V(G)$, assign it the first available color in $C_A$ if $v \in A$ and the first available color in $C_B$ if $v \in B$. Then
    \[
      \prob(\text{coloring is improper}) \leq \sum_{v \in A} \prob(S(v) \cap C_A = \emptyset) + \sum_{v \in B} \prob(S(v) \cap C_B = \emptyset) < n \cdot \frac{1}{n} = 1.
    \]
    Thus there is a proper coloring of $G$ in which each vertex $v$ receives a colour from its list $S(v)$.
  \end{proof}
\end{homeworkProblem}

\newpage

\begin{homeworkProblem}
  Let $p = p(n) = \frac{\log n + f(n)}{n}$. Show that if $f(n) \to \infty$ then the probability that $G(n, p)$ contains an isolated vertex tends to 0, and that if $f(n) \to -\infty$ then this probability tends to 1. [\textit{Hint. Apply the first and second moment methods to the number of isolated vertices. We may assume (why?) that $f(n)$ is not too large, say $|f(n)| \le \log n$.}]

  (This shows in particular that $p^*(n) = \log n / n$ is a threshold function for $G(n, p)$ to have minimum degree at least 1.)

  \begin{proof}    
    Let $X$ be the number of isolated vertices in $G(n, p)$. Then
    \[
      \E[X] = \sum_{v \in V(G)} \prob(v \text{ is isolated}) = n (1 - p)^{n - 1} \leq ne^{-np} = e^{-f(n)}.
    \]
    Thus if $f(n) \to \infty$, then $\E[X] \to 0$ and by Markov's inequality, $\prob(X > 0) \to 0$. Now suppose $f(n) \to -\infty$. Then
    \begin{align*}
      \E[X^2] 
      &= \sum_{v \in V(G)} \E[\mathbbm{1}_{d(v) = 0}^2] + \sum_{\{u, v\} \subseteq V(G)} \E[\mathbbm{1}_{d(u) = 0} \cdot \mathbbm{1}_{d(v) = 0}] \\
      &= \E[X] + \sum_{u \neq v} \prob(N(u) \cup N(v) = \emptyset) \\
      &= \E[X] + n(n - 1) (1 - p)^{2n - 3} \\
      &= \E[X] + n(n - 1)(\E[X]^2 \cdot (1 - p)^{-1}).
    \end{align*}
    But then $p \to 0$, so 
    \[
      \frac{\E[X^2]}{\E[X]^2} = \frac{1}{\E[X]} + \left(1 - \frac{1}{n}\right) (1 - p)^{-1} \to 1.
    \]
    By Chebyshev's inequality,
    \[
      \prob(X = 0) \leq \prob(|X - \E[X]| \geq \E[X]) \leq \frac{\E[X^2]}{\E[X]^2} - 1 \to 0.
    \]
  \end{proof}
\end{homeworkProblem}

\newpage

\begin{homeworkProblem}
  Let $S_{n,p}$ be a random subset of $\{1, 2, \dots, n\}$ chosen by including each element independently with probability $p$.

    \begin{enumerate}[(a)]
      \item Show that $p = n^{-2/3}$ is a threshold function for the property ``$S_{n,p}$ contains an arithmetic progression of length 3''.
      \begin{proof}
        Suppose $n^{2/3}p \to 0$. For $a \in S_{n, p}$, let $X$ be the number of arithmetic progressions of length 3. Then
        \[
          \E[X] = \sum_{a \in [n]}\sum_{d \leq (n - a)/2} p^3 = \frac{p^3}{2}\sum_{a \in [n]} n - a = \frac{p^3n(n - 1)}{4} \to 0.
        \]
        Thus by Markov's inequality, $\prob(X > 0) \to 0$. Now suppose $n^{2/3}p \to \infty$. For 3 term arithmetic progression $A \subseteq [n]$, let $\mathbbm{1}_{A}$ be the indicator random variable that $A \subseteq S_{n, p}$. Since the events $A \subseteq S_{n, p}$ and $B \subseteq S_{n, p}$ are positively correlated, $\E[\mathbbm{1}_{A} \mathbbm{1}_{B}] \geq \E[\mathbbm{1}_{A}] \E[\mathbbm{1}_{B}]$. Thus,
        \[
          \E[X^2] = \sum_{A \text{ is 3-AP}} \E[\mathbbm{1}_{A}^2] + \sum_{A \neq B \text{ are 3-APs}} \E[\mathbbm{1}_{A} \mathbbm{1}_{B}] \geq \E[X] + \sum_{A \neq B \text{ are 3-APs}} p^6.
        \]
        Note that there are $\binom{n(n - 1)/4}{2} \sim n^4/32$ distinct pairs of 3-APs in $[n]$. Thus,
        \[
          \frac{\E[X^2]}{\E[X]^2} = \frac{1}{\E[X]} + \frac{\binom{n(n - 1)/4}{2} p^6}{(p^3n(n - 1)/4)^2} \to 1.
        \]
        By Chebyshev's inequality,
        \[
          \prob(X = 0) \leq \prob(|X - \E[X]| \geq \E[X]) \leq \frac{\E[X^2]}{\E[X]^2} - 1 \to 0.
        \]
      \end{proof}
      \item Show that for $k \ge 3$ fixed, $p = n^{-2/k}$ is a threshold function for $S_{n,p}$ to contain an arithmetic progression of length $k$.
      \begin{proof}
        Suppose $n^{2/k}p \to 0$. For $a \in S_{n, p}$, let $X$ be the number of arithmetic progressions of length $k$. Each $k$-AP can be determined by the first element $a \in [n]$ and the common difference $d \leq (n - a)/(k - 1)$. Thus, the number of $k$-APs in $[n]$ is $N = n(n - 1)/2(k - 1) \sim n^2/2(k - 1)$. But then
        \[
          \E[X] = \sum_{A \text{is $k$-AP}} p^k = \frac{n(n - 1)p^k}{k - 1} \to 0.
        \]
        Thus by Markov's inequality, $\prob(X > 0) \to 0$. Now suppose $n^{2/k}p \to \infty$. For $k$-term arithmetic progression $A \subseteq [n]$, let $\mathbbm{1}_{A}$ be the indicator random variable that $A \subseteq S_{n, p}$. Since the events $A \subseteq S_{n, p}$ and $B \subseteq S_{n, p}$ are positively correlated, $\E[\mathbbm{1}_{A} \mathbbm{1}_{B}] \geq \E[\mathbbm{1}_{A}] \E[\mathbbm{1}_{B}]$. Thus,
        \[
          \E[X^2] = \sum_{A \text{ is $k$-AP}} \E[\mathbbm{1}_{A}^2] + \sum_{A \neq B \text{ are $k$-APs}} \E[\mathbbm{1}_{A} \mathbbm{1}_{B}] \geq \E[X] + \sum_{A \neq B \text{ are $k$-APs}} p^{2k}.
        \]
        Note that there are $\binom{N}{2} \sim n^4/4(k - 1)^2$ distinct pairs of $k$-APs in $[n]$. Thus,
        \[
          \frac{\E[X^2]}{\E[X]^2} \sim \frac{1}{\E[X]} + \frac{n^4p^{2k}/4(k - 1)^2}{(p^kn(n - 1)/2(k - 1))^2} \to 1.
        \]
        By Chebyshev's inequality,
        \[
          \prob(X = 0) \leq \prob(|X - \E[X]| \geq \E[X]) \leq \frac{\E[X^2]}{\E[X]^2} - 1 \to 0.
        \]
      \end{proof}
    \end{enumerate}
\end{homeworkProblem}

\end{document}