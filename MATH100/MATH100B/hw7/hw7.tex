\documentclass{article}

\usepackage{fancyhdr}
\usepackage{extramarks}
\usepackage{amsmath}
\usepackage{amsthm}
\usepackage{amsfonts}
\usepackage{tikz}
\usepackage[plain]{algorithm}
\usepackage{algpseudocode}
\usepackage{enumerate}
\usepackage{amssymb}

\usetikzlibrary{automata,positioning}

%
% Basic Document Settings
%

\topmargin=-0.45in
\evensidemargin=0in
\oddsidemargin=0in
\textwidth=6.5in
\textheight=9.0in
\headsep=0.25in

\linespread{1.1}

\pagestyle{fancy}
\lhead{\hmwkAuthorName}
\chead{\hmwkClass:\ \hmwkTitle}
\rhead{\firstxmark}
\lfoot{\lastxmark}
\cfoot{\thepage}

\renewcommand\headrulewidth{0.4pt}
\renewcommand\footrulewidth{0.4pt}

\setlength\parindent{0pt}
\setlength{\parskip}{5pt}

%
% Create Problem Sections
%

\newcommand{\enterProblemHeader}[1]{
    \nobreak\extramarks{}{Problem \arabic{#1} continued on next page\ldots}\nobreak{}
    \nobreak\extramarks{Problem \arabic{#1} (continued)}{Problem \arabic{#1} continued on next page\ldots}\nobreak{}
}

\newcommand{\exitProblemHeader}[1]{
    \nobreak\extramarks{Problem \arabic{#1} (continued)}{Problem \arabic{#1} continued on next page\ldots}\nobreak{}
    \stepcounter{#1}
    \nobreak\extramarks{Problem \arabic{#1}}{}\nobreak{}
}

\setcounter{secnumdepth}{0}
\newcounter{partCounter}
\newcounter{homeworkProblemCounter}
\setcounter{homeworkProblemCounter}{1}
\nobreak\extramarks{Problem \arabic{homeworkProblemCounter}}{}\nobreak{}

%
% Homework Problem Environment
%
% This environment takes an optional argument. When given, it will adjust the
% problem counter. This is useful for when the problems given for your
% assignment aren't sequential. See the last 3 problems of this template for an
% example.
%
\newenvironment{homeworkProblem}[1][-1]{
    \ifnum#1>0
        \setcounter{homeworkProblemCounter}{#1}
    \fi
    \section{Problem \arabic{homeworkProblemCounter}}
    \setcounter{partCounter}{1}
    \enterProblemHeader{homeworkProblemCounter}
}{
    \exitProblemHeader{homeworkProblemCounter}
}

%
% Homework Details
%   - Title
%   - Due date
%   - Class
%   - Section/Time
%   - Instructor
%   - Author
%

\newcommand{\hmwkTitle}{Homework\ \#7}
\newcommand{\hmwkDueDate}{Feb 29, 2024}
\newcommand{\hmwkClass}{MATH 100B}
\newcommand{\hmwkClassTime}{Section A02 6:00PM - 6:50PM}
\newcommand{\hmwkSectionLeader}{Castellano-Macías}
\newcommand{\hmwkClassInstructor}{Professor McKernan}
\newcommand{\hmwkSource}{Source Consulted: Textbook, Lecture, Discussion, Office Hour}
\newcommand{\hmwkAuthorName}{\textbf{Ray Tsai}}
\newcommand{\hmwkPID}{A16848188}

%
% Title Page
%

\title{
    \vspace{2in}
    \textmd{\textbf{\hmwkClass:\ \hmwkTitle}}\\
    \normalsize\vspace{0.1in}\small{Due\ on\ \hmwkDueDate\ at 12:00pm}\\
    \vspace{0.1in}\large{\textit{\hmwkClassInstructor}} \\
    \vspace{0.1in}\small\hmwkClassTime \\
    \small Section Leader: \hmwkSectionLeader \\
    \vspace{0.1in}\small\hmwkSource \\
    \vspace{3in}
}

\author{
  \hmwkAuthorName \\
  \vspace{0.1in}\small\hmwkPID
}
\date{}

\renewcommand{\part}[1]{\textbf{\large Part \Alph{partCounter}}\stepcounter{partCounter}\\}

%
% Various Helper Commands
%

% Useful for algorithms
\newcommand{\alg}[1]{\textsc{\bfseries \footnotesize #1}}

% For derivatives
\newcommand{\deriv}[1]{\frac{\mathrm{d}}{\mathrm{d}x} (#1)}

% For partial derivatives
\newcommand{\pderiv}[2]{\frac{\partial}{\partial #1} (#2)}

% Integral dx
\newcommand{\dx}{\mathrm{d}x}

% Probability commands: Expectation, Variance, Covariance, Bias
\newcommand{\Var}{\mathrm{Var}}
\newcommand{\Cov}{\mathrm{Cov}}
\newcommand{\Bias}{\mathrm{Bias}}
\newcommand*{\Z}{\mathbb{Z}}
\newcommand*{\Q}{\mathbb{Q}}
\newcommand*{\R}{\mathbb{R}}
\newcommand*{\C}{\mathbb{C}}
\newcommand*{\N}{\mathbb{N}}
\newcommand*{\F}{\mathbb{F}}
\newcommand*{\prob}{\mathds{P}}
\newcommand*{\E}{\mathds{E}}

\begin{document}

\maketitle

\pagebreak

\begin{homeworkProblem}
  Are the following true or false? If true prove them; if false give
  counterexamples.
  \begin{enumerate}[(i)]
    \item Every prime ideal is maximal.
    \begin{proof}
      False. Consider $\langle x \rangle$ in $\Z[x]$. $\Z[x]/\langle x \rangle \simeq \Z$, so
      $\langle x \rangle$ is a prime ideal but not maximal. 
    \end{proof}
    \item Every ring is a PID.
    \begin{proof}
      False. Consider $\Z[x^2, x^3]$. Notice that since $x \notin \Z[x^2, x^3]$, $x^2$ and $x^3$ are
      irreducible. But then $x^6 = (x^2)(x^2)(x^2) = (x^3)(x^3)$, so the factorization is not unique
      in $\Z[x^2, x^3]$. It follows that $\Z[x^2, x^3]$ is not a UFD, and thus not a PID.
    \end{proof}
    \item Every UFD is a PID.
    \begin{proof}
      False. Consider $\Z[x]$. Since $\Z$ is a UFD, $\Z[x]$ is a UFD, by Gauss' Lemma. Let $I =
      \langle x^2, 4 \rangle$. Note that $x \notin I$ so $I \neq \Z[x]$. Suppose for the sake of
      contradiction that $I = \langle p(x) \rangle$ for some $p(x) \in \Z[x]$. We know $p(x) = a$
      for some $a \in \Z$, otherwise $4 \notin I$. But then $a = 1$, otherwise $x^2 \notin I$. The
      result now follows from the contradiction that $I = \Z[x]$.
    \end{proof}
  \end{enumerate}
\end{homeworkProblem}

\newpage

\begin{homeworkProblem}
  If
  \[
    N_1 \subseteq N_2 \subseteq N_3 \subseteq \cdots
  \]
  is an ascending chain of submodules of an $R$-module $M$ then show that the union $N$ is a
  submodule.

  \begin{proof}
    Let $m, n \in N$. We know $m \in N_i$, $n \in N_j$, for some $i, j$. Let $k = \max(i, j)$. Since
    $m, n \in N_k$, we have $m + n, m^{-1}, 0 \in N_k$ and $m + n = n + m$. Hence, $N$ is an abelian
    group under addition. Since $m, n \in N_k$, $m, n$ follows all $4$ multiplicative conditions of
    a module, and the result follows
  \end{proof}
\end{homeworkProblem}
\end{document}