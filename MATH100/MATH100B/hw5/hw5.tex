\documentclass{article}

\usepackage{fancyhdr}
\usepackage{extramarks}
\usepackage{amsmath}
\usepackage{amsthm}
\usepackage{amsfonts}
\usepackage{tikz}
\usepackage[plain]{algorithm}
\usepackage{algpseudocode}
\usepackage{enumerate}
\usepackage{amssymb}

\usetikzlibrary{automata,positioning}

%
% Basic Document Settings
%

\topmargin=-0.45in
\evensidemargin=0in
\oddsidemargin=0in
\textwidth=6.5in
\textheight=9.0in
\headsep=0.25in

\linespread{1.1}

\pagestyle{fancy}
\lhead{\hmwkAuthorName}
\chead{\hmwkClass:\ \hmwkTitle}
\rhead{\firstxmark}
\lfoot{\lastxmark}
\cfoot{\thepage}

\renewcommand\headrulewidth{0.4pt}
\renewcommand\footrulewidth{0.4pt}

\setlength\parindent{0pt}
\setlength{\parskip}{5pt}

%
% Create Problem Sections
%

\newcommand{\enterProblemHeader}[1]{
    \nobreak\extramarks{}{Problem \arabic{#1} continued on next page\ldots}\nobreak{}
    \nobreak\extramarks{Problem \arabic{#1} (continued)}{Problem \arabic{#1} continued on next page\ldots}\nobreak{}
}

\newcommand{\exitProblemHeader}[1]{
    \nobreak\extramarks{Problem \arabic{#1} (continued)}{Problem \arabic{#1} continued on next page\ldots}\nobreak{}
    \stepcounter{#1}
    \nobreak\extramarks{Problem \arabic{#1}}{}\nobreak{}
}

\setcounter{secnumdepth}{0}
\newcounter{partCounter}
\newcounter{homeworkProblemCounter}
\setcounter{homeworkProblemCounter}{1}
\nobreak\extramarks{Problem \arabic{homeworkProblemCounter}}{}\nobreak{}

%
% Homework Problem Environment
%
% This environment takes an optional argument. When given, it will adjust the
% problem counter. This is useful for when the problems given for your
% assignment aren't sequential. See the last 3 problems of this template for an
% example.
%
\newenvironment{homeworkProblem}[1][-1]{
    \ifnum#1>0
        \setcounter{homeworkProblemCounter}{#1}
    \fi
    \section{Problem \arabic{homeworkProblemCounter}}
    \setcounter{partCounter}{1}
    \enterProblemHeader{homeworkProblemCounter}
}{
    \exitProblemHeader{homeworkProblemCounter}
}

%
% Homework Details
%   - Title
%   - Due date
%   - Class
%   - Section/Time
%   - Instructor
%   - Author
%

\newcommand{\hmwkTitle}{Homework\ \#5}
\newcommand{\hmwkDueDate}{Feb 15, 2024}
\newcommand{\hmwkClass}{MATH 100B}
\newcommand{\hmwkClassTime}{Section A02 6:00PM - 6:50PM}
\newcommand{\hmwkSectionLeader}{Castellano-Macías}
\newcommand{\hmwkClassInstructor}{Professor McKernan}
\newcommand{\hmwkSource}{Source Consulted: Textbook, Lecture, Discussion, Office Hour}
\newcommand{\hmwkAuthorName}{\textbf{Ray Tsai}}
\newcommand{\hmwkPID}{A16848188}

%
% Title Page
%

\title{
    \vspace{2in}
    \textmd{\textbf{\hmwkClass:\ \hmwkTitle}}\\
    \normalsize\vspace{0.1in}\small{Due\ on\ \hmwkDueDate\ at 12:00pm}\\
    \vspace{0.1in}\large{\textit{\hmwkClassInstructor}} \\
    \vspace{0.1in}\small\hmwkClassTime \\
    \small Section Leader: \hmwkSectionLeader \\
    \vspace{0.1in}\small\hmwkSource \\
    \vspace{3in}
}

\author{
  \hmwkAuthorName \\
  \vspace{0.1in}\small\hmwkPID
}
\date{}

\renewcommand{\part}[1]{\textbf{\large Part \Alph{partCounter}}\stepcounter{partCounter}\\}

%
% Various Helper Commands
%

% Useful for algorithms
\newcommand{\alg}[1]{\textsc{\bfseries \footnotesize #1}}

% For derivatives
\newcommand{\deriv}[1]{\frac{\mathrm{d}}{\mathrm{d}x} (#1)}

% For partial derivatives
\newcommand{\pderiv}[2]{\frac{\partial}{\partial #1} (#2)}

% Integral dx
\newcommand{\dx}{\mathrm{d}x}

% Probability commands: Expectation, Variance, Covariance, Bias
\newcommand{\Var}{\mathrm{Var}}
\newcommand{\Cov}{\mathrm{Cov}}
\newcommand{\Bias}{\mathrm{Bias}}
\newcommand*{\Z}{\mathbb{Z}}
\newcommand*{\Q}{\mathbb{Q}}
\newcommand*{\R}{\mathbb{R}}
\newcommand*{\C}{\mathbb{C}}
\newcommand*{\N}{\mathbb{N}}
\newcommand*{\F}{\mathbb{F}}
\newcommand*{\prob}{\mathds{P}}
\newcommand*{\E}{\mathds{E}}

\begin{document}

\maketitle

\pagebreak

\begin{homeworkProblem}
  Show that the following polynomials are irreducible over the field $F$ indicated.
  \begin{enumerate}[(a)]
      \item $x^2 + 7$ over $F = \R$.
      \begin{proof}
        Note that $x^2 + 7$ is of degree two, so it suffice to show that $x^2 + 7$ has no roots in $\R$, by Lemma 8.7. Replace $x$ with any real number $a$. Since $a^2$ is non-negative, $a^2 + 7$ must not be $0$, and thus it is irreducible.
      \end{proof}
      \item $x^3 - 3x + 3$ over $F = \Q$.
      \begin{proof}
        Notice that $3$ divides the coefficients of every term other than that of the greatest one and $3^2$ does not divide $3$, and thus the result follows from the Eisenstein's Criteria.
      \end{proof}
      \item $x^2 + x + 1$ over $F = \Z_2$.
      \begin{proof}
        By Lemma 8.7, since $x^2 + x + 1 = 1$ no matter what $x$ is, it is irreducible.
      \end{proof}
      \item $x^2 + 1$ over $F = \Z_{19}$.
      \begin{proof}
        By Lemma 8.7, it suffice to show that $-1$ is not a square in $\Z_{19}$. Since $(-a)^2 = a^2$, we only need to consider $a = 0, 1, \ldots, 9$. Hence,
        \begin{gather*}
          0^2 = 0, \quad 1^2 = 1, \quad 2^2 = 4, \quad 3^2 = 9, \quad 4^2 = 16 = -3 \\
          5^2 = 25 = 6, \quad 7^2 = 49 = -8, \quad 8^2 = 64 = 7, \quad 9^2 = 81 = 5,
        \end{gather*}
        and thus $-1$ is not a square in $\Z_{19}$.
      \end{proof}
      \item $x^3 - 9$ over $F = \Z_{13}$.
      \begin{proof}
        By Lemma 8.7, it suffice to show that $9$ is not a cube in $\Z_{13}$. Hence,
        \begin{gather*}
          0^3 = 0, \quad 1^3 = 1, \quad 2^3 = 8, \quad 3^3 = 1, \quad 4^3 = -1 \\
          5^3 = -5, \quad 7^3 = 5, \quad 8^3 = 5, \quad 9^3 = 1, \quad 10^3 = -1 \\
          11^3 = 5, \quad 12^2 = -1,
        \end{gather*}
        and thus $9$ is not a cube in $\Z_{13}$.
      \end{proof}
      \item $x^4 + 2x^2 + 2$ over $F = \Q$.
      \begin{proof}
        Notice that $2$ divides the coefficients of every term other than that of the greatest one and $2^2$ does not divide $2$, and thus the result follows from the Eisenstein's Criteria.
      \end{proof}
  \end{enumerate}
\end{homeworkProblem}

\newpage

\begin{homeworkProblem}
  Let $\R$ be the field of real numbers and $\C$ that of complex numbers. Show that $\R[x]/(x^2 + 1) \simeq \C$.

  \begin{proof}
    Since there exists a natural inclusion $\R \rightarrow \C$, there exists a unique ring homomorphism $\phi: \R[x] \rightarrow \C$ that sends $a_nx^n + a_{n - 1}x^{n - 1} + \cdots + a_0$ to $a_ni^n + a_{n - 1}i^{n - 1} + \cdots + a_0$, by the universal property of polynomial rings. $\phi$ is clearly surjective, as there exists $a + bx \in \R[x]$ that gets mapped to $a + bi$, for $a + bi \in \C$. Note that $\R[x]$ is an Euclidean doamin and thus a PID, so Ker $\phi$ is generated by some $f \in \R[x]$. Since $\phi(x^2 + 1) = 0$, $x^2 + 1$ is in the kernel. However, $x^2 + 1$ is irreducible in $\R[x]$, so Ker $\phi = \langle x^2 + 1 \rangle$. The result now follows from the First Isomorphism Theorem for rings.
  \end{proof}
\end{homeworkProblem}

\newpage

\begin{homeworkProblem}
  Let $F = \Z_{11}$, the integers mod 11.
  \begin{enumerate}[(a)]
      \item Let $p(x) = x^2 + 1$; show that $p(x)$ is irreducible in $F[x]$ and that $F[x]/(p(x))$ is a field having $121$ elements.
      \begin{proof}
        To show $p(x)$ is irreducible, we only need to show $-1$ is not a square in $\Z_{11}$, by Lemma 8.7. Since $(-a)^2 = a^2$, we only need to consider $a = 0, 1, \ldots, 5$. Therefore,
        \begin{gather*}
          0^2 = 0, \quad 1^2 = 1, \quad 2^2 = 4, \quad 3^2 = 9, \quad 4^2 = 16 = 5, \quad 5^2 = 25 = 3,
        \end{gather*}
        so $p(x)$ is indeed irreducible. Note that $\Z[i]/\langle 11 \rangle$ is a field of $121$ elements, proven in Midterm 1 challenge problem 2. Since there is a natural inclusion $F \hookrightarrow \Z[i]/\langle 11 \rangle$, the universal property of polynomial rings gives us a unique ring homomorphism $\phi \colon F[x] \to \Z[i]/\langle 11 \rangle$ that sends $x$ to $i$. Let $I$ be the kernel of $\phi$. Since $F[x]$ is an Euclidean domain and thus a PID, $I = \langle a \rangle$, for some noninvertible $a \in F[x]$. We already know $x^2 + 1 \in I$. However, since $x^2 + 1$ is irreducible, $x^2 + 1$ and $a$ are associates, and thus $I = \langle x^2 + 1 \rangle$. By the First Isomorphism Theorem for rings, $F[x]/\langle x^2 + 1 \rangle \simeq \Z[i]/\langle 11 \rangle$, and this completes the proof.
      \end{proof}
      \item Let $p(x) = x^3 + x + 4 \in F[x]$; show that $p(x)$ is irreducible in $F[x]$ and that $F[x]/(p(x))$ is a field having $11^3$ elements.
      \begin{proof}
        To show $p(x)$ is irreducible, we only need to show $p(x)$ does not have a root in $\Z_{11}$. Therefore,
        \begin{gather*}
          0^3 + 0 + 4 = 4, \quad 1^3 + 1 + 4 = 6, \quad 2^3 + 2 + 4 = 3, \quad 3^3 + 3 + 4 = 1 \\
          4^3 + 4 + 4 = 6, \quad 5^3 + 5 + 4 = 2, \quad 6^3 + 6 + 4 = 6, \quad 7^3 + 7 + 4 = 2 \\
          5^3 + 5 + 4 = 7, \quad 9^3 + 9 + 4 = 5, \quad 10^3 + 10 + 4 = 2,
        \end{gather*}
        so $p(x)$ is indeed irreducible. Note that since $F$ is a field, $F[x]$ is an Euclidean domain and thus a PID. Hence, for any ideal $I = \langle k \rangle$ such that $I \neq F[x]$ and contains $\langle x^3 + x + 4 \rangle$, we have $I = \langle x^3 + x + 4 \rangle$, as $x^3 + x + 4$ is irreducible. It follows that $\langle x^3 + x + 4 \rangle$ is maximal, and thus $F[x]/\langle x^3 + x + 4 \rangle$ is a field. It remains to show that $F[x]/\langle x^3 + x + 4 \rangle$ contains $11^3$ elements. Note that any $f(x) \in F[x]$ can be written in the unique form of $g(x)(x^3 + x + 4) + (ax^2 + bx + c)$, and thus $f(x) + \langle x^3 + x + 4 \rangle = (ax^2 + bx + c) + \langle x^3 + x + 4 \rangle$, for some $a, b, c \in F[x]$. Since there are $11^3$ possible sequence of $a, b, c$, $F[x]/(p(x))$ has $11^3$ elements.
      \end{proof}
  \end{enumerate}
\end{homeworkProblem}

\newpage

\begin{homeworkProblem}
  Construct a field having $p^2$ elements, for $p$ an odd prime.

  \begin{proof}
    Consider the field $\F_p[x]/\langle g(x) \rangle$, for some irreducible quadratic $g(x) \in \F_p[x]$. We first show that such $g(x)$ exists. Suppose that a monic quadratic $f(x) = x^2 + ax + b \in \F_p[x]$ is reducible. Then $f(x) = (x + m)(x + n)$, so we need to solve for $\begin{cases} m + n = a \\
      mn = b \end{cases}$. However, since $\F_p$ is a field, there exists a unique solution to $m, n$. This means that there is a bijection between the reducible monic quadratics in $\F_p[x]$ and the unordered pairs of elements in $\F_p$. Since there are ${p \choose 2} + p$ possiblilities of unordered pairs in $\F_p$, there are ${p \choose 2} + p$ reducible monic quadratics, and thus the number of irreducible monic quadratics is $p^2 - {p \choose 2} + p > 0$. Therefore, there exists an irreducible monic quadratic $g(x) \in \F_p[x]$. Note that $\F_p$ is a field, so $\F_p[x]$ is an Euclidean domain and thus a PID. Hence, $\langle g(x) \rangle$ is maximal, as $g(x)$ is irreducible, so $\F_p[x]/\langle g(x) \rangle$ is indeed a field. It remains to show that $\F_p[x]/\langle g(x) \rangle$ contains $p^2$ elements. Since $\F_p[x]$ is an Euclidean domain, any polynomial $k(x)$ in $\F_p[x]$ can be written in the unique form of $k(x) = h(x)g(x) + (\alpha x + \beta)$, as $g(x)$ is of degree two.
    Since, $k(x) + \langle g(x) \rangle = (\alpha x + \beta) + \langle g(x) \rangle$, the left cosets of $\langle g(x) \rangle$ are characterized by the remainders of polynomials in $\F_p[x]$ after divided by $g(x)$, and there are $p^2$ of them. Hence, we conclude that $\F_p[x]/\langle g(x) \rangle$ has $p^2$ elements.
  \end{proof}
\end{homeworkProblem}

\newpage

\begin{homeworkProblem}
  In Example 5, show that because $g(x)$ is irreducible in $\Q[x]$, then so is $f(x)$. 

  \begin{proof}
    Since 5 divides coefficients of all terms except for that of the largest one in $g(x) = x^4 + 5x^3 + 10x^2 + 10x + 5$ and $25$ also does not divide 5, it meets the Eisenstein Criteria and thus $g(x)$ is irreducible in $\Q[x]$. Note that the map $\Q[x] \to \Q[x]$ that sends $h(x)$ to $h(x + 1)$ is an one-to-one correspondence. Suppose for contradiction that $f(x) = w(x)u(x)$, for some nonconstant $w(x), u(x) \in \Q[x]$. Then, $w(x + 1)u(x + 1) = g(x)$, contradiction. Hence, $f(x)$ is also irreducible. 
  \end{proof}
\end{homeworkProblem}

\newpage

\begin{homeworkProblem}
  Prove that $f(x) = x^3 + 3x + 2$ is irreducible in $\Q[x]$.

  \begin{proof}
    By Gauss' Lemma, it suffices to show that $f(x)$ has not roots in $\Z[x]$. Since $x^3 + 3x + 2 = x(x^2 + 3) + 2$, we need to show that $x(x^2 + 3) \neq -2$ for any $x \in \Z$. Suppose that it is false. We know $-2$ can be factorized into $-1 \cdot 2$ or $-2 \cdot 1$, so $x$ is either $-1$ or $-2$, as $x^2 + 3 > 0$. However, $x^2 + 3 > 2$ for $x = 1, 2$, contradiction. Hence, $x(x^2 + 3) \neq -2$, so $f(x)$ is irreducible in $\Z[x]$.
  \end{proof}
\end{homeworkProblem}

\newpage

\begin{homeworkProblem}
  Show that there is an infinite number of integers $a$ such that $f(x) = x^7 + 15x^2 - 30x + a$ is irreducible in $\Q[x]$. What $a$'s do you suggest?

  \begin{proof}
    By Eisenstein's Criteria, $f(x)$ is irreducible in $\Q[x]$ if there is a prime $p$ that divides 15, $-30$, and $a$, but not 1 and $p^2$ does not divide $a^2$. We show that any integers in $S = (\langle 3 \rangle\backslash \langle 9 \rangle) \cup (\langle 5 \rangle \backslash \langle 25 \rangle)$ suffices to be $a$. Suppose $a \in S$. $a$ is a multiple of $3$ or $5$. If $a$ is a multiple of $3$, we may pick $p = 3$ and $f(x)$ would meet Eisenstein's Criteria, as $9 \nmid a$. Otherwise, we may pick $p = 5$  and $f(x)$ would also meet Eisenstein's Criteria, as $25 \nmid a$.
  \end{proof}
\end{homeworkProblem}

\newpage

\begin{homeworkProblem}
  Let $F$ be the field and $\varphi$ an automorphism of $F[x]$ such that $\varphi(a) = a$ for every $a \in F$. If $f(x) \in F[x]$, prove that $f(x)$ is irreducible in $F[x]$ if and only if $g(x) = \varphi(f(x))$ is.

  \begin{proof}
    Suppose that $f(x)$ is irreducible in $F[x]$. $f(x) \neq k(x)h(x)$, for any noninvertible $k(x), h(x) \in F[x]$. Since $\varphi$ is an automorphism, $g(x) = \varphi(f(x)) \neq \varphi(k(x)h(x)) = \varphi(k(x))\varphi(h(x))$, and thus $g(x)$ is irreducible. This also applies for $\varphi^{-1}$, and thus the converse is also true.
  \end{proof}
\end{homeworkProblem}

\newpage

\begin{homeworkProblem}
  Let $F$ be a field. Define the mapping
  \[
    \varphi : F[x] \rightarrow F[x] \quad \text{by} \quad \varphi(f(x)) = f(x + 1)
  \]
  for every $f(x) \in F[x]$. Prove that $\varphi$ is an automorphism of $F[x]$ such that $\varphi(a) = a$ for every $a \in F$.

  \begin{proof}
    Let $f(x), g(x) \in F[x]$. Suppose that $f(x) = g(x)$. Then, $\varphi(f(x)) = f(x + 1) = g(x + 1) = \varphi(g(x))$, so $\varphi$ is well defined. Since there exists $f(x - 1) \in F[x]$ such that $\varphi(f(x - 1)) = f(x)$, $\varphi$ is surjective. Let $f(x)$ be in the kernel of $\varphi$. Then, $\varphi(f(x)) = f(x + 1) = 0$, so $f(x) = 0$. Hence, the kernel is trivial, and thus $\varphi$ is injective. Since the constant polynomials do not depend on $x$, $\varphi(a) = a$, for all $a \in F$. Since $\varphi(f(x)g(x)) = f(x + 1)g(x + 1) = \varphi(f(x))\varphi(g(x))$, $\varphi(f(x) + g(x)) = f(x + 1) + g(x + 1) = \varphi(f(x)) + \varphi(g(x))$ and $\varphi(1) = 1$, $\varphi$ is an automorphism.
  \end{proof}
\end{homeworkProblem}

\newpage

\begin{homeworkProblem}
  Let $F$ be a field and $b \neq 0$ an element of $F$. Define the mapping
  \[
    \varphi : F[x] \rightarrow F[x] \quad \text{by} \quad \varphi(f(x)) = f(bx) \quad \text{for every} \quad f(x) \in F[x].
  \]
  Prove that $\varphi$ is an automorphism of $F[x]$ such that $\varphi(a) = a$ for every $a \in F$.

  \begin{proof}
    Let $f(x), g(x) \in F[x]$. Suppose that $f(x) = g(x)$. Then, $\varphi(f(x)) = f(bx) = g(bx) = \varphi(g(x))$, so $\varphi$ is well defined. Since $F$ is a field, there exists $f(b^{-1}x) \in F$ such that $\varphi(f(b^{-1}x)) = f(x)$, and thus $\varphi$ is surjective. Let $f(x)$ be in the kernel of $\varphi$. Then, $\varphi(f(x)) = f(bx) = 0$, so $f(x) = 0$. Hence, the kernel is trivial, and thus $\varphi$ is injective. Since the constant polynomials do not depend on $x$, $\varphi(a) = a$, for all $a \in F$. Since $\varphi(f(x)g(x)) = f(bx)g(bx) = \varphi(f(x))\varphi(g(x))$, $\varphi(f(x) + g(x)) = f(bx) + g(bx) = \varphi(f(x)) + \varphi(g(x))$ and $\varphi(1) = 1$, $\varphi$ is an automorphism.
  \end{proof}
\end{homeworkProblem}

\newpage

\begin{homeworkProblem}
  Let $F$ be a field, $b \neq 0$, $c$ elements of $F$. Define the mapping
  \[
    \varphi : F[x] \rightarrow F[x] \text{ by } \varphi(f(x)) = f(bx + c) \text{ for every } f(x) \in F[x].
  \]
  Prove that $\varphi$ is an automorphism of $F[x]$ such that $\varphi(a) = a$ for every $a \in F$.

  \begin{proof}
    Define the mapping
    \[
      \phi : F[x] \rightarrow F[x] \quad \text{by} \quad \phi(f(x)) = f(bx) \quad \text{for every} \quad f(x) \in F[x].
    \]
    By Problem 10, we already know $\phi$ is an automorphism of $F[x]$ such that $\phi(a) = a$ for every $a \in F$.

    Define the mapping
    \[
      \psi : F[x] \rightarrow F[x] \quad \text{by} \quad \psi(f(x)) = f(x + 1)
    \]
    for every $f(x) \in F[x]$. By Problem 9, we already know $\psi$ is an automorphism of $F[x]$ such that $\psi(a) = a$ for every $a \in F$.

    Since $$\underbrace{\psi \circ \dots \circ \psi}_{c \text{ times}} \circ \phi(f(x)) = \underbrace{\psi \circ \dots \circ \psi}_{c \text{ times}}(f(bx)) = \underbrace{\psi \circ \dots \circ \psi}_{c - 1 \text{ times}}(f(bx + 1)) = f(bx + c) = \varphi(f(x)),$$ $\varphi$ is an automorphism of $F[x]$ such that $\varphi(a) = a$ for every $a \in F$.
  \end{proof}
\end{homeworkProblem}

\newpage

\begin{homeworkProblem}
  Let $\varphi$ be an automorphism of $F[x]$, where $F$ is a field, such that $\varphi(a) = a$ for every $a \in F$. Prove that if $f(x) \in F[x]$, then $\deg \varphi(f(x)) = \deg f(x)$.

  \begin{proof}
    Since $\varphi(a) = a$ for every $a \in F$, $\varphi$ is the unique ring homomorphism corresponding to the natural inclusion $F \hookrightarrow F[x]$, by the universal property of polynomial rings. Since $\varphi$ maps $$f(x) = a_nx^n + a_{n - 1}x^{n - 1} + \cdots + a_0$$ to
    $$a_n(\varphi(x))^n + a_{n - 1}(\varphi(x))^{n - 1} + \cdots + a_0,$$ it suffices to show that $\varphi(x)$ is of degree 1, as $\deg \varphi(f(x)) = \deg \varphi(x) \deg f(x)$. $\deg \varphi(x)$ cannot be $0$, otherwise $\varphi(f(x)) \in F$, then $\varphi$ is not surjective and thus not an automorphism. Suppose for the sake of contradiction that $\deg \varphi(x) > 1$. Then, for non-constant $f(x) \in F[x]$, $\deg \varphi(f(x)) = \deg \varphi(x) \deg f(x) > \deg f(x) \geq 1$, which implies that the image of the automorphism $\phi$ does not contain polynomials of degree 1, contradiction. 
  \end{proof}
\end{homeworkProblem}

\newpage

\begin{homeworkProblem}
  Let $\varphi$ be an automorphism of $F[x]$, where $F$ is a field, such that $\varphi(a) = a$ for every $a \in F$. Prove there exist $b \neq 0$, $c$ in $F$ such that $\varphi(f(x)) = f(bx + c)$ for every $f(x) \in F[x]$.

  \begin{proof}
    Since $\varphi(a) = a$ for every $a \in F$, $\varphi$ is the unique ring homomorphism corresponding to the natural inclusion $F \hookrightarrow F[x]$, by the universal property of polynomial rings. By problem 12, we know $\varphi(x)$ is a polynomial of degree 1 if $\varphi$ is an automorphism, say $bx + c$. Then,
    \begin{align*}
      \varphi(f(x)) 
      &= \varphi(a_nx^n + a_{n - 1}x^{n - 1} + \cdots + a_0) \\
      &= a_n(\varphi(x))^n + a_{n - 1}(\varphi(x))^{n - 1} + \cdots + a_0 \\
      &= f(\varphi(x)) = f(bx + c),
    \end{align*}
    and we are done.
  \end{proof}
\end{homeworkProblem}

\newpage

\begin{homeworkProblem}
  Find a nonidentity automorphism $\varphi$ of $\Q[x]$ such that $\varphi^2$ is the identity automorphism of $\Q[x]$.

  \begin{proof}
    From the natural inclusion $\Q \hookrightarrow \Q[x]$, the universal property gives us a ring homomorphism $\phi$ such that $\phi(x) = -x$. $\phi$ is obviously injective, as its kernel is trivial. For $p(x) \in \Q[x]$, there exists $p(-x) \in \Q[x]$ such that $\phi(p(-x)) = p(x)$, so $\phi$ is surjective. It follows that $\phi$ is a bijection and thus an automorphism. Now consider $\phi^2$. $\phi^2(p(x)) = \phi(\phi(p(x))) = \phi(p(-x)) = p(x)$, and thus $\phi^2$ is the identity automorphism.
  \end{proof}
\end{homeworkProblem}

\newpage

\begin{homeworkProblem}
  Show that in Problem 14 you do not need the assumption $\varphi(a) = a$ for every $a \in \Q$ because any automorphism of $\Q[x]$ automatically satisfies $\varphi(a) = a$ for every $a \in \Q$.

  \begin{proof}
    Let $\phi$ be a ring automorphism of $\Q[x]$. We know $\phi(0) = 0$ and $\phi(1) = 1$. Since $\phi(x) + \phi(y) = \phi(x + y)$, we may prove by induction that $f(z) = f(z - 1) + f(1) = zf(1) = z$, for all $z \in \Z$. Suppose that there exists $\frac{p}{q} \in \Q$ such that $\phi(\frac{p}{q}) \neq \frac{p}{q}$, $p, q \in \Z$. Then, $p = \phi(p) = \phi(q)\phi(\frac{p}{q}) \neq q \cdot \frac{p}{q}$, contradiction. Hence, $\phi(a) = a$, for all $a \in \Q$.
  \end{proof}
\end{homeworkProblem}

\newpage

\begin{homeworkProblem}
  Let $\C$ be the field of complex numbers. Given an integer $n > 0$, exhibit an automorphism $\varphi$ of $\C[x]$ of order $n$.

  \begin{proof}
    Consider $\varphi(f(x)) = f(e^{\frac{2i\pi}{n}}x)$. By problem 11, $\varphi$ is an automorphism. Since
    \[
      \varphi^n(f(x)) = \underbrace{\varphi \circ \cdots \circ \varphi}_{n \text{ times}}(f(x)) = \varphi\left(\left(\prod^n e^{\frac{2i\pi}{n}}\right)x\right) = \varphi(e^{2i\pi}x) = \varphi(x),
    \]
    $\varphi$ is of order $n$.
  \end{proof}
\end{homeworkProblem}

\newpage

\begin{homeworkProblem}
  Given a ring $R$, let $S = R[x]$ be the ring of polynomials in $x$ over $R$, and let $T = S[y]$ be the ring of polynomials in $y$ over $S$. Show that:
  \begin{enumerate}[(a)]
      \item Any element $f(x, y)$ in $T$ has the form $\sum a_{ij}x^iy^j$, where the $a_{ij}$ are in $R$.
      \begin{proof}
        Let $f(x, y) \in T$. Since $f(x, y) \in S[y]$, $f(x, y) = \sum_j p(x)y^j = \sum_j \left(\sum_i a'_{ij}x^i\right)y^j = \sum a_{ij}x^iy^j$, for $p(x) \in S$ and $a_{ij}, a'_{ij} \in R$.
      \end{proof}
      \item In terms of the form of $f(x, y)$ in $T$ given in Part (a), give the condition for the equality of two elements $f(x, y)$ and $g(x, y)$ in $T$.
      \begin{proof}
        $f(x, y) = \sum f_{ij}x^iy^j = \sum g_{ij}x^iy^j = g(x, y)$, if and only if $f_{ij} = g_{ij}$, for all $i, j$.
      \end{proof}
      \item In terms of the form for $f(x, y)$ in Part (a), give the formula for $f(x, y) + g(x, y)$, for $f(x, y), g(x, y)$ in $T$.
      \begin{proof}
        $f(x, y) + g(x, y) = \sum f_{ij}x^iy^j + \sum g_{ij}x^iy^j = \sum (h_{ij} + g_{ij})x^iy^j$.
      \end{proof}
      \item Give the form for the product of $f(x, y)$ and $g(x, y)$ if $f(x, y)$ and $g(x, y)$ are in $T$. ($T$ is called the ring of polynomials in two variables over $R$, and is denoted by $R[x, y]$).
      \begin{proof}
        \[
          f(x, y)g(x, y) = \left(\sum f_{ij}x^iy^j\right)\left(\sum g_{ij}x^iy^j\right) = \sum
        \left(\sum_{m + n = i, p + q = j} f_{mp}g_{nq} \right)x^iy^j.
        \]
        Since the product is of the form of $\sum a_{ij}x^iy^j$, it is in $T$.
      \end{proof}
  \end{enumerate}
\end{homeworkProblem}

\newpage

\begin{homeworkProblem}
  If $D$ is an integral domain, show that $D[x, y]$ is an integral domain.

  \begin{proof}
    Since $D$ is a commutative ring, $D[x][y] \simeq D[x, y]$, by Lemma 9.12. By Lemma 7.4, $D[x]$ is also an integral domain, and thus $D[x][y]$ is also an integral domain.
  \end{proof}
\end{homeworkProblem}
\end{document}