\documentclass{article}

\usepackage{fancyhdr}
\usepackage{extramarks}
\usepackage{amsmath}
\usepackage{amsthm}
\usepackage{amsfonts}
\usepackage{tikz}
\usepackage[plain]{algorithm}
\usepackage{algpseudocode}
\usepackage{enumerate}
\usepackage{amssymb}

\usetikzlibrary{automata,positioning}

%
% Basic Document Settings
%

\topmargin=-0.45in
\evensidemargin=0in
\oddsidemargin=0in
\textwidth=6.5in
\textheight=9.0in
\headsep=0.25in

\linespread{1.1}

\pagestyle{fancy}
\lhead{\hmwkAuthorName}
\chead{\hmwkClass:\ \hmwkTitle}
\rhead{\firstxmark}
\lfoot{\lastxmark}
\cfoot{\thepage}

\renewcommand\headrulewidth{0.4pt}
\renewcommand\footrulewidth{0.4pt}

\setlength\parindent{0pt}
\setlength{\parskip}{5pt}

%
% Create Problem Sections
%

\newcommand{\enterProblemHeader}[1]{
    \nobreak\extramarks{}{Problem \arabic{#1} continued on next page\ldots}\nobreak{}
    \nobreak\extramarks{Problem \arabic{#1} (continued)}{Problem \arabic{#1} continued on next page\ldots}\nobreak{}
}

\newcommand{\exitProblemHeader}[1]{
    \nobreak\extramarks{Problem \arabic{#1} (continued)}{Problem \arabic{#1} continued on next page\ldots}\nobreak{}
    \stepcounter{#1}
    \nobreak\extramarks{Problem \arabic{#1}}{}\nobreak{}
}

\setcounter{secnumdepth}{0}
\newcounter{partCounter}
\newcounter{homeworkProblemCounter}
\setcounter{homeworkProblemCounter}{1}
\nobreak\extramarks{Problem \arabic{homeworkProblemCounter}}{}\nobreak{}

%
% Homework Problem Environment
%
% This environment takes an optional argument. When given, it will adjust the
% problem counter. This is useful for when the problems given for your
% assignment aren't sequential. See the last 3 problems of this template for an
% example.
%
\newenvironment{homeworkProblem}[1][-1]{
    \ifnum#1>0
        \setcounter{homeworkProblemCounter}{#1}
    \fi
    \section{Problem \arabic{homeworkProblemCounter}}
    \setcounter{partCounter}{1}
    \enterProblemHeader{homeworkProblemCounter}
}{
    \exitProblemHeader{homeworkProblemCounter}
}

%
% Homework Details
%   - Title
%   - Due date
%   - Class
%   - Section/Time
%   - Instructor
%   - Author
%

\newcommand{\hmwkTitle}{Homework\ \#8}
\newcommand{\hmwkDueDate}{Mar 7, 2024}
\newcommand{\hmwkClass}{MATH 100B}
\newcommand{\hmwkClassTime}{Section A02 6:00PM - 6:50PM}
\newcommand{\hmwkSectionLeader}{Castellano-Macías}
\newcommand{\hmwkClassInstructor}{Professor McKernan}
\newcommand{\hmwkSource}{Source Consulted: Textbook, Lecture, Discussion, Office Hour}
\newcommand{\hmwkAuthorName}{\textbf{Ray Tsai}}
\newcommand{\hmwkPID}{A16848188}

%
% Title Page
%

\title{
    \vspace{2in}
    \textmd{\textbf{\hmwkClass:\ \hmwkTitle}}\\
    \normalsize\vspace{0.1in}\small{Due\ on\ \hmwkDueDate\ at 12:00pm}\\
    \vspace{0.1in}\large{\textit{\hmwkClassInstructor}} \\
    \vspace{0.1in}\small\hmwkClassTime \\
    \small Section Leader: \hmwkSectionLeader \\
    \vspace{0.1in}\small\hmwkSource \\
    \vspace{3in}
}

\author{
  \hmwkAuthorName \\
  \vspace{0.1in}\small\hmwkPID
}
\date{}

\renewcommand{\part}[1]{\textbf{\large Part \Alph{partCounter}}\stepcounter{partCounter}\\}

%
% Various Helper Commands
%

% Useful for algorithms
\newcommand{\alg}[1]{\textsc{\bfseries \footnotesize #1}}

% For derivatives
\newcommand{\deriv}[1]{\frac{\mathrm{d}}{\mathrm{d}x} (#1)}

% For partial derivatives
\newcommand{\pderiv}[2]{\frac{\partial}{\partial #1} (#2)}

% Integral dx
\newcommand{\dx}{\mathrm{d}x}

% Probability commands: Expectation, Variance, Covariance, Bias
\newcommand{\Var}{\mathrm{Var}}
\newcommand{\Cov}{\mathrm{Cov}}
\newcommand{\Bias}{\mathrm{Bias}}
\newcommand*{\Z}{\mathbb{Z}}
\newcommand*{\Q}{\mathbb{Q}}
\newcommand*{\R}{\mathbb{R}}
\newcommand*{\C}{\mathbb{C}}
\newcommand*{\N}{\mathbb{N}}
\newcommand*{\F}{\mathbb{F}}
\newcommand*{\prob}{\mathds{P}}
\newcommand*{\E}{\mathds{E}}

\begin{document}

\maketitle

\pagebreak

\begin{homeworkProblem}
  Let $M$, $N$ and $P$ be $R$-modules and let $F$ be a free $R$-module of rank $n$. Show that there are isomorphisms, which are all natural (except for the last):
  \begin{enumerate}[(a)]
    \item $M \underset{R}{\otimes} N \simeq N \underset{R}{\otimes} M$.
    \begin{proof}
      Let $v: N \times M \to N \underset{R}{\otimes} M$ be the bilinear map associated with $N \underset{R}{\otimes} M$. Define $f: M \times N \to N \underset{R}{\otimes} M$ that sends $(m, n)$ to $v(n, m)$. By the universal property of tensor product, there is an induced module homomorphism $\phi: M \underset{R}{\otimes} N \to N \underset{R}{\otimes} M$. Similarly, there exsits an induced module homomorphism $\psi: N \underset{R}{\otimes} M \to M \underset{R}{\otimes} N$. By the universal property of tensor product, Hom($M \oplus N$, $M \oplus N$) and Hom($N \oplus M$, $N \oplus M$) only contain the identities. But then $\phi \circ \psi \in$ Hom($M \oplus N$, $M \oplus N$) and $\psi \circ \phi \in$ Hom($N \oplus M$, $N \oplus M$), so $\phi$ and $\psi$ are inverses. It follows that $\phi$ is a module isomorphism, so $M \underset{R}{\otimes} N \simeq N \underset{R}{\otimes} M$.
    \end{proof}
    \item $(M \underset{R}{\otimes} N)\underset{R}{\otimes} P \simeq M \underset{R}{\otimes} (N\underset{R}{\otimes}P)$.
    \begin{proof}
      For $m \in M$, define $\psi_m^B: N \times P \to (M \underset{R}{\otimes} N) \underset{R}{\otimes} P$, which sends $(n, p)$ to $(m \otimes n) \otimes p$. Note that $\psi_m^B$ is obvisouly bilinear and well-defined, and thus the universal property gives us a linear mapping $\psi_m: N \otimes P \to (M \underset{R}{\otimes} N) \otimes P$. We now define $\phi^B: M \times (N \otimes P) \to (M \underset{R}{\otimes} N) \otimes P$, which sends $(m, n \otimes p)$ to $\psi_m(n, p)$. We check that $\phi^B$ is bilinear. Let $m, m' \in M$, $r \in R$, and $v, v' \in N \underset{R}{\otimes} P$, say $v = \sum a_{ij} n_i \otimes p_j$ and $v' = \sum b_{ij} n_i \otimes p_j$. Since
      \begin{align*}
        \phi^B(m + m', v) 
        &= \psi_{m + m'}(v) \\
        &= \sum a_{ij} \psi_{m + m'}(n_i \otimes p_j) \\
        &= \sum a_{ij} ((m + m') \otimes n_i) \otimes p_j \\
        &= \sum a_{ij} \psi_{m}(n_i \otimes p_j) + \sum a_{ij} \psi_{m'}(n_i \otimes p_j) \\
        &= \psi_{m}(v) + \psi_{m'}(v) = \phi^B(m, v) + \phi^B(m', v),
      \end{align*}
      \begin{align*}
        \phi^B(m, v + v') 
        &= \psi_{m}(v + v') \\
        &= \sum (a_{ij} + b_{ij}) \psi_{m}(n_i \otimes p_j) \\
        &= \sum a_{ij} (m \otimes n_i) \otimes p_j + \sum b_{ij} (m \otimes n_i) \otimes p_j \\
        &= \sum a_{ij} \psi_{m}(n_i \otimes p_j) + \sum b_{ij} \psi_{m}(n_i \otimes p_j) \\
        &= \psi_{m}(v) + \psi_{m}(v') = \phi^B(m, v) + \phi^B(m, v'),
      \end{align*}
      \begin{align*}
        \phi^B(rm, v) 
        &= \psi_{rm}(v) \\
        &= \sum a_{ij} \psi_{rm}(n_i \otimes p_j) \\
        &= r\sum a_{ij} \psi_{m}(n_i \otimes p_j) \\
        &= r\psi_{m}(v) = r\phi^B(m, v),
      \end{align*}
      \begin{align*}
        \phi^B(m, rv) = \psi_{m}(rv) = r\psi_{m}(v) = r\phi^B(m, v),
      \end{align*}
      $\phi^B$ is indeed bilinear, so we obtain a linear $\phi: M \underset{R}{\otimes} (N \underset{R}{\otimes} P) \to (M \underset{R}{\otimes} N) \underset{R}{\otimes} P$, by the universal property. We may repeat the above process to obtain an induced linear map $\varphi: (M \underset{R}{\otimes} N) \underset{R}{\otimes} P \to M \underset{R}{\otimes} (N \underset{R}{\otimes} P)$, and thus $\phi$ and $\varphi$ are inveres of each other, by the standard uniqueness argument. The result now follows.
    \end{proof}
    \item $R \underset{R}{\otimes} M \simeq M$.
    \begin{proof}
      Define mapping $f: R \times M \to M$ that sends $(r, m)$ to $rm$. Note that $f$ is obviously bilinear. The universal property of tensor product gives us a $R$-linear mapping $\phi: R \underset{R}{\otimes} M \to M$ which sends $r \otimes m$ to $f(rm)$, that is, $rm$. Since for all $m \in M$, we have $1 \otimes m \in R \underset{R}{\otimes} M$ that is mapped to $m$ via $\phi$, so $\phi$ is surjective. Suppose $r \otimes m$ is in the kernel of $\phi$. Then $\phi(r \otimes m) = rm = 0$, so $r = 0$ or $m = 0$. But then $r \otimes m = 0$ in either case, and thus the kernel of $\phi$ is trivial. The result now follows from the first isomorphism theorem.
    \end{proof}
    \item $M \underset{R}{\otimes} (N \oplus P) \simeq (M \underset{R}{\otimes} N) \oplus (M \underset{R}{\otimes}P)$.
    \begin{proof}
      Define mapping $f: M \times (N \oplus P) \to (M \underset{R}{\otimes} N) \oplus (M \underset{R}{\otimes}P)$, which maps $(m, (n \otimes p))$ to $(m \otimes n, m \otimes p)$. This map is obviously well defined. We show that $f$ is bilinear. Suppose $m, m' \in M$, $n, n' \in N$, $p, p' \in P$, and $r \in R$. We then have
      \begin{align*}
        f(m + m', (n, p)) 
        &= ((m + m') \otimes n, (m + m') \otimes p) \\
        &= (m \otimes n, m \otimes p) + (m' \otimes n, m' \otimes p) \\
        &= f(m, (n, p)) + f(m', (n, p)),
      \end{align*}
      \begin{align*}
        f(m, (n, p) + (n', p')) 
        &= (m \otimes (n + n'), \otimes (p + p')) \\
        &= (m \otimes n, m \otimes p) + (m \otimes n', m \otimes p') \\
        &= f(m, (n, p)) + f(m, (n', p')),
      \end{align*}
      \begin{align*}
        f(rm, (n, p))
        &= (rm \otimes n, rm \otimes p) \\
        &= (r(m \otimes n), r(m \otimes p)) \\
        &= r(m \otimes n, m \otimes p) = rf(m, (n, p)),
      \end{align*}
      \begin{align*}
        f(m, r(n, p))
        &= (m \otimes rn, m \otimes rp) \\
        &= (r(m \otimes n), r(m \otimes p)) \\
        &= r(m \otimes n, m \otimes p) = rf(m, (n, p)),
      \end{align*}
      and thus $f$ is bilinear. The universal property of tensor product now gives us an induced $R$-linear mapping
      \[
        \phi: M \underset{R}{\otimes} (N \oplus P) \to (M \underset{R}{\otimes} N) \oplus (M \underset{R}{\otimes}P),
      \]
      which maps $m \otimes (n, p)$ to $f(m, (n, p))$. 

      It remains to find the inverse of $\phi$. Define $\psi_1^B: M \times N \to M \underset{R}{\otimes} (N \oplus P)$ by sending $(m, n)$ to $m \otimes (n, 0)$, and define $\psi_2^B: M \times P \to M \underset{R}{\otimes} (N \oplus P)$ by sending $(m, p)$ to $m \otimes (0, p)$. Note that both $\psi_1^B$ and $\psi_2^B$ are bilinear, so the universal property gives us linear mappings $\psi_1: M \underset{R}{\otimes} N \to M \underset{R}{\otimes} (N \oplus P)$ and $\psi_2: M \underset{R}{\otimes} P \to M \underset{R}{\otimes} (N \oplus P)$. Now define $\psi: (M \underset{R}{\otimes} N) \oplus (M \underset{R}{\otimes}P) \to M \underset{R}{\otimes} (N \oplus P)$ by sending $((m \otimes n), (m' \otimes p))$ to $\psi_1(m \otimes n) + \psi_2(m' \otimes p)$. Note that $\psi$ is linear, as both $\psi_1$ and $\psi_2$ are linear. 

      We now show that $\phi$ and $\psi$ are inverses of each other. Suppose $v \in M \underset{R}{\otimes} N$, $w \in M \underset{R}{\otimes} P$, and $x \in M \underset{R}{\otimes} (N \oplus P)$, say $v = \sum a_{ij} m_i \otimes n_j$, $w = \sum b_{ij} m_i \otimes p_j$, and $x = \sum c_{ijk} m_i \otimes (n_j, p_k)$. Since both $\phi$ and $\psi$ are linear,
      \begin{align*}
        \phi \circ \psi(v, w) 
        &= \phi (\psi_1(v) + \psi_2(w)) \\
        &= \phi(\psi_1(v)) + \phi(\psi_2(w)) \\
        &= \sum a_{ij} \phi(\psi_1(m_i \otimes n_j)) + \sum b_{ij} \phi(\psi_1(m_i \otimes p_j)) \\
        &= \sum a_{ij} \phi(m_i \otimes (n_j, 0)) + \sum b_{ij} \phi(m_i \otimes (0, p_j)) \\
        &= \sum a_{ij} (m_i \otimes n_j, 0) + \sum b_{ij} (0, m_i \otimes p_j) = (v, 0) + (0, w) = (v, w),
      \end{align*}
      \begin{align*}
        \psi \circ \phi(x) 
        &= \psi \left(\sum c_{ijk} \phi(m_i \otimes (n_j, p_k))\right) \\
        &= \sum c_{ijk} \psi(m_i \otimes n_j, m_i \otimes p_k) \\
        &= \sum c_{ijk} (\psi_1(m_i \otimes n_j) + \psi_2(m_i \otimes p_k)) \\
        &= \sum c_{ijk} (m_i \otimes (n_j, 0) + (m_i \otimes (0, p_k))) \\
        &= \sum c_{ijk} (m_i \otimes (n_j, p_k)) = x,
      \end{align*}
      and the result follows.
    \end{proof}
    \item $F \underset{R}{\otimes} M \simeq M^n$.
    \begin{proof}
      Note that $F \simeq R^n$. We show that $R^n \underset{R}{\otimes} M \simeq M^n$ by induction on $n$. The base case follows from (c). Suppose $n > 1$. By (d), we have $R^n \underset{R}{\otimes} M \simeq (R \underset{R}{\otimes} M) \oplus (R^{n - 1} \underset{R}{\otimes} M) \simeq M \oplus (R^{n - 1} \underset{R}{\otimes} M)$. The result now follows from induction.
    \end{proof}
  \end{enumerate}
\end{homeworkProblem}

\newpage

\begin{homeworkProblem}
  Let $m$ and $n$ be integers. Identify $\Z_m \underset{\Z}{\otimes} \Z_n$.
  \begin{proof}
    Let $d = \gcd(m, n)$. We show that $\Z_m \underset{\Z}{\otimes} \Z_n \simeq \Z_d$. Let $a, b \in \Z$. We first note that $0 \otimes a = b \otimes 0 = 0$. In addition, since $(ab)(1 \otimes 1) = a \otimes b$, all elements in $\Z_m \underset{\Z}{\otimes} \Z_n$ are multiples of $1 \otimes 1$, so we have a cyclic group. 
    
    Define $f: \Z_m \times \Z_n \to \Z_d$ that sends $(a, b)$ to $ab$. Suppose $(a, b) = (a', b')$. We know $a' = a + km$ and $b' = b + ln$, for some $k, l \in \Z$. But then $d$ divides $m, n$, so $a' = a$ and $b' = b$, mod $d$. Hence, $f(a, b) = ab = a'b' = f(a', b')$, so $f$ is well-defined. Since $f$ is obviously bilinear, the universal property of tensor product gives us an induced module homomorphism 
    \[
      \phi: \Z_m \underset{\Z}{\otimes} \Z_n \to \Z_d,
    \]
    which sends $1 \otimes 1$ to $f(1, 1) = 1$. 

    Consider $\psi: \Z_d \to \Z_m \underset{\Z}{\otimes} \Z_n$, which sends $k$ to $k \otimes 1$. Suppose $k' = k + \alpha d$, for some $\alpha \in \Z$. Then,
    \[
      \psi(k') = (k + \alpha d) \otimes 1 = k(1 \otimes 1) + \alpha(d(1 \otimes 1)).
    \]
    Since $d = \gcd(m, n)$, $d = pm + qn$, for some $p, q \in \Z$. But then 
    \[
      d(1 \otimes 1) = (pm + qn)(1 \otimes 1) = p(m \otimes 1) + q(1 \otimes n) = 0,
    \]
    so $\psi(k') = k(1 \otimes 1) = \psi(k)$, and thus $\psi$ is well defined. Note that $\psi$ is obviously linear.

    Since $\phi \circ \psi(k) = \phi(k(1 \otimes 1)) = k$ and $\psi \circ \phi(a \otimes b) = \psi(ab) = (ab)(1 \otimes 1) = (a \otimes b)$, $\phi$ is an module isomorphism, and the result follows.
  \end{proof}
\end{homeworkProblem}

\newpage

\begin{homeworkProblem}
  Show that if $M$ and $N$ are two finitely generated (respectively Noetherian) $R$-modules (respectively and $R$ is Noetherian) then so is $M \underset{R}{\otimes} N$. 
  \begin{proof}
    By Proposition 11.7., it suffices to show that $M \underset{R}{\otimes} N$ is finitely generated. Suppose $m_1, m_2, \ldots, m_k$ and $n_1, n_2, \ldots, n_l$ are the generators of $M$ and $N$, respectively. Let $m \otimes n \in M \underset{R}{\otimes} N$. Since $m = \sum_i a_im_i$ and $n = \sum_j b_jn_j$, for some $a_1, a_2, \ldots, a_k$, $b_1, b_2, \ldots, b_l \in R$ , we have
    \[
      m \otimes n = \sum_i a_i(m_i \otimes n) = \sum_i\sum_j a_ib_j(m_i \otimes n_j) = \sum_{i, j} c_{ij}(m_i \otimes n_j),
      \]
      where $c_{ij} = a_ib_j$. Hence, $M \underset{R}{\otimes} N$ is generated by $m_i \otimes n_j$, for finitely many $i, j$, and the result now follows.
    \end{proof}
  \end{homeworkProblem}
  
  \end{document}