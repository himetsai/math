\documentclass{article}

\usepackage{fancyhdr}
\usepackage{extramarks}
\usepackage{amsmath}
\usepackage{amsthm}
\usepackage{amsfonts}
\usepackage{tikz}
\usepackage[plain]{algorithm}
\usepackage{algpseudocode}
\usepackage{enumitem}

\usetikzlibrary{automata,positioning}

%
% Basic Document Settings
%

\topmargin=-0.45in
\evensidemargin=0in
\oddsidemargin=0in
\textwidth=6.5in
\textheight=9.0in
\headsep=0.25in

\linespread{1.1}

\pagestyle{fancy}
\lhead{\hmwkAuthorName}
\chead{\hmwkClass:\ \hmwkTitle}
\rhead{\firstxmark}
\lfoot{\lastxmark}
\cfoot{\thepage}

\renewcommand\headrulewidth{0.4pt}
\renewcommand\footrulewidth{0.4pt}

\setlength\parindent{0pt}
\setlength{\parskip}{5pt}

%
% Create Problem Sections
%

\newcommand{\enterProblemHeader}[1]{
    \nobreak\extramarks{}{Problem \arabic{#1} continued on next page\ldots}\nobreak{}
    \nobreak\extramarks{Problem \arabic{#1} (continued)}{Problem \arabic{#1} continued on next page\ldots}\nobreak{}
}

\newcommand{\exitProblemHeader}[1]{
    \nobreak\extramarks{Problem \arabic{#1} (continued)}{Problem \arabic{#1} continued on next page\ldots}\nobreak{}
    \stepcounter{#1}
    \nobreak\extramarks{Problem \arabic{#1}}{}\nobreak{}
}

\setcounter{secnumdepth}{0}
\newcounter{partCounter}
\newcounter{homeworkProblemCounter}
\setcounter{homeworkProblemCounter}{1}
\nobreak\extramarks{Problem \arabic{homeworkProblemCounter}}{}\nobreak{}

%
% Homework Problem Environment
%
% This environment takes an optional argument. When given, it will adjust the
% problem counter. This is useful for when the problems given for your
% assignment aren't sequential. See the last 3 problems of this template for an
% example.
%
\newenvironment{homeworkProblem}[1][-1]{
    \ifnum#1>0
        \setcounter{homeworkProblemCounter}{#1}
    \fi
    \section{Problem \arabic{homeworkProblemCounter}}
    \setcounter{partCounter}{1}
    \enterProblemHeader{homeworkProblemCounter}
}{
    \exitProblemHeader{homeworkProblemCounter}
}

%
% Homework Details
%   - Title
%   - Due date
%   - Class
%   - Section/Time
%   - Instructor
%   - Author
%

\newcommand{\hmwkTitle}{Homework\ \#4}
\newcommand{\hmwkDueDate}{October 26, 2023}
\newcommand{\hmwkClass}{MATH 100A}
\newcommand{\hmwkClassTime}{Section A02 5:00PM - 5:50PM}
\newcommand{\hmwkSectionLeader}{Castellano}
\newcommand{\hmwkClassInstructor}{Professor McKernan}
\newcommand{\hmwkSource}{Source Consulted: Textbook, Lecture, Discussion}
\newcommand{\hmwkAuthorName}{\textbf{Ray Tsai}}
\newcommand{\hmwkPID}{A16848188}

%
% Title Page
%

\title{
    \vspace{2in}
    \textmd{\textbf{\hmwkClass:\ \hmwkTitle}}\\
    \normalsize\vspace{0.1in}\small{Due\ on\ \hmwkDueDate\ at 12:00pm}\\
    \vspace{0.1in}\large{\textit{\hmwkClassInstructor}} \\
    \vspace{0.1in}\small\hmwkClassTime \\
    \small Section Leader: \hmwkSectionLeader \\
    \vspace{0.1in}\small\hmwkSource \\
    \vspace{3in}
}

\author{
  \hmwkAuthorName \\
  \vspace{0.1in}\small\hmwkPID
}
\date{}

\renewcommand{\part}[1]{\textbf{\large Part \Alph{partCounter}}\stepcounter{partCounter}\\}

%
% Various Helper Commands
%

% Useful for algorithms
\newcommand{\alg}[1]{\textsc{\bfseries \footnotesize #1}}

% For derivatives
\newcommand{\deriv}[1]{\frac{\mathrm{d}}{\mathrm{d}x} (#1)}

% For partial derivatives
\newcommand{\pderiv}[2]{\frac{\partial}{\partial #1} (#2)}

% Integral dx
\newcommand{\dx}{\mathrm{d}x}

% Probability commands: Expectation, Variance, Covariance, Bias
\newcommand{\Var}{\mathrm{Var}}
\newcommand{\Cov}{\mathrm{Cov}}
\newcommand{\Bias}{\mathrm{Bias}}
\newcommand*{\Z}{\mathbb{Z}}
\newcommand*{\Q}{\mathbb{Q}}
\newcommand*{\R}{\mathbb{R}}
\newcommand*{\C}{\mathbb{C}}
\newcommand*{\N}{\mathbb{N}}
\newcommand*{\prob}{\mathds{P}}
\newcommand*{\E}{\mathds{E}}

\begin{document}

\maketitle

\pagebreak

\begin{homeworkProblem}
    Find the order of all the elements of $U_{18}$. Is $U_{18}$ cyclic?

    \begin{proof}
      We know that $18 = 2\cdot 3^2$, and so $U_{18} = \{[1], [5], [7], [11], [13], [17]\}$. Notice that
      \begin{align*}
        [5]^1 &= 5 \\
        [5]^2 &= 7 \\
        [5]^3 &= 17 \\
        [5]^4 &= 13 \\
        [5]^5 &= 11 \\
        [5]^6 &= 1.
      \end{align*}
      Thus, we know that $U_{18}$ is a cyclic group. Let $[5] = a$. We can represent each element in $U_{18}$ as $a^i$, for some $0 < i \leq \varphi(18) = 6$. Hence, for all $a^i \in U_{18}$, the order of $a^i$ is equal to the least common multiple of $i$ and $\varphi(18) = 6$ divided by $i$, namely 
      \[
        o(a^i) = \frac{\text{lcm}(i, 6)}{i} = \frac{6i}{i \cdot \gcd(i, 6)} = \frac{6}{\gcd(i, 6)}.
      \]
      Therefore,
      \begin{align*}
        o([1]) &= \frac{6}{\gcd(6, 6)} = 1 \\
        o([5]) &= \frac{6}{\gcd(1, 6)} = 6 \\
        o([7]) &= \frac{6}{\gcd(2, 6)} = 3 \\
        o([11]) &= \frac{6}{\gcd(5, 6)} = 6 \\
        o([13]) &= \frac{6}{\gcd(4, 6)} = 3 \\
        o([17]) &= \frac{6}{\gcd(3, 6)} = 2.
      \end{align*}
    \end{proof}
\end{homeworkProblem}

\newpage

\begin{homeworkProblem}
  Find the order of all the elements of $U_{20}$. Is $U_{20}$ cyclic?

  \begin{proof}
    We know that $20 = 2^2\cdot 5$, and so $U_{20} = \{[1], [3], [7], [9], [11], [13], [17], [19]\}$. Notice that
    \begin{align*}
      [3]^1 &= [3] && [7]^1 = [7] && [17]^1 = [17] && [13]^1 = [13] && [9]^1 = [9] && [11]^1 = [11] && [19]^1 = [19] \\
      [3]^2 &= [9] && [7]^2 = [9] && [17]^2 = [9] && [13]^2 = [9] && [9]^2 = [1] && [11]^2 = [1] && [19]^2 = [1] \\
      [3]^3 &= [7] && [7]^3 = [3] && [17]^3 = [13] && [13]^3 = [17] \\
      [3]^4 &= [1] && [7]^4 = [1] && [17]^4 = [1] && [13]^4 = [1]
    \end{align*}
    Thus,we have
    \begin{align*}
      o([1]) &= 1 \\
      o([3]) &= 4 \\
      o([7]) &= 4 \\
      o([9]) &= 2 \\
      o([11]) &= 2 \\
      o([13]) &= 4 \\
      o([17]) &= 4 \\
      o([19]) &= 2
    \end{align*}
    Since we cannot represent all the elements as powers of a single element in $U_{20}$, we know $U_{20}$ is not a cyclic group.
  \end{proof}
\end{homeworkProblem}

\newpage

\begin{homeworkProblem}
  If $p$ is a prime number of the form $4n + 3$, show that we cannot solve 
  \[
    x^2 \equiv -1 \mod p. 
  \]
  \begin{proof}
    Suppose for the sake of contradiction that $x^2 \equiv -1 \mod p$ for some $x$. Then, $x^4 \equiv 1 \mod p$. Since $x^2 \equiv -1 \mod p$, we know $x \not\equiv \pm 1 \mod p$, and so $x^3 = x\cdot x^2 \not\equiv 1 \mod p$. Therefore, we know that $[x]$ is of order $4$ in $U_p$. By Lagrange's Theorem, we know that $4 | \varphi(p) = p - 1 = 4n + 2$, contradiction. Therefore, we cannot solve the above equation.
  \end{proof}

  \begin{proof}[Aliter]
    We can assume that $p$ does not divide $x$, otherwise we get $x^2 \equiv 0 \mod p$. Then, we know $x^{p - 1} \equiv 1 \mod p$, by Fermat's theorem. We thus get $x^{4n + 2} = (x^2)^{2n + 1} \equiv 1 \mod p$, but $(-1)^{2n + 1} \equiv -1 \mod p$, and so the equation cannot be solved.
  \end{proof}
\end{homeworkProblem}

\newpage

\begin{homeworkProblem}
  Find the products:
  \begin{enumerate}[label=(\alph*)]
    \item $\begin{pmatrix} 1 & 2 & 3 & 4 & 5 & 6 \\
      6 & 4 & 5 & 2 & 1 & 3 \end{pmatrix}\begin{pmatrix} 1 & 2 & 3 & 4 & 5 & 6 \\
      2 & 3 & 4 & 5 & 6 & 1 \end{pmatrix}$.

    \begin{proof}
      \[
        \begin{pmatrix}
          1 & 2 & 3 & 4 & 5 & 6 \\
          6 & 4 & 5 & 2 & 1 & 3
        \end{pmatrix}\begin{pmatrix}
          1 & 2 & 3 & 4 & 5 & 6 \\
          2 & 3 & 4 & 5 & 6 & 1
        \end{pmatrix} = \begin{pmatrix}
          1 & 2 & 3 & 4 & 5 & 6 \\
          4 & 5 & 2 & 1 & 3 & 6
        \end{pmatrix}
      \]
    \end{proof}

    \item $\begin{pmatrix} 1 & 2 & 3 & 4 & 5 \\
      2 & 1 & 3 & 4 & 5 \end{pmatrix}\begin{pmatrix} 1 & 2 & 3 & 4 & 5 \\
      3 & 2 & 1 & 4 & 5 \end{pmatrix}$.
    
    \begin{proof}
      \[
        \begin{pmatrix}
          1 & 2 & 3 & 4 & 5 \\
          2 & 1 & 3 & 4 & 5
        \end{pmatrix}\begin{pmatrix}
          1 & 2 & 3 & 4 & 5 \\
          3 & 2 & 1 & 4 & 5
        \end{pmatrix} = \begin{pmatrix}
          1 & 2 & 3 & 4 & 5 \\
          3 & 1 & 2 & 4 & 5
        \end{pmatrix}.
      \]
    \end{proof}

    \item $\begin{pmatrix} 1 & 2 & 3 & 4 & 5 \\
      4 & 1 & 3 & 2 & 5 \end{pmatrix}^{-1}\begin{pmatrix} 1 & 2 & 3 & 4 & 5 \\
      2 & 1 & 3 & 4 & 5 \end{pmatrix}\begin{pmatrix} 1 & 2 & 3 & 4 & 5 \\
      4 & 1 & 3 & 2 & 5 \end{pmatrix}$.
    
    \begin{proof}
      \[
        \begin{pmatrix}
          1 & 2 & 3 & 4 & 5 \\
          4 & 1 & 3 & 2 & 5
        \end{pmatrix}^{-1}\begin{pmatrix}
          1 & 2 & 3 & 4 & 5 \\
          2 & 1 & 3 & 4 & 5
        \end{pmatrix}\begin{pmatrix}
          1 & 2 & 3 & 4 & 5 \\
          4 & 1 & 3 & 2 & 5
        \end{pmatrix} = \begin{pmatrix}
          1 & 2 & 3 & 4 & 5 \\
          1 & 4 & 3 & 2 & 5
        \end{pmatrix}.
      \]
    \end{proof}
  \end{enumerate}
\end{homeworkProblem}

\newpage

\begin{homeworkProblem}
  Find the order of the product you obtained in the previous problem.

  \begin{proof}
    Since $\begin{pmatrix} 1 & 2 & 3 & 4 & 5 & 6 \\
      4 & 5 & 2 & 1 & 3 & 6 \end{pmatrix} = \begin{pmatrix} 1 & 4 \end{pmatrix}\begin{pmatrix} 5 & 3 & 2 \end{pmatrix}$, the order of it is the least common multiple of the size of the two cycles, namely $6$.

    Since $\begin{pmatrix} 1 & 2 & 3 & 4 & 5 \\
      3 & 1 & 2 & 4 & 5 \end{pmatrix}$ is a $3$-cycle in $S_5$, the order of it is $3$.

    Since $\begin{pmatrix} 1 & 2 & 3 & 4 & 5 \\
      1 & 4 & 3 & 2 & 5 \end{pmatrix}$ is a $2$-cycle in $S_5$, the order of it is $2$.
  \end{proof}
\end{homeworkProblem}

\newpage

\begin{homeworkProblem}
  Show that if $\sigma, \tau$ are two disjoint cycles, then $\sigma\tau = \tau\sigma$.

  \begin{proof}
    Let $\sigma, \tau \in S_n$, where $S = \{1, 2, \dots, n\}$. Let $i \in S$. If $i$ is not in $\sigma$ nor $\tau$, then $\sigma\tau(i) = \tau\sigma(i) = i$. Since $\sigma, \tau$ are disjoint cycles, $i$ is not in cycle $\tau$ if it is already in $\sigma$, and thus $\sigma\tau(i) = \sigma(i) = \tau\sigma(i)$. By symmetry, we also know that if $i$ is in $\tau$, we get $\tau\sigma(i) = \tau(i) = \sigma\tau(i)$, and we exausted all cases.

  \end{proof}
\end{homeworkProblem}

\newpage

\begin{homeworkProblem}
  Find the cycle decomposition and order.

  \begin{enumerate}[label=(\alph*)]
    \item $\begin{pmatrix} 1 & 2 & 3 & 4 & 5 & 6 & 7 & 8 & 9 \\
      3 & 1 & 4 & 2 & 7 & 6 & 9 & 8 & 5 \end{pmatrix}$.

    \begin{proof}
      The above permutation can be decomposed into 
      \[
        \begin{pmatrix}
          1 & 3 & 4 & 2
        \end{pmatrix}\begin{pmatrix}
          5 & 7 & 9
        \end{pmatrix},
      \]
      and the order of it is the least common multiple of the size of the disjoint cycles, namely $12$.
    \end{proof}

    \item $\begin{pmatrix} 1 & 2 & 3 & 4 & 5 & 6 & 7 \\
      7 & 6 & 5 & 4 & 3 & 2 & 1 \end{pmatrix}$.

    \begin{proof}
      The above permutation can be decomposed into 
      \[
        \begin{pmatrix}
          1 & 7
        \end{pmatrix}\begin{pmatrix}
          2 & 6
        \end{pmatrix}\begin{pmatrix}
          3 & 5
        \end{pmatrix},
      \]
      and the order of it is the least common multiple of the size of the disjoint cycles, namely $2$.
    \end{proof}

    \item $\begin{pmatrix} 1 & 2 & 3 & 4 & 5 & 6 & 7 \\
      7 & 6 & 5 & 3 & 4 & 2 & 1 \end{pmatrix}\begin{pmatrix} 1 & 2 & 3 & 4 & 5 & 6 & 7 \\
      2 & 3 & 1 & 5 & 6 & 7 & 4 \end{pmatrix}$.

    \begin{proof}
      \begin{align*}
        \begin{pmatrix}
          1 & 2 & 3 & 4 & 5 & 6 & 7 \\
          7 & 6 & 5 & 3 & 4 & 2 & 1
        \end{pmatrix}\begin{pmatrix}
          1 & 2 & 3 & 4 & 5 & 6 & 7 \\
          2 & 3 & 1 & 5 & 6 & 7 & 4
        \end{pmatrix}
        &= \begin{pmatrix}
          1 & 2 & 3 & 4 & 5 & 6 & 7 \\
          6 & 5 & 7 & 4 & 2 & 1 & 3
        \end{pmatrix} \\
        &= \begin{pmatrix}
          1 & 6
        \end{pmatrix}\begin{pmatrix}
          2 & 5
        \end{pmatrix}\begin{pmatrix}
          3 & 7
        \end{pmatrix},
      \end{align*}
      and the order of it is the least common multiple of the size of the disjoint cycles, namely $2$.
    \end{proof}
  \end{enumerate}
\end{homeworkProblem}

\newpage

\begin{homeworkProblem}
  Express as the product of disjoint cycles and find the order.

  \begin{enumerate}
    \item[(c)] $\begin{pmatrix} 1 & 2 & 3 & 4 & 5 \end{pmatrix}\begin{pmatrix} 1 & 2 & 3 & 4 & 6 \end{pmatrix}\begin{pmatrix} 1 & 2 & 3 & 4 & 7 \end{pmatrix}$.

    \begin{proof}
      \begin{align*}
        \begin{pmatrix}
          1 & 2 & 3 & 4 & 5
        \end{pmatrix}\begin{pmatrix}
          1 & 2 & 3 & 4 & 6
        \end{pmatrix}\begin{pmatrix}
          1 & 2 & 3 & 4 & 7
        \end{pmatrix}
        &= \begin{pmatrix}
          1 & 4 & 7 & 3 & 6 & 2 & 5
        \end{pmatrix}.
      \end{align*}
      Since it's a $7$-cycle, the order of it is $7$.
    \end{proof}

    \item[(d)] $\begin{pmatrix} 1 & 2 & 3 \end{pmatrix}\begin{pmatrix} 1 & 3 & 2 \end{pmatrix}$.

    \begin{proof}
      \[
        \begin{pmatrix}
          1 & 2 & 3
        \end{pmatrix}\begin{pmatrix}
          1 & 3 & 2
        \end{pmatrix} = \begin{pmatrix}
          1 & 2 & 3 \\
          1 & 2 & 3
        \end{pmatrix}.
      \]
      Since it's the identity element, the order is $1$.
    \end{proof}
  \end{enumerate}
\end{homeworkProblem}

\newpage

\begin{homeworkProblem}
  Express the permutations in the previous problem as the product of transpositions.

  \begin{proof}
    For (c),
    \begin{align*}
      \begin{pmatrix}
        1 & 2 & 3 & 4 & 5
      \end{pmatrix}\begin{pmatrix}
        1 & 2 & 3 & 4 & 6
      \end{pmatrix}\begin{pmatrix}
        1 & 2 & 3 & 4 & 7
      \end{pmatrix}
      &= \begin{pmatrix}
        1 & 4 & 7 & 3 & 6 & 2 & 5
      \end{pmatrix} \\
      &= \begin{pmatrix}
        1 & 5
      \end{pmatrix}\begin{pmatrix}
        1 & 2
      \end{pmatrix}\begin{pmatrix}
        1 & 6
      \end{pmatrix}\begin{pmatrix}
        1 & 3
      \end{pmatrix}\begin{pmatrix}
        1 & 7
      \end{pmatrix}\begin{pmatrix}
        1 & 4
      \end{pmatrix}.
    \end{align*}

    For (d),
    \begin{align*}
      \begin{pmatrix}
        1 & 2 & 3
      \end{pmatrix}\begin{pmatrix}
        1 & 3 & 2
      \end{pmatrix}
      &= \begin{pmatrix}
        1 & 3
      \end{pmatrix}\begin{pmatrix}
        1 & 2
      \end{pmatrix}\begin{pmatrix}
        1 & 2
      \end{pmatrix}\begin{pmatrix}
        1 & 3
      \end{pmatrix}.
    \end{align*}
  \end{proof}
\end{homeworkProblem}

\newpage

\begin{homeworkProblem}
  Find the conjugate of $\sigma = (1, 4, 7, 2)(3, 6, 5) \in S_7$ by $\tau = (1, 2, 3)(4, 7, 5)$. What is the order of $\sigma$ and $\tau$?

  \begin{proof}
    \begin{align*}
      \tau\sigma\tau^{-1}
      &= \begin{pmatrix}
        \tau(1) & \tau(4) & \tau(7) & \tau(2)
      \end{pmatrix}\begin{pmatrix}
        \tau(3) & \tau(6) & \tau(5)
      \end{pmatrix} \\
      &= \begin{pmatrix}
        2 & 7 & 5 & 3
      \end{pmatrix}\begin{pmatrix}
        1 & 6 & 4
      \end{pmatrix}.
    \end{align*}

    The order of $\sigma$ and $\tau$ are $12$ and $3$ respectively.
  \end{proof}
\end{homeworkProblem}

\newpage

\begin{homeworkProblem}
  Find an element $\tau \in S_7$ that carries $\sigma = (1, 2, 5)(3, 6, 7, 4)$ into $\sigma' = (3, 1, 4)(2, 7, 6, 5)$, that is find $\tau \in S_7$ such that
  \[
    \sigma' = \tau\sigma\tau^{-1}.
  \]

  \begin{proof}
    Consider 
    \[
      \tau = \begin{pmatrix}
        1 & 2 & 3 & 4 & 5 & 6 & 7 \\
        3 & 1 & 2 & 5 & 4 & 7 & 6
      \end{pmatrix}.
    \]
    \begin{align*}
      \tau\sigma\tau^{-1}
      &= \begin{pmatrix}
        \tau(1) & \tau(2) & \tau(5)
      \end{pmatrix}\begin{pmatrix}
        \tau(3) & \tau(6) & \tau(7) & \tau(4)
      \end{pmatrix} \\
      &= \begin{pmatrix}
        3 & 1 & 4
      \end{pmatrix}\begin{pmatrix}
        2 & 7 & 6 & 5
      \end{pmatrix} \\
      &= \sigma',
    \end{align*}
    and so $\tau$ is what we're looking for.
  \end{proof}
\end{homeworkProblem}
\end{document}