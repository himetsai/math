\documentclass{article}

\usepackage{fancyhdr}
\usepackage{extramarks}
\usepackage{amsmath}
\usepackage{amsthm}
\usepackage{amsfonts}
\usepackage{tikz}
\usepackage[plain]{algorithm}
\usepackage{algpseudocode}
\usepackage{enumitem}
\usepackage{amssymb}

\usetikzlibrary{automata,positioning}

%
% Basic Document Settings
%

\topmargin=-0.45in
\evensidemargin=0in
\oddsidemargin=0in
\textwidth=6.5in
\textheight=9.0in
\headsep=0.25in

\linespread{1.1}

\pagestyle{fancy}
\lhead{\hmwkAuthorName}
\chead{\hmwkClass:\ \hmwkTitle}
\rhead{\firstxmark}
\lfoot{\lastxmark}
\cfoot{\thepage}

\renewcommand\headrulewidth{0.4pt}
\renewcommand\footrulewidth{0.4pt}

\setlength\parindent{0pt}
\setlength{\parskip}{5pt}

%
% Create Problem Sections
%

\newcommand{\enterProblemHeader}[1]{
    \nobreak\extramarks{}{Problem \arabic{#1} continued on next page\ldots}\nobreak{}
    \nobreak\extramarks{Problem \arabic{#1} (continued)}{Problem \arabic{#1} continued on next page\ldots}\nobreak{}
}

\newcommand{\exitProblemHeader}[1]{
    \nobreak\extramarks{Problem \arabic{#1} (continued)}{Problem \arabic{#1} continued on next page\ldots}\nobreak{}
    \stepcounter{#1}
    \nobreak\extramarks{Problem \arabic{#1}}{}\nobreak{}
}

\setcounter{secnumdepth}{0}
\newcounter{partCounter}
\newcounter{homeworkProblemCounter}
\setcounter{homeworkProblemCounter}{1}
\nobreak\extramarks{Problem \arabic{homeworkProblemCounter}}{}\nobreak{}

%
% Homework Problem Environment
%
% This environment takes an optional argument. When given, it will adjust the
% problem counter. This is useful for when the problems given for your
% assignment aren't sequential. See the last 3 problems of this template for an
% example.
%
\newenvironment{homeworkProblem}[1][-1]{
    \ifnum#1>0
        \setcounter{homeworkProblemCounter}{#1}
    \fi
    \section{Problem \arabic{homeworkProblemCounter}}
    \setcounter{partCounter}{1}
    \enterProblemHeader{homeworkProblemCounter}
}{
    \exitProblemHeader{homeworkProblemCounter}
}

%
% Homework Details
%   - Title
%   - Due date
%   - Class
%   - Section/Time
%   - Instructor
%   - Author
%

\newcommand{\hmwkTitle}{Homework\ \#6}
\newcommand{\hmwkDueDate}{November 16, 2023}
\newcommand{\hmwkClass}{MATH 100A}
\newcommand{\hmwkClassTime}{Section A02 5:00PM - 5:50PM}
\newcommand{\hmwkSectionLeader}{Castellano}
\newcommand{\hmwkClassInstructor}{Professor McKernan}
\newcommand{\hmwkSource}{Source Consulted: Textbook, Lecture, Discussion}
\newcommand{\hmwkAuthorName}{\textbf{Ray Tsai}}
\newcommand{\hmwkPID}{A16848188}

%
% Title Page
%

\title{
    \vspace{2in}
    \textmd{\textbf{\hmwkClass:\ \hmwkTitle}}\\
    \normalsize\vspace{0.1in}\small{Due\ on\ \hmwkDueDate\ at 12:00pm}\\
    \vspace{0.1in}\large{\textit{\hmwkClassInstructor}} \\
    \vspace{0.1in}\small\hmwkClassTime \\
    \small Section Leader: \hmwkSectionLeader \\
    \vspace{0.1in}\small\hmwkSource \\
    \vspace{3in}
}

\author{
  \hmwkAuthorName \\
  \vspace{0.1in}\small\hmwkPID
}
\date{}

\renewcommand{\part}[1]{\textbf{\large Part \Alph{partCounter}}\stepcounter{partCounter}\\}

%
% Various Helper Commands
%

% Useful for algorithms
\newcommand{\alg}[1]{\textsc{\bfseries \footnotesize #1}}

% For derivatives
\newcommand{\deriv}[1]{\frac{\mathrm{d}}{\mathrm{d}x} (#1)}

% For partial derivatives
\newcommand{\pderiv}[2]{\frac{\partial}{\partial #1} (#2)}

% Integral dx
\newcommand{\dx}{\mathrm{d}x}

% Probability commands: Expectation, Variance, Covariance, Bias
\newcommand{\Var}{\mathrm{Var}}
\newcommand{\Cov}{\mathrm{Cov}}
\newcommand{\Bias}{\mathrm{Bias}}
\newcommand*{\Z}{\mathbb{Z}}
\newcommand*{\Q}{\mathbb{Q}}
\newcommand*{\R}{\mathbb{R}}
\newcommand*{\C}{\mathbb{C}}
\newcommand*{\N}{\mathbb{N}}
\newcommand*{\prob}{\mathds{P}}
\newcommand*{\E}{\mathds{E}}

\begin{document}

\maketitle

\pagebreak

\begin{homeworkProblem}
    If $G$ is a cyclic group and $N$ is a subgroup of $G$, show that $G/N$ is a cyclic group.

    \begin{proof}
        Since $G$ is cyclic, there exists some $a \in G$ that generates $G$. In other words, for all $g \in G$, $g = a^i$, for some $i \in \Z$. Since $G$ is abelian, any subgroup $N$ of $G$ is normal. Consider $G/N = \{gN \, | \, g \in G\} = \{a^iN \, | \, i \in \Z\}$. We first show that $a^iN = (aN)^i$ for all positive $i$ by induction on $i$. The base case is trivial. Suppose that $i > 1$. Then $a^iN = (a^{i - 1}N)aN = (aN)^{i - 1}aN = (aN)^{i}$, by induction. Now consider $(aN)^{-1}$. Since $(aN)^{-1}$ is the inverse of $aN$, $aN(aN)^{-1} = (aN)^0 = N = (aN)(a^{-1}N)$, and so we know $a^{-1}N = (aN)^{-1}$ and $N = a^{0}N = (aN)^0$. Then, with the same induction argument, we can show that $a^{-i}N = (aN)^{-i}$ for all positive $i$, and thus all $gN \in G/N$ can be generated by $aN$. Therefore, $G/N$ is also a cyclic group, with generator $aN$.
    \end{proof}
\end{homeworkProblem}

\newpage

\begin{homeworkProblem}
    If $G$ is an abelian group and $N$ is a subgroup of $G$, show that $G/N$ is an abelian group.

    \begin{proof}
        Since $G$ is abelian, any subgroup $N$ of $G$ is normal. Let $gN, hN \in G/N$, for some $g, h \in G$. Since $(gN)(hN) = ghN = hgN = (hN)(gN)$, $G/N$ is abelian.
    \end{proof}
\end{homeworkProblem}

\pagebreak

\begin{homeworkProblem}
    If $G$ is a group and $Z(G)$ the center of $G$, show that if $G/Z(G)$ is cyclic, then $G$ is abelian.

    \begin{proof}
        Suppose that $G/Z(G)$ is generated by some element $gZ(G)$, where $g \in G$. Then, $G/Z(G) = \{(gZ(G))^{i} \, | \, i \in \Z\} = \{g^iZ(G) \, | \, i \in \Z\}$. Let $a, b \in G$. Then, $a = g^iz$ and $b = g^jz'$, for some $i, j \in \Z$, $z, z' \in Z(G)$. This immediately follows that
        \[
            ab = (g^iz)(g^jz') = (g^{i + j})zz' = (g^j)(g^i)z'z = (g^jz')(g^iz) = ba,
        \]
        and thus $G$ is abelian.
        \end{proof}
\end{homeworkProblem}

\pagebreak

\begin{homeworkProblem}
    If $G$ is a group and $N \vartriangleleft G$ is such that $G/N$ is abelian, prove that $aba^{-1}b^{-1} \in N$ for all $a, b \in G$.

    \begin{proof}
        Let $aN, bN \in G/N$. Since $G/N$ is abelian, $abN = aNbN = bNaN = baN$, and thus $aba^{-1}b^{-1} \in N$.
    \end{proof}
\end{homeworkProblem}

\pagebreak

\begin{homeworkProblem}
    If $G$ is a group and $N \vartriangleleft G$ is such that 
    \[
        aba^{-1}b^{-1} \in N
    \]
    for all $a, b \in G$, prove that $G/N$ is abelian.

    \begin{proof}
        Let $aN, bN \in G/N$. Since $bab^{-1}a^{-1} \in N$, 
        \[
            NaNb = Nab = N(bab^{-1}a^{-1})ab = Nba = NbNa,
        \]
        and thus $G/N$ is abelian.
    \end{proof}
\end{homeworkProblem}

\pagebreak

\begin{homeworkProblem}
    Let $G$ be an abelian group (possibly infinite) and let the set $T = \{a \in G \, | \, a^m = e, m > 1 \text{ depending on } a\}$. Prove that:

    \begin{enumerate}[label=(\alph*)]
        \item $T$ is a subgroup of $G$.

        \begin{proof}
            We first note that $T$ is nonempty, as $e^l = e$ for all $l$, so $e \in T$. It suffices to show $T$ is closed under multiplication and taking inverses. Let $a, b \in T$. We know $a^m = b^n = e$ for some positive $m, n$. Let $k = mn$. Since $G$ is abelian, $(ab)^k = a^kb^k = (a^m)^n(b^n)^m = e$, and so $ab \in T$. Since $(a^{-1})^m = (a^{m})^{-1} = e$, $a^{-1} \in T$, and we are done.
        \end{proof}

        \item $G/T$ has no element -- other than its identity element -- of finite order.
        
        \begin{proof}
            We may assume there exists $g \in G\backslash T$, otherwise we are done. Then, $g$ is of infinite order. Note that for $i \in N$, $g^i \notin T$, otherwise there exists $m$ such that $(g^i)^m = g^{im} = e$, contradiction. Since $G$ is abelian, $T$ is normal, and thus $(gT)^i = g^iT$ for all $i \in \N$. This immediately follows that $g^i \notin T$ so $g^iT \neq T$, which implies that $gT$ is of infinite order. 
        \end{proof}
    \end{enumerate}
\end{homeworkProblem}

\pagebreak

\begin{homeworkProblem}
    Let $G$ be the group of all real-valued functions on the unit interval $[0, 1]$, where we define, for $f, g \in G$, addition by $(f + g)(x) = f(x) + g(x)$ for every $x \in [0, 1]$. If $N = \{f \in G \, | \, f\left(\frac{1}{4}\right) = 0\}$, prove that $G/N \simeq$ real numbers under $+$.

    \begin{proof}
        We first show that $N$ is a normal subgroup. Note that since $e(x) = 0 \in N$, $N$ is nonempty. Let $f, g \in N$. Since $(f + g)\left(\frac{1}{4}\right) = f\left(\frac{1}{4}\right) + g\left(\frac{1}{4}\right) = 0$, $f + g \in N$. We also know that $f^{-1}\left(\frac{1}{4}\right) = -f\left(\frac{1}{4}\right) = 0$, so $f^{-1} \in N$. Since addition is commutative, $G$ is abelian, and thus $N$ is indeed a normal subgroup. Define $\phi: G \rightarrow \R$ as $\phi(f) = f\left(\frac{1}{4}\right)$. Let $f, g \in G$, such that $f = g$. Then, we know $\phi(f) = f\left(\frac{1}{4}\right) = g\left(\frac{1}{4}\right) = \phi(g)$, so $\phi$ is well defined. Since $\phi(f + g) = (f + g)\left(\frac{1}{4}\right) = f\left(\frac{1}{4}\right) + g\left(\frac{1}{4}\right) = \phi(f) + \phi(g)$, $\phi$ is homomorphic. For all $x \in \R$, there exists $h \in G$, such that $\phi(h) = h\left(\frac{1}{4}\right) = x$, and thus $\phi$ is onto. Let $k \in N$. Then $\phi(k) = 0$ , and so $k \in \text{Ker
        }\phi$. Suppose that $k \in \text{Ker }\phi$, then $\phi(k) = k\left(\frac{1}{4}\right) = 0$, which implies that $k \in N$. Therefore, Ker $\phi = N$, and $G/N \simeq \R$, by the First Isomorphism Theorem.
    \end{proof}
\end{homeworkProblem}

\pagebreak

\begin{homeworkProblem}
    If $G_1, G_2$ are two groups and $G = G_1 \times G_2 = \{(a,b) \, | \, a \in G_1, b \in G_2\}$, where we define $(a,b)(c,d) = (ab,cd)$, show that:
    \begin{enumerate}[label=(\alph*)]
        \item $N = \{(a, e_2) \, | \, a \in G_1\}$, where $e_2$ is the unit element of $G_2$, is a normal subgroup of $G$.

        \begin{proof}
            $N$ is obviously not empty. Let $a, a' \in G_1$. Since $(a, e_2)(a', e_2) = (aa', e_2) \in G$ and $(a^{-1}, e_2)$, the inverse of $(a, e_2)$, is also in $G$, $N$ is a subgroup. Let $g = (g_1, g_2) \in G$. Since $g(a, e_2)g^{-1} = (g_1ag_1^{-1}, e_2) \in N$, $N$ is normal.
        \end{proof}

        \item $N \simeq G_1$
        
        \begin{proof}
            Define $\phi: N \rightarrow G_1$ as $\phi((a, e_2)) = a$. $\phi$ is obviously well defined. Let $x = (a, e_2), y = (a', e_2)$ Since $\phi(xy) = aa' = \phi(x)\phi(y)$, $\phi$ is a homomorphism. For $g_1 \in G_1$, we have $\phi((g_1, e_2)) = g_1$, and so $\phi$ is surjective. Since $G_1$ has an unique identity element $e_1$, Ker $\phi = \{(e_1, e_2)\}$, and so $\phi$ is injective. We have shown that $\phi$ is a bijective homomorphism, so $N \simeq G_1$.
        \end{proof}

        \item $G/N \simeq G_2$

        \begin{proof}
            Define $\psi: G \rightarrow G_2$ as $\psi((a, b)) = b$. Let $x = (a, b), y = (a', b')$. Since $\psi(xy) = bb' = \psi(x)\psi(y)$, $\psi$ is a homomorphism. For $g_2 \in G_2$, we have $\psi((g_1, g_2)) = g_2$ for some $g_1 \in G_1$, so $\psi$ is surjective. Since Ker $\psi = \{(a, e_2) \, | \, a \in G_1\} = N$, $G/N \simeq G_2$, by the First Isomorphism Theorem. 
        \end{proof}
    \end{enumerate}
\end{homeworkProblem}

\pagebreak

\begin{homeworkProblem}
    If $G$ is a group and $N \vartriangleleft G$, show that if $a \in G$ has finite order $o(a)$, then $Na$ in $G/N$ has finite order $m$, where $m \, | \, o(a)$.

    \begin{proof}
        Define $\phi: G \rightarrow G/N$ as $\phi(a) = aN$. We already know $\phi$ is a surjective homomorphism. Suppose that $a \in G$ is of finite order $o(a)$. Since $\phi$ is a homomorphism, $(Na)^{o(a)} = (\phi(a))^{o(a)} = \phi(a^{o(a)}) = e$, and thus $Na$ is of some order $m \leq o(a)$. This means that $(Na)^m = Na^m = N$, and so $a^m \in N$. Suppose for the sake of contradiction that $m \nmid o(a)$. Then, $o(a) = pm + q$, for some $p, q \in \Z$, $0 < q < m$. Since $e = a^{o(a)} = (a^{m})^pa^q \in N$, we know $a^q \in N$. However, we then get $(Na)^q = Na^q = N$, which implies $Na$ is of order $q < m$, contradiction. Therefore, $m \, | \, o(a)$.
    \end{proof}
\end{homeworkProblem}

\pagebreak

\begin{homeworkProblem}
    If $\varphi$ is a homomorphism of $G$ onto $G'$ and $N \vartriangleleft G$, show that $\varphi(N) \vartriangleleft G'$.

    \begin{proof}
        For all $g$, we know $\varphi(g)\varphi(N)(\varphi(g))^{-1} = \varphi(gNg^{-1})$. However, $gNg^{-1} \subset N$, and so $\phi(gNg^{-1}) \subset \phi(N)$. Therefore, $\phi(N) \vartriangleleft G'$.
    \end{proof}
\end{homeworkProblem}
\end{document}