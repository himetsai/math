\documentclass{article}
\usepackage{amsfonts, amsmath, amssymb, amsthm} % Math notations imported
\usepackage{enumitem}

\newtheorem{thm}{Theorem}
\newtheorem{prop}[thm]{Proposition}
\newtheorem{cor}[thm]{Corollary}

% title information
\title{Math 109 Week 9 Discussion}
\author{Ray Tsai}
\date{11/15/2022}

\begin{document}

\maketitle

\begin{enumerate}
    \item 
    There is a surjective function $\beta$ such that 
    \begin{gather}
        \beta: \mathbb{R} \times \mathbb{Z} \rightarrow \mathbb{R} \\
        \beta(r, z) = r.
    \end{gather}
    Since $\mathbb{R}$ is uncountable, $\mathbb{R} \times \mathbb{Z}$ is uncountable.

    \item
    There is a surjective function $f$ such that
    \begin{gather}
        f:  \mathbb{Z}^2 \rightarrow \mathbb{Z}[i] \\
            f(a, b) = a + bi.
    \end{gather}
    Since $\mathbb{Z}^2$ is countable, $\mathbb{Z}[i]$ is countable.

    \item
    Let $P$ be the set of all prime numbers. There is a surjective function $g$ such that 
    \begin{gather}
        g:  P \rightarrow \{\frac{p}{7}\,|\,p \in P \}  \\
            f(x) = \frac{x}{7}.
    \end{gather}
    Since $P \subseteq \mathbb{N}$, $\{\frac{p}{7}\,|\, p \in P \}$ is countable.

    \item
    Let $A = \{x \in \mathbb{R} \, | \, x = \frac{n\pi}{2} \, \text{or} \, \frac{ne}{3} \, n \in \mathbb{Z} \}$. There is a surjective function $h$ such that
    \[
        h: \mathbb{Z} \rightarrow A
    \]
    \[
        h(x) = \begin{cases}
            \frac{n\pi}{4}, & \text{if $n$ is even} \\
            \frac{(n - 1)e}{6}, & \text{if $n$ is odd}.
            \end{cases}
    \]
    Since $\mathbb{Z}$ is countable, $A$ is countable.

    \item
    There exists a bijective function $\alpha$ such that 
    \begin{gather}
        \alpha: \mathbb{R}^4 \rightarrow M_2(\mathbb{R}) \\
        \alpha(a, b, c, d) = \begin{pmatrix}
            a & b \\
            c & d
        \end{pmatrix}.
    \end{gather}
    Since $\mathbb{R}^4$ is uncountable, $M_2(\mathbb{R})$ is uncountable.

    \item
    Let $S = \{ax + b, \, a, b \in \mathbb{R}\}$. There exists a bijective function $r$ such that 
    \begin{gather}
        r: \mathbb{R}^2 \rightarrow S \\
        r(a, b) = ax + b.
    \end{gather}
    Since $\mathbb{R}^2$ is uncountable, $S$ is uncountable.
\end{enumerate}

\end{document}