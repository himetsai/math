\documentclass{article}
\usepackage{amsfonts, amsmath, amssymb, amsthm} % Math notations imported
\usepackage{enumitem}

\newtheorem{thm}{Theorem}
\newtheorem{prop}[thm]{Proposition}
\newtheorem{cor}[thm]{Corollary}

% title information
\title{Math 109 HW 6}
\author{Ray Tsai}
\date{11/7/2022}

% main content
\begin{document} 

% placing title information; comment out if using fancyhdr
\maketitle 

\begin{enumerate}
% Q1
\item 
\begin{prop}
    For all $n \in \mathbb{Z}_{\geq 1}$, we have 
    \begin{gather}
        1\cdot2 + 2\cdot3 + \ldots + n(n+1) = \frac{n(n+1)(n+2)}{3}.
    \end{gather}
\end{prop}
\begin{proof}
    We will proceed by induction. Let $n \in \mathbb{Z}^+$. \\
    If $n = 1$,
    \begin{align}
        1\cdot2 &= 2 \\
                &= \frac{1(1+1)(1+2)}{3} \\
                &= \frac{6}{3}.
    \end{align}
    Suppose that for some $k \in \mathbb{Z}_{\geq 1}$, we have
    \begin{gather}
        1\cdot2 + 2\cdot3 + \ldots + k(k+1) = \frac{k(k+1)(k+2)}{3}.
    \end{gather}
    We then have
    \begin{align}
        1\cdot2 + 2\cdot3 + \ldots + k(k+1) + (k+1)(k+2) &= \frac{k(k+1)(k+2)}{3} + (k+1)(k+2) \\
        &= \frac{k(k+1)(k+2) + 3(k+1)(k+2)}{3} \\
        &= \frac{(k+3)(k+1)(k+2)}{3} \\
        &= \frac{(k+1)((k+1)+1)((k+1)+2)}{3}.
    \end{align}
    Thus, if the equation is correct when $n = k$, then the equation also works when $n = k + 1$. \\

    Therefore, for all $n \in \mathbb{Z}_{\geq 1}$, we have 
    \begin{gather}
        1\cdot2 + 2\cdot3 + \ldots + n(n+1) = \frac{n(n+1)(n+2)}{3}.
    \end{gather}
\end{proof}

% Q2
\item 
\begin{prop}
    For all $n \in \mathbb{Z}_{\geq 5}$, we have $4n < 2^n$.
\end{prop}
\begin{proof}
    We will proceed by induction. Let $n \in \mathbb{Z}^+$.\\

    If $n = 5$, we have $4\cdot5 = 20 < 32 = 2^5$. \\

    Suppose that for some $k \in \mathbb{Z}_{\geq 5}$, we have $4k < 2^k$, which also means that $k < 2^{k-2}$. \\
    We then have 
    \begin{align}
        4(k+1) &= 4k + 4 \\
        2^{k+1} &= 2\cdot2^k \\
                &= 8\cdot2^{k-2}
    \end{align}
    Since $k < 2^{k-2}$, we know that $8k < 8\cdot2^{k-2} = 2^{k+1}$. \\
    Since $4m > 4$ for all $m \in \mathbb{Z}_{> 1}$, as $4(m - 1) > 0$, we have $4k + 4 < 4k + 4k = 8k$. \\Since $8k < 8\cdot2^{k-2}$, we have $4k + 4 < 8k < 2^{k+1}$ \\

    Therefore, for all $n \in \mathbb{Z}_{\geq 5}$, we have $4n < 2^n$.
\end{proof}

% Q3
\item \begin{prop}
    For all $n \in \mathbb{Z}_{\geq 0}$, we have $7|2^{n+2} + 3^{2n+1}$.
\end{prop}
\begin{proof}
    We will proceed by induction. Let $n \in \mathbb{Z}^+$. \\

    If $n = 0$, we have 
    \begin{align}
        2^{2} + 3 = 7
    \end{align}
    which is divisible by $7$.

    Suppose that for some $k \in \mathbb{Z}_{\geq 0}$, there exists some integer $m$ such that $7m = 2^{k+2} + 3^{2k+1}$. \\
    We then have 
    \begin{align}
        2^{(k+1)+2} + 3^{2(k+1)+1} &= 2\cdot2^{k+2} + 9\cdot3^{2k+1}
    \end{align}
    Since $7m = 2^{k+2} + 3^{2k+1}$, we know that $2^{k+2} = 7m - 3^{2k+1}$. \\
    We then have 
    \begin{align}
        2\cdot2^{k+2} + 9\cdot3^{2k+1} &= 2\cdot(7m - 3^{2k+1}) + 9\cdot3^{2k+1} \\
        &= 7\cdot2m + 7\cdot3^{2k+1} \\
        &= 7(2m + 3^{2k+1})
    \end{align}
    Since $2m + 3^{2k+1}$ is an integer, $ 2^{(k+1)+2} + 3^{2(k+1)+1}$ is also divisible by $7$. \\

    Therefore, for all $n \in \mathbb{Z}_{\geq 0}$, we have $7|2^{n+2} + 3^{2n+1}$.
\end{proof}

% Q4
\item 
\begin{prop}
    Every positive integer $n$ can be written as the sum of distinct powers of 2.
\end{prop}
\begin{proof}
    We will show by strong induction on $n$ that for all $n \geq 1$, we can write $n$ as the sum of distinct powers of 2.
    
    For $1$, we have $1 = 2^0$. 
    
    Let $k \in \mathbb{Z}^+$. Suppose that for all integer $j$ where $1 \leq j \leq k$, we can write $j$ as the sum of distinct powers of 2. We will show that $k + 1$ can also be written as the sum of the distinct powers of 2. Since there exists a non-negative integer $m$ such that $2^m \leq k + 1 < 2^{m+1}$, we have 
    \begin{align}
        k + 1 &= 2^m + (k + 1 - 2^m).
    \end{align}
    Since $2^m \leq k + 1$, we know that $0 \leq k + 1 - 2^m$. Let $r$ be some non-negative integer such that $r = k + 1 - 2^m$. We can then separate it into two cases, $r = 0$ and $r \neq 0$. 

    If $r = 0$, we have
    \begin{align}
        k + 1 &= 2^m.
    \end{align}
    If $r \neq 0$, since $1 \leq r \leq k$, we have
    \begin{gather}
        r = a_{m}2^{m - 1} + a_{m-1}2^{m - 2} + \ldots + a_{1}2^1 + a_{0}2^0.
    \end{gather}
    where $a_{m}, a_{m-1}, \ldots ,a_2, a_1 \in \{0, 1\}$. We then have
    \begin{align}
        k + 1 &= 2^m + r \\
              &= 2^m + a_{m}2^{m - 1} + a_{m-1}2^{m - 2} + \ldots + a_{1}2^1 + a_{0}2^0.
    \end{align}
    This shows that $k + 1$ can also be written the sum of the distinct powers of 2. 
    
    Therefore, every positive integer $n$ can be written as the sum of distinct powers of 2.
\end{proof}

% Q5
\item 
\begin{prop}
   If $S$ is a non-empty subsets of $\mathbb{Z}_{> 0}$, then $S$ has a smallest element.
\end{prop}
\begin{proof}
    We will prove by contrapositive.

    Suppose $S$ is a subset of $\mathbb{Z}_{> 0}$, which has no smallest element. We will prove that $S$ is empty. To do so, we will show by strong induction on n that for all $n \geq 1$, we have $n \notin S$.

    If $n = 1$, $n$ cannot be in $S$ because if $1 \in S$, then 
    $1$ will be the smallest integer in $S$, as it is the smallest positive integer.

    Let $k \in \mathbb{Z}_{> 0}$. Suppose for all integer $m$ where $0 < m \leq k$, we have $m \notin S$. We will show that $k + 1 \notin S$.

    Since there are no positive integers that are smaller than $k + 1$ in $S$, $k + 1 \notin S$ because if $k + 1 \in S$, then it would be the smallest element in $S$.

    Thus, $k + 1 \notin S$ if $m \notin S$ for all integer $m$ where $0 < m \leq k$.

    Therefore, $S$ is empty.
\end{proof}

% Q6
\item 
\begin{prop}
    There exists a not-empty set $S$, such that for all $x \in S$, $0 \leq x \leq 1$ and $S$ does not have the smallest element.
\end{prop}
\begin{proof}
    Consider the case where $S = \{\frac{1}{2^n}|n \in \mathbb{Z}_{\geq 0}\}$ . Suppose for the sake of contradiction that there exists a smallest element $k \in S$. Since $k = \frac{1}{2^m}$ for some non-negative integer $m$, we have $\frac{k}{2} \in S$ because $\frac{k}{2} = \frac{1}{2^{m+1}}$. However, $\frac{k}{2} < k$, which contradicts our assumption that $k$ is the smallest element in $S$. 

    Therefore, there exists a not-empty set $S$, such that for all $x \in S$, $0 \leq x \leq 1$ and $S$ does not have the smallest element.
\end{proof}

% Q7
\item 
A counter example to the proposition is the Fibonacci Sequence, since the first and third elements, which are 1 and 2, are not equal.

The error of the given induction proof is that the base case is not strong enough to support the induction hypothesis. The hypothesis is comparing the current element with elements that precede it. However, the base case does not compare an element with its previous elements, but only compared an element with itself.

% Q8
\item 
\begin{prop}
  For all integer $n$,  if $n$ is not divisible by $3$, then $n^2$ has remainder $1$ when divided by $3$.
\end{prop} 
\begin{proof}
    Let $n$ be some integer that is not divisible by $3$. We will show that $n^2$ has a remainder $1$ when divided by $3$. 
    
    By the Division Algorithm, if $n$ has a remainder $k$ when divided by $3$, $n$ can be written as $n = 3q + k$ for some integer $q$.

    Thus, since $n$ can only have a remainder of $1$ or $2$ when divided by $3$, we can separate it into 2 cases, $n = 3q + 1$ and $n = 3q + 2$, $q \in \mathbb{Z}$.

    If $n = 3q + 1$, we have
    \begin{align}
        n^2 &= (3q + 1)^2 \\
            &= 9q^2 + 6q + 1 \\
            &= 3(3q^2 + 2q) + 1.
    \end{align}
    Since $3q^2 + 2q$ is an integer, it is shown that $n^2$ has a remainder of $1$ when divided by $3$.

    If $n = 3q + 2$, we have
    \begin{align}
        n^2 &= (3q + 2)^2 \\
            &= 9q^2 + 12q + 4 \\
            &= 3(3q^2 + 4q + 1) + 1.
    \end{align}
    Since $3q^2 + 4q + 1$ is an integer, it is shown that $n^2$ has a remainder of $1$ when divided by $3$.

    Therefore, $n^2$ has a remainder of $1$ when divided by $3$.
\end{proof}
    
\end{enumerate}
\end{document}