\documentclass{article}

\usepackage{fancyhdr}
\usepackage{extramarks}
\usepackage{amsmath}
\usepackage{amsthm}
\usepackage{amsfonts}
\usepackage{tikz}
\usepackage[plain]{algorithm}
\usepackage{algpseudocode}
\usepackage{enumerate}
\usepackage{amssymb}

\usetikzlibrary{automata,positioning}

%
% Basic Document Settings
%

\topmargin=-0.45in
\evensidemargin=0in
\oddsidemargin=0in
\textwidth=6.5in
\textheight=9.0in
\headsep=0.25in

\linespread{1.1}

\pagestyle{fancy}
\lhead{\hmwkAuthorName}
\chead{\hmwkClass:\ \hmwkTitle}
\rhead{\firstxmark}
\lfoot{\lastxmark}
\cfoot{\thepage}

\renewcommand\headrulewidth{0.4pt}
\renewcommand\footrulewidth{0.4pt}

\setlength\parindent{0pt}
\setlength{\parskip}{5pt}

%
% Create Problem Sections
%

\newcommand{\enterProblemHeader}[1]{
    \nobreak\extramarks{}{Problem \arabic{#1} continued on next page\ldots}\nobreak{}
    \nobreak\extramarks{Problem \arabic{#1} (continued)}{Problem \arabic{#1} continued on next page\ldots}\nobreak{}
}

\newcommand{\exitProblemHeader}[1]{
    \nobreak\extramarks{Problem \arabic{#1} (continued)}{Problem \arabic{#1} continued on next page\ldots}\nobreak{}
    \stepcounter{#1}
    \nobreak\extramarks{Problem \arabic{#1}}{}\nobreak{}
}

\setcounter{secnumdepth}{0}
\newcounter{partCounter}
\newcounter{homeworkProblemCounter}
\setcounter{homeworkProblemCounter}{1}
\nobreak\extramarks{Problem \arabic{homeworkProblemCounter}}{}\nobreak{}

%
% Homework Problem Environment
%
% This environment takes an optional argument. When given, it will adjust the
% problem counter. This is useful for when the problems given for your
% assignment aren't sequential. See the last 3 problems of this template for an
% example.
%
\newenvironment{homeworkProblem}[1][-1]{
    \ifnum#1>0
        \setcounter{homeworkProblemCounter}{#1}
    \fi
    \section{Problem \arabic{homeworkProblemCounter}}
    \setcounter{partCounter}{1}
    \enterProblemHeader{homeworkProblemCounter}
}{
    \exitProblemHeader{homeworkProblemCounter}
}

%
% Homework Details
%   - Title
%   - Due date
%   - Class
%   - Section/Time
%   - Instructor
%   - Author
%

\newcommand{\hmwkTitle}{Homework\ \#4}
\newcommand{\hmwkDueDate}{Feb 9, 2024}
\newcommand{\hmwkClass}{MATH 140A}
\newcommand{\hmwkClassInstructor}{Professor Seward}
\newcommand{\hmwkAuthorName}{\textbf{Ray Tsai}}
\newcommand{\hmwkPID}{A16848188}

%
% Title Page
%

\title{
    \vspace{2in}
    \textmd{\textbf{\hmwkClass:\ \hmwkTitle}}\\
    \normalsize\vspace{0.1in}\small{Due\ on\ \hmwkDueDate\ at 23:59pm}\\
    \vspace{0.1in}\large{\textit{\hmwkClassInstructor}} \\
    \vspace{3in}
}

\author{
  \hmwkAuthorName \\
  \vspace{0.1in}\small\hmwkPID
}
\date{}

\renewcommand{\part}[1]{\textbf{\large Part \Alph{partCounter}}\stepcounter{partCounter}\\}

%
% Various Helper Commands
%

% Useful for algorithms
\newcommand{\alg}[1]{\textsc{\bfseries \footnotesize #1}}

% For derivatives
\newcommand{\deriv}[1]{\frac{\mathrm{d}}{\mathrm{d}x} (#1)}

% For partial derivatives
\newcommand{\pderiv}[2]{\frac{\partial}{\partial #1} (#2)}

% Integral dx
\newcommand{\dx}{\mathrm{d}x}

% Probability commands: Expectation, Variance, Covariance, Bias
\newcommand{\Var}{\mathrm{Var}}
\newcommand{\Cov}{\mathrm{Cov}}
\newcommand{\Bias}{\mathrm{Bias}}
\newcommand*{\Z}{\mathbb{Z}}
\newcommand*{\Q}{\mathbb{Q}}
\newcommand*{\R}{\mathbb{R}}
\newcommand*{\C}{\mathbb{C}}
\newcommand*{\N}{\mathbb{N}}
\newcommand*{\prob}{\mathds{P}}
\newcommand*{\E}{\mathds{E}}

\begin{document}

\maketitle

\pagebreak

\begin{homeworkProblem}
  Let $E^\circ$ denote the set of all interior points of a set $E$.

  \begin{enumerate}[(a)]
    \item Prove that $E^\circ$ is always open.
    \begin{proof}
      Let $p \in E^\circ$. Since $p$ is an interior point of $E$, there exists a neighborhood $N$ of
      $p$ such that $N \subset E$. Let $q \in N$. Since $N$ is an open set, there exists a
      neighborhood $V$ of $q$ such that $V \subset N \subset E$, and thus $q$ is also an interior
      point of $E$. Hence, $N \subset E^\circ$, so $E^\circ$ is open.
    \end{proof}
    \item Prove that $E$ is open if and only if $E^\circ = E$.
    \begin{proof}
      Suppose that $E$ is open. By definition, we know $E \subset E^{\circ}$. For $p \in E^{\circ}$,
      we know there exists a neighborhood $N$ of $p$ such that $N \subset E$. However, $p \in N
      \subset E$, and thus $E^{\circ} \subset E$. Therefore, $E^\circ = E$.

      We now assume the converse. For $p \in E$, since $p \in E^\circ$, $p$ is an interior point of
      $E$, and thus $E$ is open.
    \end{proof}
    \item If $G \subseteq E$ and $G$ is open, prove that $G \subseteq E^\circ$.
    \begin{proof}
      Let $g \in G$. Since $g$ is an interior point of $G$, there exists a neighborhood $N$ of $g$
      such that $N \subset G \subset E$. Thus, $g$ is also an interior point of $G$, and the result
      now follows.
    \end{proof}
    \item Prove that the complement of $E^\circ$ is the closure of the complement of $E$.
    \begin{proof}
      Let $p \in (E^\circ)^c$. Then, for all neighborhood $N$ of $p$, $N$ is not a subset of $E$,
      and thus $N$ contains a point in $E^c$, which makes $p$ is limit point of $E^c$. Therefore, $p
      \in \overline{E^c} = (E^c)' \cup E^c$.
    \end{proof}
    \item Do $E$ and $\overline{E}$ always have the same interiors?
    \begin{proof}
      No. Consider $E = \Q$ in $\R$. By the density of $\Q$, any real number is a limit point of
      $\Q$, so $\overline{E} = \R$. However, $\Q$ has no interiors points, but $\R$ is open. Hence,
      $\Q$ and $\R$ do not have the same interiors.
    \end{proof}
    \item Do $E$ and $E^\circ$ always have the same closures?
    \begin{proof}
      No. Consider the set $E = \{\pi\}$ in $\R$. We know that $E$ is closed so $\overline{E} = E$.
      However, $E$ does not contain any interior points, so $E^\circ = \emptyset$. It immediately
      follows that $\emptyset$ is closed, and thus $\overline{E^\circ} = \emptyset \neq
      \overline{E}$.
    \end{proof}
  \end{enumerate}
\end{homeworkProblem}

\newpage

\begin{homeworkProblem}
  Let $K \subseteq \mathbb{R}^1$ consist of $0$ and the numbers $\frac{1}{n}$ for $n =
  1,2,3,\ldots$. Prove that $K$ is compact directly from the definition (without using the
  Heine-Borel theorem).

  \begin{proof}
    Let $\{G_{\alpha}\}$ be an open cover of $K$. Then, $0$ must be in some $G_{\alpha_0}$. Since
    $G_{\alpha_0}$ is an open set, there exists $N_{\epsilon}(0) = (-\epsilon, \epsilon) \subset
    G_{\alpha_0}$. Hence, $(-\epsilon, \epsilon)$ covers all $\frac{1}{k} \in K$ such that $k >
    \frac{1}{\epsilon}$. By the archimedean property, we may find an integer $m >
    \frac{1}{\epsilon}$. For natural number $n < m$, we may find a $G_{\alpha_n}$ that covers
    $\frac{1}{n}$. Thus, with the choice of $m$ indices $\alpha_0, \alpha_1, \ldots, \alpha_{m -
    1}$, $$K \subset G_{\alpha_0} \cup G_{\alpha_0} \cup \ldots \cup G_{\alpha_{m - 1}},$$ which
    proves the compactness of $K$.
  \end{proof}
\end{homeworkProblem}

\newpage

\begin{homeworkProblem}
  Give an example of an open cover of the segment $(0, 1)$ which has no finite subcover.

  \begin{proof}
    Consider $\{G_{\alpha}\}$, where $G_{\alpha} = (0, \alpha)$, for $\alpha \in (0, 1)$.
    $\{G_{\alpha}\}$ is clearly an infinite collection. Let $p \in (0, 1)$ and $q \in (p, 1)$. Since
    there exists $G_{q}$ such that $p \in G_{q}$, $\{G_{\alpha}\}$ is an infinite open cover of $(0,
    1)$. Let $A \subset (0, 1)$ be non-empty and finite. Then, there exists $k = \max A$. Since $k <
    1$, there exists $h \in (k, 1)$ such that $h \notin G_{a} = (0, a)$, for all $a \in A$, as $a <
    k < h < 1$. Hence, $\{G_{\alpha}\}$ has no finite subcover.
  \end{proof}
\end{homeworkProblem}

\newpage

\begin{homeworkProblem}
  A metric space is called \textit{separable} if it contains a countable dense subset. Show that
  $\mathbb{R}^k$ is separable. \textit{Hint:} Consider the set of points which have only rational
  coordinates.

  \begin{proof}
    Consider $\Q^k \subset \R^k$. Note that $\Q$ is countable, so $\Q^k$ is countable, by Theorem
    2.12. Let $S$ be a non-empty open subset of $\R^k$ and let $x = (x_1, x_2, \ldots, x_k) \in S$.
    Then, we may find a ball $B_{\epsilon}(x) \subset S$, for some $\epsilon > 0$. Since $\Q$ is
    dense in $\R$, there exists $r_i \in \Q$ such that $|x_i - r_i| < \frac{\epsilon}{\sqrt{k}}$,
    for all $1 \leq i \leq k$. Let $r = (r_1, r_2, \ldots, r_k)$. Then, we may find a rational
    coordinate $r$ such that
    \[
      \lVert x - r \rVert = \left(\sum_{i = 1}^k (x_i - r_i)^2\right)^{\frac{1}{2}} < \left(\sum_{i = 1}^k \frac{\epsilon^2}{k}\right)^{\frac{1}{2}} = \epsilon.
    \]
    Hence, $r \in S$, and thus $\R^k$ contains a dense set $\Q^k$.
  \end{proof}
\end{homeworkProblem}

\newpage

\begin{homeworkProblem}
  A collection $\{V_{\alpha}\}$ of open subsets of $X$ is said to be a \textit{base} for $X$ if the
  following is true: For every $x \in X$ and every open set $G \subseteq X$ such that $x \in G$, we
  have $x \in V_{\alpha} \subseteq G$ for some $\alpha$. In other words, every open set in $X$ is
  the union of a subcollection of $\{V_{\alpha}\}$. Prove that every separable metric space has a
  \textit{countable base}. \textit{Hint:} Take all neighborhoods with rational radius and center in
  some countable dense subset of $X$.

  \begin{proof}
    Suppose that $X$ is a separable metric space. Let $S \subset X$ be a countable dense subset, and
    let $\{V_{\alpha}\}$ be the collection of neighborhoods $N_r(s)$, for all $s \in S$ and $r \in
    \Q^+$. Since $\{V_{\alpha}\}$ contains neighborhoods of countably many points with countably
    many radii, $\{V_{\alpha}\}$ is countable. Let $x \in X$ and let $G$ be an open set in $X$ such
    that $x \in G$. Then, there exists an open neighborhood $N_{\delta}(x) \subset G$, for some
    $\delta > 0$. Let $\epsilon \in \Q^+$ such that $\epsilon < \delta/2$. Since $S$ is dense in
    $X$, there exists $k \in S$ such that $d(x, k) < \epsilon$. Consider $N_{\epsilon}(k)$. Since
    $d(x, k) < \epsilon$, we already know $x \in N_{\epsilon}(k)$. However, notice that for all $b
    \in N_{\epsilon}(k)$, $d(x, b) \leq d(x, k) + d(k, b) < \frac{\epsilon}{2} + \frac{\epsilon}{2}
    < \delta$, and thus $x \in N_{\epsilon}(k) \subset N_{\delta}(x) \subset G$. It immediately
    follows that $N_{\epsilon}(k) = V_{\alpha}$, for some $\alpha$, so $\{V_{\alpha}\}$ is a
    countable base for $X$.
  \end{proof}
\end{homeworkProblem}

\newpage

\begin{homeworkProblem}
  Let $X$ be a metric space in which every infinite subset has a limit point. Prove that $X$ is
  separable. \textit{Hint:} Fix $\delta > 0$, and pick $x_1 \in X$. Having chosen $x_1, \ldots, x_j
  \in X$, choose $x_{j+1} \in X$, if possible, so that $d(x_i, x_{j+1}) \geq \delta$ for $i = 1,
  \ldots, j$. Show that this process must stop after a finite number of steps, and that $X$ can
  therefore be covered by finitely many neighborhoods of radius $\delta$. Take $\delta = 1/n$ $(n =
  1,2,3,\ldots)$, and consider the centers of the corresponding neighborhoods.

  \begin{proof}
    Suppose that the process does not terminate after a finite number of steps. Then, $\{x_i\}_{i =
    1}^{\infty}$ is an infinite subset of $X$, which means that there exists a limit point $s$ of
    $\{x_i\}_{i = 1}^{\infty}$. But then $N_{\frac{\delta}{2}}(s)$ contains at most $1$ point in
    $\{x_i\}_{i = 1}^{\infty}$, which contradicts the fact that any neighborhoods of a limit point
    contain infinitly many points. Hence, for all $p \in X$, there exists $x_i$ such that $p \in
    N_{\delta}(x_i)$, and thus $\{N_{\delta}(x_i)\}$ is a finite open cover of $X$. Let $S$ be the
    set of all $x_i$ we pick via the process, with $\delta = \frac{1}{n} (n = 1, 2, 3, \ldots)$.
    Since $S$ is a union of countably many finite sets, $S$ is countable. Let $G$ be a non-empty
    open subset of $X$, and let $g \in G$. There exists $\epsilon > 0$ such that $N_{\epsilon}(g)
    \subset G$. By the archimedean property, there exists $k \in \N$ such that $k\epsilon > 1$, and
    thus $\frac{1}{k} < \epsilon$. We know there exists $x_i \in S$ such that $g \in
    N_{\frac{1}{k}}(x_i)$. But then $d(x_i, g) < \frac{1}{k} < \epsilon$, so $x_i \in G$. Hence, $S$
    is a countable dense subset of $X$, and the result now follows.
  \end{proof}
\end{homeworkProblem}

\newpage

\begin{homeworkProblem}
  Prove that every open set in $\mathbb{R}^1$ is the union of an at most countable collection of
  disjoint segments. \textit{Hint:} Use Exercise 2.22.

  \begin{proof}
    From Exercise 2.22, we know $\R^1$ is separable. The proof of Exercise 2.23 gives us a countable
    base $\{V_{\alpha, \beta}\}$ of $\R_1$, where $V_{\alpha, \beta} = (\alpha - \beta, \alpha +
    \beta)$, $\alpha, \beta \in \Q$, $\beta > 0$. In other words, every open set in $\R^1$ is the
    union of a subcollection of $\{V_{\alpha, \beta}\}$, which is a union of an at most countable
    collection of segments. Let $S$ be an open set in $\R^1$, and let $K$ be the subcollection of
    $\{V_{\alpha, \beta}\}$ whose union of all elements is $S$. For each pair of segments $u, v \in
    K$, if $u \cap v \neq \emptyset$, then we replace $u$ and $v$ with segment $u \cup v$. We repeat
    this process until all segments in $K$ are disjoint. We call this new collection $K'$. Since the
    union of every segment in $K'$ remains to be $S$ and $|K'| \leq |K|$, $K'$ is an at most
    countable collection of disjoint segments such that $S = \bigcup_{k \in K'} k$.
  \end{proof}
\end{homeworkProblem}
\end{document}