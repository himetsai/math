\documentclass{article}

\usepackage{fancyhdr}
\usepackage{extramarks}
\usepackage{amsmath}
\usepackage{amsthm}
\usepackage{amsfonts}
\usepackage{tikz}
\usepackage[plain]{algorithm}
\usepackage{algpseudocode}
\usepackage{enumerate}
\usepackage{amssymb}

\usetikzlibrary{automata,positioning}

%
% Basic Document Settings
%

\topmargin=-0.45in
\evensidemargin=0in
\oddsidemargin=0in
\textwidth=6.5in
\textheight=9.0in
\headsep=0.25in

\linespread{1.1}

\pagestyle{fancy}
\lhead{\hmwkAuthorName}
\chead{\hmwkClass:\ \hmwkTitle}
\rhead{\firstxmark}
\lfoot{\lastxmark}
\cfoot{\thepage}

\renewcommand\headrulewidth{0.4pt}
\renewcommand\footrulewidth{0.4pt}

\setlength\parindent{0pt}
\setlength{\parskip}{5pt}

%
% Create Problem Sections
%

\newcommand{\enterProblemHeader}[1]{
    \nobreak\extramarks{}{Problem \arabic{#1} continued on next page\ldots}\nobreak{}
    \nobreak\extramarks{Problem \arabic{#1} (continued)}{Problem \arabic{#1} continued on next page\ldots}\nobreak{}
}

\newcommand{\exitProblemHeader}[1]{
    \nobreak\extramarks{Problem \arabic{#1} (continued)}{Problem \arabic{#1} continued on next page\ldots}\nobreak{}
    \stepcounter{#1}
    \nobreak\extramarks{Problem \arabic{#1}}{}\nobreak{}
}

\setcounter{secnumdepth}{0}
\newcounter{partCounter}
\newcounter{homeworkProblemCounter}
\setcounter{homeworkProblemCounter}{1}
\nobreak\extramarks{Problem \arabic{homeworkProblemCounter}}{}\nobreak{}

%
% Homework Problem Environment
%
% This environment takes an optional argument. When given, it will adjust the
% problem counter. This is useful for when the problems given for your
% assignment aren't sequential. See the last 3 problems of this template for an
% example.
%
\newenvironment{homeworkProblem}[1][-1]{
    \ifnum#1>0
        \setcounter{homeworkProblemCounter}{#1}
    \fi
    \section{Problem \arabic{homeworkProblemCounter}}
    \setcounter{partCounter}{1}
    \enterProblemHeader{homeworkProblemCounter}
}{
    \exitProblemHeader{homeworkProblemCounter}
}

%
% Homework Details
%   - Title
%   - Due date
%   - Class
%   - Section/Time
%   - Instructor
%   - Author
%

\newcommand{\hmwkTitle}{Homework\ \#1}
\newcommand{\hmwkDueDate}{Jan 19, 2023}
\newcommand{\hmwkClass}{MATH 140A}
\newcommand{\hmwkClassInstructor}{Professor Seward}
\newcommand{\hmwkAuthorName}{\textbf{Ray Tsai}}
\newcommand{\hmwkPID}{A16848188}

%
% Title Page
%

\title{
    \vspace{2in}
    \textmd{\textbf{\hmwkClass:\ \hmwkTitle}}\\
    \normalsize\vspace{0.1in}\small{Due\ on\ \hmwkDueDate\ at 23:59pm}\\
    \vspace{0.1in}\large{\textit{\hmwkClassInstructor}} \\
    \vspace{3in}
}

\author{
  \hmwkAuthorName \\
  \vspace{0.1in}\small\hmwkPID
}
\date{}

\renewcommand{\part}[1]{\textbf{\large Part \Alph{partCounter}}\stepcounter{partCounter}\\}

%
% Various Helper Commands
%

% Useful for algorithms
\newcommand{\alg}[1]{\textsc{\bfseries \footnotesize #1}}

% For derivatives
\newcommand{\deriv}[1]{\frac{\mathrm{d}}{\mathrm{d}x} (#1)}

% For partial derivatives
\newcommand{\pderiv}[2]{\frac{\partial}{\partial #1} (#2)}

% Integral dx
\newcommand{\dx}{\mathrm{d}x}

% Probability commands: Expectation, Variance, Covariance, Bias
\newcommand{\Var}{\mathrm{Var}}
\newcommand{\Cov}{\mathrm{Cov}}
\newcommand{\Bias}{\mathrm{Bias}}
\newcommand*{\Z}{\mathbb{Z}}
\newcommand*{\Q}{\mathbb{Q}}
\newcommand*{\R}{\mathbb{R}}
\newcommand*{\C}{\mathbb{C}}
\newcommand*{\N}{\mathbb{N}}
\newcommand*{\prob}{\mathds{P}}
\newcommand*{\E}{\mathds{E}}

\begin{document}

\maketitle

\pagebreak

\begin{homeworkProblem}
    If $r$ is rational $(r \neq 0)$ and $x$ is irrational, prove that $r + x$ and $rx$ are
    irrational.

    \begin{proof}
      Suppose for the sake of contradiction that $y = r + x$, $z = rx$ are rational. Since rational
      numbers are closed under addition, $x = y - r$ is also rational, contradiction. Similarly,
      since non-zero rational is closed under multiplication and taking multiplicative inverse, $x =
      \frac{z}{r}$ is rational, contradiction. Hence, $r + x$ and $rx$ are irrational.
    \end{proof}
\end{homeworkProblem}

\newpage

\begin{homeworkProblem}
  Let $E$ be a nonempty subset of an ordered set; suppose $\alpha$ is a lower bound of $E$ and
  $\beta$ is an upper bound of $E$. Prove that $\alpha \leq \beta$.

  \begin{proof}
    Since $\beta \geq e$ and $e \geq \alpha$ for $e \in E$, $\beta \geq \alpha$.
  \end{proof}
\end{homeworkProblem}

\newpage

\begin{homeworkProblem}
  Let $A$ be a nonempty set of real numbers which is bounded below. Let $-A$ be the set of all
  numbers $-x$, where $x \in A$. Prove that
    \[
        \inf A = -\sup(-A).
    \]

    \begin{proof}
      Let $y = \inf A$. Since $y \leq x$ for $x \in A$, we know $-y \geq -x$ for $-x \in -A$, so
      $-y$ is an upper bound of $-A$. Since $y$ is the greatest lower bound of $A$, there exists $a
      \in A$ such that $a < y + \epsilon$, for $\epsilon > 0$. This immediately follows that for
      $\epsilon > 0$, there exists $-a \in -A$ such that $-a > -y - \epsilon$, so $-y$ is the least
      upper bound of $-A$. In other words, $y = -\sup(-A)$.
    \end{proof}
\end{homeworkProblem}

\newpage

\begin{homeworkProblem}
  Fix $b > 1$.
  \begin{enumerate}[(a)]
    \item If $m, n, p, q$ are integers, $n > 0$, $q > 0$, and $r = \frac{m}{n} = \frac{p}{q}$, prove
    that
    \[
      (b^m)^{1/n} = (b^p)^{1/q}.
    \]
    Hence it makes sense to define $b^r = (b^m)^{1/n}$.
    \begin{proof}
      Let $x = (b^m)^{1/n}$, $y = (b^p)^{1/q}$. We know $x^n = b^m = b^{nr}$ and $y^q = b^p =
      b^{qr}$. Consider $x^{nq}$. Since $x^{nq} = b^{nrq} = y^{nq}$, we conclude that $x = y$, by
      Theorem 1.21.
    \end{proof}

    \item Prove that $b^{r + s} = b^rb^s$ if $r$ and $s$ are rational.
    \begin{proof}
      Let $r = \frac{m}{n}, s = \frac{p}{q}$, for $m, n, p, q \in \Z$ and $n, q > 0$. Consider
      $(b^{r + s})^{nq}$ and $(b^rb^s)^{nq}$. Since $(b^{r + s})^{nq} = b^{mq + np}$ and
      $(b^rb^s)^{nq} = (b^r)^{nq}(b^s)^{nq} = b^{mq}b^{np} = b^{mq + np}$, we know $(b^{r + s})^{nq}
      = (b^rb^s)^{nq}$. The result now follows by Theorem 1.21.
    \end{proof}

    \item If $x$ is real, define $B(x)$ to be the set of all numbers $b^t$, where $t$ is rational
    and $t < x$. Prove that
    \[
      b^r = \sup B(r).
    \]
    when $r$ is rational. Hence it makes sense to define
    \[
      b^x = \sup B(x).
    \]
    for every real $x$.
    \begin{proof}
      Let $b^t \in B(r)$. Suppose $r = \frac{m}{n}$, $t = \frac{p}{q}$, where $n, q > 0$. Since $r >
      t$, $mq > np$. Consider $(b^r)^{nq}$ and $(b^t)^{nq}$. We know $(b^r)^{nq} = b^{mq} > b^{np} =
      (b^t)^{nq}$. Let $y = b^r$, $x = b^t$. Since $y^{nq - 1} + y^{nq - 2}x + \dots + x^{nq - 1} >
      0$, the identity $y^{nq} - x^{nq} = (y - x)(y^{nq - 1} + y^{nq - 2}x + \dots + x^{nq - 1}) >
      0$ yields $y > x$, and thus $y = b^r$ is an upper bound of $B(r)$. 

      Let $y < b^r$ be a positive real number. We show that there exists $b^s \in B(r)$ such that
      $b^s > y$. Let $t = y^{-1}b^r$, and let $n \in \Z$ such that $n > \frac{b - 1}{t - 1}$. We
      know such $n$ exists by the archimedean property and $t > 1$. Since $a^n - 1 = (a - 1)(a^{n -
      1} + a^{n - 2} + \dots + 1) > n(a - 1)$ for $a > 1$, we know $b - 1 > n(b^{1/n} - 1) > \frac{b
      - 1}{t - 1}(b^{1/n} - 1)$. This immediately follows that $t = y^{-1}b^r > b^{1/n}$, so $b^{r -
      (1/n)} > y$. However, since $r - (1/n) \in \Q$, we have $b^{r - (1/n)} \in B(r)$. Therefore,
      $y$ is not an upper bound of $B(r)$, which shows that $b^r = \sup B(r)$.
    \end{proof}

    \item Prove that $b^{x + y} = b^xb^y$ for all real $x$ and $y$.
    \begin{proof}
      We show that $b^xb^y = \sup B(x + y)$. Note that $B(x + y) = \{b^{t} \, | \, t < x + y, t \in
      \Q\} = \{b^{g + h} \, | \, g < x, h < y, g, h \in \Q\} = \{b^{g}b^{h} \, | \, g < x, h < y, g,
      h \in \Q\}$. Let $b^{g}b^{h} \in B(x + y)$, for $g < x, h < y$. Since $b^g \in B(x)$ and $b^h
      \in B(y)$, we know $b^x > b^g$ and $b^y > b^h$. Thus $b^{x}b^{y} > b^{g}b^{h}$, so
      $b^{x}b^{y}$ is an upper bound of $B(x + y)$. Suppose that $k < b^{x}b^{y}$. Since
      $\frac{k}{b^y} < b^{x}$, there exists $b^l \in B(x)$ such that $\frac{k}{b^y} < b^l$. However,
      $\frac{k}{b^l} < b^y$, so there exists $b^s \in B(y)$ such that $\frac{k}{b^l} < b^s$. This
      immediately follows that there exists $b^lb^s \in B(x + y)$ such that $k < b^lb^s$, so
      $b^xb^y$ is the least upper bound of $B(x + y)$.
    \end{proof}
  \end{enumerate}
\end{homeworkProblem}

\newpage

\begin{homeworkProblem}
  Fix $b > 1, y > 0$, and prove that there is a unique real $x$ such that $b^x = y$.

  \begin{proof}
    Since $b > 1$, we know $b^{k}> 1$ for positive integer $k$, so the identity $b^n - 1 = (b -
    1)(b^{n - 1} + b^{n - 2} + \dots + 1)$ yields
    \begin{gather}
      b^{n} - 1 < n(b - 1),
    \end{gather}
    for positive integer $n$. Note that $b^{1/n} > 1$, otherwise $b = (b^{1/n})^n \leq 1$. Hence, we
    may apply (1) and get
    \begin{gather}
      b - 1 > n(b^{1/n} - 1).
    \end{gather}
    Suppose that $b^{w} < y$. Let $t = yb^{-w}$, and let $n > \frac{b - 1}{t - 1}$, for $w \in \R$.
    We know such $n$ exists by the Archimedean Property and $t = yb^{-w} > b^wb^{-w} = 1$. By (2),
    $b - 1 > n(b^{1/n} - 1) > \frac{b - 1}{t - 1}(b^{1/n} - 1)$, so $t > b^{1/n}$. Thus, $b^{w + (1
    / n)} < y$ for sufficiently large $n$. Similarly, when $b^{w} > y$, take $t = y^{-1}b^{w}$.
    Again, let $n > \frac{b - 1}{t - 1}$, and it follows by (2) that $t > b^{1/n}$. Thus, $y < b^{w
    - (1 / n)}$ for sufficient large $n$. 

    Let $A$ be the set of all $w$ such that $b^w < y$. We show that $x = \sup A$ satisfies $b^x =
    y$. Suppose for the sake of contradiction that $b^x > y$. Then, we know there exists $n$ such
    that $b^{x - (1 / n)} > y$. However, since $x = \sup A$, $b^{x - (1 / n)}$ must be in $A$, which
    contradicts that $b^{x - (1 / n)} > y$. Suppose for the sake of contradiction that $b^x < y$.
    There exists $n$ such that $b^{x + (1 / n)} < y$. However, this means that there exists $b^{x +
    (1 / n)} \in A$ such that $b^{x + (1 / n)} > x = \sup A$, contradiction. Therefore, $b^x = y$.
    
    Suppose that $b^{x} = b^{x'} = y$. $x$ cannot be greater or lesser than $x'$, otherwise
    $b^{x} \neq b^{x'}$, so $x$ is unique.
  \end{proof}
\end{homeworkProblem}
\end{document}