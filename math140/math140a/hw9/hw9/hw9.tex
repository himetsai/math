\documentclass{article}

\usepackage{fancyhdr}
\usepackage{extramarks}
\usepackage{amsmath}
\usepackage{amsthm}
\usepackage{amsfonts}
\usepackage{tikz}
\usepackage[plain]{algorithm}
\usepackage{algpseudocode}
\usepackage{enumerate}
\usepackage{amssymb}

\usetikzlibrary{automata,positioning}

%
% Basic Document Settings
%

\topmargin=-0.45in
\evensidemargin=0in
\oddsidemargin=0in
\textwidth=6.5in
\textheight=9.0in
\headsep=0.25in

\linespread{1.1}

\pagestyle{fancy}
\lhead{\hmwkAuthorName}
\chead{\hmwkClass:\ \hmwkTitle}
\rhead{\firstxmark}
\lfoot{\lastxmark}
\cfoot{\thepage}

\renewcommand\headrulewidth{0.4pt}
\renewcommand\footrulewidth{0.4pt}

\setlength\parindent{0pt}
\setlength{\parskip}{5pt}

%
% Create Problem Sections
%

\newcommand{\enterProblemHeader}[1]{
    \nobreak\extramarks{}{Problem \arabic{#1} continued on next page\ldots}\nobreak{}
    \nobreak\extramarks{Problem \arabic{#1} (continued)}{Problem \arabic{#1} continued on next page\ldots}\nobreak{}
}

\newcommand{\exitProblemHeader}[1]{
    \nobreak\extramarks{Problem \arabic{#1} (continued)}{Problem \arabic{#1} continued on next page\ldots}\nobreak{}
    \stepcounter{#1}
    \nobreak\extramarks{Problem \arabic{#1}}{}\nobreak{}
}

\setcounter{secnumdepth}{0}
\newcounter{partCounter}
\newcounter{homeworkProblemCounter}
\setcounter{homeworkProblemCounter}{1}
\nobreak\extramarks{Problem \arabic{homeworkProblemCounter}}{}\nobreak{}

%
% Homework Problem Environment
%
% This environment takes an optional argument. When given, it will adjust the
% problem counter. This is useful for when the problems given for your
% assignment aren't sequential. See the last 3 problems of this template for an
% example.
%
\newenvironment{homeworkProblem}[1][-1]{
    \ifnum#1>0
        \setcounter{homeworkProblemCounter}{#1}
    \fi
    \section{Problem \arabic{homeworkProblemCounter}}
    \setcounter{partCounter}{1}
    \enterProblemHeader{homeworkProblemCounter}
}{
    \exitProblemHeader{homeworkProblemCounter}
}

%
% Homework Details
%   - Title
%   - Due date
%   - Class
%   - Section/Time
%   - Instructor
%   - Author
%

\newcommand{\hmwkTitle}{Homework\ \#9}
\newcommand{\hmwkDueDate}{Jan 19, 2023}
\newcommand{\hmwkClass}{MATH 140A}
\newcommand{\hmwkClassInstructor}{Professor Seward}
\newcommand{\hmwkAuthorName}{\textbf{Ray Tsai}}
\newcommand{\hmwkPID}{A16848188}

%
% Title Page
%

\title{
    \vspace{2in}
    \textmd{\textbf{\hmwkClass:\ \hmwkTitle}}\\
    \normalsize\vspace{0.1in}\small{Due\ on\ \hmwkDueDate\ at 23:59pm}\\
    \vspace{0.1in}\large{\textit{\hmwkClassInstructor}} \\
    \vspace{3in}
}

\author{
  \hmwkAuthorName \\
  \vspace{0.1in}\small\hmwkPID
}
\date{}

\renewcommand{\part}[1]{\textbf{\large Part \Alph{partCounter}}\stepcounter{partCounter}\\}

%
% Various Helper Commands
%

% Useful for algorithms
\newcommand{\alg}[1]{\textsc{\bfseries \footnotesize #1}}

% For derivatives
\newcommand{\deriv}[1]{\frac{\mathrm{d}}{\mathrm{d}x} (#1)}

% For partial derivatives
\newcommand{\pderiv}[2]{\frac{\partial}{\partial #1} (#2)}

% Integral dx
\newcommand{\dx}{\mathrm{d}x}

% Probability commands: Expectation, Variance, Covariance, Bias
\newcommand{\Var}{\mathrm{Var}}
\newcommand{\Cov}{\mathrm{Cov}}
\newcommand{\Bias}{\mathrm{Bias}}
\newcommand*{\Z}{\mathbb{Z}}
\newcommand*{\Q}{\mathbb{Q}}
\newcommand*{\R}{\mathbb{R}}
\newcommand*{\C}{\mathbb{C}}
\newcommand*{\N}{\mathbb{N}}
\newcommand*{\prob}{\mathds{P}}
\newcommand*{\E}{\mathds{E}}

\begin{document}

\maketitle

\pagebreak

\begin{homeworkProblem}
	Suppose $f$ is a real function defined on $\R^1$ which satisfies 
	\[
		\lim_{h \to 0} [f(x + h) - f(x - h)] = 0
	\]
	for every $x \in \R^1$. Does this imply that $f$ is continuous?

	\begin{proof}
		No. Consider $f(x) = \begin{cases}
			1 & ,x = 0 \\
			0 & ,x \neq 0
		\end{cases}$. Since $f(x)$ is constant at all points other than $x = 0$, the condition stated
		above holds for $f$. But then $\lim_{x \to 0} f(x) = 0 \neq 1 = f(0)$, so $f$ is not continuous.
	\end{proof}
\end{homeworkProblem}

\newpage

\begin{homeworkProblem}
	If $f$ is a continuous mapping of a metric space $X$ into a metric space $Y$, prove that
	\[
		f(\overline{E}) \subseteq \overline{f(E)}
	\]
	for every set $E \subseteq X$. ($\overline{E}$ denotes the closure of $E$.) Show, by an example,
	that $f(\overline{E})$ can be a proper subset of $\overline{f(E)}$.

	\begin{proof}
		Note that $f(E) \subseteq \overline{f(E)}$, so it suffices to show $f(E') \subseteq
		\overline{f(E)}$. Let $p \in E'$. Since $f$ is continuous, for all $n \in \N$, there exists
		$\delta_n > 0$ such that $d_Y(f(p), f(q)) < 1/n$ for all $d_X(p, q) < \delta_n$. But then $p$ is
		a limit point of $E$, so we may pick $q_n \in E \cap N_{\delta_n}(p) \backslash \{p\}$ for each
		$n \in \N$. Now, for arbitrary $\epsilon > 0$, we may find $f(q_n) \in f(E)$, such that
		$d_Y(f(q_n), f(p)) < 1/n < \epsilon$, for large enough $n$. Hence, $f(p)$ is a limit point of
		$f(E)$, and the result follows.

		Consider $X = \Q$ and $Y = \R$. Take $f: \Q \hookrightarrow \R$ to be the natural inclusion.
		Since $\Q$ is closed in itself but $\overline{\Q} = \R$ in $\R$, $f(\overline{\Q}) = f(\Q) = \Q
		\subset \R$.
	\end{proof}
\end{homeworkProblem}

\newpage

\begin{homeworkProblem}
	Let $f$ be a continuous real function on a metric space $X$. Let $Z(f)$ (the zero set of $f$) be
	the set of all $p \in X$ at which $f(p) = 0$. Prove that $Z(f)$ is closed.

	\begin{proof}
		Since $\{0\}$ is closed in $\R$, $Z(f) = f^{-1}(0)$ is closed in $X$, by Theorem 4.8.
	\end{proof}
\end{homeworkProblem}

\newpage

\begin{homeworkProblem}
	Let $f$ and $g$ be continuous mappings of a metric space $X$ into a metric space $Y$, and let $E$
	be a dense subset of $X$. Prove that $f(E)$ is dense in $f(X)$. If $g(p) = f(p)$ for all $p \in
	E$, prove that $g(p) = f(p)$ for all $p \in X$ (In other words, a continuous function is
	determined by its values on a dense subset of its domain).

	\begin{proof}
		Let $x \in X$. Fix $\epsilon > 0$. There exists $\delta > 0$ such that $d_Y(f(x), f(y)) <
		\epsilon$, for any $y \in X$ with $d_X(x, y) < \delta$. Since $E$ is dense in $X$, there exists
		$e \in E$ such that $d_X(x, e) < \delta$. But then $d_Y(f(x), f(e)) < \epsilon$. Since
		$\epsilon$ was arbitrary, $f(E)$ is dense in $f(X)$.
		
		We now prove the second question. Let $p \in X$. Since $E$ is dense in $X$, pick $p_n \in E \cap
		(N_{\frac{1}{n}} \backslash \{p\})$, for each $n \in \N$. But then
		\[
			f(p) = \lim_{n \to \infty} f(p_n) = \lim_{n \to \infty} g(p_n) = g(p).
		\]
	\end{proof}
\end{homeworkProblem}

\newpage

\begin{homeworkProblem}
	If $f$ is a real continuous function defined on a closed set $E \subseteq \R^1$, prove that there
	exist continuous real functions on $\R^1$ such that $g(x) = f(x)$ for all $x \in E$. (Such
	functions $g$ are called continuous extensions of $f$ from $E$ to $\R^1$.) 

	\begin{proof}
		By exercise 2.29, $E^c$ is a union of countably many disjoint segments, say $E^c = \bigcup_{n =
		1} (a_n, b_n)$, where $a_i < b_i \leq a_{i + 1}$ for all $i$. Define $g_n: (a_n, b_n) \to \R$ as
		\[
			g_n(x) \mapsto \begin{cases}
				f(b_n), &a_n = -\infty \\
				f(a_n), &b_n = \infty \\
				f(a_n) + \frac{f(b_n) - f(a_n)}{b_n - a_n} \cdot (x - a_n), & \text{otherwise}
			\end{cases}.
		\]
		If we plotted out the two dimension graph of $g$, the image of each segment $(a_i, b_i)$ is a
		straight line which connects $(a_i, f(a_i))$ and $(b_i, f(b_i))$. Note that we $g_n$ is
		continuous, as it is either constant or a linear polynomial. Now define $g: \R \to \R$ as
		\[
			g(x) \mapsto \begin{cases}
				g_n(x), & x \in (a_n, b_n) \\
				f(x), & \text{otherwise}
			\end{cases}.
		\]
		Obvisouly, $g(x) = f(x)$ for all $x \in E$. Fix $\epsilon > 0$, and let $p \in \R$. Suppose $p
		\notin E$. Since $E^c$ is open, $N_r(p) \subseteq E^c$ for some $r > 0$. Since $g_n(p)$ is
		continuous, there exists $\delta \in (0, r)$, such that $|g_n(p) - g_n(q)| < \epsilon$ for all
		$q \in N_{\delta}(p)$. But then $q \in E^c$, so $g(q) = g_n(q)$, and thus $g$ is continuous at
		$p$. 
		
		Now suppose $p \in E$. Since $f$ is continuous, there eixsts $\delta > 0$ such that $|g(p) -
		g(q)| = |f(p) - f(q)| < \epsilon$, for all $q \in (p - \delta, p + \delta) \cap E$. We may
		assume $(p - \delta, p + \delta)$ intersects with some $(a_i, b_i)$, otherwise we are done. 
		
		Suppose some $(a_i, b_i) \subset (p - \delta, p + \delta)$. Since $a_i, b_i \in E$ and, by
		construction, $\min(f(a_i), f(b_i)) \leq f(q) \leq \max(f(a_i), f(b_i))$ for $q \in (a_i, b_i)$,
		we have $|g(p) - g(q)| < \epsilon$ for $q \in (a_i, b_i)$. 

		Hence, we only have to care about the case where the ends of $(p - \delta, p + \delta)$ overlap
		with other segments. Suppose $(p - \delta, p + \delta)$ partially intersects with the bottom
		part of some $(a_i, b_i)$, then we shrunk the segment to $(p - \delta, p + \delta_a)$, with $a_i
		= p + \delta_a$. Similarly, suppose $(p - \delta, p + \delta)$ partially intersects with the top
		part of some $(a_i, b_i)$, then we shrunk the segment to $(p - \delta_b, p + \delta)$, with
		$b_i = p + \delta_b$. 
		
		Thus, if $\delta_a, \delta_b > 0$, we would end up with some neighborhood $N_p = (p - \delta_b,
		p + \delta_a)$, where $\delta_a, \delta_b \leq \delta$, such that $|g(p) - g(q)| < \epsilon$ for
		all $q \in N_p$.

		Suppose $\delta_a = 0$. Then $p = a_i$, for some $i$. But then $\lim_{x \to a_i} g_i(x) =
		g_i(a_i)$. Similarly, suppose $\delta_b = 0$. Then $p = b_i$, for some $i$. But then $\lim_{x
		\to b_i} g_i(x) = g_i(b_i)$. Hence, in either case, we may still find some $\delta_a', \delta_b'
		> 0$, such that $|g(p) - g(q)| < \epsilon$ for $q \in (p - \delta_b', p + \delta_a')$, and thus
		$g(p)$ is continuous at $p$.
	\end{proof}
\end{homeworkProblem}

\newpage

\begin{homeworkProblem}
	If $f$ is defined on $E$, the graph of $f$ is the set of points $(x, f(x))$, for $x \in E$. In
	particular, if $E$ is a set of real numbers and $f$ is real-valued, the graph of $f$ is a subset
	of the plane. Suppose $E$ is compact and prove that $f$ is continuous on $E$ if and only if its
	graph is compact.

	\begin{proof}
		Suppose $f: E \to Y$ is continuous. Define $\phi: E \to E \times Y$ that sends $x$ to $(x,
		f(x))$. Fix $\epsilon > 0$. There exists $\nu > 0$ such that $d_Y(f(x), f(y)) < \epsilon/2$ for
		$y \in N_{\nu}(x)$. Take $\delta = \min(\nu, \epsilon/2)$. We then have
		\[
			d(\phi(x), \phi(y)) = d_E(x, y) + d_Y(f(x), f(y)) < \min(\nu, \epsilon/2) + \epsilon/2 \leq \epsilon,
		\]
		for all $y \in E$ with $d_E(x, y) < \delta$. Since $\phi$ is a continuous mapping of a compact
		metrix space, $\phi(E)$ is compact, by By Theorem 4.14.

		We now show the converse. Since $d(\phi(x), \phi(y)) \geq d_Y(f(x), f(y))$, it is obvious that
		$f$ is continuous if $\phi$ is continuous, as the $\delta > 0$ that can be used to show the
		continui of $\phi$ also applies to show that of $f$. Hence, suppose $f$ is not continuous at
		some point $p$, then $\phi$ is not continuous at $p$. Then, there exists a sequence $p_n$ that
		converges to $p$ while $\phi(p_n)$ does not converge to $\phi(p)$. We may assume that some
		subsequence $\phi(p_{n_i})$ converges, otherwise $\phi(E)$ is not compact, by Theorem 3.6. But
		then $\phi(p_{n_i})$ converges to some point $k$, where $f(p) \neq k$. Since $f$ is
		well-defined, $(p, k) \neq (p, f(p))$ is a limit point not contained in $\phi(E)$. Hence,
		$\phi(E)$ is not closed, and thus it is not compact, by Theorem 2.34.
	\end{proof}
\end{homeworkProblem}

\newpage

\begin{homeworkProblem}
	Let $f$ be a real uniformly continuous function on the bounded set $E$ in $\R^1$. Prove that $f$
	is bounded on $E$. Show that the conclusion is false if the boundedness of $E$ is omitted from the
	hypothesis.

	\begin{proof}
		Suppose for the sake of contradiction that $f$ is unbounded on $E$. Then, we may pick $p_n$ such
		that $|f(p_n)| > n$, for each $n \in \N$. Note that $f(p_n) \to \infty$. But $p_n$ is a sequence
		in a bounded set $E$, so there exists a convergent subsequence $p_{n_i}$, by Theorem 3.6. Fix
		$\epsilon > 0$. Since $f$ is uniformly continuous, there exists $\delta > 0$ such that $|f(x) -
		f(y)| < \epsilon$ for all $|x - y| < \delta$. Since $p_{n_i}$ is a Cauchy sequence by Theorem
		3.11, there exists integer $N$ such that $|p_{n_i} - p_{n_j}| < \delta$ for $i, j \geq N$ and
		thus $|f(p_{n_i}) - f(p_{n_j})| < \epsilon$. Hence, $|f(p_{n_i})| \leq |f(p_{n_i}) -
		f(p_{n_j})| + |f(p_{n_j})| < \epsilon + |f(p_{n_j})|$. Fixing $j$, we have $\lim_{i \to \infty}
		|f(p_{n_i})| \leq \epsilon + |f(p_{n_j})|$, contradicting our choice of $p_n$.

		Define $f: \R \to \R$ as $f(x) = x$. For $\epsilon > 0$, there exists $\delta = \epsilon$ such
		that $|f(x) - f(y)| = |x - y| < \epsilon$ for all $|x - y| < \delta$. It follows that $f$ is
		uniformly continuous but unbounded, so the conclusion is false if $E$ is not bounded.
	\end{proof}
\end{homeworkProblem}
\end{document}