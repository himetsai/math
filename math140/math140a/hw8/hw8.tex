\documentclass{article}

\usepackage{fancyhdr}
\usepackage{extramarks}
\usepackage{amsmath}
\usepackage{amsthm}
\usepackage{amsfonts}
\usepackage{tikz}
\usepackage[plain]{algorithm}
\usepackage{algpseudocode}
\usepackage{enumerate}
\usepackage{amssymb}

\usetikzlibrary{automata,positioning}

%
% Basic Document Settings
%

\topmargin=-0.45in
\evensidemargin=0in
\oddsidemargin=0in
\textwidth=6.5in
\textheight=9.0in
\headsep=0.25in

\linespread{1.1}

\pagestyle{fancy}
\lhead{\hmwkAuthorName}
\chead{\hmwkClass:\ \hmwkTitle}
\rhead{\firstxmark}
\lfoot{\lastxmark}
\cfoot{\thepage}

\renewcommand\headrulewidth{0.4pt}
\renewcommand\footrulewidth{0.4pt}

\setlength\parindent{0pt}
\setlength{\parskip}{5pt}

%
% Create Problem Sections
%

\newcommand{\enterProblemHeader}[1]{
    \nobreak\extramarks{}{Problem \arabic{#1} continued on next page\ldots}\nobreak{}
    \nobreak\extramarks{Problem \arabic{#1} (continued)}{Problem \arabic{#1} continued on next page\ldots}\nobreak{}
}

\newcommand{\exitProblemHeader}[1]{
    \nobreak\extramarks{Problem \arabic{#1} (continued)}{Problem \arabic{#1} continued on next page\ldots}\nobreak{}
    \stepcounter{#1}
    \nobreak\extramarks{Problem \arabic{#1}}{}\nobreak{}
}

\setcounter{secnumdepth}{0}
\newcounter{partCounter}
\newcounter{homeworkProblemCounter}
\setcounter{homeworkProblemCounter}{1}
\nobreak\extramarks{Problem \arabic{homeworkProblemCounter}}{}\nobreak{}

%
% Homework Problem Environment
%
% This environment takes an optional argument. When given, it will adjust the
% problem counter. This is useful for when the problems given for your
% assignment aren't sequential. See the last 3 problems of this template for an
% example.
%
\newenvironment{homeworkProblem}[1][-1]{
    \ifnum#1>0
        \setcounter{homeworkProblemCounter}{#1}
    \fi
    \section{Problem \arabic{homeworkProblemCounter}}
    \setcounter{partCounter}{1}
    \enterProblemHeader{homeworkProblemCounter}
}{
    \exitProblemHeader{homeworkProblemCounter}
}

%
% Homework Details
%   - Title
%   - Due date
%   - Class
%   - Section/Time
%   - Instructor
%   - Author
%

\newcommand{\hmwkTitle}{Homework\ \#8}
\newcommand{\hmwkDueDate}{Mar 8, 2024}
\newcommand{\hmwkClass}{MATH 140A}
\newcommand{\hmwkClassInstructor}{Professor Seward}
\newcommand{\hmwkAuthorName}{\textbf{Ray Tsai}}
\newcommand{\hmwkPID}{A16848188}

%
% Title Page
%

\title{
    \vspace{2in}
    \textmd{\textbf{\hmwkClass:\ \hmwkTitle}}\\
    \normalsize\vspace{0.1in}\small{Due\ on\ \hmwkDueDate\ at 23:59pm}\\
    \vspace{0.1in}\large{\textit{\hmwkClassInstructor}} \\
    \vspace{3in}
}

\author{
  \hmwkAuthorName \\
  \vspace{0.1in}\small\hmwkPID
}
\date{}

\renewcommand{\part}[1]{\textbf{\large Part \Alph{partCounter}}\stepcounter{partCounter}\\}

%
% Various Helper Commands
%

% Useful for algorithms
\newcommand{\alg}[1]{\textsc{\bfseries \footnotesize #1}}

% For derivatives
\newcommand{\deriv}[1]{\frac{\mathrm{d}}{\mathrm{d}x} (#1)}

% For partial derivatives
\newcommand{\pderiv}[2]{\frac{\partial}{\partial #1} (#2)}

% Integral dx
\newcommand{\dx}{\mathrm{d}x}

% Probability commands: Expectation, Variance, Covariance, Bias
\newcommand{\Var}{\mathrm{Var}}
\newcommand{\Cov}{\mathrm{Cov}}
\newcommand{\Bias}{\mathrm{Bias}}
\newcommand*{\Z}{\mathbb{Z}}
\newcommand*{\Q}{\mathbb{Q}}
\newcommand*{\R}{\mathbb{R}}
\newcommand*{\C}{\mathbb{C}}
\newcommand*{\N}{\mathbb{N}}
\newcommand*{\prob}{\mathds{P}}
\newcommand*{\E}{\mathds{E}}

\begin{document}

\maketitle

\pagebreak

\begin{homeworkProblem}
  Suppose $a_n > 0$, $s_n = a_1 + \cdots + a_n$, and $\sum a_n$ diverges.
  \begin{enumerate}[(a)]
      \item Prove that $\sum \frac{a_n}{(1 + a_n)}$ diverges.
      \begin{proof}
        Note that if $a_n > 1$, then $\frac{a_n}{a_n + 1} = 1 - \frac{1}{a_n + 1} > \frac{1}{2}$. On
        the other hand, if $a_n \leq 1$, we have $\frac{a_n}{a_n + 1} \geq \frac{a_n}{2}$. If there
        are infinitely many $n$ such that $a_n > 1$, then the series obviously diverges, as it would
        be greater than the sum of infinitely many $\frac{1}{2}$. Hence, we may assume there exists
        $N \geq 0$ such that $a_n \leq 1$ for all $n \geq N$. But then
        \[
          \sum \frac{a_n}{(1 + a_n)} \geq \sum_{n = 1}^{N - 1} \frac{a_n}{(1 + a_n)} + \frac{1}{2}\sum_{n = N}^{\infty} a_n.
        \]
        Since $\sum a_n$ diverges, $\frac{1}{2}\sum_{n = N}^{\infty} a_n$ diverges, by comparison
        test. The result now follows.
      \end{proof}
      \item Prove that 
      \[
        \frac{a_{N+1}}{s_{N+1}} + \cdots + \frac{a_{N+k}}{s_{N+k}} \geq 1 - \frac{s_N}{s_{N+k}}
      \]
      and deduce that $\sum \frac{a_n}{s_n}$ diverges.
      \begin{proof}
        We first note that 
        \[
          \frac{a_{N+1}}{s_{N+1}} + \cdots + \frac{a_{N+k}}{s_{N+k}} \geq \frac{a_{N + 1} + \dots + a_{N + k}}{s_{N + k}} =  1 - \frac{s_N}{s_{N+k}}.
        \]
        Fix $\epsilon \in (0, 1)$. Since $S_n$ is increasing and unbounded, $\frac{s_N}{s_{N+k}} \to
        0$. Hence, we may find large enough $k$ such that $\frac{s_N}{s_{N+k}} < 1 - \epsilon$. But
        then $\sum_{n = N + 1}^{N + k} \frac{a_n}{s_n} \geq \epsilon$, which fails to meet the
        Cauchy criterion.
      \end{proof}
      \item Prove that 
      \[
        \frac{a_n}{s_n^2} \leq \frac{1}{s_{n-1}} - \frac{1}{s_n}
      \]
      and deduce that $\sum \frac{a_n}{s_n^2}$ converges.
      \begin{proof}
        Since
        \[
          \frac{a_n}{s_n^2} \leq \frac{a_n}{s_{n - 1}s_n} = \frac{1}{s_{n-1}} - \frac{1}{s_n},
        \]
        the consecutive terms cancel out, and we get $\sum_{n = 1}^N \frac{a_n}{s_n^2} \leq \sum_{n
        = 2}^N \frac{1}{s_{n-1}} - \frac{1}{s_n} = \frac{1}{a_1} - \frac{1}{s_N}$. But then $s_n$ is
        increasing and unbounded, and thus
        \[
          \frac{1}{a_1} \leq \lim_{N \to \infty} \sum \frac{a_n}{s_n^2} \leq \lim_{N \to \infty} \frac{1}{a_1} - \frac{1}{s_N} = \frac{1}{a_1}.
        \]
        Hence, the series converges to $\frac{1}{a_1}$.
      \end{proof}
  \end{enumerate}
\end{homeworkProblem}

\newpage

\begin{homeworkProblem}
  Suppose $a_n > 0$ and $\sum a_n$ converges. Put 
    \[
    r_n = \sum_{m=n}^{\infty} a_m.
    \]
  \begin{enumerate}[(a)]
    \item Prove that 
    \[
      \frac{a_m}{r_m} + \dots + \frac{a_n}{r_n} > 1 - \frac{r_n}{r_m}
    \]
    if $m < n$, and deduce that $\sum \frac{a_n}{r_n}$ diverges.
    \begin{proof}
      Let $A = \sum_{n = 1}^{\infty} a_n$. We know $r_n = A - s_n$, where $s_n$ is the sum of the
      first $n - 1$ terms of $a_n$. Note that for $n > m$, we have $r_n < r_m$, as $s_{n} > s_m$.
      Hence,
      \[
        \frac{a_m}{r_m} + \dots + \frac{a_n}{r_n} > \frac{a_m + \dots + a_{n - 1}}{r_m} = \frac{r_m - r_n}{r_m} = 1 - \frac{r_n}{r_m}.
      \]
      Fix 
      
      $\lim_{n \to \infty} \sum_{k=m}^{n} \frac{a_n}{r_n}$
    \end{proof}
    \item Prove that
    \[
    \frac{a_n}{\sqrt{r_n}} < 2\left( \sqrt{r_n} - \sqrt{r_{n+1}} \right)
    \]
    and deduce that $\sum \frac{a_n}{\sqrt{r_n}}$ converges.
  \end{enumerate}
\end{homeworkProblem}
\end{document}