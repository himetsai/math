\documentclass[addpoints, 11pt]{exam}
\setlength{\headsep}{0.25in}
\setlength{\unitlength}{1in}
%
\pagestyle{head}
%
\usepackage[utf8]{inputenc} %use Unicode
\usepackage[T1]{fontenc} %European fonts

\usepackage{%
	amsmath,       %some math tools
	amssymb,       %math symbols
	graphicx,      %enhanced graphics options
	mathtools,     %extension of amsmath
	microtype,     %small typographic effects
	bm,            %bold math symbols
	% todonotes,   %adds the option \todo{...} (use fixme instead)
	stmaryrd,      %some more math symbolswork
	% nicematrix,    %nicer matrix controls
	mathrsfs,      %more math fonts
        dsfont,
}
\usepackage{url}
\usepackage{hyperref}
%
\usepackage{amsthm}
% \newtheorem*{thm11.1.7}{Theorem 11.1.7}
%
\usepackage[shortlabels]{enumitem}
\usepackage{nicematrix}
\usepackage{multicol}
%
\usepackage[normalem]{ulem}
%
\newcommand{\myCourseNumber}{Math 140A A02}
\newcommand{\myName}{Ray Tsai}
\newcommand{\myID}{A16848188}
\newcommand{\myProfessor}{Professor Mohammadi}
\newcommand{\myHmwkNumber}{1}
% \newcommand{\myExamVersion}{}

\newcommand*{\Z}{\mathbb{Z}}
\newcommand*{\Q}{\mathbb{Q}}
\newcommand*{\R}{\mathbb{R}}
\newcommand*{\C}{\mathbb{C}}
\newcommand*{\N}{\mathbb{N}}
\newcommand*{\prob}{\mathds{P}}
\newcommand*{\E}{\mathds{E}}

\definecolor{crimson}{rgb}{0.86,0.08,0.24}

\newenvironment{question}[1]{\smallskip\noindent\color{crimson}{\bf Question #1.}}{}
\allowdisplaybreaks[1]

\setlength{\parindent}{0pt}
\setlength{\parskip}{5pt}

%% define absolute value \abs{...}:
\DeclarePairedDelimiterX\abs[1]\lvert\rvert{%
  \ifblank{#1}{\:\cdot\:}{#1}
}
% define norm \norm{...}:
\DeclarePairedDelimiterX\norm[1]\lVert\rVert{%
  \ifblank{#1}{\:\cdot\:}{#1}
}
% define inner product \inner{...}{...}:
\DeclarePairedDelimiterX{\inner}[2]{\langle}{\rangle}{%
  \ifblank{#1}{\:\cdot\:}{#1},\ifblank{#2}{\:\cdot\:}{#2}
}

% define \set{...} to write sets and \given to write \set{... \given ...} for {...|...}
\newcommand*\setSymbol[1][]{
  \nonscript\:#1\vert\allowbreak\nonscript\:\mathopen{}
}
\providecommand\given{}
\DeclarePairedDelimiterX\set[1]{\lbrace}{\rbrace}{
  \renewcommand*\given{\setSymbol[\delimsize]}
  #1
}

% free group geneated by ... \free{...} or \free{... \given ...}
\DeclarePairedDelimiterX\free[1]{\langle}{\rangle}{
  \renewcommand\given{\nonscript\:\delimsize\vert\nonscript\:
    \mathopen{}}
  #1}

% define \lopen{...}{...}, \ropen{...}{...}, \open{...}{...}, \closed{...}{...} for intervals
\DeclarePairedDelimiterX\open[2](){#1,#2}
\DeclarePairedDelimiterX\lopen[2](]{#1,#2}
\DeclarePairedDelimiterX\ropen[2][){#1,#2}
\DeclarePairedDelimiterX\closed[2][]{#1,#2}

\NiceMatrixOptions{cell-space-limits = 1pt}
\newcommand*{\pmat}[1]{\begin{pNiceMatrix} #1 \end{pNiceMatrix}}
\newcommand*{\dfdx}[2]{\frac{\partial #1}{\partial #2}}

\DeclareMathOperator{\vol}{vol}
%
\pointsinmargin
\pointpoints{\thinspace point}{points}
\marginpointname{ \points}
%
\begin{document}
%
\firstpageheader{\bfseries \myCourseNumber}{\bfseries Homework \myHmwkNumber}{\bfseries \myName \\ \myID \\ \myProfessor}
%
\runningheader{}{(page \textit{\thepage}\ of \textit{\numpages})}{}
%
%

\begin{question}{A}
    Let \[
        E = \left\{\frac{5n + 8}{11n} : n \in \N \right\}.
    \]
    Compute $\sup E$ and $\inf E$. Justify your answer.
\end{question}

\begin{proof}[Solution]
    We will show that $\sup E = \frac{13}{11}$ and $\inf E = \frac{5}{11}$. Since $n \in \N$, 
    \begin{gather*}
        n \geq 1 \\
        1 \geq \frac{1}{n} \geq 0 \\
        \frac{8}{11} \geq \frac{8}{11n} \geq 0 \\
        \frac{13}{11} \geq \frac{5n + 8}{11n} \geq \frac{5}{11},
    \end{gather*}
    and thus $\frac{13}{11}$ and $\frac{5}{11}$ are a upper bound and a lower bound of $E$ respectively. 
    Let $s < \frac{13}{11}$. Since $\frac{13}{11} \in E$, $s$ is not a upper bound of $E$. Therefore, $\sup E = \frac{13}{11}$.
    
    We will now show inf $E = \frac{5}{11}$ by contradiction. Suppose for the sake of contradiction that there exists a lower bound $l$ of $E$ such that $l > \frac{5}{11}$. Then, for any $n \in \N$,
    \begin{align*}
        \frac{5n + 8}{11n} &\geq l \\
        \frac{8}{11l - 5} &\geq n,
    \end{align*}
    contradiction as $\N$ is unbounded above. Therefore, $\inf E = \frac{5}{11}$.
\end{proof}

\newpage

\begin{question}{B}
    Let $S$ and $T$ be two bounded subsets of the real numbers. Prove that
    \[
        \sup(T \cup S) = \max \{\sup T, \sup S\}.
    \]
\end{question}

\begin{proof}
    Assume without loss of generality that $\max \{\sup T, \sup S\} = \sup T$. For all $s \in S$ and $t \in T$, since $\sup T \geq t$ and $\sup T \geq \sup S \geq s$, we know $\sup T \geq x$, for all $x \in T \cup S$, which shows that $\sup T$ is an upper bound of $T \cup S$. Let $k < \sup T$. Then there exists some $p \in T \subseteq T \cup S$ such that $p > k$, and thus $k$ is not an upper bound of $T \cup S$. Therefore, the statement of the question holds.
\end{proof}

\newpage

\begin{question}{C}
    Let $S$ and $T$ be two bounded, nonempty, subsets of the set of positive real numbers. Define $ST := \{st : s \in S, t \in T\}$ and $S + T := \{s + t : s \in S, t \in T\}$. Prove that 
    \[
        \sup(ST) = (\sup S) \dot (\sup T) \text{ and } \sup(S + T) = \sup S + \sup T.
    \]

\end{question}

\begin{proof}
    We first show that $\sup(ST) = (\sup S) \dot (\sup T)$. Let $t \in T$, $s \in S$. 
    Since $s < \sup S$ and $t < \sup T$, we have $st < (\sup S)t < (\sup S)(\sup T)$, and thus $(\sup S)(\sup T)$ is an upper bound of $ST$. Let $k \in \R^{+}$, such that $k < (\sup S)(\sup T)$. Since $\frac{k}{\sup S} < \sup T$, there exists $t \in T$ such that $\frac{k}{\sup S} < t < \sup T$. Then, we also know that since $\frac{k}{t} < \sup S$, there exists $s \in S$, such that $\frac{k}{t} < s < \sup S$. Rearranged, we get $k < st \in ST$, which shows that $k$ is not an upper bound of $ST$, and thus $\sup(ST) = (\sup S)(\sup T)$.

    We now show that $\sup(S + T) = \sup S + \sup T$. Let $t \in T$, $s \in S$. Since $s < \sup S$ and $t < \sup T$, we have $s + t < \sup S + \sup T$, and thus $\sup S + \sup T$ is an upper bound of $S + T$. Let $k \in \R^{+}$, such that $k < \sup S + \sup T$. Since $k - \sup T < \sup S$, there exists $s \in S$ such that $k - \sup T < s$. Since $k - s < \sup T$, there exists $t \in T$ such that $k - s < t$, and thus we know there exist $s + t \in S + T$ such that $k < s + t < \sup S + \sup T$. Therefore, $\sup(S + T) = \sup S + \sup T$.
\end{proof}

\newpage

\begin{question}{D}
    Let $F$ be the set of all rational functions\
    \begin{gather}
        \frac{a_nx^n + a_{n - 1}x^{n - 1} + \dots + a_0}{b_mx^m + b_{m - 1}x^{m - 1} + \dots + b_0}
    \end{gather}
    where the coefficients are real numbers and $b_m \neq 0$.
\end{question}

\begin{enumerate}[(i)]
    \color{crimson}
    \item  Define addition and multiplication of two elements in $F$ to be the usual addition and multiplication of functions. Show that with this addition and multiplication, $F$ is a field.
    \normalcolor
    
    \begin{proof}
        Let $A = \{a_nx^n + a_{n - 1}x^{n - 1} + \dots + a_0 \, | \, a_n, \dots , a_0 \in \R\}$, $B = \{b_mx^m + b_{m - 1}x^{m - 1} + \dots + b_0 \, | \, b_m , \dots , b_0 \in \R - \{0\}\}$. Let $a = \frac{f_1}{g_1}, b = \frac{f_2}{g_2}, c = \frac{f_3}{g_3} \in F$. 
        
        \textbf{Associativity:}
        Since
        \[
            (a + b) + c = \frac{f_1g_2g_3 + f_2g_1g_3 + f_3g_1g_2}{g_1g_2g_3} = a + (b + c)
        \]
        and
        \[
            (ab)c = \frac{f_1f_2f_3}{g_1g_2g_3} = a(bc),
        \]
        $F$ is associative under $+$ and $\times$.

        \textbf{Commutativity:} Since 
        \[
            a + b = \frac{f_1g_2 + f_2g_1}{g_1g_2} = b + a
        \] 
        and
        \[
            ab = \frac{f_1f_2}{g_1g_2} = ba,
        \]
        $F$ is commutative under $+$ and $\times$.

        \textbf{Additive and multiplicative identity:} Since
        \[
            a + 0 = 0 + a = a
        \]
        and
        \[
            a \cdot 1 = 1 \cdot a = a,
        \]
        $F$ has additive and multiplicative identity.

        \textbf{Additive inverses: } For every $a$, we have $a^{-1} = -a \in F$, so that $a + (-a) = 0$.

        \textbf{Multiplicative inverses: } For every $a \neq 0$, we have $a^{-1} = \frac{g_1}{f_1} \in F$. Note that $f_1 \in B$. Then, we have $aa^{-1} = \frac{f_1}{g_1} \cdot \frac{g_1}{f_1} = 1$.

        \textbf{Distributivity:} Since 
        \[
            a(b + c) = \frac{f_1}{g_1} \cdot \frac{f_2g_3 + f_3g_2}{g_2g_3} = (\frac{f_1}{g_1} \cdot \frac{f_2}{g_2}) + (\frac{f_1}{g_1} \cdot \frac{f_3}{g_3}) = (ab) + (ac),
        \]
        $F$ is distributive.

        The above qualities show that $F$ is a field under addition and multiplication.
    \end{proof}

    \color{crimson}
    \item  We can define an order on $F$ as follows. A rational function like $(1)$ is positive if and only if $a_n$ and $b_m$ have the same sign, i.e. $a_nb_m > 0$. Now given two rational functions $\frac{p}{q}$ and $\frac{f}{g}$ we define:
    \[
        \frac{p}{q} > \frac{f}{g} \text{ if and only if } \frac{p}{q} - \frac{f}{g} > 0.
    \]
    Show with this ordering and the operations in part (i), $F$ is an ordered field.
    \normalcolor
    
    \begin{proof}
         We continue using the defined sets $A, B$ and elements $a, b, c \in F$ from part (i). 

         We first show that $F$ is an ordered set. Let $n_1, m_1 \in \R$, $m_1 \neq 0$, each be the leading coefficient of $f_1, g_1$. Since $\R$ is an ordered set, we know $n_1m_1$ must be either positive, negative, or equal to $0$. This indicates that for all $f \in F$, $f$ must be either positive, negative, or equal to $0$. Since $a - b \in F$, it must be either positive, negative, or equal to $0$. Therefore, since $a, b \in F$, one and only one of the following statements
         \[
            a > b, \quad b > a, \quad a = b
         \]
         is true.

        Suppose $a > b$ and $b > c$, then $\frac{f_1g_2 - f_2g_1}{g_1g_2} > 0$ and $\frac{f_2g_3 - f_3g_2}{g_2g_3} > 0$. Combining two equations, we get $\frac{f_1g_2g_3 - f_2g_1g_3 + f_2g_1g_3 - f_3g_1g_2}{g_1g_2g_3} > 0$. It follows that
        \[
            \frac{f_1g_3 - f_3g_1}{g_1g_3} = a - c > 0.
        \]
        Thus, $F$ is an ordered set since it meets the two required conditions.

        Suppose $c > b$. We know $a + c = \frac{f_1g_3 + f_3g_1}{g_1g_3}$ and $a + b = \frac{f_1g_2 + f_2g_1}{g_1g_2}$. Since $c > b$, we rearrange and get $f_3g_2 > f_2g_3$. Thus
        \begin{align*}
            f_3g_2 &> f_2g_3 \\
            f_3g_2g_1 &> f_2g_3g_1 \\
            f_1g_2g_3 + f_3g_2g_1 &> f_1g_2g_3 + f_2g_3g_1 \\
            \frac{f_1g_3 + f_3g_1}{g_1g_3} &> \frac{f_1g_2 + f_2g_1}{g_1g_2} && \text{dividing }g_1g_2g_3 \text{ on both sides} \\
            a + c &> a + b.
        \end{align*}

        Suppose $a, b$ are positive. Let $n_1, n_2, m_1, m_2 \in \R - \{0\}$ each be the leading coefficient of $f_1, f_2, g_1, g_2$, we get $n_1m_1, n_2m_2 > 0$. Since the leading coefficient of the product of two polynomials is the product of the leading coefficients of the two polynomials, we know that the leading coefficient of $f_1f_2$ and $g_1g_2$ are $n_1n_2$ and $m_1m_2$, respectively. Since $n_1m_1, n_2m_2 > 0$, $n_1n_2m_1m_2 > 0$, and thus $ab = \frac{f_1f_2}{g_1g_2}$ is also positive.

        Since all the conditions are met, $F$ is an ordered field.
    \end{proof}

    \color{crimson}
    \item  Write the following polynomials in order of increasing size using the order defined in (ii): $x^2, -x^5, 2, x + 6, 3 - 2x$.
    \normalcolor
    
    \begin{proof}[Solution]
        Since
        \begin{gather*}
            x^2 - (x + 6) = x^2 - x - 6 > 0, \\
            x + 6 - 2  = x + 4 > 0, \\
            2 - (-2x + 3) = 2x -1 > 0, \\
            -2x + 3 - (-x^5) = x^5 -2x + 3 > 0,
        \end{gather*}
        we have 
        \[
            x^2 > x + 6 > 2 > -2x + 3 > -x^5,
        \]
        by the transitivity of ordered sets.
    \end{proof}

    \color{crimson}
    \item  Show that $x > a$ for all $a \in R$.
    \normalcolor
    
    \begin{proof}
        Let $a \in \R$. Since $x - a$ has a leading coefficient of $1$, the statement holds true. 
    \end{proof}
\end{enumerate}

\newpage

\begin{question}{E1}
    If $r$ is rational ($r \neq 0$) and $x$ is irrational, prove that $r + x$ and $rx$ are irrational.
\end{question}

\begin{proof}
    Let $r = \frac{m}{n}$, for $m, n \in \Z$, $\gcd(m,n) = 1$. We first show $r + x$ to be irrational. Suppose for the sake of contradiction that $r + x = \frac{p}{q}$, for $p, q \in \Z$, $\gcd(p,q) = 1$. Then $x = \frac{p}{q} - \frac{m}{n} = \frac{mq + np}{nq} \in \Q$, contradiction. 

    We now show $rx$ to be irrational. Suppose for the sake of contradiction that $rx = \frac{k}{l}$, for $k, l \in \Z$, $\gcd(k,l) = 1$. Then $x = \frac{\frac{k}{l}}{\frac{m}{n}} = \frac{kn}{lm} \in \Q$, contradiction.

    Therefore, both $r + x$ and $rx$ are irrational.
\end{proof}

\newpage

\begin{question}{E2}
    Prove that there is no rational number whose square is 12.
\end{question}

\begin{proof}
    Let $p = \frac{m}{n}$, for $m, n \in \Z$, $\gcd(m,n) = 1$. Suppose for the sake of contradiction that $p^2 = 12$. We know $m^2 = 12n^2$, and so $m = 2k$, for $k \in \Z$.  We then have $k^2 = 3n^2$, which implies that $3 | k$. This shows that $m = 6l$, for $6 \in \Z$. Substituting it back into the equation, we get $3l^2 = n^2$, which shows that $3 | m, n$, contradiction. Therefore, the statement of the question holds true.
\end{proof}

\newpage

\begin{question}{E5}
    Let $A$ be a nonempty set of real numbers which is bounded below. Let $-A$ be the set of all numbers $-x$, where $x \in A$. Prove that
    \[
        \inf A = -\sup(-A).
    \]
\end{question}

\begin{proof}
    Let $k = \inf A$, $b \in -A$. Since $-b \in A$, we know $k < -b$. Therefore, $-k > b$, and thus $-k$ is an upper bound of $-A$. Let $m \in \R$, such that $m < -k$. Since $-m > k$, we know there exists $a \in A$, such that $-m > a$. Since $-a \in -A$ and $-a > m$, $m$ is not an upper bound of $-A$. Therefore, $k = -\sup(-A)$.
\end{proof}

\newpage

\begin{question}{E8}
    Prove that no order can be defined in the complex field that turns it into an ordered field.
\end{question}

\begin{proof}
    Let $a, b \in \C$. Suppose for the sake of contradiction that there exists some ordering such that $a > b$. We then have 
    \begin{align*}
        a &> b \\
        ia &> ib \\
        -a &> -b \\
        a &< b,
    \end{align*}
    contradiction. Thus, the statement holds true. 
\end{proof}

\end{document}
