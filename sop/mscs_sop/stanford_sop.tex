\documentclass[12pt]{article}
\usepackage[utf8]{inputenc}
\usepackage{amsmath}
\usepackage{amssymb}
\usepackage{geometry}
\geometry{letterpaper, margin=1in}
\usepackage{parskip}
\usepackage{mathpazo}
\usepackage{titling}

\setlength{\droptitle}{-0.5in}

\parindent=20pt

\title{Statement of Purpose}
\author{Ray Tsai}
\date{}

\begin{document}


\maketitle

\vspace{-0.25in}

I am applying to the Master's program in Computer Science at Stanford University to deepen my
understanding of theoretical computer science, particularly in complexity theory and the
combinatorial side of the study. My goal is to become a researcher working at the intersection of
computer science and mathematics, finding ways to bridge the two fields. I intend to pursue a Ph.D.
after the master's program to further this objective.

As a mathematics-computer science major at UC San Diego, I dedicated most of my attention to
mathematics in my undergraduate studies, primarily through honors and graduate-level coursework.
While I found beauty in purely theoretical concepts, I faced an existential crisis pursuing
mathematics for its own sake. I instead looked for a field in mathematics with intimate connection
to the real world and turned to graph theory, the study of relationships that can model a wide range
of practical problems effectively.

During my sophomore year, I began research with Professor Verstraete in extremal graph theory,
exploring a range of open problems and conjectures. My first focus was on a problem involving long
paths in Eulerian digraphs, specifically seeking a lower bound for the longest directed path based
on average degree. This problem extends the well-known Erdős-Gallai theorem and holds relevance in
algorithmic studies. Although I didn't achieve a concrete result after a quarter of trial and error,
this experience gave me firsthand insight into the perseverance and challenges integral to research.

My spirit, however, was not dampened by the failure, and I continued to explore other open problems
after gaining more mathematical maturity through my coursework. My honors thesis now centers on the
Double Turán problem, which asks for the maximum possible number of edges in $n$ subgraphs of a
complete graph $K_n$, with no pairwise intersection of these subgraphs containing a certain
forbidden structure. Through this work, I have developed a toolkit that ranges from the
probabilistic method to construction techniques. After dedicating my last summer to studying the
triangle-free case, I completed the proof with a tighter condition that each subgraph is induced,
which serves as a stepping stone for the general case. 

As I advanced in my studies, I grew interest in computational complexity theory, which studies
combinatorial problems with a computational lens grounded in real-world problems. After a quarter
spent working through Sanjeev Arora and Boaz Barak's \textit{Computational Complexity: A Modern
Approach}, I joined Professor Impagliazzo's research group, studying Multicalibration to address
unintended bias in learning models from the perspective of complexity theory. The project opened my
eyes to the unexpected connection between theoretical computer science and combinatorics. Apart from
the already commonly used combinatorial tools like the probabilistics method, the project brought
the application of combinatorics to a new level by modeling the fairness of algorithms with the
random-like structures yielded by the Szemeredi's Regularity Lemma. This experience reinforced my
interest in the interplay between the two discplines and prompted me to pursue further studies in
theoretical computer science.

With my mathematics-heavy background, I aim to build a deeper understanding of computer science
concepts to support future Ph.D. work. Stanford's Master's program offers an ideal path for this,
particularly with its Theoretical Computer Science specialization. I look forward to gaining a
rigorous grounding in computational complexity through courses like the CS 254 series, and I aim to
formalize my knowledge of Multicalibration by taking Algorithmic Fairness (CS 256), a subject I have
mostly self-studied. Additionally, I plan to learning more about combinatorial applications to
computer science, and taking courses like Matching Theory (MS\&E 319), Combinatorial Optimization
(CS 261), and Open Problems in Coding Theory (CS 351) would be a great start.

Moreover, I am excited about the research opportunities at Stanford and aim to pursue distinction in
research, particularly under the guidance of Professor Li-Yang Tan on topics in Hardness
Amplification. During my undergraduate research with Professor Impagliazzo, I encountered Professor
Tan's work,
\textit{``The Sample Complexity of Smooth Boosting and the Tightness of the Hardcore Theorem,''}
which reveals that the circuit size loss in Impagliazzo's Hardcore Theorem is inherently necessary,
regardless of the proof technique. I would like to investigate a parallel result for Yao's XOR lemma
in collaboration with Professor Tan.

I have only begun to scratch the surface of theoretical computer science, and there are still a lot
of topics to explore. I am excited about the prospect of joining Stanford's Master's in Computer
Science program to expose me to a broader range of topics in computer science through rigorous
coursework and research opportunities. With the mathematical foundation I built during my
undergraduate studies, I am confident in my ability to excel in the program and evolve into a
capable researcher. I look forward to contributing to the vibrant academic community at Stanford and
bridging the worlds of mathematics and computer science.

\end{document}
