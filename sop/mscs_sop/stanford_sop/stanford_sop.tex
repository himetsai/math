\documentclass[12pt]{article}
\usepackage[utf8]{inputenc}
\usepackage{amsmath}
\usepackage{amssymb}
\usepackage{geometry}
\geometry{letterpaper, margin=1in}
\usepackage{parskip}
\usepackage{mathpazo}
\usepackage{titling}
\usepackage{cite}  

\setlength{\droptitle}{-0.5in}

\parindent=20pt

\title{Statement of Purpose}
\author{Ray Tsai}
\date{}

\begin{document}


\maketitle

\vspace{-0.25in}

I am applying to the Master's program in Computer Science at Stanford University to deepen my
understanding of theoretical computer science, particularly in complexity theory and approximation
algorithms. I am drawn to the increasing relevance of these areas in addressing real-world
challenges, and I aim to apply my mathematical background to develop innovative solutions. Whether
in academia or industry, I aim to contribute to advancements that bridge theoretical foundations
with practical applications.

As a mathematics-computer science major at UC San Diego, I dedicated most of my attention to
mathematics in my undergraduate studies, primarily through honors and graduate-level coursework.
While I found beauty in purely theoretical concepts, I faced an existential crisis pursuing
mathematics for its own sake. I instead looked for a field in mathematics with an intimate
connection to the real world and turned to graph theory, the study of relationships that can model a
wide range of practical problems effectively.

During my sophomore year, I began my research journey by undertaking independent study under
Professor Jacques Verstraete, exploring the extensive literature on extremal graph theory. Through
the readings, I was exposed to a variety of powerful techniques for tackling extremal graph
problems, such as probabilistic methods, stability, and finite geometric constructions, which laid
the groundwork for my future research. Currently, I am working on my honors thesis centering on the
Double Turán problem, which asks for the maximum possible number of edges across $n$ subgraphs of a
complete graph $K_n$ such that no pairwise intersection of these subgraphs contain a specified
forbidden structure. After dedicating my summer to studying the triangle-free case, I established a
tight upper bound on the number of edges under the stricter condition that each subgraph is induced.
Specifically, I showed that the extremal condition is uniquely achieved when all subgraphs are
complete bipartite graphs, by recursively expanding the intersection of all subgraphs. This case
serves as a stepping stone for the project, and I am currently working on the general triangle-free
case. 

As I advanced in my studies, I also grew interested in computational complexity theory, which
studies combinatorial problems with a computational lens grounded in real-world problems. After a
quarter spent working through Sanjeev Arora and Boaz Barak's \textit{Computational Complexity: A
Modern Approach}\cite{arora2009computational}, I recently joined Professor Russell Impagliazzo's
research group, studying Multicalibration to address unintended bias in learning models from the
perspective of complexity theory. This research opened my eyes to the surprising connections between
abstract mathematical concepts and the practical applications of machine learning. Specifically, the
project brought the application of extremal combinatorics and complexity theory to a new level by
modeling the fairness of algorithms with the random-like structures yielded by Szemeredi's
Regularity Lemma. Through the project, I realized the boundless transformative power of
combinatorial tools, which adds another layer of meaning to my interest in theoretical computer
science.

Pivoting from a mathematical background, I aim to fill in the gaps in my computer science knowledge,
and Stanford's Master's program offers an ideal path for this, particularly with its Theoretical
Computer Science specialization. I look forward to gaining a rigorous grounding in computational
complexity through courses like the CS 254 series, and I aim to formalize my knowledge of
Multicalibration by taking Algorithmic Fairness (CS 256), a subject I have mostly self-studied.
Additionally, I plan to learning more about combinatorial applications to computer science, and
taking courses like Matching Theory (MS\&E 319), Combinatorial Optimization (CS 261), and Open
Problems in Coding Theory (CS 351) would be a great start.

Moreover, I am excited about the research opportunities at Stanford and aim to pursue distinction in
research, particularly under the guidance of Aviad Rubinstein on approximate algorithms. His
innovative work on algorithms for approximating the size of maximum matchings is especially
compelling. I was intrigued by his algorithm in \cite{doi:10.1137/1.9781611977554.ch151} that beats
the $\frac{1}{2}$-approximation barrier without even requiring a full traversal of the graph.
Professor Rubinstein's research direction is not only exciting but of great practical applications,
and I hope to leverage my specialization in combinatorics to develop innovative algorithms with him.

Another faculty member I want to work with is Professor Li-Yang Tan on topics in hardness
amplification. During my undergraduate research with Professor Impagliazzo, I encountered
Professor's work on the tightness of Impagliazzo's
hardcoretheorem\cite{blanc2024samplecomplexitysmoothboosting}, which reveals that the circuit size
loss in the hardcore theorem is inherently necessary, regardless of the proof technique. I would
like to investigate a parallel result for Yao's XOR lemma in collaboration with Professor Tan.

I have only begun to scratch the surface of theoretical computer science, and there are still a lot
of topics to explore. I am excited about the prospect of joining Stanford's Master's in Computer
Science program to expose me to a broader range of topics in computer science through rigorous
coursework and research opportunities. With the mathematical foundation I built during my
undergraduate studies, I am confident in my ability to excel in the program and evolve into a
capable problem-solver. I look forward to contributing to the vibrant academic community at Stanford
and bridging the worlds of mathematics and computer science.

\newpage

\bibliographystyle{plain}
\bibliography{references}
\end{document}
