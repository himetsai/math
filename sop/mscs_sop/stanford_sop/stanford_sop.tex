\documentclass[12pt]{article}
\usepackage[utf8]{inputenc}
\usepackage{amsmath}
\usepackage{amssymb}
\usepackage{geometry}
\geometry{letterpaper, margin=1in}
\usepackage{parskip}
\usepackage{mathpazo}
\usepackage{titling}
\usepackage{cite}  

\setlength{\droptitle}{-0.5in}

\parindent=20pt

\title{Statement of Purpose}
\author{Ray Tsai}
\date{}

\begin{document}


\maketitle

\vspace{-0.25in}

I am applying to the Master's program in Computer Science at Stanford University with a
specialization in theoretical computer science to translate my mathematical foundation, particularly
in combinatorics, to abilities that can address real-world problems. I am especially drawn to the
fields of approximation algorithms and cryptography, where the interplay between theory and
application is most apparent, and where I believe I can contribute to impactful advancements in
areas like secure data sharing and resource optimization.

As a mathematics-computer science major at UC San Diego, I dedicated most of my attention to
mathematics in my undergraduate studies, primarily through honors and graduate-level coursework.
While I appreciate the beauty of pure mathematics, I wasn't satisfied pursuing abstract mathematics
for its own sake. I instead looked for a field in mathematics with a closer connection to the real
world and turned to graph theory, the study of relationships that can model a wide range of
practical problems effectively.

During my sophomore year, I began my research journey by undertaking independent study under
Professor Jacques Verstraete, exploring the extensive literature on extremal graph theory. Through
the readings, I was exposed to a variety of powerful techniques for tackling extremal graph
problems, such as probabilistic methods, stability, and finite geometric constructions, which laid
the groundwork for my future research. Currently, I am working on my honors thesis centering on the
Double Turán problem, which asks for the maximum possible number of edges across $n$ subgraphs of a
complete graph $K_n$ such that no pairwise intersection of these subgraphs contain a specified
forbidden structure. After dedicating my summer to studying the triangle-free case, I established a
tight upper bound on the number of edges under the stricter condition that each subgraph is induced.
Specifically, I showed that the extremal condition is uniquely achieved when all subgraphs are
complete bipartite graphs, by recursively expanding the intersection of all subgraphs. This case
serves as a stepping stone for the project, and I am currently working on the general triangle-free
case. 

As I advanced in my studies, I also grew interested in computational complexity theory, which
studies combinatorial problems with a computational lens grounded in real-world problems. After a
quarter spent working through Sanjeev Arora and Boaz Barak's \textit{Computational Complexity: A
Modern Approach}\cite{arora2009computational}, I recently joined Professor Russell Impagliazzo's
research group, studying Multicalibration to address unintended bias in learning models from the
perspective of complexity theory. This research opened my eyes to the surprising connections between
abstract mathematical concepts and the practical applications of machine learning. Specifically, the
project brought the application of extremal combinatorics and complexity theory to a new level by
modeling the fairness of algorithms with the random-like structures yielded by Szemeredi's
Regularity Lemma. Through this project, I solidified my interest in connecting deep theoretical
results with real-world applications

Stanford's Master's program offers the ideal path to translate my mathematical expertise into a
career focused on developing practical algorithms and solutions. I envision myself contributing to
fields like kidney exchange programs and secret sharing schemes, where combinatorial tools are
applied extensively. Taking courses like Matching Theory (MS\&E 319), Combinatorial Optimization (CS
261), Algebraic Error Correcting Codes (CS 250), and Advanced Topics in Cryptography (CS 355) would
provide me the necessary background of combinatorial applications to computer science. 

Moreover, I am looking forward to the research opportunities at Stanford and aim to pursue
distinction in research, I am particularly excited by the opportunity to work with faculty like
Aviad Rubinstein and Li-Yang Tan, whose work aligns well with my interests. Professor Rubinstein's
is an leading expert in approximation algorithms, and I am interested in working on matching
algorithms with him. I was intrigued by his algorthm for approximating maximum matching size in
\cite{doi:10.1137/1.9781611977554.ch151} that beats the $\frac{1}{2}$-approximation barrier without
even requiring a full traversal of the graph, and I am curious if a better accuracy can be achieved.

Similarly, I am interested in working with Professor Li-Yang Tan on pseudorandomness. His work on
the tightness of Impagliazzo's hardcore theorem \cite{blanc2024samplecomplexitysmoothboosting},
which I encountered during my undergraduate research with Professor Impagliazzo, revealed a
fundamental limitation in hardness amplification for pseudorandom generator construction. This
result raised critical questions about the security and efficiency of PRGs, and I want to study the
potential solutions and alternative approaches under Professor Tan's guidance.

I have only begun to scratch the surface of theoretical computer science, and there are still a lot
of topics to explore. I am excited about the prospect of joining Stanford's Master's in Computer
Science program to expose me to a broader range of topics in computer science through rigorous
coursework and research opportunities. With the mathematical foundation I built during my
undergraduate studies, I am confident in my ability to excel in the program and evolve into a
capable problem-solver. I look forward to contributing to the vibrant academic community at Stanford
and bridging the worlds of mathematics and computer science.

\newpage

\bibliographystyle{plain}
\bibliography{references}
\end{document}
