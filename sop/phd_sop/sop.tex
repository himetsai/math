\documentclass[12pt]{article}
\usepackage[utf8]{inputenc}
\usepackage{amsmath}
\usepackage{amssymb}
\usepackage{geometry}
\geometry{letterpaper, margin=1in}
\usepackage{parskip}
\usepackage{mathpazo}
\usepackage{titling}

\setlength{\droptitle}{-0.5in}

\parindent=20pt

\title{Statement of Purpose}
\author{Ray Tsai}
\date{}

\begin{document}


\maketitle

\vspace{-0.25in}

I am applying to the Ph.D program in Algorithm, Combinatorics, and Optimization (ACO) at Carnegie
Mellon University to pursue my interest in combinatorics, especially in extremal graph theory and
its connections to theoretical computer science. My goal is to become a capable researcher in
combinatorics.

As a mathematics-computer science major at UC San Diego, I dedicated most of my attention to
mathematics in my undergraduate studies, primarily through honors and graduate-level coursework.
While I found beauty across various fields in mathematics, I was especially drawn to combinatorics
due to its deceiving simplicity. Combinatorial problems are often stated in friendly ways that a
middle schooler can grasp, but yet they are grounded in complex structures and reveal profound
implications once digging in.

During my sophomore year, I began research with Professor Verstraete in extremal graph theory,
exploring a range of open problems and conjectures. My first focus was on an open problem involving
long paths in Eulerian digraphs. The Erdős-Gallai theorem guarantees a path of length $k$ in a graph
of average degree $k$, but it remains unclear if the analogous result holds for directed paths in
Eulerian digraphs. Being my first research experience in mathematics, I realized the stark contrast
between coursework and research and struggled significantly in navigating through dead ends and
setbacks. Nevertheless, this challenges and difficulities I faced introduced perseverance essential
to the research process, marking a formative step in my academic journey.

My spirit, however, was not dampened, and I continued to explore other open problems after gaining
more mathematical maturity through my coursework. My honors thesis now centers on the Double Turán
problem, which asks for the maximum possible number of edges in $n$ subgraphs of a complete graph
$K_n$, with no pairwise intersection of these subgraphs containing a certain forbidden structure.
Through this work, I have developed a toolkit that ranges from the probabilistic method to
construction techniques. After dedicating my last summer to studying the triangle-free case, I
completed the proof with a tighter condition that each subgraph is induced, which serves as a
stepping stone for the general case. 

As I advanced in my studies, I grew interest in computational complexity theory, which studies
combinatorial problems with a computational lens grounded in real-world problems. After a quarter
spent working through Sanjeev Arora and Boaz Barak's \textit{Computational Complexity: A Modern
Approach}, I joined Professor Impagliazzo's research group, studying Multicalibration to address
unintended bias in learning models from the perspective of complexity theory. The project opened my
eyes to the unexpected connection between theoretical computer science and combinatorics. Apart from
the already commonly used combinatorial tools like the probabilistics method, the project brought
the application of combinatorics to a new level by modeling the fairness of algorithms with the
random-like structures yielded by the Szemeredi's Regularity Lemma. This made me realize the
boundless potential of real-world application of combinatorial tools, adding another layer of
meaning to my interest in combinatorics.

My undergraduate experiences opened my appetite for mathematical research, but it also humbled me
with the immense breadth and depth of the field. To further proceed in the realm of combinatorics,
graduate school is a crucial first step for me to achieve this goal. I plan to continue my research
in extremal combinatorics in graduate school, as well as exploring adjacent fields such as discrete
geometry, random graphs, and topics that I can bridge combinatorics with.

The ACO program at Carnegie Mellon University seems tailor made for my goals. The program's emphasis
on combinatorics and its interdisciplinary nature with computer science and operation research
perfectly matches my intended research direction. The wide range of faculty members with expertise
that covers most of combinatorics will provide me with well-rounded training and guidance in the
field. 

[Closing Paragraph]

\end{document}
