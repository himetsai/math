\documentclass[12pt]{article}
\usepackage[utf8]{inputenc}
\usepackage{amsmath}
\usepackage{amssymb}
\usepackage{geometry}
\geometry{letterpaper, margin=0.5in, bottom=1in}
\usepackage{parskip}
\usepackage{mathpazo}
\usepackage{titling}

\setlength{\droptitle}{-0.5in}

\parindent=20pt

\title{Statement of Purpose}
\author{Ray Tsai}
\date{}

\begin{document}


\maketitle

\vspace{-0.25in}

I am applying to the Ph.D program in Mathematics at XXX University to pursue my interest in
combinatorics, particularly in extremal graph theory and its connections to theoretical computer
science. 

My curiosity about combinatorics was first sparked in an honors course on graph theory, taught by
Professor Jacques Verstraete. Being my first exposure to combinatorics, the course soon intrigued me
with the unconventional nature of combinatorial problems. Unlike other fields of mathematics,
combinatorial problems often appear as discrete and uncorrelated with each other, but they
nevertheless come with surprising relationships beneath the surface.

To deepen my engagement, I proceeded to take the most challenging courses offered at UC San Diego,
including other honors and graduate-level combinatorics courses, along with seminars that exposed
showcased modern research topics in combinatorics. However, my interest in the field was truly
solidified through research experiences under the mentorship of Professor Verstraete, where I
explored extensive literature on Turán problems and worked on open problems and conjectures. 

The first open problem I attempted was a problem on long paths in Eulerian Digraphs, which asks
for the lower bound of the longest path in an Eulerian digraph. This problem is an extension of the
famous Erdős-Gallai theorem and is also of interest in theoretical computer science. I attempted the
problem with an alternative approach suggested by Professor Verstraete, which involves randomly
selecting a cyclic ordering of the vertices and analyzing the expected length of the longest
``zig-zag'' pattern in the ordering. Although I did not obtain a tangible result due to issues in
extending the zig-zag path, the experience gave me a taste of the challenges and obstacles one will
face in mathematical research.

My spirit, however, was not dampened by the failure, and I continued to explore other open problems
after gaining more mathematical maturity through the honors algebra and analysis sequences. My
honors thesis now centers on the Double Turán problem, which asks for the maximum possible number of
edges in $n$ subgraphs of a complete graph $K_n$, with no pairwise intersection of these subgraphs
containing a certain forbidden structure. Through this work, I have developed a researcher's mindset
and a toolkit that ranges from the probabilistic method to construction techniques, as well as
coding skills for searching counterexamples. After dedicating my last summer to studying the
triangle-free case, I completed the proof with a tighter condition that each subgraph is induced,
which serves as a stepping stone for the general case. Despite the difficulties and frustrations
involved in the process, I found myself more motivated than ever and continued to explore the general
case. 

Alongside my research in extremal combinatorics, I pursued theoretical computer science, drawn by
its close ties to combinatorics. After a quarter spent working through Sanjeev Arora and Boaz
Barak's \textit{Computational Complexity: A Modern Approach}, I recently joined a research project
under Professor Russell Impagliazzo on multicalibration, aiming to reduce unintended bias in
learning models. The project opened my eyes to the unexpected connection between algorithm fairness
and the Szemerédi regularity lemma through their ``random-like'' properties. This made me realize
the boundless potential of real-world application of combinatorial tools, adding another layer of
meaning to my interest in combinatorics. Recognizing the extensive real-world applications of
combinatorial tools added new depth to my interest in combinatorics and its potential impact. 

My undergraduate experiences opened my appetite for mathematical research, but it also humbled me
with the immense breadth and depth of the field. Io further proceed in the realm of combinatorics, I
realized graduate school is a crucial first step for me to achieve this goal, as is essential to
developing the skills and knowledge I need to become a successful researcher in mathematics.

I plan to continue investigating Turán problems and their generalizations in graduate school. An
interesting research direction I would like to follow is to build on my undergraduate research. The
degenerated case of the Double Turán Problem, for which the forbidden structure is bipartite, is
shown to be harder than the non-degenerated case, as seen in the classical Turán problem. Currently,
I lack the expertise to tackle this case, but I would like to explore the degenerated double further
Turán problems as I acquire more tools and insights in my graduate studies. More specifically, I
need to learn more about polynomial equations over finite fields and projective planes, which are
crucial tools for constructions in these degenerated cases.

While I have a strong interest in extremal combinatorics, I do not aim to limit myself to a
single area early in my research career. I plan to explore areas adjacent to combinatorics such as
theoretical computer science to bridge my rather specialized knowledge in extremal
combinatorics with other domains and expand my understanding of the mathematical landscape. I
believe that cultivating a diverse set of perspectives will lead to more fruitful outcomes in my
research career.

[Why this program?]

[Closing Paragraph]

\end{document}
