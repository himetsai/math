\documentclass[12pt]{article}
\usepackage[utf8]{inputenc}
\usepackage{amsmath}
\usepackage{amssymb}
\usepackage{geometry}
\geometry{letterpaper, margin=1in}
\usepackage{parskip}
\usepackage{mathpazo}
\usepackage{titling}
\usepackage{cite}  

\setlength{\droptitle}{-0.5in}

\parindent=20pt

\title{Statement of Purpose}
\author{Ray Tsai}
\date{}

\begin{document}


\maketitle

\vspace{-0.25in}

I am applying to the Ph.D program in Algorithm, Combinatorics, and Optimization (ACO) at Georgia
Tech to pursue my interest in combinatorics, especially in extremal graph theory and its
applications to adjacent fields like theoretical computer science.

As a mathematics-computer science major at UC San Diego, I dedicated most of my undergraduate
studies to mathematics, primarily through honors and graduate-level coursework. While I found beauty
across fields in mathematics, I was especially drawn to combinatorics due to its deceiving
simplicity and intimate ties to real-world problems. I aspire to become a researcher who not only
specializes in combinatorics but also bridges combinatorics with other disciplines. I believe the
ACO program at Georgia Tech is the ideal place to achieve this goal.

During my sophomore year, I began my research journey by undertaking independent study under
Professor Jacques Verstraete, exploring the extensive literature on extremal graph theory. Through
the readings, I was exposed to a variety of powerful techniques for tackling extremal graph
problems, such as probabilistic methods, stability, and finite geometric constructions, which laid
the groundwork for my future research. Currently, I am working on my honors thesis centering on the
Double Turán problem, which asks for the maximum possible number of edges across $n$ subgraphs of a
complete graph $K_n$ such that no pairwise intersection of these subgraphs contain a specified
forbidden structure. After dedicating my summer to studying the triangle-free case, I established a
tight upper bound on the number of edges under the stricter condition that each subgraph is induced.
Specifically, I showed that the extremal condition is uniquely achieved when all subgraphs are
complete bipartite graphs, by recursively expanding the intersection of all subgraphs. This case
serves as a stepping stone for the project, and I am currently working on the general triangle-free
case. 

As I advanced in my studies, I also grew interested in computational complexity theory, which
studies combinatorial problems with a computational lens grounded in real-world problems. After a
quarter spent working through Sanjeev Arora and Boaz Barak's \textit{Computational Complexity: A
Modern Approach}\cite{arora2009computational}, I joined Professor Russell Impagliazzo's research
group, studying Multicalibration to address unintended bias in learning models from the perspective
of complexity theory. The project opened my eyes to the unexpected connections between theoretical
computer science and combinatorics, as it brought the application of combinatorics to a new level by
modeling the fairness of algorithms with the random-like structures yielded by Szemeredi's
Regularity Lemma. Through the project, I realized the boundless potential of real-world application
of combinatorial tools and it adds another layer of meaning to my interest in combinatorics.

My undergraduate experiences opened my appetite for mathematical research, but it also humbled me
with the immense breadth and depth of the discipline. Graduate school is a crucial first step to
further proceed in the realm of combinatorics. I plan to continue my research in extremal
combinatorics in graduate school, as well as explore adjacent fields such as discrete geometry,
random graphs, and topics that I can bridge combinatorics with.

The ACO program at Georgia Tech seems tailor-made for my goals. With its emphasis on combinatorics
and interdisciplinary nature with computer science and industrial engineering, I can build solid
foundations in various fields while furthering my studies in combinatorics, which would realize my
aspiration to become a researcher who connects combinatorics with other disciplines. 

The diverse expertise of the faculty members in combinatorics will provide me with comprehensive
training in the field. I am particularly interested in collaborating with Professor Rose McCarty on
extremal combinatorics problems. An aspect of her research that I find intriguing is on Thomassen's
famous conjecture on girth in graphs. In her recent work \cite{du2023inducedc4freesubgraphslarge},
she gave a polynomial bound on the average degree of a graph that guarantees the existence of of a
induced $C_4$-free subgraph with a chosen average degree, and I am intrigued collaborate with her to
find an analogous result for the non-induced case. In addition to pure combinatorics, she is also
active in the realm of theoretical computer science, such as her work on model checking problem on
monadically stable graph classes\cite{dreier2023firstordermodelcheckingmonadically}, aligns well
with my interest in exploring the intersection of combinatorics and computer science under her
guidance.

As a Sudoku enthusiast, I am also excited to work with Professor Tom Kelly on problems related to
Latin squares. My first exposure to Latin squares came in a graduate combinatorics class assignment,
where I as amazed of their behavior exhibit under probabilistic and entropy-based settings.
Professor Kelly has done extensive works on random Latin squares, and I am excited to learn more
about this area under his mentorship.

Moreover, I want to collaborate with faculty beyond the math department. In particular, Professor
Dana Randall's research on the behavior of randomize algorithms in combinatorial problems is
particularly appealing to me, especially her work on the stable marriage problem\cite{inproceedings}
which studies the convergence time of using random walk to generate stable matchings. I would like
to leverage my specialization in combinatorics to contribute to her research in this area.

I have only begun to scratch the surface of combinatorics, and there are still a lot of topics to
explore. I am excited about the prospect of joining Georgia Tech's Ph.D program in ACO to receive
comprehensive training in mathematical research, and I am confident in my ability to evolve into a
capable researcher and make meaningful contributions.  

\newpage


\bibliographystyle{plain}
\bibliography{references}
\end{document}
