\documentclass[12pt]{article}
\usepackage[utf8]{inputenc}
\usepackage{amsmath}
\usepackage{amssymb}
\usepackage{geometry}
\geometry{letterpaper, margin=1in}
\usepackage{parskip}
\usepackage{mathpazo}
\usepackage{titling}
\usepackage{cite}  

\setlength{\droptitle}{-0.5in}

\parindent=20pt

\title{Statement of Purpose}
\author{Ray Tsai}
\date{}

\begin{document}

\maketitle

\vspace{-0.25in}

I am applying to the Ph.D program in Algorithm, Combinatorics, and Optimization (ACO) at Carnegie
Mellon University to pursue my interest in combinatorics, especially in extremal graph theory and
its applications to other areas. 

As a mathematics-computer science major at UC San Diego, I dedicated most of my undergraduate
studies to mathematics, primarily through honors and graduate-level coursework. While I found beauty
across fields in mathematics, I was especially drawn to combinatorics due to its deceiving
simplicity and intimate ties to real-world problems. I aspire to become a researcher who not only
specializes in combinatorics but also bridges combinatorics with other disciplines. I believe the
ACO program at Carnegie Mellon University is the ideal place to achieve this goal.

During my sophomore year, I began my research journey by directly reading under Professor Jacques
Verstraete, exploring the vast literature on extremal graph theory. Through the readings, I was
exposed to a variety of powerful techniques for tackling extremal graph problems, from probabilistic
methods to algebraic constructions, which laid the groundwork for my honors thesis centered on the
Double Turán problem. The problem asks for the maximum possible number of edges across $n$ subgraphs
of a complete graph $K_n$ such that no pairwise intersection of these subgraphs contains a specified
forbidden structure. After dedicating my summer to studying the case where the forbidden graph is
non-bipartite, I established a tight upper bound on the number of edges under the stricter condition
that each subgraph is induced. Specifically, I showed that the extremal condition is uniquely
achieved when all subgraphs extremal graphs for the forbidden graph, by recursively expanding the
intersection of all subgraphs. This case serves as a stepping stone for the project, and I am
currently working on the general case. 

As I progressed in my research, I also developed an interest in the computational aspects of
combinatorial problems. I had the opportunity to learn about computational complexity theory
directly from Professor Russell Impagliazzo, studying Arora and Barak's text, Computational
Complexity: A Modern Approach \cite{arora2009computational}. With this foundation, I then joined
Professor Impagliazzo's research group, studying Multicalibration to mitigate unintended biases to
certain subpopulations in learning models from the perspective of complexity theory. We are trying
to connect notions in smooth boosting to multicalibration. By modeling the fairness of algorithms
with random-like structures yielded by Szemeredi's Regularity Lemma, this research brought the
application of extremal combinatorics and complexity theory to a new level and further strengthened
my commitment to utilize my mathematical foundation to solve practical problems.

My undergraduate experiences opened my appetite for mathematical research, but it also humbled me
with the immense breadth and depth of the discipline. Graduate school is a crucial first step to
further proceed in the realm of combinatorics. I plan to continue my research in extremal
combinatorics in graduate school, as well as explore adjacent fields such as discrete geometry,
random graphs, and topics that I can bridge combinatorics with.

The ACO program at Carnegie Mellon University seems tailor-made for my goals. With its emphasis on
combinatorics and interdisciplinary nature with computer science and operation research, I can build
solid foundations in various fields while furthering my studies in combinatorics, which would
realize my aspiration to become a researcher who connects combinatorics with other disciplines. 

The diverse expertise of the faculty members in combinatorics will provide me with comprehensive
training in the field. I am particularly interested in collaborating with Professor Tom Bohman on
problems related to independent sets in graphs. Through my independent studies, I explored the
problem of the lower bound of independent sets in hypergraphs, and I was excited by its extensive
applications both within combinatorics and in broader contexts, information theory is one of which.
Given Professor Bohman's focus on Shannon capacities of graphs, I believe that working with him on
this topic will reveal deeper insights into the connections between combinatorics and information
theory. 

I am also excited by the prospect of collaborating with Professor Boris Bukh. I find his use of
randomized polynomials to tackle bipartite extremal graph
problems\cite{bukh2017rationalexponentsextremalgraph} very intriguing, and I would like to attempt
the Zarankiewicz problem using similar techniques under his mentorship. Additonally, I also want
explore the discrete geometry, and Professor Bukh can provide me with the necessary guidance to
apply my knowledge in extremal graph theory to this area.

Moreover, I want to collaborate with faculty beyond the math department. Although this area diverges
from my current research focus, Professor Tuomas Sandholm's projects on real-world applications of
game theory and market design are particularly appealing to me. His work on price of fairness in
kidney exchange \cite{dickerson2014price}, which involves using graph models to determine the cost
of fairness in such exchanges, is especially intriguing. It prompts me to wonder whether other
combinatorial results could be applied in this domain. I would like to leverage my specialization in
combinatorics to contribute to projects such as heart transplantation policy optimization.

I have only begun to scratch the surface of combinatorics, and there are still a lot of topics to
explore. I am excited about the prospect of joining Carnegie Mellon University's Ph.D program in ACO
to receive comprehensive training in mathematical research, and I am confident in my ability to
evolve into a capable researcher and make meaningful contributions.  

\newpage

\bibliographystyle{plain}
\bibliography{references}
\end{document}
