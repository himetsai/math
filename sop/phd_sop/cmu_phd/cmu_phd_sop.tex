\documentclass[12pt]{article}
\usepackage[utf8]{inputenc}
\usepackage{amsmath}
\usepackage{amssymb}
\usepackage{geometry}
\geometry{letterpaper, margin=1in}
\usepackage{parskip}
\usepackage{mathpazo}
\usepackage{titling}
\usepackage{cite}  

\setlength{\droptitle}{-0.5in}

\parindent=20pt

\title{Statement of Purpose}
\author{Ray Tsai}
\date{}

\begin{document}


\maketitle

\vspace{-0.25in}

I am applying to the Ph.D program in Algorithm, Combinatorics, and Optimization (ACO) at Carnegie
Mellon University to pursue my interest in combinatorics, especially in extremal graph theory and
its applications to adjacent fields like theoretical computer science.

As a mathematics-computer science major at UC San Diego, I dedicated most of my undergraduate
studies to mathematics, primarily through honors and graduate-level coursework. While I found beauty
across fields in mathematics, I was especially drawn to combinatorics due to its deceiving
simplicity and intimate ties to real-world problems. I aspire to become a researcher who not only
specializes in combinatorics but also bridges combinatorics with other disciplines. I believe the
ACO program at Carnegie Mellon University is the ideal place to achieve this goal.

During my sophomore year, I began my research journey by undertaking independent study under
Professor Jacques Verstraete, exploring the extensive literature on extremal graph theory. Through
the readings, I was exposed to a variety of powerful techniques for tackling extremal graph
problems, such as probabilistic methods, stability, and finite geometric constructions, which laid
the groundwork for my future research. Currently, I am working on my honors thesis centering on the
Double Turán problem, which asks for the maximum possible number of edges across $n$ subgraphs of a
complete graph $K_n$ such that no pairwise intersection of these subgraphs contain a specified
forbidden structure. After dedicating my summer to studying the triangle-free case, I established a
tight upper bound on the number of edges under the stricter condition that each subgraph is induced.
Specifically, I showed that the extremal condition is uniquely achieved when all subgraphs are
complete bipartite graphs, by recursively expanding the intersection of all subgraphs. This case
serves as a stepping stone for the project, and I am currently working on the general triangle-free
case. 

As I advanced in my studies, I also grew interested in computational complexity theory, which
studies combinatorial problems with a computational lens grounded in real-world problems. After a
quarter spent working through Sanjeev Arora and Boaz Barak's \textit{Computational Complexity: A
Modern Approach}\cite{arora2009computational}, I joined Professor Russell Impagliazzo's research
group, studying Multicalibration to address unintended bias in learning models from the perspective
of complexity theory. The project opened my eyes to the unexpected connections between theoretical
computer science and combinatorics, as it brought the application of combinatorics to a new level by
modeling the fairness of algorithms with the random-like structures yielded by Szemeredi's
Regularity Lemma. Through the project, I realized the boundless potential of real-world application
of combinatorial tools and it adds another layer of meaning to my interest in combinatorics.

My undergraduate experiences opened my appetite for mathematical research, but it also humbled me
with the immense breadth and depth of the discipline. Graduate school is a crucial first step to
further proceed in the realm of combinatorics. I plan to continue my research in extremal
combinatorics in graduate school, as well as explore adjacent fields such as discrete geometry,
random graphs, and topics that I can bridge combinatorics with.

The ACO program at Carnegie Mellon University seems tailor-made for my goals. With its emphasis on
combinatorics and interdisciplinary nature with computer science and operation research, I can build
solid foundations in various fields while furthering my studies in combinatorics, which would
realize my aspiration to become a researcher who connects combinatorics with other disciplines. 

The diverse expertise of the faculty members in combinatorics will provide me with comprehensive
training in the field. I am particularly interested in collaborating with Professor Tom Bohman on
problems related to independent sets in graphs. Through my independent studies, I explored the
problem of the lower bound of independent sets in hypergraphs, and I was excited by its extensive
applications both within combinatorics and in broader contexts, information theory is one of which.
Given Professor Bohman's focus on Shannon capacities of graphs, I believe that working with him on
this topic will reveal deeper insights into the connections between combinatorics and information
theory. I am also excited by the prospect of collaborating with Professor Boris Bukh. Although my
background in discrete geometry is limited, I am intrigued by the easy-to-state yet hard-to-solve
nature of problems Professor Bukh worked on, such as his past works on empty axis-paralleled
boxes\cite{bukh2021axisparallelboxes} and digital almost nets\cite{bukh2022digitalnets}.

Moreover, I want to collaborate with faculty beyond the math department. Although this area diverges
from my current research focus, Professor Tuomas Sandholm's projects on real-world applications of
game theory and market design are particularly appealing to me. His work on price of fairness in
kidney exchange \cite{dickerson2014price}, which involves using graph models to determine the cost
of fairness in such exchanges, is especially intriguing. It prompts me to wonder whether other
combinatorial results could be applied in this domain. I would like to leverage my specialization in
combinatorics to contribute to projects such as heart transplantation policy optimization.

I have only begun to scratch the surface of combinatorics, and there are still a lot of topics to
explore. I am excited about the prospect of joining Carnegie Mellon University's Ph.D program in ACO
to receive comprehensive training in mathematical research, and I am confident in my ability to
evolve into a capable researcher and make meaningful contributions.  

\newpage

\bibliographystyle{plain}
\bibliography{references}
\end{document}
