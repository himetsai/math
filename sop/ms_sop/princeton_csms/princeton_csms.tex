\documentclass[12pt]{article}
\usepackage[utf8]{inputenc}
\usepackage{amsmath}
\usepackage{amssymb}
\usepackage{geometry}
\geometry{letterpaper, margin=1in}
\usepackage{parskip}
\usepackage{mathpazo}
\usepackage{titling}
\usepackage{cite}  

\setlength{\droptitle}{-0.5in}

\parindent=20pt

\title{Statement of Purpose}
\author{Ray Tsai}
\date{}

\begin{document}


\maketitle

\vspace{-0.25in}

I am applying to the Master's program in Computer Science at Princeton University to deepen my
understanding of theoretical computer science, particularly in the combinatorial side of the study.
My goal is to become a researcher working at the intersection of computer science and mathematics,
finding ways to bridge the two fields. I intend to pursue a Ph.D. after the master's program to
further this objective.

As a mathematics-computer science major at UC San Diego, I placed most of my attention to
mathematics in my undergraduate studies, primarily through honors and graduate-level coursework.
While I appreciate the beauty of pure mathematics, I wasn't satisfied pursuing abstract mathematics
for its own sake. I instead looked for a field in mathematics with a closer connection to the real
world and turned to graph theory, the study of relationships that can model a wide range of
practical problems effectively.

During my sophomore year, I began my research journey by directly reading under Professor Jacques
Verstraete, exploring the vast literature on extremal graph theory. Through the readings, I was
exposed to a variety of powerful techniques for tackling extremal graph problems, from probabilistic
methods to algebraic constructions, which laid the groundwork for my honors thesis centered on the
Double Turán problem. The problem asks for the maximum possible number of edges across $n$ subgraphs
of a complete graph $K_n$ such that no pairwise intersection of these subgraphs contains a specified
forbidden structure. After dedicating my summer to studying the case where the forbidden graph is
non-bipartite, I established a tight upper bound on the number of edges under the stricter condition
that each subgraph is induced. Specifically, I showed that the extremal condition is uniquely
achieved when all subgraphs extremal graphs for the forbidden graph, by recursively expanding the
intersection of all subgraphs. This case serves as a stepping stone for the project, and I am
currently working on the general case. 

As I progressed in my research, I also developed an interest in the computational aspects of
combinatorial problems. I had the opportunity to learn about computational complexity theory
directly from Professor Russell Impagliazzo, studying Arora and Barak's text, Computational
Complexity: A Modern Approach \cite{arora2009computational}. With this foundation, I then joined
Professor Impagliazzo's research group, studying Multicalibration to mitigate unintended biases to
certain subpopulations in learning models from the perspective of complexity theory. We are trying
to connect notions in smooth boosting to multicalibration. By modeling the fairness of algorithms
with random-like structures yielded by Szemeredi's Regularity Lemma, this research brought the
application of extremal combinatorics and complexity theory to a new level and further strengthened
my commitment to utilize my mathematical foundation to solve practical problems.

With my mathematics-heavy background, I aim to build a deeper understanding of computer science
concepts to support future Ph.D. work. Princeton's Master's program offers an ideal path for this. I
look forward to gaining a rigorous grounding in computational complexity through courses like COS
522, and I aim to formalize my knowledge of Multicalibration by taking Fairness in Machine Learning
(COS 534), a subject I have mostly self-studied. Additionally, I plan to learning more about the
probabilistic and combinatorial side of computer science, and taking courses like Probabilistic
Algorithms (COS 527), Analytic Combinatorics (COS 488), and Information Theory (COS 585) would be a
great start.

Moreover, I am drawn to the research opportunities at Princeton and aim to pursue the M.S.E. track,
particularly under the guidance of Professor Zeev Dvir on coding theory and combinatorial geometry.
His extensively application of algebraic and combinatorial techniques to coding theory aligns with
my intended research direction. I am especially interested in his work on 2-query Locally
Correctable Code over $\mathbb{F}_p$\cite{6108225}, where he employs ideas from additive
combinatorics to prove a coding theory result and then uses this result to address another statement
in incidence geometry. This elegant interplay between different areas of mathematics and computer
science inspires me to study such interdisciplinary research. I believe that working with Professor
Dvir will further enlighten me on of these hidden connections.

Apart from my coursework and research, I am excited to take on teaching assistant roles. While I
have not yet had the opportunity to formally teach during my first three years of undergraduate
studies, my experience as a private tutor for high school students has been invaluable in shaping my
ability to break down complex concepts into simple and engaging explanations. Witnessing the moment
when a student fully understanding a new concept has been especially rewarding. This winter, I will
be tutoring the Design and Analysis of Algorithms course at UC San Diego, and I look forward to the
opportunity to serve as a TA for courses at Princeton that align with my interests, such as
Introduction to Graph Theory (COS 342).

I have only begun to scratch the surface of theoretical computer science, and there are still a lot
of topics to explore. I am excited about the prospect of joining Princeton's Master's in Computer
Science program to expose me to a broader range of topics in computer science through rigorous
coursework and research opportunities. With the mathematical foundation I built during my
undergraduate studies, I am confident in my ability to excel in the program and evolve into a
capable researcher. I look forward to contributing to the vibrant academic community at Princeton
and bridging the worlds of mathematics and computer science.

\newpage

\bibliographystyle{plain}
\bibliography{references}
\end{document}
