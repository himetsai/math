\documentclass[12pt]{article}
\usepackage[utf8]{inputenc}
\usepackage{amsmath}
\usepackage{amssymb}
\usepackage{geometry}
\geometry{letterpaper, margin=1in}
\usepackage{parskip}
\usepackage{mathpazo}
\usepackage{titling}

\setlength{\droptitle}{-0.5in}

\parindent=20pt

\title{Diversity Statement}
\author{Ray Tsai}
\date{}

\begin{document}


\maketitle

\vspace{-0.25in}

\subsection*{Prompt}

Stanford University welcomes graduate applications from individuals with a broad range of
experiences, interests, and backgrounds who would contribute to our community of scholars. We invite
you to share the lived experiences, demonstrated values, perspectives, and/or activities that shape
you as a scholar and would help you to make a distinctive contribution to Stanford University.

\subsection*{Response}

Growing up in Taiwan, I didn't receive a particularly inspiring education. Each day at school was a
marathon of passive lectures, absorbing mountains of information, and the relentless pursuit of high
scores on standardized exams. While this environment provided me with a solid academic foundation,
it left little room for genuine curiosity and exploration. Like many of my peers, I find myself
disconnected from the purpose of my studies, viewing knowledge as a mere tool for achieving academic
success.

Coming to the United States for higher education transformed my perspective. Here, learning was no
longer about passive absorption but active engagement, and the motivation and application behind the
content were thoughtfully explained. This shift allowed me to see that behind every piece of
knowledge, whether simple or complex, lies a core idea and motivation. After understanding the
motivation behind my studies, I soon found myself in love with learning and picked up a major in
mathematics, a subject I had previously found dull and intimidating. 

This stark contrast between the two educational paradigms makes me realize that quality education is
not evenly distributed, which sparked my desire to democratize access to knowledge. It is a
privilege to study abroad, and I feel a responsibility to share the wealth of my education with the
community that had nurtured me. This led to the creation of my Chinese math blog, where I aim to
share high-level mathematical concepts I find interesting to a general Taiwanese audience in an
approachable way. My goal was to show people that difficult subjects like math are not as
intimidating as they may seem and can be fun when grasping the core ideas behind them. Though modest
in scale, my blog represents a commitment to sharing the power of accessible education, and I hope
to inspire others the way college education inspired me.

At Stanford, I aim to further this mission of distributing quality education. I believe even
abstract subjects like mathematics and theoretical computer science can be made approachable and
inspiring with the right introductions. Drawing on my experiences with contrasting educational
systems, I hope to bring this perspective to the classrooms and research groups at Stanford,
bridging the knowledge gap between different educational backgrounds and disciplines and building a
more inclusive academic environment. 

\newpage

\end{document}
