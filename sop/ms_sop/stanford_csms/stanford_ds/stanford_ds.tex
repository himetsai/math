\documentclass[12pt]{article}
\usepackage[utf8]{inputenc}
\usepackage{amsmath}
\usepackage{amssymb}
\usepackage{geometry}
\geometry{letterpaper, margin=1in}
\usepackage{parskip}
\usepackage{mathpazo}
\usepackage{titling}

\setlength{\droptitle}{-0.5in}

\parindent=20pt

\title{Diversity Statement}
\author{Ray Tsai}
\date{}

\begin{document}


\maketitle

\vspace{-0.25in}

\subsection*{Prompt}

Stanford University welcomes graduate applications from individuals with a broad range of
experiences, interests, and backgrounds who would contribute to our community of scholars. We invite
you to share the lived experiences, demonstrated values, perspectives, and/or activities that shape
you as a scholar and would help you to make a distinctive contribution to Stanford University.

\subsection*{Response}

Growing up in Taiwan, I didn't receive a particularly inspiring education. Each day at school was a
marathon of passive lectures, absorbing mountains of information, and the relentless pursuit of high
scores on standardized exams. While this environment provided me a solid academic foundation, it
left little room for genuine curiosity and exploration. Like many of my peers, I find myself
disconnected to the purpose of my studies, viewing knowledge as a mere tool for achieving academic
success.

Coming to the United States for higher education transformed my perspective. Here, learning was no
longer about passive absorption but active engagement, with the intention behind every step
thoughtfully explained. This shift allowed me to see that behind every piece of knowledge, whether
simple or complex, lies a core idea and motivation. After understanding the motivation behind my
studies, I soon found myself in love with learning and picked up a major in mathematics, a subject I
had previously found dull and intimidating. 

This stark contrast between the two educational paradigms sparked my desire to democratize access to
knowledge. Recognizing the privilege of studying abroad, I felt a responsibility to give back to the
community that had shaped me. This led to the creation of my Chinese math blog, where I aim to share
high-level mathematical concepts I find interesting to a general Taiwanese audience. My goal was to
demystify the subject, showing that difficult subjects like math are not as intimidating as they may
seem, inspiring others in the same fashion my college education inspired me. Though modest in scale,
my blog represents a commitment to sharing the transformative power of accessible education.

At Stanford, I aim to further this mission by bridging gaps in understanding and fostering
inclusivity in academic spaces. I believe even abstract subjects like mathematics and theoretical
computer science can be made approachable and inspiring with the right introductions. Drawing on my
experiences with contrasting educational systems, I hope to bring this perspective to the classrooms
and research groups at Stanford, bridging the knowledge gap between different educational
backgrounds and disciplines and build a more inclusive academic environment. 

\newpage

\end{document}
