\documentclass[12pt]{article}
\usepackage[utf8]{inputenc}
\usepackage{amsmath}
\usepackage{amssymb}
\usepackage{geometry}
\geometry{letterpaper, margin=1in}
\usepackage{parskip}
\usepackage{mathpazo}
\usepackage{titling}
\usepackage{cite}  

\setlength{\droptitle}{-0.5in}

\parindent=20pt

\title{Statement of Purpose}
\author{Ray Tsai}
\date{}

\begin{document}


\maketitle

\vspace{-0.25in}

I am applying to the Master's program in Computer Science at Stanford University, specializing in
theoretical computer science to translate my mathematical foundation, particularly in combinatorics,
to abilities that can address real-world problems. My primary interests lie in combinatorial
optimization, where the interplay between theory and application is most apparent to me, but I'm
also open to exploring other areas like coding theory and cryptography. I aspire to develop
innovative algorithms and solutions to address challenges in areas like resource optimization.

As a mathematics-computer science major at UC San Diego, I placed most of my attention to
mathematics in my undergraduate studies, primarily through honors and graduate-level coursework.
While I appreciate the beauty of pure mathematics, I wasn't satisfied pursuing abstract mathematics
for its own sake. I instead looked for a field in mathematics with a closer connection to the real
world and turned to graph theory, the study of relationships that can model a wide range of
practical problems effectively.

During my sophomore year, I began my research journey by directly reading under Professor Jacques
Verstraete, exploring the vast literature on extremal graph theory. Through the readings, I was
exposed to a variety of powerful techniques for tackling extremal graph problems, from probabilistic
methods to algebraic constructions, which laid the groundwork for my honors thesis centered on the
Double Turán problem. The problem asks for the maximum possible number of edges across $n$ subgraphs
of a complete graph $K_n$ such that no pairwise intersection of these subgraphs contains a specified
forbidden structure. After dedicating my summer to studying the case where the forbidden graph is
non-bipartite, I established a tight upper bound on the number of edges under the stricter condition
that each subgraph is induced. Specifically, I showed that the extremal condition is uniquely
achieved when all subgraphs extremal graphs for the forbidden graph, by recursively expanding the
intersection of all subgraphs. This case serves as a stepping stone for the project, and I am
currently working on the general case. 

As I progressed in my research, I also developed an interest in the computational aspects of
combinatorial problems. I had the opportunity to learn about computational complexity theory
directly from Professor Russell Impagliazzo, studying Arora and Barak's text, Computational
Complexity: A Modern Approach \cite{arora2009computational}. With this foundation, I then joined
Professor Impagliazzo's research group, studying Multicalibration to mitigate unintended biases to
certain subpopulations in learning models from the perspective of complexity theory. We are trying
to connect notions in smooth boosting to multicalibration. By modeling the fairness of algorithms
with random-like structures yielded by Szemeredi's Regularity Lemma, this research brought the
application of extremal combinatorics and complexity theory to a new level and further strengthened
my commitment to utilize my mathematical foundation to solve practical problems.

Stanford's Master's program offers the ideal path to translate my mathematical expertise into a
career focused on developing practical algorithms and solutions. I aspire to contribute to fields
where combinatorial tools are applied extensively, and taking courses like Matching Theory (MS\&E
319), Combinatorial Optimization (CS 261), Algebraic Error Correcting Codes (CS 250), and Advanced
Topics in Cryptography (CS 355) would provide me with the necessary background of combinatorial
applications to computer science.

Moreover, I am looking forward to the research opportunities at Stanford and aim to pursue
distinction in research. I am most curious to investigate approximation algorithms. Classic
combinatorial optimization problems, such as finding Hamiltonian cycles and matchings in graphs,
frequently arise in real-world contexts, where exact solutions are often computationally infeasible
and rely on approximation algorithms for pratical solutions. Thus I hope to utilize my specialized
knowledge in combinatorics to advance the Pareto frontier of these algorithms, pushing the
boundaries of their efficiency and accuracy. Professor Aviad Rubinstein's expertise in approximation
algorithms and algorithmic game theory aligns with my research interest. I am especially intrigued
by his algorithm for approximating maximum matching size in
\cite{doi:10.1137/1.9781611977554.ch151}, which beats the $\frac{1}{2}$-approximation barrier
without even requiring a full traversal of the graph! Given the ubiquity of combinatorial problems
in various applications, I am excited to explore the possibilities of improving these algorithms
under his guidance, making them more accessible for practical applications. If possible, I would
also like to collaborate with Professor Itai Ashlagi from the Management Science and Engineering
department. He has contributed significantly to the area of market design and matching markets, and
I am particularly drawn to his work on kidney exchange. He previously used integer programming to
find long chains in kidney exchange that maximize kidney
transplants\cite{doi:10.1073/pnas.1421853112}, and I am curious if approximate algorithms can be
used to tackle other NP-hard problems regarding this urgent topic.

Additionally, I am interested in studying other areas of theoretical computer science that have
significant connections with combinatorics, such as cryptography and coding theory. Professor
Li-Yang Tan's work on the tightness of Impagliazzo's hardcore theorem
\cite{blanc2024samplecomplexitysmoothboosting} revealed the limitations in hardness amplification
for pseudorandom generators (PRG), and I want to study its implications on the security and
efficiency of PRGs. On the other hand, I also want to learn coding theory from Professor Mary
Wootters. I find her work on characterizing certain Secret Sharing Schemes with coding
theory\cite{blackwell2023characterizationoptimalratelinearhomomorphic} to be enlightening, and I
would like to explore other connections between coding theory and cryptography under her guidance.

There are many Stanford faculty members that I crave to learn from, and, though I'm unlikely to work
with all of them, I believe they will provide me with a deep understanding in the connection between
math and computer science. With the mathematical foundation I built during my undergraduate studies,
I am confident in my ability to excel in the program and evolve into a capable problem-solver. I
look forward to contributing to the vibrant academic community at Stanford and connecting theory
with the real-world.

\newpage

\bibliographystyle{plain}
\bibliography{references}
\end{document}
