\documentclass[12pt]{article}
\usepackage[utf8]{inputenc}
\usepackage{amsmath}
\usepackage{amssymb}
\usepackage{geometry}
\geometry{letterpaper, margin=1in}
\usepackage{parskip}
\usepackage{mathpazo}
\usepackage{titling}
\usepackage{cite}  

\setlength{\droptitle}{-0.5in}

\parindent=20pt

\title{Academic Statement}
\author{Ray Tsai}
\date{}

\begin{document}


\maketitle

\vspace{-0.25in}

I am applying to the Master's program in Computer Science at University of Illinois Urbana-Champaign
(UIUC) to deepen my understanding of theoretical computer science, particularly in complexity and
the combinatorial side of the study. My goal is to become a researcher working at the intersection
of computer science and mathematics, finding ways to bridge the two fields. I intend to pursue a
Ph.D. after the master's program to further this objective.

As a mathematics-computer science major at UC San Diego, I dedicated most of my attention to
mathematics in my undergraduate studies, primarily through honors and graduate-level coursework.
While I appreciate the beauty of pure mathematics, I wasn't satisfied pursuing abstract mathematics
for its own sake. I instead looked for a field in mathematics with a closer connection to the real
world and turned to graph theory, the study of relationships that can model a wide range of
practical problems effectively.

During my sophomore year, I began my research journey by undertaking independent study under
Professor Jacques Verstraete, exploring the extensive literature on extremal graph theory. Through
the readings, I was exposed to a variety of powerful techniques for tackling extremal graph
problems, such as probabilistic methods, stability, and finite geometric constructions, which laid
the groundwork for my future research. Currently, I am working on my honors thesis centering on the
Double Turán problem, which asks for the maximum possible number of edges across $n$ subgraphs of a
complete graph $K_n$ such that no pairwise intersection of these subgraphs contain a specified
forbidden structure. After dedicating my summer to studying the triangle-free case, I established a
tight upper bound on the number of edges under the stricter condition that each subgraph is induced.
Specifically, I showed that the extremal condition is uniquely achieved when all subgraphs are
complete bipartite graphs, by recursively expanding the intersection of all subgraphs. This case
serves as a stepping stone for the project, and I am currently working on the general triangle-free
case. 

As I advanced in my studies, I also grew interested in computational complexity theory, which
studies combinatorial problems with a computational lens grounded in real-world problems. After a
quarter spent working through Sanjeev Arora and Boaz Barak's \textit{Computational Complexity: A
Modern Approach}\cite{arora2009computational}, I joined Professor Russell Impagliazzo's research
group, studying Multicalibration to address unintended bias in learning models from the perspective
of complexity theory. The project opened my eyes to the unexpected connections between theoretical
computer science and combinatorics, as it brought the application of combinatorics to a new level by
modeling the fairness of algorithms with the random-like structures yielded by Szemeredi's
Regularity Lemma. Through the project, I realized the boundless potential of real-world application
of combinatorial tools and it adds another layer of meaning to my interest in combinatorics.

With my mathematics-heavy background, I aim to build a deeper understanding of computer science
concepts to support my future research endeavor. UIUC's Master's program offers an ideal path for
this. I look forward to getting a rigorous grounding in computational complexity through courses
like CS 579. Additionally, I plan to learn more about the combinatorial side of computer science,
and taking courses like Approximation algorithms (CS 583), Combinatorial Optimization (CS 586), and,
if offered, Expansion, Codes, and Optimization both Classical and Quantum (CS 598) would be a great
start. My interest also extends to Algorithmic Game Theory (CS 580), as it applies combinatorial and
algorithmic techniques to analyze strategic interactions in real world settings.

Moreover, I am drawn to the research opportunities at UIUC and aim to write a master's thesis,
ideally under the guidance of Professor Fernando Granha Jeronimo. His work on coding theory and
expanders has intersects significantly with combinatorics, which aligns well with my intended
research direction. Another faculty member I would like to work with is Professor Karthekeyan
Chandrasekaran. I find his old research topic on deterministic algorithms for the Lovász Local
Lemma\cite{chandrasekaran2019deterministicalgorithmslovaszlocal} especially interesting, it develops
a deterministic algorithm to construct solutions that satisfy a probabilistic statement. I am
looking forward to study more about this topic under his mentorship.

I have only begun to scratch the surface of theoretical computer science, and there are still a lot
of topics to explore. I am excited about the prospect of joining UIUC's Master's in Computer
Science program to expose me to a broader range of topics in computer science through rigorous
coursework and research opportunities. With the mathematical foundation I built during my
undergraduate studies, I am confident in my ability to excel in the program and evolve into a
capable researcher. I look forward to contributing to the vibrant academic community at UIUC
and bridging the worlds of mathematics and computer science.

\newpage

\bibliographystyle{plain}
\bibliography{references}
\end{document}
