\documentclass[12pt]{article}
\usepackage[utf8]{inputenc}
\usepackage{amsmath}
\usepackage{amssymb}
\usepackage{geometry}
\geometry{letterpaper, margin=1in}
\usepackage{parskip}
\usepackage{mathpazo}
\usepackage{titling}
\usepackage{cite}  

\setlength{\droptitle}{-0.5in}

\parindent=20pt

\title{Academic Statement}
\author{Ray Tsai}
\date{}

\begin{document}


\maketitle

\vspace{-0.25in}

I am applying to the Master's program in Computer Science at University of Illinois Urbana-Champaign
(UIUC) to deepen my understanding of theoretical computer science, particularly in coding theory and
combinatorial optimization. My goal is to become a researcher working at the intersection of
computer science and mathematics, finding ways to bridge the two fields. I intend to pursue a Ph.D.
after the master's program to further this objective.

As a mathematics-computer science major at UC San Diego, I placed most of my attention to
mathematics in my undergraduate studies, primarily through honors and graduate-level coursework.
While I appreciate the beauty of pure mathematics, I wasn't satisfied pursuing abstract mathematics
for its own sake. I instead looked for a field in mathematics with a closer connection to the real
world and turned to graph theory, the study of relationships that can model a wide range of
practical problems effectively.

During my sophomore year, I began my research journey by directly reading under Professor Jacques
Verstraete, exploring the vast literature on extremal graph theory. Through the readings, I was
exposed to a variety of powerful techniques for tackling extremal graph problems, from probabilistic
methods to algebraic constructions, which laid the groundwork for my honors thesis centered on the
Double Turán problem. The problem asks for the maximum possible number of edges across $n$ subgraphs
of a complete graph $K_n$ such that no pairwise intersection of these subgraphs contains a specified
forbidden structure. After dedicating my summer to studying the case where the forbidden graph is
non-bipartite, I established a tight upper bound on the number of edges under the stricter condition
that each subgraph is induced. Specifically, I showed that the extremal condition is uniquely
achieved when all subgraphs extremal graphs for the forbidden graph, by recursively expanding the
intersection of all subgraphs. This case serves as a stepping stone for the project, and I am
currently working on the general case. 

As I progressed in my research, I also developed an interest in the computational aspects of
combinatorial problems. I had the opportunity to learn about computational complexity theory
directly from Professor Russell Impagliazzo, studying Arora and Barak's text, Computational
Complexity: A Modern Approach \cite{arora2009computational}. With this foundation, I then joined
Professor Impagliazzo's research group, studying Multicalibration to mitigate unintended biases to
certain subpopulations in learning models from the perspective of complexity theory. We are trying
to connect notions in smooth boosting to multicalibration. By modeling the fairness of algorithms
with random-like structures yielded by Szemeredi's Regularity Lemma, this research brought the
application of extremal combinatorics and complexity theory to a new level and further strengthened
my commitment to utilize my mathematical foundation to solve practical problems.

With my mathematics-heavy background, I aim to build a deeper understanding of computer science
concepts to support my future research endeavor. UIUC's Master's program offers an ideal path for
this. I plan to learn more about the combinatorial side of the study, and taking courses like
Approximation Algorithms (CS 583), Combinatorial Optimization (CS 586), Coding Theory (ECE 556),
and, if offered, Expansion, Codes, and Optimization both Classical and Quantum (CS 598) would be a
great start.

Moreover, I am drawn to the opportunities to work with the UIUC faculty and aim to write a master's
thesis, particularly under Professor Fernando Granha Jeronimo. His work on coding theory and
expanders has intersects significantly with combinatorics, which aligns well with my intended
research direction. If possible, I would also like to join Professor Olgica Milenkovic's
Interdisciplinary Data Processing Group under the Electrical \& Computer Engineering department.
Their research focus on constructing and analyzing codes on graphs seems very interesting, and I
believe I can contribute to their research with my background in graph theory.

Another faculty member I would like to work with is Professor Ruta Mehta on envy-freen item
allocations. Her paper on EFX orientations shows a surprising connection between the envy-freeness
up to any good (EFX) property and chromatic number of a
graph\cite{zeng2024structureefxorientationsgraphs}. This topic is closely related to back background
in both extremal graph theory and complexity theory, and I believe my specialized knowledge can be
applied to this area. I also want to work with Professor Karthekeyan Chandrasekaran under the
Department of Industrial and Enterprise Systems Engineering. I find his old research topic on
deterministic algorithms for the Lovász Local
Lemma\cite{chandrasekaran2019deterministicalgorithmslovaszlocal} especially interesting, as it
develops a deterministic algorithm to construct solutions that satisfy a probabilistic statement. I
am looking forward to study more about this topic under his mentorship. 

I have only begun to scratch the surface of theoretical computer science, and there are still a lot
of topics to explore. Research at UIUC has significant intersections with both my academic
background and my research interests, which will build me a solid foundation while I shift my focus
from mathematics to computer science. I am confident in my ability to excel in the program and
evolve into a capable researcher, and I look forward to contributing to the vibrant academic
community at UIUC and bridging the worlds of mathematics and computer science.

\newpage

\bibliographystyle{plain}
\bibliography{references}
\end{document}
