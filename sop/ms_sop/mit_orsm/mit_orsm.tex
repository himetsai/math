\documentclass[12pt]{article}
\usepackage[utf8]{inputenc}
\usepackage{amsmath}
\usepackage{amssymb}
\usepackage{geometry}
\geometry{letterpaper, margin=1in}
\usepackage{parskip}
\usepackage{mathpazo}
\usepackage{titling}
\usepackage{cite}  

\setlength{\droptitle}{-0.5in}

\parindent=20pt

\title{Statement of Purpose}
\author{Ray Tsai}
\date{}

\begin{document}


\maketitle

\vspace{-0.25in}

I am applying to the Master's program in Operations Research at MIT to translate my mathematical
foundation, particularly in combinatorics, to abilities that can address real-world problems. My
primary interests lie in combinatorial and network optimization, where the interplay between theory
and application is most apparent to me, and I envision myself contributing to fields like
transportation systems and resource allocation in the industry.

As a mathematics-computer science major at UC San Diego, I placed most of my attention to
mathematics in my undergraduate studies, primarily through honors and graduate-level coursework.
While I appreciate the beauty of pure mathematics, I wasn't satisfied pursuing abstract results for
its own sake. I instead looked for a field in mathematics with a closer connection to the real world
and turned to graph theory, the study of relationships that can model a wide range of practical
problems effectively.

During my sophomore year, I began my research journey by directly reading under Professor Jacques
Verstraete, exploring the vast literature on extremal graph theory. Through the readings, I was
exposed to a variety of powerful techniques for tackling extremal graph problems, from probabilistic
methods to algebraic constructions, which laid the groundwork for my honors thesis centered on the
Double Turán problem. The problem asks for the maximum possible number of edges across $n$ subgraphs
of a complete graph $K_n$ such that no pairwise intersection of these subgraphs contains a specified
forbidden structure. After dedicating my summer to studying the case where the forbidden graph is
non-bipartite, I established a tight upper bound on the number of edges under the stricter condition
that each subgraph is induced. Specifically, I showed that the extremal condition is uniquely
achieved when all subgraphs are extremal graphs for the forbidden graph, by recursively expanding
the intersection of all subgraphs. This case serves as a stepping stone for the project, and I am
currently working on the general case. 

As I progressed in my research, I also developed an interest in the computational aspects of
combinatorial problems. I had the opportunity to learn about computational complexity theory
directly from Professor Russell Impagliazzo, studying Arora and Barak's text, Computational
Complexity: A Modern Approach \cite{arora2009computational}. With this foundation, I then joined
Professor Impagliazzo's research group, studying Multicalibration to mitigate unintended biases to
certain subpopulations in learning models from the perspective of complexity theory. We are trying
to connect notions in smooth boosting to multicalibration. By modeling the fairness of algorithms
with random-like structures yielded by Szemeredi's Regularity Lemma, this research brought the
application of extremal combinatorics and complexity theory to a new level and further strengthened
my commitment to utilize my mathematical foundation to solve practical problems.

MIT Operations Research program's rigorous theoretical training and emphasis on practical,
interdisciplinary applications perfectly aligns with my goal. I aim to contribute to fields where
combinatorial tools are applied extensively, and taking courses like Combinatorial Optimization
(18.433), Network Optimization (15.082J), Logistics Systems (15.770J), along with the extensive
operations management courses would provide with the necessary background of combinatorial
applications to real-world problems. 

I am most drawn to the opportunities to conduct independent research with MIT's faulty members,
particularly in network optimization. This field aligns best with my combinatorics background and
offers versatile applications. I am especially keen to work with Professor James Orlin. He is the
leading expert in network and combinatorial optimization, and I believe his guidance would be
invaluable for me to convert my mathematical foundation into applicable skills. One research topic I
am eager to explore is the optimization of medical evacuation (MEDEVAC) helicopter routing. This
topic is crucially important on its own right, but I am particularly invested in it due to the
increasing geopolitical tension surrounding my home country, Taiwan. More specifically, I am curious
about how the routing of MEDEVAC helicopters can be optimized in extreme situations where the
helicopters are in high demand and potentially face contested airspace, such as a military conflict.
I believe this issue can be modeled as a network optimization problem and effectively analyzed.
Furthermore, I would also like to seek advice from Professor Alexandre Jacquillat. His research on
vertiport planning for urban aerial mobility\cite{10.1287/msom.2022.1148} is highly relevant to
optimizing MEDEVAC helicopter routing in urban environments during a conflict, and his expertise
could provide crucial insights on how to adapt his methodologies to optimize MEDEVAC helicopter
routing.

I have only begun to scratch the surface of operations research, and there are still a lot of topics
to explore. I am excited about the prospect of joining MIT's Master's in Operations Research program
to expose me to a broader range of topics in operations research through rigorous coursework and
extensive research opportunities. With the mathematical foundation I built during my undergraduate
studies, I am confident in my ability to excel in the program and evolve into a capable
problem-solver. I look forward to contributing to the vibrant academic community at MIT and bridging
the worlds of mathematics and real-world problems.

\newpage

\bibliographystyle{plain}
\bibliography{references}
\end{document}
