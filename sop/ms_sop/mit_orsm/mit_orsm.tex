\documentclass[12pt]{article}
\usepackage[utf8]{inputenc}
\usepackage{amsmath}
\usepackage{amssymb}
\usepackage{geometry}
\geometry{letterpaper, margin=1in}
\usepackage{parskip}
\usepackage{mathpazo}
\usepackage{titling}
\usepackage{cite}  

\setlength{\droptitle}{-0.5in}

\parindent=20pt

\title{Statement of Purpose}
\author{Ray Tsai}
\date{}

\begin{document}


\maketitle

\vspace{-0.25in}

I am applying to the Master's program in Operations Research at MIT to translate my mathematical
foundation, particularly in combinatorics, to abilities that can address real-world problems. My
primary interests lie in combinatorial optimization, and I envision myself contributing to fields
like transportation systems and resource allocation in the industry.

As a mathematics-computer science major at UC San Diego, I placed most of my attention to
mathematics in my undergraduate studies, primarily through honors and graduate-level coursework.
While I appreciate the beauty of pure mathematics, I wasn't satisfied pursuing abstract mathematics
for its own sake. I instead looked for a field in mathematics with a closer connection to the real
world and turned to graph theory, the study of relationships that can model a wide range of
practical problems effectively.

During my sophomore year, I began my research journey by directly reading under Professor Jacques
Verstraete, exploring the vast literature on extremal graph theory. Through the readings, I was
exposed to a variety of powerful techniques for tackling extremal graph problems, such as
probabilistic methods, stability, and finite geometric constructions, which laid the groundwork for
my honors thesis centered on the Double Turán problem. The problem asks for the maximum possible
number of edges across $n$ subgraphs of a complete graph $K_n$ such that no pairwise intersection of
these subgraphs contain a specified forbidden structure. After dedicating my summer to studying the
triangle-free case, I established a tight upper bound on the number of edges under the stricter
condition that each subgraph is induced. Specifically, I showed that the extremal condition is
uniquely achieved when all subgraphs are complete bipartite graphs, by recursively expanding the
intersection of all subgraphs. This case serves as a stepping stone for the project, and I am
currently working on the general triangle-free case. 

Expanding my interests to include the computational aspects of combinatorial problems, I had the
opportunity to learn about computational complexity theory directly from Professor Russell
Impagliazzo, studying Arora and Barak's text, Computational Complexity: A Modern Approach
\cite{arora2009computational}. With this foundation, I then joined Professor Impagliazzo's research
group, studying Multicalibration to mitigate unintended biases to certain subpopulations in learning
models from the perspective of complexity theory. This research opened my eyes to the surprising
connections between abstract mathematical concepts and the practical applications of machine
learning. Specifically, the project brought the application of extremal combinatorics and complexity
theory to a new level by modeling the fairness of algorithms with the random-like structures yielded
by Szemeredi's Regularity Lemma, further strengthening my commitment to applying theoretical results
to practical problems.

Operations Research program at MIT provides rigorous theoretical training while emphasizing on
practical implementations, which perfectly aligns with my goal of translating my mathematical
expertise into a career focused on developing practical algorithms and solutions. I aim to
contributing to fields related to transportation networks, where combinatorial tools are applied
extensively, and taking courses like Combinatorial Optimization (18.433), Network Optimization
(15.082J), Logistics Systems (15.770J), and Air Transportation Operations Research (16.763J) would
provide me the necessary background of combinatorial applications and challenges in transportation.

I am particularly drawn to the research conducted at the MIT Operations Research Center, especially
in network optimization and air transportation. One research topic I am eager to explore is the
optimization of medical evacuation (MEDEVAC) helicopter routing --- a critical issue with
significant implications for emergency response scenarios. Born and raised in Taiwan, I am
particularly invested in this topic due to the potential military conflict between China and Taiwan.
Optimizing MEDEVAC helicopter routing could play a vital role in saving lives during such crises,
and I hope to contribute to this area by leveraging Professor James Orlin's expertise in network
optimization and my background in combinatorics. 

Furthermore, I am inspired by Professor Alexandre Jacquillat's extensive work on aircraft routing
and scheduling, particularly his research on vertiport planning for urban aerial
mobility\cite{10.1287/msom.2022.1148}. This work is highly relevant to optimizing MEDEVAC helicopter
routing in urban environments like Taipei during a conflict. By applying these methodologies, I aim
to develop solutions that not only address potential conflicts but also offer a general framework
for MEDEVAC routing in diverse emergency contexts. I believe my research in this area could provide
a blueprint for enhancing emergency medical response capabilities worldwide.

I have only begun to scratch the surface of operations research, and there are still a lot of topics
to explore. I am excited about the prospect of joining MIT's Master's in Operations Research program
to expose me to a broader range of topics in operations research through rigorous coursework and
extensive research opportunities. With the mathematical foundation I built during my undergraduate
studies, I am confident in my ability to excel in the program and evolve into a capable
problem-solver. I look forward to contributing to the vibrant academic community at MIT and bridging
the worlds of mathematics and real-world problems.

\newpage

\bibliographystyle{plain}
\bibliography{references}
\end{document}
