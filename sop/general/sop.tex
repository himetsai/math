\documentclass[12pt]{article}
\usepackage[utf8]{inputenc}
\usepackage{amsmath}
\usepackage{amssymb}
\usepackage{geometry}
\geometry{letterpaper, margin=1in}
\usepackage{parskip}
\usepackage{mathpazo}
\usepackage{titling}

\setlength{\droptitle}{-0.5in}

\parindent=20pt

\title{Statement of Purpose}
\author{Ray Tsai}
\date{}

\begin{document}


\maketitle

\vspace{-0.25in}

I am interested in combinatorics, particularly in extremal graph theory and its connections to
theoretical computer science. 

My curiosity about combinatorics was first sparked in an honors course on graph theory, taught by
Professor Jacques Verstraete. Being my first exposure to combinatorics, the course soon intrigued me
with the unconventional nature of combinatorial problems. Unlike other fields of mathematics,
combinatorial problems often appear as discrete and uncorrelated with each other, but they
nevertheless come with surprising relationships beneath the surface.

I deepened my engagement with combinatorics by taking advanced courses, including a graduate-level
combinatorics sequence as well as seminars that exposed me to current research topics. However, my
interest in the field was truly solidified through research experiences under the mentorship of
Professor Verstraete, where I explored extensive literature on Turán problems and worked on open
problems and conjectures. My honors thesis now centers on the Double Turán problem, which asks for
the maximum possible number of edges in $n$ subgraphs of a complete graph $K_n$, with no pairwise
intersection of these subgraphs contains a certain forbidden structure. Through this work, I have
developed a researcher's mindset and a toolkit that ranges from the probabilistic method to
construction techniques, as well as coding skills for searching counterexamples. After dedicating my
last summer to studying the triangle-free case, I completed the proof with a tighter condition that
each subgraph is induced, which serves as a stepping stone for the general case. Despite the
challenges and obstacles involed in the process, I found myself more motivated than ever and
continued to explore the general case.

Alongside my research in extremal combinatorics, I pursued theoretical computer science, drawn by
its close ties to combinatorics. After a quarter spent working through Sanjeev Arora and Boaz
Barak's \textit{Computational Complexity: A Modern Approach}, I joined a research project under
Professor Russell Impagliazzo on multicalibration, aiming to reduce unintended bias in learning
models. The project opened my eyes to the unexpected connection between algorithm fairness and the
Szemerédi regularity lemma through their ``random-like'' properties. This made me realize the
boundless potential of real-world application of combinatorial tools, adding another layer of
meaning to my interest in combinatorics. Recognizing the extensive real-world applications of
combinatorial tools added new depth to my interest in combinatorics and its potential impact. 

My undergraduate experiences opened my apetite to mathematical research. To feed my curiosity, I
want to dig further into combinatorics, which prompted me to pursue graduate studies in mathematics.
In the long term, I aspire to become a researcher in combinatorics, capable of addressing open
problems with fresh perspectives and making meaningful contributions to the math community. I see
graduate school as a crucial first step toward achieving this goal.

I plan to continue investigating Turán problems and their generalizations in graduate school. An
interesting research direction I would like to follow is to build on my undergraduate research. The
degenerated case of the Double Turán Problem, for which the forbidden structure is bipartite, is
shown to be harder than the non-degenerated case, as seen in the classical Turán problem. Currently,
I lack the expertise to tackle this case, but I would like to further explore the degenerated double
Turán problems as I acquire more tools and insights in my graduate studies. More specifically, I
need to learn more about polynomial equations over finite fields and projective planes, which are
useful tool for constructions in these degenerated cases.

While I have a strong interest in the extremal combinatorics, I do not aim to limit myself to a
single area early in my research career. I plan to explore areas adjacent to combinatorics such as
theoretical computer science, both to bridge my rather specialized knowledge in extremal
combinatorics with other domains and to expand my understanding of the mathematical landscape. I
believe that cultivating a diverse set of perspectives will lead to more fruitful outcomes in my
research career.

[Why this program?]

[Closing Paragraph]

\end{document}
