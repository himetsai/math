\documentclass[12pt]{article}
\usepackage[utf8]{inputenc}
\usepackage{amsmath}
\usepackage{amssymb}
\usepackage{geometry}
\geometry{letterpaper, margin=0.5in, bottom=1in}
\usepackage{parskip}
\usepackage{mathpazo}
\usepackage{titling}

\setlength{\droptitle}{-0.5in}

\parindent=20pt

\title{Statement of Purpose}
\author{Ray Tsai}
\date{}

\begin{document}


\maketitle

\vspace{-0.25in}

My interest lies in combinatorics, particularly in extremal graph theory and its connections
to theoretical computer science.

My curiosity about combinatorics was first sparked in an honors undergraduate course on graph
theory, taught by Professor Jacques Verstraete. Being my first exposure to combinatorics, the course
soon intrigued me with the unconventional nature of combinatorial problems. Unlike other fields of
mathematics, combinatorial problems often appear as discrete and uncorrelated with each other, but
they nevertheless come with surprising relationships beneath the surface.

However, what truly solidified my interest were the subsequent research experiences that the course
has led me to. Researching under Professor Verstraete, I was not only introduced to the vast
literature on Turán problems but also to the joy of interacting with theoretical objects and
revealing their hidden properties. The deep structure properties graphs exhibit under extremal
conditions amaze me. Simple conditions can force an extremal graph to behave in certain ways. Would
the graph lose its structure if we slightly relax the conditions? Does the same structure remain
with a slightly different forbidden graph? These questions drive me to explore extremal graphs under
varying conditions to uncover further information.

On the other hand, research under Professor Russell Impagliazzo revealed the more ``applied'' side
of combinatorics. Working on problems on multicalibration, which aims to avoid inadvertent
discrimination in learning models, the unexpected connection between Algorithm Fairness and the
Szemerédi's regularity lemma broadened my view on the possible applications of combinatorial tools
in other disciplines. Despite the theoretical nature of combinatorics, its boundless potential for
real-world applications adds another layer of meaning to my interest.

To feed my curiosity, I hope to dig further into the world of combinatorics, which prompts me to
pursue graduate studies in mathematics. In the long term, I aspire to become a researcher in
combinatorics who is capable of tackling open problems with original ideas and making meaningful
contributions, and graduate school in mathematics will be the first step toward this goal. 

I plan to continue investigating Turán problems and their generalizations in graduate school. A
possible research direction I would like to follow is to build on my undergraduate research. The
problem I'm currently working on is the Double Turán problem, which extends the classical Turán
problem by investigating the maximum number of edges in $n$ subgraphs of a complete graph $K_n$,
where no pairwise intersection of these subgraphs contains a specific forbidden structure. The
degenerated case of this problem, for which the forbidden structure is bipartite, is shown to be
harder than the non-degenerated case, as seen in the classical Turán problem. Since I am not capable
enough to tackle them in my current state of knowledge, I would like to further explore the
degenerated double Turán problems as I acquire more tools and insights in my graduate studies.

Despite my strong interest in the Turán problem, I am not looking to focus exclusively on a single
area early in my research career. I intend to explore adjacent fields, both to connect my rather
specialized knowledge in extremal combinatorics with other areas and to broaden my understanding of
the mathematical landscape. I believe that cultivating a diverse set of perspectives will lead to
more fruitful outcomes in my research career. 

Why this program

Why I'm qualified

Closing paragraph

\end{document}
