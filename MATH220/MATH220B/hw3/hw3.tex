\documentclass{article}

\usepackage{fancyhdr}
\usepackage{extramarks}
\usepackage{amsmath}
\usepackage{amsthm}
\usepackage{amsfonts}
\usepackage{tikz}
\usepackage[plain]{algorithm}
\usepackage{algpseudocode}
\usepackage{enumerate}
\usepackage{amssymb}

\usetikzlibrary{automata,positioning}

%
% Basic Document Settings
%

\topmargin=-0.45in
\evensidemargin=0in
\oddsidemargin=0in
\textwidth=6.5in
\textheight=9.0in
\headsep=0.25in

\linespread{1.1}

\pagestyle{fancy}
\lhead{\hmwkAuthorName}
\chead{\hmwkClass:\ \hmwkTitle}
\rhead{\firstxmark}
\lfoot{\lastxmark}
\cfoot{\thepage}

\renewcommand\headrulewidth{0.4pt}
\renewcommand\footrulewidth{0.4pt}

\setlength\parindent{0pt}
\setlength{\parskip}{5pt}

%
% Create Problem Sections
%

\newcommand{\enterProblemHeader}[1]{
    \nobreak\extramarks{}{Problem \arabic{#1} continued on next page\ldots}\nobreak{}
    \nobreak\extramarks{Problem \arabic{#1} (continued)}{Problem \arabic{#1} continued on next page\ldots}\nobreak{}
}

\newcommand{\exitProblemHeader}[1]{
    \nobreak\extramarks{Problem \arabic{#1} (continued)}{Problem \arabic{#1} continued on next page\ldots}\nobreak{}
    \stepcounter{#1}
    \nobreak\extramarks{Problem \arabic{#1}}{}\nobreak{}
}

\setcounter{secnumdepth}{0}
\newcounter{partCounter}
\newcounter{homeworkProblemCounter}
\setcounter{homeworkProblemCounter}{1}
\nobreak\extramarks{Problem \arabic{homeworkProblemCounter}}{}\nobreak{}

%
% Homework Problem Environment
%
% This environment takes an optional argument. When given, it will adjust the
% problem counter. This is useful for when the problems given for your
% assignment aren't sequential. See the last 3 problems of this template for an
% example.
%
\newenvironment{homeworkProblem}[1][-1]{
    \ifnum#1>0
        \setcounter{homeworkProblemCounter}{#1}
    \fi
    \section{Problem \arabic{homeworkProblemCounter}}
    \setcounter{partCounter}{1}
    \enterProblemHeader{homeworkProblemCounter}
}{
    \exitProblemHeader{homeworkProblemCounter}
}

%
% Homework Details
%   - Title
%   - Due date
%   - Class
%   - Section/Time
%   - Instructor
%   - Author
%

\newcommand{\hmwkTitle}{Homework\ \#3}
\newcommand{\hmwkDueDate}{Feb 18, 2025}
\newcommand{\hmwkClass}{MATH 220B}
\newcommand{\hmwkClassInstructor}{Professor Xiao}
\newcommand{\hmwkAuthorName}{\textbf{Ray Tsai}}
\newcommand{\hmwkPID}{A16848188}

%
% Title Page
%

\title{
    \vspace{2in}
    \textmd{\textbf{\hmwkClass:\ \hmwkTitle}}\\
    \normalsize\vspace{0.1in}\small{Due\ on\ \hmwkDueDate\ at 23:59pm}\\
    \vspace{0.1in}\large{\textit{\hmwkClassInstructor}} \\
    \vspace{3in}
}

\author{
  \hmwkAuthorName \\
  \vspace{0.1in}\small\hmwkPID
}
\date{}

\renewcommand{\part}[1]{\textbf{\large Part \Alph{partCounter}}\stepcounter{partCounter}\\}

%
% Various Helper Commands
%

% Useful for algorithms
\newcommand{\alg}[1]{\textsc{\bfseries \footnotesize #1}}

% For derivatives
\newcommand{\deriv}[1]{\frac{\mathrm{d}}{\mathrm{d}x} (#1)}

% For partial derivatives
\newcommand{\pderiv}[2]{\frac{\partial}{\partial #1} (#2)}

% Integral dx
\newcommand{\dx}{\mathrm{d}x}

% Probability commands: Expectation, Variance, Covariance, Bias
\newcommand{\Var}{\mathrm{Var}}
\newcommand{\Cov}{\mathrm{Cov}}
\newcommand{\Bias}{\mathrm{Bias}}
\newcommand*{\Z}{\mathbb{Z}}
\newcommand*{\Q}{\mathbb{Q}}
\newcommand*{\R}{\mathbb{R}}
\newcommand*{\C}{\mathbb{C}}
\newcommand*{\N}{\mathbb{N}}
\newcommand*{\prob}{\mathds{P}}
\newcommand*{\E}{\mathds{E}}

\begin{document}

\maketitle

\pagebreak

\begin{homeworkProblem}
  Prove Lemma 1.5: If $(S, d)$ is a metric space then
	\[
	\mu(s, t) = \frac{d(s, t)}{1 + d(s, t)}
	\]
	is also a metric on $S$. A set is open in $(S, d)$ iff it is open in $(S, \mu)$; a sequence is a Cauchy sequence in $(S, d)$ iff it is a Cauchy sequence in $(S, \mu)$.

	\begin{proof}
		We first show that $\mu$ is a metric. Let $s, t, u \in S$. Then $\mu(s, s) = 0$, $\mu(s, t) > 0$ if $s \neq t$, $\mu(s, t) = \mu(t, s)$. We now prove the triangle inequality. Note that
		\[
			\frac{d(s, u)}{1 + d(s, u)} \leq \frac{d(s, t) + d(t, u)}{1 + d(s, t) + d(t, u)},
		\]
		Hence, it suffices to show that for $a, b \geq 0$,
		\[
			\frac{a + b}{1 + a + b} \leq \frac{a}{1 + a} + \frac{b}{1 + b}.
		\]
		Notice
		\[
			\frac{a}{1 + a} + \frac{b}{1 + b} = 2 - \left(\frac{1}{1 + a} + \frac{1}{1 + b}\right)
		\]
		and
		\[
			\frac{a + b}{1 + a + b} = 1 - \frac{1}{1 + a + b}.
		\]
		Since
		\[
			\frac{1}{1 + a} + \frac{1}{1 + b} - 1 = \frac{1 - ab}{1 + a + b + ab} \leq \frac{1}{1 + a + b},
		\]
		we have 
		\[
			\frac{a}{1 + a} + \frac{b}{1 + b} = 2 - \left(\frac{1}{1 + a} + \frac{1}{1 + b}\right) \geq 1 - \frac{1}{1 + a + b} = \frac{a + b}{1 + a + b}.
		\]

		Since $\frac{t}{1 + t}$ is continuous and strictly increasing on $[0, \infty)$, for $\delta > 0$ there exists $\epsilon > 0$ such that $d(s, t) < \delta$ if and only if $\mu(s, t) < \epsilon$. Hence, a set $U \subseteq S$ is open in $(S, d)$ if and only if $U$ is open in $(S, \mu)$. Similarly, a sequence $\{s_n\}$ is a Cauchy sequence in $(S, \mu)$ if and only if for $\epsilon > 0$ there exists $N$ such that for all $m, n \geq N$,
		\[
			\mu(s_n, s_m) < \epsilon \iff d(s_n, s_m) < \delta,
		\]
		where the $\delta$ corresponds to $\epsilon$ as above. 
	\end{proof}
\end{homeworkProblem}

\newpage

\begin{homeworkProblem}
	Suppose $\{f_n\}$ is a sequence in $C(G, \Omega)$ which converges to $f$ and $\{z_n\}$ is a sequence in $G$ which converges to a point $z$ in $G$. Show $\lim f_n(z_n) = f(z)$.
    
	\begin{proof}
		Let $K \subseteq G$ be a compact set that contains $z$ and $\{z_n\}$. Let $\epsilon > 0$. Since $f_n \to f$ uniformly on $K$, there exists $N$ such that for all $n \geq N$,
		\[
			|f_n(x) - f(x)| < \frac{\epsilon}{2},
		\]
		for all $x \in K$. Since $z_n \to z$ and $f$ is continuous, there exists $M$ such that for all $n \geq M$,
		\[
			d|f(z_n) - f(z)| < \frac{\epsilon}{2},
		\]
		Hence, for all $n \geq \max(N, M)$,
		\[
			|f_n(z_n) - f(z)| \leq |f_n(z_n) - f_n(z)| + |f_n(z) - f(z)| < \frac{\epsilon}{2} + \frac{\epsilon}{2} = \epsilon.
		\]
	\end{proof}
\end{homeworkProblem}

\newpage

\begin{homeworkProblem}
	\textbf{(Dini's Theorem)} Consider $C(G, \mathbb{R})$ and suppose that $\{f_n\}$ is a sequence in $C(G, \mathbb{R})$ which is monotonically increasing (i.e., $f_n(z) \leq f_{n+1}(z)$ for all $z$ in $G$) and $\lim f_n(z) = f(z)$ for all $z$ in $G$, where $f \in C(G, \mathbb{R})$. Show that $f_n \to f$.

	\begin{proof}
		Let $K \subseteq G$ be compact. Fix $\epsilon > 0$. Let $g_n = f - f_n$. Let $K_n = \{x \in K \mid g_n(x) \geq \epsilon\} = g^{-1}([\epsilon, \infty))$. Since $g_n$ is continuous and $[\epsilon, \infty)$ is closed, $K_n$ is closed. But then $K_n$ is a closed subset of a compact set, so $K_n$ is compact. Since $g_{n + 1}(z) \geq g_n(z)$, we have $K_{n + 1} \subseteq K_n$. Let $z \in K$. Since $\lim_{n \to \infty} g_n(z) = 0$, we know $z \notin K_n$ for large enough $n$, and so $\bigcap_{n \geq 1} K_n = \emptyset$. But then $K_N$ is empty for some $N$. Hence, $0 \leq g_n(z) < \epsilon$ for all $z \in K$, $n \geq N$. The result now follows.
	\end{proof}
\end{homeworkProblem}

\newpage

\begin{homeworkProblem}
	\begin{enumerate}
    \item[(a)] Let $f$ be analytic on $B(0; R)$ and let 
    \[
    f(z) = \sum_{n=0}^{\infty} a_n z^n \quad \text{for } |z| < R.
    \]
    If 
    \[
    f_n(z) = \sum_{k=0}^{n} a_k z^k,
    \]
    show that $f_n \to f$ in $C(G; \mathbb{C})$.

		\begin{proof}
			Note that for any compact subset $K \subseteq B(0; R)$, there exists $r \in (0, R)$ such that $K \subseteq \overline{B}_r(0)$. Since $f$ converges on $B(0; R)$, the series $\sum_{n = 0}^\infty a_n r^n$ converges. But then by the Weierstrass M-test, $f_n$ converges to $f$ uniformly on $\overline{B}_r(0)$. The result now follows.
		\end{proof}
    
    \item[(b)] Let $G = \text{ann}(0; 0, R)$ and let $f$ be analytic on $G$ with Laurent series development
    \[
    f(z) = \sum_{n=-\infty}^{\infty} a_n z^n.
    \]
    Put 
    \[
    f_n(z) = \sum_{k=-\infty}^{n} a_k z^k
    \]
    and show that $f_n \to f$ in $C(G; \mathbb{C})$.

		\begin{proof}
			Write $f(z) = f^-(z) + f^+(z)$, with $f^-(z) = \sum_{n = -\infty}^{-1} a_n z^n$ and $f^+(z) = \sum_{n = 0}^{\infty} a_n z^n$. Let $f_n^- = \sum_{k = 1}^{n} a_{-k} z^{-k}$ and $f_n^+ = \sum_{k = 0}^{n} a_k z^k$. Note that for any compact subset $K \subseteq \text{ann}(0; 0, R)$, there exists $r_1, r_2 \in (0, R)$ such that $K \subseteq \overline{\text{ann}(0; r_1, r_2)}$. Since $f$ converges on $\text{ann}(0; 0, R)$, the series $\sum_{n = -\infty}^{-1} a_nr_1^n$ and $\sum_{n = 0}^{\infty} a_nr_2^n$ converges. By the Weierstrass M-test, $f_n^-$ converges to $f^-$ uniformly on $\overline{\text{ann}(0; r_1, r_2)}$ and $f_n^+$ converges to $f^+$ uniformly on $\overline{\text{ann}(0; r_1, r_2)}$. Since $f_n(z) = f^-(z) + f^+_n(z)$, the result follows.
		\end{proof}
	\end{enumerate}
\end{homeworkProblem}

\newpage

\begin{homeworkProblem}
	Prove Vitali's Theorem: If $G$ is a region and $\{f_n\} \subset H(G)$ is locally bounded and $f \in H(G)$ that has the property that 
	\[
	A = \{ z \in G : \lim f_n(z) = f(z) \}
	\]
	has a limit point in $G$, then $f_n \to f$.

	\begin{proof}
		Define $g_n = f_n - f$. Since $\{f_n\}$ is locally bounded, $\{g_n\}$ is locally bounded. By Montel's Theorem, theres is a converging subsequence $\{g_{n_k}\}$, say $g_{n_k} \to g$. But then $g(z) = 0$ on $A$ and $A$ has a limit point, so $g(z) = 0$ on $G$. This implies every converging subsequence of $\{g_{n}\}$ converges to $0$ on $G$, which forces $g_n \to 0$. Therefore,
		$f_n = f + g_n \to f$. 
	\end{proof}
\end{homeworkProblem}

\newpage

\begin{homeworkProblem}
	Let $D = B(0; 1)$ and for $0 < r < 1$ let $\gamma_r(t) = re^{2\pi i t}$, $0 \leq t \leq 1$. Show that a sequence $\{ f_n \}$ in $H(D)$ converges to $f$ iff
	\[
	\int_{\gamma_r} |f(z) - f_n(z)| \, |dz| \to 0 \quad \text{as } n \to \infty
	\]
	for each $r$, $0 < r < 1$.

	\begin{proof}
		Suppose that $f_n \to f$. Pick $\epsilon > 0$. Then there exists $N$ such that for all $n \geq N$, $|f(z) - f_n(z)| < \epsilon$. Hence,
		\[
			\int_{\gamma_r} |f(z) - f_n(z)| \, |dz| < \epsilon \int_{\gamma_r} |dz| = \epsilon \cdot 2\pi r \to 0,
		\]
		as $\epsilon \to 0$.

		We now show the converse. Fix $r \in (0, 1), \epsilon > 0$. Let $g_n = f(z) - f_n(z)$. Since $g_n$ is analytic,  
		\[
			|g_n(z)| = \frac{1}{2\pi}\int_{\gamma_r} \frac{g_n(w)}{w - z} \, |dw| \leq \frac{1}{2\pi r}\int_{\gamma_r} |g_n(w)| \, |dz|
		\]
		on $\overline{B}_r(0)$. Hence, $g_n(z) \to 0$ on any closed disk $B_0(r)$, $0 < r < 1$, and the result now follows.
	\end{proof}
\end{homeworkProblem}

\newpage

\begin{homeworkProblem}
	Let $\{ f_n \} \subset H(G)$ be a sequence of one-one functions which converge to $f$. Show that either $f$ is one-one or $f$ is a constant function.

	\begin{proof}
		Suppose $f$ is not one-one or constant. There exists $z_1, z_2 \in G$ such that $f(z_1) = f(z_2)$. Consider sequence $g_n(z) = f_n(z) - f_n(z_1)$. Let $g = f - f(z_1)$. Note that $g_n \to g$ and $g_n$ has at most one zero. Since $g$ is analytic, its zeros are isolated, so we may find a closed disk $D$ such that $g$ does not vanish on $\partial D$ and $z_1, z_2 \in K$. By Hurwitz's Theorem, for large enough $n$, $g_n$ and $g$ have the same number of zeros in $K$. But then $g$ has zeros $z_1$ and $z_2$ in $K$ while $g_n$ has at most one zero in $K$, contradiction.
	\end{proof}
\end{homeworkProblem}

\newpage

\begin{homeworkProblem}
	Suppose that $\{ f_n \}$ is a sequence in $H(G)$, $f$ is a non-constant function, and $f_n \to f$ in $H(G)$. Let $a \in G$ and $\alpha = f(a)$; show that there is a sequence $\{ a_n \}$ in $G$ such that:
	\begin{enumerate}
    \item[(i)] $a = \lim a_n$;
    \item[(ii)] $f_n(a_n) = \alpha$ for sufficiently large $n$.
	\end{enumerate}

	\begin{proof}
		Define $g(z) = f(z) - \alpha$. Since $g$ is analytic and non-constant, the zeros of $g$ are isolated. Hence, we may find a sequence $\{r_n\}$ such that $r_n \to 0$ and  $g$ does not vanish on $\partial B_{r_n}(a)$. Since $f_n \to f$ uniformly on closed balls, there exists $N$ such that for $n \geq N$ we have
		\[
			\max_{|z - a| = r_n} |f_n(z) - f(z)| < \min_{|z - a| = r_n} |g(z)|.
		\]
		Put $g_n(z) = f_n(z) - \alpha$. Since for $n \geq N$
		\[
			|g_n(z) - g(z)| = |f_n(z) - f(z)| < |g(z)|
		\]
		on $\partial B_{r_n}(a)$, $g_n(z)$ and $g(z)$ have the same number of zeros in $B_{r_n}(a)$, which is at least one. Let $a_n$ be a zero of $g_n(z)$ in $B_{r_n}(a)$. Then we have $f_n(a_n) = \alpha$ for all $n \geq N$. Since $r_n \to 0$, $a_n \to 0$. 
	\end{proof}
\end{homeworkProblem}

\newpage

\begin{homeworkProblem}
	Let $f$ be analytic on $G = \{z : \text{Re} \, z > 0\}$, one-one, with $\text{Re} f(z) > 0$ for all $z \in G$, and $f(a) = a$ for some real number $a$. Show that $|f'(a)| \leq 1$. 

	\begin{proof}
		Since $G$ is a simply connected region and $G \neq \C$, there is a unique analytic one-one function $g: G \to D$ such that $g(a) = 0$. Consider $h = g \circ f \circ g^{-1}$. Note that $h$ maps $D$ to $D$ and $h(0) = 0$. By Schwarz's Lemma, 
		\[
			|h'(0)| = |g'(a)f'(a)(g^{-1})'(0)| \leq 1
		\]
		But then $(g^{-1})'(0)g'(a) = (g^{-1})'(0)g'(g^{-1}(0)) = 1$, and the result now follows.
	\end{proof}
\end{homeworkProblem}
\end{document}