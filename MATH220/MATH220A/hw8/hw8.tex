\documentclass{article}

\usepackage{fancyhdr}
\usepackage{extramarks}
\usepackage{amsmath}
\usepackage{amsthm}
\usepackage{amsfonts}
\usepackage{tikz}
\usepackage[plain]{algorithm}
\usepackage{algpseudocode}
\usepackage{enumerate}
\usepackage{amssymb}

\usetikzlibrary{automata,positioning}

%
% Basic Document Settings
%

\topmargin=-0.45in
\evensidemargin=0in
\oddsidemargin=0in
\textwidth=6.5in
\textheight=9.0in
\headsep=0.25in

\linespread{1.1}

\pagestyle{fancy}
\lhead{\hmwkAuthorName}
\chead{\hmwkClass:\ \hmwkTitle}
\rhead{\firstxmark}
\lfoot{\lastxmark}
\cfoot{\thepage}

\renewcommand\headrulewidth{0.4pt}
\renewcommand\footrulewidth{0.4pt}

\setlength\parindent{0pt}
\setlength{\parskip}{5pt}

%
% Create Problem Sections
%

\newcommand{\enterProblemHeader}[1]{
    \nobreak\extramarks{}{Problem \arabic{#1} continued on next page\ldots}\nobreak{}
    \nobreak\extramarks{Problem \arabic{#1} (continued)}{Problem \arabic{#1} continued on next page\ldots}\nobreak{}
}

\newcommand{\exitProblemHeader}[1]{
    \nobreak\extramarks{Problem \arabic{#1} (continued)}{Problem \arabic{#1} continued on next page\ldots}\nobreak{}
    \stepcounter{#1}
    \nobreak\extramarks{Problem \arabic{#1}}{}\nobreak{}
}

\setcounter{secnumdepth}{0}
\newcounter{partCounter}
\newcounter{homeworkProblemCounter}
\setcounter{homeworkProblemCounter}{1}
\nobreak\extramarks{Problem \arabic{homeworkProblemCounter}}{}\nobreak{}

%
% Homework Problem Environment
%
% This environment takes an optional argument. When given, it will adjust the
% problem counter. This is useful for when the problems given for your
% assignment aren't sequential. See the last 3 problems of this template for an
% example.
%
\newenvironment{homeworkProblem}[1][-1]{
    \ifnum#1>0
        \setcounter{homeworkProblemCounter}{#1}
    \fi
    \section{Problem \arabic{homeworkProblemCounter}}
    \setcounter{partCounter}{1}
    \enterProblemHeader{homeworkProblemCounter}
}{
    \exitProblemHeader{homeworkProblemCounter}
}

%
% Homework Details
%   - Title
%   - Due date
%   - Class
%   - Section/Time
%   - Instructor
%   - Author
%

\newcommand{\hmwkTitle}{Homework\ \#8}
\newcommand{\hmwkDueDate}{Nov 22, 2024}
\newcommand{\hmwkClass}{MATH 220A}
\newcommand{\hmwkClassInstructor}{Professor Ebenfelt}
\newcommand{\hmwkAuthorName}{\textbf{Ray Tsai}}
\newcommand{\hmwkPID}{A16848188}

%
% Title Page
%

\title{
    \vspace{2in}
    \textmd{\textbf{\hmwkClass:\ \hmwkTitle}}\\
    \normalsize\vspace{0.1in}\small{Due\ on\ \hmwkDueDate\ at 23:59pm}\\
    \vspace{0.1in}\large{\textit{\hmwkClassInstructor}} \\
    \vspace{3in}
}

\author{
  \hmwkAuthorName \\
  \vspace{0.1in}\small\hmwkPID
}
\date{}

\renewcommand{\part}[1]{\textbf{\large Part \Alph{partCounter}}\stepcounter{partCounter}\\}

%
% Various Helper Commands
%

% Useful for algorithms
\newcommand{\alg}[1]{\textsc{\bfseries \footnotesize #1}}

% For derivatives
\newcommand{\deriv}[1]{\frac{\mathrm{d}}{\mathrm{d}x} (#1)}

% For partial derivatives
\newcommand{\pderiv}[2]{\frac{\partial}{\partial #1} (#2)}

% Integral dx
\newcommand{\dx}{\mathrm{d}x}

% Probability commands: Expectation, Variance, Covariance, Bias
\newcommand{\Var}{\mathrm{Var}}
\newcommand{\Cov}{\mathrm{Cov}}
\newcommand{\Bias}{\mathrm{Bias}}
\newcommand*{\Z}{\mathbb{Z}}
\newcommand*{\Q}{\mathbb{Q}}
\newcommand*{\R}{\mathbb{R}}
\newcommand*{\C}{\mathbb{C}}
\newcommand*{\N}{\mathbb{N}}
\newcommand*{\prob}{\mathds{P}}
\newcommand*{\E}{\mathds{E}}

\begin{document}

\maketitle

\pagebreak

\begin{homeworkProblem}
	\begin{enumerate}[(a)]
		\item Prove Abel's Theorem: Let $\sum a_n (z - a)^n$ have radius of convergence 1 and suppose
		that $\sum a_n$ converges to $A$. Prove that
		\[
			\lim_{r \to 1^-} \sum a_n r^n = A.
		\]
		(Hint: Find a summation formula which is the analogue of integration by parts.)

		\begin{proof}
			We may assume that $a = 0$ and $\sum a_n = A = 0$, as we can always adjust the value of $a_0$.
			Let $S_n = \sum_{k=0}^n a_k$. Then for $r \in (0, 1)$,
			\begin{align}
				\sum_{n = 0}^{\infty} a_n r^n
				&= a_0 + \sum_{n = 1}^{\infty} (S_n - S_{n - 1}) r^n \\
				&= a_0 + \sum_{n = 1}^{\infty} S_n r^n - \sum_{n = 1}^{\infty} S_{n - 1} r^n \\
				&= \sum_{n = 0}^{\infty} S_n r^n - r\sum_{n = 0}^{\infty} S_n r^{n} \\
				&= (1 - r)\sum_{n = 0}^{\infty} S_n r^n.
			\end{align}
			Pick $\epsilon > 0$. Since $\sum a_n \to 0$, there exists integer $N$ such that for all $k
			\geq N$, $|S_k| < \epsilon/2$. Then
			\[
				\left|(1 - r)\sum_{n = N}^{\infty} S_n r^n\right| \leq (1 - r)\sum_{n = N}^{\infty} |S_n| r^n \leq \frac{\epsilon}{2} (1 - r)\sum_{n = N}^{\infty} r^n = \frac{\epsilon}{2} (1 - r)\frac{r^N}{1 - r} < \epsilon/2,
			\]
			for all $r \in (0, 1)$. Since $\sum_{n = N}^{N} S_n r^n = M$ for some constant $M$, pick $r <
			(1 - \epsilon/2M, 1)$ and we have
			\[
				\left|(1 - r)\sum_{n = 0}^{N - 1} S_n r^n\right| = (1 - r)M < \epsilon/2.
			\]
			Therefore, 
			\[
				\sum a_nr^n < \epsilon
			\]	 
			for $r$ sufficiently close to 1, and the result now follows.
		\end{proof}

		\item Use Abel's Theorem to prove that $\log 2 = 1 - \frac{1}{2} + \frac{1}{3} - \cdots$.
		\begin{proof}
			Consider $\log(1 + z)$. The power series expansion of $\log(1 + z)$ is $\sum a_nz^n = \sum
			\frac{(-1)^{n + 1}}{n}z^n$, which has radius of convergence 1, as $\lim_{n \to \infty}
			\left|\frac{a_{n + 1}}{a_n}\right| = 1$. By Abel's Theorem, 
			\[
				\log 2 = \lim_{r \to 1^-} \sum \frac{(-1)^{n + 1}}{n}r^n = 1 - \frac{1}{2} + \frac{1}{3} - \cdots.
			\]
		\end{proof}
	\end{enumerate}
\end{homeworkProblem}

\newpage

\begin{homeworkProblem}
	Give the power series expansion of $\log z$ about $z = i$ and find its radius of convergence.

	\begin{proof}
		Note that the $n$th derivative of $\log z$ is $(-1)^{n + 1}z^{-n}$. Let $a_n = \frac{(-1)^{n +
		1}}{n!}i^{-n}$. Then the power series expansion of $\log z$ about $z = i$ is
		\[
			\sum a_n(z - i)^n = \sum \frac{(-1)^{n + 1}}{n!}i^{-n}(z - i)^n.
		\]
		The radius of convergence is
		\[
			R = \lim_{n \to \infty} \left|\frac{a_{n}}{a_{n + 1}}\right| = \lim_{n \to \infty} \left|\frac{n + 1}{i}\right| = \infty.
		\]
	\end{proof}
\end{homeworkProblem}

\newpage

\begin{homeworkProblem}
	Evaluate 
	\[
		\int_\gamma \frac{z^2 + 1}{z(z^2 + 4)} \, dz
	\]
	where $\gamma(s) = re^{is}$, $0 \leq s \leq 2\pi$, for all possible values of $r$, $0 < r < 2$ and
	$2 < r < \infty$.

	\begin{proof}
		Define $\phi(s, t) = \frac{i(r^2e^{2is} + t)}{r^2e^{2is} + 4t}$ for $t \in [0, 1]$ and $s \in [0,
		2\pi]$. Note that $\phi$ is continuously differentiable, and thus $g(t) = \int_{0}^{2\pi}
		\phi(s, t) \, ds$ is continuously differentiable. Notice
		\[
			g(1) = \int_0^{2\pi} \frac{i(r^2e^{2is} + 1)}{r^2e^{2is} + 4} \, ds = \int_0^{2\pi} \frac{r^2e^{2is} + 1}{re^{is}(r^2e^{2is} + 4)}ire^{is} \, ds = \int_\gamma \frac{z^2 + 1}{z(z^2 + 4)} \, dz.
		\]
		By Leibniz's Rule, 
		\[
			g'(t) = \int_{0}^{2\pi} \frac{\partial}{\partial t}\phi(s, t) \, ds = \int_{0}^{2\pi} \frac{-3ir^2e^{2is}}{(r^2e^{2is} + 4t)^2} \, ds.
		\]
		Define $\Phi(s) = \frac{3}{2}(r^2e^{2is} + 4t)^{-1}$. Since $\Phi'(s) =
		\frac{-3ir^2e^{2is}}{(r^2e^{2is} + 4t)^2}$, we have $g'(t) = \Phi(2\pi) - \Phi(0) = 0$. That is,
		$g(t)$ is constant for $t \in [0, 1]$. It now follows that
		\[
			\int_\gamma \frac{z^2 + 1}{z(z^2 + 4)} \, dz = g(1) = g(0) = \int_{0}^{2\pi} \frac{i(r^2e^{2is})}{r^2e^{2is}} \, ds = 2\pi i.
		\]
	\end{proof}
\end{homeworkProblem}

\newpage

\begin{homeworkProblem}
	Let $f$ be an entire function and suppose there is a constant $M$, an $R > 0$, and an integer $n
	\geq 1$ such that $|f(z)| \leq M|z|^n$ for $|z| > R$. Show that $f$ is a polynomial of degree
	$\leq n$.

	\begin{proof}
		Since $f$ is entire, $f$ jas a power series expansion about $z = 0$ of the form
		\[
			f(z) = \sum_{n = 0}^{\infty} \frac{f^{(n)}(0)}{n!}z^n.
		\]
		Let $\gamma(t) = re^{it}$, with $r > R$ and $t \in [0, 2\pi]$. By Corollary 2.13,
		\[
			f^{(k)}(0) = \frac{k!}{2\pi i} \int_\gamma \frac{f(w)}{w^{k + 1}} \, dw.
		\]
		For $r > R$, $|f(z)| \leq M|z|^n$ and thus
		\begin{align*}
			|f^{(k)}(0)| 
			&= \left|\frac{k!}{2\pi i} \int_\gamma \frac{f(w)}{w^{k + 1}} \, dw\right| \\
			&\leq \frac{k!}{2\pi} \int_\gamma \frac{|f(w)|}{|w|^{k + 1}} \, |dw| \\
			&\leq \frac{k!M}{2\pi} \int_\gamma |w|^{n - k - 1} \, |dw| \\
			&= \frac{k!M}{2\pi}r^{n - k - 1} \int_\gamma |dw| = k!Mr^{n - k}.
		\end{align*}
		But then $r$ can be arbitrarily large, and thus $f^{(k)}(0) = 0$ for all $k > n$. Therefore,
		\[
			f(z) = \sum_{k = 0}^{n} \frac{f^{(k)}(0)}{k!}z^k
		\]
		is a polynomial of degree $\leq n$.
	\end{proof}
\end{homeworkProblem}

\newpage

\begin{homeworkProblem}
	Find all entire functions $f$ such that $f(x) = e^x$ for $x \in \mathbb{R}$.

	\begin{proof}
		Let $g(x) = f(x) - e^x$. Then $g(x) = 0$ for $x \in \mathbb{R}$. Since $g$ is entire and $\{z
		\in \C : g(z) = 0\}$ has a limit point at $0$, $g(z) = 0$ for all $z \in \C$ by Theorem 3.7.
		Therefore, $f(x) = e^x$ for all $x \in \C$.
	\end{proof}
\end{homeworkProblem}
\end{document}