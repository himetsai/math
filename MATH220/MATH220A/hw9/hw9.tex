\documentclass{article}

\usepackage{fancyhdr}
\usepackage{extramarks}
\usepackage{amsmath}
\usepackage{amsthm}
\usepackage{amsfonts}
\usepackage{tikz}
\usepackage[plain]{algorithm}
\usepackage{algpseudocode}
\usepackage{enumerate}
\usepackage{amssymb}

\usetikzlibrary{automata,positioning}

%
% Basic Document Settings
%

\topmargin=-0.45in
\evensidemargin=0in
\oddsidemargin=0in
\textwidth=6.5in
\textheight=9.0in
\headsep=0.25in

\linespread{1.1}

\pagestyle{fancy}
\lhead{\hmwkAuthorName}
\chead{\hmwkClass:\ \hmwkTitle}
\rhead{\firstxmark}
\lfoot{\lastxmark}
\cfoot{\thepage}

\renewcommand\headrulewidth{0.4pt}
\renewcommand\footrulewidth{0.4pt}

\setlength\parindent{0pt}
\setlength{\parskip}{5pt}

%
% Create Problem Sections
%

\newcommand{\enterProblemHeader}[1]{
    \nobreak\extramarks{}{Problem \arabic{#1} continued on next page\ldots}\nobreak{}
    \nobreak\extramarks{Problem \arabic{#1} (continued)}{Problem \arabic{#1} continued on next page\ldots}\nobreak{}
}

\newcommand{\exitProblemHeader}[1]{
    \nobreak\extramarks{Problem \arabic{#1} (continued)}{Problem \arabic{#1} continued on next page\ldots}\nobreak{}
    \stepcounter{#1}
    \nobreak\extramarks{Problem \arabic{#1}}{}\nobreak{}
}

\setcounter{secnumdepth}{0}
\newcounter{partCounter}
\newcounter{homeworkProblemCounter}
\setcounter{homeworkProblemCounter}{1}
\nobreak\extramarks{Problem \arabic{homeworkProblemCounter}}{}\nobreak{}

%
% Homework Problem Environment
%
% This environment takes an optional argument. When given, it will adjust the
% problem counter. This is useful for when the problems given for your
% assignment aren't sequential. See the last 3 problems of this template for an
% example.
%
\newenvironment{homeworkProblem}[1][-1]{
    \ifnum#1>0
        \setcounter{homeworkProblemCounter}{#1}
    \fi
    \section{Problem \arabic{homeworkProblemCounter}}
    \setcounter{partCounter}{1}
    \enterProblemHeader{homeworkProblemCounter}
}{
    \exitProblemHeader{homeworkProblemCounter}
}

%
% Homework Details
%   - Title
%   - Due date
%   - Class
%   - Section/Time
%   - Instructor
%   - Author
%

\newcommand{\hmwkTitle}{Homework\ \#9}
\newcommand{\hmwkDueDate}{Dec 2, 2024}
\newcommand{\hmwkClass}{MATH 220A}
\newcommand{\hmwkClassInstructor}{Professor Ebenfelt}
\newcommand{\hmwkAuthorName}{\textbf{Ray Tsai}}
\newcommand{\hmwkPID}{A16848188}

%
% Title Page
%

\title{
    \vspace{2in}
    \textmd{\textbf{\hmwkClass:\ \hmwkTitle}}\\
    \normalsize\vspace{0.1in}\small{Due\ on\ \hmwkDueDate\ at 23:59pm}\\
    \vspace{0.1in}\large{\textit{\hmwkClassInstructor}} \\
    \vspace{3in}
}

\author{
  \hmwkAuthorName \\
  \vspace{0.1in}\small\hmwkPID
}
\date{}

\renewcommand{\part}[1]{\textbf{\large Part \Alph{partCounter}}\stepcounter{partCounter}\\}

%
% Various Helper Commands
%

% Useful for algorithms
\newcommand{\alg}[1]{\textsc{\bfseries \footnotesize #1}}

% For derivatives
\newcommand{\deriv}[1]{\frac{\mathrm{d}}{\mathrm{d}x} (#1)}

% For partial derivatives
\newcommand{\pderiv}[2]{\frac{\partial}{\partial #1} (#2)}

% Integral dx
\newcommand{\dx}{\mathrm{d}x}

% Probability commands: Expectation, Variance, Covariance, Bias
\newcommand{\Var}{\mathrm{Var}}
\newcommand{\Cov}{\mathrm{Cov}}
\newcommand{\Bias}{\mathrm{Bias}}
\newcommand*{\Z}{\mathbb{Z}}
\newcommand*{\Q}{\mathbb{Q}}
\newcommand*{\R}{\mathbb{R}}
\newcommand*{\C}{\mathbb{C}}
\newcommand*{\N}{\mathbb{N}}
\newcommand*{\prob}{\mathds{P}}
\newcommand*{\E}{\mathds{E}}

\begin{document}

\maketitle

\pagebreak

\begin{homeworkProblem}
  Let $G$ be a region and suppose that $f: G \to \mathbb{C}$ is analytic and $a \in G$ such that 
	\[
		|f(a)| \leq |f(z)|
	\]
	for all $z \in G$. Show that either $f(a) = 0$ or $f$ is constant.

	\begin{proof}
		Suppose $f(a) \neq 0$. Consider $g(z) = \frac{1}{f(z)}$. Since $g$ is analytic on $G$ and $g(a)
		\geq g(z)$ for all $z \in G$, by the maximum modulus principle, $g$ is constant, which also
		makes $f$ a constant.
	\end{proof}
\end{homeworkProblem}

\newpage

\begin{homeworkProblem}
	Let $G$ be a region and let $f$ and $g$ be analytic functions on $G$ such that 
	\[
		f(z)g(z) = 0
	\]
	for all $z \in G$. Show that either $f \equiv 0$ or $g \equiv 0$.

	\begin{proof}
		If $f$ is constant, then etiher $f \equiv 0$ or $g \equiv 0$ and we are done. Suppose $f, g$ are
		not constants. There exists a $a \in G$ such that $f(a) = 0$. By Corollary 3.10, there is an $R
		> 0$ such that $B(a, R) \subset G$ and $f(z) \neq 0$ for all $z \in B(a, R) \backslash \{a\}$.
		That is, $g(z) = 0$ for all $z \in B(a, R) \backslash \{a\}$. But then the set $\{z \in G: g(z)
		= 0\}$ has a limit point at $a$, which implies that $g \equiv 0$ by theorem 3.7, contradiction.
	\end{proof}
\end{homeworkProblem}

\newpage

\begin{homeworkProblem}
	Show that if $\gamma$ and $\sigma$ are closed rectifiable curves having the same
	initial points, then
	\begin{enumerate}[(a)]
			\item $n(\gamma; a) = -n(-\gamma; a)$ for every $a \notin \{\gamma\}$.
			\begin{proof}
				By proposition 1.17,
				\begin{align*}
					n(-\gamma; a) = \frac{1}{2\pi i} \int_{-\gamma} \frac{dz}{z - a} = -\frac{1}{2\pi i} \int_{\gamma} \frac{dz}{z - a} = -n(\gamma; a).
				\end{align*}
			\end{proof}
			\item $n(\gamma + \sigma; a) = n(\gamma; a) + n(\sigma; a)$ for every $a \notin \{\gamma\} \cup \{\sigma\}$.
			\begin{proof}
				\begin{align*}
					n(\gamma + \sigma; a)
					&= \frac{1}{2\pi i} \int_{\gamma + \sigma} \frac{dz}{z - a} \\
					&= \frac{1}{2\pi i} \int_{0}^1 \frac{((\gamma + \sigma)(t))'}{(\gamma + \sigma)(t) - a} \, dt \\ 
					&= \frac{1}{2\pi i} \left(\int_{0}^{\frac{1}{2}} \frac{(\gamma(2t))'}{\gamma(2t) - a} \, dt + \int_{\frac{1}{2}}^{1} \frac{(\sigma(2t - 1))'}{\sigma(2t - 1) - a} \, dt\right) 
				\end{align*}
				Since $\gamma(2t)$ on $[0, 1/2]$ is equivalent to $\gamma(t)$ on $[0, 1]$, and $\sigma(2t -
				1)$ on $[1/2, 1]$ is equivalent to $\sigma(t)$ on $[0, 1]$, 
				\[
					\int_{0}^{\frac{1}{2}} \frac{(\gamma(2t))'}{\gamma(2t) - a} \, dt = \int_{\gamma} \frac{dz}{z - a}, \quad \int_{\frac{1}{2}}^{1} \frac{(\sigma(2t - 1))'}{\sigma(2t - 1) - a} \, dt = \int_{\sigma} \frac{dz}{z - a}.
				\]
			\end{proof}
			The result now follows.
	\end{enumerate}
\end{homeworkProblem}

\newpage

\begin{homeworkProblem}
	Let $p(z)$ be a polynomial of degree $n$ and let $R > 0$ be sufficiently large so that $p$ never
	vanishes in $\{z : |z| \geq R\}$. If $\gamma(t) = Re^{it}$, $0 \leq t \leq 2\pi$, show that
	\[
		\int_{\gamma} \frac{p'(z)}{p(z)} \, dz = 2\pi i n.
	\]

	\begin{proof}
		Define $q(t) = p(\gamma(t))$ for $t \in [0, 1]$. Then,
		\[
			\int_{\gamma} \frac{p'(z)}{p(z)} \, dz  = \int_{0}^1 \frac{p'(\gamma(t))\gamma'(t)}{p(\gamma(t))} \, dt = \int_{0}^1 \frac{q'(t)}{q(t)} \, dt.
		\]
		Define
		\[
			g(s) = \int_{0}^s \frac{q'(t)}{q(t)} \, dt.
		\]
		Note that $g(0) = 0$, $g(1) = \int_{\gamma} \frac{p'(z)}{p(z)} \, dz,$ and $g'(s) =
		\frac{q'(s)}{q(s)}$ for $s \in [0, 1]$. But this gives
		\begin{align*}
			\frac{d}{dt} e^{-g(t)}q(t) = e^{-g(t)}q'(t) - e^{-g(t)}g'(t)q(t) = e^{-g(t)}\left(q'(t) - \frac{q'(t)}{q(t)} \cdot q(t)\right) = 0,
		\end{align*}
		so $e^{-g(t)}q(t) = e^{-g(1)}q(1) = q(0) = p(R)$ for all $t$. Since $\gamma(0) = \gamma(1)$, we
		have $q(0) = q(1)$ and thus $e^{-g(1)} = 1$. It now follows that $g(1) = 2\pi i k$ for some $k
		\in \Z$.
	\end{proof}
\end{homeworkProblem}

\newpage

\begin{homeworkProblem}
	Suppose $f: G \to \mathbb{C}$ is analytic and define $\varphi: G \times G \to \mathbb{C}$ by 
	\[
	\varphi(z, w) = \frac{f(z) - f(w)}{z - w} \quad \text{if } z \neq w \quad \text{and} \quad \varphi(z, z) = f'(z).
	\]
	Prove that $\varphi$ is continuous and for each fixed $w$, $z \mapsto \varphi(z, w)$ is analytic.

	\begin{proof}
		Fix $z_0, w_0 \in G$. Suppose $z_0 \neq w_0$. Since $f$ is continuous, and $z - w$ is nonzero
		and continuous, $varphi$ is continuous. Suppose $z_0 = w_0$. Pick $\epsilon > 0$. Since $f$ is
		analytic, there exists a $\delta_1 > 0$ such that $|w - z_0| < \delta_1$ implies 
		\[
			|f'(w) - f'(z_0)| < \epsilon/2.
		\]
		Since $f'$ exists, there exists $\delta_2 > 0$ such that for all $z, w \in G$,
		\begin{gather}
			\left|\frac{f(z) - f(w)}{z - w} - f'(w)\right| < \epsilon/2
		\end{gather}
		whenever $|z - w| < \delta_2$. Put $\delta \in (0, \min(\delta_1, \delta_2/2))$. Note that for
		all $z, w \in B_{\delta}(z_0)$, $|z - w| < 2\delta \leq \delta_2$. Hence,
		\[
			\left|\frac{f(z) - f(w)}{z - w} - f'(z)\right| \leq \left|\frac{f(z) - f(w)}{z - w} - f'(w)\right| + |f'(w) - f'(z_0)| < \epsilon
		\]
		for all $z, w \in B_{\delta}(z_0)$, $z \neq w$. Thus, $\varphi$ is continuous at $(z_0, z_0)$.

		We now show that for each fixed $w$, $z \mapsto \varphi(z, w)$ is analytic. If $z \neq w$,
		$\varphi'(z, w) = \frac{f'(z)}{z - w}$ is continuous. Suppose $z = w$. We need to show that
		\[
			\lim_{x \to z} \frac{\varphi(x, w) - \varphi(z, w)}{x - z} = \lim_{x \to z} \frac{\frac{f(x) - f(z)}{x - z} - f'(z)}{x - z}
		\]
		exists and is continuous. Consider the power series expansion of $f$ at $z$.
		\[
			f(x) = f(z) + f'(z)(x - z) + \frac{f''(z)}{2!}(x - z)^2 + \cdots
		\]
		But then
		\begin{align}
			\lim_{x \to z} \frac{\frac{f(x) - f(z)}{x - z} - f'(z)}{x - z} 
			&= \lim_{x \to z} \frac{[f'(z) + \frac{f''(z)}{2!}(x - z) + \cdots] - f'(z)}{x - z} \\
			&= \lim_{x \to z} \frac{f''(z)}{2!} + \frac{f^{(3)}(z)}{3!}(x - z) + \cdots = \frac{f''(z)}{2!}.
		\end{align}
		The result now follows from the analyticity of $f$.
	\end{proof}
\end{homeworkProblem}

\newpage

\begin{homeworkProblem}
	Give the details of the proof of Theorem 5.6: Let $G$ be an open subset of the plane and $f: G \to
	\mathbb{C}$ an analytic function. If $\gamma_1, \dots, \gamma_m$ are closed rectifiable curves in
	$G$ such that 
	\[
	n(\gamma_1; w) + \cdots + n(\gamma_m; w) = 0 \quad \text{for all } w \in \mathbb{C} - G,
	\]
	then for $a \in G - \{\gamma\}$
	\[
	f(a) \sum_{k=1}^m n(\gamma_k; a) = \sum_{k=1}^m \frac{1}{2\pi i} \int_{\gamma_k} \frac{f(z)}{z - a} \, dz.
	\]
	\begin{proof}
		We continue from the textbook. By assumption $H \cup G = \C$. Since $n(\gamma_1; w) + \cdots +
		n(\gamma_m; w)$ is integer valued and continuous, $H$ is open. Define
		\[
			g(z) = \sum_{k=1}^m \int_{\gamma_i} \varphi(z, w) \, dw,
		\]
		if $z \in G$ and 
		\[
			g(z) = \sum_{k=1}^m \int_{\gamma_i} \frac{f(w)}{w - z} \, dw,
		\]
		if $z \in H$. By the proof of Theorem 5.4, $\int_{\gamma_i} \frac{dw}{w - z}$ is well defined
		for all $i$ and $z \in G \cap H$, and so $g$ is well defined for $z \in G \cap H$. Following the
		same argument as in the proof of Theorem 5.4, $g$ is entire as it is a finite sum of entire
		functions. By Theorem 4.4, $H$ contains a neighborhood of $\infty$ in $\C_{\infty}$. Since $f$
		is bounded on $\{\gamma\}$ and $\lim_{z \to \infty} (w - z)^{-1} = 0$ uniformly for $w \in
		\{\gamma\}$, 
		\[
			\lim_{z \to \infty} g(z) = \sum_{k=1}^m \int_{\gamma_i} \lim_{z \to \infty}\frac{f(w)}{w - z} \, dw = 0.
		\]
		Hence there exists $R > 0$ such that $|g(z) \leq 1|$ for all $|z| \geq R$. Since $g$ is bounded
		on $\overline{B}_R(0)$ it follows that $g$ is a bounded entire function and hence constant by
		Liouville's Theorem. But then $\lim_{z \to \infty} g(z) = 0$ so $g \equiv 0$. Hence, for $a \in
		G \backslash \{\gamma\}$, 
		\[
			0 = \sum_{k=1}^m \int_{\gamma_i} \frac{f(w) - f(a)}{w - a} \, dw = \sum_{k=1}^m \int_{\gamma_i} \frac{f(w)}{w - a} \, dw - f(a)\sum_{k=1}^m \int_{\gamma_i} \frac{dw}{w - a},
		\]
		and the result now follows from the definition of $n(\gamma_i; a)$.
	\end{proof}
\end{homeworkProblem}
\end{document}