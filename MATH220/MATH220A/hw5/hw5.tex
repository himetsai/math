\documentclass{article}

\usepackage{fancyhdr}
\usepackage{extramarks}
\usepackage{amsmath}
\usepackage{amsthm}
\usepackage{amsfonts}
\usepackage{tikz}
\usepackage[plain]{algorithm}
\usepackage{algpseudocode}
\usepackage{enumerate}
\usepackage{amssymb}

\usetikzlibrary{automata,positioning}

%
% Basic Document Settings
%

\topmargin=-0.45in
\evensidemargin=0in
\oddsidemargin=0in
\textwidth=6.5in
\textheight=9.0in
\headsep=0.25in

\linespread{1.1}

\pagestyle{fancy}
\lhead{\hmwkAuthorName}
\chead{\hmwkClass:\ \hmwkTitle}
\rhead{\firstxmark}
\lfoot{\lastxmark}
\cfoot{\thepage}

\renewcommand\headrulewidth{0.4pt}
\renewcommand\footrulewidth{0.4pt}

\setlength\parindent{0pt}
\setlength{\parskip}{5pt}

%
% Create Problem Sections
%

\newcommand{\enterProblemHeader}[1]{
    \nobreak\extramarks{}{Problem \arabic{#1} continued on next page\ldots}\nobreak{}
    \nobreak\extramarks{Problem \arabic{#1} (continued)}{Problem \arabic{#1} continued on next page\ldots}\nobreak{}
}

\newcommand{\exitProblemHeader}[1]{
    \nobreak\extramarks{Problem \arabic{#1} (continued)}{Problem \arabic{#1} continued on next page\ldots}\nobreak{}
    \stepcounter{#1}
    \nobreak\extramarks{Problem \arabic{#1}}{}\nobreak{}
}

\setcounter{secnumdepth}{0}
\newcounter{partCounter}
\newcounter{homeworkProblemCounter}
\setcounter{homeworkProblemCounter}{1}
\nobreak\extramarks{Problem \arabic{homeworkProblemCounter}}{}\nobreak{}

%
% Homework Problem Environment
%
% This environment takes an optional argument. When given, it will adjust the
% problem counter. This is useful for when the problems given for your
% assignment aren't sequential. See the last 3 problems of this template for an
% example.
%
\newenvironment{homeworkProblem}[1][-1]{
    \ifnum#1>0
        \setcounter{homeworkProblemCounter}{#1}
    \fi
    \section{Problem \arabic{homeworkProblemCounter}}
    \setcounter{partCounter}{1}
    \enterProblemHeader{homeworkProblemCounter}
}{
    \exitProblemHeader{homeworkProblemCounter}
}

%
% Homework Details
%   - Title
%   - Due date
%   - Class
%   - Section/Time
%   - Instructor
%   - Author
%

\newcommand{\hmwkTitle}{Homework\ \#5}
\newcommand{\hmwkDueDate}{Nov 1, 2024}
\newcommand{\hmwkClass}{MATH 220A}
\newcommand{\hmwkClassInstructor}{Professor Ebenfelt}
\newcommand{\hmwkAuthorName}{\textbf{Ray Tsai}}
\newcommand{\hmwkPID}{A16848188}

%
% Title Page
%

\title{
    \vspace{2in}
    \textmd{\textbf{\hmwkClass:\ \hmwkTitle}}\\
    \normalsize\vspace{0.1in}\small{Due\ on\ \hmwkDueDate\ at 23:59pm}\\
    \vspace{0.1in}\large{\textit{\hmwkClassInstructor}} \\
    \vspace{3in}
}

\author{
  \hmwkAuthorName \\
  \vspace{0.1in}\small\hmwkPID
}
\date{}

\renewcommand{\part}[1]{\textbf{\large Part \Alph{partCounter}}\stepcounter{partCounter}\\}

%
% Various Helper Commands
%

% Useful for algorithms
\newcommand{\alg}[1]{\textsc{\bfseries \footnotesize #1}}

% For derivatives
\newcommand{\deriv}[1]{\frac{\mathrm{d}}{\mathrm{d}x} (#1)}

% For partial derivatives
\newcommand{\pderiv}[2]{\frac{\partial}{\partial #1} (#2)}

% Integral dx
\newcommand{\dx}{\mathrm{d}x}

% Probability commands: Expectation, Variance, Covariance, Bias
\newcommand{\Var}{\mathrm{Var}}
\newcommand{\Cov}{\mathrm{Cov}}
\newcommand{\Bias}{\mathrm{Bias}}
\newcommand*{\Z}{\mathbb{Z}}
\newcommand*{\Q}{\mathbb{Q}}
\newcommand*{\R}{\mathbb{R}}
\newcommand*{\C}{\mathbb{C}}
\newcommand*{\N}{\mathbb{N}}
\newcommand*{\prob}{\mathds{P}}
\newcommand*{\E}{\mathds{E}}

\begin{document}

\maketitle

\pagebreak

\begin{homeworkProblem}
	Suppose $f: G \to \mathbb{C}$ is analytic and that $G$ is connected. Show that if $f(z)$ is real
	for all $z$ in $G$ then $f$ is constant.

	\begin{proof}
		Put $f(x + iy) = u(x, y) + iv(x, y)$, where $u, v$ are real-valued functions. Since $f$ is
		real-valued, $v(x, y) = 0$ for all $x, y \in G$. By the Cauchy-Riemann equations,
		\[
			u_x = v_y = 0 \quad \quad u_y = -v_y = 0.
		\]
		But then $u$ is constant, and thus $f$ is constant.
	\end{proof}
\end{homeworkProblem}

\newpage

\begin{homeworkProblem}
	Find an open connected set $G \subset \mathbb{C}$ and two continuous functions $f$ and $g$ defined
	on $G$ such that $f(z)^2 = g(z)^2 = 1 - z^2$ for all $z$ in $G$. Can you make $G$ maximal? Are $f$
	and $g$ analytic?

	\begin{proof}
		Let $G = (\C \backslash \R) \cup [-1, 1]$. Consider $f(z) = \exp(\frac{1}{2}Log(1 - z^2))$ and
		$g(z) = \exp(\frac{1}{2}Log(1 - z^2))$. Then $f(z)^2 = g(z)^2 = 1 - z^2$ for all $z \in G$.
		Notice that $G$ is maximal in $\C$, as any larger set would make $1 - z^2 \in \R_{\leq 0}$,
		which makes $Log(1 - z^2)$ undefined. Since $f, g$ are compositions of analytic functions, they
		are analytic.
	\end{proof}
\end{homeworkProblem}

\newpage

\begin{homeworkProblem}
	Let $G$ be a region and define $G^* = \{ z : \overline{z} \in G \}$. If $f : G \rightarrow
	\mathbb{C}$ is analytic, prove that $f^* : G^* \rightarrow \mathbb{C}$, defined by $f^*(z) =
	\overline{f(\overline{z})}$, is also analytic.

	\begin{proof}
		Let $z = x + iy$ and $f(z) = u(x, y) + iv(x, y)$. Then $f^*(z) = u(x, -y) - iv(x, -y)$.
		By the Cauchy-Riemann equations, $u_x = v_y$ and $u_y = -v_x$, and so
		\[
			\partial_x u(x, -y) = -\partial_y v(x, -y) = \partial_y [-v(x, -y)], \quad \partial_y u(x, -y) = \partial_x v(x, -y) = -\partial_x [-v(x, -y)].
		\]
		Thus, $f^*$ is analytic.
	\end{proof}
\end{homeworkProblem}

\newpage

\begin{homeworkProblem}
	Prove that there is no branch of the logarithm defined on $G = \mathbb{C} \setminus \{ 0 \}$.
	(Hint: Suppose such a branch exists and compare this with the principal branch.)

	\begin{proof}
		Denote $Log$ as the principal branch of the logarithm and let $H$ be its domain. Suppose there
		exists a branch of the logarithm $f$ defined on $G$. There exists $k \in \Z$ such that $f(z) =
		Log(z) + i2\pi k$, for all $z \in H$. Consider the limit of $Log$ at $z = -1$. Approaching from
		above and below the real axis, we get
		\[
			\lim_{\theta \to \pi} \log |z| + i\theta = i\pi \neq -i\pi = \lim_{\theta \to -\pi} \log |z| + i\theta,
		\]
		so $\lim_{z \to -1} Log(z)$ does not exist. But then $\lim_{z \to -1} f(z)$ does not exist,
		contradicting the continuity of $f$.
	\end{proof}
\end{homeworkProblem}
\end{document}