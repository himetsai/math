\documentclass{article}

\usepackage{fancyhdr}
\usepackage{extramarks}
\usepackage{amsmath}
\usepackage{amsthm}
\usepackage{amsfonts}
\usepackage{tikz}
\usepackage[plain]{algorithm}
\usepackage{algpseudocode}
\usepackage{enumerate}
\usepackage{amssymb}

\usetikzlibrary{automata,positioning}

%
% Basic Document Settings
%

\topmargin=-0.45in
\evensidemargin=0in
\oddsidemargin=0in
\textwidth=6.5in
\textheight=9.0in
\headsep=0.25in

\linespread{1.1}

\pagestyle{fancy}
\lhead{\hmwkAuthorName}
\chead{\hmwkClass:\ \hmwkTitle}
\rhead{\firstxmark}
\lfoot{\lastxmark}
\cfoot{\thepage}

\renewcommand\headrulewidth{0.4pt}
\renewcommand\footrulewidth{0.4pt}

\setlength\parindent{0pt}
\setlength{\parskip}{5pt}

%
% Create Problem Sections
%

\newcommand{\enterProblemHeader}[1]{
    \nobreak\extramarks{}{Problem \arabic{#1} continued on next page\ldots}\nobreak{}
    \nobreak\extramarks{Problem \arabic{#1} (continued)}{Problem \arabic{#1} continued on next page\ldots}\nobreak{}
}

\newcommand{\exitProblemHeader}[1]{
    \nobreak\extramarks{Problem \arabic{#1} (continued)}{Problem \arabic{#1} continued on next page\ldots}\nobreak{}
    \stepcounter{#1}
    \nobreak\extramarks{Problem \arabic{#1}}{}\nobreak{}
}

\setcounter{secnumdepth}{0}
\newcounter{partCounter}
\newcounter{homeworkProblemCounter}
\setcounter{homeworkProblemCounter}{1}
\nobreak\extramarks{Problem \arabic{homeworkProblemCounter}}{}\nobreak{}

%
% Homework Problem Environment
%
% This environment takes an optional argument. When given, it will adjust the
% problem counter. This is useful for when the problems given for your
% assignment aren't sequential. See the last 3 problems of this template for an
% example.
%
\newenvironment{homeworkProblem}[1][-1]{
    \ifnum#1>0
        \setcounter{homeworkProblemCounter}{#1}
    \fi
    \section{Problem \arabic{homeworkProblemCounter}}
    \setcounter{partCounter}{1}
    \enterProblemHeader{homeworkProblemCounter}
}{
    \exitProblemHeader{homeworkProblemCounter}
}

%
% Homework Details
%   - Title
%   - Due date
%   - Class
%   - Section/Time
%   - Instructor
%   - Author
%

\newcommand{\hmwkTitle}{Homework\ \#3}
\newcommand{\hmwkDueDate}{Oct 18, 2024}
\newcommand{\hmwkClass}{MATH 220A}
\newcommand{\hmwkClassInstructor}{Professor Ebenfelt}
\newcommand{\hmwkAuthorName}{\textbf{Ray Tsai}}
\newcommand{\hmwkPID}{A16848188}

%
% Title Page
%

\title{
    \vspace{2in}
    \textmd{\textbf{\hmwkClass:\ \hmwkTitle}}\\
    \normalsize\vspace{0.1in}\small{Due\ on\ \hmwkDueDate\ at 23:59pm}\\
    \vspace{0.1in}\large{\textit{\hmwkClassInstructor}} \\
    \vspace{3in}
}

\author{
  \hmwkAuthorName \\
  \vspace{0.1in}\small\hmwkPID
}
\date{}

\renewcommand{\part}[1]{\textbf{\large Part \Alph{partCounter}}\stepcounter{partCounter}\\}

%
% Various Helper Commands
%

% Useful for algorithms
\newcommand{\alg}[1]{\textsc{\bfseries \footnotesize #1}}

% For derivatives
\newcommand{\deriv}[1]{\frac{\mathrm{d}}{\mathrm{d}x} (#1)}

% For partial derivatives
\newcommand{\pderiv}[2]{\frac{\partial}{\partial #1} (#2)}

% Integral dx
\newcommand{\dx}{\mathrm{d}x}

% Probability commands: Expectation, Variance, Covariance, Bias
\newcommand{\Var}{\mathrm{Var}}
\newcommand{\Cov}{\mathrm{Cov}}
\newcommand{\Bias}{\mathrm{Bias}}
\newcommand*{\Z}{\mathbb{Z}}
\newcommand*{\Q}{\mathbb{Q}}
\newcommand*{\R}{\mathbb{R}}
\newcommand*{\C}{\mathbb{C}}
\newcommand*{\N}{\mathbb{N}}
\newcommand*{\prob}{\mathds{P}}
\newcommand*{\E}{\mathds{E}}

\begin{document}

\maketitle

\pagebreak

\begin{homeworkProblem}
  Show that the closure of a totally bounded set is totally bounded.

  \begin{proof}
    Suppose not. Let $X$ be a totally bounded set. Let $\epsilon > 0$. There exist finite number of
    points $x_1, \ldots, x_n \in X$ such that $X \subset \bigcup_{i = 1}^n B_{\epsilon/2}(x_i)$. But
    then
    \[
      \overline{X} \subset \overline{\bigcup_{i = 1}^n B_{\epsilon/2}(x_i)} \subset \bigcup_{i = 1}^n \overline{B_{\epsilon/2}(x_i)} \subset \bigcup_{i = 1}^n B_{\epsilon}(x_i).
    \]
  \end{proof}
\end{homeworkProblem}

\newpage

\begin{homeworkProblem}
  We say that $f: X \to \mathbb{C}$ is bounded if there is a constant $M > 0$ with $|f(x)| \leq M
  \text{ for all } x \in X. $ Show that if $f$ and $g$ are bounded uniformly continuous (Lipschitz)
  functions from $X$ into $\mathbb{C}$, then so is $fg$.

  \begin{proof}
    Since there exist $M, N$ such that $|f(x)| \leq M$ and $|g(x)| \leq N$ for all $x \in X$, 
    \[
      |fg(x)| = |f(x)g(x)| \leq |f(x)||g(x)| \leq MN
    \]
    for all $x \in X$, and thus $fg$ is bounded. Now, let $\epsilon > 0$. Since $f$ and $g$ are
    uniformly continuous, there exists $\nu$ such that $|f(x) - f(y)| < \epsilon/(M + N)$ and $|g(x)
    - g(y)| < \epsilon/(M + N)$ whenever $d(x, y) < \delta$. Then,
    \begin{align*}
      |fg(x) - fg(y)| 
      &= |f(x)g(x) + f(x)g(y) - f(x)g(y) - f(y)g(y)| \\
      &= |f(x)(g(x) - g(y)) + g(y)(f(x) - f(y))| \\
      &\leq |f(x)||(g(x) - g(y))| + |g(y)||(f(x) - f(y))| \\
      &< \epsilon M/(M + N) + \epsilon N/(M + N) < \epsilon,
    \end{align*}
    whenever $d(x, y) < \delta$. Thus, $fg$ is uniformly continuous.

    Suppose that $f$ and $g$ are Lipschitz functions. Then, there exists $K$ such that $|f(x) -
    f(y)|, |g(x) - g(y)| \leq Kd(x, y)$ for all $x, y \in X$. Through the same calculation as above,
    we have 
    \[
      |fg(x) - fg(y)| \leq |f(x)||(g(x) - g(y))| + |g(y)||(f(x) - f(y))| \leq K(M + N)d(x, y),
    \]
    and thus $fg$ is Lipschitz.
  \end{proof}
\end{homeworkProblem}

\newpage

\begin{homeworkProblem}
  Suppose $f: X \to \Omega$ is uniformly continuous; show that if $\{x_n\}$ is a Cauchy sequence in
  $X$, then $\{f(x_n)\}$ is a Cauchy sequence in $\Omega$. Is this still true if we only assume that
  $f$ is continuous? (Prove or give a counterexample.)

  \begin{proof}
    Let $d$ and $\rho$ each denote the metric on $X$ and $\Omega$, respectively. Pick $\epsilon >
    0$. Since $f$ is uniformly continuous, there exists $\delta > 0$ such that $d(x, y) < \delta$
    implies $\rho(f(x), f(y)) < \epsilon$. Since $\{x_n\}$ is Cauchy, there exists $N$ such that
    $d(x_n, x_m) < \delta$ whenever $n, m \geq N$. But then $\rho(f(x_n), f(x_m)) < \epsilon$
    whenever $n, m \geq N$, and thus $\{f(x_n)\}$ is Cauchy.

    If $f$ is only continuous, then the statement is not necessarily true. Consider the sequence
    $\{\frac{1}{n}\}_{n \in \N}$ and function $f: (0, 1) \to \R$ defined by $f(x) = \frac{1}{x}$.
    $\{\frac{1}{n}\}_{n \in \N}$ is Cauchy as it converges to $0$. We also know that $f$ is
    continuous. But then $\{f(n)\}_{n \in \N} \to \infty$ so it is not Cauchy.
  \end{proof}
\end{homeworkProblem}

\newpage

\begin{homeworkProblem}
  Recall the definition of a dense set (1.14). Suppose that $\Omega$ is a complete metric space and
  that $f: (D, d) \to (\Omega, \rho)$ is uniformly continuous, where $D$ is dense in $(X, d)$. Use
  the last problem to show that there is a uniformly continuous function $g: X \to \Omega$ with
  $g(x) = f(x)$ for every $x$ in $D$.

  \begin{proof}
    Let $x \in X$. Since $D$ is dense in $X$, there exists a sequence $\{x_n\} \subseteq D$ such
    that $x_n \to x$, and so $\{x_n\}$ is Cauchy. Since $f$ is uniformly continuous, $\{f(x_n)\}$ is
    also Cauchy, by the result of the previous problem. Since $\Omega$ is complete, $\{f(x_n)\}$
    converges to some $y \in \Omega$. Define $g: X \to \Omega$ by $g(x) = y$. Note that $g(x) =
    f(x)$ for all $x \in D$, as $g(x) = \lim_{n \to \infty} f(x_n) = f(x)$. 
    
    We claim that $g$ is uniformly continuous. Pick $\epsilon > 0$. Since $f$ is uniformly
    continuous, there exists $\delta$ such that $\rho(f(x), f(y)) < \frac{\epsilon}{3}$ whenever
    $d(x, y) < \delta$. Suppose $x, y \in X$ with $d(x, y) < \frac{\delta}{3}$. There exist
    sequences $\{x_n\}, \{y_n\} \subseteq D$ with $x_n \to x$ and $y_n \to y$, and thus there exists
    $N_1$ such that $d(x_n, x), d(y_n, y) < \frac{\delta}{3}$ whenever $n \geq N_1$. Since $d(x_n,
    y_n) \leq d(x_n, x) + d(x, y) + d(y, y_n) < \delta$, we have $\rho(f(x_n), f(y_n)) <
    \frac{\epsilon}{3}$ for all $n \geq N_1$. Since $f(x_n) \to g(x)$ and $f(y_n) \to g(y)$, there
    exists $N_2$ such that $\rho(f(x_n), g(x)), \rho(f(y_n), g(y)) < \frac{\epsilon}{3}$ whenever $n
    \geq N_2$. It now follows that for all $d(x, y) < \frac{\delta}{3}$, we may find $n \geq
    \max(N_1, N_2)$ such that
    \[
      \rho(g(x), g(y)) \leq \rho(g(x), f(x_n)) + \rho(f(x_n), f(y_n)) + \rho(f(y_n), g(y)) < \epsilon.
    \]
  \end{proof}
\end{homeworkProblem}

\newpage

\begin{homeworkProblem}
  Let $G$ be an open subset of $\mathbb{C}$ and let $P$ be a polygon in $G$ from $a$ to $b$. Use
  Theorems 5.15 and 5.17 to show that there is a polygon $Q \subseteq G$ from $a$ to $b$ which is
  composed of line segments that are parallel to either the real or imaginary axes.

  \begin{proof}
    Since $P$ is a polygon, $P = [z_1, z_2] \cup [z_{n}, z_{n + 1}]$ is a union of finitely many
    line intervals, where $z_1 = a, z_2, \ldots, z_n, z_{n + 1} = b \in G$. But then each $[z_k,
    z_{k + 1}]$ is compact, so $P$ is compact. By theorem 5.17, we have $d(\C \backslash G, P) > 0$.
    For each interval $[z_k, z_{k + 1}]$ in $P$, define function $f_k: [z_k, z_{k + 1}] \to \R$ as
    the Manhattan distanct from $z \in [z_k, z_{k + 1}]$ to $z_k$ on the complex plane, i.e. $f_k(z)
    = |Re(z) - Re(z_k)| + |Im(z) - Im(z_k)|$. We claim that $f_k$ is continuous. Pick $\epsilon > 0$
    and let $z \in [z_k, z_{k + 1}]$. Let $\nu \in (0, \epsilon/\pi)$. Since every point in $[z_k,
    z_{k + 1}]$ are on the same line, $Re(w) - Re(z_k)$ have the same sign for all $w \in [z_k, z_{k
    + 1}]$, and thus $\left||Re(z) - Re(z_k)| - |Re(w) - Re(z_k)|\right| = |Re(z) - Re(w)|$. Hence,
    for all $w \in B_{\nu}(z) \cap [z_k, z_{k + 1}]$,
    \begin{align*}
      |f_k(z) - f_k(w)| &= |(|Re(z) - Re(z_k)| + |Im(z) - Im(z_k)|) - (|Re(w) - Re(z_k)| + |Im(w) - Im(z_k)|)| \\
      &\leq |(|Re(z) - Re(z_k)| - |Re(w) - Re(z_k)|)| + |(|Im(z) - Im(z_k)| - |Im(w) - Im(z_k)|)| \\
      &= |Re(z) - Re(w)| + |Im(z) - Im(w)| < \pi d(z, w) < \epsilon,
    \end{align*}
    where the last inequality follows from the fact that the perimeter of a triangle inscribed in a
    circle is less than the circumference of the circle. Thus, $f_k$ is continuous for all $k$. By
    theorem 5.15, $f_k$ is uniformly continuous, so there exists $\delta$ such that for all $z, w
    \in [z_k, z_{k + 1}]$ with $d(z, w) < \delta$, we have $|f_k(z) - f_k(w)| < d(\C \backslash G,
    P)$. We may now partition $[z_k, z_{k + 1}]$ into finitely many intervals of length less than
    $\delta$, with endpoints $z_k = w_0, w_1, \ldots, w_m = z_{k + 1}$. Since $|f_k(w_i) - f_k(w_{i
    + 1})| < d(\C \backslash G, P)$ for all $i$,
    \[
      [Re(w_i), Re(w_{i + 1})] \cup [Im(w_i), Im(w_{i + 1})] \subset G.
    \]
    The result now follows.
  \end{proof}
\end{homeworkProblem}

\newpage

\begin{homeworkProblem}
  Let $\{f_n\}$ be a sequence of uniformly continuous functions from $(X, d)$ into $(\Omega, \rho)$
  and suppose that $f = u-\lim f_n$ exists. Prove that $f$ is uniformly continuous. If each $f_n$ is
  a Lipschitz function with constant $M_n$ and $\sup M_n < \infty$, show that $f$ is a Lipschitz
  function. If $\sup M_n = \infty$, show that $f$ may fail to be Lipschitz.

  \begin{proof}
    Pick $\epsilon > 0$. There exists $n$ such that $\rho(f_n(x), f(x)) < \epsilon/3$ for all $x \in
    X$. Since $f_n$ is uniformly continuous, there exists $\delta$ such that $\rho(f_n(x), f_n(y)) <
    \epsilon/3$ whenever $d(x, y) < \delta$. Then, whenever $d(x, y) < \delta$,
    \[
      \rho(f(x), f(y)) \leq \rho(f(x), f_n(x)) + \rho(f_n(x), f_n(y)) + \rho(f_n(y), f(y)) < \epsilon,
    \]
    and thus $f$ is uniformly continuous.

    Suppose that each $f_n$ is Lipschitz with constant $M_n$ and $\sup M_n < \infty$. Given $x, y
    \in X$, there exists $n$ such that $\rho(f_n(z), f(z)) < d(x, y)$ for all $z \in X$. It now
    follows that
    \[
      \rho(f(x), f(y)) \leq \rho(f(x), f_n(x)) + \rho(f_n(x), f_n(y)) + \rho(f_n(y), f(y)) < (M_n + 2) d(x, y).
    \]

    However, this does not work in the general case. Consider $f: [0, 1] \to \R$ defined by $f(x) =
    \sqrt{x}$. Given any $K \in \R$, pick $\epsilon \in (0, \frac{1}{K^2})$ and we have
    \[
      |f(\epsilon) - f(0)| = \left|\sqrt{\epsilon} - 0\right| = \sqrt{\epsilon} > K\epsilon.
    \]
    Hence, $f$ is not Lipschitz. But then by the Weierstrass approximation theorem, there exists a
    sequence of polynomials $\{p_n\}$ on $[a, b]$ such that $p_n \to f$ uniformly. Since $\sup_{x
    \in [a, b]} |p_n'(x)| < \infty$ for all $n$, 
    \[
      |p_n(x) - p_n(y)| < 2|p_n'(x)||x - y|
    \]
    for all $x, y \in [a, b]$, which makes $p_n$ Lipschitz.

  \end{proof}
\end{homeworkProblem}
\end{document}