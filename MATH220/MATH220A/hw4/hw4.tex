\documentclass{article}

\usepackage{fancyhdr}
\usepackage{extramarks}
\usepackage{amsmath}
\usepackage{amsthm}
\usepackage{amsfonts}
\usepackage{tikz}
\usepackage[plain]{algorithm}
\usepackage{algpseudocode}
\usepackage{enumerate}
\usepackage{amssymb}

\usetikzlibrary{automata,positioning}

%
% Basic Document Settings
%

\topmargin=-0.45in
\evensidemargin=0in
\oddsidemargin=0in
\textwidth=6.5in
\textheight=9.0in
\headsep=0.25in

\linespread{1.1}

\pagestyle{fancy}
\lhead{\hmwkAuthorName}
\chead{\hmwkClass:\ \hmwkTitle}
\rhead{\firstxmark}
\lfoot{\lastxmark}
\cfoot{\thepage}

\renewcommand\headrulewidth{0.4pt}
\renewcommand\footrulewidth{0.4pt}

\setlength\parindent{0pt}
\setlength{\parskip}{5pt}

%
% Create Problem Sections
%

\newcommand{\enterProblemHeader}[1]{
    \nobreak\extramarks{}{Problem \arabic{#1} continued on next page\ldots}\nobreak{}
    \nobreak\extramarks{Problem \arabic{#1} (continued)}{Problem \arabic{#1} continued on next page\ldots}\nobreak{}
}

\newcommand{\exitProblemHeader}[1]{
    \nobreak\extramarks{Problem \arabic{#1} (continued)}{Problem \arabic{#1} continued on next page\ldots}\nobreak{}
    \stepcounter{#1}
    \nobreak\extramarks{Problem \arabic{#1}}{}\nobreak{}
}

\setcounter{secnumdepth}{0}
\newcounter{partCounter}
\newcounter{homeworkProblemCounter}
\setcounter{homeworkProblemCounter}{1}
\nobreak\extramarks{Problem \arabic{homeworkProblemCounter}}{}\nobreak{}

%
% Homework Problem Environment
%
% This environment takes an optional argument. When given, it will adjust the
% problem counter. This is useful for when the problems given for your
% assignment aren't sequential. See the last 3 problems of this template for an
% example.
%
\newenvironment{homeworkProblem}[1][-1]{
    \ifnum#1>0
        \setcounter{homeworkProblemCounter}{#1}
    \fi
    \section{Problem \arabic{homeworkProblemCounter}}
    \setcounter{partCounter}{1}
    \enterProblemHeader{homeworkProblemCounter}
}{
    \exitProblemHeader{homeworkProblemCounter}
}

%
% Homework Details
%   - Title
%   - Due date
%   - Class
%   - Section/Time
%   - Instructor
%   - Author
%

\newcommand{\hmwkTitle}{Homework\ \#7}
\newcommand{\hmwkDueDate}{Nov 15, 2024}
\newcommand{\hmwkClass}{MATH 220A}
\newcommand{\hmwkClassInstructor}{Professor Ebenfelt}
\newcommand{\hmwkAuthorName}{\textbf{Ray Tsai}}
\newcommand{\hmwkPID}{A16848188}

%
% Title Page
%

\title{
    \vspace{2in}
    \textmd{\textbf{\hmwkClass:\ \hmwkTitle}}\\
    \normalsize\vspace{0.1in}\small{Due\ on\ \hmwkDueDate\ at 23:59pm}\\
    \vspace{0.1in}\large{\textit{\hmwkClassInstructor}} \\
    \vspace{3in}
}

\author{
  \hmwkAuthorName \\
  \vspace{0.1in}\small\hmwkPID
}
\date{}

\renewcommand{\part}[1]{\textbf{\large Part \Alph{partCounter}}\stepcounter{partCounter}\\}

%
% Various Helper Commands
%

% Useful for algorithms
\newcommand{\alg}[1]{\textsc{\bfseries \footnotesize #1}}

% For derivatives
\newcommand{\deriv}[1]{\frac{\mathrm{d}}{\mathrm{d}x} (#1)}

% For partial derivatives
\newcommand{\pderiv}[2]{\frac{\partial}{\partial #1} (#2)}

% Integral dx
\newcommand{\dx}{\mathrm{d}x}

% Probability commands: Expectation, Variance, Covariance, Bias
\newcommand{\Var}{\mathrm{Var}}
\newcommand{\Cov}{\mathrm{Cov}}
\newcommand{\Bias}{\mathrm{Bias}}
\newcommand*{\Z}{\mathbb{Z}}
\newcommand*{\Q}{\mathbb{Q}}
\newcommand*{\R}{\mathbb{R}}
\newcommand*{\C}{\mathbb{C}}
\newcommand*{\N}{\mathbb{N}}
\newcommand*{\prob}{\mathds{P}}
\newcommand*{\E}{\mathds{E}}

\begin{document}

\maketitle

\pagebreak

\begin{homeworkProblem}
	Prove that $\limsup (a_n + b_n) \leq \limsup a_n + \limsup b_n$ and $\liminf (a_n + b_n) \geq \liminf a_n + \liminf b_n$ for $\{a_n\}$ and $\{b_n\}$ sequences of real numbers.

	\begin{proof}
		Let $A = \limsup a_n$ and $B = \limsup b_n$. Pick $\epsilon > 0$. Then there exists $N_1, N_2$ such that $a_n \leq A + \epsilon/2$ for all $n \geq N_1$, and $b_n \leq B + \epsilon/2$ for all $n \geq N_2$. Put $N = \max (N_1, N_2)$. Then for all $n \geq N$, we have $a_n + b_n \leq A + B + \epsilon$. But then $\epsilon$ is arbitrary, and thus $\limsup (a_n + b_n) \leq A + B$. 

		Let $A = \liminf a_n$ and $B = \liminf b_n$. Pick $\epsilon > 0$. Then there exists $N_1, N_2$ such that $a_n \geq A - \epsilon/2$ for all $n \geq N_1$, and $b_n \geq B - \epsilon/2$ for all $n \geq N_2$. Put $N = \max (N_1, N_2)$. Then for all $n \geq N$, we have $a_n + b_n \geq A + B - \epsilon$. But then $\epsilon$ is arbitrary, and thus $\liminf (a_n + b_n) \geq A + B$.
	\end{proof}
\end{homeworkProblem}

\newpage

\begin{homeworkProblem}
	Find the radius of convergence for each of the following power series:

	\begin{enumerate}[(a)]
		\item $\sum_{n=0}^{\infty} a^n z^n$, $a \in \mathbb{C}$
		\begin{proof}
			By the comparison test, the radius of convergence is $R = \lim |a^n/a^{n + 1}| = \frac{1}{|a|}$ when $a \neq 0$, and $R = \infty$ when $a = 0$.
		\end{proof}
		\item $\sum_{n=0}^{\infty} a^{n^2} z^n$, $a \in \mathbb{C}$
		\begin{proof}
			By the comparison test, the radius of convergence is
			\[
				R = \lim |a^{n^2}/a^{(n + 1)^2}| = \lim |a^{-2n - 1}| = \begin{cases}
					0 & \text{if } |a| > 1 \\
					1 & \text{if } |a| = 1 \\
					\infty & \text{if } |a| < 1
				\end{cases}.
			\]
		\end{proof}
		\item $\sum_{n=0}^{\infty} k^n z^n$, $k$ an integer $\neq 0$
		\begin{proof}
			By the comparison test, the radius of convergence is $R = \lim |k^n/k^{n + 1}| = \frac{1}{|k|}$.
		\end{proof}
		\item $\sum_{n=0}^{\infty} z^{n!}$
		\begin{proof}
			Note that
			\[
				\sum_{n=0}^{\infty} z^{n!} = \sum_{k=0}^{\infty} a_kz^{k},
			\]
			where $a_1 = 2$, $a_k = 1$ if $k = n!$ for $n \in \Z_{\geq 2}$, and $a_k = 0$ otherwise. Then by the root test, the radius of convergence is 
			\[
				R = \frac{1}{\limsup |a_k|^{1/k}} = 1.
			\]
		\end{proof}
	\end{enumerate}
\end{homeworkProblem}

\newpage

\begin{homeworkProblem}
	Show that the radius of convergence of the power series
	\[
		\sum_{n=1}^{\infty} \frac{(-1)^n}{n} z^{n(n+1)}
	\]
	is 1, and discuss convergence for $z = 1$, $-1$, and $i$. (Hint: The $n$th coefficient of this series is not $(-1)^n / n$.)

	\begin{proof}
		Note that
		\[
			\sum_{n=1}^{\infty} \frac{(-1)^n}{n} z^{n(n+1)} = \sum_{k=0}^{\infty} a_kz^{k},
		\]
		where $a_k = \frac{(-1)^n}{n}$ if there exists $n$ such that $k = n(n + 1)$, otherwise $a_k = 0$. Then by the root test, 
		\[
			\frac{1}{R} = \limsup |a_k|^{1/k} = \limsup \left| \frac{(-1)^n}{n} \right|^{1/n(n+1)} = \limsup n^{-1/n(n+1)} = \limsup e^{-\ln n / n(n + 1)} = 1,
		\]
		as $\lim \frac{\ln n}{n(n + 1)} = 0$. Thus the radius of convergence is $R = 1$.

		When $z = 1$,
		\[
			\sum_{n=1}^{\infty} \frac{(-1)^n}{n} z^{n(n+1)} = \sum_{n=1}^{\infty} \frac{(-1)^n}{n},
		\]
		which converges by the alternating test. 

		When $z = -1$, since $n(n + 1)$ is even,
		\[
			\sum_{n=1}^{\infty} \frac{(-1)^n}{n} z^{n(n+1)} = \sum_{n=1}^{\infty} \frac{(-1)^{n + n(n+1)}}{n} = \sum_{n=1}^{\infty} \frac{(-1)^{n}}{n},
		\]
		so the series again converges by the alternating test.

		When $z = i$, since $n(n + 1)$ is even
		\[
			\sum_{n=1}^{\infty} \frac{(-1)^n}{n} z^{n(n+1)} = \sum_{n=1}^{\infty} \frac{(-1)^{n + \frac{n(n + 1)}{2}}}{n} = \sum_{n=1}^{\infty} \frac{(-1)^{\frac{n(n + 3)}{2}}}{n} = \sum_{n=1}^{\infty} a_n.
		\]
		where
		\[
			a_{n} = \frac{(-1)^{\frac{n(n + 3)}{2}}}{n} = \begin{cases}
				\frac{1}{n} & \text{if } n \equiv 0, 1 \pmod{4} \\
				-\frac{1}{n} & \text{if } n \equiv 2, 3 \pmod{4} \end{cases}.
		\] 	
		Put $b_0 = a_1$, $b_k = a_{2k} + a_{2k + 1} = (-1)^{k}\left(\frac{1}{2k} + \frac{1}{2k + 1}\right)$ for $k \geq 1$. Then $\sum_{n = 1}^{\infty} a_n = \sum_{k = 0}^{\infty} b_k$. But then $|b_k|$ decreases monotonically and $\lim b_k = 0$, so the series converges by the alternating test.
	\end{proof}
\end{homeworkProblem}

\newpage

\begin{homeworkProblem}
	Show that $f(z) = |z|^2 = x^2 + y^2$ has a derivative only at the origin.

	\begin{proof}
		Suppose that $f'(z)$ exists for some $z \in \C$. Then
		\[
			f'(z) = \lim_{h \to 0} \frac{f(z + h) - f(h)}{h} = \lim_{h \to 0} \frac{(z + h)(\bar{z} + \bar{h}) - z\bar{z}}{h} = \lim_{h \to 0} \frac{z\bar{h} + \overline{z\bar{h}} + h\bar{h}}{h} = \lim_{h \to 0} \frac{2Re(z\bar{h})}{h} + \bar{h}.
		\]
		Suppose $\{h_n\} \to 0$. If $\{h_n\} \subseteq \R$, then
		\[
			f'(z) = \lim_{n \to \infty} \frac{2Re(zh_n)}{h_n} + h_n = \lim_{n \to \infty} \frac{2h_nx}{h_n} = 2x.
		\]
		If $\{h_n\} \subseteq i\R$, then
		\[
			f'(z) = \lim_{n \to \infty} \frac{2Re(-zh_n)}{h_n} - h_n = \lim_{n \to \infty} \frac{2h_ny}{ih_n} = -2yi.
		\]
		Since $f'(z) = 2x = 2yi$, we must have $x = y = 0$, so $z = 0$. Thus $f'(z)$ only exists at the origin.
	\end{proof}
\end{homeworkProblem}

\newpage

\begin{homeworkProblem}
	Describe the following sets: 
	\begin{enumerate}[(a)]
		\item $\{z: e^z = i\}$
		\begin{proof}
			Put $z = x + iy$, where $x, y \in \R$. We have $e^z = e^{x + iy} = e^xe^{iy} = i$. Then $e^x = 1$ and $e^{iy} = \cos y + i\sin y = i$. Hence $x = 0$ and $y = \frac{\pi}{2} + 2\pi k$ for some $k \in \Z$, which yields
			\[
				\{z: e^z = i\} = \left\{\frac{(4k + 1)i\pi}{2} \mid k \in \Z\right\}.
			\]
		\end{proof}
		\item $\{z: e^z = -1\}$
		\begin{proof}
			Put $z = x + iy$, where $x, y \in \R$. We have $e^z = e^{x + iy} = e^xe^{iy} = -1$. Then $e^x = 1$ and $e^{iy} = \cos y + i\sin y = -1$. Hence $x = 0$ and $y = \pi + 2\pi k$ for some $k \in \Z$, which yields
			\[
				\{z: e^z = -1\} = \left\{(2k + 1)i\pi \mid k \in \Z\right\}.
			\]
		\end{proof}
		\item $\{z: e^z = -i\}$
		\begin{proof}
			Put $z = x + iy$, where $x, y \in \R$. We have $e^z = e^{x + iy} = e^xe^{iy} = -i$. Then $e^x = 1$ and $e^{iy} = \cos y + i\sin y = -i$. Hence $x = 0$ and $y = -\frac{\pi}{2} + 2\pi k$ for some $k \in \Z$, which yields
			\[
				\{z: e^z = -i\} = \left\{\frac{(4k - 1)i\pi}{2} \mid k \in \Z\right\}.
			\]
		\end{proof}
		\item $\{z: \cos z = 0\}$
		\begin{proof}
			Put $z = x + iy$, where $x, y \in \R$. Since $\cos z = \frac{1}{2}(e^{iz} + e^{-iz}) = 0$, we have $e^{2iz} = e^{-2y}e^{2ix} = -1$. Hence, $y = 0$ and $x = \frac{\pi}{2} + \pi k$. Thus,
			\[
				\{z: \cos z = 0\} = \left\{\frac{(2k + 1)\pi}{2} \mid k \in \Z\right\}.
			\]
		\end{proof}
		\item $\{z: \sin z = 0\}$
		\begin{proof}
			Put $z = x + iy$, with $x, y \in \R$. Since $\sin z = \frac{1}{2i}(e^{iz} - e^{-iz}) = 0$, we have $e^{2iz} = e^{-2y}e^{2ix} = 1$. Hence, $y = 0$ and $x = \pi k$ for some $k \in \Z$. Thus,
			\[
				\{z: \cos z = 0\} = \left\{k\pi \mid k \in \Z\right\}.
			\]
		\end{proof}
	\end{enumerate}
\end{homeworkProblem}

\newpage

\begin{homeworkProblem}
	Prove the following generalization of Proposition 2.20. Let $G$ and $\Omega$ be open in $\mathbb{C}$ and suppose $f$ and $h$ are functions defined on $G$, $g: \Omega \to \mathbb{C}$ and suppose that $f(G) \subseteq \Omega$. Suppose that $g$ and $h$ are analytic, $g'(\omega) \neq 0$ for any $\omega$, that $f$ is continuous, $h$ is one-to-one, and that they satisfy $h(z) = g(f(z))$ for $z$ in $G$. Show that $f$ is analytic. Give a formula for $f'(z)$.

	\begin{proof}
		Let $z \in \C$. Since $h$ is injective, $g(f(z + k)) = h(z + k) \neq h(z) = g(f(z))$ for all $k \neq 0$, and so $f(z + k) \neq f(z)$ for all $k \neq 0$. Since $h$ is analytic, 
		\[
			h'(z) = \lim_{k \to 0} \frac{h(z + k) - h(z)}{k} = \lim_{k \to 0} \frac{g(f(z + k)) - g(f(z))}{k} = \lim_{k \to 0} \frac{g(f(z + k)) - g(f(z))}{f(z + k) - f(z)} \cdot \frac{f(z + k) - f(z)}{k}.
		\] 
		But then $f$ is continuous, so $f(z + k) \to f(z)$ as $k \to 0$, and thus 
		\[
			\lim_{k \to 0} \frac{g(f(z + k)) - g(f(z))}{f(z + k) - f(z)} = g'(f(z)).
		\]
		Hence,
		\[
			h'(z) = g'(f(z)) \lim_{k \to 0} \frac{f(z + k) - f(z)}{k},
		\]
		so $f'(z) = \lim_{k \to 0} \frac{f(z + k) - f(z)}{k} = \frac{h'(z)}{g'(f(z))}$ exists.
	\end{proof}
\end{homeworkProblem}
\end{document}