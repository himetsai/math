\documentclass[a4paper]{article}

\usepackage[english]{babel}
\usepackage[utf8]{inputenc}
\usepackage{amsmath}
\usepackage{amsthm}
\usepackage{amsfonts}
\usepackage{graphicx}
\usepackage{xcolor}
\usepackage{enumerate}
\usepackage[colorinlistoftodos]{todonotes}

\setlength{\topmargin}{-0.5in}
\setlength{\textheight}{9.5in}
\setlength{\headsep}{10pt}

\setlength{\textwidth}{6.5in}
\setlength{\oddsidemargin}{0pt}
\setlength{\evensidemargin}{0pt}
\setlength{\parskip}{5pt}

\addto{\captionsenglish}{\renewcommand{\abstractname}{}}

\graphicspath{ {./images} }

\newtheorem{theorem}{Theorem}[section]
\newtheorem{conjecture}[theorem]{Conjecture}
\newtheorem{corollary}{Corollary}[theorem]
\newtheorem{lemma}[theorem]{Lemma}
\newtheorem{example}[theorem]{Example}

\title{\textsc{MATH 264A Lecture Notes}}

\author{Instructor: Lutz Warnke \\ \small{Compiled by Ray Tsai}}

\date{}

\begin{document}

\maketitle

\begin{abstract}
  This note is for the graduate combinatorics course MATH 264A at UC San Diego, taught by Professor
  Lutz Warnke in 2024 Fall. The proofs below are merely my attempts of recreating the contents in
  lectures, which might not be accurate representations of what was actually taught. 
\end{abstract}

\section{Some Basic Tools}

This lecture introduces some simple but powerful tools.

\subsection*{Inductive Approaches}

This method is done by changing the size of the problem, e.g. adding an vertex or an edge in a
graph.

\begin{example}
  Every $n$-vertex graph with maximum degree $\Delta$ has $\geq \beta^n$ valid vertex colorings with
  $\leq \lceil \Delta + \beta \rceil$ colors.
\end{example}

\begin{proof}
  Color the vertices $v_1, v_2, \ldots, v_n$ sequentially. Since $v_i$ has $\leq \Delta$ neighbors
  already colored, there are $\geq \lceil \Delta + \beta \rceil - \Delta \geq \beta$ choices to
  color $v_i$. Define $N_i$ as $\#$ valid colorings of $v_1, \ldots, v_i$. Then, the
  \textit{Telescoping Product} now yields
  \[
    N_n = \frac{N_n}{N_{n - 1}} \cdot \frac{N_{n - 2}}{N_{n - 1}} \cdots \frac{N_{1}}{N_{0}} \cdot N_0 \geq \beta^n,
  \]
  as $N_0 = 1$.
\end{proof}

Despite being an extremely basic technique, induction can prove several advanced theorems if used
artfully. The following are some exciting theorems which can be proven by induction:

\begin{enumerate}
  \item Strengthen Lovász Local Lemma (LLL)
  \item Chromatic number of triangle-free graph with max-degree $\Delta$ is $\leq (1 +
  o(1))\frac{\Delta}{\log \Delta}$ as $\Delta \to \infty$.
  \item Almost all triangle-free graphs are bipartite.
\end{enumerate}

\subsection*{Double Counting/Switching}

Also known as the Pertubation method, e.g. change of location of edges.

\begin{example}
  Find the $\# \Pi \in S_n$ without fix-points, i.e. $\Pi(i) \neq i$ for all $i$.
\end{example}

\begin{proof}
  We prove this by a basic approach which consists of several steps:

  \textbf{Step 1: Define the ``Switching Operation.''} Let $S_{n, k}$ be the set of permutations
  with $k$ fix-points. Define the switching operation to transform $\pi \in S$ to $\pi' \in S_{n,
  1}$.

  \textbf{Step 2: Consider the auxiliary bipartite graph.} Let $S_{n, 0}, S_{n, 1}$ be parts of
  the bipartite graph. Connect $\pi \in S_{n, 0}$ with $\pi' \in S_{n, 1}$ if $\pi'$ results from
  $\pi$ through the switching operation.

  \textbf{Step 3: Double count the degrees.}
  \[
    \sum_{\pi \in S_{n, 0}} \deg \pi = \sum_{\pi' \in S_{n, 1}} \deg \pi'
  \]

  \textbf{Step 4: Degree essentially transfers to ratio.} Suppose $\deg \pi \approx a$ and $\deg
  \pi' \approx b$, for all $\pi \in S_{n, 0}$ and $\pi' \in S_{n, 1}$. Then,
  \[
    \frac{|S_{n, 0}|}{|S_{n, 1}|} \approx \frac{b}{a}.
  \]
\end{proof}

This method can be applied to count $d$-regular graphs with certain properties, i.e. random model
without independence.

\subsection*{Asymptotic Methods}

Rather than finding the close form of a discrete function, sometimes it is significantly easier to
approximate the function in asymptotic settings.

\subsubsection*{Bootstrapping}

Suppose we have an equation $w(z)e^{w(z)} = z$ and we try to extract $w(z)$. By bootstrapping, $w(z)
= \ln z - \ln \ln z + o(1)$.

\subsubsection*{Integral-Approximation}

As the title suggests, this method estimates a summation $\sum_{k \in I} f(k)$ with its integral
counterpart $\int_I f(x) \, dx$. For example, the the summation derived from the Fibonacci Tiliing
Problem can be estimated by the Laplace-Method, i.e.
\[
  \sum_{0 \leq k \leq \frac{n}{2}} \binom{n - k}{k} \sim \frac{1}{\sqrt{5}}\left(\frac{1 + \sqrt{5}}{2}\right)^{n + 1} \quad n \to \infty.
\]

\newpage

\section{Inductive Counting}

This lecture introduces two exmaples the on inductive counting approach, which often improves the
Lovász Local Lemma. The basic approach of this method is to first generalize the problem then use
structural induction on, say, the number of vertices or edges. It often involves extending from
smaller cases then counting the number of ``bad" extensions based on some observed patterns of them.

\subsection*{Non-repetitive Words}

A word $w$ is defined as a sequence of symbols from alphabet $\mathcal{A}$. A word $w$ is
non-repetitive if no word appears twice in $w$ consecutively. For example,
$\underline{ab}\underline{ab}$ and $b\underline{abc}\underline{abc}$ are repetitive words, while
$ab$ and $aba$ are non-repetitive words.

The following Theorem is proven by Thue in 1906:
\begin{theorem}
  Let $\mathcal{A}$ be a $3$-symbol alphabet. Then for all $n \geq 1$, there exists a non-repetitive
  word $a_1\ldots a_n$, with $a_i \in \mathcal{A}$ for all $i$.
\end{theorem}

A conjecture on non-repetitive words is that

\begin{conjecture}
  Let $L_1, \ldots, L_n$ be subsets of an alphabet $\mathcal{A}$, with $|L_i| = 3$ for all $i$. Then
  for $n \geq 1$, there exists a non-repetitive word $a_1\ldots a_n$, with $a_i \in L_i$ for all
  $i$.
\end{conjecture}

This conjecture remains open till this day (2024), but we can prove a slightly weaker version of it
using induction:

\begin{theorem}
  Let $L_1, \ldots, L_n$ be subsets of an alphabet $\mathcal{A}$, with $|L_i| = 4$ for all $i$. Then
  for $n \geq 1$, there exists are at least $2^n$ non-repetitive words $a_1\ldots a_n$, with $a_i
  \in L_i$ for all $i$.
\end{theorem}

\begin{proof}
  We first generalize the problem. Define $N_k$ as the number of non-repetitive words of $a_1\ldots
  a_k$, with $a_i \in L_i$ for all $i \in [k]$. It suffices to show $N_k \geq 2^k$. Put $\beta = 2$.
  We proceed by induction on $k \geq 1$ to show that $\frac{N_k}{N_{k - 1}} \geq \beta$.
  
  \textbf{Base Case:} We already know $N_0 = 1$. Since there are $|L_1| = 4$ choices for the first
  symbol, $\frac{N_1}{N_0} = N_1 = 4 \geq \beta$, so the base case is done.

  \textbf{Inductive Step:} Suppose $k \geq 2$. Here we use symbols from $L_k$ to (try to) extend
  each $N_{k - 1}$ non-repetitive words of length $k - 1$. Hence we may write
  \[
    \# \text{all extensions} = N_{k - 1} \cdot |L_k| = N_k + B,
  \]
  where $B$ is the number of repetitive (bad) extensions . By construction, repetition must happen
  at the end of the extension, say the last $j$ symbols. But then those $j$ symbols uniquely
  determines the $j$ symbols preceeding them. Summing over all possible $j$'s, we get $B \leq
  \sum_{j \in [k/2]} N_{k - j}$. By induction, $N_{k - j} \leq N_{k - 1} \left(\frac{1}{2}\right)^{j
  - 1}$, and thus
  \[
    B \leq N_{k - 1}\sum_{j \in [k/2]} \left(\frac{1}{\beta}\right)^{j - 1} \leq N_{k - 1} \cdot \frac{\beta}{\beta - 1}.
  \]
  It immediately follows that when $\beta = 2$
  \[
    N_k = N_{k - 1} \cdot |L_k| - B \geq N_{k - 1} \cdot |L_i| - \frac{\beta}{\beta - 1} \cdot N_{k - 1} \geq \beta N_{k - 1},
  \]
  and this completes the induction.
\end{proof}

\newpage

\subsection*{Lower Bound on Size of Intependent Sets}

\begin{theorem}
  Let $V_1, \ldots, V_s \subseteq V$ be disjoint sets of size $|V_{i}| \geq 4\Delta$. Then there
  exist $(2\Delta)^s$ intependent sets $I \subseteq V$ which contains one vertex from each $V_i$.
\end{theorem}

\begin{proof}
  Again, we first generalize the problem for $T \subseteq [s]$. Define $N_T$ as the number of
  independent sets $I \subseteq V_T = \bigcup_{j \in T} V_j$ that contains one vertex from each
  $V_j$. It suffices to show that $N_T \geq (2\Delta)^{|T|}$. Put $\beta = 2\Delta$. We proceed by
  induction on $|T|$ to show that for all $x \in T$, $\frac{N_T}{N_{T - \{x\}}} \geq \beta$. 

  \textbf{Base Case:} Trivial, as $\frac{N_1}{N_1} = N_1 = |V_i| \geq 4\Delta \geq \beta$.

  \textbf{Inductive Step:} Suppose $|T| \geq 2$. Here we try to use vertex from $V_x$ to extend the
  $N_{T - \{x\}}$ many valid independent sets for $V_{T - \{x\}}$. Hence we write
  \[
    N_{T - \{x\}} \cdot |V_x| = N_T + B,
  \]
  where $B$ is again the number of invalid (bad) extensions. In particular, we need to count the
  number of possible ways that our extension contains an edge from $V_x$ to an independent set in
  $V_{T - \{x\}}$. Hence,
  \[
    B \leq |V_x| \cdot \Delta \cdot \max_{z \in T - \{x, z\}} N_{T - \{x, z\}} \leq \frac{4\Delta^2}{\beta} \cdot N_{T - \{x\}},
  \]
  by induction. It now follows that when $\beta = 2\Delta$,
  \[
    N_T = N_{T - \{x\}} \cdot |V_x| - B \geq 4\Delta \left(1 - \frac{\Delta}{\beta}\right)N_{T - \{x\}} \geq \beta N_{T - \{x\}},
  \]
  and this completes the induction.
\end{proof}

\end{document}