\documentclass[a4paper]{article}

\usepackage[english]{babel}
\usepackage[utf8]{inputenc}
\usepackage{amsmath}
\usepackage{amsthm}
\usepackage{amsfonts}
\usepackage{graphicx}
\usepackage{xcolor}
\usepackage{enumerate}
\usepackage[colorinlistoftodos]{todonotes}
\usepackage[margin=1in]{geometry}
\usepackage{array}

\setlength{\topmargin}{-0.5in}
\setlength{\textheight}{9.5in}
\setlength{\headsep}{10pt}

\setlength{\textwidth}{6.5in}
\setlength{\oddsidemargin}{0pt}
\setlength{\evensidemargin}{0pt}
\setlength{\parskip}{5pt}
\setlength\fboxsep{6pt}

\addto{\captionsenglish}{\renewcommand{\abstractname}{}}

\graphicspath{ {./images} }

\newtheorem{theorem}{Theorem}[section]
\newtheorem{conjecture}[theorem]{Conjecture}
\newtheorem{corollary}{Corollary}[theorem]
\newtheorem{lemma}[theorem]{Lemma}
\newtheorem{example}[theorem]{Example}

\title{\textsc{MATH 264A Lecture Notes}}

\author{Instructor: Lutz Warnke}

\date{}

\begin{document}

\begin{center}
\fbox{%
  \begin{minipage}{0.95\linewidth}
    \textbf{MATH 264A: Combinatorics}
    \hfill
    \textbf{Fall 2024}

    \begin{center}
      {\Large Lecture 1: Some Basic Tools}\\[.5ex]
      
      \vspace{1ex}

      {\large October 1, 2024}

      \vspace{1ex}

      \noindent
      \emph{Lecturer: Lutz Warnke}
      \hfill
      \emph{Scribes: Ray Tsai}
    \end{center}
  \end{minipage}
}
\end{center}
\subsection*{Inductive Approaches}

This method is done by changing the size of the problem, e.g. adding an vertex or an edge in a graph.

\begin{example}
  Every $n$-vertex graph with maximum degree $\Delta$ has $\geq \beta^n$ valid vertex colorings with $\leq \lceil \Delta + \beta \rceil$ colors.
\end{example}

\begin{proof}
  Color the vertices $v_1, v_2, \ldots, v_n$ sequentially. Since $v_i$ has $\leq \Delta$ neighbors already colored, there are $\geq \lceil \Delta + \beta \rceil - \Delta \geq \beta$ choices to color $v_i$. Define $N_i$ as $\#$ valid colorings of $v_1, \ldots, v_i$. Then, the \textit{Telescoping Product} now yields
  \[
    N_n = \frac{N_n}{N_{n - 1}} \cdot \frac{N_{n - 2}}{N_{n - 1}} \cdots \frac{N_{1}}{N_{0}} \cdot N_0 \geq \beta^n,
  \]
  as $N_0 = 1$.
\end{proof}

Despite being an extremely basic technique, induction can prove several advanced theorems if used artfully. The following are some exciting theorems which can be proven by induction:

\begin{enumerate}
  \item Strengthen Lovász Local Lemma (LLL)
  \item Chromatic number of triangle-free graph with max-degree $\Delta$ is $\leq (1 + o(1))\frac{\Delta}{\log \Delta}$ as $\Delta \to \infty$.
  \item Almost all triangle-free graphs are bipartite.
\end{enumerate}

\subsection*{Double Counting/Switching}

Also known as the Pertubation method, e.g. change of location of edges.

\begin{example}
  Find the $\# \Pi \in S_n$ without fix-points, i.e. $\Pi(i) \neq i$ for all $i$.
\end{example}

\begin{proof}
  We prove this by a basic approach which consists of several steps:

  \textbf{Step 1: Define the ``Switching Operation.''} Let $S_{n, k}$ be the set of permutations with $k$ fix-points. Define the switching operation to transform $\pi \in S$ to $\pi' \in S_{n, 1}$.

  \textbf{Step 2: Consider the auxiliary bipartite graph.} Let $S_{n, 0}, S_{n, 1}$ be parts of the bipartite graph. Connect $\pi \in S_{n, 0}$ with $\pi' \in S_{n, 1}$ if $\pi'$ results from $\pi$ through the switching operation.

  \textbf{Step 3: Double count the degrees.}
  \[
    \sum_{\pi \in S_{n, 0}} \deg \pi = \sum_{\pi' \in S_{n, 1}} \deg \pi'
  \]

  \textbf{Step 4: Degree essentially transfers to ratio.} Suppose $\deg \pi \approx a$ and $\deg \pi' \approx b$, for all $\pi \in S_{n, 0}$ and $\pi' \in S_{n, 1}$. Then,
  \[
    \frac{|S_{n, 0}|}{|S_{n, 1}|} \approx \frac{b}{a}.
  \]
\end{proof}

This method can be applied to count $d$-regular graphs with certain properties, i.e. random model without independence.

\subsection*{Asymptotic Methods}

Rather than finding the close form of a discrete function, sometimes it is significantly easier to approximate the function in asymptotic settings.

\subsubsection*{Bootstrapping}

Suppose we have an equation $w(z)e^{w(z)} = z$ and we try to extract $w(z)$. By bootstrapping, $w(z) = \ln z - \ln \ln z + o(1)$.

\subsubsection*{Integral-Approximation}

As the title suggests, this method estimates a summation $\sum_{k \in I} f(k)$ with its integral counterpart $\int_I f(x) \, dx$. For example, the the summation derived from the Fibonacci Tiliing Problem can be estimated by the Laplace-Method, i.e.
\[
  \sum_{0 \leq k \leq \frac{n}{2}} \binom{n - k}{k} \sim \frac{1}{\sqrt{5}}\left(\frac{1 + \sqrt{5}}{2}\right)^{n + 1} \quad n \to \infty.
\]

\end{document}