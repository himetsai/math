\documentclass{article}

\usepackage{fancyhdr}
\usepackage{extramarks}
\usepackage{amsmath}
\usepackage{amsthm}
\usepackage{amsfonts}
\usepackage{tikz}
\usepackage[plain]{algorithm}
\usepackage{algpseudocode}
\usepackage{enumerate}
\usepackage{amssymb}
\usepackage{dsfont}

\usetikzlibrary{automata,positioning}

%
% Basic Document Settings
%

\topmargin=-0.45in
\evensidemargin=0in
\oddsidemargin=0in
\textwidth=6.5in
\textheight=9.0in
\headsep=0.25in

\linespread{1.1}

\pagestyle{fancy}
\lhead{\hmwkAuthorName}
\chead{\hmwkClass:\ \hmwkTitle}
\rhead{\firstxmark}
\lfoot{\lastxmark}
\cfoot{\thepage}

\renewcommand\headrulewidth{0.4pt}
\renewcommand\footrulewidth{0.4pt}

\setlength\parindent{0pt}
\setlength{\parskip}{5pt}

%
% Create Problem Sections
%

\newcommand{\enterProblemHeader}[1]{
    \nobreak\extramarks{}{Problem \arabic{#1} continued on next page\ldots}\nobreak{}
    \nobreak\extramarks{Problem \arabic{#1} (continued)}{Problem \arabic{#1} continued on next page\ldots}\nobreak{}
}

\newcommand{\exitProblemHeader}[1]{
    \nobreak\extramarks{Problem \arabic{#1} (continued)}{Problem \arabic{#1} continued on next page\ldots}\nobreak{}
    \stepcounter{#1}
    \nobreak\extramarks{Problem \arabic{#1}}{}\nobreak{}
}

\setcounter{secnumdepth}{0}
\newcounter{partCounter}
\newcounter{homeworkProblemCounter}
\setcounter{homeworkProblemCounter}{1}
\nobreak\extramarks{Problem \arabic{homeworkProblemCounter}}{}\nobreak{}

%
% Homework Problem Environment
%
% This environment takes an optional argument. When given, it will adjust the
% problem counter. This is useful for when the problems given for your
% assignment aren't sequential. See the last 3 problems of this template for an
% example.
%
\newenvironment{homeworkProblem}[1][-1]{
    \ifnum#1>0
        \setcounter{homeworkProblemCounter}{#1}
    \fi
    \section{Problem \arabic{homeworkProblemCounter}}
    \setcounter{partCounter}{1}
    \enterProblemHeader{homeworkProblemCounter}
}{
    \exitProblemHeader{homeworkProblemCounter}
}

%
% Homework Details
%   - Title
%   - Due date
%   - Class
%   - Section/Time
%   - Instructor
%   - Author
%

\newcommand{\hmwkTitle}{Homework}
\newcommand{\hmwkDueDate}{Nov 2, 2024}
\newcommand{\hmwkClass}{MATH 264B}
\newcommand{\hmwkClassInstructor}{Professor Rhodes}
\newcommand{\hmwkAuthorName}{\textbf{Ray Tsai}}
\newcommand{\hmwkPID}{A16848188}

%
% Title Page
%

\title{
    \vspace{2in}
    \textmd{\textbf{\hmwkClass:\ \hmwkTitle}}\\
    \vspace{0.1in}\large{\textit{\hmwkClassInstructor}} \\
    \vspace{3in}
}

\author{
  \hmwkAuthorName \\
  \vspace{0.1in}\small\hmwkPID
}
\date{}

\renewcommand{\part}[1]{\textbf{\large Part \Alph{partCounter}}\stepcounter{partCounter}\\}

%
% Various Helper Commands
%

% Useful for algorithms
\newcommand{\alg}[1]{\textsc{\bfseries \footnotesize #1}}

% For derivatives
\newcommand{\deriv}[1]{\frac{\mathrm{d}}{\mathrm{d}x} (#1)}

% For partial derivatives
\newcommand{\pderiv}[2]{\frac{\partial}{\partial #1} (#2)}

% Integral dx
\newcommand{\dx}{\mathrm{d}x}

% Probability commands: Expectation, Variance, Covariance, Bias
\newcommand{\Var}{\mathrm{Var}}
\newcommand{\Cov}{\mathrm{Cov}}
\newcommand{\Bias}{\mathrm{Bias}}
\newcommand*{\Z}{\mathbb{Z}}
\newcommand*{\Q}{\mathbb{Q}}
\newcommand*{\R}{\mathbb{R}}
\newcommand*{\C}{\mathbb{C}}
\newcommand*{\N}{\mathbb{N}}
\newcommand*{\prob}{\mathds{P}}
\newcommand*{\E}{\mathds{E}}
\newcommand*{\sym}{\mathfrak{S}}

\begin{document}

\maketitle

\pagebreak

\begin{homeworkProblem}
  Let $n, m \in \mathbb{Z}_{\geq 0}$. Give a \textit{combinatorial} proof that
  \[
  \sum_{i=0}^n \binom{m + i}{i} = \binom{m + n + 1}{n}.
  \]
  That is, interpret both sides as the cardinality of a set, and find a bijection between these
  sets.

  \begin{proof}
    It suffices to show that 
    \[
    \sum_{i=0}^n \binom{m + i}{m} = \binom{m + n + 1}{m + 1}.
    \]
    Let $C_i$ be the set of all $m$-element subsets of $[m + i]$, and let $S$ be the set of all $(m
    + 1)$-element subsets of $[m + n + 1]$. Consider the map $f: \bigsqcup_{i = 1}^n C_i \to S$ by
    sending $A \in C_i$ to $A \cup \{m + i + 1\} \in S$. This mapping is a bijection as we may
    recover $A$ by removing the largest element of $f(A)$. Thus, $|\bigsqcup_{i = 1}^n C_i| = |S|$,
    and the result now follows.
  \end{proof}
\end{homeworkProblem}

\newpage

\begin{homeworkProblem}
  Let $\text{des} : \sym_n \to \mathbb{Z}_{\geq 0}$ be the descent statistic
  \[
  \text{des}(w) := \#\{1 \leq i \leq n-1 : w(i) > w(i+1)\}
  \]
  and consider the \textit{Eulerian polynomial}
  \[
  A_n(t) := \sum_{w \in \sym_n} t^{\text{des}(w)}.
  \]

  Prove that $A_n(2) = [A_n(t)]_{t=2}$ is the number of ordered set partitions of $[n]$.

  \begin{proof}
    We say that a ordered partition is in canonical form if the elements of each block are in
    descending order. Let $P_n$ be the set of all ordered set partitions of $[n]$. Define the
    operation $\phi: P_n \to \sym_n$ by erasing the brackets of an ordered partition in
    canonical form and interpreting the resulting string as a permutation. It is clear that $\phi$
    is well-defined. Now consider the reverse operation $\psi: \sym_n \to 2^{P_n}$ by sending
    $w \in \sym_n$ to $\{p \in P_n : \phi(p) = w\}$, the set of all ordered partitions whose
    canonical form resembles $w$ after erasing the brackets. Note that 
    \[
      |P_n| = \sum_{p \in P_n} |\phi(p)| = \sum_{w \in \sym_n} |\psi(w)|,
    \]
    and so it suffices to show that $|\psi(w)| = 2^{\text{des}(w)}$. To see this, we start from the
    ordered singleton partition $p_0 \in \psi(w)$. Reading $p_0$ from left to right, we may choose
    to combine a block with its preceding block whenever a descend occurs, and the resulting
    partition will still be in $\psi(w)$. This gives us $2^{\text{des}(w)}$ ways to partition $w$
    into blocks.
  \end{proof}  
\end{homeworkProblem}

\newpage

\begin{homeworkProblem}
  How many (strong) compositions of $n$ have an even number of even parts?

  \begin{proof}
    Let $E_n$ be the set of all compositions of $n$ with even number of even parts, and let $O_n$ be
    the set of all compositions of $n$ with odd number of even parts. We show $|E_n| = 2^{n - 2}$
    for $n \geq 2$ by proving that $|E_n| = |O_n|$. Consider the operation $\phi: E_n \to O_n$ by
    sending the composition $(\alpha_1, \ldots, \alpha_k)$ to $(\alpha_1, \ldots, \alpha_k - 1, 1)$
    if $\alpha_k > 1$ and send $(\alpha_1, \ldots, \alpha_k)$ to $(\alpha_1, \ldots, \alpha_{k - 1}
    + 1)$ if $\alpha_k = 1$. Notice that $\phi(\phi(\alpha_1, \ldots, \alpha_k)) = (\alpha_1, \ldots,
    \alpha_k)$, so $\phi$ is an inversion. 
  \end{proof}
\end{homeworkProblem}

\end{document}