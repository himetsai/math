\documentclass{article}

\usepackage{fancyhdr}
\usepackage{extramarks}
\usepackage{amsmath}
\usepackage{amsthm}
\usepackage{amsfonts}
\usepackage{tikz}
\usepackage[plain]{algorithm}
\usepackage{algpseudocode}
\usepackage{enumerate}
\usepackage{amssymb}
\usepackage{dsfont}

\usetikzlibrary{automata,positioning}

%
% Basic Document Settings
%

\topmargin=-0.45in
\evensidemargin=0in
\oddsidemargin=0in
\textwidth=6.5in
\textheight=9.0in
\headsep=0.25in

\linespread{1.1}

\pagestyle{fancy}
\lhead{\hmwkAuthorName}
\chead{\hmwkClass:\ \hmwkTitle}
\rhead{\firstxmark}
\lfoot{\lastxmark}
\cfoot{\thepage}

\renewcommand\headrulewidth{0.4pt}
\renewcommand\footrulewidth{0.4pt}

\setlength\parindent{0pt}
\setlength{\parskip}{5pt}

%
% Create Problem Sections
%

\newcommand{\enterProblemHeader}[1]{
    \nobreak\extramarks{}{Problem \arabic{#1} continued on next page\ldots}\nobreak{}
    \nobreak\extramarks{Problem \arabic{#1} (continued)}{Problem \arabic{#1} continued on next page\ldots}\nobreak{}
}

\newcommand{\exitProblemHeader}[1]{
    \nobreak\extramarks{Problem \arabic{#1} (continued)}{Problem \arabic{#1} continued on next page\ldots}\nobreak{}
    \stepcounter{#1}
    \nobreak\extramarks{Problem \arabic{#1}}{}\nobreak{}
}

\setcounter{secnumdepth}{0}
\newcounter{partCounter}
\newcounter{homeworkProblemCounter}
\setcounter{homeworkProblemCounter}{1}
\nobreak\extramarks{Problem \arabic{homeworkProblemCounter}}{}\nobreak{}

%
% Homework Problem Environment
%
% This environment takes an optional argument. When given, it will adjust the
% problem counter. This is useful for when the problems given for your
% assignment aren't sequential. See the last 3 problems of this template for an
% example.
%
\newenvironment{homeworkProblem}[1][-1]{
    \ifnum#1>0
        \setcounter{homeworkProblemCounter}{#1}
    \fi
    \section{Problem \arabic{homeworkProblemCounter}}
    \setcounter{partCounter}{1}
    \enterProblemHeader{homeworkProblemCounter}
}{
    \exitProblemHeader{homeworkProblemCounter}
}

%
% Homework Details
%   - Title
%   - Due date
%   - Class
%   - Section/Time
%   - Instructor
%   - Author
%

\newcommand{\hmwkTitle}{Homework}
\newcommand{\hmwkDueDate}{Mar 10, 2025}
\newcommand{\hmwkClass}{MATH 264B}
\newcommand{\hmwkClassInstructor}{Professor Rhoades}
\newcommand{\hmwkAuthorName}{\textbf{Ray Tsai}}
\newcommand{\hmwkPID}{A16848188}

%
% Title Page
%

\title{
    \vspace{2in}
    \textmd{\textbf{\hmwkClass:\ \hmwkTitle}}\\
    \vspace{0.1in}\large{\textit{\hmwkClassInstructor}} \\
    \vspace{3in}
}

\author{
  \hmwkAuthorName \\
  \vspace{0.1in}\small\hmwkPID
}
\date{}

\renewcommand{\part}[1]{\textbf{\large Part \Alph{partCounter}}\stepcounter{partCounter}\\}

%
% Various Helper Commands
%

% Useful for algorithms
\newcommand{\alg}[1]{\textsc{\bfseries \footnotesize #1}}

% For derivatives
\newcommand{\deriv}[1]{\frac{\mathrm{d}}{\mathrm{d}x} (#1)}

% For partial derivatives
\newcommand{\pderiv}[2]{\frac{\partial}{\partial #1} (#2)}

% Integral dx
\newcommand{\dx}{\mathrm{d}x}

% Probability commands: Expectation, Variance, Covariance, Bias
\newcommand{\Var}{\mathrm{Var}}
\newcommand{\Cov}{\mathrm{Cov}}
\newcommand{\Bias}{\mathrm{Bias}}
\newcommand*{\Z}{\mathbb{Z}}
\newcommand*{\Q}{\mathbb{Q}}
\newcommand*{\R}{\mathbb{R}}
\newcommand*{\C}{\mathbb{C}}
\newcommand*{\N}{\mathbb{N}}
\newcommand*{\F}{\mathbb{F}}
\newcommand*{\prob}{\mathds{P}}
\newcommand*{\E}{\mathds{E}}
\newcommand*{\sym}{\mathfrak{S}}

\begin{document}

\maketitle

\pagebreak

\begin{homeworkProblem}
  Let $n, m \in \mathbb{Z}_{\geq 0}$. Give a \textit{combinatorial} proof that
  \[
  \sum_{i=0}^n \binom{m + i}{i} = \binom{m + n + 1}{n}.
  \]
  That is, interpret both sides as the cardinality of a set, and find a bijection between these
  sets.

  \begin{proof}
    It suffices to show that 
    \[
    \sum_{i=0}^n \binom{m + i}{m} = \binom{m + n + 1}{m + 1}.
    \]
    Let $C_i$ be the set of all $m$-element subsets of $[m + i]$, and let $S$ be the set of all $(m
    + 1)$-element subsets of $[m + n + 1]$. Consider the map $f: \bigsqcup_{i = 1}^n C_i \to S$ by
    sending $A \in C_i$ to $A \cup \{m + i + 1\} \in S$. This mapping is a bijection as we may
    recover $A$ by removing the largest element of $f(A)$. Thus, $|\bigsqcup_{i = 1}^n C_i| = |S|$,
    and the result now follows.
  \end{proof}
\end{homeworkProblem}

\newpage

\begin{homeworkProblem}
  Let $\text{des} : \sym_n \to \mathbb{Z}_{\geq 0}$ be the descent statistic
  \[
  \text{des}(w) := \#\{1 \leq i \leq n-1 : w(i) > w(i+1)\}
  \]
  and consider the \textit{Eulerian polynomial}
  \[
  A_n(t) := \sum_{w \in \sym_n} t^{\text{des}(w)}.
  \]

  Prove that $A_n(2) = [A_n(t)]_{t=2}$ is the number of ordered set partitions of $[n]$.

  \begin{proof}
    We say that a ordered partition is in canonical form if the elements of each block are in
    descending order. Let $P_n$ be the set of all ordered set partitions of $[n]$. Define the
    operation $\phi: P_n \to \sym_n$ by erasing the brackets of an ordered partition in
    canonical form and interpreting the resulting string as a permutation. It is clear that $\phi$
    is well-defined. Now consider the reverse operation $\psi: \sym_n \to 2^{P_n}$ by sending
    $w \in \sym_n$ to $\{p \in P_n : \phi(p) = w\}$, the set of all ordered partitions whose
    canonical form resembles $w$ after erasing the brackets. Note that 
    \[
      |P_n| = \sum_{p \in P_n} |\phi(p)| = \sum_{w \in \sym_n} |\psi(w)|,
    \]
    and so it suffices to show that $|\psi(w)| = 2^{\text{des}(w)}$. To see this, we start from the
    ordered singleton partition $p_0 \in \psi(w)$. Reading $p_0$ from left to right, we may choose
    to combine a block with its preceding block whenever a descend occurs, and the resulting
    partition will still be in $\psi(w)$. This gives us $2^{\text{des}(w)}$ ways to partition $w$
    into blocks.
  \end{proof}  
\end{homeworkProblem}

\newpage

\begin{homeworkProblem}
  How many (strong) compositions of $n$ have an even number of even parts?

  \begin{proof}
    Let $E_n$ be the set of all compositions of $n$ with even number of even parts, and let $O_n$ be
    the set of all compositions of $n$ with odd number of even parts. We show $|E_n| = 2^{n - 2}$
    for $n \geq 2$ by proving that $|E_n| = |O_n|$. Consider the operation $\phi: E_n \to O_n$ by
    sending the composition $(\alpha_1, \ldots, \alpha_k)$ to $(\alpha_1, \ldots, \alpha_k - 1, 1)$
    if $\alpha_k > 1$ and send $(\alpha_1, \ldots, \alpha_k)$ to $(\alpha_1, \ldots, \alpha_{k - 1}
    + 1)$ if $\alpha_k = 1$. Notice that $\phi(\phi(\alpha_1, \ldots, \alpha_k)) = (\alpha_1, \ldots,
    \alpha_k)$, so $\phi$ is a bijection. But then ther eare $2^{n - 1}$ compositions of $n$, so $E_n = 2^{n - 2}$.
  \end{proof}
\end{homeworkProblem}

\newpage

\begin{homeworkProblem}
  For $1 \leq i \leq n - 1$, let $s_i$ be the adjacent transposition $(i, i+1) \in \sym_n$. It is known that the set $S = \{ s_1, \dots, s_{n-1} \}$ generates the group $\sym_n$. For $w \in \sym_n$, the \textit{Coxeter length} $\ell_{\mathcal{S}}(w)$ is the minimum number $r$ so that $w = s_{i_1} \cdots s_{i_r}$ for some $1 \leq i_1, \dots, i_r \leq n$. Prove that $\ell_{\mathcal{S}}(w) = \operatorname{inv}(w)$ for all $w \in \sym_n$.

  \begin{proof}
    Let $w \in \sym_n$. Consider the bubble sorting algorithm that rearranges the identity permutation by swapping adjacent numbers. Let $w^{(i)}$ be the result after the $i$th iteration of the algorithm. Note $w^{(0)}$ is the identity permutation. Hence, in the $i$th iteration, we shift the $i$th number of $w$ to the $i$th position. For all $i$, notice that $w^{(i)}_{j} = w_j$ for $1 \leq j \leq i$, and $w^{(i)}_j < w^{(i)}_k$ for $i < j < k \leq n$. Thus if $w_{i} = w^{(i - 1)}_j$, then we know $j \geq i$ and the $i$th iteration of the algorithm would take $j - i$ adjacent transpositions to move $w_i$ to the $i$th position. But then for $i < k < j$, we know $w_i = w^{(i - 1)}_j > w^{(i - 1)}_k$ and $w^{(i - 1)}_k = w_m$ for some $m > i$. Additionally, $w_k = w^{(i - 1)}_k$ for $1 \leq k < i$ so the numbers sorted before $w^{(i - 1)}_j$ will not contribute to the number of inversions in $w$ with respect to $w_i = w^{(i - 1)}_j$. Hence, let $L(w)$ be the number of adjacent transpositions used to create $w$ with this algorithm, then $\ell_S(w) \leq L(w) = \text{inv}(w)$. It remains to show that $\ell_S(w) \geq \text{inv}(w)$. Notice that the identity permutation is a product of $0$ adjacent transpositions, and each transposition increases the number of inversions of a permutation by at most $1$. Hence, we need at least $\text{inv}(w)$ adjacent transpositions to produce a permutation with $\text{inv}(w)$ inversions, and thus $\ell_S(w) \geq \text{inv}(w)$.
  \end{proof}
\end{homeworkProblem}

\newpage

\begin{homeworkProblem}
  The set $T = \{(i\;j) : 1 \leq i < j \leq n\}$ of all transpositions generates the symmetric group $\sym_n$. For $w \in \sym_n$, the \emph{absolute length} $\ell_T(w)$ is defined to be the minimum number $r$ so that $w = t_1 t_2 \cdots t_r$ for some $t_1, t_2, \ldots, t_r \in T$. Prove that $\ell_T(w) \;=\; n - \mathrm{cyc}(w)$.

  \begin{proof}
    Let $w = c_1\ldots c_k \in \sym_n$, where $c_1, \ldots, c_k$ are disjoint cycles and each $c_i$ is a $m_i$-cycle. Note that $c_i$ is a product of $m_i - 1$ transpositions, so $w$ can be written as a product of $\sum_{i = 1}^k (m_i - 1) = \left(\sum_{i = 1}^k m_k\right) - k = n - k$ transpositions. Thus, $\ell_T(w) \leq n - k$. It remains to show that $\ell_T(w) \geq n - k$. Notice the identity permutation is a product of $n$ disjoint $1$-cycles, and each transposition decreases the number of disjoint cycles of a permutation by at most $1$. It now follows that we need at least $n - k$ transpositions to produce a permutation with $k$ cycles, and thus $\ell_T(w) \geq n - k$.
  \end{proof}
\end{homeworkProblem}

\newpage

\begin{homeworkProblem}
  Prove the following identity of formal power series using the theory of partitions:
  \[
    \prod_{i \geq 1} \frac{1}{1 - x^i y} = \sum_{k \geq 0} \frac{x^{k^2} y^k}{(1 - x)(1 - x^2) \cdots (1 - x^k)(1 - yx)(1 - yx^2) \cdots (1 - yx^k)}.
  \]

  \begin{proof}
    Note that the left-hand-side is the generating function for partitions, where the exponent of $y$ represents the number of parts. Given $k \geq 0$, we show how to generate a partition with a $k \times k$ Durfee square. Start with a $k \times k$ Durfee square, this has generating function $x^{k^2}y$. We may choose two partitions with at most $k$ parts to add to the right and bottom sides of the Durfee square. The generating  function for partition with at most $k$ parts is $\frac{1}{(1 - x)(1 - x^2) \cdots (1 - x^k)}$. However, each part of the bottom partition contributes to an addition part to the whole partition. Hence, we need to use the generating function which records the number of parts, which is $\frac{1}{(1 - yx)(1 - yx^2) \cdots (1 - yx^k)}$. For partitions with a $k \times k$ Durfee square, we now have the generating function $x^{k^2}y \cdot \frac{1}{(1 - x)(1 - x^2) \cdots (1 - x^k)} \cdot \frac{1}{(1 - yx)(1 - yx^2) \cdots (1 - yx^k)}$. This gives us the right-hand-side.
  \end{proof}
\end{homeworkProblem}

\newpage

\begin{homeworkProblem}
  Prove the following identity of formal power series using the theory of partitions:
  \[
    \prod_{i \geq 1} (1 + x^{2i-1} y) = \sum_{k \geq 0} \frac{x^{k^2} y^k}{(1 - x^2)(1 - x^4) \cdots (1 - x^{2k})}.
  \]

  \begin{proof}
    Note that the left-hand-side equals the generating function for partitions into distinct odd parts where the exponent of $y$ represents the number of parts. Let $P_k$ be the set of such partitions of $k$ parts. Given $\lambda \in P_k$, $|\lambda| \geq 1 + 3 + \cdots + (2k - 1) = k^2$, as $\lambda_i \geq 2i - 1$. Hence, for $\lambda \in P_k$, we may write $\lambda_i = 2i - i + 2\mu_i$, where $\mu_i$ is even and $\mu_1 \leq \mu_2 \leq \cdots \leq \mu_k$. That is, we may generate $P_k$ by starting with a partition of $k^2$ into $k$ distinct odd parts, and we choose an non-decreasing sequence of even numbers $(\mu_1, \ldots, \mu_k)$ to add to the corresponding odd parts. This gives us the right-hand-side.
  \end{proof}
\end{homeworkProblem}

\newpage

\begin{homeworkProblem}
  Let $\alpha = (\alpha_1, \dots, \alpha_r) \models n$ be a composition of a positive integer $n$ and let $\mathbb{F}$ be an arbitrary field. Let $P_{\alpha} \subseteq GL_n(\mathbb{F})$ be the parabolic subgroup of block upper-triangular invertible matrices whose diagonal blocks have sizes $\alpha_1 \times \alpha_1, \dots, \alpha_r \times \alpha_r$. Describe a bijection between cosets $GL_n(\mathbb{F})/P_{\alpha}$ and the family of flags $W_{\bullet} = (W_0 \subset W_1 \subset \cdots \subset W_r)$ of subspaces of $\mathbb{F}^n$ such that $\dim W_i = \alpha_1 + \cdots + \alpha_i$.

  \begin{proof}
    Consider flag $0 = V_0 \subset V_1 \subset \cdots \subset V_r = \mathbb{F}^n$, where $V_i$ is the span of the first $\alpha_1 + \cdots + \alpha_i$ standard basis vector.  For $g \in GL_n(\mathbb{F})$, define the map by sending $gP_\alpha$ to $g(V_0)\subset g(V_1) \subset \cdots \subset g(V_r)$. Notice $P_\alpha(V_i) = V_i$ for all $i$, so $P_\alpha$ is a stabilizer of the flag $V_0 \subset V_1 \subset \cdots \subset V_r$. Hence, this map is well-defined as $gP_\alpha(V_i) = gV_i$, and it is obviously bijective.
  \end{proof}
\end{homeworkProblem}

\newpage

\begin{homeworkProblem}
  Let $n > 1$ be an integer. Use a sign-reversing involution to prove the identity
  \[
  \sum_{w \in \sym_n} (-1)^{\operatorname{inv}(w)} = 0.
  \]
  Your involution should also prove the identity
  \[
  \sum_{w \in \sym_n} (-1)^{\operatorname{cyc}(w)} = 0.
  \]

  \begin{proof}
    Consider the involution $f: \sym_n \to \sym_n$ which swaps the last two elements of the permutation. If $w(n) > w(n - 1)$, then $f$ would increase $\operatorname{inv}(w)$ by 1. Otherwise, $f$ would decrease $\operatorname{inv}(w)$ by 1. That is, $f$ changes the parity of $\operatorname{inv}(w)$, and so
    \[
      \sum_{w \in \sym_n} (-1)^{\operatorname{inv}(w)} = \frac{1}{2}\left(\sum_{w \in \sym_n} (-1)^{\operatorname{inv}(w)} + (-1)^{\operatorname{inv}(f(w))}\right) = 0,
    \]
    On the other hand, consider $f$'s effect on the number of cycles of $w$. Suppose $n$ and $n - 1$ are in the same cycle of $w$, say $c_1c_2\ldots c_kc_1$ where $n = c_i, w(n) = c_{i + 1}, n - 1 = c_j, w(n - 1) = c_{j + 1}$, for some $1\leq i < j \leq k$. Then $f$ would split the cycle into two cycles, $c_i c_{j + 1} \ldots c_1 \ldots c_i$ and $c_{i + 1} \ldots c_j c_{i + 1}$. Otherwise, assume that $n$ and $n - 1$ are in different cycles of $w$, say $c_{i_1}c_{i_2} \ldots c_{i_k}$ and $c_{j_1}c_{j_2} \ldots c_{j_l}$, where $n = c_{i_1}, w(n) = c_{i_2}, n - 1 = c_{j_1}, w(n - 1) = c_{j_2}$. Then $f$ would merge the two cycles into one cycle, $c_{i_1}c_{j_2} \ldots c_{j_l}c_{j_1} \ldots c_{i_2} \ldots c_{i_k}c_{i_1}$. Hence, $f$ changes the parity of $\operatorname{cyc}(w)$, and so
    \[
      \sum_{w \in \sym_n} (-1)^{\operatorname{cyc}(w)} = \frac{1}{2}\left(\sum_{w \in \sym_n} (-1)^{\operatorname{cyc}(w)} + (-1)^{\operatorname{cyc}(f(w))}\right) = 0.
    \]
  \end{proof}
\end{homeworkProblem}

\newpage

\begin{homeworkProblem}
  Let $n$ and $m$ be positive integers and let $f(n,m)$ be the number of $0,1$-matrices of dimensions $n \times m$ which do not contain any zero rows or columns. Find a formula for $f(n,m)$.


  \begin{proof}
    For $S \subseteq [m]$, let $g(S)$ be the number of $0,1$-matrices of dimensions $n \times m$ that have no zero rows but have zero columns indexed by $S$, and let $h(S)$ be the number of $0,1$-matrices of dimensions $n \times m$ that have no zero rows and have zero columns exactly at the columns indexed by $S$. Note that for all $S \subseteq [m]$,
    \[
      g(S) = \sum_{S \subseteq T \subseteq [m]} h(T).
    \]
    Given $T \subseteq [m]$, there are $(2^{m - |T|} - 1)^n$ ways to choose $n$ rows that is not completely zero in the columns not indexed by $T$ and zero in the columns indexed by $T$, so $g(T) = (2^{m - |T|} - 1)^n$. Principle of Inclusion-Exclusion now yields
    \[
      f(n, m) = h(\emptyset) = \sum_{T \subseteq [n]} (-1)^{|T|} g(T) = \sum_{k = 1}^m (-1)^{k}\binom{m}{k}(2^{m - k} - 1)^n.
    \]
  \end{proof}
\end{homeworkProblem}

\newpage

\begin{homeworkProblem}
  What is the average number of fixed points of a permutation $w \in \mathcal{S}_n$?

  \begin{proof}
    Let $X$ be the expected number of fixed points in a random permutation $w \in \mathcal{S}_n$, and let $X_i$ be the indicator for the event that $w(i) = i$. Then
    \[
      \E[X] = \sum_{i = 1}^n \E[X_i] = \sum_{i = 1}^n \prob(w(i) = i) = \sum_{i = 1}^n \frac{1}{n} = 1.
    \]
  \end{proof}
\end{homeworkProblem}

\newpage

\begin{homeworkProblem}
  Let $q$ be a prime power and let $\mathbb{F}_q$ be the finite field with $q$ elements. For $k \leq n$, find a formula for the number of $\mathbb{F}_q$-linear surjections $\varphi : \mathbb{F}_q^n \twoheadrightarrow \mathbb{F}_q^k$.

  \begin{proof}
    For each subsapce $U$ of $\F^k_q$, define $f(U)$ to be the number of linear maps $\mathbb{F}_q^n \rightarrow \mathbb{F}_q^k$ whose image lies in $U$, and define $g(U)$ to be the number of linear maps $\mathbb{F}_q^n \rightarrow \mathbb{F}_q^k$ whose image is $U$. Obviously,
    \[
      f(U) = \sum_{V \text{ subspace of } U} g(V).
    \]
    By Möbius inversion on the lattice of subspaces of $\F^k_q$, 
    \[
      g(\F_q^k) = \sum_{V \text{ subspace of } U} \mu(V, \F_q^k)f(V).
    \]
    Since $\mu(V, \F_q^k) = (-1)^{k - \dim(V)}q^{\binom{k - \dim(V)}{2}}$ and $f(V) = q^{n \cdot \dim V}$,
    \[
      g(\F_q^k) = \sum_{i = 1}^k \begin{bmatrix}
        k \\ i
      \end{bmatrix}_q (-1)^{k - i}q^{\binom{k - i}{2}}q^{ni}.
    \]
  \end{proof}
\end{homeworkProblem}

\newpage

\begin{homeworkProblem}
  Let $P$ be a finite poset which has a minimum element $\hat{0}$. Prove that the order complex $\Delta(P)$ is contractible. (This means that $\Delta(P)$ can be continuously deformed to a point. For example, the 2-dimensional disc $D^2$ is contractible whereas its boundary circle $S^1$ is not.)

  \begin{proof}
    Note that $\hat{0}$ is minimal and any chain in $P$ can be extended to another chain in $P$ that includes $\hat{0}$ as a unique minimum element. Hence, all maximal simplices in $\Delta(P)$ are simplices of $\{\hat{0}\}$. Hence, given a maximal chain $C$ in $P$, we continuously deform the corresponding simplex of $C$ by iteratively removing the elements of starting from the end of the chain to $\hat{0}$. This deformation iteratively reduces the dimension of the corresponding simplex of $C$ until it is a point at $\hat{0}$. Hence, $\Delta(P)$ is contractible.

    (I have no idea what I wrote.)
  \end{proof}
\end{homeworkProblem}

\newpage

\begin{homeworkProblem}
  Let $\Pi_n$ be the lattice of set partitions of $[n]$, partially ordered by $\pi \leq \sigma$ if and only if $\pi$ refines $\sigma$. These exercises outline an alternative proof that
  \[
  \mu_{\Pi_n}[\hat{0}, \hat{1}] = (-1)^{n-1} \cdot (n-1)!.
  \]
  \begin{enumerate}[(a)]
    \item Let $\pi = \{B_1 \mid \cdots \mid B_k\}$ be a set partition of $[n]$. Give a poset isomorphism
    \[
    [\hat{0}, \pi]_{\Pi_n} \cong [\hat{0}, \hat{1}]_{\Pi_{|B_1|}} \times \cdots \times [\hat{0}, \hat{1}]_{\Pi_{|B_k|}}.
    \]
    \begin{proof}
      Let $\sigma \in [\hat{0}, \pi]_{\Pi_n}$. Note that each block of $\sigma$ is a subset of some unique $B_i$. Hence, we may index the blocks of $\sigma$ that refines $B_i$ as $B_{i_1}, B_{i_2}, \ldots, B_{i_{m_1}}$. Define $f: [\hat{0}, \pi]_{\Pi_n} \to [\hat{0}, \hat{1}]_{\Pi_{|B_1|}} \times \cdots \times [\hat{0}, \hat{1}]_{\Pi_{|B_k|}}$ by 
      \[
      f(\sigma) = (\sigma_{B_1}, \ldots, \sigma_{B_k}),
      \]
      where $\sigma_{B_i} = \{B_{i_1} \mid \cdots \mid B_{i_{m_i}}\} \in \Pi_{|B_i|}$. Since the operation of $f$ can be reversed, $f$ is well-defined and bijective. Suppose $x, y \in [\hat{0}, \pi]_{\Pi_n}$ such that $x \leq y$. Then $x_{B_i} \leq y_{B_i}$ for all $i$, so $f(x) \leq f(y)$. Hence, $f$ is order-preserving. 
    \end{proof}

    \item We have a natural map $\varphi : \mathcal{S}_n \to \Pi_n$ where $\varphi(w)$ is the set partition of $[n]$ whose blocks are the cycles of the permutation $w \in \mathcal{S}_n$. If $\pi = \{B_1 \mid \cdots \mid B_k\}$ is a set partition of $[n]$, prove that the pre-image $\varphi^{-1}(\pi)$ has size
    \[
    \# \varphi^{-1}(\pi) = (|B_1| - 1)! \cdots (|B_k| - 1)!.
    \]
    \begin{proof}
      Since there are $(r - 1)!$ ways to arrange $r$ distinct elements into a cycle, for each $B_i$ there are $(|B_i| - 1)!$ possible cycles that would result in $B_i$ as a block. The result now follows.
    \end{proof}

    \item Use the results in Problems 1 and 2 to deduce that $\mu_{\Pi_n}[\hat{0}, \hat{1}]$. (Use induction on $n$. It may be helpful to remember a result about the alternating sum of Stirling numbers of the first kind.)
    \begin{proof}
      We proceed on induction on $n$. For $n = 1$, $\Pi_1$ only has one partition, so $\mu_{\Pi_1}[\hat{0}, \hat{1}] = 1$. Suppose $n \geq 2$. The Möbius function satisfies
      \[
        \mu_{\Pi_n}[\hat{0}, \hat{1}] = -\sum_{\pi < \hat{1}} \mu_{\Pi_n}[\hat{0}, \pi].
      \]
      For $\pi < \hat{1}$,
      \[
        \mu_{\Pi_n}[\hat{0}, \pi] = \mu_{\Pi_{|B_1|}}[\hat{0}, \hat{1}] \cdots \mu_{\Pi_{|B_{k_\pi}|}}[\hat{0}, \hat{1}] = \prod_{i = 1}^k (-1)^{|B_i| - 1}(|B_i| - 1)! = (-1)^{n - k}(|B_1| - 1)! \cdots (|B_{k_\pi}| - 1)!,
      \]
      by (a) and induction. But then by (b),
      \[
        \mu_{\Pi_n}[\hat{0}, \hat{1}] = -\sum_{\pi < \hat{1}}(-1)^{n - k_\pi} \cdot \# \varphi^{-1}(\pi) = (-1)^{n - 1}\sum_{k = 2}^n (-1)^{k}c(n, k).
      \]
      Since $\sum_{k = 1}^n (-1)^{k}c(n, k) = 0$, 
      \[
        \mu_{\Pi_n}[\hat{0}, \hat{1}] = (-1)^{n - 1}c(n, 1) = (-1)^{n - 1}(n - 1)!.
      \]
    \end{proof}

  \end{enumerate}
\end{homeworkProblem}

\newpage

\begin{homeworkProblem}
  Let $\mathcal{A}$ be the arrangement in $\mathbb{R}^n$ and let $R$ be a region of $\mathcal{A}$. The \textit{recession cone} of $R$ is
  \[
  \text{Rec}(R) := \{ v \in \mathbb{R}^n \mid R + v \subseteq R \}.
  \]
  Prove that $\text{Rec}(R)$ is a \textit{cone} in $\mathbb{R}^n$. That is, prove that if $v_1, \dots, v_r \in \text{Rec}(R)$ and if $a_1, \dots, a_r > 0$ we have
  \[
  a_1 v_1 + \cdots + a_r v_r \in \text{Rec}(R).
  \]

  \begin{proof}
    Let $x \in R$. By definition of $\text{Rec}(R)$, we have $x + v_r \in R$ for all $i$. Since $R$ is convex, $x + tv_r \in R$ for all $t \in [0, 1]$. But then given any $a_r > 0$, we may repeatedly apply the above argument to show that $x + a_rv_r \in R$. The result now follows from induction on $r$.
  \end{proof}
\end{homeworkProblem}

\newpage

\begin{homeworkProblem}
  Let $\mathcal{A}$ be the arrangement in $\mathbb{R}^n$ with hyperplanes $x_i = x_j$ for $1 \leq i < j \leq n$ and 
  \[
  x_1 + x_2 + \cdots + x_n = 0.
  \]
  Prove that $\mathcal{A}$ has characteristic polynomial
  \[
  \chi_{\mathcal{A}}(t) = (t - 1)^2 (t - 2)(t - 3) \cdots (t - n + 1).
  \]

  \begin{proof}
    We now use the finite field method to compute $\chi_{\mathcal{A}}(t)$. Let $q$ be a large prime power and let $\mathbb{F}_q$ be the finite field with $q$ elements. The finite field method yields
    \begin{align*}
      \chi_{\mathcal{A}}(q) 
      &= \# \left(\F_q^n \backslash \bigcup_{H^{(p)} \in \mathcal{A}^{(p)}} H^{(p)}\right) \\ 
      &= \#\{x \in \F_q^n \mid x_i \neq x_j \text{ for $i \neq j$ and } x_1 + \cdots + x_n \neq 0\} \\
      &= \#\{x \in \F_q^n \mid x_i \neq x_j \text{ for $i \neq j$}\} - \#\{x \in \F_q^n \mid x_i \neq x_j \text{ for $i \neq j$ and } x_1 + \cdots + x_n = 0\}.
    \end{align*}
    For $a \in \F_q$, define
    \[
      S_a = \#\{x \in \F_q^n \mid x_i \neq x_j \text{ for $i \neq j$ and } x_1 + \cdots + x_n = a\}.
    \]
    Let $a, b \in \F_q$. Since $q$ is prime, there exists unique $k \in \F_q$ such that $a + nk \equiv b \pmod q$. Define $f_{a, b}: S_a \to S_b$ by
    \[
      f_{a, b}(x_1, \ldots, x_n) = (x_1 + k, x_2 + k, \ldots, x_n + k).
    \]
    Define $f_{b, a}$ in the same fashion, and we get $f_{a, b} \circ f_{b, a} = f_{b, a} \circ f_{a, b}$ is the identity map. Hence, $|S_a| = |S_b|$ for all $a, b \in \F_q$, and so
    \[
      |S_0| = \frac{1}{q} \cdot \#\{x \in \F_q^n \mid x_i \neq x_j \text{ for $i \neq j$}\}.
    \]
    Since 
    \[
      \#\{x \in \F_q^n \mid x_i \neq x_j \text{ for $i \neq j$}\} = q(q - 1)(q - 2) \cdots (q - n + 1),
    \]
    we have
    \[
      \chi_{\mathcal{A}}(q) = (q - 1)^2 (q - 2)(q - 3) \cdots (q - n + 1).
    \]
    The result now follows.
  \end{proof}
\end{homeworkProblem}
\end{document}