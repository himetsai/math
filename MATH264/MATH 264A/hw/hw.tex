\documentclass{article}

\usepackage{fancyhdr}
\usepackage{extramarks}
\usepackage{amsmath}
\usepackage{amsthm}
\usepackage{amsfonts}
\usepackage{tikz}
\usepackage[plain]{algorithm}
\usepackage{algpseudocode}
\usepackage{enumerate}
\usepackage{amssymb}
\usepackage{dsfont}

\usetikzlibrary{automata,positioning}

%
% Basic Document Settings
%

\topmargin=-0.45in
\evensidemargin=0in
\oddsidemargin=0in
\textwidth=6.5in
\textheight=9.0in
\headsep=0.25in

\linespread{1.1}

\pagestyle{fancy}
\lhead{\hmwkAuthorName}
\chead{\hmwkClass:\ \hmwkTitle}
\rhead{\firstxmark}
\lfoot{\lastxmark}
\cfoot{\thepage}

\renewcommand\headrulewidth{0.4pt}
\renewcommand\footrulewidth{0.4pt}

\setlength\parindent{0pt}
\setlength{\parskip}{5pt}

%
% Create Problem Sections
%

\newcommand{\enterProblemHeader}[1]{
    \nobreak\extramarks{}{Problem \arabic{#1} continued on next page\ldots}\nobreak{}
    \nobreak\extramarks{Problem \arabic{#1} (continued)}{Problem \arabic{#1} continued on next page\ldots}\nobreak{}
}

\newcommand{\exitProblemHeader}[1]{
    \nobreak\extramarks{Problem \arabic{#1} (continued)}{Problem \arabic{#1} continued on next page\ldots}\nobreak{}
    \stepcounter{#1}
    \nobreak\extramarks{Problem \arabic{#1}}{}\nobreak{}
}

\setcounter{secnumdepth}{0}
\newcounter{partCounter}
\newcounter{homeworkProblemCounter}
\setcounter{homeworkProblemCounter}{1}
\nobreak\extramarks{Problem \arabic{homeworkProblemCounter}}{}\nobreak{}

%
% Homework Problem Environment
%
% This environment takes an optional argument. When given, it will adjust the
% problem counter. This is useful for when the problems given for your
% assignment aren't sequential. See the last 3 problems of this template for an
% example.
%
\newenvironment{homeworkProblem}[1][-1]{
    \ifnum#1>0
        \setcounter{homeworkProblemCounter}{#1}
    \fi
    \section{Problem \arabic{homeworkProblemCounter}}
    \setcounter{partCounter}{1}
    \enterProblemHeader{homeworkProblemCounter}
}{
    \exitProblemHeader{homeworkProblemCounter}
}

%
% Homework Details
%   - Title
%   - Due date
%   - Class
%   - Section/Time
%   - Instructor
%   - Author
%

\newcommand{\hmwkTitle}{Homework}
\newcommand{\hmwkDueDate}{Nov 2, 2024}
\newcommand{\hmwkClass}{MATH 264A}
\newcommand{\hmwkClassInstructor}{Professor Warnke}
\newcommand{\hmwkAuthorName}{\textbf{Ray Tsai}}
\newcommand{\hmwkPID}{A16848188}

%
% Title Page
%

\title{
    \vspace{2in}
    \textmd{\textbf{\hmwkClass:\ \hmwkTitle}}\\
    \normalsize\vspace{0.1in}\small{Due\ on\ \hmwkDueDate\ at 23:59pm}\\
    \vspace{0.1in}\large{\textit{\hmwkClassInstructor}} \\
    \vspace{3in}
}

\author{
  \hmwkAuthorName \\
  \vspace{0.1in}\small\hmwkPID
}
\date{}

\renewcommand{\part}[1]{\textbf{\large Part \Alph{partCounter}}\stepcounter{partCounter}\\}

%
% Various Helper Commands
%

% Useful for algorithms
\newcommand{\alg}[1]{\textsc{\bfseries \footnotesize #1}}

% For derivatives
\newcommand{\deriv}[1]{\frac{\mathrm{d}}{\mathrm{d}x} (#1)}

% For partial derivatives
\newcommand{\pderiv}[2]{\frac{\partial}{\partial #1} (#2)}

% Integral dx
\newcommand{\dx}{\mathrm{d}x}

% Probability commands: Expectation, Variance, Covariance, Bias
\newcommand{\Var}{\mathrm{Var}}
\newcommand{\Cov}{\mathrm{Cov}}
\newcommand{\Bias}{\mathrm{Bias}}
\newcommand*{\Z}{\mathbb{Z}}
\newcommand*{\Q}{\mathbb{Q}}
\newcommand*{\R}{\mathbb{R}}
\newcommand*{\C}{\mathbb{C}}
\newcommand*{\N}{\mathbb{N}}
\newcommand*{\prob}{\mathds{P}}
\newcommand*{\E}{\mathds{E}}

\begin{document}

\maketitle

\pagebreak

\begin{homeworkProblem}
  \begin{enumerate}[(a)]
    \item Let $G = (V, E)$ be an $n$-vertex graph and suppose that each vertex $v \in V$ is
    associated with a list $S(v)$ of colors of size at least $4r$, where $r$ is a positive integer.
    Suppose also that for each $v \in V$ and each $c \in S(v)$, there are at most $r$ neighbors $u$
    of $v$ such that $c \in S(u)$. Using induction, prove that there are at least $(2r)^n$ proper
    colorings of $G$ under which each vertex $v$ receives a color from its list $S(v)$.

    \textit{Remark}: a proof using the classical Lovász Local Lemma (LLL) requires $2e r \approx
    5.44r$ instead of $4r$.

    \begin{proof}
      For $S \subseteq V$, define $N_S$ as the number of proper colorings of $G$ under which $v$
      receives a color from its list $S(v)$ for all $v \in S$. It suffices to show that for all $T
      \subseteq V$ and $x \in T$, $N_T/N_{T\backslash \{x\}} \geq 2r$. We proceed by induction on
      $|T|$. When $T = \{x\}$, $N_T = S(x)$ and $N_{T\backslash \{x\}} = N_{\emptyset} = 1$, we have
      $N_T/N_{T\backslash \{x\}} \geq 4r \geq 2r$. Suppose $|T| > 2$. Then,
      \[
        |S(x)| \cdot N_{T\backslash \{x\}} = N_T + B,
      \]
      where $B$ is the number of improper colorings of $T$ which becomes proper if we ignore the
      color of $x$. Notice that any element counted by $B$ contains a vertex $u$ which is a neighbor
      of $v$ that shares the same color $c$ as $x$ and a proper coloring of $T \backslash \{u, x\}$.
      This yields the upper bound
      \[
        B \leq |S(x)| \cdot (\# \text{choice of vertex $u$}) \cdot N_{T \backslash \{u, x\}} \leq 4r \cdot r \cdot N_{T \backslash \{u, x\}}.
      \]
      By induction, $N_{T \backslash \{u, x\}} \leq \frac{1}{2r} \cdot N_{T \backslash \{x\}}$, and
      thus 
      \[
        N_T = |S(x)| \cdot N_{T\backslash \{x\}} - B \geq 4r N_{T\backslash \{x\}} - 4r \cdot r \cdot \frac{1}{2r} \cdot N_{T\backslash \{x\}} = 2r N_{T\backslash \{x\}},
      \]
      and this completes the induction.
    \end{proof}

    \item A $k$-SAT formula is an expression such as
    \[
    (x_1 \text{ OR } x_4 \text{ OR } \overline{x_6}) \text{ AND } (x_1 \text{ OR } \overline{x_2} \text{ OR } x_5),
    \]
    where the variables $x_i$ take values true or false, $\overline{x_i}$ means not $x_i$, and $k$
    distinct variables or their negations are OR-ed together in each clause. A formula is called
    satisfiable if there is an assignment of values to the variables making the expression true.
    Suppose that in a given $k$-SAT formula $\Phi$ no variable lies in more than $2^k / (ek)$
    clauses. Using induction, prove that $\Phi$ has at least $(2 - 2/k)^n$ many satisfying
    assignments (which in fact remains true if we relax the assumed $2^k / (ek)$ upper bound to $2^k
    / k \cdot (1 - 1/k)^{k-1}$).

    \begin{proof}
      Let $\Phi_i$ denote the $k$-SAT formula which AND's together all the clauses of $\Phi$ that
      involve only the first $i$ variables. Define $N_i$ as the number of satisfying assignments of
      $\Phi_i$. We show that $N_i\geq (2 - 2/k)^i$ by induction on $i \geq k$. And then I'm stuck.
    \end{proof}

    \item Let $\mathcal{A}$ be an alphabet, and let $\mathcal{F}$ be a set of forbidden strings.
    Assume that there exists $\beta > 0$ such that 
    \[
    |\mathcal{A}| - \sum_{f \in \mathcal{F}} \beta^{1 - |f|} \geq \beta.
    \]
    Using induction, prove that there exists at least $\beta^n$ words of length $n$ over alphabet
    $\mathcal{A}$ that avoid all the substrings in $\mathcal{F}$.

    \begin{proof}
      Define $N_k$ to be the set of words of length $n$ over alphabet $\mathcal{A}$ that avoid all
      the substrings in $\mathcal{F}$. We show that $|N_k|/|N_{k - 1}| \geq \beta$ by induciton on $k
      \geq 1$. Since $|N_0| = 1$, 
      \[
        \frac{|N_1|}{|N_0|} = |N_1| \geq |\mathcal{A}| - |\{f \in F \mid |f| = 1\}| \geq \beta,
      \]
      as $|\{f \in F \mid |f| = 1\}| \leq \sum_{f \in \mathcal{F}} \beta^{1 - |f|}$. Hence, the base
      case holds. Suppose $k > 1$. Let $B$ denote the set of words over $\mathcal{A}$ of the form
      $a_1\ldots a_k$ that contains some substring in $F$, with $a_1\ldots a_{k - 1} \in N_{k - 1}$.
      Then
      \[
        |\mathcal{A}| \cdot |N_{k - 1}| = |N_k| + |B|.
      \]
      By construction, any word in $B$ consists of a forbidden string $f$ at the end and some word
      in $N_{k - |f|}$ at the beginning. Summing over all $f$, we have the bound
      \[
        |B| \leq \sum_{f \in \mathcal{F}} N_{k - |f|}.
      \]
      By induction, $N_{k - |f|} \leq \beta^{1 - |f|} \cdot |N_{k - 1}|$, and thus
      \[
        |B| \leq |N_{k - 1}|\sum_{f \in \mathcal{F}} \beta^{1 - |f|}.
      \]
      Therefore, 
      \[
        |N_k| = |\mathcal{A}| \cdot |N_{k - 1}| - B \geq |N_{k - 1}|\left(|\mathcal{A}| - \sum_{f \in \mathcal{F}} \beta^{1 - |f|}\right) \geq \beta |N_{k - 1}|,
      \]
      and this completes the induction. It now follows that 
      \[
        |N_n| = \frac{|N_n|}{|N_{n - 1}|} \cdot \frac{|N_{n - 1}|}{|N_{n - 2}|} \cdots \frac{|N_1|}{|N_0|} \geq \beta^n.
      \]
    \end{proof}
  \end{enumerate}
\end{homeworkProblem}

\newpage

\begin{homeworkProblem}
  In the inductive proof of the `almost all triangle-free graphs are 2-colorable' result, we defined
  the following sets (using $\Gamma(v)$ and $\Gamma(S)$ to denote the set of neighbors of a vertex
  $v$ or set of vertices $S$):

  \begin{itemize}
    \item $\mathcal{C}ol_2(n)$ is the set of all 2-colorable graphs on $n$ vertices.
    \item $\mathcal{T}(n)$ is the set of all triangle-free graphs on $n$ vertices.
    \item $\mathcal{A}(n) \subseteq \mathcal{T}(n)$ is the subset of graphs containing a vertex $v$
    such that $|\Gamma(v)| \leq \log_2 n$.
    \item $\mathcal{B}(n) \subseteq \mathcal{T}(n)$ is the subset of graphs containing a vertex set
    $Q$ of size $|Q| = \log_2 n$, such that $|\Gamma(Q)| \leq (1/2 - 1/1000)n$.
    \item $\mathcal{D}(n) \subseteq \mathcal{T}(n) \setminus (\mathcal{A}(n) \cup \mathcal{B}(n))$
    is the subset of graphs containing an edge $\{x, y\}$ and vertex sets $Q_x \subseteq \Gamma(x)$
    and $Q_y \subseteq \Gamma(y)$, such that $|Q_x| = \log_2 n$, $|Q_y| = \log_2 n$, and
    $|\Gamma(Q_x) \cap \Gamma(Q_y)| \geq n/100$.
  \end{itemize}

  \begin{enumerate}[(a)]
    \item Prove that $|\mathcal{D}(n)| / |\mathcal{T}(n - 2)| \leq 2^{(1 - 1/2000)n}$ for all
    sufficiently large $n \geq n_0$.
    \begin{proof}
      To generate a graph in $\mathcal{D}(n)$, we first pick two vertices $x, y$ to be adjacent and
      then place a triangle free graph on the remaining $n - 2$ vertices. Lastly, we pick two
      subsets from the $n - 2$ vertices to be $x$ and $y$'s neighbors respectively. Since the graph
      is not in $\mathcal{A}(n)$, $|\Gamma(x)|, |\Gamma(y)| > \log_2 n$. But then the graph is also
      not in $\mathcal{B}(n)$, so $\Gamma(\Gamma(x)), \Gamma(\Gamma(y))$ each have size greater than
      $(1/2 - 1/1000)n$. Since the graph is triangle-free, $x$ cannot be adjacent to any vertex in
      $\Gamma(\Gamma(x))$ and similarly for $y$, and thus $|\Gamma(x)|, |\Gamma(y)| \leq (1/2 +
      1/1000)n$. This yields the bound
      \[
        \mathcal{D}(n) \leq \binom{n}{2} \cdot |\mathcal{T}(n - 2)| \cdot 2^{(1/2 - 1/1000)n} \cdot 2^{(1/2 - 1/1000)n} \leq 2^{(1 - 1/2000)n} \cdot |\mathcal{T}(n - 2)|,
      \]
      for large enough $n$.
    \end{proof}
    \item Prove that $|\mathcal{C}ol_2(n)| / |\mathcal{C}ol_2(n - 1)| \geq 2^{\frac{1}{2}(n - 1)}$
    for all sufficiently large $n \geq n_0$.
    \begin{proof}
      Each graph in $\mathcal{C}ol_2(n)$ consists of vertex $n$, a graph $H \in \mathcal{C}ol_2(n -
      1)$, and edges between $n$ and $H$. Since $H$ is bipartite, $H$ contains an independent set
      $I_H$ of size at least $2^{\frac{1}{2}(n - 1)}$. By picking a graph $H$ from
      $\mathcal{C}ol_2(n - 1)$ and adding some edges between $n$ and $I_H$, we may uniquely generate
      a graph in $\mathcal{C}ol_2(n)$. Therefore,
      \[
        |\mathcal{C}ol_2(n)| \geq |\mathcal{C}ol_2(n - 1)| \cdot 2^{\frac{1}{2}(n - 1)},
      \]
      and the result now follows.
    \end{proof}
  \end{enumerate}
\end{homeworkProblem}

\newpage

\begin{homeworkProblem}
  In this problem we discuss in more detail one calculation in the proof of the Bregman's Theorem
  from class. Given a bipartite graph \( G = (L \cup R, E) \) with \( |L| = |R| = n \), fix a
  perfect matching \( M \) of \( G \). For each vertex \( i \in L \), there thus is a unique vertex
  \( X_i \) such that \( \{i, X_i\} \in M \) is a matching edge. Now write \( R_i \) for the set of
  \( j \in L \) for which \( \{j, X_j\} \in M \) and \( \{i, X_j\} \in E \), i.e., the set of
  vertices \( j \in L \) for which there is a matching edge in \( M \) that contains \( j \) and a
  neighbor of \( i \). By construction, we have \( |R_i| = \deg_G(i) \). Using a (permutation)
  counting argument, show that for each vertex \( i \in L \) and \( 1 \leq j \leq \deg_G(i) \) we
  have

  \[
    \mathbb{P}(\text{vertex } i \text{ appears in } \pi \text{ in the } j\text{th position among the vertices in } R_i) = \frac{1}{\deg_G(i)}.
  \]

  \begin{proof}
    \begin{align*}
      &\mathbb{P}(\text{vertex } i \text{ in the } j\text{th position among the vertices in } R_i \text{ in } \pi ) \\
      &= \frac{\# \{\text{permutations of $R_i$ with $i$ appearing in the $j$th position}\}}{\# \{\text{permutations of $R_i$}\}} \\
      &= \frac{\# \{\text{permutations of $R_i \backslash \{i\}$}\}}{\# \{\text{permutations of $R_i$}\}} \\
      &= \frac{(|R_i| - 1)!}{|R_i|!} = \frac{1}{\deg_G(i)}.
    \end{align*}
  \end{proof}
\end{homeworkProblem}

\newpage

\begin{homeworkProblem}
  \begin{enumerate}[(a)]
    \item An order $n$ Latin square is an $n \times n$ matrix in which each row and column is a
    permutation of $[n]$. Using the entropy method, prove that the number of order $n$ Latin squares
    is at most 
    \[
    L(n) \leq \left( (1 + o(1)) \frac{n}{e^2} \right)^{n^2} 
    \]
    as $n \to \infty$.

    \begin{proof}
      Choose $X$ uniformly at random from all order $n$ Latin squares. Let $X_{i}$ denote the $i$th
      row and let $X_{ij}$ denote the $i$th row $j$th column of $X$. Independently and uniformly
      take a random $a_{ij}$ from $[0, 1]$ for all $i, j \in [n]$, and denote $A = (a_{ij})$. Let
      $R_{ij}(A) = \{X_{il}: a_{il} < a_{ij}\}$ and $C_{ij}(A) = \{X_{kj}: a_{kj} < a_{ij}\}$. Note
      that $\prob(a_{ij} = a_{kl}) = 0$ for all $(i, j) \neq (k, l)$. By the Chain rule,
      \[
        \log_2 L(n) = H(X) = \E_A\sum_{i, j \in [n]} H(X_{ij} \mid X_{kl}: a_{kl} < a_{ij}).
      \]
      Let $N_{ij}(A, X) = |[n] - R_{ij}(A) \cup C_{ij}(A)|$. Rewriting the entropy in expectation
      form, 
      \[
        H(X_{ij} \mid X_{kl}: a_{kl} < a_{ij}) = \E_X[\log_2 N_{ij}(A, X)],
      \]
      and thus
      \[
        H(X) = \E_A\sum_{i, j \in [n]} \E_X[\log_2 N_{ij}(A, X)] = \E_X \sum_{i, j \in [n]} \E_A[\log_2 N_{ij}(A, X) \mid X].
      \]
      By Jensen's inequality,
      \[
        \E_A[\log_2 N_{ij}(A, X) \mid X] \leq \E_{a_{ij}}[\log_2 \E_{A \backslash \{a_{ij}\}} [N_{ij}(A, X) \mid X, a_{ij}]].
      \]
      Note that each row and column of a given $X$ is a fixed permutation of $[n]$, so each $m \in
      [n]$ must be assigned to some square in a given row or column. Let $m \in [n]$. Since we are
      conditioning on $a_{ij}$, if $m = X_{ij}$, then $\prob[m \notin (R_{ij}(A) \cup C_{ij}(A))] =
      1$. Otherwise, if $m \neq X_{ij}$,
      \[
        \prob[m \notin (R_{ij}(A) \cup C_{ij}(A))] = \prob(m \notin R_{ij}(A))\prob(m \notin C_{ij}(A)) = (1 - a_{ij})^2,
      \]
      and thus
      \[
        \E_{A \backslash \{a_{ij}\}} [N_{ij}(A, X) \mid X, a_{ij}] = \sum_{m \in [n]} \prob[m \notin (R_{ij}(A) \cup C_{ij}(A))] = 1 + (n - 1)(1 - a_{ij})^2.
      \]
      This yields
      \[
        \E_{a_{ij}}[\log_2 \E_{A \backslash \{a_{ij}\}} [N_{ij}(A, X) \mid X, a_{ij}]] = \int_{0}^1 \log_2 n(1 - x)^2 \, dx \leq \log_2(n) - 2 + o(1).
      \]

      It now follows that
      \[
        H(X) \leq \E_X \sum_{i, j \in [n]} \E_{a_{ij}}[\log_2(n) - 2 + o(1)] \leq n^2(\log_2(n) - 2 + o(1)),
      \]
      and thus the result.
    \end{proof}

    \newpage
    
    \item Sudoku puzzles are order 9 Latin squares, divided into 9 smaller $3 \times 3$ blocks, with the additional constraint that each block must contain all the symbols $\{1, \ldots, 9\}$. Sudoku squares of order $N$ can be defined similarly for any square number $N = n^2$ (to clarify: normal Sudoku puzzles are simply order 9 Sudoku squares). Using the entropy method, prove that the number of order $N$ Sudoku squares is at most 
    \[
    S(n) \leq \left( (1 + o(1)) \frac{N}{e^3} \right)^{N^2} 
    \]
    as $N \to \infty$.

    \begin{proof}
      Choose $X$ uniformly at random from all order $n$ Latin squares. Let $X_{i}$ denote the $i$th
      row and let $X_{ij}$ denote the $i$th row $j$th column of $X$. Independently and uniformly
      take a random $a_{ij}$ from $[0, 1]$ for all $i, j \in [n]$, and denote $A = (a_{ij})$. 
      Define
      \begin{itemize}
        \item $R_{ij}(A) = \{X_{il}: a_{il} < a_{ij}\}$ 
        \item $C_{ij}(A) = \{X_{kj}: a_{kj} < a_{ij}\}$
        \item $B_{ij}(A) = \{X_{kl}: a_{kl} < a_{ij} \text{ and } X_{kl} \text{ in the same block
        as } X_{ij}\}$.
      \end{itemize}
      Note that $\prob(a_{ij} = a_{kl}) = 0$ for all $(i, j) \neq (k, l)$. By the Chain
      rule,
      \[
        \log_2 L(n) = H(X) = \E_A\sum_{i, j \in [n]} H(X_{ij} \mid X_{kl}: a_{kl} < a_{ij}).
      \]
      Let $N_{ij}(A, X) = |[N] - R_{ij}(A) \cup C_{ij}(A) \cup B_{ij}(A)|$. Rewriting the entropy in
      expectation form, 
      \[
        H(X_{ij} \mid X_{kl}: a_{kl} < a_{ij}) = \E_X[\log_2 N_{ij}(A, X)],
      \]
      and thus
      \[
        H(X) = \E_A\sum_{i, j \in [n]} \E_X[\log_2 N_{ij}(A, X)] = \E_X \sum_{i, j \in [n]} \E_A[\log_2 N_{ij}(A, X) \mid X].
      \]
      By Jensen's inequality,
      \[
        \E_A[\log_2 N_{ij}(A, X) \mid X] \leq \E_{a_{ij}}[\log_2 \E_{A \backslash \{a_{ij}\}} [N_{ij}(A, X) \mid X, a_{ij}]].
      \]
      Note that each row, column, and block of a given $X$ is a fixed permutation of $[n]$, so each
      $m \in [n]$ must be assigned to some square in a given row, column, or block. Let $m \in [n]$.
      Since we are conditioning on $a_{ij}$, if $m = X_{ij}$, then $\prob[m \notin (R_{ij}(A) \cup
      C_{ij}(A) \cup B_{ij}(A))] = 1$. Otherwise, if $m \neq X_{ij}$,
      \[
        \prob[m \notin (R_{ij}(A) \cup C_{ij}(A) \cup B_{ij}(A))] = \prob(m \notin R_{ij}(A))\prob(m \notin C_{ij}(A))\prob(m \notin B_{ij}(A)) = (1 - a_{ij})^3,
      \]
      and thus
      \[
        \E_{A \backslash \{a_{ij}\}} [N_{ij}(A, X) \mid X, a_{ij}] = \sum_{m \in [N]} \prob[m \notin (R_{ij}(A) \cup C_{ij}(A) \cup B_{ij}(A))] = 1 + (N - 1)(1 - a_{ij})^3.
      \]
      Hence,
      \[
        \E_{a_{ij}}[\log_2 \E_{A \backslash \{a_{ij}\}} [N_{ij}(A, X) \mid X, a_{ij}]] = \int_{0}^1 \log_2 N(1 - x)^3 \, dx \leq \log_2(N) - 3 + o(1).
      \]

      It now follows that
      \[
        H(X) \leq \E_X \sum_{i, j \in [n]} \E_{a_{ij}}[\log_2 (1 + (N - 1)(1 - a_{ij})^3)] \leq N^2(\log_2(N) - 3) + o(1),
      \]
      and thus the result. 
    \end{proof}
  \end{enumerate}
\end{homeworkProblem}
\end{document}