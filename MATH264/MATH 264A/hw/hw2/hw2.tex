\documentclass{article}

\usepackage{fancyhdr}
\usepackage{extramarks}
\usepackage{amsmath}
\usepackage{amsthm}
\usepackage{amsfonts}
\usepackage{tikz}
\usepackage[plain]{algorithm}
\usepackage{algpseudocode}
\usepackage{enumerate}
\usepackage{amssymb}
\usepackage{dsfont}

\usetikzlibrary{automata,positioning}

%
% Basic Document Settings
%

\topmargin=-0.45in
\evensidemargin=0in
\oddsidemargin=0in
\textwidth=6.5in
\textheight=9.0in
\headsep=0.25in

\linespread{1.1}

\pagestyle{fancy}
\lhead{\hmwkAuthorName}
\chead{\hmwkClass:\ \hmwkTitle}
\rhead{\firstxmark}
\lfoot{\lastxmark}
\cfoot{\thepage}

\renewcommand\headrulewidth{0.4pt}
\renewcommand\footrulewidth{0.4pt}

\setlength\parindent{0pt}
\setlength{\parskip}{5pt}

%
% Create Problem Sections
%

\newcommand{\enterProblemHeader}[1]{
    \nobreak\extramarks{}{Problem \arabic{#1} continued on next page\ldots}\nobreak{}
    \nobreak\extramarks{Problem \arabic{#1} (continued)}{Problem \arabic{#1} continued on next page\ldots}\nobreak{}
}

\newcommand{\exitProblemHeader}[1]{
    \nobreak\extramarks{Problem \arabic{#1} (continued)}{Problem \arabic{#1} continued on next page\ldots}\nobreak{}
    \stepcounter{#1}
    \nobreak\extramarks{Problem \arabic{#1}}{}\nobreak{}
}

\setcounter{secnumdepth}{0}
\newcounter{partCounter}
\newcounter{homeworkProblemCounter}
\setcounter{homeworkProblemCounter}{1}
\nobreak\extramarks{Problem \arabic{homeworkProblemCounter}}{}\nobreak{}

%
% Homework Problem Environment
%
% This environment takes an optional argument. When given, it will adjust the
% problem counter. This is useful for when the problems given for your
% assignment aren't sequential. See the last 3 problems of this template for an
% example.
%
\newenvironment{homeworkProblem}[1][-1]{
    \ifnum#1>0
        \setcounter{homeworkProblemCounter}{#1}
    \fi
    \section{Problem \arabic{homeworkProblemCounter}}
    \setcounter{partCounter}{1}
    \enterProblemHeader{homeworkProblemCounter}
}{
    \exitProblemHeader{homeworkProblemCounter}
}

%
% Homework Details
%   - Title
%   - Due date
%   - Class
%   - Section/Time
%   - Instructor
%   - Author
%

\newcommand{\hmwkTitle}{Homework 2}
\newcommand{\hmwkDueDate}{Nov 26, 2024}
\newcommand{\hmwkClass}{MATH 264A}
\newcommand{\hmwkClassInstructor}{Professor Warnke}
\newcommand{\hmwkAuthorName}{\textbf{Ray Tsai}}
\newcommand{\hmwkPID}{A16848188}

%
% Title Page
%

\title{
    \vspace{2in}
    \textmd{\textbf{\hmwkClass:\ \hmwkTitle}}\\
    \normalsize\vspace{0.1in}\small{Due\ on\ \hmwkDueDate\ at 23:59pm}\\
    \vspace{0.1in}\large{\textit{\hmwkClassInstructor}} \\
    \vspace{3in}
}

\author{
  \hmwkAuthorName \\
  \vspace{0.1in}\small\hmwkPID
}
\date{}

\renewcommand{\part}[1]{\textbf{\large Part \Alph{partCounter}}\stepcounter{partCounter}\\}

%
% Various Helper Commands
%

% Useful for algorithms
\newcommand{\alg}[1]{\textsc{\bfseries \footnotesize #1}}

% For derivatives
\newcommand{\deriv}[1]{\frac{\mathrm{d}}{\mathrm{d}x} (#1)}

% For partial derivatives
\newcommand{\pderiv}[2]{\frac{\partial}{\partial #1} (#2)}

% Integral dx
\newcommand{\dx}{\mathrm{d}x}

% Probability commands: Expectation, Variance, Covariance, Bias
\newcommand{\Var}{\mathrm{Var}}
\newcommand{\Cov}{\mathrm{Cov}}
\newcommand{\Bias}{\mathrm{Bias}}
\newcommand*{\Z}{\mathbb{Z}}
\newcommand*{\Q}{\mathbb{Q}}
\newcommand*{\R}{\mathbb{R}}
\newcommand*{\C}{\mathbb{C}}
\newcommand*{\N}{\mathbb{N}}
\newcommand*{\prob}{\mathds{P}}
\newcommand*{\E}{\mathds{E}}

\begin{document}

\maketitle

\pagebreak

\begin{homeworkProblem}
  This problem illustrates how variants of the usual switching arguments can sometimes lead to
  fairly simple proofs (if one is content with bounds instead of asymptotics). Let $\mathcal{N}_i$
  denote the set of permutations of $[n]$ with exactly $i$ fixed points. Picking a random
  permutation $\pi_n \in S_n$ uniformly at random, below we give a simple and robust proof of the
  non-trivial bound
  \[
    \mathbb{P}(\pi_n \text{ has no fixed points}) \geq \frac{1}{3} - o(1) \quad \text{as } n \to \infty
  \]

\begin{enumerate}[(a)]
    \item Using a basic switching argument, show that $\lvert \mathcal{N}_1 \rvert / \lvert
    \mathcal{N}_0 \rvert \leq 1 + O(1/n)$.

    \begin{proof}
      Define switching operation $\phi: \mathcal{N}_1 \to 2^{\mathcal{N}_0}$ by sending $\pi \in
      \mathcal{N}_1$ to the set of all possible permutations $\pi'$ obtained by switching the fixed
      point $k$ with any other element in $[n] \backslash \{k\}$. 

      \textbf{Forward Switching:}
      
      Let $\pi \in \mathcal{N}_1$ with $\pi(k) = k$. Notice if permutation $\pi'$ is
      defined by
      \[
        \pi'(i) = \begin{cases}
          \pi(j) & \text{if } i = k \\
          \pi(k) = k & \text{if } i = j \\
          \pi(i) & \text{otherwise}
        \end{cases}
      \] 
      for some $j \neq k$, then $\pi' \in \mathcal{N}_0$. Since there are $n - 1$ choices for $j$,
      we have $\lvert \phi(\pi) \rvert = n - 1$. 

      \textbf{Reverse Switching:}

      Number of ways $\pi' \in \mathcal{N}_0$ can be obtained by switching some $\pi \in
      \mathcal{N}_1$ is
      \begin{align*}
        \lvert \phi^{-1}(\pi') \rvert  = (\# \text{choices for fixed point $k$})(\# \text{valid points in $\pi'$ to swap with $k$})
      \end{align*}
      There are $n$ choices for $k$. Notice that we need to swap $k$ with $\pi'(k)$ to possibly
      obtain a permutation with $k$ as its only fixed point. If $\pi'(k) = j$ and $\pi'(j) = k$,
      then swapping $k$ with $j$ will yield a permutation with both $k$ and $j$ as fixed points.
      Hence, the number of valid points to swap with $k$ is $\leq 1$, and thus $\lvert
      \phi^{-1}(\pi') \rvert \leq n$.

      \textbf{Double Counting:}

      \[
        \sum_{\pi \in \mathcal{N}_1} \lvert \phi(\pi) \rvert = \sum_{\pi' \in \mathcal{N}_0} \lvert \phi^{-1}(\pi') \rvert \quad \implies \quad \lvert \mathcal{N}_1 \rvert/\lvert \mathcal{N}_0 \rvert  \leq  \frac{n - 1}{n} = 1 + O(1/n).
      \]
    \end{proof}

    \item Using a simple counting argument, show that in a random permutation, the expected number
    of fixed points is equal to one.

    \begin{proof}
      Let $X$ be the number of fixed points in a random permutation $\pi_n \in S_n$, and let $X_i$
      be the indicator for the event that $\pi_n(i) = i$. Then,
      \[
        \E_{\pi_n}[X] = \sum_{i = 1}^{n} \E_{\pi_n}[X_i] = \sum_{i = 1}^{n} \prob(i \text{ is a fixed point}) = \sum_{i = 1}^{n} \frac{\# \text{$\pi \in S_n$ with $\pi(i) = i$}}{|S_n|} = \sum_{i = 1}^n \frac{1}{n} = 1.
      \]
    \end{proof}

    \newpage

    \item By combining these estimates, conclude that 
    \[
      \mathbb{P}(\pi_n \text{ has no fixed points}) = \frac{\lvert \mathcal{N}_0 \rvert}{n!} \geq \frac{1}{3} - O(1/n).
    \]
    \begin{proof}
      Since $\sum_{i = 0}^n \frac{\lvert \mathcal{N}_i \rvert}{n!} = 1$, combining (b) we have
      \[
        2\left(1 - \frac{\lvert \mathcal{N}_0 \rvert}{n!} - \frac{\lvert \mathcal{N}_1 \rvert}{n!}\right) = 2\sum_{i = 2}^n \frac{\lvert \mathcal{N}_i \rvert}{n!} \leq \sum_{i = 2}^{n} i \cdot \frac{\lvert \mathcal{N}_i \rvert}{n!} = \E_{\pi_n}[X] - \frac{|\mathcal{N}_1|}{n!} = 1 - \frac{|\mathcal{N}_1|}{n!}.
      \]
      Rearranging yields
      \[
        2 \cdot \frac{\lvert \mathcal{N}_0 \rvert}{n!} \geq 1 - \frac{\lvert \mathcal{N}_1 \rvert}{n!}
      \]
      It now follows $\lvert \mathcal{N}_1 \rvert \leq (1 + O(1/n))\lvert \mathcal{N}_0 \rvert$ from
      (a) that
      \[
        2 \cdot \frac{\lvert \mathcal{N}_0 \rvert}{n!} \geq 1 - (1 + O(1/n))\frac{\lvert \mathcal{N}_0 \rvert}{n!} \quad \implies \quad \frac{\lvert \mathcal{N}_0 \rvert}{n!} \geq \frac{1}{3} - O(1/n).
      \]
    \end{proof}
\end{enumerate}
\end{homeworkProblem}

\end{document}