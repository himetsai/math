\documentclass{article}

\usepackage{fancyhdr}
\usepackage{extramarks}
\usepackage{amsmath}
\usepackage{amsthm}
\usepackage{amsfonts}
\usepackage{tikz}
\usepackage[plain]{algorithm}
\usepackage{algpseudocode}
\usepackage{enumerate}
\usepackage{amssymb}
\usepackage[margin=1in]{geometry}

\newcommand{\st}{~\mid~}
\newcommand{\ind}{$~~~$}
\usepackage{xcolor}

\graphicspath{ {./../images} }

\usetikzlibrary{automata,positioning}

%
% Basic Document Settings
%

\topmargin=-0.45in
\evensidemargin=0in
\oddsidemargin=0in
\textwidth=6.5in
\textheight=9.0in
\headsep=0.25in

\linespread{1.1}

\pagestyle{fancy}
\lhead{\hmwkAuthorName}
\chead{\hmwkClass:\ \hmwkTitle}
\rhead{\firstxmark}
\lfoot{\lastxmark}
\cfoot{\thepage}

\renewcommand\headrulewidth{0.4pt}
\renewcommand\footrulewidth{0.4pt}

\setlength\parindent{0pt}
\setlength{\parskip}{5pt}

%
% Create Problem Sections
%

\newcommand{\enterProblemHeader}[1]{
    \nobreak\extramarks{}{Problem \arabic{#1} continued on next page\ldots}\nobreak{}
    \nobreak\extramarks{Problem \arabic{#1} (continued)}{Problem \arabic{#1} continued on next page\ldots}\nobreak{}
}

\newcommand{\exitProblemHeader}[1]{
    \nobreak\extramarks{Problem \arabic{#1} (continued)}{Problem \arabic{#1} continued on next page\ldots}\nobreak{}
    \stepcounter{#1}
    \nobreak\extramarks{Problem \arabic{#1}}{}\nobreak{}
}

\setcounter{secnumdepth}{0}
\newcounter{partCounter}
\newcounter{homeworkProblemCounter}
\setcounter{homeworkProblemCounter}{1}
\nobreak\extramarks{Problem \arabic{homeworkProblemCounter}}{}\nobreak{}

%
% Homework Problem Environment
%
% This environment takes an optional argument. When given, it will adjust the
% problem counter. This is useful for when the problems given for your
% assignment aren't sequential. See the last 3 problems of this template for an
% example.
%
\newenvironment{homeworkProblem}[1][-1]{
    \ifnum#1>0
        \setcounter{homeworkProblemCounter}{#1}
    \fi
    \section{Problem \arabic{homeworkProblemCounter}}
    \setcounter{partCounter}{1}
    \enterProblemHeader{homeworkProblemCounter}
}{
    \exitProblemHeader{homeworkProblemCounter}
}

%
% Homework Details
%   - Title
%   - Due date
%   - Class
%   - Section/Time
%   - Instructor
%   - Author
%

\newcommand{\hmwkTitle}{Homework\ \#4}
\newcommand{\hmwkDueDate}{May 9, 2024}
\newcommand{\hmwkClass}{CSE 101}
\newcommand{\hmwkClassInstructor}{Professor Jones}
\newcommand{\hmwkAuthorName}{\textbf{Ray Tsai}}
\newcommand{\hmwkPID}{A16848188}

%
% Title Page
%

\title{
    \vspace{2in}
    \textmd{\textbf{\hmwkClass:\ \hmwkTitle}}\\
    \normalsize\vspace{0.1in}\small{Due\ on\ \hmwkDueDate\ at 23:59pm}\\
    \vspace{0.1in}\large{\textit{\hmwkClassInstructor}} \\
    \vspace{3in}
}

\author{
  \hmwkAuthorName \\
  \vspace{0.1in}\small\hmwkPID
}
\date{}

\renewcommand{\part}[1]{\textbf{\large Part \Alph{partCounter}}\stepcounter{partCounter}\\}

%
% Various Helper Commands
%

% Useful for algorithms
\newcommand{\alg}[1]{\textsc{\bfseries \footnotesize #1}}

% For derivatives
\newcommand{\deriv}[1]{\frac{\mathrm{d}}{\mathrm{d}x} (#1)}

% For partial derivatives
\newcommand{\pderiv}[2]{\frac{\partial}{\partial #1} (#2)}

% Integral dx
\newcommand{\dx}{\mathrm{d}x}

% Probability commands: Expectation, Variance, Covariance, Bias
\newcommand{\Var}{\mathrm{Var}}
\newcommand{\Cov}{\mathrm{Cov}}
\newcommand{\Bias}{\mathrm{Bias}}
\newcommand*{\Z}{\mathbb{Z}}
\newcommand*{\Q}{\mathbb{Q}}
\newcommand*{\R}{\mathbb{R}}
\newcommand*{\C}{\mathbb{C}}
\newcommand*{\N}{\mathbb{N}}
\newcommand*{\prob}{\mathds{P}}
\newcommand*{\E}{\mathds{E}}

\begin{document}

\maketitle

\pagebreak

\begin{homeworkProblem}
  You are given a connected graph $G$ with $n$ vertices and with positive edge weights $w: E \to
  \R^+$. You wish to find a connected subgraph of $G$ with $n-1$ vertices that has minimum total
  cost.

  \begin{proof}

    Consider the following algorithm on $G$:

    Set the cost of $T$ to $\infty$. For each vertex $v \in V$, run Kruskal's algorithm on $G - v$
    and get $T_v$. If $T_v$ has exactly $n$ edges $cost(T_v) \leq cost(T)$, update $T$ to $T_v$.
    After all iterations, return $T$.

    \textbf{Runtime Analysis: }

    Running Kruskal takes $O(|E|\log |V|)$ time. Note that we don't actually need to remove $v$ from
    $G$, as we may just ignore $v$ while running Kruskal. Checking the the connectedness and and the
    minimality of $T_v$ only takes constant time. Hence, each iteration takes $O(|E|\log |V|)$ time.
    Since the algorithm runs Kruskal's algorithm $|V|$ times, it takes $O(|V||E|\log |E|)$ time in
    total.

    \textbf{Justification for correctness: }

    We show that $T$ is the  MST of $n - 1$ vertices with minimum total cost. 
    
    We first show that $T = T_v$ for some $v$. In particular, we need to show that $T$ is guaranteed
    to be updated. Note that $G$ is a connected graph, which contains a spanning tree $S$. Remove
    any leaf $u$ from $S$ yields a spanning tree of $n - 1$ vertices, which ensures that $G$
    contains a connected subgraph with $n - 1$ vertices. But then running Kruskal on $G - u$ gives a
    spanning tree on $n - 1$ vertices, and this guarantees $T$ to be updated to some $T_u$.
    Additionally, $T$ is only updated to $T_v$ if $T_v$ has $n$ edges. Since Kruskal ensures $T_v$
    has no cycles, $T_v$ is a connected tree of $n - 1$ vertices. Therefore, $T$ is a MST of some $n
    - 1$ vertices at the end of the algorithm.

    Now we show that $T$ is optimal. Let $T'$ be a connected subgraph of $G$ of $n - 1$ vertices
    such that $cost(T') < cost(T)$. Say $T = T_v$ is a MST on $V \backslash \{v\}$ and $T'$ is a
    subgraph on $V \backslash \{u\}$, for some $v, u \in V$. Since Kruskal gives an MST $T_u$ on $G
    - u$, we know $cost(T_u) \leq cost(T')$. But then the algorithm picked $T_v$ over $T_u$, so
    \[
      cost(T') > cost(T_v) \geq cost(T_u) \geq cost(T'),
    \]
    contradiction. Hence, $T$ is optimal.
  \end{proof}
\end{homeworkProblem}

\newpage

\begin{homeworkProblem}
  Suppose you are managing a computer network of $n$ computing sites. Between some pairs of
  computing sites, there is a link that has a positive initialization time. When you turn on the
  entire network, all links start initializing at the same time. The whole network is not
  operational until all links have been initialized. You wish to remove links so that the network
  stays connected yet you have minimized the maximum initializion time.

  The network is given to you as a connected undirected graph with positive edge weights. You can
  assume that $|E|=O(|V|)$.

  Design an algorithm that achieves this goal.

  \begin{proof}
    Consider running Kruskal's algorithm on the network graph and return the resulting MST $T$.

    We already know Kruskal's algorithm takes $O(|E|\log |V|)$ time. But since $|E| = O(V)$, this
    algorithm only takes $O(|V|\log |V|)$ time.

    We now give a justification for the correctness of the algorithm, particularly the optimality of
    $T$. Let $e$ be the last edge added to $T$. Say $e$ has weight $w(e)$. We know $e$ is the
    heaviest edge in $T$. Consider $T - \{e\}$. Since $T$ is a tree, $T - \{e\}$ has two components,
    which splits the vertex set $V$ into two subsets, say $X$ and $V - X$. Note that by the choice
    of Kruskal's algorithm, $e$ is the lightest edge which connects $X$ and $V - X$. But then any
    spanning graph of $G$ must contain an edge which connects $X$ and $V - X$. Hence, $w(e)$ is a
    lower bound of initialization time of any connected network, which is achieved by $T$. 
  \end{proof}
\end{homeworkProblem}

\newpage

\begin{homeworkProblem}
  You are given two sets of $n$ points each on the number line: $(A[1],\dots, A[n])$ and
  $(B[1],\dots,B[n])$ (all values are different and each list is sorted in increasing order.)

  You wish to pair up the points (one from the first list and one from the second list):

  $$[(A[1],B[i_1]), (A[2],B[i_2]),\dots, (A[n], B[i_n])]$$ 
  
  (such that each point is in exactly one pair.)

  You wish to pair them up in such a way to minimize:

  $$\sum_{k=1}^n |A[k] - B[i_k]|$$

  \begin{enumerate}[(a)]
  \item
  Describe this problem as we have done in class in terms of:
  \begin{itemize}
  \item
  {\bf Input:} Arrays $A, B$ of size $n$ in increasing order and all values are distinct.
  \item
  {\bf Solution Format:} $[(A[1],B[i_1]), (A[2],B[i_2]),\dots, (A[n], B[i_n])]$. 
  \item
  {\bf Constraints:} $i_{\alpha} \neq i_{\beta}$ if $\alpha \neq \beta$.
  \item
  {\bf Objective:} Minimize cost value $\sum_{k=1}^n |A[k] - B[i_k]|$.
  \end{itemize}

  \item
  {\bf Candidate Greedy Strategy I:} Find the pair $A[i],B[j]$ that is the closest (with the
  smallest overall $|A[i] - B[j]|$ distance) and pair $(A[i],B[j])$ (break ties by choosing the
  smaller value of $A[i]$.) Remove $A[i]$ from the first list and remove $B[j]$ from the second list
  and repeat on the remaining points until all points are paired.

  Either prove that this strategy always yields an optimal solution or give a counterexample to show
  that it is not always optimal.

  \begin{proof}
    Consider $A = [1, 4]$, $B = [-3, 2]$. The algorithm would first pair up $1$ and $2$, then $4$
    and $-3$. Hence, the algorithm would return $x = [(1, 2), (4, -3)]$, which has a cost value of
    $8$. But then there exists a pairing $y = [(1, -3), (4, 2)]$, which has a cost value of $6$.
    Hence, this strategy is not optimal.
  \end{proof}

  \item
  {\bf Candidate Greedy Strategy II:}

  Pair up $(A[1],B[1]), (A[2],B[2]), \dots, (A[n],B[n])$.

  Either prove that this strategy always yields an optimal solution or give a counterexample to show
  that it is not always optimal.

  \begin{proof}
    This algorithm is correct. Given input $A, B$, suppose $\sigma$ be a non-trivial permutation of
    $B$. We show that there exists a permutation $\sigma'$ such that
    \[
      \sum_{k=1}^n |A[k] - B[\sigma'(k)]| < \sum_{k=1}^n |A[k] - B[\sigma(k)]|.
    \]
    Since $\sigma$ is non-trivial, there exists some $i$ such that $\sigma(i) \neq i$. Let $k$ be
    the smallest such index. We know $\sigma(k) > k$. Let $l = \sigma^{-1}(k)$. Since $k$ is the
    smallest index which deviates, $l$ must be larger than $k$. 
    
    We now show that
    \begin{gather}
      |A[k] - B[\sigma(k)]| + |A[l] - B[k]| \geq |A[k] - B[k]| + |A[l] - B[\sigma(k)]|.
    \end{gather}
    Since $A[k] < A[l]$ and $B[k] < B[\sigma(k)]$, there are \textit{only} 6 possible cases:

    \textbf{Case 1: } $A[k] \leq A[l] \leq B[k] \leq B[\sigma(k)]$.
    \[
      (B[\sigma(k)] - A[k]) + (B[k] - A[l]) = (B[k] - A[k]) + (B[\sigma(k)] - A[l]).
    \]
    
    \textbf{Case 2: } $A[k] \leq B[k] \leq A[l] \leq B[\sigma(k)]$.
    \begin{align*}
      (B[\sigma(k)] - A[k]) + (A[l] - B[k]) 
      &= (B[k] - A[k]) + (B[\sigma(k)] - A[l]) + 2(A[l] - B[k]) \\
      &\geq (B[k] - A[k]) + (B[\sigma(k)] - A[l]) 
    \end{align*}
    

    \textbf{Case 3: } $A[k] \leq B[k] \leq B[\sigma(k)] \leq A[l]$.
    \begin{align*}
      (B[\sigma(k)] - A[k]) + (A[l] - B[k]) 
      &= (B[k] - A[k]) + (A[l] - B[\sigma(k)]) + 2(A[l] - B[\sigma(k)]) \\
      &\geq (B[k] - A[k]) + (B[\sigma(k)] - A[l]) 
    \end{align*}

    \textbf{Case 4: } $B[k] \leq A[k] \leq A[l] \leq B[\sigma(k)]$.
    \begin{align*}
      (B[\sigma(k)] - A[k]) + (A[l] - B[k]) 
      &= (A[k] - B[k]) + (B[\sigma(k)] - A[l]) + 2(A[l] - A[k]) \\
      &\geq (B[k] - A[k]) + (B[\sigma(k)] - A[l]) 
    \end{align*}

    \textbf{Case 5: } $B[k] \leq A[k] \leq B[\sigma(k)] \leq A[l]$.
    \begin{align*}
      (B[\sigma(k)] - A[k]) + (A[l] - B[k]) 
      &= (A[k] - B[k]) + (A[l] - B[\sigma(k)]) + 2(B[\sigma(k)] - A[k]) \\
      &\geq (B[k] - A[k]) + (B[\sigma(k)] - A[l]) 
    \end{align*}

    \textbf{Case 6: } $B[k] \leq B[\sigma(k)] \leq A[k] \leq A[l]$.
    \begin{align*}
      (A[k] - B[\sigma(k)]) + (A[l] - B[k]) &= (A[k] - B[k]) + (A[l] - B[\sigma(k)])
    \end{align*}
    Hence, let $\sigma'$ be $\sigma$ after swapping the values of $\sigma(k)$ and $\sigma(l)$.
    By (1),
    \begin{align*}
      \sum_{i=1}^n &|A[i] - B[\sigma(i)]| - \sum_{i=1}^n |A[i] - B[\sigma'(i)]| \\
      &= |A[k] - B[\sigma(k)]| + |A[l] - B[k]| - |A[k] - B[k]| - |A[l] - B[\sigma(k)]| > 0,
    \end{align*}
    and thus $\sigma'$ achieves a lower cost value.

    We now show that the greedy solution is optimal for any input of size $n \geq 1$ by induction on
    $n$. The base case is trivial, as there is only one possible solution. Suppose $n \geq 2$.
    Assume that the greedy strategy is optimal for all inputs of size less than $n$. Consider an
    input $A, B$ of size $n$. Let $OS$ be any arbitrary solution of $A, B$. We may assume that
    $OS(A, B)$ mismatches some terms of $A, B$, otherwise $OS = GS$ and we are done. Say the first
    mismatch in $OS(A, B)$ happens at index $k$, with $A[k]$ paired with $B[k']$ and $A[l]$ paired
    with $B[k]$, for some $l, k' > k$. By our claim at the start, there exists a better solution
    $OS'$ which agrees with all terms of $OS(A, B)$ except that it swaps the first mismatch of $OS$.
    Let $A' = A \backslash \{A[k]\}$ and $B' = B \backslash \{B[k]\}$. Define
    \[
      C([(A[1],B[i_1]), \dots, (A[n], B[i_n])]) = \sum_{k=1}^n |A[k] - B[i_k]|.
    \]
    By induction,
    \begin{align*}
      C(OS(A, B)) > C(OS'(A, B)) 
      &= C(\{(A[k], B[k])\} \cup S(A', B')) \\
      &\geq C(\{(A[k], B[k])\} \cup GS(A', B')) \\
      &= C(GS(A, B)).
    \end{align*}
  \end{proof}

  \item
  {\bf Candidate Greedy Strategy III:} Let $B[j]$ be the closest point to $A[1]$ in the $B$ list
  (break ties by choosing the lower $B$ value). Pair up $(A[1],B[j])$ and remove $A[1]$ and $B[j]$
  from the lists and continue with $A[2]$ until all points are paired.

  Either prove that this strategy always yields an optimal solution or give a counterexample to show
  that it is not always optimal.

  \begin{proof}
    Consider $A = [1, 4]$, $B = [-3, 2]$. The algorithm would first pair up $1$ and $2$, then $4$
    and $-3$. Hence, the algorithm would return $x = [(1, 2), (4, -3)]$, which has a cost value of
    $8$. But then there exists a pairing $y = [(1, -3), (4, 2)]$, which has a cost value of $6$.
    Hence, this strategy is not optimal.
  \end{proof}
  \end{enumerate}
\end{homeworkProblem}

\newpage

\begin{homeworkProblem}
  Suppose you are driving along a road in an electric car. The battery of the electric car can bring
  you $x[0]$ miles. There are battery stations along the way at positive positions $D[1],\dots D[n]$
  (in sorted order.) Each battery station can  \emph{replace} your battery and give you a new
  battery that can bring you a certain number of miles. The distances of the batteries are given in
  the array $x[0],x[1],\dots,x[n]$

  You wish to start at position $0$ with a full battery and end at position $D[n]$ replacing the
  fewest batteries.

  \begin{enumerate}[(a)]
  \item
  Describe this problem as we have done in class in terms of:
  \begin{itemize}
  \item
  {\bf Input:} A sorted list of $n$ battery station positions and a list of $n + 1$ distances of the
  batteries. 
  \item
  {\bf Solution Format:} An increasing list of indices $X = [x_1, \dots, x_k]$, where $1 \leq x_i \leq n$.
  \item
  {\bf Constraints:} Let $x_0 = 0$. For all $x_i \in X$,
  \[
    x[x_{i - 1}] + D[x_{i - 1}] \geq D[x_i].
  \]
  \item
  {\bf Objective:} Reach $D[n]$ and minimize $|X|$.
  \end{itemize}

  \item
  {\bf Candidate Greedy Strategy I:} Travel to the farthest battery station without exceeding $x[0]$
  miles. Replace the battery at that station and repeat the process starting from that station until
  you can reach position $D[n]$.

  Either prove that this strategy always yields an optimal solution or give a counterexample to show
  that it is not always optimal.

  \begin{proof}
    Consider $D = [0, 1, 3, 6]$, $x = [3, 5, 1, 0]$. The strategy would traval to station 2 first
    and replace to a battery with distance $x[2] = 1$. But then the next station is $3$ unit
    distances away, so the strategy fails to arrive at the destination. However, there exists
    solution $X = \{1, 3\}$ which could reach the destination, and thus the strategy is not optimal.
  \end{proof}

  \item
  {\bf Candidate Greedy Strategy II:}

  Travel to the battery station with the largest $x[i]$ value without exceeding $x[0]$ miles.
  Replace the battery at that station and repeat the process starting from that station until you
  can reach position $D[n]$ and then go directly there.

  Either prove that this strategy always yields an optimal solution or give a counterexample to show
  that it is not always optimal.

  \begin{proof}
    Consider $D = [0, 1, 3, 4]$ and $x = [3, 2, 1, 0]$. The strategy would first stop at station 1
    then station 2 and finally arrive at the destination. However, there exists solution $X = \{2,
    3\}$, which reaches the destination with lesser stops, so this strategy is not optimal.
  \end{proof}

  \break

  \item
  {\bf Candidate Greedy Strategy III:} 

  Travel to the battery station with the largest $D[i] + x[i]$ value (in other words, the battery
  that can take you the farthest down the road.) Replace the battery at that station and repeat the
  process starting from that station until you can reach position $D[n]$ and then go directly there.

  Either prove that this strategy always yields an optimal solution or give a counterexample to show
  that it is not always optimal.

  \begin{proof}
    Given input $D$ and $x$ where $D$ is increasing, Let $OS = [y_1, \dots, y_{m}]$ be some
    solution which contains some index $y_i$ such that station $y_i$ does not have the largest
    $D[y_i] + x[y_i]$ value reachable from station $y_{i - 1}$. We show that there exists solution
    $OS' = [z_1, \dots, z_{k}]$ such that $z_{k} \geq y_{m}$ and $k \leq m$.

    Let $y_i$ be the smallest index in $OS$ such that station $y_i$ does not have the largest
    $D[y_i] + x[y_i]$ value reachable from station $y_{i - 1}$. Construct $OS' = [z_1, \dots,
    z_{k}]$ such that $y_j = z_j$ for all $j \neq i$ and make $z_i$ the station with the largest
    $D[s] + x[s]$ value reachable from station $z_{i - 1}$. Replace all stations in $OS'$ after
    $z_i$ with $n$ if $D[z_i] + x[z_i] \geq D[n]$. Note that since $D[z_i] + x[z_i] > D[y_i] +
    x[y_i]$, station $y_{i + 1}$ is reachable from station $z_i$, and thus $OS'$ is a valid
    solution. 
    
    If $D[z_i] + x[z_i] \geq D[n]$, then obviously $z_{k} \geq y_{m}$ and $k \leq m$. Otherwise,
    $OS$ and $OS'$ agree with all terms except for the $i$th one, so obviously $z_{k} \geq y_{m}$
    and $k \leq m$. Hence, $OS'$ is at least as good a solution as $OS$.

    We now show that for any input of size $n \geq 2$, the greedy solution is optimal. When $n = 2$,
    there is only one possible solution, so the base case is done. Suppose $n \geq 3$. Assume that
    the greedy straregy is optimal for all valid inputs of size less than $n$. Given input $D$ and
    $x$ of size $n$, let $OS = [y_1, \dots, y_{m}]$ be any solution. We may assume that some
    stations in $OS$ does not have the station with the largest $D + x$ value reachable from the
    previous station, otherwise $OS = GS$ and we are done. Let $y_i$ be the first such. The argument
    at the start yields a just as good or better solution $OS' = [z_1, \dots, z_{k}]$. Let $D'$ be
    the list of the first $z_i + 1$ entries of $D$ and let $D''$ be the list such that $D''[k] = 0$
    and $D''[k] = D[z_i + k] - D[z_i + k - 1]$ for $1 \leq k \leq n - z_i$. Additionally, let $x'$
    be the list of the first $z_i + 1$ entries of $x$ and let $x''$ be the rest of $x$. Then, by the
    construction of $OS'$ and induction,
    \begin{align*}
      |OS(D, x)| \geq |OS'(D, x)| 
      &= |GS(D', x') \circ S(D'', x'')| \\
      &\geq |GS(D', x') \circ GS(D'', x'')| \\
      &= |GS(D, x)|.
    \end{align*}
    Define $last(L)$ to be the last element of a list $L$. Then, again by the construction of $OS'$
    and induction
    \begin{align*}
      last(OS(D, x)) \leq last(OS'(D, x))
      &= last(S(D'', x'')) + z_i \\
      &\leq last(GS(D'', x'')) + z_i \\
      &= last(GS(D, x)).
    \end{align*}
  \end{proof}
  \end{enumerate}
\end{homeworkProblem}
\end{document}