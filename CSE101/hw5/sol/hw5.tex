\documentclass{article}

\usepackage{fancyhdr}
\usepackage{extramarks}
\usepackage{amsmath}
\usepackage{amsthm}
\usepackage{amsfonts}
\usepackage{tikz}
\usepackage[plain]{algorithm}
\usepackage{algpseudocode}
\usepackage{enumerate}
\usepackage{amssymb}
\usepackage[margin=1in]{geometry}

\newcommand{\st}{~\mid~}
\newcommand{\ind}{$~~~$}
\usepackage{xcolor}

\graphicspath{ {./../images} }

\usetikzlibrary{automata,positioning}

%
% Basic Document Settings
%

\topmargin=-0.45in
\evensidemargin=0in
\oddsidemargin=0in
\textwidth=6.5in
\textheight=9.0in
\headsep=0.25in

\linespread{1.1}

\pagestyle{fancy}
\lhead{\hmwkAuthorName}
\chead{\hmwkClass:\ \hmwkTitle}
\rhead{\firstxmark}
\lfoot{\lastxmark}
\cfoot{\thepage}

\renewcommand\headrulewidth{0.4pt}
\renewcommand\footrulewidth{0.4pt}

\setlength\parindent{0pt}
\setlength{\parskip}{5pt}

%
% Create Problem Sections
%

\newcommand{\enterProblemHeader}[1]{
    \nobreak\extramarks{}{Problem \arabic{#1} continued on next page\ldots}\nobreak{}
    \nobreak\extramarks{Problem \arabic{#1} (continued)}{Problem \arabic{#1} continued on next page\ldots}\nobreak{}
}

\newcommand{\exitProblemHeader}[1]{
    \nobreak\extramarks{Problem \arabic{#1} (continued)}{Problem \arabic{#1} continued on next page\ldots}\nobreak{}
    \stepcounter{#1}
    \nobreak\extramarks{Problem \arabic{#1}}{}\nobreak{}
}

\setcounter{secnumdepth}{0}
\newcounter{partCounter}
\newcounter{homeworkProblemCounter}
\setcounter{homeworkProblemCounter}{1}
\nobreak\extramarks{Problem \arabic{homeworkProblemCounter}}{}\nobreak{}

%
% Homework Problem Environment
%
% This environment takes an optional argument. When given, it will adjust the
% problem counter. This is useful for when the problems given for your
% assignment aren't sequential. See the last 3 problems of this template for an
% example.
%
\newenvironment{homeworkProblem}[1][-1]{
    \ifnum#1>0
        \setcounter{homeworkProblemCounter}{#1}
    \fi
    \section{Problem \arabic{homeworkProblemCounter}}
    \setcounter{partCounter}{1}
    \enterProblemHeader{homeworkProblemCounter}
}{
    \exitProblemHeader{homeworkProblemCounter}
}

%
% Homework Details
%   - Title
%   - Due date
%   - Class
%   - Section/Time
%   - Instructor
%   - Author
%

\newcommand{\hmwkTitle}{Homework\ \#5}
\newcommand{\hmwkDueDate}{May 16, 2024}
\newcommand{\hmwkClass}{CSE 101}
\newcommand{\hmwkClassInstructor}{Professor Jones}
\newcommand{\hmwkAuthorName}{\textbf{Ray Tsai, Kevin Yu}}

%
% Title Page
%

\title{
    \vspace{2in}
    \textmd{\textbf{\hmwkClass:\ \hmwkTitle}}\\
    \normalsize\vspace{0.1in}\small{Due\ on\ \hmwkDueDate\ at 23:59pm}\\
    \vspace{0.1in}\large{\textit{\hmwkClassInstructor}} \\
    \vspace{3in}
}

\author{
  \hmwkAuthorName
}
\date{}

\renewcommand{\part}[1]{\textbf{\large Part \Alph{partCounter}}\stepcounter{partCounter}\\}

%
% Various Helper Commands
%

% Useful for algorithms
\newcommand{\alg}[1]{\textsc{\bfseries \footnotesize #1}}

% For derivatives
\newcommand{\deriv}[1]{\frac{\mathrm{d}}{\mathrm{d}x} (#1)}

% For partial derivatives
\newcommand{\pderiv}[2]{\frac{\partial}{\partial #1} (#2)}

% Integral dx
\newcommand{\dx}{\mathrm{d}x}

% Probability commands: Expectation, Variance, Covariance, Bias
\newcommand{\Var}{\mathrm{Var}}
\newcommand{\Cov}{\mathrm{Cov}}
\newcommand{\Bias}{\mathrm{Bias}}
\newcommand*{\Z}{\mathbb{Z}}
\newcommand*{\Q}{\mathbb{Q}}
\newcommand*{\R}{\mathbb{R}}
\newcommand*{\C}{\mathbb{C}}
\newcommand*{\N}{\mathbb{N}}
\newcommand*{\prob}{\mathds{P}}
\newcommand*{\E}{\mathds{E}}

\begin{document}

\maketitle

\pagebreak

\begin{homeworkProblem}
  Consider the following divide and conquer algorithm that claims to find an MST when the input is a
  complete graph $G$ with positive edge weights:

  {\bf Algorithm Description:} Given an undirected complete graph $G=(V,E)$ with positive edge
  weights where $V=[v_1,\dots,v_n]$,
  \begin{itemize}
  \item If $n=1$ then return the empty set of edges. 
  \item
  Otherwise, split the set of vertices into two sets: $V' = [v_1,\dots,v_{\lfloor n/2\rfloor}]$ and
  $V'' = [v_{\lfloor n/2\rfloor}+1,\dots,v_n]$.
  \item
  Create two new graphs $G' = (V',E')$ and $G'' = (V'',E'')$ where $E'\subseteq E$ is the set of
  edges with both endpoints in $V'$ and $E''\subseteq E$ is the set of edges with both endpoints in
  $V''$.
  \item
  Recursively run the algorithm on $G'$ and $G''$ to get $T'$ and $T''$, respectively. Find the
  lightest edge that connects a vertex in $T'$ to a vertex in $T''$ and call that edge $e$.
  \item
  Return $T' \cup T'' \cup \{e\}$.
  \end{itemize}


  Disprove the correctness of this algorithm by giving a counterexample.

  \begin{proof}
    Consider $G = C_4$, where the edge $\{v_3, v_4\}$ has weight $2$ and the remaining edges each
    has weight $1$. The algorithm recurses on subgraph $G''$ with vertex set $V'' = [v_3, v_4]$, so
    the resulting spanning tree $T$ contains the edge $\{v_3, v_4\}$. Since $T$ has $3$ edges with
    an edges of weight $2$, the total cost of $T$ is $4$. But then $\{\{2, v_i\}: i \neq 2\} \subset
    E$ spans $G$ with a total weight of $3$, as it only uses edges of weight 1.
  \end{proof}
\end{homeworkProblem}

\newpage

\begin{homeworkProblem}
  You are given an increasing sequence of integers: $(A[1],A[2],\dots,A[n])$. Design an algorithm
  that determines (returns TRUE or FALSE) if there exists an index $i$ such that $A[i]=i$.

  Your algorithm should run in $O(\log n)$ time.

  \begin{proof}
    We first give a description of the algorithm.

    \textbf{Algorithm Description:}

    Let $l = 1$ and $r = n$. While $l < r$: put $m = \lfloor(l + r)/2\rfloor$. If $A[m] = m$, return
    TRUE. If $A[m] < m$, put $l = m + 1$. Otherwise, put $r = m$. After the loop, if $A[l] = l$,
    return TRUE. Otherwise, return FALSE.

    \textbf{Justification of Correctness:}

    Let $l_k$ and $r_k$ denote the value of $l$ and $r$ at the end of the $k$th iteration of the
    loop, respectively ($0$th iteration means before the loop starts). Notice that $r_{k} \geq r_{k
    + 1} \geq l_{k + 1} \geq l_{k}$, for all $k \geq 0$
    
    We show that for all indices $i < l_k$ and $j > r_k$, $A[i] < i$ and $A[j] > j$ by induction on
    $k \geq 0$. At the start, $l_k = 1$ and $r_k = n$. Hence, no elements are outside the range of
    $l_k$ and $r_k$, and so the base case $k = 0$ is done. 
    
    Suppose $k \geq 1$. Assume that for all indices $i < l_{k - 1}$ and $j > r_{k - 1}$, we have
    $A[i] < i$ and $A[j] > j$. There are three cases:

    \textbf{Case 1: } $A[m] = m$. 

    The loop terminates without changing the values of $l$ and $r$. By induction, $A[i] < i$ and
    $A[j] > j$, for all indices $i < l_{k - 1} = l_k$ and $j > r_{k - 1} = r_k$.

    \textbf{Case 2: } $A[m] < m$. 

    $l_k$ is set to $m + 1$ and $r_k = r_{k - 1}$. By induction, $A[j] > j$ for all $j > r_{k - 1} =
    r_k$, so it remains to show that $A[i] < i$ for all $i \leq m$. Since the sequence of integers
    $(A[1],A[2],\dots,A[n])$ is strictly increasing, we may observe that
    \[
      A[i] \leq A[m] - (m - i),
    \]
    for all $i \leq m$. But then $A[m] - m < 0$, so indeed
    \[
      A[i] \leq A[m] - (m - i) = (A[m] - m) + i < i,
    \]
    for all $i \leq m$.

    \textbf{Case 3: } $A[m] > m$. 

    $r_k$ is set to $m$ and $l_k = l_{k - 1}$. By induction, $A[i] < i$ for all $i < l_{k - 1} =
    l_k$, so it remains to show that $A[j] < j$ for all $j > m$. Since the sequence of integers
    $(A[1],A[2],\dots,A[n])$ is strictly increasing, we may observe that
    \[
      A[j] \geq A[m] + (j - m),
    \]
    for all $j > m$. But then $A[m] - m > 0$, so indeed
    \[
      A[j] \geq A[m] + (j - m) = (A[m] - m) + j > j,
    \]
    for all $j > m$.

    And this completes the induction. Note that the loop breaks half way only if there exists some
    $A[m] = m$ and the algorithm returns TRUE. Now suppose the loop is terminated by the natural
    condition. Since $r_{k} \geq r_{k + 1} \geq l_{k + 1} \geq l_{k}$ for all $k \geq 0$, we must
    have $l = r$. But then by our induction result, $A[i] \neq i$ for all index $i \neq r$. Hence,
    there exists $A[i] = i$ for some $i$ if and only if $A[r] = r$, and the result now follows.

    \textbf{Runtime Analysis:}

    Since every iteration of the loop cuts out half the current list, the loop will iterate at most
    $\log n$ times until $l$ meets $r$, given an input list of size $n$. Checking and updating $l$
    or $r$ only take constant time. Hence, in total, the algorithm has a runtime of $O(\log n)$.
  \end{proof}
\end{homeworkProblem}

\newpage

\begin{homeworkProblem}
  You are given a list of $n$ ordered pairs $[(x_1,f_1),\dots,(x_n,f_n)]$. This list describes a
  list of length $\sum f_i$ that contains $f_1$ copies of the value $x_1$, $f_2$ copies of the value
  $x_2$ and so on.

  You wish to find the median value of this list in expected runtime of $O(n)$. (You can assume that
  $\sum f_i$ is odd.)

  \begin{proof}
    We give a description of the algorithm:

    \textbf{Algorithm Description:}

    Let $\ell([(x_1,f_1),\dots,(x_u,f_u)])$ denote the length of the list described by
    $[(x_1,f_1),\dots,(x_u,f_u)]$, namely $\sum_{i = 1}^u f_i$.

    We first define $Selection(L = [(x_1,f_1),\dots,(x_m,f_m)], k)$, which takes in a list $L$ of
    ordered pairs and an integer $k$, and outputs the $k$th smallest number in the list described in
    $L$:

    If $|L| = 1$, return $x_1$. Otherwise, pick $x_v$ randomly from $L$. Split $L$ into $L_l$,
    $[(x_v, f_v)]$, and $L_r$, where $L_l$ contains all the ordered pairs with $x_i$ less than $x_v$
    and $L_r$ contains the ordered pairs with $x_i$ greater than $x_v$. If $k \leq \ell(L_l)$,
    return $Selection(L_l, k)$. Else, if $k \leq \ell(L_l) + f_v$, return $x_v$. Otherwise, return
    $Selection(L_r, k - \ell(L_l) - f_v)$.

    Now for finding the median value of the list described in $L$, we simply run $Selection(L,
    \lceil \frac{n}{2} \rceil)$.

    \textbf{Runtime Analysis:}

    Since we select the pivot $x_v$ uniformly at random, the input list $L$ will be split into a
    list $L_l$ of length $v - 1$ and a list $L_r$ of length $n - v$. Hence, when we recurse on
    $L_l$, $L_r$, it will take time proportional to $max(v - 1, n - v)$. Note that if
    $\frac{n}{4}\leq v - 1 \leq \frac{3}{4}n$, then $max(v - 1, n - v) \leq \frac{3}{4}n$.
    Otherwise, $\frac{3}{4}n \leq max(v - 1, n - v) < n$. Let $ET(n)$ denote the expected runtime
    for $Selection$ on a list of length $n$. It now follows that
    \[
      ET(n) \leq \frac{1}{2}ET\left(\frac{3}{4}n\right) + \frac{1}{2}ET(n) + cn,
    \]
    where the $cn$ term derived from the splitting process of $L$. But then 
    \[
      ET(n) \leq ET\left(\frac{3}{4}n\right) + cn,
    \]
    and thus
    \[
      ET(n) \in O(n).
    \]
    by the Master Theorem.
  \end{proof}
\end{homeworkProblem}

\newpage

\begin{homeworkProblem}
  \begin{enumerate}[(a)]
    \item
    Let $T(n)$ be the runtime of a divide and conquer algorithm on an input of size $n$. The
    algorithm has $6$ recursive calls each of size $n/4$ and the non-recursive part takes
    $O(n^{1.5})$ time. Use the Master theorem to find the best Big-Oh runtime.

    \begin{proof}
      We first note that
      \[
        T(n) = 6T(n/4) + cn^{1.5}.
      \]
      By the Master Theorem, 
      \[
        T(n) \in O(n^{1.5}),
      \]
      as $6 < 4^{1.5} = 8$.
    \end{proof}
    
    \item
    Let $R(n)$ be the runtime of a divide and conquer algorithm on an input of size $n$. The
    algorithm has $1$ recursive call of size $n/2$ and the non-recursive part takes $O(\log n)$
    time. Find the best Big-Oh runtime.
    \begin{proof}
      We first note that
      \[
        R(n) = R(n/2) + c\log n.
      \]
      Consider the levels of recurrence of this algorithm. Since the algorithm has $1$ recursive
      call of size $n/2$, there are $\log n$ levels of recurrence, with $1$ recursive call per
      level. It now follows that
      \begin{align*}
        R(n) 
        &= R(n/2) + c\log n \\
        &= \left(R(n/4) + c\log \frac{n}{2}\right) + c\log n \\
        &= c\sum_{k = 0}^{\log n} \log \frac{n}{2^k} \\
        &= c\sum_{k = 0}^{\log n} (\log n - k) \\
        &= c\log^2 n - c\sum_{k = 0}^{\log n} k \\
        &= c\log^2 n - \frac{c(\log n + 1)\log n}{2} \\
        &\in O\left(\log^2 n\right).
      \end{align*}
      \end{proof}
    
    \item
    Let $S(n)$ be the runtime of a divide and conquer algorithm on an input of size $n$. The
    algorithm has $2$ recursive calls each of size $2n/3$ and the non-recursive part takes $O(n)$
    time. Find the best Big-Oh runtime.
    \begin{proof}
      We first note that
      \[
        S(n) = 2T(2n/3) + cn.
      \]
      By the Master Theorem, 
      \[
        S(n) \in O(n^{\log_{3/2} 2}) = O(n^{\frac{\log 2}{\log 3 - \log 2}}) \approx O(n^{1.71}),
      \]
      as $2 > 3/2$.
    \end{proof}
    \end{enumerate}
\end{homeworkProblem}
\end{document}