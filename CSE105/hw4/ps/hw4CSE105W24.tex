\documentclass[12pt, oneside]{article}

\usepackage[letterpaper, scale=0.8, centering]{geometry}
\usepackage{fancyhdr}
\setlength{\parindent}{0em}
\setlength{\parskip}{1em}

\pagestyle{fancy}
\fancyhf{}
\renewcommand{\headrulewidth}{0pt}
\rfoot{{\footnotesize Copyright Mia Minnes, 2024, Version \today~(\thepage)}}

\usepackage{titlesec}

\author{CSE105W24}

\newcommand{\instructions}{{\bf For all HW assignments:} Weekly homework 
may be done individually or in groups of up to 3 students. 
You may switch HW partners for different HW assignments. 
Please ensure your name(s) and PID(s) are clearly visible on the first page of your homework submission 
and then upload the PDF to Gradescope. If working in a group, submit only one submission per group: 
one partner uploads the submission through their Gradescope account and then adds the other group member(s) 
to the Gradescope submission by selecting their name(s) in the ``Add Group Members" dialog box. 
You will need to re-add your group member(s) every time you resubmit a new version of your assignment.
 Each homework question will be graded either for correctness (including clear and precise explanations and 
 justifications of all answers) or fair effort completeness. 
 For ``graded for correctness''
 questions: collaboration is allowed only with CSE 105 students in your group; 
 if your group has questions about a problem, you may ask in drop-in 
 help hours or post a private post (visible only to the Instructors) on Piazza.
 For ``graded for completeness''
 questions: collaboration is allowed with any CSE 105 students this quarter; 
 if your group has questions about a problem, you may ask in drop-in 
 help hours or post a public post on Piazza.

All submitted homework for this class must be typed. 
You can use a word processing editor if you like (Microsoft Word, Open Office, Notepad, Vim, Google Docs, etc.) 
but you might find it useful to take this opportunity to learn LaTeX. 
LaTeX is a markup language used widely in computer science and mathematics. 
The homework assignments are typed using LaTeX and you can use the source files 
as templates for typesetting your solutions.
To generate state diagrams of machines, we recommend using Flap.js
or JFLAP. Photographs of clearly hand-drawn diagrams may also be used. We recommend that you
submit early drafts to Gradescope so that in case of any technical difficulties, at least some of your
work is present. You may update your submission as many times as you'd like up to the deadline.


{\bf Integrity reminders}
\begin{itemize}
\item Problems should be solved together, not divided up between the partners. The homework is
designed to give you practice with the main concepts and techniques of the course, 
while getting to know and learn from your classmates.
\item You may not collaborate on homework questions graded for correctness with anyone other than your group members.
You may ask questions about the homework in office hours (of the instructor, TAs, and/or tutors) and 
on Piazza (as private notes viewable only to the Instructors).  
You \emph{cannot} use any online resources about the course content other than the class material 
from this quarter -- this is primarily to ensure that we all use consistent notation and
definitions (aligned with the textbook) and also to protect the learning experience you will have when
the `aha' moments of solving the problem authentically happen.
\item Do not share written solutions or partial solutions for homework with 
other students in the class who are not in your group. Doing so would dilute their learning 
experience and detract from their success in the class.
\end{itemize}

}

\newcommand{\gradeCorrect}{({\it Graded for correctness}) }
\newcommand{\gradeCorrectFirst}{\gradeCorrect\footnote{This means your solution 
will be evaluated not only on the correctness of your answers, but on your ability
to present your ideas clearly and logically. You should explain how you 
arrived at your conclusions, using
mathematically sound reasoning. Whether you use formal proof techniques or 
write a more informal argument
for why something is true, your answers should always be well-supported. 
Your goal should be to convince the
reader that your results and methods are sound.} }
\newcommand{\gradeComplete}{({\it Graded for completeness}) }
\newcommand{\gradeCompleteFirst}{\gradeComplete\footnote{This means you will 
get full credit so long as your submission demonstrates honest effort to 
answer the question. You will not be penalized for incorrect answers. 
To demonstrate your honest effort in answering the question, we 
expect you to include your attempt to answer *each* part of the question. 
If you get stuck with your attempt, you can still demonstrate 
your effort by explaining where you got stuck and what 
you did to try to get unstuck.} }

\usepackage{tikz}
\usetikzlibrary{automata,positioning,arrows}

\usepackage{amssymb,amsmath,pifont,amsfonts,comment,enumerate,enumitem}
\usepackage{currfile,xstring,hyperref,tabularx,graphicx,wasysym}
\usepackage[labelformat=empty]{caption}
\usepackage{xcolor}
\usepackage{multicol,multirow,array,listings,tabularx,lastpage,textcomp,booktabs}

\lstnewenvironment{algorithm}[1][] {   
    \lstset{ mathescape=true,
        frame=tB,
        numbers=left, 
        numberstyle=\tiny,
        basicstyle=\rmfamily\scriptsize, 
        keywordstyle=\color{black}\bfseries,
        keywords={,procedure, div, for, to, input, output, return, datatype, function, in, if, else, foreach, while, begin, end, }
        numbers=left,
        xleftmargin=.04\textwidth,
        #1
    }
}
{}

\newcommand\abs[1]{\lvert~#1~\rvert}
\newcommand{\st}{\mid}

\newcommand{\cmark}{\ding{51}}
\newcommand{\xmark}{\ding{55}}
 
\newcommand{\SUBSTRING}{\textsc{Substring}}
\newcommand{\REP}{\textsc{Rep}}
\newcommand{\blank}{\scalebox{1.5}{\textvisiblespace}}
 
\title{HW4CSE105W24: Homework assignment 4}
\date{Due: February 29th at 5pm (no penalty late submission until 8am next morning), via Gradescope}

\begin{document}
\maketitle
\thispagestyle{fancy}

{\bf In this assignment,}
You will  practice analyzing, designing, and working with Turing machines.
You will use general constructions and specific machines to explore the classes 
of recognizable and decidable languages. 
You will explore various ways to encode machines as strings so that 
computational problems can be recognized.

{\bf Resources}: To review the topics 
for this assignment, see the class material from Weeks 5-7.
We will post frequently asked questions and our answers to them in a 
pinned Piazza post.

{\bf Reading and extra practice problems}:  
Sipser Sections 3.1, 3.3, 4.1
Chapter 3 exercises 3.1, 3.2, 3.5, 3.8.
Chapter 4 exercises 4.1, 4.2, 4.3, 4.4, 4.5.

\instructions

You will submit this assignment via Gradescope
(\href{https://www.gradescope.com}{https://www.gradescope.com}) 
in the assignment called ``hw4CSE105W24''.

{\bf Assigned questions}
\begin{enumerate}[wide, labelwidth=!, labelindent=0pt]


\item\textbf{Classifying languages} (10 points):
Our first example of a more complicated Turing machine was of a Turing machine 
that recognized the language $\{w \# w \mid w \in\{0,1\}^*\}$, which we 
know is not context-free. The language
\[
    \{0^n 1^n 2^n \mid n \geq 0\}
\]
is also not context-free. 

\begin{enumerate}
    \item\gradeCorrectFirst Give an implementation-level description of a Turing machine 
    that recognizes this language.
    \item\gradeCompleteFirst Draw a state diagram of the Turing machine you gave in part (a) and trace
    the computation of this Turing machine on the input $012$. You may use all 
    our usual conventions for state diagrams of Turing machines 
    (we do not include the node for the reject state $qrej$ and any missing transitions 
    in the state diagram have value $(qrej,\square,R)$; 
    $b \to R$ label means $b \to b, R$ ).
\end{enumerate}


\item \textbf{Deciders, Recognizers, Decidability, and Recognizability} (15 points):
For this question, consider the alphabet $\Sigma = \{0,1\}$.
\begin{enumerate}
\item[(a)]\gradeCorrect Give an example of a finite, nonempty language over $\Sigma$ and 
two different Turing machines that recognize it: one that is a decider and one that is not.
A complete solution will include a precise definition for your example language, 
along with {\bf both} a state diagram and an implementation-level description 
of each Turing machines, along with a brief explanation of why each of them recognizes
the language and why one is a decider and there other is not.

\item[(b)]\gradeCorrect True or false: There is a Turing machine that is not a decider that recognizes 
the empty set. A complete solution will include a witness Turing machine (given by 
state diagram or implementation-level description or high-level description) and a justification 
for why it's not a decider and why it does not accept any strings, or a complete and correct
justification for why there is no such Turing machine.

\item[(c)]\gradeCorrect True or false: There is a Turing machine that is not a decider that recognizes 
the set of all string $\Sigma^*$.  A complete solution will include a witness Turing machine 
(given by 
state diagram or implementation-level description or high-level description) and a justification 
for why it's not a decider and why it accept each string over $\{0,1\}$, or a complete and correct
justification for why there is no such Turing machine.
\end{enumerate}

\item\textbf{Closure} (15 points):
Suppose $M$ is a Turing machine over the alphabet $\{0,1\}$. 
Let $s_1, s_2, \ldots$ be a list of all strings in 
$\{0,1\}^*$ in string (shortlex) order.
We define a new Turing machine 
by giving its high-level description as follows: 
\begin{align*}
   M_{new} &= ``\text{On input }w:\\
    &\text{1. For $n = 1, 2, \ldots$}\\
    &\text{2.~~~For $j = 1, 2, \ldots n$} \\
    &\text{3.~~~For $k = 1, 2, \ldots, n$} \\
    &\text{4.~~~~~~Run the computation of $M$ on $s_jws_k$}\\
    &\text{5.~~~~~~If it accepts, accept.}\\
    &\text{6.~~~~~~If it rejects, go to the next iteration of the loop"}\\
\end{align*}

Recall the definitions we have: 
For languages $L_1, L_2$ over the alphabet $\Sigma = \{0,1\}$, we have the 
associated sets of strings
\[
   SUBSTRING(L_1) = \{ w \in \Sigma^* ~|~ \text{there exist } a,b \in \Sigma^* \text{ such that } awb \in L_1\}
\]
and 
\[
   L_1 \circ L_2 = \{ w \in \Sigma^* ~|~ w = uv \text{ for some strings } u \in L_1 \text{ and } v \in L_2 \}
\]
We say that self-set-wise concatenation of the set $L_1$ is $L_1 \circ L_1$.


{\it Note: there was a bug in the version of this assignment 
that was first released.}

\begin{enumerate}
\item[(a)]\gradeComplete Prove that this Turing machine construction 
{\bf cannot} be used to prove that the
class of decidable languages over $\{0,1\}$ is closed under {\bf either} of the above operations
($SUBSTRING$ or self-set-wise concatenation).
A complete answer will give a counterexample or general description why the construction doesn't work for 
both operations.
\item[(b)] \gradeCorrect Prove that this Turing machine construction cannot be used to prove that the
class of recognizable languages over $\{0,1\}$ is closed under the $SUBSTRING$ set operation. In particular, 
give a counterexample of a specific language $L_1$ and Turing machine $M_1$ recognizing it where $M_{new}$ does not
recognize $SUBSTRING(L_1)$.
\item[(c)] \gradeComplete Define a new construction by slightly modifiying this one that can be used to prove  that the
class of recognizable languages over $\{0,1\}$ is closed under $SUBSTRING$. Justify that 
your construction works. The proof of correctness for the closure claim can be structured like: 
``Let $L_1 $ be a recognizable language over $\{0,1\}$ and 
assume we are given a Turing machine $M_1$ so that $L(M_1) = L_1$. Consider the new Turing machine 
$M_{new}$ defined above. We will show that $L(M_{new}) = SUBSTRING(L_1) $... {\it complete the proof
by proving subset inclusion in two directions, by tracing the relevant Turing machine computations}''
\end{enumerate}

\item \textbf{Computational problems} (10 points):
Recall the definitions of some example computational problems from class

\hspace{-30pt}
    \begin{tabular}{|lcl|}
    \hline
    \multicolumn{3}{|l|}{{\bf  Acceptance problem} } \\
    & & \\
    \ldots for DFA & $A_{DFA}$ & $\{ \langle B,w \rangle \mid  \text{$B$ is a  DFA that accepts input 
    string $w$}\}$ \\
    \ldots for NFA & $A_{NFA}$ & $\{ \langle B,w \rangle \mid  \text{$B$ is a  NFA that accepts input 
    string $w$}\}$ \\
    \ldots for regular expressions & $A_{REX}$ & $\{ \langle R,w \rangle \mid  \text{$R$ is a  regular
    expression that generates input string $w$}\}$ \\
    \ldots for CFG & $A_{CFG}$ & $\{ \langle G,w \rangle \mid  \text{$G$ is a context-free grammar 
    that generates input string $w$}\}$ \\
    \ldots for PDA & $A_{PDA}$ & $\{ \langle B,w \rangle \mid  \text{$B$ is a PDA that accepts input string $w$}\}$ \\
    & & \\
    \hline
    \multicolumn{3}{|l|}{{\bf Language emptiness  testing} } \\
    & & \\
    \ldots for DFA & $E_{DFA}$ & $\{ \langle A \rangle \mid  \text{$A$ is a  DFA and  $L(A) = \emptyset$\}}$ \\
    \ldots for NFA & $E_{NFA}$ & $\{ \langle A\rangle \mid  \text{$A$ is a NFA and  $L(A) = \emptyset$\}}$ \\
    \ldots for regular expressions & $E_{REX}$ & $\{ \langle R \rangle \mid  \text{$R$ is a  regular
    expression and  $L(R) = \emptyset$\}}$ \\
    \ldots for CFG & $E_{CFG}$ & $\{ \langle G \rangle \mid  \text{$G$ is a context-free grammar 
    and  $L(G) = \emptyset$\}}$ \\
    \ldots for PDA & $E_{PDA}$ & $\{ \langle A \rangle \mid  \text{$A$ is a PDA and  $L(A) = \emptyset$\}}$ \\
    & & \\
    \hline
    \multicolumn{3}{|l|}{{\bf Language equality testing} } \\
    & & \\
    \ldots for DFA & $EQ_{DFA}$ & $\{ \langle A, B \rangle \mid  \text{$A$ and $B$ are DFAs and  $L(A) =L(B)$\}}$\\
    \ldots for NFA & $EQ_{NFA}$ & $\{ \langle A, B \rangle \mid  \text{$A$ and $B$ are NFAs and  $L(A) =L(B)$\}}$\\
    \ldots for regular expressions & $EQ_{REX}$ & $\{ \langle R, R' \rangle \mid  \text{$R$ and $R'$ are regular
    expressions and  $L(R) =L(R')$\}}$\\
    \ldots for CFG & $EQ_{CFG}$ & $\{ \langle G, G' \rangle \mid  \text{$G$ and $G'$ are CFGs and  $L(G) =L(G')$\}}$ \\
    \ldots for PDA & $EQ_{PDA}$ & $\{ \langle A, B \rangle \mid  \text{$A$ and $B$ are PDAs and  $L(A) =L(B)$\}}$ \\
    \hline
    \end{tabular}

\begin{enumerate}
    \item[(a)] \gradeComplete Pick five of the computational problems above and give 
    examples (preferably different from the ones we talked about in class) of strings that are
    in each of the corresponding languages. Remember to use the 
    notation $\langle \cdots \rangle$ to denote the string encoding of relevant objects.
    {\it Extension, not for credit:} Explain why it's hard to write a specific string of 
    $0$s and $1$s and make a claim about membership in one of these sets.
    \item[(b)] \gradeComplete Computational problems can also be defined about Turing machines.
    Consider the two high-level descriptions of Turing machines below.
    Reverse-engineer them to define the computational problem that is being
    recognized, where $L(M_{DFA})$ is the language corresponding to this computational
    problem about DFA and $L(M_{TM})$ is the language corresponding to this computational
    problem about Turing machines. {\it Hint}: the computational problem is not acceptance,
    language emptiness, or language equality (but is related to one of them).

    Let $s_1, s_2, \ldots$ be a list of all strings in 
    $\{0,1\}^*$ in string (shortlex) order. Consider the following Turing machines
    \begin{align*}
        M_{DFA} &= ``\text{On input $\langle D \rangle$ where $D$ is a DFA}:\\
         &\text{1. for $i=1, 2, 3, \ldots$} \\
         &\text{2.~~~ Run $D$ on $s_i$} \\
         &\text{3.~~~~If it accepts, accept.}\\
         &\text{4.~~~~If it rejects, go to the next iteration of the loop"}\\
     \end{align*}
     and
     \begin{align*}
        M_{TM} &= ``\text{On input $\langle T \rangle$ where $T$ is a Turing machine}:\\
         &\text{1. for $i=1, 2, 3, \ldots$} \\
         &\text{2.~~~ Run $T$ for $i$ steps on each input $s_1, s_2, \ldots, s_i$ in turn} \\
         &\text{3.~~~~If $T$ has accepted any of these, accept.}\\
         &\text{4.~~~~Otherwise, go to the next iteration of the loop"}\\
     \end{align*}
\end{enumerate}


\end{enumerate}
\end{document}