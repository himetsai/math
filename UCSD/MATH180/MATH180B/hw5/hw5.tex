\documentclass{article}

\usepackage{fancyhdr}
\usepackage{extramarks}
\usepackage{amsmath}
\usepackage{amsthm}
\usepackage{amsfonts}
\usepackage{tikz}
\usepackage[plain]{algorithm}
\usepackage{algpseudocode}
\usepackage{enumerate}
\usepackage{amssymb}
\usepackage{dsfont}

\usetikzlibrary{automata,positioning}

%
% Basic Document Settings
%

\topmargin=-0.45in
\evensidemargin=0in
\oddsidemargin=0in
\textwidth=6.5in
\textheight=9.0in
\headsep=0.25in

\linespread{1.1}

\pagestyle{fancy}
\lhead{\hmwkAuthorName}
\chead{\hmwkClass:\ \hmwkTitle}
\rhead{\firstxmark}
\lfoot{\lastxmark}
\cfoot{\thepage}

\renewcommand\headrulewidth{0.4pt}
\renewcommand\footrulewidth{0.4pt}

\setlength\parindent{0pt}
\setlength{\parskip}{5pt}

%
% Create Problem Sections
%

\newcommand{\enterProblemHeader}[1]{
    \nobreak\extramarks{}{Problem \arabic{#1} continued on next page\ldots}\nobreak{}
    \nobreak\extramarks{Problem \arabic{#1} (continued)}{Problem \arabic{#1} continued on next page\ldots}\nobreak{}
}

\newcommand{\exitProblemHeader}[1]{
    \nobreak\extramarks{Problem \arabic{#1} (continued)}{Problem \arabic{#1} continued on next page\ldots}\nobreak{}
    \stepcounter{#1}
    \nobreak\extramarks{Problem \arabic{#1}}{}\nobreak{}
}

\setcounter{secnumdepth}{0}
\newcounter{partCounter}
\newcounter{homeworkProblemCounter}
\setcounter{homeworkProblemCounter}{1}
\nobreak\extramarks{Problem \arabic{homeworkProblemCounter}}{}\nobreak{}

%
% Homework Problem Environment
%
% This environment takes an optional argument. When given, it will adjust the
% problem counter. This is useful for when the problems given for your
% assignment aren't sequential. See the last 3 problems of this template for an
% example.
%
\newenvironment{homeworkProblem}[1][-1]{
    \ifnum#1>0
        \setcounter{homeworkProblemCounter}{#1}
    \fi
    \section{Problem \arabic{homeworkProblemCounter}}
    \setcounter{partCounter}{1}
    \enterProblemHeader{homeworkProblemCounter}
}{
    \exitProblemHeader{homeworkProblemCounter}
}

%
% Homework Details
%   - Title
%   - Due date
%   - Class
%   - Section/Time
%   - Instructor
%   - Author
%

\newcommand{\hmwkTitle}{Homework\ \#5}
\newcommand{\hmwkDueDate}{Feb 23, 2024}
\newcommand{\hmwkClass}{MATH 180B}
\newcommand{\hmwkClassInstructor}{Professor Carfagnini}
\newcommand{\hmwkAuthorName}{\textbf{Ray Tsai}}
\newcommand{\hmwkPID}{A16848188}

%
% Title Page
%

\title{
    \vspace{2in}
    \textmd{\textbf{\hmwkClass:\ \hmwkTitle}}\\
    \normalsize\vspace{0.1in}\small{Due\ on\ \hmwkDueDate\ at 23:59pm}\\
    \vspace{0.1in}\large{\textit{\hmwkClassInstructor}} \\
    \vspace{3in}
}

\author{
  \hmwkAuthorName \\
  \vspace{0.1in}\small\hmwkPID
}
\date{}

\renewcommand{\part}[1]{\textbf{\large Part \Alph{partCounter}}\stepcounter{partCounter}\\}

%
% Various Helper Commands
%

% Useful for algorithms
\newcommand{\alg}[1]{\textsc{\bfseries \footnotesize #1}}

% For derivatives
\newcommand{\deriv}[1]{\frac{\mathrm{d}}{\mathrm{d}x} (#1)}

% For partial derivatives
\newcommand{\pderiv}[2]{\frac{\partial}{\partial #1} (#2)}

% Integral dx
\newcommand{\dx}{\mathrm{d}x}

% Probability commands: Expectation, Variance, Covariance, Bias
\newcommand{\Var}{\mathrm{Var}}
\newcommand{\Cov}{\mathrm{Cov}}
\newcommand{\Bias}{\mathrm{Bias}}
\newcommand*{\Z}{\mathbb{Z}}
\newcommand*{\Q}{\mathbb{Q}}
\newcommand*{\R}{\mathbb{R}}
\newcommand*{\C}{\mathbb{C}}
\newcommand*{\N}{\mathbb{N}}
\newcommand*{\p}{\mathds{P}}
\newcommand*{\E}{\mathds{E}}

\begin{document}

\maketitle

\pagebreak

\begin{homeworkProblem}
  A Markov chain $X_0, X_1, X_2, \ldots$ has the transition probability matrix
  \begin{center}
    $P$ = \begin{tabular}{c ||c c c||} \multicolumn{1}{c}{\phantom{A}} & \multicolumn{1}{c}{$0$} & \multicolumn{1}{c}{$1$} & \multicolumn{1}{c}{$2$} \\
      0 & 0.3 & 0.2 & 0.5 \\
      1 & 0.5 & 0.1 & 0.4 \\
      2 & 0.5 & 0.2 & 0.3 \\
    \end{tabular}
  \end{center}

  Every period that the process spends in state 0 incurs a cost of \$2. Every period that the process spends in state 1 incurs a cost of \$5. Every period that the process spends in state 2 incurs a cost of \$3. What is the long run cost per period associated with this Markov chain?

  \begin{proof}
    Since $P$ is regular, $P$ has a limiting distribution $\pi = (\pi_0, \pi_1, \pi_2)^T$. We solve for 
    \[
      \begin{cases}
        (I - P^T)\pi = 0 \\
        \sum_i \pi_i = 1
      \end{cases}
    \]
    and get $\pi = (\frac{5}{12}, \frac{2}{11}, \frac{53}{132})$. Since $\pi_i$ can also be interpreted as the long run mean fraction of time the process spent in state $i$, long run cost per period is $\frac{5}{12} \cdot \$2 + \frac{2}{11} \cdot \$5 + \frac{53}{132} \cdot \$3 = \$\frac{389}{132} \approx \$2.94697$.
  \end{proof}
\end{homeworkProblem}

\newpage

\begin{homeworkProblem}
  Five balls are distributed between two urns, labeled A and B. Each period, an urn is selected at random, and if it is not empty, a ball from that urn is removed and placed into the other urn. In the long run what fraction of time is urn A empty?

  \begin{proof}
    Let $\{X_n\}$ denote the number of balls in urn $A$ at step $n$. The transition probability matrix for this process is
    \begin{center}
      $P$ = \begin{tabular}{c ||c c c c c c||} \multicolumn{1}{c}{\phantom{A}} & \multicolumn{1}{c}{$0$} & \multicolumn{1}{c}{$1$} & \multicolumn{1}{c}{$2$} & \multicolumn{1}{c}{$3$} & \multicolumn{1}{c}{$4$} & \multicolumn{1}{c}{$5$} \\
        0 & $\frac{1}{2}$ & $\frac{1}{2}$ & 0 & 0 & 0 & 0 \\
        1 & $\frac{1}{2}$ & 0 & $\frac{1}{2}$ & 0 & 0 & 0 \\
        2 & 0 & $\frac{1}{2}$ & 0 & $\frac{1}{2}$ & 0 & 0 \\
        3 & 0 & $\frac{1}{2}$ & 0 & $\frac{1}{2}$ & 0 & 0 \\
        4 & 0 & 0 & $\frac{1}{2}$ & 0 & $\frac{1}{2}$ & 0 \\
        5 & 0 & 0 & 0 & 0 & $\frac{1}{2}$ & $\frac{1}{2}$ \\
      \end{tabular}.
    \end{center}
    Since $P_{00} > 0$ and every pair of states $i, j$ obviously communicates, $P$ is doubly stochasatic and regular, and thus the long run mean fraction of time urn $A$ is empty is $\frac{1}{6}$.
  \end{proof}
\end{homeworkProblem}

\newpage

\begin{homeworkProblem}
  A Markov chain has the transition probability matrix
  \begin{center}
    $P$ = \begin{tabular}{c ||c c c c c c||} \multicolumn{1}{c}{\phantom{A}} & \multicolumn{1}{c}{$0$} & \multicolumn{1}{c}{$1$} & \multicolumn{1}{c}{$2$} & \multicolumn{1}{c}{$3$} & \multicolumn{1}{c}{$4$} & \multicolumn{1}{c}{$5$} \\
      0  & $\alpha_1$ & $\alpha_2$ & $\alpha_3$ & $\alpha_4$ & $\alpha_5$ & $\alpha_6$ \\
      1  & 1   & 0   & 0   & 0   & 0   & 0   \\
      2  & 0   & 1   & 0   & 0   & 0   & 0   \\
      3  & 0   & 0   & 1   & 0   & 0   & 0   \\
      4  & 0   & 0   & 0   & 1   & 0   & 0   \\
      5  & 0   & 0   & 0   & 0   & 1   & 0   \\
    \end{tabular}.
  \end{center}
  where $\alpha_i \geq 0$, $i = 1, \ldots, 6$, and $\alpha_1 + \ldots + \alpha_6 = 1$. Determine the limiting probability of being in state 0.

  \begin{proof}
    Suppose $\alpha_6 = 0$, then we treast $P$ as a transition matrix for states $0$ to $4$. Then, since $0$ communicates with every state, every pair of states $i, j$ are accessible. With $P_{00} > 0$, we know $P$ has a limiting distribution $\pi = (\pi_0, \dots, \pi_5)$. We then get the system of equations
    \[
      \begin{cases}
        \pi_0 = \alpha_1\pi_0 + \pi_1 \\
        \pi_1 = \alpha_2\pi_0 + \pi_2 \\
        \pi_2 = \alpha_3\pi_0 + \pi_3 \\
        \pi_3 = \alpha_4\pi_0 + \pi_4 \\
        \pi_4 = \alpha_5\pi_0 + \pi_5 \\
        \pi_5 = \alpha_6\pi_0 \\
        \pi_0 + \pi_1 + \pi_2 + \pi_3 + \pi_4 + \pi_5 = 1
      \end{cases}.
    \]
    Solving for it, we get 
    \[
      \begin{cases}
        \pi_1 = (\alpha_2 + \alpha_3 + \alpha_4 + \alpha_5 + \alpha_6)\pi_0 \\
        \pi_2 = (\alpha_3 + \alpha_4 + \alpha_5 + \alpha_6)\pi_0 \\
        \pi_3 = (\alpha_4 + \alpha_5 + \alpha_6)\pi_0 \\
        \pi_4 = (\alpha_5 + \alpha_6)\pi_0 \\
        \pi_5 = \alpha_6\pi_0 \\
      \end{cases},
    \]
    and thus the limiting probability of being in state 0 is $\pi_0 = (1 + \alpha_2 + 2\alpha_3 + 3\alpha_4 + 4\alpha_5 + 5\alpha_6)^{-1}$.
  \end{proof}
\end{homeworkProblem}

\newpage

\begin{homeworkProblem}
  Consider a Markov chain with transition probability matrix

  \begin{center}
    $P$ = \begin{tabular}{||c c c c c||} $p_0$ & $p_1$ & $p_{2}$ & $\cdots$ & $p_N$ \\
      $p_N$ & $p_0$ & $p_{1}$ & $\cdots$ & $p_{N-1}$ \\
      $p_{N-1}$ & $p_N$ & $p_{0}$ & $\cdots$ & $p_{N-2}$ \\
      $\vdots$ & $\vdots$ & $\vdots$ & & $\vdots$ \\
      $p_1$ & $p_2$ & $p_3$ & $\cdots$ & $p_0$ \\
    \end{tabular}
  \end{center}
  

  where $0 < p_0 < 1$ and $p_0 + p_1 + \cdots + p_N = 1$. Determine the limiting distribution.

  \begin{proof}
    We already know that $P$ is aperiodic, as $P_{ii} = p_0 > 0$. Since $p_0 + p_1 + \cdots + p_N = 1$ and $p_0 < 1$, there exsits $p_i > 0$ for some $i \neq 0$, there exists a directed hamiltonian cycle in the state transition diagram, and thus $P$ is an irreducible stochastic matrix. It follows that the limiting distribution of $P$ exists. Since $P$ is doubly stochastic, the limiting distribution is $\left(\frac{1}{N + 1}, \dots, \frac{1}{N + 1}\right)$.
  \end{proof}
\end{homeworkProblem}

\newpage

\begin{homeworkProblem}
  A component of a computer has an active life, measured in discrete units, that is a random variable $\xi$, where
  \begin{center}
    \begin{tabular}{c|cccc}
      $k$ & 1 & 2 & 3 & 4 \\
      \hline
      $\Pr\{\xi = k\}$ & 0.1 & 0.3 & 0.2 & 0.4 \\
    \end{tabular}
  \end{center}

  Suppose that one starts with a fresh component, and each component is replaced by a new component upon failure. Let $X_n$ be the \textit{remaining life} of the component in service at the \textit{end} of period $n$. When $X_n = 0$, a new item is placed into service at the \textit{start} of the next period.

  \begin{enumerate}[(a)]
    \item Set up the transition probability matrix for $\{X_n\}$.
    \begin{proof}
      The transition probability matrix is 
      \begin{center}
        $P$ = \begin{tabular}{c ||c c c c||} \multicolumn{1}{c}{\phantom{A}} & \multicolumn{1}{c}{$0$} & \multicolumn{1}{c}{$1$} & \multicolumn{1}{c}{$2$} & \multicolumn{1}{c}{$3$} \\
          0 & $0.1$ & $0.3$ & $0.2$ & $0.4$ \\
          1 & $1$ & $0$ & $0$ & $0$ \\
          2 & $0$ & $1$ & $0$ & $0$ \\
          3 & $0$ & $0$ & $1$ & $0$ \\
        \end{tabular}
      \end{center}
    \end{proof}
    \item By showing that the chain is regular and solving for the limiting distribution, determine the long run probability that the item in service at the end of a period has no remaining life and therefore will be replaced.
    \begin{proof}
      $P$ is aperiodic, as $P_{00} > 0$. Notice that for all state $i > 0$, 0 is accessible from state $i$ by following the path $i \to i - 1 \to \dots \to 0$. Since we also know that every positive state is accessible from $0$, $P$ is irreducible.  It follows that $P$ is regular, so the limiting distribution $\pi = (\pi_0, \dots, \pi_4)$ exists. We then get the system of equations
      \[
        \begin{cases}
          \pi_0 = 0.1\pi_0 + \pi_1 \\
          \pi_1 = 0.3\pi_0 + \pi_2 \\
          \pi_2 = 0.2\pi_0 + \pi_3 \\
          \pi_3 = 0.4\pi_0 \\
          \pi_0 + \pi_1 + \pi_2 + \pi_3 = 1
        \end{cases},
      \]
      and solving it gives $\pi = (\frac{10}{29}, \frac{9}{29}, \frac{6}{29}, \frac{4}{29})$.
    \end{proof}
    \item Relate this to the mean life of a component.
    \begin{proof}
      Notice that $E[\xi]\pi_0 = (1 \cdot 0.1 + 2 \cdot 0.3 + 3 \cdot 0.2 + 4 \cdot 0.4) \cdot \frac{10}{29} = 1$. This implies that the number of periods that we make replacements of component multiplied by the mean life of a component approaches the number of periods of the process in the long run.
    \end{proof}
  \end{enumerate}
\end{homeworkProblem}

\newpage

\begin{homeworkProblem}
  Consider a computer system that fails on a given day with probability $p$ and remains ``up'' with probability $q = 1 - p$. Suppose the repair time is a random variable $N$ having the probability mass function $p(k) = \beta(1 - \beta)^{k-1}$ for $k = 1, 2, \ldots$, where $0 < \beta < 1$. Let $X_n = 1$ if the computer is operating on day $n$ and $X_1 = 0$ if not. Show that $\{X_n\}$ is a Markov chain with transition matrix

  \begin{center}
    \begin{tabular}{c ||c c||} \multicolumn{1}{c}{\phantom{A}} & \multicolumn{1}{c}{$0$} & \multicolumn{1}{c}{$1$} \\
      0 & $\alpha$ & $\beta$ \\
      1 & $p$ & $q$ \\
    \end{tabular}
  \end{center}
  and $\alpha = 1 - \beta$. Determine the long run probability that the computer is operating in terms of $\alpha, \beta, p$, and $q$.

  \begin{proof}
    We already know $\p(X_n = i \mid X_{n - 1} = 1, \dots, X_0 = i_0) = \p(X_n = i \mid X_{n - 1} = 1)$. Notice that $p(k)$ is a geometric distribution. Hence, on any given day that our computer is broken, the probability of it being repaired on that day is $\beta$ regardless of when the repairment was incurred. Hence, $\{X_n\}$ is a Markov chain and has the above transition matrix. Name that matrix $P$. Since $p, \beta > 0$ and $\beta < 1$, $P$ is irreducible and aperiodic, and thus the limiting distribution $\pi = (\pi_0, \pi_1)^T$ exists. We then solve for $(I - P^T)\pi = 0$ and get $\pi_0 = \frac{p}{\beta + p}$ and $\pi_1 = \frac{\beta}{\beta + p}$.
  \end{proof}
\end{homeworkProblem}
\end{document}