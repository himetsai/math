\documentclass{article}

\usepackage{fancyhdr}
\usepackage{extramarks}
\usepackage{amsmath}
\usepackage{amsthm}
\usepackage{amsfonts}
\usepackage{tikz}
\usepackage[plain]{algorithm}
\usepackage{algpseudocode}
\usepackage{enumerate}
\usepackage{amssymb}
\usepackage{dsfont}

\usetikzlibrary{automata,positioning}

%
% Basic Document Settings
%

\topmargin=-0.45in
\evensidemargin=0in
\oddsidemargin=0in
\textwidth=6.5in
\textheight=9.0in
\headsep=0.25in

\linespread{1.1}

\pagestyle{fancy}
\lhead{\hmwkAuthorName}
\chead{\hmwkClass:\ \hmwkTitle}
\rhead{\firstxmark}
\lfoot{\lastxmark}
\cfoot{\thepage}

\renewcommand\headrulewidth{0.4pt}
\renewcommand\footrulewidth{0.4pt}

\setlength\parindent{0pt}
\setlength{\parskip}{5pt}

%
% Create Problem Sections
%

\newcommand{\enterProblemHeader}[1]{ \nobreak\extramarks{}{Problem \arabic{#1} continued on next
    page\ldots}\nobreak{} \nobreak\extramarks{Problem \arabic{#1} (continued)}{Problem \arabic{#1}
    continued on next page\ldots}\nobreak{} }

\newcommand{\exitProblemHeader}[1]{ \nobreak\extramarks{Problem \arabic{#1} (continued)}{Problem
    \arabic{#1} continued on next page\ldots}\nobreak{}
    \stepcounter{#1}
    \nobreak\extramarks{Problem \arabic{#1}}{}\nobreak{}
}

\setcounter{secnumdepth}{0}
\newcounter{partCounter}
\newcounter{homeworkProblemCounter}
\setcounter{homeworkProblemCounter}{1}
\nobreak\extramarks{Problem \arabic{homeworkProblemCounter}}{}\nobreak{}

%
% Homework Problem Environment
%
% This environment takes an optional argument. When given, it will adjust the
% problem counter. This is useful for when the problems given for your
% assignment aren't sequential. See the last 3 problems of this template for an
% example.
%
\newenvironment{homeworkProblem}[1][-1]{
    \ifnum#1>0
        \setcounter{homeworkProblemCounter}{#1}
    \fi
    \section{Problem \arabic{homeworkProblemCounter}}
    \setcounter{partCounter}{1}
    \enterProblemHeader{homeworkProblemCounter}
}{
    \exitProblemHeader{homeworkProblemCounter}
}

%
% Homework Details
%   - Title
%   - Due date
%   - Class
%   - Section/Time
%   - Instructor
%   - Author
%

\newcommand{\hmwkTitle}{Homework\ \#6}
\newcommand{\hmwkDueDate}{Jun 3, 2024}
\newcommand{\hmwkClass}{MATH 188}
\newcommand{\hmwkClassInstructor}{Professor Kunnawalkam Elayavalli}
\newcommand{\hmwkAuthorName}{\textbf{Ray Tsai}}
\newcommand{\hmwkPID}{A16848188}

%
% Title Page
%

\title{
    \vspace{2in}
    \textmd{\textbf{\hmwkClass:\ \hmwkTitle}}\\
    \normalsize\vspace{0.1in}\small{Due\ on\ \hmwkDueDate\ at 23:59pm}\\
    \vspace{0.1in}\large{\textit{\hmwkClassInstructor}} \\
    \vspace{3in}
}

\author{
  \hmwkAuthorName \\
  \vspace{0.1in}\small\hmwkPID
}
\date{}

\renewcommand{\part}[1]{\textbf{\large Part \Alph{partCounter}}\stepcounter{partCounter}\\}

%
% Various Helper Commands
%

% Useful for algorithms
\newcommand{\alg}[1]{\textsc{\bfseries \footnotesize #1}}

% For derivatives
\newcommand{\deriv}[1]{\frac{\mathrm{d}}{\mathrm{d}x} (#1)}

% For partial derivatives
\newcommand{\pderiv}[2]{\frac{\partial}{\partial #1} (#2)}

% Integral dx
\newcommand{\dx}{\mathrm{d}x}

% Probability commands: Expectation, Variance, Covariance, Bias
\newcommand{\Var}{\mathrm{Var}}
\newcommand{\Cov}{\mathrm{Cov}}
\newcommand{\Bias}{\mathrm{Bias}}
\newcommand*{\Z}{\mathbb{Z}}
\newcommand*{\Q}{\mathbb{Q}}
\newcommand*{\R}{\mathbb{R}}
\newcommand*{\C}{\mathbb{C}}
\newcommand*{\N}{\mathbb{N}}
\newcommand*{\p}{\mathds{P}}
\newcommand*{\E}{\mathds{E}}

\begin{document}

\maketitle

\pagebreak

\begin{homeworkProblem}
  The following exercise gives another proof of Cayley's formula, and at the same time provides new information that our proof doesn't give.

  Let $n \geq 1$ and let $x_1, \dots, x_n$ be variables. Given a labeled tree $T$ with vertices $1, \dots, n$, define the monomial $x(T) = x_1^{d_1} \cdots x_n^{d_n}$ where $d_i$ is the degree of vertex $i$, i.e., the number of edges containing $i$. Define $\textbf{C}_n = \sum_T x(T)$ where the sum is over all labeled trees $T$ with vertices $1, \dots, n$. Also define
  \[
    \textbf{D}_n = x_1 x_2 \cdots x_n (x_1 + x_2 + \cdots + x_n)^{n-2}.
  \]

  \begin{enumerate}[(a)]
    \item Given a polynomial $p(x_1, \dots, x_n)$, let $p^{(i)}$ be the result of plugging in $x_i = 0$ into the partial derivative $\frac{\partial p}{\partial x_i}$, i.e., the coefficient of $x_i$ if you think of the other variables as constants. If $n \geq 2$, show that
    \[
      \textbf{C}_n^{(n)} = (x_1 + x_2 + \cdots + x_{n-1})\textbf{C}_{n-1},
    \]
    \[
      \textbf{D}_n^{(n)} = (x_1 + x_2 + \cdots + x_{n-1})\textbf{D}_{n-1}.
    \]
    \begin{proof}
      Since a tree is connected, all vertices has positive degree. Hence, we have
      \begin{gather}
        \textbf{C}_n^{(n)} = \left.\frac{\partial}{\partial x_n} \sum_{T} x(T)\right|_{x_n = 0} = \sum_{T} \left.d_nx_1^{d_1} \cdots x_n^{d_n - 1}\right|_{x_n = 0} = \sum_{T; d_n = 1} x_1^{d_1} \cdots x_{n - 1}^{d_{n - 1}}.
      \end{gather}
      A tree must contains a vertex of degree 1. Given a $T$ with $d_n = 1$, suppose $j$ is the only neighbor of $n$. Then, $x(T) = x_nx_jx(T_{n - 1})$, where $T_{n - 1} = T - \{n\}$ a labeled tree with vertex set $[n - 1]$. On the other hand, given a labeled tree $T_{n - 1}$ with vertex set $[n - 1]$, we may choose a vertex which connects to $n$ and get $T$ with $d_n = 1$, with $x(T) = x_nx_jx(T_{n - 1})$. It now follows that
      \begin{align*}
        \textbf{C}_n^{(n)} 
        &= \left.\frac{\partial}{\partial x_n} \sum_{T} x(T)\right|_{x_n = 0} \\
        &= \left.\frac{\partial}{\partial x_n} \sum_{T; d_n = 1} x(T)\right|_{x_n = 0} + \left.\frac{\partial}{\partial x_n} \sum_{T; d_n \neq 1} x(T)\right|_{x_n = 0}\\
        &= \left.\sum_{j = 1}^{n - 1}\sum_{\substack{T; d_n = 1, \\ \{j, n\} \in e(T)}} \frac{\partial}{\partial x_n} x_nx_jx(T_{n - 1})\right|_{x_n = 0} \\
        &= \sum_{j = 1}^{n - 1} x_j\sum_{T_{n - 1}} x(T_{n - 1}) = \sum_{j = 1}^{n - 1} x_j\textbf{C}_{n-1}.
      \end{align*}

      On the other hand,
      \begin{align*}
        \textbf{D}_n^{(n)}
        = x_1 x_2 \cdots x_{n - 1} &(x_1 + x_2 + \cdots + x_n)^{n-2} \\
        &+ (n - 2)x_1 x_2 \cdots x_{n} (x_1 + x_2 + \cdots + x_n)^{n-3} |_{x_n = 0} \\
        &= x_1 x_2 \cdots x_{n - 1} (x_1 + x_2 + \cdots + x_{n - 1})^{n-2} \\
        &= (x_1 + x_2 + \cdots + x_{n - 1})\textbf{D}_{n - 1}.
      \end{align*}
    \end{proof}

    \break

    \item Assuming that $\textbf{C}_{n-1} = \textbf{D}_{n-1}$ show that $\textbf{C}_n^{(i)} = \textbf{D}_n^{(i)}$ for all $i = 1, \dots, n$.
    \begin{proof}
      Define $\textbf{C}_{[n] - \{i\}} = \sum_{T_{[n] - \{i\}}} x(T_{[n] - \{i\}})$, where the sum is over all labeled trees $T_{[n] - \{i\}}$ with vertices $[n] - \{i\}$. Also define $\textbf{D}_{[n] - \{i\}} = x_1 x_2 \cdots x_nx^{-1}_i(x_1 + x_2 + \cdots + x_n - x_i)^{n-3}$.
      
      Using the same argument in (a), we may show that
      \[
        \textbf{C}_n^{(i)} = \sum_{T; d_i = 1} \prod_{j \neq i} x_j^{d_j} = \sum_{j = 1, j \neq i}^{n} x_j\textbf{C}_{[n] - \{i\}},
      \]
      for all $i$. On the other hand, for all $i$,
      \begin{align*}
        \textbf{D}_n^{(i)}
        = x_i^{-1}x_1 x_2 \cdots x_{n} &(x_1 + x_2 + \cdots + x_n)^{n-2} \\
        &+ (n - 2)x_1 x_2 \cdots x_{n} (x_1 + x_2 + \cdots + x_n)^{n-3} |_{x_i = 0} \\
        &= x_1 x_2 \cdots x_{n}x_i^{-1} (x_1 + x_2 + \cdots + x_{n} - x_i)^{n-2} \\
        &= (x_1 + x_2 + \cdots + x_{n} - x_i)\textbf{D}_{[n] - \{i\}}.
      \end{align*}
      Note that the only differences between $\textbf{C}_{n-1}, \textbf{C}_{[n] - \{i\}}$ and between $\textbf{D}_{n-1}, \textbf{D}_{[n] - \{i\}}$ are the indexing of the variables. Hence, $\textbf{C}_{n-1} = \textbf{D}_{n-1}$ also implies that $\textbf{C}_{[n] - \{i\}} = \textbf{D}_{[n] - \{i\}}$. It now follows that
      \[
        \textbf{C}_n^{(i)} = \sum_{j = 1, j \neq i}^{n} x_j\textbf{C}_{[n] - \{i\}} = \sum_{j = 1, j \neq i}^{n} x_j\textbf{D}_{[n] - \{i\}} = \textbf{D}_n^{(i)},
      \]
      for all $i$.
    \end{proof}
    \item Conclude that $\textbf{C}_n = \textbf{D}_n$ for all $n \geq 1$.
    \begin{proof}
      We proceed by induction on $n$. When $n = 1$, there are only one label tree, which is a singleton. Hence, $\textbf{C}_1 = 1 = x_1x_1^{-1} = \textbf{D}_1$. Suppose $n \geq 2$. Since each tree has at least a leaf, each term in $\textbf{C}_n$ has some $x_i$ with power 1. On the other hand, note that each term in the expansion of $(x_1 + x_2 + \cdots + x_n)^{n-2}$ misses at least one $x_i$, and thus each term in $\textbf{D}_n$ has some $x_i$ with degree 1. By induction and (b), we have $\textbf{C}_{n}^{(i)} = \textbf{D}_{n}^{(i)}$ for all $i \in [n]$. It follows that the terms with some single degree $x_i$ are equal in $\textbf{C}_n$ and $\textbf{D}_n$, which entail every term. Hence, $\textbf{C}_n = \textbf{D}_n$.
    \end{proof}
  \end{enumerate}
\end{homeworkProblem}

\newpage

\begin{homeworkProblem}
  How many ways are there to list the letters of the word MATHEMATICS so that no two consecutive letters are the same?

  \begin{proof}
    The repeated characters in the word MATHEMATICS are A, M, and T. Let $S_A, S_M, S_T$ each be the set of ways to list MATHEMATICS with consecutive A, M, T, respectively. By inclusion-exclusion, the number of arrangements of MATHEMATICS with consecutive same characters is
    \[
      |S_A \cup S_M \cup S_T| = |S_A| + |S_M| + |S_T| - |S_A \cap S_M| - |S_A \cap S_T| - |S_T \cap S_T| + |S_A \cap S_M \cap S_T|.
    \]
    Since A, M, and T each appears exactly twice in MATHEMATICS, 
    \[
      |S_A \cup S_M \cup S_T| = 3|S_A| - 3|S_A \cap S_M| + |S_A \cap S_M \cap S_T|,
    \]
    by symmetry. Notice that to count elements in $S_A$, we may view $AA$ as a single character, and
    \[
      |S_A| = \frac{10!}{2!2!}.
    \]
    Similarily, to count elements in $S_A \cap S_M$, we may $AA$ and $MM$ as characters and get
    \[
      |S_A \cap S_M| = \frac{9!}{2!}.
    \]
    Using the same idea, we get
    \[
      |S_A \cap S_M \cap S_T| = 8!.
    \]
    Hence, 
    \[
      |S_A \cup S_M \cup S_T| = 3 \cdot \frac{10!}{2!2!} - 3 \cdot \frac{9!}{2!} + 8!.
    \]
    In total there are $\frac{11!}{2!2!2!}$ arrangements of MATHEMATICS, so the number of arrangements of MATHEMATICS with no consecutive repeated characters is
    \[
      \frac{11!}{2!2!2!} - 3 \cdot \frac{10!}{2!2!} + 3 \cdot \frac{9!}{2!} - 8! = 2772000.
    \]
  \end{proof}
\end{homeworkProblem}

\newpage

\begin{homeworkProblem}
  Let $n \geq 2$ be an integer. We have $n$ married couples ($2n$ people in total).
  \begin{enumerate}[(a)]
    \item How many ways can we have the $2n$ people stand in a line so that no person is standing next to their spouse?
    \begin{proof}
      Let $L$ be the set of all arrangements of $2n$ people in a line. Put $A = A_1 \cup \cdots \cup A_n$, where $A_i$ is set of ways to line up $n$ couples with the $i$th couple standing next to each other. Let $S \subseteq [n]$. Define
      \begin{gather*}
        f(S) = |\{x \in A  \mid x \in A_i \text{ if and only if } i \in S\}|, \\
        g(S) = |\{x \in A \mid x \in A_i \text{ if } i \in S\}|.
      \end{gather*}
      Note that $g(\emptyset) = |A|$ and $f(\emptyset) = 0$. To calculate $g(S) = |\cap_{i \in S} A_i|$, we may view each couple in $S$ as a unit of people and account the ordering of each unit. Then, we would have $2n - |S|$ unit of people, with $|S|$ units each having 2 arrangements, so
      \[
        g(S) = (2n - |S|)!2^{|S|}.
      \]
      By inclusion-exclusion,
      \[
        |A| = \sum_{\substack{S \subseteq [n] \\ S \neq \emptyset}} (-1)^{|S| - 1} (2n - |S|)!2^{|S|} = \sum_{k = 1}^{n} (-1)^{k - 1}\binom{n}{k}(2n - k)!2^{k}
      \]
      Since we are calculating the case where no person stands next to their spouse, we have
      \[
        |L \backslash A| = |L| - |A| = (2n)! - \sum_{k = 1}^{n} (-1)^{k - 1}\binom{n}{k}(2n - k)!2^{k} = \sum_{k = 0}^{n} (-2)^{k}\binom{n}{k}(2n - k)!.
      \]
    \end{proof}
    \item Same as (a), but replace ``line'' by ``circle''.
    \begin{proof} 
      Let $L$ be the set of all arrangements of $2n$ people in a line. Put $A = A_1 \cup \cdots \cup A_n$, where $A_i$ is set of ways to line up $n$ couples with the $i$th couple standing next to each other. Let $S \subseteq [n]$. Define
      \begin{gather*}
        f(S) = |\{x \in A  \mid x \in A_i \text{ if and only if } i \in S\}|, \\
        g(S) = |\{x \in A \mid x \in A_i \text{ if } i \in S\}|.
      \end{gather*}
      Note that $g(\emptyset) = |A|$ and $f(\emptyset) = 0$. To calculate $g(S) = |\cap_{i \in S} A_i|$, we may view each couple in $S$ as a unit of people and account the ordering of each unit. Then, we would have $2n - |S|$ unit of people, with $|S|$ units each having 2 arrangements, so
      \[
        g(S) = \frac{1}{2n - |S|} \cdot (2n - |S|)!2^{|S|} = (2n - |S| - 1)!2^{|S|}.
      \]
      Note that we divide by $2n - |S|$ to disregard shifting the circle. By inclusion-exclusion,
      \[
        |A| = \sum_{\substack{S \subseteq [n] \\ S \neq \emptyset}} (-1)^{|S| - 1} (2n - |S| - 1)!2^{|S|} = \sum_{k = 1}^{n} (-1)^{k - 1}\binom{n}{k}(2n - k - 1)!2^{k}
      \]
      Hence, the number of ways to have $n$ couples stand in a circle with no person standign next to their spouse is
      \[
        |L \backslash A| = \frac{1}{2n} \cdot (2n)! - \sum_{k = 1}^{n} (-1)^{k - 1}\binom{n}{k}(2n - k - 1)!2^{k} = \sum_{k = 0}^{n} (-2)^{k}\binom{n}{k}(2n - k - 1)!.
      \]
    \end{proof}
  \end{enumerate}
\end{homeworkProblem}

\newpage

\begin{homeworkProblem}
  Let $q$ be a prime power and $n$ a positive integer. Let $V$ be an $n$-dimensional $\textbf{F}_q$-vector space and let $P$ be the poset whose elements are linear subspaces of $V$ with the ordering $X \leq Y$ if $X$ is contained in $Y$. Show that the Möbius function of $P$ is given by
  \[
    \mu(X, Y) = (-1)^d q^{\binom{d}{2}}
  \]
  where $d = \dim Y - \dim X$. 

  \begin{proof}
    For $X \leq Y$, it suffices to show that 
    \[
      \delta_{X, Y} = \sum_{U \in [X, Y]} (-1)^{d_u} q^{\binom{d_u}{2}},
    \]
    where $d_u = \dim U - \dim X$. Put $x = \dim X$ and $y = \dim Y$. The number of vector spaces of dimension $k$ in the interval $[X, Y]$ is $\begin{bmatrix} d \\
      k - x \end{bmatrix}_q$. It now follows that
    \begin{align*}
      \sum_{U \in [X, Y]} (-1)^{d_u} q^{\binom{d_u}{2}} 
      &= \sum_{k = x}^y \begin{bmatrix} 
        d \\
        k - x
      \end{bmatrix}_q (-1)^{k - x} q^{\binom{k - x}{2}} \\
      &= \sum_{i = 0}^{d} \begin{bmatrix} 
        d \\
        i
      \end{bmatrix}_q (-1)^{i} q^{\binom{i}{2}}.
    \end{align*}
    By Theorem 3.2.4 from Sagen,
    \[
      \sum_{i = 0}^{d} \begin{bmatrix} 
        d \\
        i
      \end{bmatrix}_q (-1)^{i} q^{\binom{i}{2}} = \begin{cases}
        \prod_{i = 0}^{d - 1} (1 - q^{i}) & d > 0 \\
        1 & d = 0
      \end{cases} = \delta_{X, Y}.
    \]
  \end{proof}
\end{homeworkProblem}

\newpage

\begin{homeworkProblem}
  Let $\Pi_n$ be the poset of set partitions of $[n]$ and let $\mu$ be its Möbius function. Write a formula for the number of connected labeled graphs with vertex set $[n]$ using $\mu$.

  \begin{proof}
    Let $\mathcal{G}$ be the set of all labeled graphs with vertex set $[n]$, and let $P = \{S_1, \dots, S_m\} \in \Pi_n$. Define  
    \begin{gather*}
      f(P) = |\{G \in \mathcal{G} \mid i, j \text{ connected in } G \text{ if and only if } i, j \in S_k \text{ for some } k\}|, \\
      g(P) = |\{G \in \mathcal{G} \mid i, j \text{ connected in } G \text{ only if } i, j \in S_k \text{ for some } k\}| = \prod_{k = 1}^m 2^{\binom{|S_k|}{2}} = 2^{\sum_{k = 1}^m \binom{|S_k|}{2}}.
    \end{gather*}
    Note that $f(\{[n]\})$ is the number of connected labeled graphs with vertex set $[n]$ and $g(\{[n]\}) = 2^{\binom{n}{2}}$. By definition, $g(P) = \sum_{Q \leq P} f(Q)$. It now follows by the Möbius inversion that
    \[
      f(\{[n]\}) = \sum_{P \in \Pi_n}\mu(P, \{[n]\})g(P) = \sum_{P \in \Pi_n}\mu(P, \{[n]\}) 2^{\sum_{S \in P} \binom{|S|}{2}}.
    \]
  \end{proof}
\end{homeworkProblem}

\newpage

\begin{homeworkProblem}
  $F(x) = \sum_{n \geq 0} f_n x^n$ is a formal power series that satisfies the following identity:
  \[
    F(x) = \exp\left(\frac{x}{2} (F(x) + 1)\right).
  \]
  Find a formula for $f_n$.
  \begin{proof}
    We first note that $f_0 = F(0) = 1$. Adding 1 then multiplying $\frac{x}{2}$ on both sides of the given identity yields
    \[
      \frac{x}{2}(F(x) + 1) = \frac{x}{2}\left[\exp\left(\frac{x}{2} (F(x) + 1)\right) + 1\right].
    \]
    Take $G(x) = \frac{1}{2}(e^x + 1)$ and $A(x) = \frac{x}{2}(F(x) + 1)$. Since $A(0) = 0$ and $G(0) \neq 0$, the Lagrange inversion formula gives
    \begin{align*}
      \frac{n + 1}{2}[x^{n}]F(x) 
      &= (n + 1)[x^{n + 1}]A(x) \\
      &= [x^{n}](G(x)^{n + 1}) \\
      &= [x^n]\frac{1}{2^{n + 1}}(e^x + 1)^{n + 1} \\
      &= [x^n]\frac{1}{2^{n + 1}}\sum_{k = 0}^{n + 1} \binom{n + 1}{k} e^{kx} \\
      &= \frac{1}{2^{n + 1}}\sum_{k = 0}^{n + 1} \binom{n + 1}{k} \frac{k^n}{n!}
    \end{align*}
    That is, for $n \geq 1$,
    \[
      f_n = \frac{1}{2^n(n + 1)!}\sum_{k = 0}^{n + 1} \binom{n + 1}{k} k^n.
    \]
  \end{proof}
\end{homeworkProblem}

\newpage

\begin{homeworkProblem}
  Reminder: Lagrange's version of the Taylor remainder theorem says this: if $f(x)$ is an infinitely differentiable function whose Taylor series at 0 converges at $x = r$, then there exists $\xi$ between 0 and $r$ such that
  \[
    f(r) - \sum_{i=0}^n \frac{f^{(i)}(0)}{i!} r^i = \frac{f^{(n+1)}(\xi)}{(n+1)!} r^{n+1}.
  \]
  Use the Taylor remainder theorem to show that
  \[
    \left| \frac{1}{e} - \sum_{i=0}^n \frac{(-1)^i}{i!} \right| \leq \frac{1}{(n+1)!}
  \]
  and conclude from this that the number of derangements of $n$ objects is inside the closed interval
  \[
    \left[ \frac{n!}{e} - \frac{1}{n+1}, \frac{n!}{e} + \frac{1}{n+1} \right].
  \]
  In particular, show that it is the closest integer to $n!/e$.

  \begin{proof}
    Consider $f(r) = e^{-r}$. When $r = 1$, Taylor remainder theorem yields 
    \[
      e^{-1} - \sum_{i=0}^n \frac{(-1)^{i}}{i!} = \frac{(-1)^{n + 1}e^{-\xi}}{(n+1)!},
    \]
    for some $\xi$ between $0$ and $1$. But then $|e^{-\xi}| \leq 1$, so 
    \[
      \left|\frac{1}{e} - \sum_{i=0}^n \frac{(-1)^{i}}{i!}\right| = \left|\frac{(-1)^{n + 1}e^{-\xi}}{(n+1)!}\right| \leq \frac{1}{(n + 1)!}.
    \]
    Let $D(n)$ be the number of derangements of size $n$. It now follows from Theorem 6.14 that
    \[
      \frac{n!}{e} - \frac{1}{n + 1} \leq D(n) = n!\sum_{i=0}^n \frac{(-1)^{i}}{i!} \leq \frac{n!}{e} + \frac{1}{n + 1}.
    \]
    Obviously $\frac{n!}{e}$ is not an integer. Since $D(n)$ is an integer and $\frac{1}{n + 1} \leq \frac{1}{2}$ for all $n \geq 1$, $D(n)$ is the closest integer to $\frac{n!}{e}$.
  \end{proof}
\end{homeworkProblem}

\newpage

\begin{homeworkProblem}
  Let $d_n$ be the number of derangements of $[n]$, and let
  \[
    D(x) = \sum_{n \geq 0} \frac{d_n}{n!} x^n.
  \]
  \begin{enumerate}[(a)]
    \item Using the structure interpretation for products of EGF, show that
    \[
      D(x)e^x = \frac{1}{1 - x}.
    \]
    \begin{proof}
      Let $D(S)$ denote the set of derangements of $S$. Define structures $\alpha(S) = D(S)$ and $\beta(S) = \{0\}$. Hte product of two structures is
      \[
        (\alpha \cdot \beta)(S) = \bigsqcup_{T \subseteq S} D(T) \times \{0\} \simeq \bigsqcup_{T \subseteq S} D(T).
      \]
      But then given a derangement of some subset $T \subseteq S$, we get a permutation $\sigma$ of $S$ with $\sigma(i) = i$ if and only if $i \in S \backslash T$. On the other hand, given a permutation $\sigma$ of $S$, we get a derangement of $T = \{i \in S \mid \sigma(i) \neq i\} \subseteq S$. Hence, $\bigsqcup_{T \subseteq S} D(T) \simeq S_n$, the symmetry group of degree $n$. It now follows that $(\alpha \cdot \beta)(S) = |S|!$, and so
      \[
        D(x)e^x = E_{\alpha \cdot \beta}(x) = \sum_{n \geq 0} x^n = \frac{1}{1 - x}.
      \]
    \end{proof}
    \item Show how this implies the formula we previously obtained:
    \[
      d_n = \sum_{i=0}^n (-1)^i \frac{n!}{i!}.
    \]
    \begin{proof}
      Rearranging the result of (a), we get
      \[
        D(x) = \frac{1}{1 - x}e^{-x},
      \]
      and so
      \[
        d_n = n![x^n]\frac{1}{1 - x}e^{-x} = \sum_{i = 0}^n (-1)^i\frac{n!}{i!}.
      \]
    \end{proof}
\end{enumerate}
\end{homeworkProblem}

\newpage

\begin{homeworkProblem}
  For a positive integer $n$, define
  \[
    f(n) = |\{ i \in \mathbb{Z} \mid 1 \leq i \leq n, \gcd(n, i) = 1 \}|.
  \]
  \begin{enumerate}[(a)]
    \item Show that
    \[
      n = \sum_{d|n} f(d)
    \]
    where the sum is over all positive integers $d$ that divide $n$.
    \begin{proof}
      \begin{align*}
        \sum_{d|n} f(d) 
        &= \sum_{d|n} f(n/d) \\
        &= \sum_{d|n} |\{i \in \mathbb{Z} \mid 1 \leq i \leq n/d, \gcd(n/d, i) = 1\}| \\
        &= \sum_{d|n} |\{i \in \mathbb{Z} \mid 1 \leq i \leq n, \gcd(n, i) = d\}| \\
        &= n.
      \end{align*}
    \end{proof}
    \item Use Möbius inversion to show that
    \[
      f(n) = n \prod_{p|n} \left( 1 - \frac{1}{p} \right)
    \]
    where the product is over the primes $p$ that divide $n$.

    \begin{proof}
      Define $g(n) = n$, for $n \in \N$. In (a), we already established that $g(n) = \sum_{d|n} f(d)$. By the Möbius inversion formula,
      \begin{align*}
        f(n) 
        &= \sum_{d|n} g(d)\mu(d, n) \\
        &= \sum_{d|n} d\mu(d, n) \\
        &= \sum_{d|n} \frac{n}{d}\mu\left(\frac{n}{d}, n\right) \\
        &= n\sum_{d|n} \frac{1}{d}\mu\left(\frac{n}{d}, n\right) \\
        &= n\sum_{\substack{d = p_1\cdots p_k \\ p_i | n \text{ and distinct}}} \frac{(-1)^k}{d} \\
        &= n\sum_{\substack{d = p_1\cdots p_k \\ p_i | n \text{ and distinct}}} \left(-\frac{1}{p_1}\right) \cdots \left(-\frac{1}{p_k}\right) \\
        &= n\prod_{p|n} \left( 1 - \frac{1}{p} \right).
      \end{align*}
    \end{proof}
  \end{enumerate}
\end{homeworkProblem}

\newpage

\begin{homeworkProblem}
  There are $n$ people sitting at a circular table. How many ways can they rearrange seats so that no one sits next to someone they were sitting next to before?

  \begin{proof}
    idk bro.
  \end{proof}
\end{homeworkProblem}

\newpage

\begin{homeworkProblem}
  Let $q$ be a prime power and let $N_n$ be the number of monic irreducible polynomials of degree $n$ with coefficients in $\textbf{F}_q$:
  \begin{enumerate}[(a)]
    \item Using that polynomials over a field satisfy unique factorization, show that
    \[
      (1 - q x)^{-1} = \prod_{d \geq 1} (1 - x^d)^{-N_d}
    \]
    \begin{proof}
      By the binomial theorem,
      \[
        (1 - x^d)^{-N_d} = \sum_{k \geq 0} \binom{-N_d}{k}(x^{d})^k.
      \]
      Note that $\binom{-N_d}{k}$ is the number of ways to pick a multiset of size $k$ from $N_d$ elements. Given a monic polynomial, we may view its factorization as a multiset of irreducible polynomials. Hence, $[x^{dk}](1 - x^d)^{-N_d}$ is the number of ways to pick a monic polynomial whose factorization is $k$ irreducible polynomials of degree $d$. But then
      \[
        [x^n]\prod_{d \geq 1} (1 - x^d)^{-N_d} = [x^n]\prod_{d \geq 1} \sum_{k \geq 0} \binom{-N_d}{k}(x^{d})^k
      \]
      is just the number monic polynomials of degree $n$. Since there are $n$ undetermined coefficients in a monic polynomial of degree $n$, there are $q^n$ monic polynomials of degree $n$. In other words,
      \[
        \prod_{d \geq 1} (1 - x^d)^{-N_d} = (1 + qx + q^2x^2 + \cdots) = (1 - q x)^{-1}.
      \]
    \end{proof}
    \item Take the logarithmic derivative of (a) and compare the coefficient of $x^{n-1}$ to get
    \[
    q^n = \sum_{d|n} d N_d.
    \]
    \begin{proof}
      \[
        \mathcal{L}((1 - qx)^{-1}) = q(1 - qx)(1 - qx)^{-2} = q(1 - qx)^{-1}.
      \]
      \begin{align*}
        \mathcal{L}\left(\prod_{d \geq 1} (1 - x^d)^{-N_d}\right) 
        &= \sum_{d \geq 1} N_d\mathcal{L}((1 - x^d)^{-1}) \\
        &= \sum_{d \geq 1} dN_dx^{d - 1}(1 - x^d)^{-1} \\
        &= \sum_{d \geq 1} dN_d(x^{d - 1} + x^{2d - 1} + x^{3d - 1} + \cdots).
      \end{align*}
      Hence,
      \begin{align*}
        q^n 
        &= [x^{n - 1}]q(1 - qx)^{-1} \\
        &= [x^{n - 1}]\sum_{d \geq 1} dN_dx^{d - 1}(1 - x^d)^{-1} \\
        &= [x^{n}]\sum_{d \geq 1} dN_d(x^{d} + x^{2d} + x^{3d} + \cdots) \\
        &= \sum_{d|n} dN_d.
      \end{align*}
    \end{proof}
    \item Use Möbius inversion to get a formula for $N_n$.
    \begin{proof}
      Since $q^n = \sum_{d|n} dN_d$, by Möbius inversion,
      \begin{align*}
        nN_d
        &= \sum_{d|n} q^d\mu(d, n),
      \end{align*}
      and so 
      \[
        N_d = \frac{1}{n}\sum_{d|n} q^d\mu(d, n).
      \]
    \end{proof}
  \end{enumerate}
\end{homeworkProblem}
\end{document}