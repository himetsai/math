\documentclass{article}

\usepackage{fancyhdr}
\usepackage{extramarks}
\usepackage{amsmath}
\usepackage{amsthm}
\usepackage{amsfonts}
\usepackage{tikz}
\usepackage[plain]{algorithm}
\usepackage{algpseudocode}
\usepackage{enumerate}
\usepackage{amssymb}
\usepackage{dsfont}
\usepackage{hyperref}

\usetikzlibrary{automata,positioning}

%
% Basic Document Settings
%

\topmargin=-0.45in
\evensidemargin=0in
\oddsidemargin=0in
\textwidth=6.5in
\textheight=9.0in
\headsep=0.25in

\linespread{1.1}

\pagestyle{fancy}
\lhead{\hmwkAuthorName}
\chead{\hmwkClass:\ \hmwkTitle}
\rhead{\firstxmark}
\lfoot{\lastxmark}
\cfoot{\thepage}

\renewcommand\headrulewidth{0.4pt}
\renewcommand\footrulewidth{0.4pt}

\setlength\parindent{0pt}
\setlength{\parskip}{5pt}

%
% Create Problem Sections
%

\newcommand{\enterProblemHeader}[1]{ \nobreak\extramarks{}{Problem \arabic{#1} continued on next
    page\ldots}\nobreak{} \nobreak\extramarks{Problem \arabic{#1} (continued)}{Problem \arabic{#1}
    continued on next page\ldots}\nobreak{} }

\newcommand{\exitProblemHeader}[1]{ \nobreak\extramarks{Problem \arabic{#1} (continued)}{Problem
    \arabic{#1} continued on next page\ldots}\nobreak{}
    \stepcounter{#1}
    \nobreak\extramarks{Problem \arabic{#1}}{}\nobreak{}
}

\setcounter{secnumdepth}{0}
\newcounter{partCounter}
\newcounter{homeworkProblemCounter}
\setcounter{homeworkProblemCounter}{1}
\nobreak\extramarks{Problem \arabic{homeworkProblemCounter}}{}\nobreak{}

%
% Homework Problem Environment
%
% This environment takes an optional argument. When given, it will adjust the
% problem counter. This is useful for when the problems given for your
% assignment aren't sequential. See the last 3 problems of this template for an
% example.
%
\newenvironment{homeworkProblem}[1][-1]{
    \ifnum#1>0
        \setcounter{homeworkProblemCounter}{#1}
    \fi
    \section{Problem \arabic{homeworkProblemCounter}}
    \setcounter{partCounter}{1}
    \enterProblemHeader{homeworkProblemCounter}
}{
    \exitProblemHeader{homeworkProblemCounter}
}

%
% Homework Details
%   - Title
%   - Due date
%   - Class
%   - Section/Time
%   - Instructor
%   - Author
%

\newcommand{\hmwkTitle}{Homework\ \#4}
\newcommand{\hmwkDueDate}{May 15, 2024}
\newcommand{\hmwkClass}{MATH 188}
\newcommand{\hmwkClassInstructor}{Professor Kunnawalkam Elayavalli}
\newcommand{\hmwkAuthorName}{\textbf{Ray Tsai}}
\newcommand{\hmwkPID}{A16848188}

%
% Title Page
%

\title{
    \vspace{2in}
    \textmd{\textbf{\hmwkClass:\ \hmwkTitle}}\\
    \normalsize\vspace{0.1in}\small{Due\ on\ \hmwkDueDate\ at 23:59pm}\\
    \vspace{0.1in}\large{\textit{\hmwkClassInstructor}} \\
    \vspace{3in}
}

\author{
  \hmwkAuthorName \\
  \vspace{0.1in}\small\hmwkPID
}
\date{}

\renewcommand{\part}[1]{\textbf{\large Part \Alph{partCounter}}\stepcounter{partCounter}\\}

%
% Various Helper Commands
%

% Useful for algorithms
\newcommand{\alg}[1]{\textsc{\bfseries \footnotesize #1}}

% For derivatives
\newcommand{\deriv}[1]{\frac{\mathrm{d}}{\mathrm{d}x} (#1)}

% For partial derivatives
\newcommand{\pderiv}[2]{\frac{\partial}{\partial #1} (#2)}

% Integral dx
\newcommand{\dx}{\mathrm{d}x}

% Probability commands: Expectation, Variance, Covariance, Bias
\newcommand{\Var}{\mathrm{Var}}
\newcommand{\Cov}{\mathrm{Cov}}
\newcommand{\Bias}{\mathrm{Bias}}
\newcommand*{\Z}{\mathbb{Z}}
\newcommand*{\Q}{\mathbb{Q}}
\newcommand*{\R}{\mathbb{R}}
\newcommand*{\C}{\mathbb{C}}
\newcommand*{\N}{\mathbb{N}}
\newcommand*{\p}{\mathds{P}}
\newcommand*{\E}{\mathds{E}}

\begin{document}

\maketitle

\pagebreak

\begin{homeworkProblem}
  \begin{enumerate}[(a)]
    \item Let $r$ be a fixed nonnegative integer. Show that both $S(n+r, n)$ and $c(n+r, n)$ are polynomial functions of $n$ of degree $2r$ for $n \geq 0$.
    \begin{proof}
      We first prove the case for $S(n + r, n)$. Consider the number $k$ of non-singleton blocks in a partition of $[n + r]$ with $n$ blocks. To count the number of partitions with exactly $k$ non-singleton blocks, we first pick the $n - k$ elements from $[n + r]$ that are in singletons, and then we calculate the number $a_{k, r}$ of possible orientations of the remaining $k + r$ elements. Note that $a_{k, r}$ is not dependent on $n$. Hence, summing over all possible $k$, we have
      \[
        S(n + r, n) = \sum_{k = 1}^{r} \binom{n + r}{n - k}a_{k, r} = \sum_{k = 1}^{r} \binom{n + r}{r + k}a_{k, r}.
      \]
      But then $\binom{n + r}{r + k}a_{k, r}$ is a polynomial of $n$ of degree $r + k$. Since $r + k$ goes up to $2r$ exactly once, $S(n + r, n)$ is a polynomial of $n$ of degree $2r$.
      
      The similar argument works for $c(n + r, n)$. Consider the number $k$ of non-trivial cycles in a permutation of size $n + r$ with $n$ disjoint cycles. To count the number of permutation with exactly $k$ non-trivial cycles, we first pick the $n - k$ elements from $[n + r]$ such that each of them are cycles on its own, and then we calculate the number $a_{k, r}$ of possible cycle formations of the remaining $k + r$ elements. Note that $b_{k, r}$ is not dependent on $n$. Hence, summing over all possible $k$, we have
      \[
        c(n + r, n) = \sum_{k = 1}^{r} \binom{n + r}{n - k}b_{k, r} = \sum_{k = 1}^{r} \binom{n + r}{r + k}b_{k, r}.
      \]
      But then $\binom{n + r}{r + k}b_{k, r}$ is a polynomial of $n$ of degree $r + k$. Since $r + k$ goes up to $2r$ exactly once, $c(n + r, n)$ is a polynomial of $n$ of degree $2r$.
    \end{proof}
    \item Compute these polynomials for $r = 2, 3$.
    \begin{proof}
      We first compute $S(n + r, n)$ for $r = 2, 3$. When $r = 2$, there are either $1$ or $2$ non-singleton blocks. If there is only one non-singleton block, then $3$ elements are in a block and the remaining elements each form a singleton, which has $\binom{n + 2}{3}$ possibilities. If there are $2$ non-singleton blocks, then there are $2$ blocks of size 2 and $n - 2$ singletons, which has $3\binom{n + 2}{4}$ possibilities. Hence, $S(n + 2, n) = \binom{n + 2}{3} + 3\binom{n + 2}{4}$.When $r = 3$, the number of non-singleton blocks ranges from $1$ to $3$. If there is only one non-singleton block, then $4$ elements are in a block and the remaining elements each form a singleton, which has $\binom{n + 3}{4}$ possibilities. If there are $2$ non-singleton blocks, then there is a block of size 2, a block of size 3, and $n - 2$ singletons, which has $\binom{5}{2}\binom{n + 3}{5} = 10\binom{n + 3}{5}$ possibilities. If there are $3$ non-singleton blocks, then there are 3 blocks of size 2 and
      all singletons for the rest, which has $\frac{1}{3!}\binom{6}{2}\binom{4}{2}\binom{n + 3}{6} = 15\binom{n + 3}{6}$ possibilities. Hence, $S(n + 3, n) = \binom{n + 3}{4} + 10\binom{n + 3}{5} + 15\binom{n + 3}{6}$.

      We now compute $c(n + r, n)$ for $r = 2, 3$. When $r = 2$, there are either $1$ or $2$ non-trivial cycles. If there is only one non-trivial cycle, then there is a $3$-cycle and $n - 1$ singletons, which has $2\binom{n + 2}{3}$ possibilities. If there are $2$ non-trivial cycles, then there are $2$ transpositions and $n - 2$ singletons, which has $3\binom{n + 2}{4}$ possibilities. Hence, $c(n + 2, n) = 2\binom{n + 2}{3} + 3\binom{n + 2}{4}$. When $r = 3$, the number of non-trivial cycles ranges from $1$ to $3$. If there is only one non-trivial cycle, then there is a $4$-cycle and $n - 1$ singletons, which has $3!\binom{n + 3}{4}$ possibilities. If there are $2$ non-trivial cycles, then there is a transposition, a $3$-cycle, and $n - 2$ singletons, which has $2\binom{5}{2}\binom{n + 3}{5} = 20\binom{n + 3}{5}$ possibilities. If there are $3$ non-trivial cycles, then there are 3 transpositions and all singletons for the rest, which has $\frac{1}{3!}\binom{6}{2}\binom{4}{2}\binom{n +
      3}{6} = 15\binom{n + 3}{6}$ possibilities. Hence, $c(n + 3, n) = 6\binom{n + 3}{4} + 20\binom{n + 3}{5} + 15\binom{n + 3}{6}$.
    \end{proof}
  \end{enumerate}
\end{homeworkProblem}

\newpage

\begin{homeworkProblem}
  For $n > 0$, let $a_n$ be the number of partitions of $n$ such that every part appears at most twice, and let $b_n$ be the number of partitions of $n$ such that no part is divisible by $3$. Set $a_0 = b_0 = 1$. Show that $a_n = b_n$ for all $n$.

  \begin{proof}
    Let $A(x)$ be the generating function of $a_n$ and $B(x)$ be the generating function of $b_n$. Since $a_n$ is the number of partitions of $n$ such that every part appears at most twice, 
    \[
      A(x) = \sum_{n \geq 0} a_nx^n = \prod_{i \geq 1} (1 + x^i + x^{2i}),
    \]
    as we either choose 1, $x^i$, or $x^{2i}$ from the $i$th term, when multiplying out the right side. What we get then is $x^N$ where $N$ where $N$ is the sum of the $i$ where we chose $x^i$ or $x^{2i}$. But we get $x^N$ one time for every partition of $N$ into parts which repeat at most once, so the coefficient is $a_N$.

    On the other hand, since $b_n$ is the number of partitions of $n$ such that no part is divisible by $3$,
    \[
      B(x) = \sum_{n \geq 0} b_nx^n = \prod_{i \geq 1, 3 \nmid i} \frac{1}{1 - x^i} = \frac{\prod_{i \geq 1} \frac{1}{1 - x^i}}{\prod_{i \geq 1} \frac{1}{1 - x^{3i}}} = \prod_{i \geq 1}\frac{1 - x^{3i}}{1 - x^{i}}.
    \]
    But then notice that $1 + x^i + x^{2i} = \frac{1 - x^{3i}}{1 - x^i}$ for all $i$. Hence, 
    \[
      A(x) = \prod_{i \geq 1}\frac{1 - x^{3i}}{1 - x^{i}} = B(x),
    \]
    and the result now follows.
  \end{proof}
\end{homeworkProblem}

\newpage

\begin{homeworkProblem}
  Let $y$ be a variable. Prove the following generalization of Example 3.27:
  \[
    \prod_{i \geq 0} (1 + x^{2i+1} y) = \sum_{r \geq 0} \frac{x^{r^2} y^r}{(1-x^2)(1-x^4) \cdots (1-x^{2r})}
  \]

  \begin{proof}
    Notice that $[y^kx^n]\prod_{i \geq 0} (1 + x^{2i+1} y)$ is counting the number of partitions of $n$ with $k$ distinct odd parts, as the exponent of the $y$ term indicates the number of times we picked the $x^{2i + 1}y$ term when expanding the multiplication. On the other hand, from Example 3.27 we know
    \[
      [y^kx^n]\sum_{r \geq 0} \frac{x^{r^2} y^r}{(1-x^2)(1-x^4) \cdots (1-x^{2r})} = [y^kx^n]\sum_{r \geq 0}  y^r \left(x^{r^2} \sum_{n \geq 0} p_{\leq r}(n)x^{2n}\right)
    \] 
    is counting the number of self-conjugate partitions of $n$ with a Durfee square of size $k$. We now show that there is a bijection between the set of self-conjugate partitions with Durfee square of size $r$ and the set of partition with $r$ distinct odd parts. Given a self-conjugate partition of $n$ which has a Durfee square of size $r$, we may use the reversible transformation described in Theorem 3.26 to obtain a new partition of $n$ with $r$ distinct odd parts, and thus the bijection. The result now follows from the bijection.
  \end{proof}
\end{homeworkProblem}

\newpage

\begin{homeworkProblem}

  \begin{enumerate}[(a)]
    \item Use the following $q$-analogue of Pascal's identity (you don't need to prove it)
    \[
      \begin{bmatrix}
        n \\
        k
      \end{bmatrix}_q = q^k \begin{bmatrix}
        n - 1 \\
        k
      \end{bmatrix}_q + \begin{bmatrix}
        n - 1 \\
        k - 1
      \end{bmatrix}_q \quad \text{for } n \geq k > 0
    \]
    to show that if $d$ is a non-negative integer, then
    \[
    \sum_{n \geq 0} \begin{bmatrix}
      n + d \\
      n
    \end{bmatrix}_q x^n = \prod_{i=0}^d (1-q^ix)^{-1} = \frac{1}{(1-x)(1-qx) \cdots (1-q^dx)}
    \]

    \begin{proof}
      We proceed by induction on $d$. If $d = 0$, then $\sum_{n \geq 0} \begin{bmatrix} n \\
        n \end{bmatrix}_q x^n = \sum_{n \geq 0} x^n = \frac{1}{1-x},$ and the base case is done. Suppose $d \geq 1$. Then,
      \begin{align*}
        \sum_{n \geq 0} \begin{bmatrix}
          n + d \\
          n
        \end{bmatrix}_q x^n
        &= 1 + \sum_{n \geq 1} q^n\begin{bmatrix}
          n + (d - 1) \\
          n
        \end{bmatrix}_q x^n + \sum_{n \geq 1} \begin{bmatrix}
          (n - 1) + d \\
          n - 1
        \end{bmatrix}_q x^n \\
        &= \sum_{n \geq 0} \begin{bmatrix}
          n + (d - 1) \\
          n
        \end{bmatrix}_q (qx)^n + x\sum_{n \geq 0} \begin{bmatrix}
          n + d \\
          n
        \end{bmatrix}_q x^n \\
        &= \frac{1}{1 - x}\sum_{n \geq 0} \begin{bmatrix}
          n + (d - 1) \\
          n
        \end{bmatrix}_q (qx)^n \\
        &= \frac{1}{(1-x)(1-qx) \cdots (1-q^dx)},
      \end{align*}
      where the last equality follows from induction.
    \end{proof}
    
    \item Give a direct explanation (i.e., independent of the Schubert decomposition explanation from lecture) for why the coefficient of $x^n$ of the right side is the sum $\sum q^{|\lambda|}$ over all integer partitions $\lambda$ whose Young diagram fits in the $n \times d$ rectangle.
    \begin{proof}
      Note that
      \begin{align*}
        [x^n]\prod_{i=0}^d (1-q^ix)^{-1} 
        &= \sum_{\substack{(a_0, \dots, a_d) \in \Z^d \\ a_0 + \cdots + a_d = n}} q^{a_1 + \cdots + da_d}.
      \end{align*}
      We now show the bijection between the weak compositions of $n$ with $d + 1$ parts and the integer partitions $\lambda$ whose Young diagram fits in the $n \times d$ rectangle. 
      
      Given an integer partitions $\lambda$ whose Young diagram fits in the $n \times d$ rectangle, let $a_i$ be the number of parts of $\lambda$ that are equal to $i \geq 1$ and put $a_0 = n - a_1 - \cdots - a_d$. Then, $(a_0, a_1, \dots, a_d)$ is a weak composition of $n$.

      On the other hand, given a weak compositions $(a_0, \dots, a_d)$ of $n$, there is an integer partition $\lambda$ with $a_i$ number of $i$'s for all $i \geq 1$. Since each part of $\lambda$ is at most $d$ and $\ell(\lambda) \leq n$, the Young diagram of $\lambda$ fits in the $n \times d$ rectangle. 

      But then $a_1 + \cdots + da_d = |\lambda|$, and thus
      \[
        \sum_{\substack{(a_0, \dots, a_d) \in \Z^d \\ a_0 + \cdots + a_d = n}} q^{a_1 + \cdots + da_d} = \sum q^{|\lambda|}.
      \]
    \end{proof}
  \end{enumerate}
\end{homeworkProblem}

\newpage

\begin{homeworkProblem}
  Let $V, W$ be $\textbf{F}_q$-vector spaces with $\dim V = n$ and $\dim W = m$.
  \begin{enumerate}[(a)]
    \item How many linear maps $V \rightarrow W$ are there?
    \begin{proof}
      Consider the number of ways we can map the canonical basis vectors $e_1, \dots, e_n$ of $V$ to some vectors in $W$. Since there are $q^m$ choices of vectors for each $e_i$ to be sent to, there are $q^{mn}$ choices in total. Hence, there are $q^{mn}$ linear maps $V \to W$.
    \end{proof}
    \item Suppose $n \geq m$. How many surjective linear maps $V \rightarrow W$ are there?
    \begin{proof}
      By the universal property of a quotient and the First Isomorphism Theorem, any surjective linear map $\phi: V \to W$ corresponds to a unique induced isomorphism $u: V/ \text{Ker }\phi \to W$. Note that $\text{Ker }\phi$ is of $(n - m)$-dimension and the number of isomorphisms $V/ \text{Ker }\phi \to W$ is equal to $|\textbf{GL}_m(\textbf{F}_q)|$. Hence, there is a bijection between the set of surjective linear maps $V \rightarrow W$ and $\textbf{Gr}_{n - m}(\textbf{F}^n_q) \times \textbf{GL}_m(\textbf{F}_q)$. But then by Theorem 3.34 and 3.35,
      \[
        |\textbf{Gr}_{n - m}(\textbf{F}^n_q)| = \begin{bmatrix} 
          n \\
          m
        \end{bmatrix}_q, \quad |\textbf{GL}_m(\textbf{F}_q)| = \prod_{i = 0}^{m - 1} (q^m - q^i),
      \]
      and thus there are $\begin{bmatrix} n \\
        m \end{bmatrix}_q\prod_{i = 0}^{m - 1} (q^m - q^i) = \prod_{i = 0}^{m - 1} (q^n - q^i)$ surjective linear maps $V \to W$.
    \end{proof}
    \item Pick $k \leq \min(m, n)$. How many rank $k$ linear maps $V \rightarrow W$ are there?
    \begin{proof}
      By the universal property of a quotient and the First Isomorphism Theorem, any linear map $\phi: V \to W$ of rank $k$ corresponds to a unique induced isomorphism from $V/ K$ to some $k$-dimensional subspace $U$ of $W$, where $K$ is the kernel of $\phi$. Note that $K$ is of $(n - k)$-dimension and the number of isomorphisms $V/ K \to W$ is equal to $|\textbf{GL}_k(\textbf{F}_q)|$. Since there are $|\textbf{Gr}_{n - k}(\textbf{F}^n_q)|$ choices for $K$, $|\textbf{Gr}_{k}(\textbf{F}^m_q)|$ choices for $U$, and $|\textbf{GL}_k(\textbf{F}_q)|$ choices for isomorphisms $V/ K \to W$, there are 
      \[
        \begin{bmatrix} 
          n \\
          k 
        \end{bmatrix}_q\begin{bmatrix} 
          m \\
          k 
        \end{bmatrix}_q\prod_{i = 0}^{k - 1} (q^k - q^i) = \begin{bmatrix} 
          m \\
          k 
        \end{bmatrix}_q\prod_{i = 0}^{k - 1} (q^n - q^i)
      \]
      rank $k$ linear maps $V \to W$, by Theorem 3.34 and 3.35.
    \end{proof}
  \end{enumerate}
\end{homeworkProblem}

\newpage

\begin{homeworkProblem}
  Prove
  \[
    \sum_{n \geq 1} x^{n(n - 1)/2} = \prod_{n \geq 1} \frac{1 - x^{2n}}{1 - x^{2n - 1}}.
  \]

  \begin{proof}
    By the Jacobi triple product, 
    \[
      \prod_{n \geq 1} (1 - x^{2n})(1 + x^{2n - 1}y)(1 + x^{2n - 1}y^{-1}) = \sum_{n = -\infty}^{\infty} x^{n^2}y^{n}.
    \]
    Hence, substituting both $x$ and $y$ as $\sqrt{x}$, we have
    \begin{align*}
      \prod_{n \geq 1} (1 - x^{n})(1 + x^{n})(1 + x^{n - 1}) 
      &= \sum_{n = -\infty}^{\infty} x^{n(n + 1)/2} \\
      &= 1 + \sum_{n \geq 1} x^{n(n + 1)/2} + x^{n(n - 1)/2} \\
      &= \sum_{n \geq 0} x^{n(n + 1)/2} + \sum_{n \geq 1} x^{n(n - 1)/2} \\
      &= 2\sum_{n \geq 1} x^{n(n - 1)/2}.
    \end{align*}
    It now follows that
    \begin{align*}
      \sum_{n \geq 1} x^{n(n - 1)/2}
      &= \frac{1}{2}\prod_{n \geq 1} (1 - x^{n})(1 + x^{n})(1 + x^{n - 1}) \\
      &= \frac{1}{2}\prod_{n \geq 1} (1 - x^{2n})(1 + x^{n - 1}) \\
      &= \left(\prod_{n \geq 1} (1 - x^{2n})\right)\left(\frac{1}{2}\prod_{n \geq 0} (1 + x^{n})\right) \\
      &= \left(\prod_{n \geq 1} (1 - x^{2n})\right)\left(\prod_{n \geq 1} (1 + x^{n})\right) \\
      &= \prod_{n \geq 1} \frac{1 - x^{2n}}{1 - x^{2n - 1}},
    \end{align*}
    where the last step follows from Theorem 3.25.
  \end{proof}
  Source cited: \url{https://www.math.uwaterloo.ca/~dmjackso/CO630/JTPID.pdf}
\end{homeworkProblem}

\newpage

\begin{homeworkProblem}
  Pick integers satisfying $1 \leq k_1 < k_2 < \cdots < k_r \leq n$. Let $X$ be the set of subspaces $W_1, \dots, W_r$ of $F_q^n$ such that $\dim W_i = k_i$ for all $i$ and $W_i \subset W_{i+1}$ for $i < r$.
  \begin{enumerate}[(a)]
    \item Find a formula for $|X|$ by generalizing Example 3.39, i.e., use a $q$-analogue of a multinomial coefficient.
    \begin{proof}
      For any $n \geq k_r > \cdots > k_1 \geq 1$, we show that there are $\begin{bmatrix} n \\
        k_1, k_2 - k_1, \dots, n - k_r \end{bmatrix}_q$ ways of picking subspaces $W_1, \dots, W_r$ of $\textbf{F}^n_q$ by induction on $r$. If $r = 1$, it is obvious that there are $\begin{bmatrix} n \\
        k_1 \end{bmatrix}_q$ ways of picking $W_1$. Suppose $r \geq 2$. There are $\begin{bmatrix} n \\
        k_r \end{bmatrix}_q$ ways of picking $W_r$. But then by induction, there are $\begin{bmatrix} k_r \\
        k_1, k_2 - k_1, \dots, k_r - k_{r - 1} \end{bmatrix}_q$ ways of picking $W_1, \dots, W_{r - 1}$ which are contained in $W_r$. It now follows that there are 
      \[
        \begin{bmatrix} 
          n \\
          k_r
        \end{bmatrix}_q\begin{bmatrix} 
          k_r \\
          k_1, k_2 - k_1, \dots, k_r - k_{r - 1}
        \end{bmatrix}_q = \begin{bmatrix} 
          n \\
          k_1, k_2 - k_1, \dots, n - k_r
        \end{bmatrix}_q
      \]
      ways of picking $W_1, \dots, W_r$ of $\textbf{F}_q^n$.
    \end{proof}
    \item $|X|$ is also a polynomial in $q$; find an explicit description of this polynomial using a generalization of the Schubert decomposition of the Grassmannian.
    \begin{proof}
      By the Schubert decomposition of $\textbf{Gr}_k(\textbf{F}_q^n)$, we have
      \[
        \begin{bmatrix} 
          n \\
          k
        \end{bmatrix}_q = |\textbf{Gr}_k(\textbf{F}_q^n)| = \sum_{\lambda 
        \subseteq k \times (n - k)} q^{|\lambda|},
      \]
      where $\lambda$ is any integer partition whose Young diagram fits into the $k \times (n - k)$ box. It now follows that,
      \begin{align*}
        |X| 
        &= \begin{bmatrix} 
          n \\
          k_1, k_2 - k_1, \dots, n - k_r
        \end{bmatrix}_q \\
        &= \begin{bmatrix} 
          n \\
          k_r
        \end{bmatrix}_q\begin{bmatrix} 
          k_r \\
          k_{r - 1}
        \end{bmatrix}_q \cdots \begin{bmatrix} 
          k_2 \\
          k_1
        \end{bmatrix}_q \\
        &= \left(\sum_{\lambda 
        \subseteq k_r \times (n - k_r)} q^{|\lambda|}\right)\left(\sum_{\lambda 
        \subseteq k_{r - 1} \times (k_r - k_{r - 1})} q^{|\lambda|}\right) \cdots \left(\sum_{\lambda 
        \subseteq k_1 \times (k_2 - k_1)} q^{|\lambda|}\right).
      \end{align*}
    \end{proof}
  \end{enumerate}
\end{homeworkProblem}
\end{document}