\documentclass{article}

\usepackage{fancyhdr}
\usepackage{extramarks}
\usepackage{amsmath}
\usepackage{amsthm}
\usepackage{amsfonts}
\usepackage{tikz}
\usepackage[plain]{algorithm}
\usepackage{algpseudocode}
\usepackage{enumerate}
\usepackage{amssymb}
\usepackage{dsfont}

\usetikzlibrary{automata,positioning}

%
% Basic Document Settings
%

\topmargin=-0.45in
\evensidemargin=0in
\oddsidemargin=0in
\textwidth=6.5in
\textheight=9.0in
\headsep=0.25in

\linespread{1.1}

\pagestyle{fancy}
\lhead{\hmwkAuthorName}
\chead{\hmwkClass:\ \hmwkTitle}
\rhead{\firstxmark}
\lfoot{\lastxmark}
\cfoot{\thepage}

\renewcommand\headrulewidth{0.4pt}
\renewcommand\footrulewidth{0.4pt}

\setlength\parindent{0pt}
\setlength{\parskip}{5pt}

%
% Create Problem Sections
%

\newcommand{\enterProblemHeader}[1]{
    \nobreak\extramarks{}{Problem \arabic{#1} continued on next page\ldots}\nobreak{}
    \nobreak\extramarks{Problem \arabic{#1} (continued)}{Problem \arabic{#1} continued on next page\ldots}\nobreak{}
}

\newcommand{\exitProblemHeader}[1]{
    \nobreak\extramarks{Problem \arabic{#1} (continued)}{Problem \arabic{#1} continued on next page\ldots}\nobreak{}
    \stepcounter{#1}
    \nobreak\extramarks{Problem \arabic{#1}}{}\nobreak{}
}

\setcounter{secnumdepth}{0}
\newcounter{partCounter}
\newcounter{homeworkProblemCounter}
\setcounter{homeworkProblemCounter}{1}
\nobreak\extramarks{Problem \arabic{homeworkProblemCounter}}{}\nobreak{}

%
% Homework Problem Environment
%
% This environment takes an optional argument. When given, it will adjust the
% problem counter. This is useful for when the problems given for your
% assignment aren't sequential. See the last 3 problems of this template for an
% example.
%
\newenvironment{homeworkProblem}[1][-1]{
    \ifnum#1>0
        \setcounter{homeworkProblemCounter}{#1}
    \fi
    \section{Problem \arabic{homeworkProblemCounter}}
    \setcounter{partCounter}{1}
    \enterProblemHeader{homeworkProblemCounter}
}{
    \exitProblemHeader{homeworkProblemCounter}
}

%
% Homework Details
%   - Title
%   - Due date
%   - Class
%   - Section/Time
%   - Instructor
%   - Author
%

\newcommand{\hmwkTitle}{Homework\ \#2}
\newcommand{\hmwkDueDate}{Jan 22, 2025}
\newcommand{\hmwkClass}{MATH 190A}
\newcommand{\hmwkClassTime}{Section A02 8:00AM - 8:50AM}
\newcommand{\hmwkSectionLeader}{Zhiyuan Jiang}
\newcommand{\hmwkClassInstructor}{Professor McKernan}
\newcommand{\hmwkSource}{Source Consulted: Textbook, Lecture, Discussion}
\newcommand{\hmwkAuthorName}{\textbf{Ray Tsai}}
\newcommand{\hmwkPID}{A16848188}

%
% Title Page
%

\title{
    \vspace{2in}
    \textmd{\textbf{\hmwkClass:\ \hmwkTitle}}\\
    \normalsize\vspace{0.1in}\small{Due\ on\ \hmwkDueDate\ at 12:00pm}\\
    \vspace{0.1in}\large{\textit{\hmwkClassInstructor}} \\
    \vspace{0.1in}\small\hmwkClassTime \\
    \small Section Leader: \hmwkSectionLeader \\
    \vspace{0.1in}\small\hmwkSource \\
    \vspace{3in}
}

\author{
  \hmwkAuthorName \\
  \vspace{0.1in}\small\hmwkPID
}
\date{}

\renewcommand{\part}[1]{\textbf{\large Part \Alph{partCounter}}\stepcounter{partCounter}\\}

%
% Various Helper Commands
%

% Useful for algorithms
\newcommand{\alg}[1]{\textsc{\bfseries \footnotesize #1}}

% For derivatives
\newcommand{\deriv}[1]{\frac{\mathrm{d}}{\mathrm{d}x} (#1)}

% For partial derivatives
\newcommand{\pderiv}[2]{\frac{\partial}{\partial #1} (#2)}

% Integral dx
\newcommand{\dx}{\mathrm{d}x}

% Probability commands: Expectation, Variance, Covariance, Bias
\newcommand{\Var}{\mathrm{Var}}
\newcommand{\Cov}{\mathrm{Cov}}
\newcommand{\Bias}{\mathrm{Bias}}
\newcommand*{\Z}{\mathbb{Z}}
\newcommand*{\Q}{\mathbb{Q}}
\newcommand*{\R}{\mathbb{R}}
\newcommand*{\C}{\mathbb{C}}
\newcommand*{\N}{\mathbb{N}}
\newcommand*{\prob}{\mathds{P}}
\newcommand*{\E}{\mathds{E}}
\newcommand*{\T}{\mathcal{T}}

\begin{document}

\maketitle

\pagebreak

\begin{homeworkProblem}
	Let $X$ be a set. Give $X$ the topology $\mathcal{T}$ where every finite set is closed, plus $X$. If $Y$ is a subset of $X$ then determine
	\begin{enumerate}[(i)]
		\item The interior of $Y$.
		\begin{proof}
			If $X \backslash Y$ is finite, then $Y \in \T$ and so $\text{int}(Y) = Y$. Otherwise, $X \backslash Y$ is infinite and so $Y$ does not contain a set whose complement is finite. Thus, $\text{int}(Y) = \emptyset$.
		\end{proof}
		\item The closure of $Y$.
		\begin{proof}
			If $Y$ is finite, then $Y$ is closed so $Y = \overline{Y}$. If $Y$ is infinite, then there is no finite set that contains $Y$ so $\overline{Y} = X$. 
		\end{proof}
		\item The boundary of $Y$.

		\begin{proof}
			By (i) and (ii),
			\[
				\partial Y = \begin{cases}
					\emptyset & \text{if } Y \text{ is finite and $X \backslash Y$ is finite}  \\
					Y & \text{if } Y \text{ is finite and $X \backslash Y$ is infinite} \\
					X \backslash Y & \text{if } Y \text{ is infinite and $X \backslash Y$ is finite}  \\
					X & \text{if } Y \text{ is infinite and $X \backslash Y$ is infinite}
				\end{cases}
			\]
		\end{proof}
	\end{enumerate}
\end{homeworkProblem}

\newpage

\begin{homeworkProblem}
	Let $a < b \in \mathbb{R}$ and let $Y = [a, b) \subset \mathbb{R}$. What is
	\begin{enumerate}[(i)]
		\item The interior of $Y$?
		\begin{proof}
			Obviously, the largest open set contained in $Y$ is $(a, b)$, so $\text{int}(Y) = (a, b)$.
		\end{proof}
		\item The closure of $Y$?
		\begin{proof}
			Obviously, the smalled closed set contained in $Y$ is $[a, b]$, so $\overline{Y} = [a, b]$.
		\end{proof}
		\item The boundary of $Y$?
		\begin{proof}
			\[
				\partial Y = \overline{Y} \backslash \text{int}(Y) = \{a, b\}.
			\]
		\end{proof}
	\end{enumerate}
\end{homeworkProblem}

\newpage

\begin{homeworkProblem}
	Let $(X, d)$ be a metric space, let $a \in X$ be a point of $X$ and let $r > 0$ be a positive real. Let
	\[
		B = \overline{B_r}(a) = \{x \in X \mid d(a, x) \leq r\}.
	\]
	Show that $B$ is closed in $X$. $\overline{B_r}(a)$ is called the \emph{closed ball of radius $r$ centred about $a$}.

	\begin{proof}
		Let $x \in X \backslash B$. Let $\epsilon = d(a, x) - r > 0$. Then $B_{\epsilon}(x) \subset X \backslash B$, so $X \backslash B$ is covered by a collection of open sets. Thus, $X \backslash B$ is open and the result follows.
	\end{proof}
\end{homeworkProblem}

\newpage

\begin{homeworkProblem}
	True or false? If true then give a proof and if false then give a counterexample.
	\begin{enumerate}[(i)]
		\item Let $(X, d)$ be a metric space, let $a \in X$ be a point of $X$ and let $r > 0$ be a positive real. Let $Y = B_r(a)$ be the \emph{open ball of radius $r$ centred about $a$}. Then the closure of $Y$ is the closed ball of radius $r$ centred about $a$.
		
		\begin{proof}
			True. By problem 3, $\overline{B}_r(a)$ is closed, so it suffices to show $\overline{B}_r(a)$ is the smallest closed set that contains $B_r(a)$. Suppose not. Let $C$ be a closed set such that $B_r(a) \subset C \subset \overline{B}_r(a)$. Then there exists some $x$ such that $d(a, x) = r$ and $x \notin C$. But then there does not exist $s > 0$ such that $B_s(x) \cap B_r(a) \neq \emptyset$. Thus $X \backslash C$ is not open and so $C$ is not closed, contradiction.
		\end{proof}

		\item If $(X, \mathcal{T})$ is a topological space and $Y \subset X$ is a subset then
		\[
		\overline{X \setminus Y} = X \setminus \operatorname{int}(Y).
		\]

		\begin{proof}
			True. Since $C = \overline{X \setminus Y}$ is the smalled closed set that contains $X \setminus Y$, we know $X \backslash C$ is the largest open set that contains $Y$. Thus, $X \backslash C = \operatorname{int}(Y)$ and the result follows.
		\end{proof}

		\item If $(X, \mathcal{T})$ is a topological space and $Y \subset X$ and $Z \subset X$ are two subsets then
		\[
		\operatorname{int}(Y \cup Z) = \operatorname{int}(Y) \cup \operatorname{int}(Z).
		\]

		\begin{proof}
			False. Let $X = \R$ and $Y = [0, 1]$ and $Z = [1, 2]$. Then $\operatorname{int}(Y \cup Z) = (0, 2)$ but $\operatorname{int}(Y) \cup \operatorname{int}(Z) = (0, 1) \cup (1, 2) = (0, 1) \cup [1, 2] = (0, 2)$.
		\end{proof}
		\item The integers $\mathbb{Z}$ are dense in the reals $\mathbb{R}$.
		\begin{proof}
			False. Let $x \in \R \backslash \Z$. There exists some $n \in \Z$ such that $n < x < n + 1$. Let $r = \min(x - n, n - x + 1)$. Then $B_{r}(x) \cap \Z = \emptyset$, so $\Z$ is not dense in $\R$ by lemma 4.12.
		\end{proof}
		\item The rationals $\mathbb{Q}$ are dense in the reals $\mathbb{R}$.
		\begin{proof}
			True. Let $x \in \R$ and $r > 0$. Fix $q$ such that $\frac{1}{q} < r$. There exists some $p$ such that $\frac{p - 1}{q} \leq x \leq \frac{p}{q} < x + r$. Thus $\frac{p}{q} \in B_r(x) \cap \Q$, so $\Q$ is dense.
		\end{proof}
	\end{enumerate}
\end{homeworkProblem}

\newpage

\begin{homeworkProblem}
	Show that every open subset of $\mathbb{R}$ is a disjoint union of open intervals.

	\begin{proof}
		Let $U$ be an open set of $\R$. Then $U$ is a union of a collection of open intervals. Suppose $I_1 = (a, b)$ and $I_2 = (c, d)$ are two open intervals in $U$ such that $I_1 \cap I_2 \neq \emptyset$. Then $I_1 \cup I_2 = (\min(a, c), \max(b, d))$ is an open interval contained in $U$. Thus, $U$ is a disjoint union of open intervals.
	\end{proof}
\end{homeworkProblem}

\newpage

\begin{homeworkProblem}
	Let $(X, \mathcal{T})$ be a topological space. Starting with any subset $Y \subset X$ (and any $X$) what is the maximum number of distinct subsets one can obtain by taking the closure and the complement (as many times as you please, in whatever order you please)?

	\begin{proof}
		14?
	\end{proof}
\end{homeworkProblem}

\end{document}