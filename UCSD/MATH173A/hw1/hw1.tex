\documentclass{article}

\usepackage{fancyhdr}
\usepackage{extramarks}
\usepackage{amsmath}
\usepackage{amsthm}
\usepackage{amsfonts}
\usepackage{tikz}
\usepackage[plain]{algorithm}
\usepackage{algpseudocode}
\usepackage{enumerate}
\usepackage{amssymb}
\usepackage{mathtools}

\usetikzlibrary{automata,positioning}

%
% Basic Document Settings
%

\topmargin=-0.45in
\evensidemargin=0in
\oddsidemargin=0in
\textwidth=6.5in
\textheight=9.0in
\headsep=0.25in

\linespread{1.1}

\pagestyle{fancy}
\lhead{\hmwkAuthorName}
\chead{\hmwkClass:\ \hmwkTitle}
\rhead{\firstxmark}
\lfoot{\lastxmark}
\cfoot{\thepage}

\renewcommand\headrulewidth{0.4pt}
\renewcommand\footrulewidth{0.4pt}

\setlength\parindent{0pt}
\setlength{\parskip}{5pt}

%
% Create Problem Sections
%

\newcommand{\enterProblemHeader}[1]{
    \nobreak\extramarks{}{Problem \arabic{#1} continued on next page\ldots}\nobreak{}
    \nobreak\extramarks{Problem \arabic{#1} (continued)}{Problem \arabic{#1} continued on next page\ldots}\nobreak{}
}

\newcommand{\exitProblemHeader}[1]{
    \nobreak\extramarks{Problem \arabic{#1} (continued)}{Problem \arabic{#1} continued on next page\ldots}\nobreak{}
    \stepcounter{#1}
    \nobreak\extramarks{Problem \arabic{#1}}{}\nobreak{}
}

\setcounter{secnumdepth}{0}
\newcounter{partCounter}
\newcounter{homeworkProblemCounter}
\setcounter{homeworkProblemCounter}{1}
\nobreak\extramarks{Problem \arabic{homeworkProblemCounter}}{}\nobreak{}

%
% Homework Problem Environment
%
% This environment takes an optional argument. When given, it will adjust the
% problem counter. This is useful for when the problems given for your
% assignment aren't sequential. See the last 3 problems of this template for an
% example.
%
\newenvironment{homeworkProblem}[1][-1]{
    \ifnum#1>0
        \setcounter{homeworkProblemCounter}{#1}
    \fi
    \section{Problem \arabic{homeworkProblemCounter}}
    \setcounter{partCounter}{1}
    \enterProblemHeader{homeworkProblemCounter}
}{
    \exitProblemHeader{homeworkProblemCounter}
}

%
% Homework Details
%   - Title
%   - Due date
%   - Class
%   - Section/Time
%   - Instructor
%   - Author
%

\newcommand{\hmwkTitle}{Homework\ \#1}
\newcommand{\hmwkDueDate}{Oct 15, 2024}
\newcommand{\hmwkClass}{MATH 173A}
\newcommand{\hmwkClassInstructor}{Professor Cloninger}
\newcommand{\hmwkAuthorName}{\textbf{Ray Tsai}}
\newcommand{\hmwkPID}{A16848188}

%
% Title Page
%

\title{
    \vspace{2in}
    \textmd{\textbf{\hmwkClass:\ \hmwkTitle}}\\
    \normalsize\vspace{0.1in}\small{Due\ on\ \hmwkDueDate\ at 23:59pm}\\
    \vspace{0.1in}\large{\textit{\hmwkClassInstructor}} \\
    \vspace{3in}
}

\author{
  \hmwkAuthorName \\
  \vspace{0.1in}\small\hmwkPID
}
\date{}

\renewcommand{\part}[1]{\textbf{\large Part \Alph{partCounter}}\stepcounter{partCounter}\\}

%
% Various Helper Commands
%

% Useful for algorithms
\newcommand{\alg}[1]{\textsc{\bfseries \footnotesize #1}}

% For derivatives
\newcommand{\deriv}[1]{\frac{\mathrm{d}}{\mathrm{d}x} (#1)}

% For partial derivatives
\newcommand{\pderiv}[2]{\frac{\partial}{\partial #1} (#2)}

% Integral dx
\newcommand{\dx}{\mathrm{d}x}

% Probability commands: Expectation, Variance, Covariance, Bias
\newcommand{\Var}{\mathrm{Var}}
\newcommand{\Cov}{\mathrm{Cov}}
\newcommand{\Bias}{\mathrm{Bias}}
\newcommand*{\Z}{\mathbb{Z}}
\newcommand*{\Q}{\mathbb{Q}}
\newcommand*{\R}{\mathbb{R}}
\newcommand*{\C}{\mathbb{C}}
\newcommand*{\N}{\mathbb{N}}
\newcommand*{\prob}{\mathds{P}}
\newcommand*{\E}{\mathds{E}}

\begin{document}

\maketitle

\pagebreak

\begin{homeworkProblem}
	Use the definition of convex functions to answer the following:

	\begin{enumerate}[(a)]
		\item Show that $f : \mathbb{R}^d \to \mathbb{R}$ given by $f(x_1, \ldots, x_d) = \|x\|_2^2 = \sum_{i=1}^{d} x_i^2$ is convex.

    \begin{proof}
      $f$ is continuously differentiable, with $\nabla f(x) = 2x$. But then, for any $x, y \in \mathbb{R}^d$, 
      \[
        f(x) + \nabla f(x)^T(y - x) = x^Tx + 2x^T(y - x) = 2x^Ty - x^Tx = \|y\|^2_2 -\|x - y\|_2^2 \leq f(y),
      \]
      so $f$ is convex.
    \end{proof}
		
		\item Show that $f : \mathbb{R} \to \mathbb{R}$ given by $f(x) = |x|$ is convex.

    \begin{proof}
      Let $x, y \in \R$. By the triangle inequality, 
      \[
        f(tx + (1 - t)y) = |tx + (1 - t)y| \leq t|x| + (1 - t)|y| = tf(x) + (1 - t)f(y),
      \]
      for any $t \in [0, 1]$. The result now follows.
    \end{proof}
		
		\item For (b), show that $f$ is not strictly convex.

		\begin{proof}
      Consider $x, y \geq 0$. Then,
      \[
        f(tx + (1 - t)y) = tx + (1 - t)y = tf(x) + (1 - t)f(y),
      \]
      for all $t \in [0, 1]$. Hence, $f$ is not strictly convex.
    \end{proof}
		
		\item Show that $f : \mathbb{R} \to \mathbb{R}$ given by $f(x) = \sqrt{|x|}$ is not convex.
		\begin{proof}
      Let $x = 1, y = 4$. Take $t = \frac{1}{2}$. Then,
      \[
        f\left(t \cdot 1 + (1 - t) \cdot 4\right) = f\left(\frac{5}{2}\right) = \sqrt{\frac{5}{2}},
      \]
      but
      \[
        tf(1) + (1 - t)f(4) = \frac{1}{2} + 1 = \frac{3}{2} < \sqrt{\frac{5}{2}}.
      \]
      Hence, $f$ is not convex.
    \end{proof}
	\end{enumerate}
\end{homeworkProblem}

\newpage

\begin{homeworkProblem}
  Use the definition of convex sets to answer the following:

  \begin{enumerate}[(a)]
    \item Show that if the sets $S$ and $T$ are convex, then $S \cap T$ is convex.
    \begin{proof}
      Let $x, y \in S \cap T$. Since $S$ and $T$ are convex, for any $t \in [0, 1]$, $tx + (1 - t)y \in S$ and $tx + (1 - t)y \in T$. But then $tx + (1 - t)y \in S \cap T$, so $S \cap T$ is convex.
    \end{proof}
    
    \item Show that the intersection of any number of convex sets is convex.
    \begin{proof}
      Let $S_1, S_2, \ldots, S_n$ be convex sets. We proceed by induction on $n \geq 2$. (a) yields the base case. For $n > 2$, $S_1 \cap S_2 \cap \ldots \cap S_{n - 1}$ is convex by induction, and thus $S_1 \cap S_2 \cap \ldots \cap S_{n - 1} \cap S_n$ is convex by (a).
    \end{proof}
    
    \item A hyperplane in $\mathbb{R}^d$ is a set of points of the form $\{x : a^T x = b\}$ where $a \in \mathbb{R}^d$ and $b \in \mathbb{R}$. Show that hyperplanes are convex.

    \begin{proof}
      Let $\Gamma$ be a hyperplane $\{x : a^T x = b\}$ in $\R^d$. Let $x, y \in \Gamma$. Then, for any $t \in [0, 1]$, 
      \[
        a^T(tx + (1 - t)y) = t(a^Tx) + (1 - t)(a^Ty) = tb + (1 - t)b = b,
      \]
      so $tx + (1 - t)y \in \Gamma$. Hence, $\Gamma$ is convex.
    \end{proof}
  \end{enumerate}
\end{homeworkProblem}

\newpage

\begin{homeworkProblem}
  Use the definition of convex functions and sets to answer the following. Let $f : \mathbb{R}^n \to \mathbb{R}$ be a function and define the set

  \[
    E_f = \{(x, w) \in \mathbb{R}^{n+1} \mid x \in \mathbb{R}^n, \, w \in \mathbb{R}, \, f(x) \leq w\}.
  \]

  \begin{enumerate}[(a)]
    \item Show that for all $x \in \mathbb{R}^n$, $(x, f(x)) \in E_f$.
    \begin{proof}
      Put $w = f(x)$. Since $w = f(x) \geq f(x)$, $(x, f(x)) \in E_f$.
    \end{proof}
    
    \item Show that if $f$ is a convex function, then $E_f$ is a convex set.
    \begin{proof}
      Let $(x_1, w_1), (x_2, w_2) \in E_f$. Since $f$ is convex, for $t \in [0, 1]$.
      \[
        f(tx_1 + (1 - t)x_2) \leq tf(x_1) + (1 - t)f(x_2) \leq tw_1 + (1 - t)w_2,
      \]
      But then $(tx_1 + (1 - t)x_2, tw_1 + (1 - t)w_2) \in E_f$, so $E_f$ is convex.
    \end{proof}
    
    \item Show conversely that if $E_f$ is a convex set, then $f$ is a convex function.

    \begin{proof}
      Let $x_1, x_2 \in \R^n$. Since $E_f$ is convex, 
      \[
        t(x_1, f(x_1)) + (1 - t)(x_2, f(x_2)) = (tx_1 + (1 - t)x_2, tf(x_1) + (1 - t)f(x_2)) \in E_f
      \]
      for all $t \in [0, 1]$. But then $f(tx_1 + (1 - t)x_2) \leq tf(x_1) + (1 - t)f(x_2)$, so $f$ is convex.
    \end{proof}
  \end{enumerate}
\end{homeworkProblem}

\newpage

\begin{homeworkProblem}
  Find the gradient and Hessian of the following functions, and determine whether the functions are convex.

  \begin{enumerate}[(a)]
      \item $f : \mathbb{R}^2 \to \mathbb{R}$ given by $f(x_1, x_2) = \frac{1}{2}x_1^4 + x_1x_2 - e^{x_2}$.
      \begin{proof}
        \[
          \nabla f(x) = (2x_1^3 + x_2, x_1 - e^{x_2}), \quad
          \nabla^2 f(x) = \begin{bmatrix}
            6x_1^2 & 1 \\
            1 & -e^{x_2}
          \end{bmatrix}.
        \]
        Since $\det(\nabla^2 f(0, 0)) = -1 < 0$, $\nabla^2 f(x)$ is not positive semidefinite. It now follows that $f$ is not convex, as $f$ is twice continuously differentiable.
      \end{proof}
      
      \item $f : \mathbb{R}^d \to \mathbb{R}$ given by $f(x) = \langle a, x \rangle^2 + \langle b, x \rangle$.

      \begin{proof}
        Since $f(x) = (a^Tx)^2 + b^Tx$, using the chain rule we have
        \[
          \nabla f(x) = 2a^Txa + b, \quad \nabla^2 f(x) = 2aa^T.
        \]
        Since $x^T\nabla^2 f(x)x^T = (a^Tx)^T(a^Tx) = \|a^Tx\|^2 \geq 0$ for all $x \in \R^d$, $\nabla^2 f(x)$ is positive semidefinite, so $f$ is convex. It now follows that $f$ is not convex, as $f$ is twice continuously differentiable.
      \end{proof}
  \end{enumerate}
\end{homeworkProblem}

\newpage

\begin{homeworkProblem}
  For each problem below, find the gradient and show your work.
  \begin{enumerate}[(a)]
    \item $f:\mathbb{R}^n\rightarrow\mathbb{R}$ for $f(x) = \|x\|_2^2$.
    \begin{proof}
      Since $f(x) = x^Tx$,
      \[
        \nabla f(x) = x^T(I + I^T) = 2x.
      \]
    \end{proof}

    \item $f:\mathbb{R}^n\rightarrow\mathbb{R}$ for $f(x) = \|Ax\|_2^2$ where $A\in\mathbb{R}^{m\times n}$. 

    \begin{proof}
      Since $f(x) = x^T(A^TA)x$,
      \[
        \nabla f(x) = 2x^T(A^TA) = 2(Ax)^TA = 2A^TAx.
      \]
    \end{proof}

    \item $f:\mathbb{R}^n\rightarrow\mathbb{R}$ for $f(x) = \|Ax-b\|_2^2$ for $A\in\mathbb{R}^{m\times n}$ and $b\in \mathbb{R}^m$.

    \begin{proof}
      Since $f(x) = (Ax - b)^T(Ax - b) = \|Ax\|^2_2 - 2b^TAx + \|b\|^2_2$, by (b)
      \[
        \nabla f(x) = 2A^TAx - 2A^Tb = 2A^T(Ax - b).
      \]
    \end{proof}

    \item $f:\mathbb{R}^n\rightarrow\mathbb{R}$ for $f(x) = \|Ax-b\|_2^2 + \gamma \|x\| _2^2$ for $A\in\mathbb{R}^{m\times n}$ and $b\in \mathbb{R}^m$ and $\gamma>0$.

    \begin{proof}
      By (a) and (c),
      \[
        \nabla f(x) = 2A^T(Ax - b) + 2\gamma x = 2(A^TA + \gamma I)x - 2A^Tb. 
      \]
    \end{proof}
  \end{enumerate}
\end{homeworkProblem}

\newpage

\begin{homeworkProblem}
  This problem builds on the results from problem 5.
  \begin{enumerate}[(a)]
    \item For part 5(c), use the Hessian of $f(x)$ to show that $f$ is convex. Under what conditions is $f$ strictly convex?
    \begin{proof}
      By chain rule, $\nabla^2 f = 2A^TA$. But then
      \[
        x^T(\nabla^2 f)x = 2(Ax)^TAx = 2\|Ax\|^2_2 \geq 0
      \]
      for all $x \in \R^n$, so $\nabla^2 f(x)$ is positive semidefinite, and thus $f$ is convex. $f$ strictly convex when $x^T\nabla^2 f(x)x = 2\|Ax\|^2_2 > 0$ for all $x \neq 0$. This is true when $A$ has full rank.
    \end{proof}
    \item For 5(d), show that $f(x)$ is always strictly convex.
    \begin{proof}
      It suffices to show that $\nabla^2 f = 2(A^TA + \gamma I)$ is positive definite. Since $\gamma > 0$
      \[
        x^T(\nabla^2 f)x = 2(Ax)^TAx + \gamma x^Tx = 2\|Ax\|^2_2 + \gamma \|x\|_2^2 > 0,
      \]
      for all $x \neq 0$. The result now follows.
    \end{proof}
  \end{enumerate}
\end{homeworkProblem}
\end{document}
