\documentclass{article}

\usepackage{fancyhdr}
\usepackage{extramarks}
\usepackage{amsmath}
\usepackage{amsthm}
\usepackage{amssymb}
\usepackage{amsfonts}
\usepackage{tikz}
\usepackage[plain]{algorithm}
\usepackage{algpseudocode}
\usepackage{enumitem}
\usepackage{cancel}

\usetikzlibrary{automata,positioning}

%
% Basic Document Settings
%

\topmargin=-0.45in
\evensidemargin=0in
\oddsidemargin=0in
\textwidth=6.5in
\textheight=9.0in
\headsep=0.25in

\linespread{1.1}

\pagestyle{fancy}
\lhead{\hmwkAuthorName}
\chead{\textsc{\hmwkHeader}}
\cfoot{\thepage}

\renewcommand\headrulewidth{0.4pt}
\renewcommand\footrulewidth{0.4pt}

\setlength\parindent{0pt}
\setlength{\parskip}{5pt}

\newenvironment{topic}[1]{\subsection*{#1}}{}
\newenvironment{lemma}[1]{\subsection*{Lemma #1.}}{}
\newenvironment{defn}{\subsection*{Definition.}}{}
\newenvironment{defnlemma}[1]{\subsection*{Definition-Lemma #1.}}{}
\newenvironment{corollary}[1]{
    \def\temp{#1}\def\null{&}\ifx\temp\null
        \subsection*{Corollary.}
    \else
        \subsection*{Corollary #1.}
    \fi
    
}{}

%
% Homework Details
%   - Title
%   - Due date
%   - Class
%   - Section/Time
%   - Instructor
%   - Author
%


\newcommand{\hmwkInstitution}{University of California San Diego}
\newcommand{\hmwkTitle}{\textsc{ECE 271 Notes}}
\newcommand{\hmwkHeader}{Statistical Learning Notes}
\newcommand{\hmwkInstructor}{Prof. Nuno Vasconcelos}
\newcommand{\hmwkAuthorName}{Ray Tsai}

%
% Title Page
%
\title{
    \vspace{2in}
    \textsc{\Large\hmwkInstitution} \\
    \vspace{0.2in}
    \textmd{\textbf{\hmwkTitle}}\\
    \vspace{0.2in}\large{Instructor: \textit{\hmwkInstructor}}
}

\author{
  Organized by \hmwkAuthorName
}
\date{}

\renewcommand{\part}[1]{\textbf{\large Part \Alph{partCounter}}\stepcounter{partCounter}\\}

%
% Various Helper Commands
%

% Useful for algorithms
\newcommand{\alg}[1]{\textsc{\bfseries \footnotesize #1}}

% For derivatives
\newcommand{\deriv}[1]{\frac{\mathrm{d}}{\mathrm{d}x} (#1)}

% For partial derivatives
\newcommand{\pderiv}[2]{\frac{\partial}{\partial #1} (#2)}

% Integral dx
\newcommand{\dx}{\mathrm{d}x}

% Probability commands: Expectation, Variance, Covariance, Bias
\newcommand{\Var}{\mathrm{Var}}
\newcommand{\Cov}{\mathrm{Cov}}
\newcommand{\Bias}{\mathrm{Bias}}
\newcommand*{\Z}{\mathbb{Z}}
\newcommand*{\Q}{\mathbb{Q}}
\newcommand*{\R}{\mathbb{R}}
\newcommand*{\C}{\mathbb{C}}
\newcommand*{\N}{\mathbb{N}}
\newcommand*{\prob}{\mathds{P}}
\newcommand*{\E}{\mathds{E}}
\newcommand*{\G}{\mathcal{G}}
\newcommand*{\D}{\mathcal{D}}

\begin{document}


\maketitle

\thispagestyle{empty}
\clearpage
\pagenumbering{arabic} 

\pagebreak

\rhead{ECE 271A}

% \begin{center}
%     \section*{\\ Bayesian Decision Theory}
% \end{center}

% \vspace{0.2in}


\begin{topic}{Bayes Decision Rule}
    \begin{align*}
        g^*(x) 
        &= \underset{g(x)}{\arg \min} \, \sum_i P_{Y|X}(i|x)L[g(x), i] \\
        &= \underset{i}{\arg \max} \, P_{Y|X}(i|x) && \text{(for 0-1 loss function)} \\
        &= \underset{i}{\arg \max} \, P_{X|Y}(x|i)P_Y(i) && \text{(for 0-1 loss function)} \\
        &= \underset{i}{\arg \max} \, \log P_{X|Y}(x|i) + \log P_Y(i). && \text{(for 0-1 loss function)}
    \end{align*}
    
    For binary classification, the likelihood ratio form is: pick $0$ if $\frac{P_{X|Y}(x|0)}{P_{X|Y}(x|1)} > T^* = \frac{P_Y(1)}{P_X(0)}$.
\end{topic}

\begin{topic}{Associated Risk}
    \[
        R^* = \int P_X(x) \sum_{i \neq g^*(x)} P_{Y|X}(i|x) dx = \int P_{Y,X}(y \neq g^*(x), x) dx \quad \text{(For 0-1 loss function)}
    \]
\end{topic}

\begin{topic}{Gaussian Classifier}
    For single variable, we assume $\sigma_i = \sigma$ and pick 0 if
    \[
        x < \frac{\mu_1 + \mu_0}{2} + \frac{1}{\frac{\mu_1 - \mu_0}{\sigma^2}}\log \frac{P_Y(0)}{P_Y(1)}.
    \]
    Generalizing it to multiple variables, we assume $\Sigma_i = \Sigma$, then the BDR becomes
    \[
        i^*(x) = \underset{i}{\arg \min}[d(x, \mu_i) + \alpha_i],
    \]
    where $d(x, y) = (x - y)^T\Sigma^{-1}(x - y)$ and $\alpha_i = \cancel{\log\left[(2\pi)^d|\Sigma|\right]} - 2\log P_Y(i)$.

    Alternatively,
    \[
        i^*(x) = \underset{i}{\arg \max} \, g_i(x),
    \]
    where $g_i(x) = w_i^Tx + w_{i0}$, $w_i = \Sigma^{-1}\mu_i$, and $w_{i0} = -\frac{1}{2}\mu_i^T\Sigma^{-1}\mu_i + \log P_Y(i)$.
\end{topic}

\begin{topic}{Geometric Interpretation}
    Thus, the hyperplane between class $0$ and $1$ is 
    \[
        g_0(x) - g_1(x) = w^Tx + b = 0,
    \]
    where $w = \Sigma^{-1}(\mu_0 - \mu_1)$ and $b = -\frac{(\mu_0 + \mu_1)^T\Sigma^{-1}(\mu_0 - \mu_1)}{2} + \log \frac{P_Y(0)}{P_Y(1)}$.

    It could also be rewritten as
    \[
        w^T(x - x_0) = 0,
    \]
    where $w = \Sigma^{-1}(\mu_0 - \mu_1)$ and $x_0 = \frac{\mu_0 + \mu_1}{2} - \frac{1}{(\mu_0 - \mu_1)^T\Sigma^{-1}(\mu_0 - \mu_1)} \log \frac{P_Y(0)}{P_Y(1)}(\mu_0 - \mu_1)$
\end{topic}

\begin{topic}{Gaussian Distribution Transformation}
    Let $x \sim N(\mu, \Sigma)$, and let $y = A^Tx$, for some matrix $A$. Then, $y \sim N(A^T\mu, A^T\Sigma A)$. A special case of this is the whitening transform $A_w = \Phi\Lambda^{-1/2}$, where $\Phi$ is the matrix of orthonormal eigenvectors of $\Sigma$, and $\Lambda$ is the diagonal matrix of eigenvalues of $\Sigma$.
\end{topic}

\begin{topic}{Sigmoid}
    Suppose that $g_1(x) = 1 - g_0(x)$. Then, we can rewrite
    \[
        g_0(x) = \frac{1}{1 + \frac{P_{X|Y}(x|1)P_Y(1)}{P_{X|Y}(x|0)P_Y(0)}} = \frac{1}{1 + \exp\{d_0(x, \mu_0) - d_1(x, \mu_1) + \alpha_0 - \alpha_1\}},
    \]
    where $d(x, y) = (x - y)^T\Sigma^{-1}(x - y)$ and $\alpha_i = \log \left[(2\pi)^d|\Sigma_i|\right] - 2\log P_Y(i)$.
\end{topic}

\begin{topic}{Maximum Likelihood Estimation}
    Solve for
    \[
        \theta^* = \underset{\Theta}{\arg \max} \, P_{X;\Theta}(\D;\theta) = \underset{\Theta}{\arg \max} \, \log P_{X|\Theta}(\D;\theta).
    \]
    Consider the Gaussian example: 
    
    Given a sample $\D = \{x_1, \dots , x_n\}$ of independent points, where $P_X(x_i) = \frac{1}{\sqrt{(2\pi)^d|\Sigma|}}e^{-\frac{1}{2}(x_i - \mu)^T\Sigma^{-1}(x_i - \mu)}$.

    Then, the likelihood $L(x_1, \dots, x_n|\mu, \sigma) = \prod_{i = 1}^n P_X(x_i)$. We take the gradient of the natural log of $L$ with respect to $\mu$ and get
    \begin{align*}
        \nabla_{\mu} (\log L) 
        &= \nabla_{\mu} \left(-\frac{1}{2}\log[(2\pi)^d|\Sigma|] - \frac{1}{2} \sum_{i = 1}^n (x_i - \mu)^T\Sigma^{-1}(x_i - \mu)\right) \\
        &= \sum_{i = 1}^n \Sigma^{-1}(x_i - \mu) = \sum_{i = 1}^n x_i - \sum_{i = 1}^n \mu  = 0.
    \end{align*}
    Thus, we get $\hat{\mu} = \frac{1}{n}\sum_{i = 1}^n x_i$. By taking the Hessian, we get $\nabla_{\mu}^2(\log L) = -\sum_{i = 1}^n \Sigma^{-1} = -n\Sigma^{-1}$. Since the covariance matrix $\Sigma$ is positive definite, $-n\Sigma^{-1}$ is negative definite. Thus $\hat{\mu}$ is the maximum point.

    In addition, the MLE of the covariance matrix is
    \[
        \hat{\Sigma} = \frac{1}{n}\sum_{i = 1}^n (x_i - \mu)(x_i - \mu)^T.
    \]
\end{topic}

\begin{topic}{Bias and Variance}
    \begin{gather*}
        Bias(\hat{\theta}) = E[\hat{\theta} - \theta], \quad Var(\hat{\theta}) = E\left\{(\hat{\theta} - E[\hat{\theta}])^2\right\}, \\
        MSE(\hat{\theta}) = E\left[(\hat{\theta} - \theta)^2\right] = Var(\hat{\theta}) + Bias^2(\hat{\theta}).
    \end{gather*}
\end{topic}

\begin{topic}{Least Squares}
    Consider an overdetermined system $\Phi\theta = z$, where we attempt to minimize $\lVert z - \Phi\theta \rVert$, the least square solution is
    \[
        \theta^* = (\Phi^T\Phi)^{-1}\Phi^Tz.
    \]
    For a overdetermined system $W\Phi\theta = Wz$, where we attempt to minimize $(z - \Phi\theta)^TW^TW(z - \Phi\theta)$, the least square solution is
    \[
        \theta^* = (\Phi^TW^TW\Phi)^{-1}\Phi^TW^TWz.
    \]
\end{topic}

\begin{topic}{Bayesian Estimation}
    Pick $i$ if
    \[
        i^{*}(x) = \underset{i}{\arg \max} \, P_{X|Y, T}(x|i, \D_i)P_Y(i),
    \]
    where the class conditional is the predictive distribution
    \[
        P_{X|Y, T}(x|i, \D_i) = \int P_{X|Y, \Theta}(x|i, \theta)P_{\Theta|Y, T}(\theta|i, \D_i) \, d\theta = E_{\Theta|Y, T}[P_{X|i, \Theta}(x|\theta) \, | \, T = \D_i].
    \]
    For the multivariate Gaussian case, suppose
    \[
        P_{T|\mu}(\D|\mu) = \G(\D, \mu, \Sigma), \quad P_{\mu}(\mu) = \G(\mu, \mu_0, \Sigma_0),
    \]
    for known $\Sigma, \mu_0, \Sigma_0$. The posterior distribution is $P_{\mu|T}(\mu|\D) = \G(\mu, \mu_n, \Sigma_n)$, where 
    \begin{align*}
        \Sigma_n &= \Sigma_0A^{-1}\frac{1}{n}\Sigma \Rightarrow \Sigma_n^{-1} = n\Sigma^{-1} + \Sigma_0^{-1}, \\
        \mu_n &= \Sigma_0A^{-1}\mu_{ML} + \frac{1}{n}\Sigma A^{-1}\mu_0, \\
        A &= \Sigma_0 + \frac{1}{n}\Sigma.
    \end{align*}
    Then, the predictive distribution is
    \begin{align*}
        P_{X|T}(x|\D)
        &= \int P_{X|\mu}(x|\mu)P_{\mu|T}(\mu|\D) \, d\mu \\
        &= \int \G(x, \mu, \Sigma)\G(\mu, \mu_n, \Sigma_n) \, d\mu \\
        &= \int \G(x - \mu, 0, \Sigma)\G(\mu, \mu_n, \Sigma_n) \, d\mu \\
        &= \G(x, 0, \Sigma) * \G(x, \mu_n, \Sigma_n) = \G(x, \mu_n, \Sigma + \Sigma_n).
    \end{align*}
    Note that for non-informative prior, $\displaystyle{\lim_{|\Sigma_0| \to \infty}} \mu_n = \mu_{ML}$ and $\displaystyle{\lim_{|\Sigma_0| \to \infty}} \Sigma_n = \frac{1}{n}\Sigma = \Sigma_{ML}$, so
    \[
        P_{X|T}(x|\D) = \G(x, \mu_n, \Sigma + \Sigma_n) = \G\left(x, \mu_{ML}, \left(1 + \frac{1}{n}\right)\Sigma\right).
    \]
\end{topic}

\begin{topic}{MAP Estimation}
    \[
        \theta_{MAP} = \underset{\theta}{\arg \max} \, P_{\Theta|T}(\theta|\D) = \underset{\theta}{\arg \max} \, P_{T|\Theta}(\D|\theta)P_{\Theta}(\theta),
    \]
    and this makes the predictive distribution equal to
    \[
        P_{X|T}(x|\D) = P_{X|\Theta}(x|\theta_{MAP}) = \G(x, \mu_{ML}, \Sigma)
    \]
    Note that for the MAP estimator approaches the ML estimator as the sample size increases, i.e. $\theta_{MAP} \rightarrow \theta_{ML}$ as $n \rightarrow \infty$.
\end{topic}

\pagebreak

\begin{topic}{Expectation-maximization}
    \begin{enumerate}
        \item write down the likelihood of the complete data (can drop terms irrelevant to $Z$ and $\Psi$)
                \[
                    P_{X, Z}(\D, z; \Psi) = \left(\prod_{i = 1}^n P_{X| Z}(x_i| z; \Psi)\right)P_Z(z; \Psi).
                \]
        \item \textbf{E-step}: write down the $Q$ function
                \[
                    Q(\Psi; \Psi^{(n)}) = E_{Z|X;\Psi^{(n)}}[\log P_{X, Z}(\D, z; \Psi) \, | \, \D].
                \]
        \item \textbf{M-step}: update $\Psi$, i.e.
                \[
                    \Psi^{(n + 1)} = \underset{\Psi}{\arg \max} \, Q(\Psi; \Psi^{(n)}).
                \]
    \end{enumerate}
\end{topic}

\begin{topic}{EM for Mixtures}
    Represent the class variable as $z = e_j = (\underbrace{0, \ldots, 1_j, \ldots, 0}_{C \text{ entries}})^T$. The complete data log likelihood is 
    \[
        \log P_{X, Z}(\D, \{z_1, \dots, z_n\}; \Psi) = \log \prod_{i = 1}^n \prod_{j = 1}^C [P_{X|Z}(x|e_j, \Psi)\pi_j]^{z_{ij}} = \sum_{i, j} z_{ij} \log [P_{X|Z}(x|e_j, \Psi)\pi_j].
    \]
    Thus, in E-step,
    \[
        Q(\Psi; \Psi^{(n)}) = \sum_{i, j} h_{ij}\log [P_{X|Z}(x|e_j, \Psi)\pi_j]
    \]
    where $h_{ij} = E_{Z|X;\Psi^{(n)}}[z_{ij}|\D]$. Hence, we only have to compute
    \[
        h_{ij} = E_{Z|X;\Psi^{(n)}}[z_{ij}|\D] = P_{Z|X}(z_{ij} = 1|x_i; \Psi^{(n)}) = P_{Z|X}(e_j|x_i; \Psi^{(n)})
    \]
    In M-step, we compute
    \[
        \Psi^{(n + 1)} = \underset{\Psi}{\arg \max} \, \sum_{i, j} h_{ij}\log [P_{X|Z}(x|e_j, \Psi)\pi_j].
    \]
    For Gaussiam mixure, we may solve for $h_{ij}$ first then take the Lagrangian $L = Q(\Psi; \Psi^{(n)}) + \lambda\left(\sum_{j = 1}^C \pi_j - 1\right)$ to solve for the parameters. Here are the results:
    \begin{align*}
        &h_{ij} = \frac{\G\left(x_i, \mu_j^{(n)}, \sigma_j^{(n)}\right)\pi_j^{(n)}}{\sum_k^C \G\left(x_i, \mu_k^{(n)}, \sigma_k^{(n)}\right)\pi_k^{(n)}}
        &\pi_j^{(n + 1)} = \frac{1}{n}\sum_{i = 1}^n h_{ij} \\
        &\mu_j^{(n + 1)} = \frac{\sum_{i}^n h_{ij}x_i}{\sum_{i}^n h_{ij}}
        &\sigma^{2(n + 1)}_j = \frac{\sum_{i}^n h_{ij}(x_i - \mu_j)^2}{\sum_{i}^n h_{ij}}
    \end{align*}
\end{topic}

\begin{topic}{MAP-EM}
    \begin{enumerate}
        \item \textbf{E-step}: compute
        \[
            E_{Z|X, \Psi}[\log P_{\Psi|X, Z} (\Psi| \D, z)\, | \, \D, \Psi^{(n)}] \Rightarrow Q(\Psi|\Psi^{(n)}) + \log P_{\Psi}(\Psi) \quad (\text{only need to compute }Q)
        \]
        \item \textbf{M-step}: compute
        \[
            \Psi^{(n + 1)} = \underset{\Psi}{\arg \max} \, \{Q(\Psi|\Psi^{(n)}) + \log P_{\Psi}(\Psi)\}.
        \]
    \end{enumerate}
\end{topic}
\end{document}