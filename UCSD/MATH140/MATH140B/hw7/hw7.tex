\documentclass{article}

\usepackage{fancyhdr}
\usepackage{extramarks}
\usepackage{amsmath}
\usepackage{amsthm}
\usepackage{amsfonts}
\usepackage{tikz}
\usepackage[plain]{algorithm}
\usepackage{algpseudocode}
\usepackage{enumerate}
\usepackage{amssymb}
\usepackage{mathrsfs}
\usepackage{mathtools}
\usepackage{amsmath}

\usetikzlibrary{automata,positioning}

%
% Basic Document Settings
%

\topmargin=-0.45in
\evensidemargin=0in
\oddsidemargin=0in
\textwidth=6.5in
\textheight=9.0in
\headsep=0.25in

\linespread{1.1}

\pagestyle{fancy}
\lhead{\hmwkAuthorName}
\chead{\hmwkClass:\ \hmwkTitle}
\rhead{\firstxmark}
\lfoot{\lastxmark}
\cfoot{\thepage}

\renewcommand\headrulewidth{0.4pt}
\renewcommand\footrulewidth{0.4pt}

\setlength\parindent{0pt}
\setlength{\parskip}{5pt}

%
% Create Problem Sections
%

\newcommand{\enterProblemHeader}[1]{
    \nobreak\extramarks{}{Problem \arabic{#1} continued on next page\ldots}\nobreak{}
    \nobreak\extramarks{Problem \arabic{#1} (continued)}{Problem \arabic{#1} continued on next page\ldots}\nobreak{}
}

\newcommand{\exitProblemHeader}[1]{
    \nobreak\extramarks{Problem \arabic{#1} (continued)}{Problem \arabic{#1} continued on next page\ldots}\nobreak{}
    \stepcounter{#1}
    \nobreak\extramarks{Problem \arabic{#1}}{}\nobreak{}
}

\setcounter{secnumdepth}{0}
\newcounter{partCounter}
\newcounter{homeworkProblemCounter}
\setcounter{homeworkProblemCounter}{1}
\nobreak\extramarks{Problem \arabic{homeworkProblemCounter}}{}\nobreak{}

%
% Homework Problem Environment
%
% This environment takes an optional argument. When given, it will adjust the
% problem counter. This is useful for when the problems given for your
% assignment aren't sequential. See the last 3 problems of this template for an
% example.
%
\newenvironment{homeworkProblem}[1][-1]{
    \ifnum#1>0
        \setcounter{homeworkProblemCounter}{#1}
    \fi
    \section{Problem \arabic{homeworkProblemCounter}}
    \setcounter{partCounter}{1}
    \enterProblemHeader{homeworkProblemCounter}
}{
    \exitProblemHeader{homeworkProblemCounter}
}

%
% Homework Details
%   - Title
%   - Due date
%   - Class
%   - Section/Time
%   - Instructor
%   - Author
%

\newcommand{\hmwkTitle}{Homework\ \#7}
\newcommand{\hmwkDueDate}{May 29, 2024}
\newcommand{\hmwkClass}{MATH 140B}
\newcommand{\hmwkClassInstructor}{Professor Seward}
\newcommand{\hmwkAuthorName}{\textbf{Ray Tsai}}
\newcommand{\hmwkPID}{A16848188}

%
% Title Page
%

\title{
    \vspace{2in}
    \textmd{\textbf{\hmwkClass:\ \hmwkTitle}}\\
    \normalsize\vspace{0.1in}\small{Due\ on\ \hmwkDueDate\ at 23:59pm}\\
    \vspace{0.1in}\large{\textit{\hmwkClassInstructor}} \\
    \vspace{3in}
}

\author{
  \hmwkAuthorName \\
  \vspace{0.1in}\small\hmwkPID
}
\date{}

\renewcommand{\part}[1]{\textbf{\large Part \Alph{partCounter}}\stepcounter{partCounter}\\}

%
% Various Helper Commands
%

% define norm \norm{...}:
\DeclarePairedDelimiterX\norm[1]\lVert\rVert{{#1}}

% Useful for algorithms
\newcommand{\alg}[1]{\textsc{\bfseries \footnotesize #1}}

% For derivatives
\newcommand{\deriv}[1]{\frac{\mathrm{d}}{\mathrm{d}x} (#1)}

% For partial derivatives
\newcommand{\pderiv}[2]{\frac{\partial}{\partial #1} (#2)}

% Integral dx
\newcommand{\dx}{\mathrm{d}x}

% Probability commands: Expectation, Variance, Covariance, Bias
\newcommand{\Var}{\mathrm{Var}}
\newcommand{\Cov}{\mathrm{Cov}}
\newcommand{\Bias}{\mathrm{Bias}}
\newcommand*{\Z}{\mathbb{Z}}
\newcommand*{\Q}{\mathbb{Q}}
\newcommand*{\R}{\mathbb{R}}
\newcommand*{\C}{\mathbb{C}}
\newcommand*{\N}{\mathbb{N}}
\newcommand*{\prob}{\mathds{P}}
\newcommand*{\E}{\mathds{E}}

\begin{document}

\maketitle

\pagebreak

\begin{homeworkProblem}
  Let $K$ be the unit circle in the complex plane (i.e., the set of all $z$ with $|z| = 1$), and let $\mathscr{A}$ be the algebra of all functions of the form
  \[
    f(e^{i\theta}) = \sum_{n=0}^N c_n e^{in\theta} \quad (\theta \text{ real}).
  \]
  Then $\mathscr{A}$ separates points on $K$ and $\mathscr{A}$ vanishes at no point of $K$, but nevertheless there are continuous functions on $K$ which are not in the uniform closure of $\mathscr{A}$.

  \textbf{Hint:} For every $f \in \mathscr{A}$
  \[
    \int_0^{2\pi} f(e^{i\theta}) e^{i\theta} \, d\theta = 0,
  \]
  and this is also true for every $f$ in the closure of $\mathscr{A}$.

  \begin{proof}
    Since $\mathscr{A}$ contains the identity function, $\mathscr{A}$ seperate points and vanishes at no points of $K$. We now show that there exists functions not in the uniform closure of $\mathscr{A}$. Suppose
    \[
      f(e^{i\theta}) = \sum_{n=0}^N c_n e^{in\theta}.
    \]
    Then, 
    \begin{align*}
      \int_0^{2\pi} f(e^{i\theta}) e^{i\theta} \, d\theta
       &= \sum_{n=0}^N c_n \int_0^{2\pi} e^{in\theta} e^{i\theta} \, d\theta \\
       &= \sum_{n=0}^N c_n \int_0^{2\pi} e^{(n + 1)i\theta} \, d\theta \\
       &= \sum_{n=0}^N c_n \int_0^{2\pi} \cos((n + 1)\theta)  \, d\theta + \sum_{n=0}^N ic_n \int_0^{2\pi} \sin((n + 1)\theta)  \, d\theta \\
       &= \sum_{n=0}^N \frac{c_n}{n + 1} \int_0^{2\pi(n + 1)} \cos(u)  \, du + \sum_{n=0}^N \frac{ic_n}{n + 1} \int_0^{2\pi(n + 1)} \sin(u)  \, du = 0.
    \end{align*}
    Now suppose $g$ is a limit point of $\mathscr{A}$. There exists a sequence $\{f_m\}$ of functions from $\mathscr{A}$ which converges to $g$ uniformly. Then,
    \begin{align*}
      \int_0^{2\pi} g(e^{i\theta}) e^{i\theta} \, d\theta
       &= \lim_{m \to \infty} \int_0^{2\pi} f_m(e^{i\theta}) e^{i\theta} \, d\theta = 0.
    \end{align*}
    Now consider $g(e^{i\theta}) = e^{-i\theta}$. We have
    \[
      \int_0^{2\pi} g(e^{i\theta}) e^{i\theta} \, d\theta = \int_0^{2\pi} e^{-i\theta} e^{i\theta} \, d\theta = \int_0^{2\pi} d\theta = 2\pi.
    \]
    But then $g$ is not in the uniform closure of $\mathscr{A}$.
  \end{proof}
\end{homeworkProblem}

\newpage

\begin{homeworkProblem}
  Define
  \[
    f(x) = \begin{cases} 
      e^{-1/x^2} & \text{if } x \neq 0, \\ 
      0 & \text{if } x = 0.
    \end{cases}
  \]
  Prove that $f$ has derivatives of all orders at $x = 0$ and that $f^{(n)}(0) = 0$ for $n = 1, 2, 3, \dots$.

  \begin{proof}
    We proceed by induction on $n$ to show that
    \[
      f^{(n)}(x) = \begin{cases} 
        p_n(1/x)e^{-1/x^2} & \text{if } x \neq 0, \\ 
        0 & \text{if } x = 0,
      \end{cases}
    \]
    where $p_n(1/x)$ is some polynomial function on $\frac{1}{x}$. Suppose $n = 1$. By Theorem 8.6(f),
    \[
      f'(0) = \lim_{h \to 0} h^{-1}e^{-1/h^2} = 0.
    \]
    On the other hand, 
    \[
      f'(x) = 2x^{-3}e^{-1/x^2}
    \]
    if $x \neq 0$, and the base case is done. Now suppose $n \geq 2$. If $x = 0$,
    \begin{align*}
      f^{(n)}(0)
      &= \lim_{h \to 0} h^{-1}(f^{(n - 1)}(h) - f^{(n - 1)}(0)) \\
      &= \lim_{h \to 0} (h^{-1}p_{n - 1}(1/h))e^{-1/h^2} = 0,
    \end{align*}
    by induction and Theorem 8.6(f). If $n \neq 0$, by induction,
    \begin{align*}
      f^{(n)}(x)
      &= (p_{n - 1}(1/x)e^{-1/x^2})' \\
      &= (p_{n - 1}(1/x))'e^{-1/x^2} + p_{n - 1}(1/x)(e^{-1/x^2})' \\
      &= \frac{1}{x}p'_{n - 1}(1/x)e^{-1/x^2} + 2x^{-3}p_{n - 1}(1/x)e^{-1/x^2} \\
      &= p_{n}(1/x)e^{-1/x^2},
    \end{align*}
    for some polynomial $p_{n}(1/x)$.

  \end{proof}
\end{homeworkProblem}

\newpage

\begin{homeworkProblem}
  Prove the following limit relations:
  \begin{enumerate}[(a)]
    \item $\lim_{x \to 0} \frac{b^x - 1}{x} = \log b \quad (b > 0).$
    \begin{proof}
      By L'Hopital's rule,
      \[
        \lim_{x \to 0} \frac{b^x - 1}{x} = \lim_{x \to 0} \frac{e^{x\log b} - 1}{x} = \lim_{x \to 0} e^{x\log b}\log b = \log b.
      \]
    \end{proof}
    \item $\lim_{x \to 0} \frac{\log(1 + x)}{x} = 1.$
    \begin{proof}
      By L'Hopital's rule,
      \[
        \lim_{x \to 0} \frac{\log(1 + x)}{x} = \lim_{x \to 0} \frac{1}{1 + x} = 1.
      \]
    \end{proof}
    \item $\lim_{x \to 0} (1 + x)^{1/x} = e.$
    \begin{proof}
      By (b),
      \[
        \lim_{x \to 0} (1 + x)^{1/x} = \lim_{x \to 0} e^{\frac{\log (1 + x)}{x}} = e.
      \]
    \end{proof}
    \item $\lim_{n \to \infty} \left(1 + \frac{x}{n}\right)^n = e^x.$
    \begin{proof}
      Fix $x$. Put $y = \left(1 + \frac{x}{n}\right)^n$. Then,
      \[
        \lim_{n \to \infty} \log y = \lim_{n \to \infty} n\log\left(1 + \frac{x}{n}\right).
      \]
      By L'Hopital's rule,
      \begin{align*}
        \lim_{n \to \infty} n\log\left(1 + \frac{x}{n}\right) 
        &= \lim_{n \to \infty} \frac{\log\left(1 + \frac{x}{n}\right)}{\frac{1}{n}} \\
        &= \lim_{a \to 0} \frac{\log\left(1 + ax\right)}{a} \\
        &= \lim_{a \to 0} \frac{x}{1 + ax} = x.
      \end{align*}
      It now follows that
      \[
        \lim_{n \to \infty} \left(1 + \frac{x}{n}\right)^n = e^{\log y} = e^x.
      \]
    \end{proof}
  \end{enumerate}
\end{homeworkProblem}

\newpage

\begin{homeworkProblem}
  \begin{enumerate}[(a)]
    \item $\lim_{x \to 0} \frac{e -(1 + x)^{1/x}}{x}.$
    \begin{proof}
      Exercise 4(c) shows that $\lim_{x \to 0} (1 + x)^{1/x} = e$, and so we may apply L'Hopital's rule, and get
      \[
        \lim_{x \to 0} \frac{e -(1 + x)^{1/x}}{x} = \lim_{x \to 0} -\left(\frac{1}{x}\log(1 + x)\right)'(1 + x)^{1/x} = e\lim_{x \to 0} \frac{\log(1 + x) - \frac{x}{1 + x}}{x^2}.
      \]
      Again by L'Hopital's rule
      \[
        e\lim_{x \to 0} \frac{\log(1 + x) - \frac{x}{1 + x}}{x^2} = e\lim_{x \to 0} \frac{1}{2(1 + x)^2} = e/2.
      \]
    \end{proof}
    \item $\lim_{n \to \infty} \frac{n}{\log n} \left(n^{1/n} - 1\right).$
    \begin{proof}
      We first note that $\frac{\log n}{n} \to 0$ as $n \to \infty$. Hence,
      \[
        \lim_{n \to \infty} \frac{n}{\log n} \left(n^{1/n} - 1\right) = \lim_{n \to \infty} \frac{e^{\frac{\log n}{n}} - 1}{\frac{\log n}{n}} = \lim_{a \to 0} \frac{e^{a} - 1}{a}.
      \]
      But then this is just the derivative of $e^x$ at $x = 0$, which is $1$.
    \end{proof}
    \item $\lim_{x \to 0} \frac{\tan x - x}{x(1 - \cos x)}.$
    \begin{proof}
      We apply L'Hopital's rule three times and get,
      \begin{align*}
        \lim_{x \to 0} \frac{\tan x - x}{x(1 - \cos x)}
        &= \lim_{x \to 0} \frac{\sin x - x\cos x}{x\cos x(1 - \cos x)} \\
        &= \lim_{x \to 0} \frac{x\sin x}{(\cos x - x\sin x)(1 - \cos x) + x\cos x\sin x} \\
        &= \lim_{x \to 0} \frac{\sin x+x\cos x}{-2x\sin^2 x +\left(4\cos x - 2\right)\sin x +2x\cos^2 x -x\cos x} \\
        &= \lim_{x \to 0} \frac{2\cos x -x\sin x}{-6\sin^2 x +\left(x-8x\cos x\right)\sin x +6\cos^2 x -3\cos x} \\
        &= \frac{2}{3}.
      \end{align*}
    \end{proof}
    \item $\lim_{x \to 0} \frac{x - \sin x}{\tan x - x}.$
    \begin{proof}
      We apply L'Hopital's rule three times and get,
      \begin{align*}
        \lim_{x \to 0} \frac{x - \sin x}{\tan x - x}
        &= \lim_{x \to 0} \frac{x\cos x - \cos x \sin x}{\sin x - x\cos x} \\
        &= \lim_{x \to 0} \frac{\sin^2 x -x\sin x -\cos^2 x +\cos x}{x \sin x} \\
        &= \lim_{x \to 0} \frac{(4 \cos x - 2)\sin x - x \cos x}{\sin x + x \cos x} \\
        &= \lim_{x \to 0} \frac{-4\sin^2 x +x\sin x +4\cos^2 x -3\cos x}{2\cos x -x\sin x} = \frac{1}{2}.
      \end{align*}
    \end{proof}
  \end{enumerate}
\end{homeworkProblem}

\newpage

\begin{homeworkProblem}
  Suppose $f(x)f(y) = f(x + y)$ for all real $x$ and $y$.
  \begin{enumerate}[(a)]
      \item Assuming that $f$ is differentiable and not zero, prove that
      \[
        f(x) = e^{cx},
      \]
      where $c$ is a constant.
      \begin{proof}
        It is obvious that for $x \neq 0$, $f(x) = f(x)f(0)$, so $f(0) = 1$. We also note that
        \[
          f'(x) = \lim_{h \to 0} \frac{f(x + h) - f(x)}{h} = f(x)\lim_{h \to 0} \frac{f(h) - 1}{h} = f'(0)f(x).
        \]
        Let $c = f'(0)$ and consider $g(x) = e^{cx}f(-x)$. Since $g(0) = 1$ and
        \[
          g'(x) = ce^{cx}f(-x) - e^{cx}f'(0)f(-x) = e^{cx}f(-x)(c - f'(0)) = 0,
        \]
        we have $g(x) = 1$ for all $x$. But then
        \[
          e^{cx}f(-x) = f(0) = f(x)f(-x).
        \]
        Since $f$ is non zero,
        \[
          e^{-cx} = f(x).
        \]
      \end{proof}
      \item Prove the same thing, assuming only that $f$ is continuous.
      \begin{proof}
        Given $r = p/q \in \Q$, we have 
        \[
          f(r) = f\left(p \cdot \frac{1}{q}\right) = (f(1/q))^n = (f(1)^{1/q})^p = (f(1))^r.
        \]
        Hence, 
        \[
          \log f(r) = r \log f(1).
        \]
        Put $c = \log f(1)$. Then,
        \[
          f(r) = e^{\log f(r)} = e^{cr},
        \]
        so $f(r) = e^{cr}$ for $r \in \Q$. Now for $x \in \R$, we have
        \[
          e^{cx} = \sup e^{cr} = \sup f(r) = f(x) \quad (r < x, r \in \Q)
        \]
        as $\Q$ is dense in $\R$ and $f$ is continuous.
      \end{proof}
  \end{enumerate}
\end{homeworkProblem}
\end{document}