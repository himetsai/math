\documentclass{article}

\usepackage{fancyhdr}
\usepackage{extramarks}
\usepackage{amsmath}
\usepackage{amsthm}
\usepackage{amsfonts}
\usepackage{tikz}
\usepackage[plain]{algorithm}
\usepackage{algpseudocode}
\usepackage{enumerate}
\usepackage{amssymb}
\usepackage{mathrsfs}
\usepackage{mathtools}
\usepackage{amsmath}

\usetikzlibrary{automata,positioning}

%
% Basic Document Settings
%

\topmargin=-0.45in
\evensidemargin=0in
\oddsidemargin=0in
\textwidth=6.5in
\textheight=9.0in
\headsep=0.25in

\linespread{1.1}

\pagestyle{fancy}
\lhead{\hmwkAuthorName}
\chead{\hmwkClass:\ \hmwkTitle}
\rhead{\firstxmark}
\lfoot{\lastxmark}
\cfoot{\thepage}

\renewcommand\headrulewidth{0.4pt}
\renewcommand\footrulewidth{0.4pt}

\setlength\parindent{0pt}
\setlength{\parskip}{5pt}

%
% Create Problem Sections
%

\newcommand{\enterProblemHeader}[1]{
    \nobreak\extramarks{}{Problem \arabic{#1} continued on next page\ldots}\nobreak{}
    \nobreak\extramarks{Problem \arabic{#1} (continued)}{Problem \arabic{#1} continued on next page\ldots}\nobreak{}
}

\newcommand{\exitProblemHeader}[1]{
    \nobreak\extramarks{Problem \arabic{#1} (continued)}{Problem \arabic{#1} continued on next page\ldots}\nobreak{}
    \stepcounter{#1}
    \nobreak\extramarks{Problem \arabic{#1}}{}\nobreak{}
}

\setcounter{secnumdepth}{0}
\newcounter{partCounter}
\newcounter{homeworkProblemCounter}
\setcounter{homeworkProblemCounter}{1}
\nobreak\extramarks{Problem \arabic{homeworkProblemCounter}}{}\nobreak{}

%
% Homework Problem Environment
%
% This environment takes an optional argument. When given, it will adjust the
% problem counter. This is useful for when the problems given for your
% assignment aren't sequential. See the last 3 problems of this template for an
% example.
%
\newenvironment{homeworkProblem}[1][-1]{
    \ifnum#1>0
        \setcounter{homeworkProblemCounter}{#1}
    \fi
    \section{Problem \arabic{homeworkProblemCounter}}
    \setcounter{partCounter}{1}
    \enterProblemHeader{homeworkProblemCounter}
}{
    \exitProblemHeader{homeworkProblemCounter}
}

%
% Homework Details
%   - Title
%   - Due date
%   - Class
%   - Section/Time
%   - Instructor
%   - Author
%

\newcommand{\hmwkTitle}{Homework\ \#5}
\newcommand{\hmwkDueDate}{May 10, 2024}
\newcommand{\hmwkClass}{MATH 140B}
\newcommand{\hmwkClassInstructor}{Professor Seward}
\newcommand{\hmwkAuthorName}{\textbf{Ray Tsai}}
\newcommand{\hmwkPID}{A16848188}

%
% Title Page
%

\title{
    \vspace{2in}
    \textmd{\textbf{\hmwkClass:\ \hmwkTitle}}\\
    \normalsize\vspace{0.1in}\small{Due\ on\ \hmwkDueDate\ at 23:59pm}\\
    \vspace{0.1in}\large{\textit{\hmwkClassInstructor}} \\
    \vspace{3in}
}

\author{
  \hmwkAuthorName \\
  \vspace{0.1in}\small\hmwkPID
}
\date{}

\renewcommand{\part}[1]{\textbf{\large Part \Alph{partCounter}}\stepcounter{partCounter}\\}

%
% Various Helper Commands
%

% define norm \norm{...}:
\DeclarePairedDelimiterX\norm[1]\lVert\rVert{{#1}}

% Useful for algorithms
\newcommand{\alg}[1]{\textsc{\bfseries \footnotesize #1}}

% For derivatives
\newcommand{\deriv}[1]{\frac{\mathrm{d}}{\mathrm{d}x} (#1)}

% For partial derivatives
\newcommand{\pderiv}[2]{\frac{\partial}{\partial #1} (#2)}

% Integral dx
\newcommand{\dx}{\mathrm{d}x}

% Probability commands: Expectation, Variance, Covariance, Bias
\newcommand{\Var}{\mathrm{Var}}
\newcommand{\Cov}{\mathrm{Cov}}
\newcommand{\Bias}{\mathrm{Bias}}
\newcommand*{\Z}{\mathbb{Z}}
\newcommand*{\Q}{\mathbb{Q}}
\newcommand*{\R}{\mathbb{R}}
\newcommand*{\C}{\mathbb{C}}
\newcommand*{\N}{\mathbb{N}}
\newcommand*{\prob}{\mathds{P}}
\newcommand*{\E}{\mathds{E}}

\begin{document}

\maketitle

\pagebreak

\begin{homeworkProblem}
  If $(f_n)$ and $(g_n)$ converge uniformly on a set $E$, prove that $(f_n + g_n)$ converges uniformly on $E$. If, in addition, $(f_n)$ and $(g_n)$ are sequences of bounded functions, prove that $(f_n g_n)$ converges uniformly on $E$.

  \begin{proof}
    Pick $\epsilon > 0$. Since $(f_n)$ and $(g_n)$ converge uniformly, there exists $N, M$ such that for all $x \in E$, $|f_{n_1}(x) - f_{n_2}(x)| \leq \epsilon/2$ and $|g_{m_1}(x) - g_{m_2}(x)| \leq \epsilon/2$, for all $n_1, n_2 \geq N$ and $m_1, m_2 \geq M$. Put $L = \max(N, M)$. For all $m, n \geq L$,
    \begin{align*}
      |(f_n + g_n)(x) - (f_m + g_m)(x)| 
      &= |(f_n(x) - f_m(x)) + (g_n(x) - g_m(x))| \\
      &\leq |(f_n(x) - f_m(x))| + |(g_n(x) - g_m(x))| \leq \epsilon,
    \end{align*}
    for all $x \in E$. Hence, $(f_n + g_n)$ converges uniformly.

    Now suppose that there exists $B > 0$ such that $\sup_x |f_n(x)| < B$ and $\sup_x |g_n(x)| < B$ for all $n$. Since $(f_n)$ and $(g_n)$ converge uniformly, there exists $N, M$ such that for all $x \in E$, $|f_{n_1}(x) - f_{n_2}(x)| \leq \epsilon/2B$ and $|g_{m_1}(x) - g_{m_2}(x)| \leq \epsilon/2B$, for all $n_1, n_2 \geq N$ and $m_1, m_2 \geq M$. Put $L = \max(N, M)$. For all $m, n \geq L$. Then, for all $m, n \geq L$,
    \begin{align*}
      |(f_ng_n)(x) - (f_mg_m)(x)|
      &= |(f_ng_n)(x) - (f_mg_n)(x) + (f_mg_n)(x) - (f_mg_m)(x)| \\
      &\leq |f_n(x)g_n(x) - f_m(x)g_n(x)| + |f_m(x)g_n(x) - f_m(x)g_m(x)| \\
      &< B|(f_n(x) - f_m(x))| + B|(g_n(x) - g_m(x))| \leq \epsilon,
    \end{align*}
    for all $x \in E$. Hence, $(f_ng_n)$ converges uniformly.
  \end{proof}
\end{homeworkProblem}

\newpage

\begin{homeworkProblem}
  Construct sequences $(f_n)$, $(g_n)$ which converge uniformly on some set $E$, but such that $(f_n g_n)$ does not converge uniformly on $E$ (of course, $(f_n g_n)$ must converge on $E$).

  \begin{proof}
    Consider $f_n(x) = x$ and $g_n(x) = \frac{1}{n}$ on $\R^+$. Since $f_n$ remains the same for all $n$, so it converges uniformly to $f(x) = x$. Given any $\epsilon > 0$, $|g_n| < \epsilon$ for $n > \frac{1}{\epsilon}$, and thus $g_n$ converges uniformly to $0$. But then there always exists $x > n$ such that $(f_n g_n)(x) > 1$. Hence, $\sup_x |(f_n g_n)(x) - 0| > 1$, $(f_n g_n)$ does not converge uniformly.
  \end{proof}
\end{homeworkProblem}

\newpage

\begin{homeworkProblem}
  Consider
  \[
    f(x) = \sum_{n=1}^{\infty} \frac{1}{1 + n^2 x}.
  \]
  For what values of $x$ does the series converge absolutely? On what intervals does it converge uniformly? On what intervals does it fail to converge uniformly? Is $f$ continuous wherever the series converges? Is $f$ bounded?
  
  \begin{proof}
    Notice that when $x = 0$, $f(x) = \sum_{n = 1}^{\infty} 1$ diverges. Additionally, when $x = -\frac{1}{n^2}$ for some $n$, the $n$th term in $f(x)$ is not well-defined, and thus the infinite sum is also not well-defined. For $x \neq 0$ and $x \neq -\frac{1}{n^2}$ for all $n$,
    \[
      \sum_{n=1}^{\infty} \left|\frac{1}{1 + n^2 x}\right| \leq \sum_{n=1}^{\infty} \left|\frac{1}{xn^2}\right| = \frac{1}{|x|}\sum_{n=1}^{\infty} \frac{1}{n^2},
    \]
    which converges as $\sum_{n=1}^{\infty} \frac{1}{n^2}$ converges. Hence, $f(x)$ converges absolutely if and only of $x \neq 0$ and $x \neq -\frac{1}{n^2}$ for any $n$.

    Let $E = \R \backslash (\{0\} \cup \{-\frac{1}{n^2} \, | \, n \in \N\})$. We show that $f(x)$ converges uniformly on $E_{\delta} = E \backslash (-\delta, \delta)$, for any $\delta > 0$. Pick $\epsilon > 0$. Let $M_n = \frac{2}{\delta n^2}$. For $x > 0$, 
    \[
      \left|\frac{1}{1 + n^2x}\right| \leq \frac{1}{n^2} \cdot \frac{1}{\delta} < M_n.
    \]
    For $x < 0$, 
    \[
      \left|\frac{1}{1 + n^2x}\right| < \frac{1}{\frac{1}{2}|x|} \cdot \frac{1}{n^2} \leq \frac{1}{\frac{1}{2}\delta} \cdot \frac{1}{n^2} = M_n,
    \]
    when $n \geq N$ for some $N$. Hence, each term of $f(x)$ is bounded by $M_n$ when $n \geq N$. Define $f'(x) = \sum_{n = N}^{\infty} \frac{1}{1 + n^2x}$. Since each term of $f'(x)$ is bounded by $M_n$ and $\sum M_n$ converges, $f'(x)$ converges uniformly on $E_{\delta}$, and thus $f(x) = \sum_{n = 1}^{N - 1} \frac{1}{1 + n^2x} + f'(x)$ also converges uniformly on $E_{\delta}$.



    $f(x)$ trivially fails to converge uniformly on any interval which contains either $0$ or $-\frac{1}{n^2}$ for some $n$. We now show that $f(x)$ fails to converge uniformly on $(0, \delta]$ for any $\delta > 0$. Let $f_m(x) = \sum_{n = 1}^m \frac{1}{1 + n^2 x}$. Suppose for the sake of contradiction that there eixsts $M$ such that for all $m \geq M$, 
    \[
      \left|f(x) - f_m(x)\right| < \frac{1}{4}.
    \]
    Pick $x_0 \in (0, \delta]$ small enough such that $\lceil \frac{1}{\sqrt{x_0}} \rceil > M$. Let $N = \lceil \frac{1}{\sqrt{x_0}} \rceil$. Then, $N^2 \geq \frac{1}{x} \geq (N - 1)^2$. But then
    \begin{align*}
      |f(x) - f_{N - 1}(x)|
      &= \left|\sum_{n = N}^{\infty} \frac{1}{1 + n^2x}\right| \\
      &\geq \left|\sum_{n = N}^{\infty} \frac{1}{2n^2x}\right| \\
      &= \frac{1}{2x}\sum_{n = N}^{\infty} \frac{1}{n^2} \\
      &\geq \frac{(N - 1)^2}{2N^2} \geq \frac{1}{4},
    \end{align*}
    contradiction.

    Since $\frac{1}{1 + n^2x}$ is continuous on $E$ for all $n \in \N$, the partial sums of $f(x)$ is continuous on $E$. Given any point $x \in E$, pick $\delta \in (0, |x|)$. By Theorem 7.12, since $f$ uniformly converges on $E_{\delta}$, $f$ is continuous on $E_{\delta}$, and thus $f$ is continuous on $x$. Hence, $f$ is continuous whenever the series converges.

    $f$ is not bounded. Given any $M > 0$, pick $x = \frac{1}{4M^2}$. Then,
    \begin{align*}
      |f(x)|
      &= \left|\sum_{n = 1}^{\infty} \frac{1}{1 + n^2x}\right| \\
      &= \left|\sum_{n = 1}^{2M} \frac{1}{1 + n^2x} + \sum_{n = 2M + 1}^{\infty} \frac{1}{1 + n^2x}\right| \\
      &> 2M\left(\frac{1}{1 + (2M)^2x}\right) = M.
    \end{align*}
  \end{proof}
\end{homeworkProblem}

\newpage

\begin{homeworkProblem}
  For $n = 1, 2, 3, \dots,$ and $x$ real, put
  \[
    f_n(x) = \frac{x}{1 + nx^2}.
  \]
  Show that $(f_n)$ converges uniformly to a function $f$, and that the equation
  \[
    f'(x) = \lim_{n \to \infty} f'_n(x)
  \]
  is correct if $x \neq 0$, but false if $x = 0$.

  \begin{proof}

    We show that $(f_n)$ converges to $f(x) = 0$. Pick $\epsilon > 0$. Put $N > \frac{1}{4\epsilon^2}$. Note that
    \[
      \left|\frac{x}{1 + nx^2}\right| = \left|\frac{1}{\frac{1}{x} + nx}\right|.
    \]
    By AM-GM, $\frac{1}{x} + nx \geq 2\sqrt{n}$. It follows that for $n \geq N$,
    \[
      \left|\frac{x}{1 + nx^2}\right| \leq \frac{1}{2\sqrt{n}} < \epsilon,
    \]
    and thus $(f_n)$ converges to $0$ uniformly.

    Note that $f_n'(x) = \frac{1 - nx^2}{(1 + nx^2)^2}$. In particular, $f_n'(0) = 1$. When $x \neq 0$, $\lim_{n \to \infty} f_n'(x) = 0 = f'(x)$. But then if $x = 0$, $\lim_{n \to \infty} f_n'(0) = 1 \neq f'(x)$.
  \end{proof}
\end{homeworkProblem}

\newpage

\begin{homeworkProblem}
  Let $(f_n)$ be a sequence of continuous functions which converges uniformly to a function $f$ on a set $E$. Prove that
  \[
    \lim_{n \to \infty} f_n(x_n) = f(x)
  \]
  for every sequence of points $x_n \in E$ such that $x_n \to x$, and $x \in E$. Is the converse of this true?

  \begin{proof}
    By Theorem 7.12, since $f_n$ is continuous for all $n$, $f$ is continuous, which implies $\lim_{n \to \infty} f(x_n) = f(x)$. Hence, it suffices to show that
    \[
      \lim_{n \to \infty} f_n(x_n) = \lim_{n \to \infty} f(x_n).
    \]
    Pick $\epsilon > 0$. Since $(f_n)$ uniformly converges to $f$, there exists $N$ such that
    \[
      |f_n(x) - f(x)| < \epsilon,
    \] 
    for all $n \geq N$. But then
    \[
      |f_n(x_n) - f(x_n)| < \epsilon,
    \] 
    for all $n \geq N$, and the result now follows.

    However, the converse to this is not true. Consider $f_n(x) = x^n$ on $[0, 1)$ and $f(x) = 0$. Let $(x_n)$ be a sequence in $[0, 1)$ which converges to some $x \in E$. Since $|x_n| < 1$, 
    \[
      \lim_{n \to \infty} f_n(x_n) = \lim_{n \to \infty} x_n^n = 0 = f(x).
    \]
    But then $(f_n)$ does not converge uniformly, as for any $\epsilon \in (0, 1)$, there exists $x > \sqrt[n]{\epsilon}$ in $[0, 1)$ such that $x^n > \epsilon$.
  \end{proof}
\end{homeworkProblem}

\newpage

\begin{homeworkProblem}
  Letting $(x)$ denote the fractional part of the real number $x$ (see Exercise 4.16 for the definition), consider the function
  \[
    f(x) = \sum_{n=1}^{\infty} \frac{(nx)}{n^2}
  \]
  for $x$ real. Find all discontinuities of $f$, and show that they form a countable dense set. Show that $f$ is nevertheless Riemann-integrable on every bounded interval.
  
  \begin{proof}
    We show that $f(x)$ is discontinuous for all $x \in \Q$, which is obviously a countable dense set. We first note that the partial sums $f(x)$ converges uniformly as $\sum_{n=1}^{\infty} \frac{1}{n^2}$ converges, by Theorem 7.10.
    
    Notice that $(nx)$ is discontinuous if and only if $nx \in \Z$ if and only if $x = p/q$, where $n$ is a multiple of $q$. Hence, for any irrational $x$, since the partial sums of $f(x)$ is continuous, $f(x)$ is continuous on $x$, by Theorem 7.12.
    
    Now suppose $x \in \Q$, say $x = p/q$. Define $f_q'(x) = \sum_{k = 1}^{\infty} \frac{(kqx)}{[kq]^2}$ and consider $f_q(x) = f(x) - f_q'(x)$. Note that
    \[
      f_q'(x-) = \sum_{k = 1}^{\infty} \frac{1}{[kq]^2} \neq 0 = f_q'(x),
    \]
    and thus $f_q'(x)$ is discontinuous on $x$. Since $f'_q$ contains all terms which are discontinuous on $x$, all terms of $f_q(x)$ are continuous on $x$, and thus the partial sum of $f_q(x)$ is continuous on $x$. Again we know that the partial sums of $f_q(x)$ converge uniformly, by Theorem 7.10. By Theorem 7.12, $f_q(x)$ is continuous on $x$. But then $f'_q(x) = f_q(x) - f(x)$ is discontinuous on $x$, so $f(x)$ is discontinuous on $x$. Hence, $f(x)$ is discontinuous on $x$ if and only if $x \in \Q$.

    Since $(nx)/n^2$ is piece-wise continuous, $(nx)/n^2 \in \mathscr{R}$, and thus $\sum_{n = 1}^m (nx)/n^2 \in \mathscr{R}$. It now follows that the partial sums of $f(x)$ converges uniformly on any given bounded interval, so $f$ is Riemann-integrable on every bounded interval, by Theorem 7.16.
  \end{proof}
\end{homeworkProblem}

\newpage

\begin{homeworkProblem}
  Let $f$ be a continuous real function on $\mathbb{R}^1$ with the following properties: $0 \leq f(t) \leq 1$, and
  \[
    f(t) = \begin{cases} 
      0 & \text{for } 0 \leq t \leq \frac{2}{3}, \\
      1 & \text{for } \frac{2}{3} \leq t \leq 1.
    \end{cases}
  \]
  $f(t+2) = f(t)$ for every $t$, and

  Put $\Phi(t) = (x(t), y(t))$, where
  \[
    x(t) = \sum_{n=1}^{\infty} 2^{-n} f(3^{2n-1} t), \quad y(t) = \sum_{n=1}^{\infty} 2^{-n} f(3^{2n} t).
  \]
  Prove that $\Phi$ is continuous and that $\Phi$ maps $I = [0, 1]$ onto the unit square $I^2 \subseteq \mathbb{R}^2$. In fact, show that $\Phi$ maps the Cantor set onto $I^2$.

  \begin{proof}
    We first note that both $x(t)$ and $y(t)$ converges uniformly as $\sum_{n = 1}^{\infty} 2^{-n} = 1$ converges, by Theorem 7.10. Since $f$ is continuous, the partial sums of both $x(t)$ and $y(t)$ are continuous, and thus $x(t)$ and $y(t)$ are continuous, by Theorem 7.16. It now follows from Theorem 4.10 that $\Phi$ is continuous.

    We now show that $\Phi$ maps $I = [0, 1]$ onto $I^2$. Notice thay each $(x_0, y_0) \in I^2$ has the form
    \[
      x_0 = \sum_{n=1}^{\infty} 2^{-n} a_{2n-1}, \quad y_0 = \sum_{n=1}^{\infty} 2^{-n} a_{2n},
    \]
    where each $a_i$ is $0$ or $1$. Let $t_0 = \sum_{i=1}^{\infty} 3^{-i-1} (2a_i)$. By Exercise 3.19, $t_0 = \sum_{i=1}^{\infty} 3^{-i - 1} (2a_i)$ is in the Cantor set. Since 
    \[
      3^kt_0 = \sum_{i=1}^{\infty} 3^{-i + k - 1} (2a_i) = 2\sum_{i=1}^{k - 1} 3^{-i + k - 1} a_i + \sum_{i=0}^{\infty} 3^{-i - 1} (2a_{i + k}),
    \]
    we know $f(3^kt_0) = f\left(\sum_{i=0}^{\infty} 3^{-i - 1} (2a_{i + k})\right)$. But then
    \[
      \sum_{i=0}^{\infty} 3^{-i - 1} (2a_{i + k}) = \frac{2}{3}a_k + \frac{2}{3}\sum_{i = 1}^{\infty} 3^{-i} a_{i + k},
    \]
    and 
    \[
      0 \leq \frac{2}{3}\sum_{i = 1}^{\infty} 3^{-i} a_{i + k} \leq \frac{2}{3} \cdot \frac{\frac{1}{3}}{1 - \frac{1}{3}} = \frac{1}{3},
    \]
    so $\sum_{i=0}^{\infty} 3^{-i - 1} (2a_{i + k}) \in [0, \frac{2}{3}]$ if $a_k = 0$ and $\sum_{i=0}^{\infty} 3^{-i - 1} (2a_{i + k}) \in [\frac{2}{3}, 1]$ otherwise. Hence, $f(3^kt_0) = a_k$. It now follows that $x(t_0) = x_0$ and $y(t_0) = y_0$, and so $\Phi$ maps the Cantor set $C \subset I$ onto $I^2$.
  \end{proof}
\end{homeworkProblem}
\end{document}