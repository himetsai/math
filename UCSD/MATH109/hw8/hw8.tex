\documentclass{article}
\usepackage{amsfonts, amsmath, amssymb, amsthm} % Math notations imported
\usepackage{enumitem}

\newtheorem{thm}{Theorem}
\newtheorem{prop}[thm]{Proposition}
\newtheorem{cor}[thm]{Corollary}

% title information
\title{Math 109 HW 8}
\author{Ray Tsai}
\date{11/21/2022}

% main content
\begin{document} 

% placing title information; comment out if using fancyhdr
\maketitle 

\begin{enumerate}
% Q1
\item \begin{enumerate}
\item \begin{prop}
    $\sim$ is equivalent.
\end{prop}
\begin{proof}
    We will show that $\sim$ is reflexive, symmetric, and transitive.

    Reflexive: Let $(a, b) \in \mathbb{R}^2$. We will show that $(a, b) \sim (a, b)$. Since there exists $k = 1 \in \mathbb{R}$ such that $(a, b) = (ka, kb)$, $(a, b) \sim (a, b)$.

    Symmetric: Let $(a, b) \sim (c, d)$. We will show that $(c, d) \sim (a, b)$. Since $(a, b) \sim (c, d)$, we know that there exists $k \in \mathbb{R}$ such that $(a, b) = (kc, kd)$. Since $k \neq 0$, we can let $m = \frac{1}{k} \in \mathbb{R}$. We then get $(ma, mb) = (kmc, kmd) = (c, d)$, which shows that $(c, d) ~ (a, b)$.

    Transitive: Let $(a, b) \sim (c, d)$, $(c, d) \sim (e, f)$. We will show that $(a, b) \sim (e, f)$. Since $(a, b) \sim (c, d)$ and $(c, d) \sim (e, f)$, we have $(a, b) = (kc, kd)$ and $(c, d) = (me, mf)$, $k, m \in \mathbb{R}_{\neq 0}$. We then have $(a, b) = (kc, kd) = (kme, kmf)$. Since $km \in \mathbb{R}_{\neq 0}$, we have $(a, b) \sim (e, f)$.

    Therefore, $\sim$ is reflexive, symmetric, and transitive.
\end{proof}

\item \begin{prop}
    $\sim$ is not equivalent.
\end{prop}
\begin{proof}
    Consider the case $(1, 0), (0, 0) \in \mathbb{R}^2$. Since $1^2 + 0^2 = 1 \geq 0^2 + 0^2$, we have $(1, 0) \sim (0, 0)$. However, $(0, 0) \not\sim (1, 0)$ because $0^2 + 0^2 = 1 < 1^2 + 0^2$. Therefore, $\sim$ is not equivalent.
\end{proof}
\end{enumerate}

% Q2
\item 
\begin{prop}
    $\approx$ is an equivalent relation.
\end{prop}
\begin{proof}
    We will show that $\approx$ is reflexive, symmetric, and transitive.

    Reflexive: Let $a \in A$. We will show that $a \approx a$. Since $\sim$ is an equivalent relation and $f(a) = f(a)$, we know that $f(a) \sim f(a)$ by the reflexive property. Therefore, since $f(a) \sim f(a)$, we have $a \approx a$.

    Symmetric: Let $a_1 \approx a_2$. We will show that $a_2 \approx a_1$. Since $a_1 \approx a_2$, we know that $f(a_1) \sim f(a_2)$. By the symmetric property of $\sim$, we have $f(a_2) \sim f(a_1)$, which shows that $a_2 \approx a_1$.

    Transitive: Let $a_1 \approx a_2$, $a_2 \approx a_3$. We will show that $a_1 \approx a_3$. Since $a_1 \approx a_2$, $a_2 \approx a_3$, we know that $f(a_1) \sim f(a_2)$ and $f(a_2) \sim f(a_3)$. By the transitive property of $\sim$, we have $f(a_1) \sim f(a_3)$, which shows that $a_1 \approx a_3$.

    Therefore, $\approx$ is an equivalent relation.
\end{proof}

% Q3
\item 
\begin{enumerate}
    \item 
    \begin{prop}
        $S/\sim$ has $3$ elements.
    \end{prop}
    \begin{proof}
        We know that for all $m \in S$, $1 \leq m \leq 15$, $m \in \mathbb{Z}$. Since the $2$ is the smallest integer, $1$ has $0$ prime factors. Since $2 \in S$, we know that $S/\sim$ contains a equivalent class for elements that have $1$ prime factor. The second smallest integer is $3$. Since $2 \cdot 3  = 6 \in S$, we know that $S/\sim$ contains a equivalent class for elements that have $2$ prime factor. The third smallest integer is $5$. The smallest integer that has $3$ or more prime factors is $2\cdot3\cdot5 = 30 \notin S$, as it is greater than $15$. Since all the integers that have $3$ or more prime factors are greater than $15$, there does not exist an equivalent class for them. Therefore, $S/\sim$ has $3$ elements, namely integers that have $0, 1, 2$ prime factors respectively.
    \end{proof}

    \item 
    $6$ has $2$ prime factors, so the equivalent class containing $6$ is $\{6, 10, 12, 14, 15\} \subseteq S$.
\end{enumerate}

% Q4
\item 
Let $a,b,c,d,n \in \mathbb{Z}$ such that $a \equiv b \pmod{n}$ and $c \equiv d \pmod{n}$.
\begin{enumerate}
\item 
\begin{prop}
     $a + c \equiv b + d \pmod{n}$.
\end{prop}
\begin{proof}
    Let $a = nk_1 + b$, $c = nk_2 + d$, $k_1, k_2 \in \mathbb{Z}$. We will show that $a + c \equiv b + d \pmod{n}$. We know that 
    \begin{align}
        a + c &\equiv (nk_1 + b) + (nk_2 + d) \pmod{n} \\
              &\equiv n(k_1 + k_2) + b + d \pmod{n} \\
              &\equiv b + d \pmod{n}.
    \end{align}
    Therefore, $a + c \equiv b + d \pmod{n}$.
\end{proof}

\item 
\begin{prop}
     $ac \equiv bd \pmod{n}$.
\end{prop}
\begin{proof}
    Let $a = nk_1 + b$, $c = nk_2 + d$, $k_1, k_2 \in \mathbb{Z}$. We will show that $ac \equiv bd \pmod{n}$. We know that 
    \begin{align}
        ac &\equiv (nk_1 + b)(nk_2 + d) \pmod{n} \\
              &\equiv n(nk_1k_2 + k_1d + k_2b) + bd \pmod{n} \\
              &\equiv bd \pmod{n}.
    \end{align}
    Therefore, $ac \equiv bd \pmod{n}$.
\end{proof}

\item 
\begin{prop}
     $a^m \equiv b^m \pmod{n}$ for all $m \in \mathbb{Z}_{> 0}$.
\end{prop}
\begin{proof}
    We will proceed by induction on $m$.

    Suppose $m = 1$, we have $a \equiv b \pmod{n}$.

    Suppose that $a^m \equiv b^m \pmod{n}$ for some $m$. We will show that $a^{m+1} \equiv b^{m+1} \pmod{n}$. Since $a \equiv b \pmod{n}$ and the induction hypothesis, we know that $a\cdot a^m \equiv b\cdot b^m \pmod{n}$ by Q4.b. Thus, $a^{m+1} \equiv b^{m+1} \pmod{n}$ if $a^m \equiv b^m \pmod{n}$.

    Therefore, $a^m \equiv b^m \pmod{n}$ for all $m \in \mathbb{Z}_{> 0}$.
\end{proof}
\end{enumerate}

% Q5
\item 
\begin{prop}
   $13^{145} \equiv 13 \pmod{21}$
\end{prop}
\begin{proof}
    \begin{align}
        13^{145} &\equiv 13^{12\cdot12 + 1} \pmod{21} \\
                 &\equiv 13\cdot(13^{12})^{12} \pmod{21}\\
                 &\equiv 13\cdot(1)^{12} \pmod{21} \\
                 &\equiv 13 \pmod{21}
    \end{align}
\end{proof}

% Q6
\item 
\begin{prop}
    $2^{101} \equiv 4 \pmod{7}.$
\end{prop}
\begin{proof}
    Since
    \begin{align}
        2^1 &\equiv 2 \pmod{7} \\
        2^2 &\equiv 4 \pmod{7} \\
        2^3 &\equiv 1 \pmod{7}, 
    \end{align}
    we know that 
    \begin{align}
        2^{101} &\equiv 2^{3\cdot33 + 2} \pmod{7} \\
                &\equiv 2^2\cdot(2^{3})^{33} \pmod{7} \\
                &\equiv 4\cdot(1)^{33} \pmod{7} \\
                &\equiv 4 \pmod{7}.
    \end{align}
\end{proof}

% Q7
\item 
\begin{prop}
    The possible congruence classes are $[1], [-2]$.
\end{prop}
\begin{proof}
    Let $2x + 3 \equiv -1 \pmod{6}$ for some congruence class $x$. We then have $2x \equiv -4 \equiv 2 \pmod{6}$ by Q4.a. By Q4.b, we can cancel the $2$ on all sides, which shows that $x \equiv -2 \pmod{6}$ or $x \equiv 1 \pmod{6}$. 

    Therefore, the possible congruence classes are $[1], [-2]$.
\end{proof}

% Q8
\item 
\begin{prop}
    There does not exist integers $x, y$ such that $x^3 + 7y^2 = 3$.
\end{prop}
\begin{proof}
    We will prove by contradiction. Let $x, y$ be integers. Suppose for the sake of contradiction that $x^3 + 7y^2 = 3$. By taking modulo $7$ of the equation, we have $x^3 + 7y^2 \equiv x^3 \equiv 3 \pmod{7}$. However, there does not exist $x$ such that $x^3 \equiv 3 \pmod{7}$, since 
    \begin{align}
              0^3 &\equiv 0 \pmod{7} \\
        (\pm 1)^3 &\equiv \pm 1 \pmod{7} \\
        (\pm 2)^3 &\equiv \pm 8 \pmod{7} \\
                  &\equiv \pm 1 \pmod{7} \\
        (\pm 3)^3 &\equiv \pm 27 \pmod{7} \\
                  &\equiv \pm 1 \pmod{7},
    \end{align}
    none of which are congruent to $3$, which contradicts our assumption.

    Therefore, there does not exist integers $x$ such that $x^3 + 7y^2 = 3$.
\end{proof}

% Q9
\item 
\begin{prop}
    If $n \equiv 3 \pmod{4}$, then there does not exist integers  $x, y$ such that $x^2 + y^2 = n$.
\end{prop}
\begin{proof}
    We will prove by contradiction. Let $x, y \in \mathbb{Z}$. Suppose for the sake of contradiction that $n \equiv 3 \pmod{4}$. By taking modulo $4$ of $x^2 + y^2 = n$, we have $x^2 + y^2 \equiv 3 \pmod{4}$. Since 
    \begin{align}
              0^2 &\equiv 0 \pmod{4} \\
        (\pm 1)^2 &\equiv 1 \pmod{4} \\
              2^2 &\equiv 0 \pmod{4},
    \end{align}
    $x^2 \pmod{4}$ and $y^2 \pmod{4}$ can only be congruent to $0$ or $1$. 

    If $x^2 \equiv 0 \pmod{4}$, then $y^2 \pmod{4}$ must be congruent to $3$, which is impossible.

    If $x^2 \equiv 1 \pmod{4}$, then $y^2 \pmod{4}$ must be congruent to $2$, which is also impossible.

    This contradicts our assumption and shows that there does not exist $x, y \in \mathbb{Z}$ such that $x^2 + y^2 = n$.
\end{proof}
    
\end{enumerate}
\end{document}