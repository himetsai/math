\documentclass{article}

\usepackage{fancyhdr}
\usepackage{extramarks}
\usepackage{amsmath}
\usepackage{amsthm}
\usepackage{amsfonts}
\usepackage{tikz}
\usepackage[plain]{algorithm}
\usepackage{algpseudocode}
\usepackage{enumerate}
\usepackage{amssymb}
\usepackage{mathrsfs}
\usepackage{mathtools}

\usetikzlibrary{automata,positioning}

%
% Basic Document Settings
%

\topmargin=-0.45in
\evensidemargin=0in
\oddsidemargin=0in
\textwidth=6.5in
\textheight=9.0in
\headsep=0.25in

\linespread{1.1}

\pagestyle{fancy}
\lhead{\hmwkAuthorName}
\chead{\hmwkClass:\ \hmwkTitle}
\rhead{\firstxmark}
\lfoot{\lastxmark}
\cfoot{\thepage}

\renewcommand\headrulewidth{0.4pt}
\renewcommand\footrulewidth{0.4pt}

\setlength\parindent{0pt}
\setlength{\parskip}{5pt}

%
% Create Problem Sections
%

\newcommand{\enterProblemHeader}[1]{
    \nobreak\extramarks{}{Problem \arabic{#1} continued on next page\ldots}\nobreak{}
    \nobreak\extramarks{Problem \arabic{#1} (continued)}{Problem \arabic{#1} continued on next page\ldots}\nobreak{}
}

\newcommand{\exitProblemHeader}[1]{
    \nobreak\extramarks{Problem \arabic{#1} (continued)}{Problem \arabic{#1} continued on next page\ldots}\nobreak{}
    \stepcounter{#1}
    \nobreak\extramarks{Problem \arabic{#1}}{}\nobreak{}
}

\setcounter{secnumdepth}{0}
\newcounter{partCounter}
\newcounter{homeworkProblemCounter}
\setcounter{homeworkProblemCounter}{1}
\nobreak\extramarks{Problem \arabic{homeworkProblemCounter}}{}\nobreak{}

%
% Homework Problem Environment
%
% This environment takes an optional argument. When given, it will adjust the
% problem counter. This is useful for when the problems given for your
% assignment aren't sequential. See the last 3 problems of this template for an
% example.
%
\newenvironment{homeworkProblem}[1][-1]{
    \ifnum#1>0
        \setcounter{homeworkProblemCounter}{#1}
    \fi
    \section{Problem \arabic{homeworkProblemCounter}}
    \setcounter{partCounter}{1}
    \enterProblemHeader{homeworkProblemCounter}
}{
    \exitProblemHeader{homeworkProblemCounter}
}

%
% Homework Details
%   - Title
%   - Due date
%   - Class
%   - Section/Time
%   - Instructor
%   - Author
%

\newcommand{\hmwkTitle}{Homework\ \#7}
\newcommand{\hmwkDueDate}{Nov 15, 2024}
\newcommand{\hmwkClass}{MATH 220A}
\newcommand{\hmwkClassInstructor}{Professor Ebenfelt}
\newcommand{\hmwkAuthorName}{\textbf{Ray Tsai}}
\newcommand{\hmwkPID}{A16848188}

%
% Title Page
%

\title{
    \vspace{2in}
    \textmd{\textbf{\hmwkClass:\ \hmwkTitle}}\\
    \normalsize\vspace{0.1in}\small{Due\ on\ \hmwkDueDate\ at 23:59pm}\\
    \vspace{0.1in}\large{\textit{\hmwkClassInstructor}} \\
    \vspace{3in}
}

\author{
  \hmwkAuthorName \\
  \vspace{0.1in}\small\hmwkPID
}
\date{}

\renewcommand{\part}[1]{\textbf{\large Part \Alph{partCounter}}\stepcounter{partCounter}\\}

%
% Various Helper Commands
%

% Useful for algorithms
\newcommand{\alg}[1]{\textsc{\bfseries \footnotesize #1}}

% For derivatives
\newcommand{\deriv}[1]{\frac{\mathrm{d}}{\mathrm{d}x} (#1)}

% For partial derivatives
\newcommand{\pderiv}[2]{\frac{\partial}{\partial #1} (#2)}

% Integral dx
\newcommand{\dx}{\mathrm{d}x}

% Probability commands: Expectation, Variance, Covariance, Bias
\newcommand{\Var}{\mathrm{Var}}
\newcommand{\Cov}{\mathrm{Cov}}
\newcommand{\Bias}{\mathrm{Bias}}
\newcommand*{\Z}{\mathbb{Z}}
\newcommand*{\Q}{\mathbb{Q}}
\newcommand*{\R}{\mathbb{R}}
\newcommand*{\C}{\mathbb{C}}
\newcommand*{\N}{\mathbb{N}}
\newcommand*{\prob}{\mathds{P}}
\newcommand*{\E}{\mathds{E}}

\begin{document}

\maketitle

\pagebreak

\begin{homeworkProblem}
	Show that if $F \subset X$ is closed and connected, then for every pair of points $a, b$ in $F$ and each $\epsilon > 0$, there are points $z_0, z_1, \ldots, z_n$ in $F$ with $z_0 = a$, $z_n = b$, and $d(z_{k-1}, z_k) < \epsilon$ for $1 \leq k \leq n$. Is the hypothesis that $F$ be closed needed? If $F$ is a set which satisfies this property, then $F$ is not necessarily connected, even if $F$ is closed. Give an example to illustrate this.

	\begin{proof}
		We give a proof without assuming that $F$ is closed. Suppose there exists $a, b \in F$ and $\epsilon > 0$ such that there do not exist $z_0, z_1, \ldots, z_n \in F$ with $z_0 = a$, $z_n = b$, and $d(z_{k-1}, z_k) < \epsilon$ for $1 \leq k \leq n$. Define
		\begin{gather*}
			A \coloneq \{z \in F \mid \exists z_0 = a, z_1, \ldots, z_n = z, d(z_{k-1}, z_k) < \epsilon, \forall 1 \leq k \leq n\}, \\
			B \coloneq \{z \in F \mid \exists z_0 = b, z_1, \ldots, z_n = z, d(z_{k-1}, z_k) < \epsilon, \forall 1 \leq k \leq n\}.
		\end{gather*}
		Then $A \cap B = \emptyset$ and $A, B \neq \emptyset$, as $a \in A$ and $b \in B$. Let $x \in A$. There exists $z_0 = a, z_1, \ldots, z_n = x$ such that $d(z_{k-1}, z_k) < \epsilon$ for all $1 \leq k \leq n$. For any point $y \in B(x, \epsilon)$, putting $z_{n + 1} \in y$ shows that $y \in A$. Same argument applies to $B$, and so $A$ and $B$ are open sets. Hence, we may assume that $C = F \backslash (A \cup B)$ is nonempty, otherwise $F$ is disconnected. Let $x \in C$. If there exists $z_0 = a, z_1, \ldots, z_n = y, d(z_{k-1}, z_k) < \epsilon, \forall 1 \leq k \leq n$ for some $y \in B(x, \epsilon)$, then putting $z_{n + 1} \in x$ shows that $x \in A$, contradiction. Same argument works for $B$. Thus, $B(x, \epsilon) \subset C$, $C$ is open. But then $C$ and $A \cup B$ are open sets that separates $F$, so $F$ is disconnected, contradiction. The result now follows.

		We now give an counter example of a closed set $F$ that satisfies the property but is disconnected. Put $F_1 = \{(x, 1/x) \mid x \in \R^+\}$ and $F_2 = \{(x, -1/x) \mid x \in \R^{-}\}$. Consider $F = F_1 \cup F_2 \in \R^2$. $F$ is closed as its complement $\R^2 \backslash F$ is obviously open. For $a = (x, 1/x) \in F_1$, $B_{|x|}(a) \cap F \subset F_1$ and so $F_1$ is open. But then $F_1, F_2$ are symmetric about the $y$ axis, so the same argument applies for $F_2$. Hence, $F$ is disconnected. We now show that $F$ is closed. Let $a = (x, y) \in \R^2 \backslash F$. 

		It remains to show that for all pair of points $a, b$ in $F$ and each $\epsilon > 0$, there are points $z_0, z_1, \ldots, z_n$ in $F$ with $z_0 = a$, $z_n = b$, and $d(z_{k-1}, z_k) < \epsilon$ for $1 \leq k \leq n$. Let $a, b \in F$ and fix $\epsilon > 0$. Since $F_1$ and $F_2$ are each connected, we may assume that $a \in F_1$ and $b \in F_2$. Let $x \in (0, \epsilon/2)$ and put $a' = (x, 1/x) \in F_1$, $b' = (x, -\frac{1}{x}) \in F_2$. Since $d(a', b') < \epsilon$, there exists $z_0 = a, z_1, \ldots, z_m = a' \in F_1$ and $z_{m + 1} = b', z_{m + 2}, \ldots, z_n = b \in F_2$ that satisfies the property.
		
		
	\end{proof}
\end{homeworkProblem}

\newpage

\begin{homeworkProblem}
	Let $z_n, z$ be points in $\mathbb{C}$ and let $d$ be the metric on $\mathbb{C}_\infty$. Show that $|z_n - z| \to 0$ if and only if $d(z_n, z) \to 0$. Also show that if $|z_n| \to \infty$ then $\{z_n\}$ is Cauchy in $\mathbb{C}_\infty$. (Must $\{z_n\}$ converge in $\mathbb{C}_\infty$?)

	\begin{proof}
		For $z_n, z \in \C$, the distance function on $\mathbb{C}_\infty$ is defined as 
		\[
			d(z_n, z) \coloneq \frac{2|z - z_n|}{\sqrt{(1 + |z|^2)(1 + |z_n|^2)}}.
		\]
		Since $|z_n - z| \to 0$, we have $d(z_n, z) \to 0$ as the numerator goes to $0$ and the demominator is at least 1. Conversely, suppose for sake of contradiction that $|z_n - z|$ does not converge to $0$ as $d(z_n, n) \to 0$. Since the numerator of $d(z_n, z)$ is not $0$, $d(z_n, n) \to 0$ converges to $0$ only if the demominator $\sqrt{(1 + |z|^2)(1 + |z_n|^2)}$ approaches $\infty$. But then $|z_n| \to \infty$, contradiction.

		Fix $\epsilon > 0$. Note that $d(z_n, \infty) = \frac{2}{\sqrt{(1 + |z_n|^2)}} \to 0$ as $|z_n| \to \infty$. Hence, there exists large enough $n_0$ such that for all $n, m > n_0$, $d(z_n, \infty), d(z_m, \infty) < \epsilon/2$. The result now follows that
		\[
			d(z_n, z_m) \leq d(z_n, \infty) + d(z_m, \infty) < \epsilon,
		\] 
		for all $n, m > n_0$.
	\end{proof}
\end{homeworkProblem}

\newpage

\begin{homeworkProblem}
	Put a metric $d$ on $\mathbb{R}$ such that $|x_n - x| \to 0$ if and only if $d(x_n, x) \to 0$, but that $\{x_n\}$ is a Cauchy sequence in $(\mathbb{R}, d)$ when $|x_n| \to \infty$. (Hint: Take inspiration from $\mathbb{C}_\infty$.)

	\begin{proof}
		Define
		\[
			d(x, y) \coloneq \frac{2|x - y|}{\sqrt{(1 + x^2)(1 + y^2)}},
		\]
		for real $x, y$. Since $d$ is merely the real number case of the metric on $\C_{\infty}$, $d$ is a metric on $\R$ and the statement ``$|x_n - x| \to 0$ if and only if $d(x_n, x) \to 0$'' follows from the same argument as the previous problem. Now, suppose $|x_n| \to \infty$. Fix $\epsilon > 0$. Pick $N > 4/\epsilon$. There exists $n_0$ such that for all $n > n_0$, $|x_n| > N$. But then
		\[
			d(x_n, x_m) = \frac{2|x_n - x_m|}{\sqrt{(1 + x_n^2)(1 + x_m^2)}} \leq \frac{2|x_n| + |x_m|}{\sqrt{x_n^2x_m^2}} = 2\left(\frac{1}{|x_n|} + \frac{1}{|x_m|}\right) < \frac{4}{N} < \epsilon,
		\]
		for all $n, m > n_0$. The result now follows.
	\end{proof}
\end{homeworkProblem}

\newpage

\begin{homeworkProblem}
	Prove the converse of proposition 4.4: A set $K \subset X$ is compact if every collection $\mathscr{F}$ of closed subsets of $K$ with the finite intersection property has nonempty intersection.

	\begin{proof}
		We prove the contrapositive. Suppose $K$ is not compact. There exists an open cover $\{U_{\alpha}\}$ of $K$ such that no finite subcover exists. Let $\mathscr{F}$ be the collection of closed subsets $\{K \backslash U_\alpha\}$. Given any finite subcollection $\{K \backslash U_{\alpha_i}\}_{i = 1}^n$, the intersection $\bigcap_{i = 1}^n K \backslash U_i = K \backslash \left(\bigcup_{i = 1}^n U_{\alpha_i}\right) \neq \emptyset$, and so $\mathscr{F}$ has the finite interseciton property. But then $\bigcap_{\alpha} K \backslash U_\alpha = K \backslash \left(\bigcup_{\alpha} U_{\alpha}\right) = \emptyset$. 
	\end{proof}
\end{homeworkProblem}

\newpage

\begin{homeworkProblem}
	Show that the union of a finite number of compact sets is compact.

	\begin{proof}
		Let $K_1, K_2, \ldots, K_n$ be compact sets, and put $\{U_{\alpha}\}$ as an open cover of $K_1 \cup K_2 \cup \ldots \cup K_n$. Since $K_1$ is compact, there exists a finite subcover $\{U_{\alpha_1}, \ldots, U_{\alpha_{n_1}}\}$ of $K_1$. Similarly, there exists a finite subcover $\{U_{\alpha_{n_1 + 1}}, \ldots, U_{\alpha_{n_2}}\}$ of $K_2$, and so on. But then the union of these finite subcovers is a finite subcover of $K_1 \cup K_2 \cup \ldots \cup K_n$, and the result now follows.
	\end{proof}
\end{homeworkProblem}
\end{document}