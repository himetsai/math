\documentclass{article}

\usepackage{fancyhdr}
\usepackage{extramarks}
\usepackage{amsmath}
\usepackage{amsthm}
\usepackage{amsfonts}
\usepackage{tikz}
\usepackage[plain]{algorithm}
\usepackage{algpseudocode}
\usepackage{enumerate}
\usepackage{amssymb}

\usetikzlibrary{automata,positioning}

%
% Basic Document Settings
%

\topmargin=-0.45in
\evensidemargin=0in
\oddsidemargin=0in
\textwidth=6.5in
\textheight=9.0in
\headsep=0.25in

\linespread{1.1}

\pagestyle{fancy}
\lhead{\hmwkAuthorName}
\chead{\hmwkClass:\ \hmwkTitle}
\rhead{\firstxmark}
\lfoot{\lastxmark}
\cfoot{\thepage}

\renewcommand\headrulewidth{0.4pt}
\renewcommand\footrulewidth{0.4pt}

\setlength\parindent{0pt}
\setlength{\parskip}{5pt}

%
% Create Problem Sections
%

\newcommand{\enterProblemHeader}[1]{
    \nobreak\extramarks{}{Problem \arabic{#1} continued on next page\ldots}\nobreak{}
    \nobreak\extramarks{Problem \arabic{#1} (continued)}{Problem \arabic{#1} continued on next page\ldots}\nobreak{}
}

\newcommand{\exitProblemHeader}[1]{
    \nobreak\extramarks{Problem \arabic{#1} (continued)}{Problem \arabic{#1} continued on next page\ldots}\nobreak{}
    \stepcounter{#1}
    \nobreak\extramarks{Problem \arabic{#1}}{}\nobreak{}
}

\setcounter{secnumdepth}{0}
\newcounter{partCounter}
\newcounter{homeworkProblemCounter}
\setcounter{homeworkProblemCounter}{1}
\nobreak\extramarks{Problem \arabic{homeworkProblemCounter}}{}\nobreak{}

%
% Homework Problem Environment
%
% This environment takes an optional argument. When given, it will adjust the
% problem counter. This is useful for when the problems given for your
% assignment aren't sequential. See the last 3 problems of this template for an
% example.
%
\newenvironment{homeworkProblem}[1][-1]{
    \ifnum#1>0
        \setcounter{homeworkProblemCounter}{#1}
    \fi
    \section{Problem \arabic{homeworkProblemCounter}}
    \setcounter{partCounter}{1}
    \enterProblemHeader{homeworkProblemCounter}
}{
    \exitProblemHeader{homeworkProblemCounter}
}

%
% Homework Details
%   - Title
%   - Due date
%   - Class
%   - Section/Time
%   - Instructor
%   - Author
%

\newcommand{\hmwkTitle}{Homework\ \#1}
\newcommand{\hmwkDueDate}{Jan 26, 2025}
\newcommand{\hmwkClass}{MATH 220B}
\newcommand{\hmwkClassInstructor}{Professor Xiao}
\newcommand{\hmwkAuthorName}{\textbf{Ray Tsai}}
\newcommand{\hmwkPID}{A16848188}

%
% Title Page
%

\title{
    \vspace{2in}
    \textmd{\textbf{\hmwkClass:\ \hmwkTitle}}\\
    \normalsize\vspace{0.1in}\small{Due\ on\ \hmwkDueDate\ at 23:59pm}\\
    \vspace{0.1in}\large{\textit{\hmwkClassInstructor}} \\
    \vspace{3in}
}

\author{
  \hmwkAuthorName \\
  \vspace{0.1in}\small\hmwkPID
}
\date{}

\renewcommand{\part}[1]{\textbf{\large Part \Alph{partCounter}}\stepcounter{partCounter}\\}

%
% Various Helper Commands
%

% Useful for algorithms
\newcommand{\alg}[1]{\textsc{\bfseries \footnotesize #1}}

% For derivatives
\newcommand{\deriv}[1]{\frac{\mathrm{d}}{\mathrm{d}x} (#1)}

% For partial derivatives
\newcommand{\pderiv}[2]{\frac{\partial}{\partial #1} (#2)}

% Integral dx
\newcommand{\dx}{\mathrm{d}x}

% Probability commands: Expectation, Variance, Covariance, Bias
\newcommand{\Var}{\mathrm{Var}}
\newcommand{\Cov}{\mathrm{Cov}}
\newcommand{\Bias}{\mathrm{Bias}}
\newcommand*{\Z}{\mathbb{Z}}
\newcommand*{\Q}{\mathbb{Q}}
\newcommand*{\R}{\mathbb{R}}
\newcommand*{\C}{\mathbb{C}}
\newcommand*{\N}{\mathbb{N}}
\newcommand*{\prob}{\mathds{P}}
\newcommand*{\E}{\mathds{E}}

\begin{document}

\maketitle

\pagebreak

\begin{homeworkProblem}
	Each of the following functions $f$ has an isolated singularity at $z = 0$. Determine its nature; if it is a removable singularity define $f(0)$ so that $f$ is analytic at $z = 0$; if it is a pole find the singular part; if it is an essential singularity determine $f(\{z : 0 < |z| < \delta\})$ for arbitrarily small values of $\delta$.

	\begin{enumerate}
		\item[(a)] $f(z) = \frac{\sin z}{z}$
		\begin{proof}
			$f$ has a removable singularity at $z = 0$. Since the power series expansion of $\sin z$ is $z - \frac{z^3}{3!} + \frac{z^5}{5!} - \cdots$, 
			\[
				f(z) = 1 - \frac{z^2}{3!} + \frac{z^4}{5!} - \cdots
			\]
			and so defining $f(0) = \lim_{z \to 0} f(z) = 1$ makes $f$ analytic.
		\end{proof}
		\item[(b)] $f(z) = \frac{\cos z}{z}$
		\begin{proof}
			Since the power series expansion of $\cos z$ is $1 - \frac{z^2}{2!} + \frac{z^4}{4!} - \cdots$, 
			\[
				f(z) = \frac{1}{z} - \frac{z}{2!} + \frac{z^3}{4!} - \cdots,
			\]
			and so the singluar part of $f$ is $\frac{1}{z}$.
		\end{proof}
		\item[(j)]$f(z) = z^n \sin\left(\frac{1}{z}\right)$
		\begin{proof}
			Note that the Laurent expansion
			\[
				f(z) = \sum_{k = 0}^{\infty} \frac{(-1)^{k}}{(2k + 1)!}z^{n - k} = \sum_{k = -\infty}^n \frac{(-1)^{n - k}}{[2(n - k) + 1]!}z^{k}
			\]
			has infinitely many terms with negative powers of $z$ and so $f$ has an essential singularity at $z = 0$.

		\end{proof}
	\end{enumerate}
\end{homeworkProblem}

\newpage

\begin{homeworkProblem}
	Let $f(z) = \frac{1}{z(z-1)(z-2)}$; give the Laurent Expansion of $f(z)$ in each of the following annuli:
	\begin{enumerate}
		\item[(b)] $\text{ann}(0; 1, 2)$
		\begin{proof}
			By partial fractions decomposition,
			\[
				f(z) = \frac{1}{z(z-1)(z-2)} = \frac{1}{2z} - \frac{1}{z - 1} + \frac{1}{2(z - 2)}.
			\]
			Since $|z| > 1$,
			\[
				\frac{1}{z - 1} = \frac{1}{z} \cdot \frac{1}{1 - \frac{1}{z}} = \frac{1}{z} \sum_{n = 0}^{\infty} \frac{1}{z^n} = \sum_{n = 1}^{\infty} \frac{1}{z^{n}}.
			\]
			Since $0 < |z| < 2$,
			\[
				\frac{1}{z - 2} = -\frac{1}{2} \cdot \frac{1}{1 - \frac{z}{2}} = - \sum_{n = 0}^{\infty} \frac{z^n}{2^{n + 1}}.
			\]
			Hence, the Laurent expansion of $f$ in $\text{ann}(0; 1, 2)$ is
			\[
				f(z) = \frac{1/2}{z} - \sum_{n = 1}^{\infty} \frac{1}{z^{n}} - \sum_{n = 0}^{\infty} \frac{1}{2^{n + 2}}z^n.
			\]
		\end{proof}
		\item[(c)] $\text{ann}(0; 2, \infty)$
		\begin{proof}
			By partial fractions decomposition,
			\[
				f(z) = \frac{1}{z(z-1)(z-2)} = \frac{1}{2z} - \frac{1}{z - 1} + \frac{1}{2(z - 2)}.
			\]
			Since $|z| > 2$,
			\[
				\frac{1}{z - 1} = \frac{1}{z} \cdot \frac{1}{1 - \frac{1}{z}} = \frac{1}{z} \sum_{n = 0}^{\infty} \frac{1}{z^n} = \sum_{n = 1}^{\infty} \frac{1}{z^{n}}.
			\]
			and
			\[
				\frac{1}{z - 2} = \frac{1}{z} \cdot \frac{1}{1 - \frac{2}{z}} = \sum_{n = 1}^{\infty} \frac{2^{n - 1}}{z^{n}}.
			\]
			Hence, the Laurent expansion of $f$ in $\text{ann}(0; 1, 2)$ is
			\[
				f(z) = \frac{1/2}{z} - \sum_{n = 1}^{\infty} \frac{1}{z^{n}} + \sum_{n = 1}^{\infty} \frac{2^{n - 2}}{z^{n}} = \frac{1/2}{z} + \sum_{n = 1}^{\infty} \frac{2^{n - 2} - 1}{z^{n}}.
			\]
		\end{proof}
	\end{enumerate}
\end{homeworkProblem}

\newpage

\begin{homeworkProblem}
	If $f: G \to \mathbb{C}$ is analytic except for poles, show that the poles of $f$ cannot have a limit point in $G$.

	\begin{proof}
		Suppose $a \in G$ is a limit point of poles of $f$. Then $a$ cannot be a pole, as there does not exist $R > 0$ such that $B_R(a) \backslash \{a\}$ is analytic. By the Open Mapping Theorem, there exists $r > 0$ such that $B_r(f(a)) \subseteq f(G \backslash \{\text{poles}\})$. Since $f$ is continuous, $f^{-1}(B_r(f(a)))$ is open in $G \backslash \{\text{poles}\}$ and contains $a$. But then $a$ is a limit point of poles, so there does not exist an open neighborhood of $a$ that is contained in $G \backslash \{\text{poles}\}$, contradiction.
	\end{proof}
\end{homeworkProblem}

\newpage

\begin{homeworkProblem}
	Suppose that $f$ has an essential singularity at $z = a$. Prove the following strengthened version of the Casorati-Weierstrass Theorem. If $c \in \mathbb{C}$ and $\epsilon > 0$ are given, then for each $\delta > 0$ there is a number $\alpha$, $\lvert c - \alpha \rvert < \epsilon$, such that $f(z) = \alpha$ has infinitely many solutions in $B(a; \delta)$.

	\begin{proof}
		We may assume that $a = 0$, otherwise we work with $f(z - a)$ instead. For $n \in \N$, let $S_n = f[\text{ann}(0, 0, \delta/(n + 1))]$, the image of $f(z)$ for all $0 < |z| < \delta/(n + 1)$. Let $\alpha_1 = c$ and $r_1 = \epsilon$. For $n \in \N$, since $S_n$ is dense by the Casorati-Weierstrass Theorem, there exists $\alpha_{n + 1} \in S_n \cap B_{r_n}(\alpha_n)$. By the Open Mapping Theorem, $S_n$ is open, so there exists $r_{n + 1} \in (0, \epsilon/(n + 1e))$ such that $\overline{B}_{r_{n + 1}}(\alpha_{n + 1}) \subseteq S_n \cap B_{r_n}(\alpha_n)$. By iterating this process, we obtain sequences $\{\alpha_n\}_{n \in \N}$ and $\{r_n\}_{n \in \N}$. Since $r_n \to 0$, $\alpha_n$ is Cauchy and thus converges to some $\alpha \in B_\epsilon(c) \cap \bigcap_{n = 1}^\infty S_n$, and the result follows.
	\end{proof}
\end{homeworkProblem}

\newpage

\begin{homeworkProblem}
	Let $R > 0$ and $G = \{ z : |z| > R \}$; a function $f: G \to \C$ has a \textit{removable singularity}, a \textit{pole}, or an \textit{essential singularity at infinity} if $f(z^{-1})$ has, respectively, a removable singularity, a pole, or an essential singularity at $z = 0$. If $f$ has a pole at $\infty$, then the order of the pole is the order of the pole of $f(z^{-1})$ at $z = 0$.
	\begin{enumerate}[(a)]
		\item Prove that an entire function has a removable singularity at infinity iff it is a constant.
		\begin{proof}
			If an entire function $f$ has a removable singularity at $\infty$, then $f(z^{-1})$ has a removable singularity at $z = 0$. But then $f(z^{-1})$ is bounded around $0$, which implies $f$ is bounded in any neighborhood of $\infty$. By Liouville's Theorem, $f$ is constant. 

			If $f(z) = c$, then $f(z^{-1}) = c$ and so $f$ has a removable singularity at $\infty$.
		\end{proof}
		\item Prove that an entire function has a pole at infinity of order $m$ iff it is a polynomial of degree $m$.
		\begin{proof}
			If an entire function $f$ has a pole at $\infty$ of order $m$, then $f(z^{-1})z^m$ has a removable singularity at $z = 0$. But then $f(z^{-1})z^m$ is bounded in any neighborhood of $0$, so $f(z^{-1})z^m$, which implies $f(z)z^{-m}$ is bounded in any neighborhood of $\infty$. By Liouville's Theorem, $f(z)z^{-m}$ is constant, and thus $f(z)$ is a polynomial of degree $m$.

			If $f(z) = a_mz^m + \cdots + a_0$ with $a_m \neq 0$, then $z^mf(z^{-1}) = \lim_{z \to 0} z^mf(z^{-1}) = a_0z^m + \cdots + a_m$, which has a removable singularity at $z = 0$. Thus, $f$ has a pole at $\infty$ of order $m$.
		\end{proof}
	\end{enumerate}
\end{homeworkProblem}

\newpage

\begin{homeworkProblem}
	Calculate the following integrals:
	\begin{enumerate}[(a)]
		\item $\int_{0}^{\infty} \frac{x^2 \, dx}{x^4 + x^2 + 1}$
		\begin{proof}
			Put $f(z) = \frac{z^2}{z^4 + z^2 + 1}$. Since $f$ is even, $\int_{0}^{\infty} f(x) \, dx = \frac{1}{2}\int_{-\infty}^{\infty} f(x) \, dx$. Note that $f$ has poles at $a_1 = e^{i\pi/3}$ and $a_2 = e^{2i\pi/3}$. For $R > 1$, let $\gamma_R = Re^{it}$, $0 \leq t \leq \pi$. Since $a_1, a_2$ are enclosed by $\gamma_R \cup [-R, R]$, by the Residue Theorem,
			\[
				\int_{-R}^{R} f(x) \, dx = 2\pi i [\text{Res}(f, a_1) + \text{Res}(f, a_2)] - \int_{\gamma_R} f(z) \, dz.
			\]
			Calculating the residues, we have
			\begin{align*}
				\text{Res}(f, a_1) &= \frac{e^{2i\pi/3}}{4e^{i\pi} + 2e^{i\pi/3}} = \frac{-\frac{1}{2} + \frac{i\sqrt{3}}{2}}{-4 + 1 + i\sqrt{3}} = \frac{1}{4} - \frac{\sqrt{3}}{12}i, \\
				\text{Res}(f, a_2) &= \frac{e^{4i\pi/3}}{4e^{2i\pi} + 2e^{2i\pi/3}} = \frac{-\frac{1}{2} - \frac{i\sqrt{3}}{2}}{4 - 1 + i\sqrt{3}} = -\frac{1}{4} - \frac{\sqrt{3}}{12}i.
			\end{align*}
			As $R \to \infty$,
			\[
				\left|\int_{\gamma_R} f(z) \, dz\right| \leq \int_{\gamma_R} \frac{|z|^2|}{|z^4 + z^2 + 1|} \, |dz| \leq \int_{\gamma_R} \frac{|z|^2}{|z|^4 - |z|^2 - 1} \, |dz| = \frac{\pi R^3}{R^4 - R^2 - 1} \to 0.
			\]
			Hence, combining the above results, 
			\[
				\int_{0}^{\infty} \frac{x^2 \, dx}{x^4 + x^2 + 1} = \frac{1}{2}\lim_{R \to \infty} \int_{-R}^{R} f(x) \, dx = \pi i\left(\frac{1}{4} - \frac{\sqrt{3}}{12}i - \frac{1}{4} - \frac{\sqrt{3}}{12}i\right) = \frac{\pi}{2\sqrt{3}}.
			\]
		\end{proof}
		\item $\int_{0}^{\infty} \frac{\cos x - 1}{x^2} \, dx$
		\begin{proof}
			Put $f(z) = \frac{e^{iz} - 1}{z^2}$. Let $R > r > 0$ and define $\gamma_{r}(t) = re^{-it}$, $0 \leq t \leq \pi$, and $\gamma_{R}(t) = Re^{it}$, $0 \leq t \leq \pi$. Let $\gamma = [-R, -r] \cup \gamma_r \cup [r, R] \cup \gamma_R$. Since $f$ is analytic on $\C \backslash \{0\}$, 
			\[
				\int_\gamma f(z) \, dz = \int_{-R}^{-r} f(z) \, dz + \int_{\gamma_r} f(z) \, dz + \int_{r}^{R} f(z) \, dz + \int_{\gamma_R} f(z) \, dz = 0
			\]
			Note that 
			\[
				\int_{-R}^{-r} \frac{e^{iz} - 1}{z^2} \, dz = \int_{r}^{R} \frac{e^{-iz} - 1}{z^2} \, dz,
			\]
			and so 
			\[
				\int_{-R}^{-r} f(z) \, dz + \int_{r}^{R} f(z) \, dz = \int_{r}^{R} \frac{e^{iz} + e^{-iz} - 2}{z^2} \, dz = 2\int_{r}^{R} \frac{\cos(z) - 1}{z^2} \, dz.
			\]
			Hence,
			\[
				\int_{r}^{R} \frac{\cos(z) - 1}{z^2} \, dz = - \frac{1}{2}\int_{\gamma_r} f(z) \, dz - \frac{1}{2}\int_{\gamma_R} f(z) \, dz.
			\]
			As $R \to \infty$, 
			\[
				\left|\int_{\gamma_R} f(z) \, dz\right| \leq \int_{\gamma_R} \frac{|e^{iz} - 1|}{|z|^2} \, |dz| \leq \frac{1}{R^2}\left(\int_{\gamma_R} |e^{iz}| \, |dz| + \int_{\gamma_R} |dz|\right) = \frac{2}{R^2}\int_{\gamma_R} |dz| = \frac{2\pi}{R} \to 0.
			\]
			On the other hand, since $f(z) = \frac{i}{z} - \frac{1}{2} - \frac{iz}{6} + \cdots$, 
			\begin{align*}
				\lim_{r \to 0} \int_{\gamma_r} f(z) \, dz 
				&= \lim_{r \to 0} i\int_{\gamma_r} \frac{1}{z} \, dz + \left( - \lim_{r \to 0} \int_{\gamma_r} \frac{1}{2} \, dz + \lim_{r \to 0} \int_{\gamma_r} \frac{iz}{6} \, dz - \cdots \right) \\
				&= \lim_{r \to 0} i\int_{\gamma_r} \frac{1}{z} \, dz \\
				&= \lim_{r \to 0} i\int_{0}^{\pi} \frac{-ire^{-it}}{re^{-it}} \, dt = \pi.
			\end{align*}
			Thus,
			\[
				\int_{0}^{\infty} \frac{\cos(z) - 1}{z^2} \, dz = - \frac{1}{2} \lim_{r \to 0}\int_{\gamma_r} f(z) \, dz = -\frac{\pi}{2}.
			\]
		\end{proof}
	\end{enumerate}
\end{homeworkProblem}

\newpage

\begin{homeworkProblem}
	Verify the following equation:
	\[
		\int_{0}^{\pi/2} \frac{d\theta}{a + \sin^2 \theta} = \frac{\pi}{2 \sqrt{a(a+1)}}, \quad \text{if } a > 0;
	\]

	\begin{proof}
		Note that
		\[
			\int_{0}^{\pi/2} \frac{d\theta}{a + \sin^2 \theta} = \int_{0}^{\pi} \frac{1}{2a + 1 - \cos \theta} \, d\theta,
		\]
		and since $\cos \theta = \cos -\theta$,
		\[
			\int_{0}^{\pi} \frac{1}{2a + 1 - \cos \theta} \, d\theta = \frac{1}{2}\int_{0}^{2\pi} \frac{1}{2a + 1 - \cos \theta} \, d\theta.
		\]
		Put $z = e^{i\theta}$ and we have
		\[
			\int_{0}^{2\pi} \frac{1}{2a + 1 - \cos \theta} \, d\theta = \int_{|z| = 1} \frac{1}{2a + 1 - \frac{z + z^{-1}}{2}} \, \frac{dz}{iz} = 2i\int_{|z| = 1} \frac{1}{z^2 - (4a + 2)z + 1} \, dz.
		\]
		Let $f(z) = \frac{1}{z^2 - (4a + 2)z + 1}$. $f(z)$ have simple poles at $z = 2a + 1 \pm 2\sqrt{a(a + 1)}$. Since $|2a + 1 + 2\sqrt{a(a + 1)}| > 1$, by the Residue Theorem, 
		\[
			\int_{|z| = 1} \frac{1}{z^2 - (4a + 2)z + 1} \, dz = 2\pi i \text{Res}(f, 2a + 1 - 2\sqrt{a(a + 1)}).
		\]
		Since
		\[
			\text{Res}(f, 2a + 1 - 2\sqrt{a(a + 1)}) = \left.\frac{1}{2z - (4a + 2)}\right|_{z = 2a + 1 - 2\sqrt{a(a + 1)}} = -\frac{1}{4\sqrt{a(a + 1)}},
		\]
		we have
		\[
			\int_{0}^{\pi} \frac{1}{2a + 1 - \cos \theta} \, d\theta = \frac{1}{2} \cdot 2i \cdot 2\pi i \cdot -\frac{1}{4\sqrt{a(a + 1)}} = \frac{\pi}{2\sqrt{a(a + 1)}},
		\]
		for $a > 0$.
	\end{proof}
\end{homeworkProblem}

\newpage

\begin{homeworkProblem}
	Suppose that $f$ has a simple pole at $z = a$ and let $g$ be analytic in an open set containing $a$. Show that
	\[
		\text{Res}(fg; a) = g(a) \, \text{Res}(f; a).
	\]

	\begin{proof}
		\[
			\text{Res}(fg; a) = \lim_{z \to a} (z - a)f(z)g(z) = g(a)\lim_{z \to a} (z - a)f(z) = g(a) \, \text{Res}(f; a).
		\]
	\end{proof}
\end{homeworkProblem}

\newpage

\begin{homeworkProblem}
	Use the previous result to show that if $G$ is a region and $f$ is analytic in $G$ except for simple poles at $a_1, \ldots, a_n$; and if $g$ is analytic in $G$, then
	\[
		\frac{1}{2\pi i} \int_\gamma fg = \sum_{k=1}^n n(\gamma; a_k) \, g(a_k) \, \text{Res}(f; a_k)
	\]
	for any closed rectifiable curve $\gamma$ not passing through $a_1, \ldots, a_n$ such that $\gamma \approx 0$ in $G$.

	\begin{proof}
		By the Residue Theorem and the last problem,
		\[
			\frac{1}{2\pi i} \int_\gamma fg = \sum_{k=1}^n n(\gamma; a_k) \, \text{Res}(fg; a_k) = \sum_{k=1}^n n(\gamma; a_k) \, g(a_k) \, \text{Res}(f; a_k).
		\]
	\end{proof}
\end{homeworkProblem}
\end{document}