\documentclass{article}

\usepackage{fancyhdr}
\usepackage{extramarks}
\usepackage{amsmath}
\usepackage{amsthm}
\usepackage{amsfonts}
\usepackage{tikz}
\usepackage[plain]{algorithm}
\usepackage{algpseudocode}
\usepackage{enumerate}
\usepackage{amssymb}

\usetikzlibrary{automata,positioning}

%
% Basic Document Settings
%

\topmargin=-0.45in
\evensidemargin=0in
\oddsidemargin=0in
\textwidth=6.5in
\textheight=9.0in
\headsep=0.25in

\linespread{1.1}

\pagestyle{fancy}
\lhead{\hmwkAuthorName}
\chead{\hmwkClass:\ \hmwkTitle}
\rhead{\firstxmark}
\lfoot{\lastxmark}
\cfoot{\thepage}

\renewcommand\headrulewidth{0.4pt}
\renewcommand\footrulewidth{0.4pt}

\setlength\parindent{0pt}
\setlength{\parskip}{5pt}

%
% Create Problem Sections
%

\newcommand{\enterProblemHeader}[1]{
    \nobreak\extramarks{}{Problem \arabic{#1} continued on next page\ldots}\nobreak{}
    \nobreak\extramarks{Problem \arabic{#1} (continued)}{Problem \arabic{#1} continued on next page\ldots}\nobreak{}
}

\newcommand{\exitProblemHeader}[1]{
    \nobreak\extramarks{Problem \arabic{#1} (continued)}{Problem \arabic{#1} continued on next page\ldots}\nobreak{}
    \stepcounter{#1}
    \nobreak\extramarks{Problem \arabic{#1}}{}\nobreak{}
}

\setcounter{secnumdepth}{0}
\newcounter{partCounter}
\newcounter{homeworkProblemCounter}
\setcounter{homeworkProblemCounter}{1}
\nobreak\extramarks{Problem \arabic{homeworkProblemCounter}}{}\nobreak{}

%
% Homework Problem Environment
%
% This environment takes an optional argument. When given, it will adjust the
% problem counter. This is useful for when the problems given for your
% assignment aren't sequential. See the last 3 problems of this template for an
% example.
%
\newenvironment{homeworkProblem}[1][-1]{
    \ifnum#1>0
        \setcounter{homeworkProblemCounter}{#1}
    \fi
    \section{Problem \arabic{homeworkProblemCounter}}
    \setcounter{partCounter}{1}
    \enterProblemHeader{homeworkProblemCounter}
}{
    \exitProblemHeader{homeworkProblemCounter}
}

%
% Homework Details
%   - Title
%   - Due date
%   - Class
%   - Section/Time
%   - Instructor
%   - Author
%

\newcommand{\hmwkTitle}{Homework\ \#5}
\newcommand{\hmwkDueDate}{Mar 14, 2025}
\newcommand{\hmwkClass}{MATH 220B}
\newcommand{\hmwkClassInstructor}{Professor Xiao}
\newcommand{\hmwkAuthorName}{\textbf{Ray Tsai}}
\newcommand{\hmwkPID}{A16848188}

%
% Title Page
%

\title{
    \vspace{2in}
    \textmd{\textbf{\hmwkClass:\ \hmwkTitle}}\\
    \normalsize\vspace{0.1in}\small{Due\ on\ \hmwkDueDate\ at 23:59pm}\\
    \vspace{0.1in}\large{\textit{\hmwkClassInstructor}} \\
    \vspace{3in}
}

\author{
  \hmwkAuthorName \\
  \vspace{0.1in}\small\hmwkPID
}
\date{}

\renewcommand{\part}[1]{\textbf{\large Part \Alph{partCounter}}\stepcounter{partCounter}\\}

%
% Various Helper Commands
%

% Useful for algorithms
\newcommand{\alg}[1]{\textsc{\bfseries \footnotesize #1}}

% For derivatives
\newcommand{\deriv}[1]{\frac{\mathrm{d}}{\mathrm{d}x} (#1)}

% For partial derivatives
\newcommand{\pderiv}[2]{\frac{\partial}{\partial #1} (#2)}

% Integral dx
\newcommand{\dx}{\mathrm{d}x}

% Probability commands: Expectation, Variance, Covariance, Bias
\newcommand{\Var}{\mathrm{Var}}
\newcommand{\Cov}{\mathrm{Cov}}
\newcommand{\Bias}{\mathrm{Bias}}
\newcommand*{\Z}{\mathbb{Z}}
\newcommand*{\Q}{\mathbb{Q}}
\newcommand*{\R}{\mathbb{R}}
\newcommand*{\C}{\mathbb{C}}
\newcommand*{\N}{\mathbb{N}}
\newcommand*{\prob}{\mathds{P}}
\newcommand*{\E}{\mathds{E}}

\begin{document}

\maketitle

\pagebreak

\begin{homeworkProblem}
	Let $f$ and $g$ be analytic functions on a region $G$ and show that there are analytic functions $f_1, g_1$, and $h$ on $G$ such that 
	\[
	f(z) = h(z) f_1(z) \quad \text{and} \quad g(z) = h(z) g_1(z)
	\]
	for all $z$ in $G$; and $f_1$ and $g_1$ have no common zeros.

	\begin{proof}
		Let $Z_f, Z_g$ be the sets of zeros of $f, g$ respectively counted with multiplicity. Theorem 5.15 yields a analytic function $h(z)$ on $G$ such that $h$ admits zeros on $Z_f \cap Z_g$. Let $f_1, g_1$ such that 
		\[
		f(z) = h(z) f_1(z) \quad \text{and} \quad g(z) = h(z) g_1(z).
		\]
		We know $h$, $f_1$ and $g_1$ are analytic on $G$. Also, since $h(z)$ contains all the common zeros of $f(z)$ and $g(z)$, $f_1$ and $g_1$ have no common zeros.
	\end{proof}
\end{homeworkProblem}

\newpage

\begin{homeworkProblem}
	\begin{enumerate}[(a)]
		\item Let $0 < |a| < 1$ and $|z| \leq r < 1$; show that 
		\[
		\left| \frac{a + |a| z}{(1 - \bar{a} z)a} \right| \leq \frac{1 + r}{1 - r}.
		\]
		\begin{proof}
			\[
				\left| \frac{a + |a| z}{(1 - \bar{a} z)a} \right| = \left| \frac{1 + \frac{|a|}{a}z}{1 - \bar{a}z}\right|.
			\]
			By the triangle inequality,
			\[
				\left|1 + \frac{|a|}{a}z\right| \leq 1 + |z| \leq 1 + r,
			\]
			and
			\[
				|1 - \bar{a}z| \geq 1 - |\bar{a}||z| \geq 1 - |a|r \geq 1 - r.
			\]
			The result now follows.
		\end{proof}

		\item Let $\{ a_n \}$ be a sequence of complex numbers with $0 < |a_n| < 1$ and 
		\[
		\sum (1 - |a_n|) < \infty.
		\]
		Show that the infinite product 
		\[
		B(z) = \prod_{n=1}^{\infty} \frac{|a_n|}{a_n} \left( \frac{a_n - z}{1 - \bar{a_n} z} \right)
		\]
		converges in $H(B(0;1))$ and that $|B(z)| \leq 1$. What are the zeros of $B$? ($B(z)$ is called a \textit{Blaschke Product}.)

		\begin{proof}
			Let $K$ be a compact set. $K$ is contained in $\overline{B}_r(0)$ for some $r < 1$. Let $B_n(z) = \frac{|a_n|}{a_n} \left( \frac{a_n - z}{1 - \bar{a_n} z}\right)$. By (a), 
			\[
				|B_n(z) - 1| = (1 - |a_n|)\left|\frac{a_n + |a_n|z}{(1 - \overline{a_n}z)a_n}\right| \leq \frac{1 + r}{1 - r} (1 - |a_n|)
			\]
			for $z \in K$, and thus $\sum |B_n(z) - 1| \leq \frac{1 + r}{1 - r} \sum (1 - |a_n|) < \infty$. But then $B(z) = \prod B_n(z)$ converges uniformly and absolutely on $K$. Also note that $B_n(z)$ is an automorphism on the unit disk with a pole at $\frac{1}{\overline{a_n}} \notin \overline{B}_1(0)$ and a zero at $a_n$. Hence, $B(z) \leq \prod 1 = 1$ and $B(z)$ has zeros at $a_n$.
		\end{proof}

		\item Find a sequence $\{ a_n \}$ in $B(0;1)$ such that 
		\[
		\sum (1 - |a_n|) < \infty
		\]
		and every number $e^{i\theta}$ is a limit point of $\{ a_n \}$.

		\begin{proof}
			Consider $a_n = e^{i\pi n/\sqrt{2}}(1 - 2^{-n})$. Then 
			\[
				\sum (1 - |a_n|) \leq \sum |e^{i\pi n/\sqrt{2}} - a_n| \leq \sum 2^{-n} < \infty.
			\]
			Since $\sqrt{2}$ is irrational, the set $\{ e^{i\pi n/\sqrt{2}} \mid n \in \N \}$ is dense in the unit circle. That is, for each $e^{i\theta}$, there exists a sequence $\{ n_k \}$ such that $e^{i\pi n_k/\sqrt{2}} \to e^{i\theta}$. Hence, $a_{n_k} = e^{i\pi n_k/\sqrt{2}}(1 - 2^{-n_k}) \to e^{i\theta}$.
		\end{proof}
	\end{enumerate}
\end{homeworkProblem}

\newpage

\begin{homeworkProblem}
	Let 
	\[
	f = \frac{1}{(z-1)(z-5)}.
	\]

	\begin{enumerate}[(a)]
    \item Prove that there is a sequence of rational functions $R_n(z)$ whose poles can only occur at $2$ and $6$ such that 
    \[
    \lim_{n \to \infty} \sup_{3 \leq |z| \leq 4} |f(z) - R_n(z)| = 0. \tag{1}
    \]
		\begin{proof}
			Pick $\epsilon > 0$. Let $K = \overline{\text{ann}(0; 3, 4)}$, and let $E = \{2, 6, \infty\}$. Since $K$ is compact and $E$ contains a pole from each component of $\C_{\infty} \backslash K$, Runge's theorem yields a rational function $R_n(z)$ whose poles can only occur in $E$ and 
			\[
				|f(z) - R_n(z)| < \epsilon,
			\]
			for $z \in K$. The result now follows.
		\end{proof}

    \item Does there exist a sequence of rational functions $R_n(z)$ whose poles can only occur at $6$ such that (1) holds? Justify your answer.

    \begin{proof}
			No. Suppose for the sake of contradiction that there exists such a sequence $\{R_n\}$. Since $R_n$ is analytic on $B_2(0)$, $\int_{|z| = 2} R_n(z) \, dz = 0$, and so $\int_{|z| = 2} R_n(z) \, dz \to 0$. But then $\int_{|z| = 2} f(z) \, dz = -\frac{\pi i}{2}$, contradiction.
		\end{proof}
	\end{enumerate}
\end{homeworkProblem}

\newpage

\begin{homeworkProblem}
Let 
\[
G = \{ z \in \mathbb{C} : |z| < 1 \text{ and } |z - \frac{1}{3}| > \frac{2}{3} \};
\]
and let $K$ be the closure of $G$:
\[
K = \{ z \in \mathbb{C} : |z| \leq 1 \text{ and } |z - \frac{1}{3}| \geq \frac{2}{3} \}.
\]
Let $A(K)$ be the space of continuous functions on $K$ that are analytic on $G$ equipped with the uniform norm on $K$. For the purposes of this problem, a Laurent polynomial is a function of the form 
\[
\sum_{n=-N}^{N} a_n z^n.
\]
Determine whether the following are true or false. Justify your answer.

	\begin{enumerate}[(a)]
    \item The set of polynomials is dense in $H(G)$.
    \begin{proof}
			True. Since $\C_{\infty} \backslash G$ is connected, for $f \in H(G)$ there exists a sequence of polynomials $\{p_n\}$ on $G$ such that $p_n \to f$ uniformly, by Corollary 1.15. That is, the set of polynomials is dense in $H(G)$.
		\end{proof}
    \item The set of polynomials is dense in $A(K)$.
    \begin{proof}
			False. Consider $f(z) = 1/z$ on $K$. We know $\int_{|z| = 1} f = 2\pi i$. But then $\int_{|z| = 1} p_n = 0$ for any polynomial $p_n$, and there does not exist $\{p_n\}$ that converges to $f$ uniformly on $K$. Hence, the set of polynomials is not dense in $A(K)$.
		\end{proof}
    \item If $f$ is analytic on a neighborhood of $K$, then $f$ can be uniformly approximated on $K$ by Laurent polynomials.
    
		\begin{proof}
			True. Let $E = \{0, \infty\}$. Since $E$ meets each component of $\C_{\infty} \backslash K$, Runge's Theorem furnishes a sequences of rational function $\{R_n\}$ which only have poles in $E$ that converges uniformly to $f$ on $K$. But then $R_n$ are Laurent polynomials. 
		\end{proof}
	\end{enumerate}
\end{homeworkProblem}
\end{document}