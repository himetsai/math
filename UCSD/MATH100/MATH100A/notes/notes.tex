\documentclass{article}

\usepackage{fancyhdr}
\usepackage{extramarks}
\usepackage{amsmath}
\usepackage{amsthm}
\usepackage{amssymb}
\usepackage{amsfonts}
\usepackage{tikz}
\usepackage[plain]{algorithm}
\usepackage{algpseudocode}
\usepackage{enumitem}

\usetikzlibrary{automata,positioning}

%
% Basic Document Settings
%

\topmargin=-0.45in
\evensidemargin=0in
\oddsidemargin=0in
\textwidth=6.5in
\textheight=9.0in
\headsep=0.25in

\linespread{1.1}

\pagestyle{fancy}
\lhead{\hmwkAuthorName}
\chead{\textsc{\hmwkHeader}}
\cfoot{\thepage}

\renewcommand\headrulewidth{0.4pt}
\renewcommand\footrulewidth{0.4pt}

\setlength\parindent{0pt}
\setlength{\parskip}{5pt}

\newenvironment{thm}[1]{\subsection*{Theorem #1.}}{}
\newenvironment{lemma}[1]{\subsection*{Lemma #1.}}{}
\newenvironment{defn}{\subsection*{Definition.}}{}
\newenvironment{defnlemma}[1]{\subsection*{Definition-Lemma #1.}}{}
\newenvironment{corollary}[1]{
    \def\temp{#1}\def\null{&}\ifx\temp\null
        \subsection*{Corollary.}
    \else
        \subsection*{Corollary #1.}
    \fi
    
}{}

%
% Homework Details
%   - Title
%   - Due date
%   - Class
%   - Section/Time
%   - Instructor
%   - Author
%


\newcommand{\hmwkInstitution}{University of California San Diego}
\newcommand{\hmwkTitle}{\textsc{MATH 100 Notes}}
\newcommand{\hmwkHeader}{Abstract Algebra Notes}
\newcommand{\hmwkTextbook}{Abstract Algebra by I.N. Herstein (3rd ed.)}
\newcommand{\hmwkAuthorName}{Ray Tsai}

%
% Title Page
%
\title{
    \vspace{2in}
    \textsc{\Large\hmwkInstitution} \\
    \vspace{0.2in}
    \textmd{\textbf{\hmwkTitle}}\\
    \vspace{0.2in}\large{Textbook: \textit{\hmwkTextbook}}
}

\author{
  Organized by \hmwkAuthorName
}
\date{}

\renewcommand{\part}[1]{\textbf{\large Part \Alph{partCounter}}\stepcounter{partCounter}\\}

%
% Various Helper Commands
%

% Useful for algorithms
\newcommand{\alg}[1]{\textsc{\bfseries \footnotesize #1}}

% For derivatives
\newcommand{\deriv}[1]{\frac{\mathrm{d}}{\mathrm{d}x} (#1)}

% For partial derivatives
\newcommand{\pderiv}[2]{\frac{\partial}{\partial #1} (#2)}

% Integral dx
\newcommand{\dx}{\mathrm{d}x}

% Probability commands: Expectation, Variance, Covariance, Bias
\newcommand{\Var}{\mathrm{Var}}
\newcommand{\Cov}{\mathrm{Cov}}
\newcommand{\Bias}{\mathrm{Bias}}
\newcommand*{\Z}{\mathbb{Z}}
\newcommand*{\Q}{\mathbb{Q}}
\newcommand*{\R}{\mathbb{R}}
\newcommand*{\C}{\mathbb{C}}
\newcommand*{\N}{\mathbb{N}}
\newcommand*{\prob}{\mathds{P}}
\newcommand*{\E}{\mathds{E}}

\begin{document}


\maketitle

\thispagestyle{empty}
\clearpage
\pagenumbering{arabic} 

\pagebreak

\rhead{MATH 100A}

\begin{center}
    \section*{\\ MATH 100A}
\end{center}

\vspace{0.2in}

\begin{defn}
    A nonempty set $G$ is said to be a \textit{group} if in $G$ there is defined an operation $*$ such that:
    \begin{enumerate}[label=(\alph*)]
        \item $a, b \in G$ implies that $a * b \in G$. (\textit{Closure})
        \item Given $a, b ,c \in G$, then $a * (b * c) = (a * b) * c$. (\textit{Associativity})
        \item There exists a special element $e \in G$ such that $a * e = e * a = a$ for all $a \in G$. (\textit{Identity element})
        \item For every $a \in G$ there exists an element $b \in G$ such that $a * b = b * a = e$. (\textit{Inverse element})
    \end{enumerate}
\end{defn}

\begin{lemma}{1.3.1}
    If $h: S \rightarrow T$, $g: T \rightarrow U$, and $f: U \rightarrow V$, then $f \circ (g \circ h) = (f \circ g) \circ h$.
    
    \textit{Note: overpowered for checking associativity}
\end{lemma}

\begin{defn}
    A group $G$ is said to be a \textit{abelian} if $a * b = b * a$ , for all $a, b \in G$.
\end{defn}

\begin{lemma}{2.2.1}
    If $G$ is a group, then:
    \begin{enumerate}[label=(\alph*)]
        \item Its identity element is \textit{unique}.
        \item Every $a \in G$ has a \textit{unique} inverse $a^{-1} \in G$.
        \item If $a \in G$, $(a^{-1})^{-1} = a$.
        \item For $a, b \in G$, $(ab)^{-1} = b^{-1}a^{-1}$.
    \end{enumerate}
\end{lemma}

\begin{lemma}{2.2.2}
    In any group $G$ and $a, b, c \in G$, we have:
    \begin{enumerate}[label=(\alph*)]
        \item If $ab = ac$, then $b = c$.
        \item If $ba = ca$, then $b = c$.
    \end{enumerate}
\end{lemma}

\begin{defn}
    A nonempty subset, $H$, of a group $G$ is called a \textit{subgroup} of $G$ if, relative to the product in $G$, $H$ itself forms a group.
\end{defn}

\begin{lemma}{2.3.1}
    A nonempty subset $A \subset G$ is a subgroup of $G$ if and only if $A$ is closed with respect to the operation of $G$ and, given $a \in A$, then $a^{-1} \in A$.
\end{lemma}

\begin{defnlemma}{8.4}
    Let $G$ be a group, and let $S \subseteq G$. The \textit{subgroup generated by $S$}, denoted as $\langle S \rangle $, is the smallest subgroup containing $S$.

    \textit{Note: From Lecture 5.}
\end{defnlemma}

\begin{defn}
    The \textit{cyclic subgroup of} $G$ generated by $a$ is a set $\{a^i \, | \, i \in \Z\}$. It is denoted $\langle a \rangle$.
\end{defn}

\begin{defnlemma}{6.5}
    Let $G$ be a group, and let $g \in G$. The \textit{centralizer} of $g$ is defined to be
    \[
        C(g) = \{h \in G \, | \, hg = gh\}.  
    \]
    Then, $C(g)$ is a subgroup of $G$.

    \textit{Note: From Lecture 3.}
\end{defnlemma}

\begin{lemma}{2.3.2}
    Suppose that $G$ is a group and $H$ a nonempty \textit{finite} subset of $G$ closed under the product in $G$. Then $H$ is a subgroup of $G$.
\end{lemma}

\begin{corollary}{&}
    If $G$ is a finite group and $H$ a nonempty subset of $G$ closed under multiplication, then $H$ is a subgroup of $G$.
\end{corollary}

\begin{defn}
    A relation is $\sim$ on a set $S$ is called an \textit{equivalence relation} if, for all $a, b, c \in S$, it satisfies:
    \begin{enumerate}[label=(\alph*)]
        \item $a \sim a$. (\textit{reflexivity})
        \item $a \sim b$ implies that $b \sim a$. (\textit{symmetry})
        \item $a \sim b, b \sim c$ implies that $a \sim c$. (\textit{transitivity})
    \end{enumerate}
\end{defn}

\begin{lemma}{7.2}
    Let $G$ be a group and let $H$ be a subgroup. Let $\sim$ be the relation on $G$ if and only if $b^{-1}a \in H$. Then $\sim$ is an equivalence relation.

    \textit{Note: From Lecture 4.}
\end{lemma}

\begin{defn}
    If $\sim$ is an equivalence relation on $S$, then $[a]$, the \textit{class} of $a$, is defined by $[a] = \{b \in S \, | \, b \sim a\}$.
\end{defn}

\begin{thm}{2.4.1}
    If $\sim$ is an equivalence relation on $S$, then $S = \cup[a]$, where this union runs over one element from each class, and where $[a] \neq [b]$ implies that $[a] \cap [b] = \emptyset$. That is, $\sim$ \textit{partition} $S$ into equivalence classes.
\end{thm}

\begin{defnlemma}{7.7}
    Let $G$ be a group and let $H$ be a subgroup. Let $g \in G$.
    \[
        [g] = gH = \{gh \, | \, h \in H\}  
    \]
    $gH$ is called a \textit{left coset}.

    \textit{Note: From Lecture 4.}
\end{defnlemma}

\begin{defn}
    Let $G$ be a group and let $H$ be a subgroup. The \textit{index} of $H$ in $G$, denoted $[G; H]$, is equal the number of left cosets of $H$ in $G$.

    \textit{Note: From Lecture 4.}
\end{defn}

\begin{thm}{2.4.2 (Lagrange's Theorem)}
    Let $G$ be a group and let $H$ be a subgroup. Then
    \[
        |H| \cdot [G; H] = |G|.
    \]
    In particular, if $G$ is finite, then the order of $H$ divides the order of $G$.

    \textit{Note: From Lecture 4.}
\end{thm}

\begin{lemma}{8.3}
    Let $G$ be a group and let $H_i$, $i \in I$ be a collection of subgroups. Then $\bigcap_{i \in I} H_i$ is a subgroup.

    \textit{Note: From Lecture 5.}
\end{lemma}

\begin{thm}{2.4.3}
    A group $G$ of prime order is cyclic.
\end{thm}

\begin{defn}
    If $G$ is finite, then the \textit{order} of $a$, written $o(a)$, is the \textit{least positive integer $m$} such that $a^m = e$.

    \textit{Note: $o(a) = |\langle a \rangle|$.}
\end{defn}

\begin{thm}{2.4.4}
    If $G$ is finite and $a \in G$, then $o(a) \, | \, |G|$.
\end{thm}

\begin{thm}{2.4.5}
    If $G$ is a finite group of order $n$, then $a^n = e$ for all $a \in G$.
\end{thm}

\begin{lemma}{9.3}
    Let $G$ be a cyclic group generated by $a$. Then,
    \begin{enumerate}[label=(\alph*)]
        \item $G$ is abelian.
        \item If $G$ is infinite, then $G = \{a^i \, | \, i \in \Z\}$.
        \item If $G$ is of finite $n$, then $G$ is precisely $\{e, a, a^2, \dots, a^{n - 1}\}$.
    \end{enumerate}

    \textit{Note: From Lecture 5.}
\end{lemma}

\begin{thm}{2.4.6}
    $\Z_n$ forms a cyclic group under the addition $[a] + [b] = [a + b]$.
\end{thm}

\begin{defn}
    The \textit{Euler $\varphi$-function}, $\varphi(n)$, is defined by $\varphi(1) = 1$ and, for $n > 1$, $\varphi(n) = $ the number of positive integers $m$ with $1 \leq m < n$ such that $(m, n) = 1$.
\end{defn}

\begin{thm}{2.4.7}
    $U_n$ forms an abelian group, under the product $[a][b] = [ab]$, of order $\varphi(n)$.
\end{thm}

\begin{thm}{2.4.8 (Euler)}
    If $a$ is an integer relatively prime to $n$, then $a^{\varphi(n)} \equiv 1 \mod n$.
\end{thm}

\begin{corollary}{(Fermat)}
    If $p$ is a prime and $p \nmid a$, then
    \[
        a^{p-1} \equiv 1 \mod p.
    \]
    For any integer $b$, $b^p \equiv b \mod p$.
\end{corollary}
 
\end{document}