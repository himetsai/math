\documentclass{article}

\usepackage{fancyhdr}
\usepackage{extramarks}
\usepackage{amsmath}
\usepackage{amsthm}
\usepackage{amsfonts}
\usepackage{tikz}
\usepackage[plain]{algorithm}
\usepackage{algpseudocode}
\usepackage{enumitem}

\usetikzlibrary{automata,positioning}

%
% Basic Document Settings
%

\topmargin=-0.45in
\evensidemargin=0in
\oddsidemargin=0in
\textwidth=6.5in
\textheight=9.0in
\headsep=0.25in

\linespread{1.1}

\pagestyle{fancy}
\lhead{\hmwkAuthorName}
\chead{\hmwkClass:\ \hmwkTitle}
\rhead{\firstxmark}
\lfoot{\lastxmark}
\cfoot{\thepage}

\renewcommand\headrulewidth{0.4pt}
\renewcommand\footrulewidth{0.4pt}

\setlength\parindent{0pt}
\setlength{\parskip}{5pt}

%
% Create Problem Sections
%

\newcommand{\enterProblemHeader}[1]{
    \nobreak\extramarks{}{Problem \arabic{#1} continued on next page\ldots}\nobreak{}
    \nobreak\extramarks{Problem \arabic{#1} (continued)}{Problem \arabic{#1} continued on next page\ldots}\nobreak{}
}

\newcommand{\exitProblemHeader}[1]{
    \nobreak\extramarks{Problem \arabic{#1} (continued)}{Problem \arabic{#1} continued on next page\ldots}\nobreak{}
    \stepcounter{#1}
    \nobreak\extramarks{Problem \arabic{#1}}{}\nobreak{}
}

\setcounter{secnumdepth}{0}
\newcounter{partCounter}
\newcounter{homeworkProblemCounter}
\setcounter{homeworkProblemCounter}{1}
\nobreak\extramarks{Problem \arabic{homeworkProblemCounter}}{}\nobreak{}

%
% Homework Problem Environment
%
% This environment takes an optional argument. When given, it will adjust the
% problem counter. This is useful for when the problems given for your
% assignment aren't sequential. See the last 3 problems of this template for an
% example.
%
\newenvironment{homeworkProblem}[1][-1]{
    \ifnum#1>0
        \setcounter{homeworkProblemCounter}{#1}
    \fi
    \section{Problem \arabic{homeworkProblemCounter}}
    \setcounter{partCounter}{1}
    \enterProblemHeader{homeworkProblemCounter}
}{
    \exitProblemHeader{homeworkProblemCounter}
}

%
% Homework Details
%   - Title
%   - Due date
%   - Class
%   - Section/Time
%   - Instructor
%   - Author
%

\newcommand{\hmwkTitle}{Homework\ \#5}
\newcommand{\hmwkDueDate}{November 9, 2023}
\newcommand{\hmwkClass}{MATH 100A}
\newcommand{\hmwkClassTime}{Section A02 5:00PM - 5:50PM}
\newcommand{\hmwkSectionLeader}{Castellano}
\newcommand{\hmwkClassInstructor}{Professor McKernan}
\newcommand{\hmwkSource}{Source Consulted: Textbook, Lecture, Discussion}
\newcommand{\hmwkAuthorName}{\textbf{Ray Tsai}}
\newcommand{\hmwkPID}{A16848188}

%
% Title Page
%

\title{
    \vspace{2in}
    \textmd{\textbf{\hmwkClass:\ \hmwkTitle}}\\
    \normalsize\vspace{0.1in}\small{Due\ on\ \hmwkDueDate\ at 12:00pm}\\
    \vspace{0.1in}\large{\textit{\hmwkClassInstructor}} \\
    \vspace{0.1in}\small\hmwkClassTime \\
    \small Section Leader: \hmwkSectionLeader \\
    \vspace{0.1in}\small\hmwkSource \\
    \vspace{3in}
}

\author{
  \hmwkAuthorName \\
  \vspace{0.1in}\small\hmwkPID
}
\date{}

\renewcommand{\part}[1]{\textbf{\large Part \Alph{partCounter}}\stepcounter{partCounter}\\}

%
% Various Helper Commands
%

% Useful for algorithms
\newcommand{\alg}[1]{\textsc{\bfseries \footnotesize #1}}

% For derivatives
\newcommand{\deriv}[1]{\frac{\mathrm{d}}{\mathrm{d}x} (#1)}

% For partial derivatives
\newcommand{\pderiv}[2]{\frac{\partial}{\partial #1} (#2)}

% Integral dx
\newcommand{\dx}{\mathrm{d}x}

% Probability commands: Expectation, Variance, Covariance, Bias
\newcommand{\Var}{\mathrm{Var}}
\newcommand{\Cov}{\mathrm{Cov}}
\newcommand{\Bias}{\mathrm{Bias}}
\newcommand*{\Z}{\mathbb{Z}}
\newcommand*{\Q}{\mathbb{Q}}
\newcommand*{\R}{\mathbb{R}}
\newcommand*{\C}{\mathbb{C}}
\newcommand*{\N}{\mathbb{N}}
\newcommand*{\prob}{\mathds{P}}
\newcommand*{\E}{\mathds{E}}

\begin{document}

\maketitle

\pagebreak

\begin{homeworkProblem}

    Determine in each of the part if the given mapping is a homomorphism. If so, identify its kernal and whether or not the mapping is 1-1 or onto.
    
    \begin{enumerate}[label=(\alph*)]
        \item $G = \Z$ under $+$, $G' = \Z_n$, $\phi(a) = [a]$ for $a \in \Z$.

        \begin{proof}
            Let $a, b \in \Z$. Since $\phi(a)\phi(b) = [a] + [b] = [a + b] = \phi(ab)$, $\phi$ is indeed a homomorphism. The kermal of $\phi$ is the set of elements $a \in G$ such that $\phi(a) = [0]$, namely Ker $\phi = \{a \in G \, | \, a = kn, k \in \Z\}$. Since Ker $\phi$ is not trivial, $\phi$ is not a 1-1 mapping. Lastly, $\phi$ is obviously onto, as for all $[a] \in \Z_n$ we have $\phi(a) = [a]$.
        \end{proof}

        \item $G$ group, $\phi: G \rightarrow G$ defined by $\phi(a) = a^{-1}$ for $a \in G$.

        \begin{proof}
            Let $a, b \in G$. Since $\phi(a)\phi(b) = a^{-1}b^{-1} = (ba)^{-1} \neq (ab)^{-1} = \phi(ab)$, $\phi$ is not a homomorphism unless $G$ is abelian.
        \end{proof}

        \item $G$ abelian group, $\phi: G \rightarrow G$ defined by $\phi(a) = a^{-1}$ for $a \in G$.

        \begin{proof}
            We already know $\phi$ is a homomorphism, from part $b$. The kernal of $\phi$ is simply $\{e\}$ because $e$ is the only element that has $e$ as its inverse, and so $\phi$ is a 1-1 mapping. Since for all $c \in G$, we have $\phi(c^{-1}) = c$, $\phi$ is also onto.
        \end{proof}

        \item $G$ group of all non-zero real numbers under multiplication, $G' = \{1, -1\}$, $\phi(r) = 1$ if $r$ is positive, $\phi(r) = -1$ if $r$ is negative.

        \begin{proof}
            Let $a, b \in G$. If $a, b$ has the same sign, we know both $ab$ and $\phi(a)\phi(b)$ are positive, and so $\phi(a)\phi(b) = 1 = \phi(ab)$. The converse also holds true, as both $ab$ and $\phi(a)\phi(b)$ are negative, which implies $\phi(a)\phi(b) = -1 = \phi(ab)$. Thus, $\phi$ is indeed a homomorphism. The kernal of $\phi$ is the set of all non-zero real numbers that get map to $1$, which contains all positive real numbers. Thus, $\phi$ is not a 1-1 mapping. However, since we can map $1$ and $-1$ to themselves from $G$ to $G'$ respectively, $\phi$ is onto.
        \end{proof}

        \item $G$ and abelian group, $n > 1$ a fixed integer, and $\phi: G \rightarrow G$ defined by $\phi(a) = a^n$ for $a \in G$.

        \begin{proof}
            Let $a, b \in G$. Since $G$ is abelian, $\phi(a)\phi(b) = a^nb^n = (ab)^n = \phi(ab)$, and so $\phi$ is a homomorphism. The kernal of $\phi$ is the set of elements $a \in G$ such that $a^n = e$, which means that the order of $a$ must divide $n$ for $a$ to be in Ker $\phi$. Thus, $\phi$ is not injective unless $o(a) \not | \, n$ for all $a$. Also, we claim that $\phi$ is not onto. Consider a group of order $2$, namely $G = \{e, a\}$, and let $n = 2$. $G$ is obviously abelian. Notice that $n = |G|$, so $\phi(g) = e$ for all $g \in G$. This implies that there does not exist $g$ such that $\phi(g) = a$, which implies that $\phi$ is not onto.
        \end{proof}
    \end{enumerate}
\end{homeworkProblem}

\pagebreak

\begin{homeworkProblem}
    Verify that in Example 9 of Section 1, the set $H = \{i, g, g^2, g^3\}$ is a normal subgroup of $G$, the dihedral group of order 8.

    \begin{proof}
        We first prove that $gf = fg^{-1}$. Note that since $e = g^4$, $g^{-1} = g^3 = (y, -x)$. On LHS, we have
        \[
            (g * f)(x, y) = g(f(x, y)) = g(-x, y) = (-y, -x).
        \]
        On RHS, we have
        \[
            (f * g^{-1})(x, y) = f(g^3(x,y)) = f(y, -x) = (-y, -x),
        \]
        and thus $g * f = f * g^{-1} = (-y, -x)$.

        We then show that $g^nf = fg^{-n}$ by induction. The base case $gf = fg^{-1}$ is done above. For $n > 1$, we get
        \begin{gather}
            g^nf = g(g^{n-1}f) = (gf)g^{-(n - 1)} = fg^{-n},
        \end{gather}
        by induction.

        Let $a = f^ig^jf^{-i} \in f^iHf^{-i}$. We can assume $f^i = f$, otherwise $a = ig^ji = g^j \in H$, and we are done. By the result we proved above, $a = fg^jf^{-1} = g^jff^{-1} = g^j \in H$. Thus, we know $f^iHf^{-i} \subset H$.

        Let $b = f^kg^l \in G$. Then, we know $bg^jb^{-1} = f^kg^{l}g^jg^{-l}f^{-k} = f^kg^{j}f^{-k} \in f^iHf^{-i} \subset H$, and thus we know $H$ is a normal subgroup of $G$.
    \end{proof}
\end{homeworkProblem}

\pagebreak

\begin{homeworkProblem}
    Prove that if $Z(G)$ is the center of $G$, then $Z(G) \vartriangleleft G$.

    \begin{proof}
        Let $z \in Z(G)$ and $g \in G$. We know $zg = gz$, and so $gzg^{-1} = z \in Z(G)$. Thus, $gZ(G)g^{-1} \subset Z(G)$ for all $g$, and we are done.
    \end{proof}
\end{homeworkProblem}

\pagebreak

\begin{homeworkProblem}
    If $N \vartriangleleft G$ and $M \vartriangleleft G$ and $MN = \{mn \, | \, m \in M, n \in N\}$, prove that $MN$ is a subgroup of $G$ and that $MN \vartriangleleft G$.

    \begin{proof}
        We first check that $MN$ is a subgroup of $G$. Since $M, N$ are normal subgroups, they are non-empty, and so $MN$ is non-empty. 

        Let $m_1n_1, m_2n_2 \in MN$, where $m_1, m_2 \in M$ and $n_1, n_2 \in N$. Since $N \vartriangleleft G$, we know $n_1m_2 = m_2n_1'$, for some $n_1' \in N$. This immediately follows that $(m_1n_1)(m_2n_2) = m_1(m_2n_1')n_2 = mn$, for some $m = m_1m_2 \in M$ and $n = n_1'n_2 \in N$, and thus $MN$ is closed under multiplication.

        Since $N \vartriangleleft G$, $(m_1n_1)^{-1} = n_1^{-1}m_1^{-1} = m_1^{-1}n' \in MN$, for some $n' \in N$. Thus, $MN$ is closed under inverse, and so $MN$ is indeed a subgroup of $G$.

        We now prove that $MN \vartriangleleft G$ Let $gmng^{-1} \in gMNg^{-1}$, where $g \in G$, $m \in M$, and $n \in N$. Since $N, M \vartriangleleft G$, $gmng^{-1} = gmg^{-1}n' = gg^{-1}m'n' = m'n' \in MN$, for some $m' \in M$ and $n' \in N$. Thus, $gMNg^{-1} \subset MN$, and this completes the proof.
    \end{proof}
\end{homeworkProblem}

\pagebreak

\begin{homeworkProblem}
    Let $G = S_3$, the symmetric group of degree 3 and let $H = \{i, f\}$, where $f(x_1) = x_2, f(x_2) = x_1, f(x_3) = x_3$.

    \begin{enumerate}[label=(\alph*)]
        \item Write down all the left cosets of $H$ in $G$.
        
        \begin{proof}
            We know $S_3 = \{a, b, c, d, f, i\}$, where 
            \begin{align*}
                a = (1, 2, 3) && b = (1, 3, 2) && c = (2, 3) \\
                d = (1, 3) && f = (1, 2) && i = ().
            \end{align*}
            Then, the left cosets of $H$ are $iH = \{i, f\}, aH = \{a, d\}, bH = \{b, c\}$.
        \end{proof}

        \item Write down all the right cosets of $H$ in $G$.

        \begin{proof}
            The right cosets are $Hi = \{i, f\}, Ha = \{a, c\}, Hb = \{b, d\}$.
        \end{proof}

        \item Is every left coset of $H$ a right coset of $H$?
        
        \begin{proof}
            No. $aH \neq Ha$.
        \end{proof}
    \end{enumerate}
\end{homeworkProblem}

\pagebreak

\begin{homeworkProblem}
    Let $G$ be a group such that all subgroups of $G$ are normal in $G$. If $a, b \in G$, prove that $ba = a^jb$ for some $j$.

    \begin{proof}
        Since $\langle a \rangle$ is a subgroup of $G$ and all subgroups of $G$ are normal, $bab^{-1} \in \langle a \rangle$, and so $bab^{-1} = a^j$ for some $j$. This immediately follows that $ba = a^jb$.
    \end{proof}
\end{homeworkProblem}

\pagebreak

\begin{homeworkProblem}
    If $G$ is a group and $a \in G$, define $\sigma_a: G \rightarrow G$ by $\sigma_a(g) = aga^{-1}$. We saw in Example 9 in this section that $\sigma_a$ is an isomorphism of $G$ onto itself, so $\sigma_a \in A(G)$, the group of all 1-1 mappings of $G$ (as a set) onto itself. Define $\psi:G \rightarrow A(G)$ by $\psi(a) = \sigma_a$ for all $a \in G$. Prove that:

    \begin{enumerate}[label=(\alph*)]
        \item $\psi$ is a homomorphism of $G$ into $A(G)$.
        
        \begin{proof}
            Let $a, b \in G$. Since $\psi(a)\psi(b) = \sigma_a\circ\sigma_b(g) = (ab)g(b^{-1}a^{-1}) = \psi(ab)$, $\psi$ is a homomorphism.
        \end{proof}

        \item Ker $\psi = Z(G)$, the center of $G$.
        
        \begin{proof}
            Note that the identity elemtnt of $A(G)$ is the identity mapping $\sigma_e(g) = g$. Let $a \in \text{Ker }\psi$. Then $\sigma_a(g) = aga^{-1} = g$. This immediately follows that $ag = ga$, for all $g \in G$, and so $a \in Z(G)$, which implies Ker $\psi \subset Z(G)$. Let $b \in Z(G)$. Since $bg = gb$ for all $g \in G$, we know $\sigma_b = bgb^{-1} = g$, so $Z(G) \subset \text{Ker }\psi$. Therefore, we conclude that Ker $\psi = Z(G)$.
        \end{proof}
    \end{enumerate}
\end{homeworkProblem}

\pagebreak

\begin{homeworkProblem}
    Let $\theta, \psi$ be automorphism of $G$, and let $\theta\psi$ be the product of $\theta$ and $\psi$ as mappings on $G$. Prove that $\theta\psi$ is an automorphism of $G$, and that $\theta^{-1}$ is an automorphism of $G$, so that the set of all automorphisms of $G$ is itself a group.

    \begin{proof}
        Let $a, b \in G$. We first show that the set of all automorphisms of $G$ is closed under multiplication. We know
        \[
            \theta\psi(a)\theta\psi(b) = \theta(\psi(a))\theta(\psi(b)) = \theta(\psi(a)\psi(b)) = \theta(\psi(ab)) = \theta\psi(ab),
        \]
        so $\theta\psi$ is a homomorphism. This immediately follows that since $\theta$ and $\psi$ are bijective mappings, their composition $\theta\psi$ is also bijective, which makes $\theta\psi$ an automorphism. Since $\theta: G \rightarrow G$ is a bijective mapping, there exists a bijective mapping $\theta^{-1}: G \rightarrow G$, such that $\theta\theta^{-1}(g) = \theta^{-1}\theta(g) = g$. Thus, $\theta^{-1}$ is also an automorphism, and this completes the proof.
    \end{proof}
\end{homeworkProblem}

\pagebreak

\begin{homeworkProblem}
    If $G$ is a nonabelian group of order 6, prove that $G \simeq S_3$.

    \begin{proof}
        We first show that there must exists an element in $G$ that is of order $2$. Let $G = \{e, a, b, c, d, f\}$, where $e$ is the identity element. By Lagrange's Theorem, we know the orders of the elements in $G$ must be one of $1, 2, 3, 6$. Notice that $G$ is nonabelian, so $G$ is not a cyclic group, which implies that no element in $G$ is of order 6. Suppose for the sake of contradiction that there are no elements in $G$ that are of order $2$. Then, each of the non-identity elements in $G$ must have an order of $3$. Suppose without loss of generality that $c = a^2$ and $d = b^2$. We investigate on $f^2$. $f^2$ cannot be $a$, otherwise $c =a^2 = f^4 = f$. $f^2$ cannot be $c$, otherwise $a = a^4 = c^2 = f^4 = f$. The same arguments apply for $b$ and $d$, and thus we reach a contradiction. Suppose that $f$ is the element of order $2$ in $G$. Let $H = \{e, f\}$ be the cyclic subgroup of $G$, and let $S = \{Hk \, | \, k \in G\}$ be the set of all right cosets of $H$ in $G$. Define,
        for $g \in G$, $T_g: S \rightarrow S$ by $T_g(Hk) = Hkg^{-1}$. Notice that since $|S| = [G:H] = 3$, $A(S) \simeq S_3$. For $m, n \in G$, we know $T_mT_n(Hk) = T_m(Hkn^{-1}) = Hkn^{-1}m^{-1} = Hk(mn)^{-1} = T_{mn}(Hk)$, and so the function $\psi: G \rightarrow A(S) \simeq S_3$ defined by $\psi(g) = T_g$ is a homomorphism. We now show that $\psi$ is injective by investigating its kernal. Suppose that $l \in \text{Ker }\psi$. Then $\psi(l) = T_l = T_e$. This implies that $Hl^{-1} = T_l(H) = T_e(H) = H$, and so $l \in H$. Consider $T_l(Hk)$, for some $k \neq f$. $T_l(Hk) = Hkl^{-1} = Hk$, and so $klk^{-1} \in H$. Suppose for the sake of contradiction that $l = f$. $kfk^{-1} \neq e$, otherwise we get $f = e$, contradiction. Thus we can assume $kfk^{-1} = f$, namely $kf = fk$. Notice that since $\langle f, k \rangle$ contains a subgroup $H$ of order 2, by Lagrange's Theorem it must have even order, and so $\langle f, k \rangle$ is of order $6$ and thus it generates $G$. However,
        since $f$ and $k$ commute, $\langle f, k \rangle = G$ is abelian, contradiction. Therefore, we know $kfk^{-1} \notin H$, and so $l = e$. It immediately follows that $\psi$ is injective since Ker $\psi$ is trivial, and this completes the proof.
    \end{proof}
\end{homeworkProblem}

\pagebreak

\begin{homeworkProblem}
    If $G$ is the group of all nonzero real numbers under multiplication and $N$ is the subgroup of all positive real numbers, write out $G/N$ by exhibiting the cosets of $N$ in $G$, and construct the multiplication in $G/N$.

    \begin{proof}
        Since multiplication is commutative for real numbers, $gN = Ng$ for all $g \in G$, and thus $N$ is normal. Notice that $gN = N$ if $g$ is positive and $gN = -N$, the set of all negative real numbers, if $g$ is negative. Thus, $G/N = \{N, -N\} = \{[1], [-1]\}$, where $[g] = \{x \in G \, | \, xg^{-1} \in N\}$. Since $N$ is normal in $G$, $G/N$ is relative to the operation $[a][b] = [ab]$, for $a, b \in G$.
    \end{proof}
\end{homeworkProblem}

\pagebreak

\begin{homeworkProblem}
    If $G$ is the group of nonzero real numbers under multiplication and $N = \{1, -1\}$, show how you can "identify" $G/N$ as the group of all positive real numbers under multiplication. What are the cosets of $N$ in $G$?

    \begin{proof}
        Since multiplication is commutative for real numbers, $gN = Ng$ for all $g \in G$, and thus $N$ is normal. Notice that $gN = \{g, -g\}$, which implies that numbers of the same absolute value are put into the same calss, namely $G/N = \{[a] \, | \, a \in R_{>0}\}$. Since $N$ is normal in $G$, $G/N$ is relative to the operation $[a][b] = [ab]$, for $a, b \in G$, and this makes $G/N$ the group of all positive real numbers under multiplication. The cosets of $N$ in $G$ is simply all the elements in $G/N$ by definition.
    \end{proof}
\end{homeworkProblem}

\pagebreak

\begin{homeworkProblem}
    If $G$ is a group and $N \vartriangleleft G$, show that if $\bar{M}$ is a subgroup of $G/N$ and $M = \{a \in G \, | \, Na \in \bar{M}\}$, then $M$ is a subgroup of $G$, and $N \subset M$.

    \begin{proof}
        Let $a, b \in M$. We know $Na, Nb \in \bar{M}$. Since $N$ is normal and $\bar{M}$ is a subgroup, $NaNb = N(ab) \in M$, so $ab \in M$. Thus, $M$ is closed under multiplication. Since $N$ is the identity element in $G/N$, we know $N \in \bar{M}$, and so there exists $Nc \in \bar{M}$ such that $NaNc = Nac = N$. This immediately follows that there exists $n' \in N$ such that $n'ac = e$, and so we get $a^{-1} = cn'$. We can easily check that $a^{-1} \in M$, as $Na^{-1} = Ncn' = Nc \in \bar{M}$. Thus, $M$ is also closed under taking inverse, and so $M$ is indeed a subgroup of $G$. We already know $N \in \bar{M}$, so if $n \in N$, then $Nn = N \in \bar{M}$, and thus $N \subset M$.
    \end{proof}
\end{homeworkProblem}

\pagebreak

\begin{homeworkProblem}
    If $\bar{M}$ in Problem 3 is normal in $G/N$, show that the $M$ defined is normal in $G$.

    \begin{proof}
        Let $m \in M$ and $g \in G$. Since $\bar{M}$ is normal in $G/N$, $NgNmNg^{-1} = N(gmg^{-1}) \in \bar{M}$, and thus $gmg^{-1} \in M$. Therefore, $M$ is normal in $G$.
    \end{proof}
\end{homeworkProblem}
\end{document}