\documentclass{article}

\usepackage{fancyhdr}
\usepackage{extramarks}
\usepackage{amsmath}
\usepackage{amsthm}
\usepackage{amsfonts}
\usepackage{tikz}
\usepackage[plain]{algorithm}
\usepackage{algpseudocode}
\usepackage{enumerate}
\usepackage{amssymb}
\usepackage[margin=1in]{geometry}

\newcommand{\st}{~\mid~}
\newcommand{\ind}{$~~~$}
\usepackage{xcolor}

\graphicspath{ {./../images} }

\usetikzlibrary{automata,positioning}

%
% Basic Document Settings
%

\topmargin=-0.45in
\evensidemargin=0in
\oddsidemargin=0in
\textwidth=6.5in
\textheight=9.0in
\headsep=0.25in

\linespread{1.1}

\pagestyle{fancy}
\lhead{\hmwkAuthorName}
\chead{\hmwkClass:\ \hmwkTitle}
\rhead{\firstxmark}
\lfoot{\lastxmark}
\cfoot{\thepage}

\renewcommand\headrulewidth{0.4pt}
\renewcommand\footrulewidth{0.4pt}

\setlength\parindent{0pt}
\setlength{\parskip}{5pt}

%
% Create Problem Sections
%

\newcommand{\enterProblemHeader}[1]{
    \nobreak\extramarks{}{Problem \arabic{#1} continued on next page\ldots}\nobreak{}
    \nobreak\extramarks{Problem \arabic{#1} (continued)}{Problem \arabic{#1} continued on next page\ldots}\nobreak{}
}

\newcommand{\exitProblemHeader}[1]{
    \nobreak\extramarks{Problem \arabic{#1} (continued)}{Problem \arabic{#1} continued on next page\ldots}\nobreak{}
    \stepcounter{#1}
    \nobreak\extramarks{Problem \arabic{#1}}{}\nobreak{}
}

\setcounter{secnumdepth}{0}
\newcounter{partCounter}
\newcounter{homeworkProblemCounter}
\setcounter{homeworkProblemCounter}{1}
\nobreak\extramarks{Problem \arabic{homeworkProblemCounter}}{}\nobreak{}

%
% Homework Problem Environment
%
% This environment takes an optional argument. When given, it will adjust the
% problem counter. This is useful for when the problems given for your
% assignment aren't sequential. See the last 3 problems of this template for an
% example.
%
\newenvironment{homeworkProblem}[1][-1]{
    \ifnum#1>0
        \setcounter{homeworkProblemCounter}{#1}
    \fi
    \section{Problem \arabic{homeworkProblemCounter}}
    \setcounter{partCounter}{1}
    \enterProblemHeader{homeworkProblemCounter}
}{
    \exitProblemHeader{homeworkProblemCounter}
}

%
% Homework Details
%   - Title
%   - Due date
%   - Class
%   - Section/Time
%   - Instructor
%   - Author
%

\newcommand{\hmwkTitle}{Homework\ \#7}
\newcommand{\hmwkDueDate}{Jun 7, 2024}
\newcommand{\hmwkClass}{CSE 101}
\newcommand{\hmwkClassInstructor}{Professor Jones}
\newcommand{\hmwkAuthorName}{\textbf{Ray Tsai, Kevin Yu}}
\newcommand{\hmwkPID}{A16848188}

%
% Title Page
%

\title{
    \vspace{2in}
    \textmd{\textbf{\hmwkClass:\ \hmwkTitle}}\\
    \normalsize\vspace{0.1in}\small{Due\ on\ \hmwkDueDate\ at 23:59pm}\\
    \vspace{0.1in}\large{\textit{\hmwkClassInstructor}} \\
    \vspace{3in}
}

\author{
  \hmwkAuthorName
}
\date{}

\renewcommand{\part}[1]{\textbf{\large Part \Alph{partCounter}}\stepcounter{partCounter}\\}

%
% Various Helper Commands
%

% Useful for algorithms
\newcommand{\alg}[1]{\textsc{\bfseries \footnotesize #1}}

% For derivatives
\newcommand{\deriv}[1]{\frac{\mathrm{d}}{\mathrm{d}x} (#1)}

% For partial derivatives
\newcommand{\pderiv}[2]{\frac{\partial}{\partial #1} (#2)}

% Integral dx
\newcommand{\dx}{\mathrm{d}x}

% Probability commands: Expectation, Variance, Covariance, Bias
\newcommand{\Var}{\mathrm{Var}}
\newcommand{\Cov}{\mathrm{Cov}}
\newcommand{\Bias}{\mathrm{Bias}}
\newcommand*{\Z}{\mathbb{Z}}
\newcommand*{\Q}{\mathbb{Q}}
\newcommand*{\R}{\mathbb{R}}
\newcommand*{\C}{\mathbb{C}}
\newcommand*{\N}{\mathbb{N}}
\newcommand*{\prob}{\mathds{P}}
\newcommand*{\E}{\mathds{E}}

\begin{document}

\maketitle

\pagebreak

\begin{homeworkProblem}
  You are given a rooted binary tree $T$ on $n$ vertices named $0,\dots, n-1$. The rooted tree is given to you as an adjacency list of children. (You can assume that the vertices are ordered by levels starting from the lowest level (i.e., that vertex $n-1$ is the root.)

  Each vertex $i$ has a positive value $v[i]$.

  You wish to find the path with maximum total sum of vertex values.

  \begin{quote}
    1: Define the subproblems:
    \begin{quote}
      % Let $IN[i]$ be the maximum total sum of vertex values of a path that {\bf includes} vertex $i$ and consists only of vertices hanging from the tree rooted at $i$.
    
      Let $PATH[i]$ be the maximum total sum of vertex values of a path consists only of vertices hanging from the tree rooted at $i$.

      Let $HEIGHT[i]$ be the maximum total sum of vertex values of a path that starts at $i$ and goes to a leaf node in the subtree hanging from $i$.
      \end{quote}
    
    2: Define and evaluate the base cases.
    \begin{quote}
      For each vertex $i$ with no children, 
      \[
        PATH[i] = v[i], \quad HEIGHT[i] = v[i].
      \]
    \end{quote}
    
    3: Establish the recurrence for the tabulation.
    \begin{quote}
      For vertex $i$ with list of children $C$,
      \begin{gather*}
        PATH[i] = \max\left(\max_{c \in C} PATH[i], v[i] + \sum_{c \in C} HEIGHT[c]\right) \\
        HEIGHT[i] = v[i] + \max_{c \in C} HEIGHT[c]
      \end{gather*}
    \end{quote}
    
    4: Determine the order of subproblems:
    \begin{quote}
      Order the subproblems from 0 to $n - 1$.
    \end{quote}
    
    5: Final form of output.
    \[
      PATH[n - 1].
    \]
    \end{quote}
\end{homeworkProblem}

\newpage

\begin{homeworkProblem}
  There is a long straight road with $n$ houses located at positions $x_1,x_2,\dots,x_n$ (measured in miles from the start of the road.) You can assume that $x_1<x_2<\dots < x_n$.

  You have a budget to build $F$ firestations along the road. You want to find positions of the $F$ firestations that minimizes the total distance from each house to its nearest firestation. (You can build a firestation anywhere along the road. You can even build a firestation at the same location of a house.)

  Complete the setup for a DP tabulation algorithm for this problem where the output is the minimum total distance from each house to its nearest firestation.

  \begin{quote}
  1: Define the subproblems:
  \begin{quote}
  Let $c[i,k]$ be defined to be minimum total distance of all houses $\{x_1,\dots,x_i\}$ with a budget of $k$ firestations.
  \end{quote}

  2: Define and evaluate the base cases.

  \begin{quote}
  Notice that the solution to the problem with only 1 firestation is to put the firestation at the location of the middle house.
  $$c[i,1] = \sum_{k=1}^i |x_k-x_{\lceil i/2\rceil}|.$$ Since we can just put a firestation at the location of the house when there is only one house,
  $$c[1,j] = 0.$$
  \end{quote}

  3: Establish the recurrence for the tabulation.
  \begin{quote}
    For $i, j > 1$
    \[
      c[i, j] = \min_{l < i}\left(c[l, j - 1] + \sum_{k=l + 1}^i |x_k-x_{\lceil i/2\rceil}|\right)
    \]
  \end{quote}
  

  4: Determine the order of subproblems:
  \begin{quote}
    Order the subproblems in row-major order. That is, we are traversing through $c$ one row $i$ at a time, from $1$ to $n$, and within row $i$ we traverse through the columns from $1$ to $F$. Each row is completed before moving on to the next row.
  \end{quote}

  5: Final form of output.
  \[
    c[n, F].
  \]
  \end{quote}

\end{homeworkProblem}
\end{document}