\documentclass{article}

\usepackage{fancyhdr}
\usepackage{extramarks}
\usepackage{amsmath}
\usepackage{amsthm}
\usepackage{amsfonts}
\usepackage{tikz}
\usepackage[plain]{algorithm}
\usepackage{algpseudocode}
\usepackage{enumerate}
\usepackage{amssymb}
\usepackage{dsfont}

\usetikzlibrary{automata,positioning}

%
% Basic Document Settings
%

\topmargin=-0.45in \evensidemargin=0in \oddsidemargin=0in \textwidth=6.5in \textheight=9.0in \headsep=0.25in

\linespread{1.1}

\pagestyle{fancy}
\lhead{\hmwkAuthorName}
\chead{\hmwkClass:\ \hmwkTitle}
\rhead{\firstxmark}
\lfoot{\lastxmark}
\cfoot{\thepage}

\renewcommand\headrulewidth{0.4pt} \renewcommand\footrulewidth{0.4pt}

\setlength\parindent{0pt}
\setlength{\parskip}{5pt}

%
% Create Problem Sections
%

\newcommand{\enterProblemHeader}[1]{ \nobreak\extramarks{}{Problem \arabic{#1} continued on next page\ldots}\nobreak{} \nobreak\extramarks{Problem \arabic{#1} (continued)}{Problem \arabic{#1} continued on next page\ldots}\nobreak{} }

\newcommand{\exitProblemHeader}[1]{ \nobreak\extramarks{Problem \arabic{#1} (continued)}{Problem \arabic{#1} continued on next page\ldots}\nobreak{}
    \stepcounter{#1}
    \nobreak\extramarks{Problem \arabic{#1}}{}\nobreak{} }

\setcounter{secnumdepth}{0}
\newcounter{partCounter}
\newcounter{homeworkProblemCounter}
\setcounter{homeworkProblemCounter}{1}
\nobreak\extramarks{Problem \arabic{homeworkProblemCounter}}{}\nobreak{}

%
% Homework Problem Environment
%
% This environment takes an optional argument. When given, it will adjust the problem counter. This is useful for when the problems given for your assignment aren't sequential. See the last 3 problems of this template for an example.
%
\newenvironment{homeworkProblem}[1][-1]{ \ifnum#1>0
        \setcounter{homeworkProblemCounter}{#1}
    \fi
    \section{Problem \arabic{homeworkProblemCounter}}
    \setcounter{partCounter}{1}
    \enterProblemHeader{homeworkProblemCounter} }{ \exitProblemHeader{homeworkProblemCounter} }

%
% Homework Details
%   - Title
%   - Due date
%   - Class
%   - Section/Time
%   - Instructor
%   - Author
%

\newcommand{\hmwkTitle}{Homework\ \#1}
\newcommand{\hmwkDueDate}{Jan 15, 2025}
\newcommand{\hmwkClass}{MATH 190A}
\newcommand{\hmwkClassTime}{Section A02 8:00AM - 8:50AM}
\newcommand{\hmwkSectionLeader}{Zhiyuan Jiang}
\newcommand{\hmwkClassInstructor}{Professor McKernan}
\newcommand{\hmwkSource}{Source Consulted: Textbook, Lecture, Discussion}
\newcommand{\hmwkAuthorName}{\textbf{Ray Tsai}}
\newcommand{\hmwkPID}{A16848188}

%
% Title Page
%

\title{
    \vspace{2in}
    \textmd{\textbf{\hmwkClass:\ \hmwkTitle}}\\
    \normalsize\vspace{0.1in}\small{Due\ on\ \hmwkDueDate\ at 12:00pm}\\
    \vspace{0.1in}\large{\textit{\hmwkClassInstructor}} \\
    \vspace{0.1in}\small\hmwkClassTime \\
    \small Section Leader: \hmwkSectionLeader \\
    \vspace{0.1in}\small\hmwkSource \\
    \vspace{3in}
}

\author{ \hmwkAuthorName \\
  \vspace{0.1in}\small\hmwkPID }
\date{}

\renewcommand{\part}[1]{\textbf{\large Part \Alph{partCounter}}\stepcounter{partCounter}\\}

%
% Various Helper Commands
%

% Useful for algorithms
\newcommand{\alg}[1]{\textsc{\bfseries \footnotesize #1}}

% For derivatives
\newcommand{\deriv}[1]{\frac{\mathrm{d}}{\mathrm{d}x} (#1)}

% For partial derivatives
\newcommand{\pderiv}[2]{\frac{\partial}{\partial #1} (#2)}

% Integral dx
\newcommand{\dx}{\mathrm{d}x}

% Probability commands: Expectation, Variance, Covariance, Bias
\newcommand{\Var}{\mathrm{Var}} 
\newcommand{\Cov}{\mathrm{Cov}} 
\newcommand{\Bias}{\mathrm{Bias}} 
\newcommand*{\Z}{\mathbb{Z}} 
\newcommand*{\Q}{\mathbb{Q}} 
\newcommand*{\R}{\mathbb{R}}
\newcommand*{\C}{\mathbb{C}} 
\newcommand*{\N}{\mathbb{N}} 
\newcommand*{\prob}{\mathds{P}} 
\newcommand*{\E}{\mathds{E}} 
\newcommand*{\T}{\mathcal{T}}

\begin{document}

\maketitle

\pagebreak

\begin{homeworkProblem}
  Find all topologies on the set 
  \[
    X = \{a, b, c\}.
  \]

  \begin{proof}
    \begin{enumerate}
      \item $\{\emptyset, X\}$.
      \item $\{\emptyset, X, \{a\}\}$.
      \item $\{\emptyset, X, \{b\}\}$.
      \item $\{\emptyset, X, \{c\}\}$.
      \item $\{\emptyset, X, \{a, b\}\}$.
      \item $\{\emptyset, X, \{a, c\}\}$.
      \item $\{\emptyset, X, \{b, c\}\}$.
      \item $\{\emptyset, X, \{b, c\}, \{a\}\}$.
      \item $\{\emptyset, X, \{a, c\}, \{b\}\}$.
      \item $\{\emptyset, X, \{a, b\}, \{c\}\}$.
      \item $\{\emptyset, X, \{a, b\}, \{a\}\}$.
      \item $\{\emptyset, X, \{a, b\}, \{b\}\}$.
      \item $\{\emptyset, X, \{a, b\}, \{a\}, \{b\}\}$.
      \item $\{\emptyset, X, \{a, c\}, \{a\}\}$.
      \item $\{\emptyset, X, \{a, c\}, \{c\}\}$.
      \item $\{\emptyset, X, \{a, c\}, \{a\}, \{c\}\}$.
      \item $\{\emptyset, X, \{b, c\}, \{b\}\}$.
      \item $\{\emptyset, X, \{b, c\}, \{c\}\}$.
      \item $\{\emptyset, X, \{b, c\}, \{b\}, \{c\}\}$.
      \item $\{\emptyset, X, \{a, b\}, \{a, c\}, \{a\}\}$.
      \item $\{\emptyset, X, \{a, b\}, \{b, c\}, \{b\}\}$.
      \item $\{\emptyset, X, \{a, c\}, \{b, c\}, \{c\}\}$.
      \item $\{\emptyset, X, \{a, b\}, \{b, c\}, \{a\}, \{b\}\}$.
      \item $\{\emptyset, X, \{a, b\}, \{b, c\}, \{b\}, \{c\}\}$.
      \item $\{\emptyset, X, \{a, c\}, \{b, c\}, \{a\}, \{c\}\}$.
      \item $\{\emptyset, X, \{a, c\}, \{b, c\}, \{b\}, \{c\}\}$.
      \item $\{\emptyset, X, \{a, b\}, \{a, c\}, \{a\}, \{b\}\}$.
      \item $\{\emptyset, X, \{a, b\}, \{a, c\}, \{a\}, \{c\}\}$.
      \item $\{\emptyset, X, \{a, b\}, \{a, c\}, \{b, c\}, \{a\}, \{b\}, \{c\}\}$.
    \end{enumerate}
  \end{proof}
\end{homeworkProblem}

\newpage

\begin{homeworkProblem}
  Find a topology $\mathcal{T}$ on the set 
  \[
    X = \{a, b, c, d\}
  \]
  such that a subset is open if and only if it is closed and yet $(X, \mathcal{T})$ is neither the trivial nor the discrete topology.

  \begin{proof}
    Consider $\T = \{\emptyset, \{a, b\}, \{c, d\}, X\}$. Note that for $S \subseteq X$, $S \in \T$ if and only if $\T \backslash S \in \T$. Hence, a subset of $\T$ is open if and only if it is closed.
  \end{proof}
\end{homeworkProblem}

\newpage

\begin{homeworkProblem}
  Let $X$ be a set and define the function 
  \[
    d: X \times X \to \mathbb{R}
  \]
  by the rule
  \[
    d(x, y) = \begin{cases} 
      0 & \text{if } x = y, \\
      1 & \text{otherwise}.
    \end{cases}
  \]
  \begin{enumerate}[(i)]
    \item Show that $(X, d)$ is a metric space.
    \begin{proof}
      Let $x, y, z \in X$. Obviously, $d(x, x) = 0$, $d(x, y) > 0$ if $x \neq y$, and $d(x, y) = d(y, x)$. If $x = z$, then $d(x, z) \leq d(x, y) + d(y, z)$ is satisfied. If $x \neq z$, then either $x \neq y$ or $y \neq z$, so $d(x, z) \leq d(x, y) + d(y, z)$ is satisfied. Hence, $(X, d)$ is a metric space.
    \end{proof}
    \item What is the associated topology?
    \begin{proof}
      The associated topology is the set of all possible unions of open balls in $X$. Let $S \subseteq X$. Since each singleton set is an open ball in $X$ and $S = \bigcup_{x \in S} \{x\}$, $S$ is an open set in the associated topology. Therefore, the associated topology is the discrete topology.
    \end{proof}
  \end{enumerate}
\end{homeworkProblem}

\newpage

\begin{homeworkProblem}
  True or false? If true, then give a proof and if false, then give a counterexample.
    \begin{enumerate}[(i)]
      \item Let $\mathcal{T}_1$ and $\mathcal{T}_2$ be two topologies on the same set $X$. Then we can always compare $\mathcal{T}_1$ and $\mathcal{T}_2$ (that is, either $\mathcal{T}_1$ is coarser than $\mathcal{T}_2$ or $\mathcal{T}_1$ is finer than $\mathcal{T}_2$).

      \begin{proof}
        False. Consider the set $X = \{a, b, c\}$ and topologies $T_1 = \{\emptyset, X, \{a\}\}$ and $T_2 = \{\emptyset, X, \{b\}\}$. Since $\{a\} \notin T_2$ and $\{b\} \notin T_1$, we cannot comapre $T_1$ and $T_2$.
      \end{proof}
      
      \item The topology $(X, \mathcal{T})$ associated to a metric space $(X, d)$ is never the trivial topology.

      \begin{proof}
        False. Consider $X = \{a\}$. Then the topology associated to the metric space $(X, d)$ is the trivial topology.
      \end{proof}
      
      \item If $(X, d_1)$ and $(X, d_2)$ are two metric spaces that give the same topology on $X$, then $d_1 = d_2$.

      \begin{proof}
        False. Consider $X = \{a, b\}$, $d_1(a, b) = 1$ and $d_2(a, b) = 2$. Then the topologies of both metric spaces are $\{\emptyset, X, \{a\}, \{b\}\}$.
      \end{proof}
      
      \item The set $[a, b) \subset \mathbb{R}$ is open, in the Euclidean topology, where $a < b$.
      \begin{proof}
        False, as for all $\epsilon > 0$, $(a - \epsilon, a + \epsilon) \notin [a, b)$. 
      \end{proof}
      
      \item The set $(a, \infty) \subset \mathbb{R}$ is open, in the Euclidean topology.
      \begin{proof}
        True, as for all $x \in (a, \infty)$, there exists $\epsilon = x - a > 0$ such that $(x - \epsilon, x + \epsilon) \subseteq (a, \infty)$.
      \end{proof}
      
      \item Let $X$ be any set. Then the set of all infinite subsets of $X$, union the empty set, is a topology on $X$.

      \begin{proof}
        False. Consider $X = \Z$. $\{0\} = \Z_{\leq 0} \cap \Z_{\geq 0}$ is not in the topology.
      \end{proof}
      
      \item The Euclidean topology on $\mathbb{R}$ and the topology on $\mathbb{R}$ given by the metric $d(x, y) = |x - y|$ are the same.

      \begin{proof}
        True. Since the Euclidean topology is the set of all unions of open intervals and the metric topology is the set of all unions of open balls, the two topologies are the same.
      \end{proof}
    \end{enumerate}
\end{homeworkProblem}

\newpage

\begin{homeworkProblem}
  Let $X = \mathbb{R}$ and let 
    \[
    \mathcal{T} = \{(a, \infty) \mid a \in \mathbb{R} \cup \{-\infty, \infty\}\}.
    \]
    \begin{enumerate}[(i)]
      \item Show that $\mathcal{T}$ is a topology on $\mathbb{R}$.
      \begin{proof}
        First note that $(\infty, \infty) = \emptyset \in \T$, $(-\infty, \infty) = X \in \T$. Let $S = \{(a_i, \infty)\}_{i \in I}$ be a collection of open sets in $\T$. Let $a = \inf_{i \in I} a_i$. Then $\bigcup_{U \in S} U = \bigcup_{i \in I} (a_i, \infty) = (a, \infty) \in \T$. Now suppose $S = \{(a_i, \infty)\}_{i = 1}^n$. Then $\bigcap_{i = 1}^n (a_i, \infty) = (\max_{1 \leq i \leq n} a_i, \infty) \in \T$. Hence, $\T$ is a topology on $\R$.
      \end{proof}
      \item Is this topology coarser or finer than the Euclidean topology?
      \begin{proof}
        Since for all $(a, \infty) \in \T$, $(a, \infty)$ is open in the Euclidean topology but $(-1, 1) \notin \T$, $\T$ is coarser than the Euclidean topology.
      \end{proof}
    \end{enumerate}
\end{homeworkProblem}
\end{document}