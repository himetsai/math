\documentclass{article}

\usepackage{fancyhdr}
\usepackage{extramarks}
\usepackage{amsmath}
\usepackage{amsthm}
\usepackage{amsfonts}
\usepackage{tikz}
\usepackage[plain]{algorithm}
\usepackage{algpseudocode}
\usepackage{enumerate}
\usepackage{amssymb}
\usepackage{dsfont}

\usetikzlibrary{automata,positioning}

%
% Basic Document Settings
%

\topmargin=-0.45in
\evensidemargin=0in
\oddsidemargin=0in
\textwidth=6.5in
\textheight=9.0in
\headsep=0.25in

\linespread{1.1}

\pagestyle{fancy}
\lhead{\hmwkAuthorName}
\chead{\hmwkClass:\ \hmwkTitle}
\rhead{\firstxmark}
\lfoot{\lastxmark}
\cfoot{\thepage}

\renewcommand\headrulewidth{0.4pt}
\renewcommand\footrulewidth{0.4pt}

\setlength\parindent{0pt}
\setlength{\parskip}{5pt}

%
% Create Problem Sections
%

\newcommand{\enterProblemHeader}[1]{
    \nobreak\extramarks{}{Problem \arabic{#1} continued on next page\ldots}\nobreak{}
    \nobreak\extramarks{Problem \arabic{#1} (continued)}{Problem \arabic{#1} continued on next page\ldots}\nobreak{}
}

\newcommand{\exitProblemHeader}[1]{
    \nobreak\extramarks{Problem \arabic{#1} (continued)}{Problem \arabic{#1} continued on next page\ldots}\nobreak{}
    \stepcounter{#1}
    \nobreak\extramarks{Problem \arabic{#1}}{}\nobreak{}
}

\setcounter{secnumdepth}{0}
\newcounter{partCounter}
\newcounter{homeworkProblemCounter}
\setcounter{homeworkProblemCounter}{1}
\nobreak\extramarks{Problem \arabic{homeworkProblemCounter}}{}\nobreak{}

%
% Homework Problem Environment
%
% This environment takes an optional argument. When given, it will adjust the
% problem counter. This is useful for when the problems given for your
% assignment aren't sequential. See the last 3 problems of this template for an
% example.
%
\newenvironment{homeworkProblem}[1][-1]{
    \ifnum#1>0
        \setcounter{homeworkProblemCounter}{#1}
    \fi
    \section{Problem \arabic{homeworkProblemCounter}}
    \setcounter{partCounter}{1}
    \enterProblemHeader{homeworkProblemCounter}
}{
    \exitProblemHeader{homeworkProblemCounter}
}

%
% Homework Details
%   - Title
%   - Due date
%   - Class
%   - Section/Time
%   - Instructor
%   - Author
%

\newcommand{\hmwkTitle}{Homework\ \#6}
\newcommand{\hmwkDueDate}{Feb 19, 2025}
\newcommand{\hmwkClass}{MATH 190A}
\newcommand{\hmwkClassTime}{Section A02 8:00AM - 8:50AM}
\newcommand{\hmwkSectionLeader}{Zhiyuan Jiang}
\newcommand{\hmwkClassInstructor}{Professor McKernan}
\newcommand{\hmwkSource}{Source Consulted: Textbook, Lecture, Discussion}
\newcommand{\hmwkAuthorName}{\textbf{Ray Tsai}}
\newcommand{\hmwkPID}{A16848188}

%
% Title Page
%

\title{
    \vspace{2in}
    \textmd{\textbf{\hmwkClass:\ \hmwkTitle}}\\
    \normalsize\vspace{0.1in}\small{Due\ on\ \hmwkDueDate\ at 12:00pm}\\
    \vspace{0.1in}\large{\textit{\hmwkClassInstructor}} \\
    \vspace{0.1in}\small\hmwkClassTime \\
    \small Section Leader: \hmwkSectionLeader \\
    \vspace{0.1in}\small\hmwkSource \\
    \vspace{3in}
}

\author{
  \hmwkAuthorName \\
  \vspace{0.1in}\small\hmwkPID
}
\date{}

\renewcommand{\part}[1]{\textbf{\large Part \Alph{partCounter}}\stepcounter{partCounter}\\}

%
% Various Helper Commands
%

% Useful for algorithms
\newcommand{\alg}[1]{\textsc{\bfseries \footnotesize #1}}

% For derivatives
\newcommand{\deriv}[1]{\frac{\mathrm{d}}{\mathrm{d}x} (#1)}

% For partial derivatives
\newcommand{\pderiv}[2]{\frac{\partial}{\partial #1} (#2)}

% Integral dx
\newcommand{\dx}{\mathrm{d}x}

% Probability commands: Expectation, Variance, Covariance, Bias
\newcommand{\Var}{\mathrm{Var}}
\newcommand{\Cov}{\mathrm{Cov}}
\newcommand{\Bias}{\mathrm{Bias}}
\newcommand*{\Z}{\mathbb{Z}}
\newcommand*{\Q}{\mathbb{Q}}
\newcommand*{\R}{\mathbb{R}}
\newcommand*{\C}{\mathbb{C}}
\newcommand*{\N}{\mathbb{N}}
\newcommand*{\prob}{\mathds{P}}
\newcommand*{\E}{\mathds{E}}
\newcommand*{\T}{\mathcal{T}}

\begin{document}

\maketitle

\pagebreak

\begin{homeworkProblem}
  Let $X$ be the topological space whose closed sets are the finite sets plus the whole of $X$. When is $X$ connected?

	\begin{proof}
		If $X$ is finite, then $X$ has the discrete topology, and is disconnected. Suppose $X$ is infinite. Let $Y \subset X$ be open and $Y$ is not empty or $X$. Then $X \backslash Y$ is finite, so $Y$ is infinite. But then $Y$ is not closed. Hence, $X$ is connected if and only if $X$ is infinite.
	\end{proof}
\end{homeworkProblem}

\newpage

\begin{homeworkProblem}
	Let $X$ be a topological space and let $Y$ be a connected subspace. If  
	\[
		Y \subset Z \subset \overline{Y}
	\]
	then prove that $Z$ is connected.

	\begin{proof}
		Let $A$ be the connected component of $Z$ that contains $Y$. Since $A$ is closed and contains $Y$, $Z \subseteq \overline{Y} \subseteq A \subseteq Z$. Hence, $Z$ is connected.
	\end{proof}
\end{homeworkProblem}

\newpage

\begin{homeworkProblem}
	Let $Y \subset \mathbb{R}^n$ be a subset.  
	\begin{enumerate}[(i)]
    \item We say that $Y$ is \textbf{convex} if the line between any two points $p$ and $q$ of $Y$ is contained in $Y$,
    \[
    \{tp + (1-t)q \mid t \in [0,1] \} \subset Y.
    \]
    Show that if $Y$ is convex then it is path-connected.  

		\begin{proof}
			Let $p, q \in Y$. By definition, the line between $p$ and $q$ is contained in $Y$, and thus $Y$ is path-connected.
		\end{proof}

    \item We say that $Y$ is \textbf{star-shaped} about $y_0$ if for any point $y \in Y$ the line connecting $y_0$ to $y$ is contained in $Y$.  
    Show that if $Y$ is star-shaped then it is path-connected.
		
		\begin{proof}
			Let $p, q \in Y$. There is a path from $p$ to $y_0$ and a path from $y_0$ to $q$. By connecting the two paths, we have a path from $p$ to $q$. Thus, $Y$ is path-connected.
		\end{proof}
	\end{enumerate}
\end{homeworkProblem}

\newpage

\begin{homeworkProblem}
	True or false? If true then give a proof and if false then give a counterexample.  
	\begin{enumerate}[(i)]
    \item The set  
    \[
    \mathbb{Q}^2 = \{ (a,b) \mid a, b \in \mathbb{Q} \}
    \]
    is connected.

		\begin{proof}
			False. Let $A = \{a \in \Q, a > \sqrt{2}\}$ and $B = \{b \in \Q, b < \sqrt{2}\}$. Then $(A \times \Q) \cap (B \times \Q) = \emptyset$ and $\Q^2 = (A \times \Q) \cup (B \times \Q)$. But then $A = (\sqrt{2}, \infty) \cap \Q$  and $B = (-\infty, \sqrt{2}) \cap \Q$ are open. Thus, $\Q^2$ is disconnected.
		\end{proof}

    \item The set  
    \[
    \mathbb{Q}^2 = \{ (a,b) \mid a, b \in \mathbb{Q} \}
    \]
    is path-connected.

		\begin{proof}
			False. Since $$\mathbb{Q}^2$$ is disconnected, it cannot be path-connected.
		\end{proof}

    \item The set  
    \[
    \mathbb{R}^2 \setminus \mathbb{Q}^2
    \]
    is path-connected.

		\begin{proof}
			True. For $a \in \R \backslash \Q$, the lines $\{(a, r) \mid r \in \R\}$ and $\{(r, a) \mid r \in \R\}$ are in $\R^2 \backslash \Q^2$. Let $p, q \in \mathbb{R}^2 \setminus \mathbb{Q}^2$. Assume that $p_1, q_1 \notin \Q$. Then for some $k \in (\R \backslash \Q) \cap (p_2, q_2)$, the path 
			\[
				\{(p_1, r) \mid p_2 \leq r \leq k\} \cup \{(r, k) \mid p_1 \leq r \leq q_1\} \cup \{(q_1, r) \mid k \leq r \leq q_2\}
			\]
			is a path from $p$ to $q$ and is contained in $\R^2 \backslash \Q^2$. We may also find  
			a path from $p$ to $q$ with a similar appraoch if $p_1, q_2 \notin \Q$. The result now follows.
		\end{proof}

    \item The set  
    \[
    \mathbb{R}^2 \setminus \mathbb{Q}^2
    \]
    is connected.

		\begin{proof}
			True. Since $\mathbb{R}^2 \setminus \mathbb{Q}^2$ is path-connected, it is also connected.
		\end{proof}

    \item If $f: X \to Y$ is continuous and surjective and $X$ is path-connected then $Y$ is path-connected.
    
		\begin{proof}
		True. Let $y_1, y_2 \in Y$. Since $f: X \to Y$ is surjective, there exist points $x_1, x_2 \in X$ such that $f(x_1)=y_1$ and $f(x_2)=y_2$. As $X$ is path-connected, there is a continuous path $\gamma : [0,1] \to X$ with $\gamma(0)=x_1$ and $\gamma(1)=x_2$. Then the composition $f\circ\gamma : [0,1] \to Y$ is continuous, with $(f\circ\gamma)(0)=y_1$ and $(f\circ\gamma)(1)=y_2$. Hence, $Y$ is path-connected.
		\end{proof}

    \item If $X$ and $Y$ are path-connected topological spaces then $X \times Y$ is path-connected.
    
		\begin{proof}
		True. Let $(x_1,y_1)$ and $(x_2,y_2)$ be any two points in $X \times Y$. Since $X$ is path-connected, there exists a continuous path $\gamma : [0,1] \to X$ such that $\gamma(0)=x_1$ and $\gamma(1)=x_2$. Similarly, there exists a continuous path $\sigma : [0,1] \to Y$ with $\sigma(0)=y_1$ and $\sigma(1)=y_2$. 

		Define a path $\eta : [0,1] \to X \times Y$ by
		\[
		\eta(t) =
		\begin{cases}
		\bigl(\gamma(2t),\, y_1\bigr) & \text{if } 0 \le t \le \tfrac{1}{2},\\[1mm]
		\bigl(x_2,\, \sigma(2t-1)\bigr) & \text{if } \tfrac{1}{2} \le t \le 1.
		\end{cases}
		\]
		Then $\eta(0)=(x_1,y_1)$ and $\eta(1)=(x_2,y_2)$. Since the concatenation of continuous functions is continuous, $\eta$ is a continuous path connecting $(x_1,y_1)$ to $(x_2,y_2)$. Therefore, $X \times Y$ is path-connected.
		\end{proof}

    \item The path components of a topological space are closed.
    
		\begin{proof}
			False. Let $B = \{(x, \sin(1 /x)) \in x \in (0, 1]\}$ and $A = \{(0, y) \in y \in [-1, 1]\}$. Consider $X = A \cup B \subset \R^2$. We know $X = \overline{B}$, so $B$ is not closed. But then $B$ is a path component of $X$. 
		\end{proof}

    \item The connected components of a topological space are open.
    \begin{proof}
			True. let $C$ be a connected component of $X$. Let $x \in C$. For any connected subsets $U$ containing $x$, we have $U \subseteq C$. Let $\{U_\alpha\}$ be the collection of connected subsets in $C$. Then $C = \bigcup_\alpha U_\alpha$. But then $C$ is a union of open sets. 
		\end{proof}
	\end{enumerate}
\end{homeworkProblem}

\newpage

\begin{homeworkProblem}
	Let $X$ be a connected topological space. We say that a point $x$ is a \textbf{cut point} if $X - \{ x \}$ is disconnected.  
	\begin{enumerate}[(i)]
    \item Let $Y \subset \mathbb{R}^2$ be the union of two closed disks that intersect at one point (so that the boundary circles are tangent). Identify the cut points.
    
		\begin{proof}
		Let $D_1$ and $D_2$ be the two disks. We may assume that $D_1 = \overline{B}_1(1, 0)$ and $D_2 = \overline{B}_1(-1, 0)$. Let $x$ be the point of tangency, i.e $x = (0, 0)$. If $x$ is removed, then $D_1$ and $D_2$ are disjoint. But then $D_1 \backslash \{0\} = Y \cap ((-2, 0) \times \R)$ and $D_2 \backslash \{0\} = Y \cap ((0, 1) \times \R)$ are open sets. Hence, $x$ is a cut point of $Y$.
		\end{proof}

    \item If $f: X \to Y$ is a homeomorphism then show that $x \in X$ is a cut point if and only if $y = f(x)$ is a cut point.
    
		\begin{proof}
			Suppose $x \in X$ is a cut point. Then $X \backslash \{x\}$ is disconnected. Let $U$ and $V$ be disjoint open sets such that $X - \{x\} = U \cup V$. Then $f(U), f(V)$ are disjoint open sets in $Y$ such that $f(X - \{x\}) = Y - \{y\} = f(U) \cup f(V)$. But then $Y = f(X)$ is connected, and so $y = f(x)$ is a cut point.
		\end{proof}

    \item Show that $[0,1]$ and $(0,1)$ are not homeomorphic.
		\begin{proof}
			Note that every point in $(0, 1)$ is a cut point. But $0$ and $1$ are not cut points in $[0, 1]$. Since homeomorphisms perserve cut points, $[0, 1]$ and $(0, 1)$ are not homeomorphic.
		\end{proof}

    \item Give a complete list of all intervals in $\mathbb{R}$, up to homeomorphism.
    
		\begin{proof}
			$(0, 1), [0, 1), [0, 1]$. 
		\end{proof}

    \item Show that if $\mathbb{R}$ is homeomorphic to $\mathbb{R}^n$ then $n = 1$.
    \begin{proof}
			Note that every point in $(0, 1)$ is a cut point. But $\mathbb{R}^n$ have no cut points for $n \geq 2$. Since homeomorphisms perserve cut points, $R$ is homeomorphic to $\mathbb{R}^n$ only if $n = 1$.
		\end{proof}
	\end{enumerate}
\end{homeworkProblem}
\end{document}