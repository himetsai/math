\documentclass{article}

\usepackage{fancyhdr}
\usepackage{extramarks}
\usepackage{amsmath}
\usepackage{amsthm}
\usepackage{amsfonts}
\usepackage{tikz}
\usepackage[plain]{algorithm}
\usepackage{algpseudocode}
\usepackage{enumerate}
\usepackage{amssymb}
\usepackage{dsfont}

\usetikzlibrary{automata,positioning}

%
% Basic Document Settings
%

\topmargin=-0.45in
\evensidemargin=0in
\oddsidemargin=0in
\textwidth=6.5in
\textheight=9.0in
\headsep=0.25in

\linespread{1.1}

\pagestyle{fancy}
\lhead{\hmwkAuthorName}
\chead{\hmwkClass:\ \hmwkTitle}
\rhead{\firstxmark}
\lfoot{\lastxmark}
\cfoot{\thepage}

\renewcommand\headrulewidth{0.4pt}
\renewcommand\footrulewidth{0.4pt}

\setlength\parindent{0pt}
\setlength{\parskip}{5pt}

%
% Create Problem Sections
%

\newcommand{\enterProblemHeader}[1]{
    \nobreak\extramarks{}{Problem \arabic{#1} continued on next page\ldots}\nobreak{}
    \nobreak\extramarks{Problem \arabic{#1} (continued)}{Problem \arabic{#1} continued on next page\ldots}\nobreak{}
}

\newcommand{\exitProblemHeader}[1]{
    \nobreak\extramarks{Problem \arabic{#1} (continued)}{Problem \arabic{#1} continued on next page\ldots}\nobreak{}
    \stepcounter{#1}
    \nobreak\extramarks{Problem \arabic{#1}}{}\nobreak{}
}

\setcounter{secnumdepth}{0}
\newcounter{partCounter}
\newcounter{homeworkProblemCounter}
\setcounter{homeworkProblemCounter}{1}
\nobreak\extramarks{Problem \arabic{homeworkProblemCounter}}{}\nobreak{}

%
% Homework Problem Environment
%
% This environment takes an optional argument. When given, it will adjust the
% problem counter. This is useful for when the problems given for your
% assignment aren't sequential. See the last 3 problems of this template for an
% example.
%
\newenvironment{homeworkProblem}[1][-1]{
    \ifnum#1>0
        \setcounter{homeworkProblemCounter}{#1}
    \fi
    \section{Problem \arabic{homeworkProblemCounter}}
    \setcounter{partCounter}{1}
    \enterProblemHeader{homeworkProblemCounter}
}{
    \exitProblemHeader{homeworkProblemCounter}
}

%
% Homework Details
%   - Title
%   - Due date
%   - Class
%   - Section/Time
%   - Instructor
%   - Author
%

\newcommand{\hmwkTitle}{Homework\ \#3}
\newcommand{\hmwkDueDate}{Jan 29, 2025}
\newcommand{\hmwkClass}{MATH 190A}
\newcommand{\hmwkClassTime}{Section A02 8:00AM - 8:50AM}
\newcommand{\hmwkSectionLeader}{Zhiyuan Jiang}
\newcommand{\hmwkClassInstructor}{Professor McKernan}
\newcommand{\hmwkSource}{Source Consulted: Textbook, Lecture, Discussion}
\newcommand{\hmwkAuthorName}{\textbf{Ray Tsai}}
\newcommand{\hmwkPID}{A16848188}

%
% Title Page
%

\title{
    \vspace{2in}
    \textmd{\textbf{\hmwkClass:\ \hmwkTitle}}\\
    \normalsize\vspace{0.1in}\small{Due\ on\ \hmwkDueDate\ at 12:00pm}\\
    \vspace{0.1in}\large{\textit{\hmwkClassInstructor}} \\
    \vspace{0.1in}\small\hmwkClassTime \\
    \small Section Leader: \hmwkSectionLeader \\
    \vspace{0.1in}\small\hmwkSource \\
    \vspace{3in}
}

\author{
  \hmwkAuthorName \\
  \vspace{0.1in}\small\hmwkPID
}
\date{}

\renewcommand{\part}[1]{\textbf{\large Part \Alph{partCounter}}\stepcounter{partCounter}\\}

%
% Various Helper Commands
%

% Useful for algorithms
\newcommand{\alg}[1]{\textsc{\bfseries \footnotesize #1}}

% For derivatives
\newcommand{\deriv}[1]{\frac{\mathrm{d}}{\mathrm{d}x} (#1)}

% For partial derivatives
\newcommand{\pderiv}[2]{\frac{\partial}{\partial #1} (#2)}

% Integral dx
\newcommand{\dx}{\mathrm{d}x}

% Probability commands: Expectation, Variance, Covariance, Bias
\newcommand{\Var}{\mathrm{Var}}
\newcommand{\Cov}{\mathrm{Cov}}
\newcommand{\Bias}{\mathrm{Bias}}
\newcommand*{\Z}{\mathbb{Z}}
\newcommand*{\Q}{\mathbb{Q}}
\newcommand*{\R}{\mathbb{R}}
\newcommand*{\C}{\mathbb{C}}
\newcommand*{\N}{\mathbb{N}}
\newcommand*{\prob}{\mathds{P}}
\newcommand*{\E}{\mathds{E}}
\newcommand*{\T}{\mathcal{T}}

\begin{document}

\maketitle

\pagebreak

\begin{homeworkProblem}
	Let $(X, \mathcal{T})$ be a topological space. If $A \subset X$ is any subset then we say that $x \in X$ is an \textbf{accumulation point} if the closure of $A \setminus \{ x \}$ contains $x$. Show that the closure of $A$ is the union of $A$ and all of its accumulation points.

	\begin{proof}
		It suffices to show that the set of accumulation points of $A$ that are not in $A$ equals $\overline{A} \backslash A$. If $x$ is an accumulation point of $A$ and $x \notin A$, then $x$ contained in the closure of $A \backslash \{x\} = A$. Now suppose $x \in \overline{A} \backslash A$. Then $x$ is in the closure of $A \backslash \{x\} = A$, so $x$ is an accumulation point of $A$.
	\end{proof}
\end{homeworkProblem}

\newpage

\begin{homeworkProblem}
	Let $(X, \mathcal{T})$ be a topological space with basis $\mathcal{B}$ and let $(Y, \mathcal{S})$ be a topological space with basis $\mathcal{C}$.  
	Show that
	\[
		\mathcal{D} = \{ B \times C \mid B \in \mathcal{B}, C \in \mathcal{C} \}
	\]
	is a basis for the product topology on $X \times Y$.

	\begin{proof}
		Note that
		\[
			X \times Y = \bigcup_{B \in \mathcal{B}} B \times \bigcup_{C \in \mathcal{C}} C = \bigcup_{B \in \mathcal{B}, C \in \mathcal{C}} B \times C = \bigcup_{D \in \mathcal{D}} D,
		\]
		so $\mathcal{D}$ covers $X \times Y$. 

		Suppose $D_1, D_2 \in \mathcal{D}$. Then $D_1 = B_1 \times C_1$ and $D_2 = B_2 \times C_2$, and thus
		\[
			D_1 \cap D_2 = (B_1 \cap B_2) \times (C_1 \cap C_2) \in \mathcal{D}.
		\]
	\end{proof}
\end{homeworkProblem}

\newpage

\begin{homeworkProblem}
	Let $(X, \mathcal{T})$ be a topological space. We say that $(X, \mathcal{T})$ is \textbf{Hausdorff} if for any two points $x \neq y \in X$ we may find two disjoint neighborhoods $F$ and $G$ of $x$ and $y$. Show that the following are equivalent:
	\begin{enumerate}[(i)]
		\item $(X, \mathcal{T})$ is Hausdorff.
		\item For any two points $x \neq y$ we can find two disjoint open subsets $U$ and $V$ such that $x \in U$ and $y \in V$.
		\item For any two points $x \neq y$ we can find a closed neighborhood $A$ of $x$ not containing $y$ (that is, $y \notin A$).
		\item The diagonal
		\[
		\Delta = \{ (x, x) \mid X \times X \}
		\]
		is closed in the product topology.
	\end{enumerate}

	\begin{proof}
		(i) to (ii): If $F$ and $G$ are disjoint neighborhoods of $x$ and $y$, then $\text{int}(F)$, $\text{int}(G)$ are disjoint open sets containing $x$ and $y$.

		(ii) to (iii): If $U$ and $V$ are disjoint open sets such that $x \in U$ and $y \in V$, then $A = V^c$ is a closed neighborhood of $x$ not containing $y$.

		(iii) to (iv): Suppose $x, y \in X$ such that $x \neq y$. Then there exists a closed neighborhood $A$ of $x$ that does not contain $y$. But then $\text{int}(A) \times A^c$ is an open neighborhood of $(x, y)$ that does not intersect with $\Delta$. Hence, $\Delta^c$ is open.

		(iv) to (i): Since $\Delta^c$ is open, for each $x \neq y$ there exists an open set $U \times V \subseteq \Delta^c$ containing $(x, y)$, where $U, V \subseteq X$. For $(a, b) \in U \times V$, since $U \times V \cap \Delta = \emptyset$, $a \neq b$. Thus, $U$ and $V$ are disjoint neighborhoods of $x$ and $y$.
	\end{proof}
\end{homeworkProblem}

\newpage

\begin{homeworkProblem}
	True or false? If true then give a proof and if false then give a counterexample.
	\begin{enumerate}[(i)]
		\item If $(X, \mathcal{T})$ is a topological space and $Y \subset X$ is a subset and $U \subset Y$ is open in the subspace topology then $U$ is open in $X$.
		\begin{proof}
			False. Consider $X = \R$, and $\T$ is the Eclidean topology. If $Y = [0, 1]$, then $U = (0, 1]$ is open in $Y$ but not in $X$.
		\end{proof}
		\item If $(X, \mathcal{T})$ is a Hausdorff topological space then every singleton subset $\{ x \}$ is closed.
		\begin{proof}
			True. Let $x \in X$. Then for any $y \in X$ with $y \neq x$, there exist closed neighborhood $U_y$ of $x$ that does not contain $y$. But then
			\[
				\bigcup_{y \in X, x \neq y} U_y^c  = X \backslash \{ x \}
			\]
			is open.
		\end{proof}
		\item If $(X, \mathcal{T})$ is a topological space and every singleton subset is closed then $(X, \mathcal{T})$ is Hausdorff.
		\begin{proof}
			False. Consider the topology given in homework 2 problem 1 and let $X$ be infinite. Every singleton is closed, but it is not Hausdorff, as any two non-empty open sets intersect.
		\end{proof}
		\item If $(X, \mathcal{T})$ and $(Y, \mathcal{S})$ are Hausdorff topological spaces then the product $(X \times Y, \mathcal{R})$ is Hausdorff.
		\begin{proof}
			True. For distinct points $(x_1, y_y), (x_2, y_2)$, there exists $U_1, U_2, V_1, V_2$ such that $x_1 \in U_1$, $x_2 \in U_2$, $y_1 \in V_1$, $y_2 \in V_2$ and $U_1 \cap U_2 = \emptyset$, $V_1 \cap V_2 = \emptyset$. Then $U_1 \times V_1$ and $U_2 \times V_2$ are disjoint neighborhoods of $(x_1, y_1)$ and $(x_2, y_2)$.
		\end{proof}
		\item If $(X, \mathcal{T})$ and $(Y, \mathcal{S})$ are two topological spaces and $A \subset X, B \subset Y$ then
		\[
		\overline{A \times B} = \overline{A} \times \overline{B}
		\]
		in the product topology on $X \times Y$.
		\begin{proof}
			True. Let $(x, y) \in \overline{A \times B}$. Then any open neighborhoods $U \times V$ of $(x, y)$ intersects with $A \times B$. This implies any open neighborhoods $U$ of $x$ intersects with $A$ and any open neighborhood $V$ of $y$ intersects with $B$. Thus, $x \in \overline{A}$ and $y \in \overline{B}$.

			On the other hand, Let $(x, y) \in \overline{A} \times \overline{B}$. Then any open neighborhoods $U$ of $x$ intersects with $A$ and any open neighborhood $V$ of $y$ intersects with $B$. But then any open neighborhoods $U \times V$ of $(x, y)$ intersects with $A \times B$.
		\end{proof}
		\item Every subspace of a Hausdorff topological space is Hausdorff.
		\begin{proof}
			True. Let $x, y$ be distinct points in the subspace $Y$ of $X$. Then there exist disjoint neighborhoods $U, V$ of $x$ and $y$ in $X$. But then $U \cap Y$ and $V \cap Y$ are disjoint neighborhoods of $x$ and $y$ in $Y$.
		\end{proof}
	\end{enumerate}
\end{homeworkProblem}

\newpage

\begin{homeworkProblem}
	If $(X, d)$ is a metric space then the induced topological space $(X, \mathcal{T})$ is Hausdorff.

	\begin{proof}
		Let $x, y \in X$ such that $x \neq y$, and let $r = d(x, y)/2$. Then the open balls $B(x, r)$ and $B(y, r)$ are disjoint neighborhoods of $x$ and $y$.
	\end{proof}
\end{homeworkProblem}

\end{document}