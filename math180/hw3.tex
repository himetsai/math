\documentclass[addpoints, 11pt]{exam}
\setlength{\headsep}{0.25in}
\setlength{\unitlength}{1in}
%
\pagestyle{head}
%
\usepackage[utf8]{inputenc} %use Unicode
\usepackage[T1]{fontenc} %European fonts

\usepackage{%
	amsmath,       %some math tools
	amssymb,       %math symbols
	graphicx,      %enhanced graphics options
	mathtools,     %extension of amsmath
	microtype,     %small typographic effects
	bm,            %bold math symbols
	% todonotes,   %adds the option \todo{...} (use fixme instead)
	stmaryrd,      %some more math symbolswork
	% nicematrix,    %nicer matrix controls
	mathrsfs,      %more math fonts
        dsfont,
}
\usepackage{url}
\usepackage{hyperref}
%
\usepackage{amsthm}
% \newtheorem*{thm11.1.7}{Theorem 11.1.7}
%
\usepackage[shortlabels]{enumitem}
\usepackage{nicematrix}
\usepackage{multicol}
%
\usepackage[normalem]{ulem}
%
\newcommand{\myCourseNumber}{Math 180A}
\newcommand{\myName}{Ray Tsai}
\newcommand{\myID}{A16848188}
\newcommand{\myProfessor}{Professor Carfagnini}
\newcommand{\myHmwkNumber}{3}
% \newcommand{\myExamVersion}{}

\newcommand*{\Z}{\mathbb{Z}}
\newcommand*{\Q}{\mathbb{Q}}
\newcommand*{\R}{\mathbb{R}}
\newcommand*{\C}{\mathbb{C}}
\newcommand*{\N}{\mathbb{N}}
\newcommand*{\prob}{\mathds{P}}

%% define absolute value \abs{...}:
\DeclarePairedDelimiterX\abs[1]\lvert\rvert{%
  \ifblank{#1}{\:\cdot\:}{#1}
}
% define norm \norm{...}:
\DeclarePairedDelimiterX\norm[1]\lVert\rVert{%
  \ifblank{#1}{\:\cdot\:}{#1}
}
% define inner product \inner{...}{...}:
\DeclarePairedDelimiterX{\inner}[2]{\langle}{\rangle}{%
  \ifblank{#1}{\:\cdot\:}{#1},\ifblank{#2}{\:\cdot\:}{#2}
}

% define \set{...} to write sets and \given to write \set{... \given ...} for {...|...}
\newcommand*\setSymbol[1][]{
  \nonscript\:#1\vert\allowbreak\nonscript\:\mathopen{}
}
\providecommand\given{}
\DeclarePairedDelimiterX\set[1]{\lbrace}{\rbrace}{
  \renewcommand*\given{\setSymbol[\delimsize]}
  #1
}

% free group geneated by ... \free{...} or \free{... \given ...}
\DeclarePairedDelimiterX\free[1]{\langle}{\rangle}{
  \renewcommand\given{\nonscript\:\delimsize\vert\nonscript\:
    \mathopen{}}
  #1}

% define \lopen{...}{...}, \ropen{...}{...}, \open{...}{...}, \closed{...}{...} for intervals
\DeclarePairedDelimiterX\open[2](){#1,#2}
\DeclarePairedDelimiterX\lopen[2](]{#1,#2}
\DeclarePairedDelimiterX\ropen[2][){#1,#2}
\DeclarePairedDelimiterX\closed[2][]{#1,#2}

\NiceMatrixOptions{cell-space-limits = 1pt}
\newcommand*{\pmat}[1]{\begin{pNiceMatrix} #1 \end{pNiceMatrix}}
\newcommand*{\dfdx}[2]{\frac{\partial #1}{\partial #2}}

\DeclareMathOperator{\vol}{vol}
%
\pointsinmargin
\pointpoints{\thinspace point}{points}
\marginpointname{ \points}
%
\begin{document}
%
\firstpageheader{\bfseries \myCourseNumber}{\bfseries Homework \myHmwkNumber}{\bfseries \myName \\ \myID \\ \myProfessor}
%
\runningheader{}{(page \textit{\thepage}\ of \textit{\numpages})}{}
%
%

\begin{description}
    \item[Question 1]  Let $c > 0$ and $X \sim \text{Unif}[0, c]$. Show that the random variable $Y = c - X$ has the same cumulative distribution function as $X$ and hence also the same density function.
    \begin{proof}[Solution]
    Let $F$ be the cumulative distribution function. For $x \in [0, c]$, 
        \begin{align}
            F_X(x) 
            &= \prob(X \leq x) = \frac{x}{c}, \\
            F_Y(x) 
            &= \prob(Y \leq x) \\
            &= \prob(X \geq c - x) = \frac{x}{c}.
        \end{align}
        Thus, $X$ and $Y$ have the same cumulative distribution.

        Since the density function $p_X(x) = \frac{d}{dx}F_X(x) = \frac{d}{dx}F_Y(x) = p_Y(x) = \frac{1}{c}$, the density functions of $X,Y$ are the same.
    \end{proof}
    

    \newpage

    \item[Question 2]  Parts (a) and (b) ask for an example of a random variable $X$ whose cumulative distribution function $F(x)$ satisfies $F(1) = \frac{1}{3}$, $F(2) = \frac{3}{4}$, and $F(3) = 1$.

    \begin{enumerate}[(a)]
        \item Make $X$ discrete and give its probability mass function.

        \begin{proof}[Solution]
            An urn contains 12 balls with 4 of them numbered 1, 5 of them numbered 2, and 3 of them numbered 3. Let $X$ be the number of the selected ball. The mass function \[
                p(x) = 
                \begin{cases}
                    \frac{1}{3}, & x = 1 \\
                    \frac{5}{12}, & x = 2 \\
                    \frac{1}{4}, & x = 3
                \end{cases}.
            \]
            Since $F(1) = p(1) = \frac{1}{3}$, $F(2) = p(1) + p(2) = \frac{1}{3} + \frac{5}{12} = \frac{3}{4}$, and $F(3) = 1$, $X$ satisfies the given condition.
        \end{proof}

        \item Make $X$ continuous and give its probability density function.

        \begin{proof}[Solution]
            Let $a = 4-\sqrt{13}$. Define distribution
            \[
                F(x) = \begin{cases}
            0, & x < a \\
            -\frac{1}{12}x^2 + \frac{2}{3}x - \frac{1}{4}, & x \in [a, 3] \\
            1, & x > 3.
        \end{cases}
            \]
            Let $X \sim F$. Since $F(1) = \frac{1}{3}$, $F(2) = \frac{3}{4}$, and $F(3) = 1$, $X$ satisfies the given condition. The density function of $X$ is $p(x) = F'(x) = \begin{cases}
                -\frac{1}{6} + \frac{2}{3}, & x \in [a, 3] \\
                0, & x \notin [a, 3]
            \end{cases}$. 
        \end{proof}
    \end{enumerate}

    \newpage

    \item[Question 3] Let $A$ and $B$ be two disjoint events. Under what conditions are they independent?

    \begin{proof}[Solution]
        Suppose $A,B$ are independent. Then $\prob(AB) = \prob(A)\prob(B) = 0$. Thus, $A,B$ are independent if $\prob(A)$ or $\prob(B)$ is 0.
    \end{proof}

    \newpage

    \item[Question 4] Suppose that the events $A, B$, and $C$ are mutually independent with
    \[
        \prob(A) = \frac{1}{2}, \quad \prob(B) = \frac{1}{3}, \quad \prob(C) = \frac{1}{4}.
    \]
    Compute $\prob(AB \cup C)$.

    \begin{proof}[Solution]
        \begin{align}
            \prob(AB \cup C)
            &= \prob(AB) + \prob(C) - \prob(ABC) \\
            &= \frac{1}{2} \cdot \frac{1}{3} + \frac{1}{4} - \frac{1}{2} \cdot \frac{1}{3} \cdot \frac{1}{4} \\
            &= \frac{3}{8}.
        \end{align}
    \end{proof}

    \newpage

    \item[Question 5]  Let $n$ be a fixed integer and $c \in R$. Let us consider the function
    \[
        p(x) = \begin{cases}
            cx, & x \in [n] \\
            0, & \text{otherwise}
        \end{cases}.
    \]
    Find the value of $c$ so that $p(x)$ is the probability mass function of a random variable $X$. 

    \begin{proof}[Solution]
    \begin{gather}
        \sum_{x \in [n]} p(x) = c\sum_{x \in [n]} x = \frac{cn(n+1)}{2} = 1, \\
        c = \frac{2}{n^2 + n}.
    \end{gather}
    \end{proof}

    \newpage

    \item[Question 6]  Let us consider the function
    \[
        f(x) = \begin{cases}
            cxe^x, & x \in (0,3) \\
            0, & \text{otherwise}
        \end{cases},
    \]
    for some $c \in \R$. Is it possible to find a value for $c$ so that $f(x)$ is the probability density function of a random variable $X$?

    \begin{proof}[Solution]
    Suppose that $\int^{\infty}_{-\infty} f(x) dx = c\int^3_0 xe^x dx = c[(x-1)e^x]^3_0 = c(2e^3 + 1) \leq 1$, $f(x)$ can be a probability density function for $0 \leq c \leq \frac{1}{2e^3 + 1}$.
    \end{proof}

    \newpage

    \item[Question 7]  A fair coin is flipped twice. Let $X$ be the number of heads observed.

    \begin{enumerate}[(a)]
        \item Give the possible values and probability mass function for $X$.

        \begin{proof}[Solution]
            The possible values are $0,1,2$. The probability mass function $p(x) = \frac{{2 \choose x}}{4}$, for $x \in \{0, 1, 2\}$.
        \end{proof}

        \item Find $\prob(X \geq 1)$ and $\prob(X > 1)$.

        \begin{proof}[Solution]
            \begin{gather}
                \prob(X \geq 1) = p(1) + p(2) = \frac{3}{4} \\
                \prob(X > 1) = p(2) = \frac{1}{4}.
            \end{gather}
        \end{proof}
    \end{enumerate}

    \newpage

    \item[Question 8] Suppose that $X$ is a discrete random variable with possible values $\N = \{1, 2, 3, \dots\}$, and probability mass function
    \[
        p_X(k) = \frac{c}{k(k+1)},
    \]
    for some constant $c > 0$. What is the value of $c$?

    \begin{proof}[Solution]
        \begin{align}
            \sum_{k \in \N} p_X(k) 
            &= c\sum_{k \in \N} \frac{1}{k(k+1)} \\
            &= c\sum_{k \in \N} \frac{1}{k} - \frac{1}{k + 1} \\
            &= c \cdot \left(1 - \frac{1}{\infty}\right) = 1.
        \end{align}
        Therefore, $c = 1$.
    \end{proof}

    \newpage

    \item[Question 9] We choose a number from the set $\{10, 11, 12, . . . , 99\}$ uniformly at random.

    \begin{enumerate}[(a)]
        \item Let $X$ be the first digit and $Y$ be the second digit of the chosen number. Show that $X$ and $Y$ are independent random variables.

        \begin{proof}[Solution]
            The possible values for $X$ are $1, \dots, 9$, $Y$ are $0, 1, \dots, 9$. Let $d_1$ be a possible value of $X$ and $d_2$ be a possible value of $Y$. There are 9 numbers that start with $d_1$ and 10 numbers that end with $d_2$, and there is only one number that starts with $d_1$ and ends with $d_2$. Since the number is chosen uniformly at random, $\prob(X = d_1) = \frac{1}{9}$ and $\prob(Y = d_2) = \frac{1}{10}$. Since \[
                \prob(X = d_1 \text{ and } Y = d_2) = \frac{1}{90} = \prob(X = d_1)\prob(Y = d_2),
            \]
            $X$, $Y$ are independent variables.
        \end{proof}

        \item  Let $X$ be the first digit of the chosen number and $Z$ be the sum of the two digits. Show that $X$ and $Z$ are not independent.

        \begin{proof}[Solution]
            The possible values for $X$ are $1, \dots, 9$, and $Z$ are $1, \dots, 18$. Since it's impossible for the sum of the digits to be 18 when the first digit is 1, $\prob(X = 1 \text{ and } Z = 18) = 0$. Since 99 is the only number whose sum of digits is 18, $\prob(Z = 18) = \frac{1}{90}$.
            Since 
            \begin{gather}
                \prob(X = 1)\prob(Z = 18) = \frac{1}{9} \cdot \frac{1}{90}, \\
                \prob(X = 1 \text{ and } Z = 18) = 0,
            \end{gather}
            $X, Z$ are not independent.
            
        \end{proof}
    \end{enumerate}
    
\end{description}

\end{document}
