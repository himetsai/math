\documentclass[addpoints, 11pt]{exam}
\setlength{\headsep}{0.25in}
\setlength{\unitlength}{1in}
%
\pagestyle{head}
%
\usepackage[utf8]{inputenc} %use Unicode
\usepackage[T1]{fontenc} %European fonts

\usepackage{%
	amsmath,       %some math tools
	amssymb,       %math symbols
	graphicx,      %enhanced graphics options
	mathtools,     %extension of amsmath
	microtype,     %small typographic effects
	bm,            %bold math symbols
	% todonotes,   %adds the option \todo{...} (use fixme instead)
	stmaryrd,      %some more math symbolswork
	% nicematrix,    %nicer matrix controls
	mathrsfs,      %more math fonts
        dsfont,
}
\usepackage{url}
\usepackage{hyperref}
%
\usepackage{amsthm}
% \newtheorem*{thm11.1.7}{Theorem 11.1.7}
%
\usepackage[shortlabels]{enumitem}
\usepackage{nicematrix}
\usepackage{multicol}
%
\usepackage[normalem]{ulem}
%
\newcommand{\myCourseNumber}{Math 180A}
\newcommand{\myName}{Ray Tsai}
\newcommand{\myID}{A16848188}
\newcommand{\myProfessor}{Professor Carfagnini}
\newcommand{\myHmwkNumber}{4}
% \newcommand{\myExamVersion}{}

\newcommand*{\Z}{\mathbb{Z}}
\newcommand*{\Q}{\mathbb{Q}}
\newcommand*{\R}{\mathbb{R}}
\newcommand*{\C}{\mathbb{C}}
\newcommand*{\N}{\mathbb{N}}
\newcommand*{\p}{\mathds{P}}
\newcommand*{\E}{\mathds{E}}

%% define absolute value \abs{...}:
\DeclarePairedDelimiterX\abs[1]\lvert\rvert{%
  \ifblank{#1}{\:\cdot\:}{#1}
}
% define norm \norm{...}:
\DeclarePairedDelimiterX\norm[1]\lVert\rVert{%
  \ifblank{#1}{\:\cdot\:}{#1}
}
% define inner product \inner{...}{...}:
\DeclarePairedDelimiterX{\inner}[2]{\langle}{\rangle}{%
  \ifblank{#1}{\:\cdot\:}{#1},\ifblank{#2}{\:\cdot\:}{#2}
}

% define \set{...} to write sets and \given to write \set{... \given ...} for {...|...}
\newcommand*\setSymbol[1][]{
  \nonscript\:#1\vert\allowbreak\nonscript\:\mathopen{}
}
\providecommand\given{}
\DeclarePairedDelimiterX\set[1]{\lbrace}{\rbrace}{
  \renewcommand*\given{\setSymbol[\delimsize]}
  #1
}

% free group geneated by ... \free{...} or \free{... \given ...}
\DeclarePairedDelimiterX\free[1]{\langle}{\rangle}{
  \renewcommand\given{\nonscript\:\delimsize\vert\nonscript\:
    \mathopen{}}
  #1}

% define \lopen{...}{...}, \ropen{...}{...}, \open{...}{...}, \closed{...}{...} for intervals
\DeclarePairedDelimiterX\open[2](){#1,#2}
\DeclarePairedDelimiterX\lopen[2](]{#1,#2}
\DeclarePairedDelimiterX\ropen[2][){#1,#2}
\DeclarePairedDelimiterX\closed[2][]{#1,#2}

\NiceMatrixOptions{cell-space-limits = 1pt}
\newcommand*{\pmat}[1]{\begin{pNiceMatrix} #1 \end{pNiceMatrix}}
\newcommand*{\dfdx}[2]{\frac{\partial #1}{\partial #2}}

\DeclareMathOperator{\vol}{vol}
%
\pointsinmargin
\pointpoints{\thinspace point}{points}
\marginpointname{ \points}
%
\begin{document}
%
\firstpageheader{\bfseries \myCourseNumber}{\bfseries Homework \myHmwkNumber}{\bfseries \myName \\ \myID \\ \myProfessor}
%
\runningheader{}{(page \textit{\thepage}\ of \textit{\numpages})}{}
%
%

\begin{description}
    \item[Question 1] Let X be a discrete random variable with possible values $\{1, 2, 3, 4, 5\}$ and probability mass function
        \begin{center}
        \begin{tabular}{c c c c c c} 
         $x$ & 1 & 2 & 3 & 4 & 5 \\ 
         $p(x)$ & $\frac{1}{7}$ & $\frac{1}{14}$ & $\frac{3}{14}$ & $\frac{2}{7}$ & $\frac{2}{7}$ \\ 
        \end{tabular}
        \end{center}
    \begin{enumerate}[(a)]
        \item Compute $\p(X \leq 3)$
        \begin{proof}[Solution]
            $\p(X \leq 3) = \frac{1}{7} + \frac{1}{14} + \frac{3}{14} = \frac{3}{7}$.
        \end{proof}

        \item Compute $\p(X < 3)$
        \begin{proof}[Solution]
            $\p(X < 3) = \frac{1}{7} + \frac{1}{14} = \frac{3}{14}$.
        \end{proof}

        \item Compute $\p(X < 4.12 \, | \, X > 1.638).$
        \begin{proof}[Solution]

        \begin{align*}
            \p(X < 4.12 \text{ and } X > 1.638) 
            &= \frac{1}{14} + \frac{3}{14} + \frac{2}{7} = \frac{4}{7}, \\
            \p(X > 1.638) 
            &= \frac{1}{14} + \frac{3}{14} + \frac{2}{7} + \frac{2}{7} = \frac{6}{7}, \\
            \p(X < 4.12 \, | \, X > 1.638) 
            &= \frac{\p(X < 4.12 \text{ and } X > 1.638)}{\p(X > 1.638)} = \frac{2}{3}.
        \end{align*}
        \end{proof}
    \end{enumerate}

    \newpage

    \item[Question 2] Let $X$ be a continuous random variable with density function
    \[
        f(x) = \begin{cases}
            3e^{-3x} & \text{if }x > 0 \\
            0 & \text{otherwise}
        \end{cases}
    \]
    \begin{enumerate}[(a)]
        \item Verify that $f$ is a density function.
        \begin{proof}[Solution]
        $f$ is a density function because
            \begin{align*}
                \int^{\infty}_{-\infty} f(x) dx
                &= \int^{\infty}_{0} 3e^{-3x} dx \\
                &= \left.-e^{-3x} \right|^{\infty}_0 \\
                &= 1.
            \end{align*}
        \end{proof}

        \item Compute $\p(-1 < X < 1)$.
        \begin{proof}[Solution]
            \begin{align*}
                \p(-1 < X < 1) 
                &= \int^{1}_{0} 3e^{-3x} dx \\
                &= \left.-e^{-3x} \right|^{1}_{0} \\
                &= 1 - e^{-3}.
            \end{align*}
        \end{proof}

        \item Compute $\p(0 < X < 1)$.
        \begin{proof}[Solution]
            \begin{align*}
                \p(0 < X < 1) 
                &= \int^{1}_{0} 3e^{-3x} dx \\
                &= \left.-e^{-3x} \right|^{1}_{0} \\
                &= 1 - e^{-3}.
            \end{align*}
        \end{proof}

        \item Compute $\p(X < 5)$.
        \begin{proof}[Solution]
            \begin{align*}
                \p(X < 5) 
                &= \int^{5}_{0} 3e^{-3x} dx \\
                &= \left.-e^{-3x} \right|^{5}_{0} \\
                &= 1 - e^{-15}.
            \end{align*}
        \end{proof}

        \item Compute $\E[X]$ and Var($X$).
        \begin{proof}[Solution]
            \begin{align*}
                \E[X] 
                &= \int^{\infty}_{-\infty} xf(x) dx \\
                &= \int^{\infty}_{0} 3xe^{-3x} dx \\
                &= \left.-xe^{-3x}\right|^{\infty}_0 + \int^{\infty}_{0} e^{-3x} dx \\
                &= -e^{-3x}\left.\left(x + \frac{1}{3}\right)\right|^{\infty}_0 \\
                &= \frac{1}{3}.
            \end{align*}

            \begin{align*}
                \text{Var}(X) 
                &= \E[X^2] - (\E[X])^2 \\
                &= \int^{\infty}_{0} 3x^2e^{-3x} dx - \frac{1}{9} \\
                &= \left.-x^2e^{-3x}\right|^{\infty}_0 + \frac{2}{3}\E[X] - \frac{1}{9} = \frac{1}{9}.
            \end{align*}
        \end{proof}
    \end{enumerate}

    \newpage

    \item[Question 3] Let $X$ be a continuous random variable with a cumulative distribution function
    \[
        F(x) = \begin{cases}
            0 & \text{if }x < \sqrt{2} \\
            x^2 - 2 & \text{if }\sqrt{2} \leq x < \sqrt{3} \\
            1 & \text{if }\sqrt{3} \leq x \\
        \end{cases}
    \]
    \begin{enumerate}[(a)]
        \item Find $\p(X = 1.6)$.
        \begin{proof}[Solution]
            $\p(X = 1.6) = F(1.6) - F(1.6) = 0$.
        \end{proof}

        \item Compute $\p(1 \leq X \leq \frac{3}{2})$.
        \begin{proof}[Solution]
            $\p(1 \leq X \leq \frac{3}{2}) = F(\frac{3}{2}) - F(1) = \frac{1}{4}$.
        \end{proof}

        \item Compute $\p(1 < X \leq \frac{3}{2})$.
        \begin{proof}[Solution]
            $\p(1 < X \leq \frac{3}{2}) = F(\frac{3}{2}) - F(1) = \frac{1}{4}$.
        \end{proof}

        \item Find the probability density function of $X$.
        \begin{proof}[Solution]
        Let $p(x)$ be the probability density function of $X$.
            \begin{align*}
                p(x) 
                &= \frac{d}{dx}F(x) \\
                &= \begin{cases}
                    2x, & x \in [\sqrt{2}, \sqrt{3}) \\
                    0, & x \notin [\sqrt{2}, \sqrt{3}).
                \end{cases}
            \end{align*}
        \end{proof}

        \item Find the smallest interval $[a, b]$ such that $\p(a \leq X \leq b) = 1$
        \begin{proof}[Solution]
            We first notice that $F(x)$ is an increasing function and $0 \leq F(x) \leq 1$. Since $F(x) = 0$ if $x \leq \sqrt{2}$ and $F(x) = 1$ if $x \geq \sqrt{3}$, the smallest interval $[a, b]$ is $[\sqrt{2}, \sqrt{3}]$.
        \end{proof}
    \end{enumerate}

    \newpage

    \item[Question 4] Let $X$ be a continuous random variable with a density function given by
    \[
        f_X(x) = \begin{cases}
            \frac{1}{2}x^{-\frac{3}{2}} &\text{if }1 < x < \infty \\
            0 & \text{otherwise}
        \end{cases}.
    \]
    Compute $\E[X]$, Var($X$), and $\E[X^{\frac{1}{4}}]$.

    \begin{proof}[Solution]
        \begin{align*}
            \E[X]
            &= \int^{\infty}_{-\infty} xf_X(x) dx \\
            &= \int^{\infty}_1 \frac{1}{2}x^{-\frac{1}{2}} dx \\
            &= \left[x^{\frac{1}{2}}\right]^{\infty}_1 \rightarrow \infty.
        \end{align*}
        Thus, $\E[X]$ is divergent.

        \begin{align*}
            \text{Var}(X)
            &= \E[X^2] - \E[X]^2 \\
            &= \int^{\infty}_{1} \frac{1}{2}x^{\frac{1}{2}} dx - \E[X]^2 \\
            &= \left[\frac{1}{3}x^{\frac{3}{2}}\right]^{\infty}_1 - \left(\left[x^{\frac{1}{2}}\right]^{\infty}_1\right)^2 \\
            &= \lim_{x \to \infty} \frac{1}{3}x^{\frac{3}{2}} - x + 2x^{\frac{1}{2}} - \frac{4}{3} \rightarrow \infty
        \end{align*}
        Thus, Var($X$) is divergent.

        \begin{align*}
            \E[X^{\frac{1}{4}}]
            &= \int^{\infty}_{-\infty} x^{\frac{1}{4}}f_X(x) dx \\
            &= \int^{\infty}_1 \frac{1}{2}x^{-\frac{5}{4}} dx \\
            &= \left[-2x^{-\frac{1}{4}}\right]^{\infty}_1 = 2.
        \end{align*}
    \end{proof}

    \newpage

    \item[Question 5] Let $X \sim \text{Unif}([a, b])$. Compute its expectation and variance.

    \begin{proof}[Solution]
        \begin{align*}
            \E[X]
            &= \int^b_a \frac{x}{b - a} dx \\
            &= \left.\frac{x^2}{2(b - a)}\right|^b_a \\
            &= \frac{b + a}{2}.
        \end{align*}

        \begin{align*}
            \text{Var}(X) 
            &= \E[X^2] - \E[X]^2 \\
            &= -\left(\frac{b + a}{2}\right)^2 + \int^b_a \frac{x^2}{b - a} dx \\
            &= \left.\frac{x^3}{3(b - a)}\right|^b_a - \left(\frac{b + a}{2}\right)^2 \\
            &= \frac{b^3 - a^3}{3(b - a)} - \left(\frac{b + a}{2}\right)^2 \\
            &= \frac{a^2 + ab + b^2}{3} - \frac{a^2 + 2ab + b^2}{4} \\
            &= \frac{(b - a)^2}{12}.
        \end{align*}
    \end{proof}

    \newpage

    \item[Question 6] Show that Var$(X) = \E[X^2] - \E[X]^2 $.

    \begin{proof}
        Let $p(x)$ be the density function of $X$, and let $\mu = \E[X] = \int^{\infty}_{-\infty} xp(x) dx$. Thus,
        \begin{align*}
            \text{Var}(X) 
            &= \int^{\infty}_{-\infty} (x - \mu)^2p(x) dx \\
            &= \int^{\infty}_{-\infty} x^2p(x) - 2\mu xp(x) + \mu^2p(x) dx \\
            &= \int^{\infty}_{-\infty} x^2p(x) dx - 2\mu\int^{\infty}_{-\infty} xp(x) dx + \mu^2\int^{\infty}_{-\infty} p(x) dx \\
            &= \E[X^2] - 2\mu^2 + \mu^2 \\
            &= \E[X^2] - \E[X]^2.
        \end{align*}
    \end{proof}

    \newpage

    \item[Question 7]  Let X be a continuous random variable with a cumulative distribution function given by
    \[
        F(x) = \begin{cases}
            \frac{x}{x+1} & \text{if }x \geq 0 \\
            0 & \text{if }x < 0
        \end{cases}.
    \]
    \begin{enumerate}[(a)]
        \item Find the probability density function of $X$.
        \begin{proof}[Solution]
            Let $f(x)$ be the probability density function of $X$.
            \begin{align*}
                f(x)
                &= \frac{d}{dx}F(x) \\
                &= \begin{cases}
                        \frac{1}{(x+1)^2} & \text{if }x \geq 0 \\
                        0 & \text{if }x < 0
                    \end{cases}.
            \end{align*}
        \end{proof}

        \item Calculate $\E[(1 + X)^2e^{-2X}]$.
        \begin{proof}[Solution]
            \begin{align*}
                \E[(1 + X)^2e^{-2X}]
                &= \int^{\infty}_{-\infty} (1 + x)^2e^{-2x}f(x) dx \\
                &= \int^{\infty}_{0} \frac{(1 + x)^2e^{-2x}}{(x+1)^2} dx \\
                &= \int^{\infty}_{0} e^{-2x} dx \\
                &= \left[-\frac{1}{2}e^{-2x}\right]^{\infty}_0 = \frac{1}{2}.
            \end{align*}
        \end{proof}
    \end{enumerate}

    \newpage

    \item[Question 8] Let $X \sim \text{Geom}(p)$ be a geometric random variable with parameter $p$. Compute its variance.

    \begin{proof}[Solution]
    By the law of total expectation,
        \begin{align*}
            \E[X] 
            &= p \cdot 1 + (1 - p)(\E[X] + 1) \\
            &= \E[X](1 - p) + 1 \\
            &= \frac{1}{p}.
        \end{align*}
        
        \begin{align*}
            \text{Var}(X)
            &= \E[X^2] - \E[X]^2 \\
            &= \E[X(X - 1)] + \frac{1}{p} - \frac{1}{p^2} \\
            &=  \frac{1}{p} - \frac{1}{p^2} + \sum^{\infty}_{n = 1} n(n - 1)p(1 - p)^{n - 1} \\
            &= \frac{1}{p} - \frac{1}{p^2} + p(1 - p)\sum^{\infty}_{n = 0} n(n - 1)(1 - p)^{n - 2} \\
            &= \frac{1}{p} - \frac{1}{p^2} + p(1 - p)\sum^{\infty}_{n = 0} \frac{d^2}{d(1 - p)^2}(1 - p)^{n} \\
            &= \frac{1}{p} - \frac{1}{p^2} + p(1 - p)\left(\frac{d^2}{d(1- 
            p)^2}\frac{1}{p}\right) \\
            &= \frac{1}{p} - \frac{1}{p^2} + \frac{2(1 - p)}{p^2} \\
            &= \frac{1 - p}{p^2}.
        \end{align*}
    \end{proof}
    
\end{description}

\end{document}
