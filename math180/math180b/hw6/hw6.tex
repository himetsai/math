\documentclass{article}

\usepackage{fancyhdr}
\usepackage{extramarks}
\usepackage{amsmath}
\usepackage{amsthm}
\usepackage{amsfonts}
\usepackage{tikz}
\usepackage[plain]{algorithm}
\usepackage{algpseudocode}
\usepackage{enumerate}
\usepackage{amssymb}
\usepackage{dsfont}

\usetikzlibrary{automata,positioning}

%
% Basic Document Settings
%

\topmargin=-0.45in
\evensidemargin=0in
\oddsidemargin=0in
\textwidth=6.5in
\textheight=9.0in
\headsep=0.25in

\linespread{1.1}

\pagestyle{fancy}
\lhead{\hmwkAuthorName}
\chead{\hmwkClass:\ \hmwkTitle}
\rhead{\firstxmark}
\lfoot{\lastxmark}
\cfoot{\thepage}

\renewcommand\headrulewidth{0.4pt}
\renewcommand\footrulewidth{0.4pt}

\setlength\parindent{0pt}
\setlength{\parskip}{5pt}

%
% Create Problem Sections
%

\newcommand{\enterProblemHeader}[1]{
    \nobreak\extramarks{}{Problem \arabic{#1} continued on next page\ldots}\nobreak{}
    \nobreak\extramarks{Problem \arabic{#1} (continued)}{Problem \arabic{#1} continued on next page\ldots}\nobreak{}
}

\newcommand{\exitProblemHeader}[1]{
    \nobreak\extramarks{Problem \arabic{#1} (continued)}{Problem \arabic{#1} continued on next page\ldots}\nobreak{}
    \stepcounter{#1}
    \nobreak\extramarks{Problem \arabic{#1}}{}\nobreak{}
}

\setcounter{secnumdepth}{0}
\newcounter{partCounter}
\newcounter{homeworkProblemCounter}
\setcounter{homeworkProblemCounter}{1}
\nobreak\extramarks{Problem \arabic{homeworkProblemCounter}}{}\nobreak{}

%
% Homework Problem Environment
%
% This environment takes an optional argument. When given, it will adjust the
% problem counter. This is useful for when the problems given for your
% assignment aren't sequential. See the last 3 problems of this template for an
% example.
%
\newenvironment{homeworkProblem}[1][-1]{
    \ifnum#1>0
        \setcounter{homeworkProblemCounter}{#1}
    \fi
    \section{Problem \arabic{homeworkProblemCounter}}
    \setcounter{partCounter}{1}
    \enterProblemHeader{homeworkProblemCounter}
}{
    \exitProblemHeader{homeworkProblemCounter}
}

%
% Homework Details
%   - Title
%   - Due date
%   - Class
%   - Section/Time
%   - Instructor
%   - Author
%

\newcommand{\hmwkTitle}{Homework\ \#6}
\newcommand{\hmwkDueDate}{Mar 08, 2024}
\newcommand{\hmwkClass}{MATH 180B}
\newcommand{\hmwkClassInstructor}{Professor Carfagnini}
\newcommand{\hmwkAuthorName}{\textbf{Ray Tsai}}
\newcommand{\hmwkPID}{A16848188}

%
% Title Page
%

\title{
    \vspace{2in}
    \textmd{\textbf{\hmwkClass:\ \hmwkTitle}}\\
    \normalsize\vspace{0.1in}\small{Due\ on\ \hmwkDueDate\ at 23:59pm}\\
    \vspace{0.1in}\large{\textit{\hmwkClassInstructor}} \\
    \vspace{3in}
}

\author{
  \hmwkAuthorName \\
  \vspace{0.1in}\small\hmwkPID
}
\date{}

\renewcommand{\part}[1]{\textbf{\large Part \Alph{partCounter}}\stepcounter{partCounter}\\}

%
% Various Helper Commands
%

% Useful for algorithms
\newcommand{\alg}[1]{\textsc{\bfseries \footnotesize #1}}

% For derivatives
\newcommand{\deriv}[1]{\frac{\mathrm{d}}{\mathrm{d}x} (#1)}

% For partial derivatives
\newcommand{\pderiv}[2]{\frac{\partial}{\partial #1} (#2)}

% Integral dx
\newcommand{\dx}{\mathrm{d}x}

% Probability commands: Expectation, Variance, Covariance, Bias
\newcommand{\Var}{\mathrm{Var}}
\newcommand{\Cov}{\mathrm{Cov}}
\newcommand{\Bias}{\mathrm{Bias}}
\newcommand*{\Z}{\mathbb{Z}}
\newcommand*{\Q}{\mathbb{Q}}
\newcommand*{\R}{\mathbb{R}}
\newcommand*{\C}{\mathbb{C}}
\newcommand*{\N}{\mathbb{N}}
\newcommand*{\p}{\mathds{P}}
\newcommand*{\E}{\mathds{E}}

\begin{document}

\maketitle

\pagebreak

\begin{homeworkProblem}
  Which states are transient and which are recurrent in the Markov chain whose transition
  probability matrix is
  \begin{center}
    \begin{tabular}{c ||c c c c c c||}
      \multicolumn{1}{c}{\phantom{A}} & \multicolumn{1}{c}{$0$} &
      \multicolumn{1}{c}{$1$} & \multicolumn{1}{c}{$2$} & \multicolumn{1}{c}{$3$} &
      \multicolumn{1}{c}{$4$} & \multicolumn{1}{c}{$5$} \\
      0 & $\frac{1}{3}$ & 0 & $\frac{1}{3}$ & 0 & 0 & $\frac{1}{3}$ \\
      1 & $\frac{1}{2}$ & $\frac{1}{4}$ & $\frac{1}{4}$ & 0 & 0 & 0 \\
      2 & 0 & 0 & 0 & 0 & 1 & 0 \\
      3 & $\frac{1}{4}$ & $\frac{1}{4}$ & $\frac{1}{4}$ & 0 & 0 & $\frac{1}{4}$ \\
      4 & 0 & 0 & 1 & 0 & 0 & 0 \\
      5 & 0 & 0 & 0 & 0 & 0 & 1 \\
    \end{tabular} ?
  \end{center}
  \begin{proof}
    The communicating classes of the matrix are $\{0\}, \{1\}, \{3\}, \{5\}, \{2, 4\}$. Note that
    $\{5\}$ and $\{2, 4\}$ are closed classes, so states $2, 4, 5$ are recurrent. In addition since
    $0, 1, 3$ are connected all to closed classes, they will eventually get stuck in those closed
    classes and never return to the original state, and thus they are transient. 
  \end{proof}
\end{homeworkProblem}

\newpage

\begin{homeworkProblem}
  Determine the communicating classes and period for each state of the Markov chain whose transition
  probability matrix is

  \begin{center}
    \begin{tabular}{c ||c c c c c c||}
      \multicolumn{1}{c}{\phantom{A}} & \multicolumn{1}{c}{$0$} &
      \multicolumn{1}{c}{$1$} & \multicolumn{1}{c}{$2$} & \multicolumn{1}{c}{$3$} &
      \multicolumn{1}{c}{$4$} & \multicolumn{1}{c}{$5$} \\
      0 & $\frac{1}{2}$ & 0 & 0 & 0 & $\frac{1}{2}$ & 0 \\
      1 & 0 & 0 & 1 & 0 & 0 & 0 \\
      2 & 0 & 0 & 0 & 1 & 0 & 0 \\
      3 & 0 & 0 & 0 & 0 & 1 & 0 \\
      4 & 0 & 0 & 0 & 0 & 0 & 1 \\
      5 & 0 & 0 & $\frac{1}{3}$ & $\frac{1}{3}$ & 0 & $\frac{1}{3}$ \\
    \end{tabular} ?
  \end{center}

  \begin{proof}
    Since states each of $0, 1$ is not accessible to any other classes, they each form their own
    classes. Since there exists a cycle $2 \to 3 \to 4 \to 5 \to 2$ which passes through all the
    rest of the states, they all belong to a class. Hence, the communicating classes are $\{0\},
    \{1\}, \{2, 3, 4, 5\}$. Since $0$ and $5$ are both connected to itself, $\{0\}$ and $\{2, 3, 4,
    5\}$ have period 1. Since 1 is not accessible from any states including itself, $d(1) = 0$. 
  \end{proof}
\end{homeworkProblem}

\newpage

\begin{homeworkProblem}
  Recall the first return distribution
  \[
  f_{ii}^{(n)} = \Pr\{X_1 \neq i, X_2 \neq j,\ldots, X_{n-1} \neq i, X_n = i \mid X_0 = i\} \text{ for } n = 1,2,\ldots,
  \]
  with $f_{ii}^{(0)} = 0$ by convention. Using equation (4.16), determine $f_{00}^{(n)}, n =
  1,2,3,4,$ for the Markov chain whose transition probability matrix is
  \begin{center}
    \begin{tabular}{c ||c c c c||}
      \multicolumn{1}{c}{\phantom{A}} & \multicolumn{1}{c}{$0$} &
      \multicolumn{1}{c}{$1$} & \multicolumn{1}{c}{$2$} & \multicolumn{1}{c}{$3$} \\
      0 & 0 & $\frac{1}{2}$ & 0 & $\frac{1}{2}$ \\
      1 & 0 & 0 & 1 & 0 \\
      2 & 0 & 0 & 0 & 1 \\
      3 & $\frac{1}{2}$ & 0 & 0 & $\frac{1}{2}$ \\
    \end{tabular}.
  \end{center}

  \begin{proof}
    Call that matrix $P$. Note that 
    \begin{center}
      $P^2 = $ \begin{tabular}{c ||c c c c||}
        \multicolumn{1}{c}{\phantom{A}} & \multicolumn{1}{c}{$0$} &
        \multicolumn{1}{c}{$1$} & \multicolumn{1}{c}{$2$} & \multicolumn{1}{c}{$3$} \\
        0 & $\frac{1}{4}$ & 0 & $\frac{1}{2}$ & $\frac{1}{4}$ \\
        1 & 0 & 0 & 0 & 1 \\
        2 & $\frac{1}{2}$ & 0 & 0 & $\frac{1}{2}$ \\
        3 & $\frac{1}{4}$ & $\frac{1}{4}$ & 0 & $\frac{1}{2}$ \\
      \end{tabular}, \quad 
      $P^3 = $ \begin{tabular}{c ||c c c c||}
        \multicolumn{1}{c}{\phantom{A}} & \multicolumn{1}{c}{$0$} &
        \multicolumn{1}{c}{$1$} & \multicolumn{1}{c}{$2$} & \multicolumn{1}{c}{$3$} \\
        0 & $\frac{1}{8}$ & $\frac{1}{8}$ & 0 & $\frac{3}{4}$ \\
        1 & $\frac{1}{2}$ & 0 & 0 & $\frac{1}{2}$ \\
        2 & $\frac{1}{4}$ & $\frac{1}{4}$ & 0 & $\frac{1}{2}$ \\
        3 & $\frac{1}{4}$ & $\frac{1}{8}$ & $\frac{1}{4}$ & $\frac{3}{8}$ \\
      \end{tabular}, \quad
      $P^4 = $ \begin{tabular}{c ||c c c c||}
        \multicolumn{1}{c}{\phantom{A}} & \multicolumn{1}{c}{$0$} &
        \multicolumn{1}{c}{$1$} & \multicolumn{1}{c}{$2$} & \multicolumn{1}{c}{$3$} \\
        0 & $\frac{3}{8}$ & $\frac{1}{16}$ & $\frac{1}{8}$ & $\frac{7}{16}$ \\
        1 & $\frac{1}{4}$ & $\frac{1}{4}$ & 0 & $\frac{1}{2}$ \\
        2 & $\frac{1}{4}$ & $\frac{1}{8}$ & $\frac{1}{4}$ & $\frac{3}{8}$ \\
        3 & $\frac{3}{16}$ & $\frac{1}{8}$ & $\frac{1}{8}$ & $\frac{9}{16}$ \\
      \end{tabular}.
    \end{center}
    Hence, we get
    \begin{gather*}
      P^1_{00} = 0 = f^{(0)}_{00}P^1_{00} + f^{(1)}_{00}P^0_{00} = f^{(1)}_{00} \\
      P^2_{00} = \frac{1}{4} = f^{(0)}_{00}P^2_{00} + f^{(1)}_{00}P^1_{00} + f^{(2)}_{00}P^0_{00} = f^{(2)}_{00} \\
      P^3_{00} = \frac{1}{8} = f^{(0)}_{00}P^3_{00} + f^{(1)}_{00}P^2_{00} + f^{(2)}_{00}P^1_{00} + f^{(3)}_{00}P^0_{00} = f^{(3)}_{00} \\
      P^4_{00} = \frac{3}{8} = f^{(0)}_{00}P^4_{00} + f^{(1)}_{00}P^3_{00} + f^{(2)}_{00}P^2_{00} + f^{(3)}_{00}P^1_{00} + f^{(4)}_{00}P^0_{00} = \frac{1}{4} \cdot \frac{1}{4} + f^{(4)}_{00},
    \end{gather*}
    and thus $f^{(1)}_{00} = 0, f^{(2)}_{00} = \frac{1}{4}, f^{(3)}_{00} = \frac{1}{8}, f^{(4)}_{00}
    = \frac{5}{16}$.
  \end{proof}
\end{homeworkProblem}

\newpage

\begin{homeworkProblem}
  Let \{$\alpha_i : i = 1, 2, \ldots$\} be a probability distribution, and consider the Markov chain
  whose transition probability matrix is

  \begin{center}
  \begin{tabular}{c ||c c c c c c c||}
  \multicolumn{1}{c}{} & \multicolumn{1}{c}{0} & \multicolumn{1}{c}{1} & \multicolumn{1}{c}{2} &
  \multicolumn{1}{c}{3} & \multicolumn{1}{c}{4} & \multicolumn{1}{c}{5} &
  \multicolumn{1}{c}{$\cdots$} \\
  0 & $\alpha_1$ & $\alpha_2$ & $\alpha_3$ & $\alpha_4$ & $\alpha_5$ & $\alpha_6$ & $\cdots$ \\
  1 & 1 & 0 & 0 & 0 & 0 & 0 & $\cdots$ \\
  2 & 0 & 1 & 0 & 0 & 0 & 0 & $\cdots$ \\
  3 & 0 & 0 & 1 & 0 & 0 & 0 & $\cdots$ \\
  4 & 0 & 0 & 0 & 1 & 0 & 0 & $\cdots$ \\
  $\vdots$ & $\vdots$ & $\vdots$ & $\vdots$ & $\vdots$ & $\vdots$ & $\vdots$ & $\ddots$ \\
  \end{tabular}
  \end{center}

  What condition on the probability distribution \{$\alpha_i : i = 1, 2, \ldots$\} is necessary and
  sufficient in order that a limiting distribution exist, and what is this limiting distribution?
  Assume $\alpha_1 > 0$ and $\alpha_2 > 0$ so that the chain is aperiodic.

  \begin{proof}
    Call that matrix $P$. We show that $\sum_{k = 1}^{\infty} k\alpha_k < \infty$ is the necessary
    and sufficient condition to the existence of the limiting distribution.
    
    $P$ is obviously aperiodic and irreducible. Hence, if state $0$ is
    recurrent, every state in $P$ is recurrent. Let $R = \inf \{n \geq 1; X_n = 0\}$. Note that $P(R
    \leq k \mid X_0 = 0) = \sum_{i = 1}^{k} f^{(i)}_{00} = \sum_{i = 1}^{k} \alpha_i$. Clearly,
    $\lim\limits_{k \to \infty} \sum_{i = 1}^{k} \alpha_i = 1$. It follows that $f_{00} =
    \lim\limits_{k \to \infty} \sum_{i = 1}^{k} f^{(i)}_{00} = 1$, so $P$ is indeed recurrent. 

    Suppose that $\sum_{k = 1}^{\infty} k\alpha_k < \infty$. Since $m_0 = E[R \mid X_0 = 0] =
    \sum_{k = 1}^{\infty} k\alpha_k < \infty$, the state 0 is positively recurrent, with
    \begin{align*}
      \pi_0 &= \lim\limits_{n \to \infty} P^{(n)}_{00} = \frac{1}{m_0} = \frac{1}{\sum_{k = 1}^{\infty} k\alpha_k} \\
      \pi_1 &= (1 - \alpha_1)\pi_0 \\
      \pi_2 &= (1 - \alpha_1 - \alpha_2)\pi_0 \\
      &\vdots \\
      \pi_n &= \left(1 - \sum_{i = 1}^n \alpha_i\right)\pi_0 \\
      & \vdots
    \end{align*}
    Hence, the limiting distribution is $\pi_k = \frac{1 - \sum_{i = 1}^k \alpha_i}{m_0}$.

    Suppose that $\sum_{k = 1}^{\infty} k\alpha_k = \infty$. Then, $m_0 = \infty$. It follows that
    $\pi_0 = \frac{1}{m_0} = 0$, so the limiting distribution does not exist.
  \end{proof}
\end{homeworkProblem}

\newpage

\begin{homeworkProblem}
  Determine the period of state 0 in the Markov chain whose transition probability matrix is

  \begin{center}
    $P = $ \begin{tabular}{c ||c c c c c c c c||}
    \multicolumn{1}{c}{} & \multicolumn{1}{c}{3} & \multicolumn{1}{c}{2} & \multicolumn{1}{c}{1} &
    \multicolumn{1}{c}{0} & \multicolumn{1}{c}{-1} & \multicolumn{1}{c}{-2} &
    \multicolumn{1}{c}{-3} & \multicolumn{1}{c}{-4} \\
    3 & 0 & 0 & 1 & 0 & 0 & 0 & 0 & 0 \\
    2 & 1 & 0 & 0 & 0 & 0 & 0 & 0 & 0 \\
    1 & 0 & 1 & 0 & 0 & 0 & 0 & 0 & 0 \\
    0 & 0 & 0 & $\frac{1}{2}$ & 0 & $\frac{1}{2}$ & 0 & 0 & 0 \\
    -1 & 0 & 0 & 0 & 0 & 0 & 1 & 0 & 0 \\
    -2 & 0 & 0 & 0 & 0 & 0 & 0 & 1 & 0 \\
    -3 & 0 & 0 & 0 & 0 & 0 & 0 & 0 & 1 \\
    -4 & 0 & 0 & 0 & 1 & 0 & 0 & 0 & 0 \\
    \end{tabular}
  \end{center}

  \begin{proof}
    There are two communicating classes in $P$, namely $\{1, 2, 3\}$ and $\{0, -1, -2, -3, -4\}$.
    Note that $0$ is accessible to both classes. But since $\{1, 2, 3\}$ is a closed class, there
    are no path to return to 0 after entering that class. Hence, we may focus on the class $\{0, -1,
    -2, -3, -4\}$. Since the only path to return to 0 is via the cycle $0 \to -1 \to -2 \to -3 \to
    -4 \to 0$, the state $0$ is of period 5.
  \end{proof}
\end{homeworkProblem}

\newpage

\begin{homeworkProblem}
  A Markov chain on states $0, 1, \ldots$ has transition probabilities
  \[ 
    P_{ij} = \frac{1}{i+2} \quad \text{for } j = 0, 1, \ldots, i, i+1.
  \]
  Find the stationary distribution.

  \begin{proof}
    We solve for $\pi = \pi P$ and get $\pi_0 = \sum_{i = 0}^{\infty} \frac{x_i}{i + 2}$ and $\pi_k
    = \sum_{i = k}^{\infty} \frac{x_{i - 1}}{i + 1}$, for $k > 0$. Hence, we get $\pi_k =
    \frac{\pi_0}{k!}$. Since $\sum_{i = 0}^{\infty} \pi_i = \pi_0\sum_{k = 0}^{\infty} \frac{1}{k!}
    = e\pi_0 = 1$, we get $\pi_0 = \frac{1}{e}$, and so $\pi_k = \frac{1}{k!e}$.
  \end{proof}
\end{homeworkProblem}
\end{document}