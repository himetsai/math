\documentclass{article}

\usepackage{fancyhdr}
\usepackage{extramarks}
\usepackage{amsmath}
\usepackage{amsthm}
\usepackage{amsfonts}
\usepackage{tikz}
\usepackage[plain]{algorithm}
\usepackage{algpseudocode}
\usepackage{enumerate}
\usepackage{amssymb}
\usepackage{dsfont}

\usetikzlibrary{automata,positioning}

%
% Basic Document Settings
%

\topmargin=-0.45in
\evensidemargin=0in
\oddsidemargin=0in
\textwidth=6.5in
\textheight=9.0in
\headsep=0.25in

\linespread{1.1}

\pagestyle{fancy}
\lhead{\hmwkAuthorName}
\chead{\hmwkClass:\ \hmwkTitle}
\rhead{\firstxmark}
\lfoot{\lastxmark}
\cfoot{\thepage}

\renewcommand\headrulewidth{0.4pt}
\renewcommand\footrulewidth{0.4pt}

\setlength\parindent{0pt}
\setlength{\parskip}{5pt}

%
% Create Problem Sections
%

\newcommand{\enterProblemHeader}[1]{
    \nobreak\extramarks{}{Problem \arabic{#1} continued on next page\ldots}\nobreak{}
    \nobreak\extramarks{Problem \arabic{#1} (continued)}{Problem \arabic{#1} continued on next page\ldots}\nobreak{}
}

\newcommand{\exitProblemHeader}[1]{
    \nobreak\extramarks{Problem \arabic{#1} (continued)}{Problem \arabic{#1} continued on next page\ldots}\nobreak{}
    \stepcounter{#1}
    \nobreak\extramarks{Problem \arabic{#1}}{}\nobreak{}
}

\setcounter{secnumdepth}{0}
\newcounter{partCounter}
\newcounter{homeworkProblemCounter}
\setcounter{homeworkProblemCounter}{1}
\nobreak\extramarks{Problem \arabic{homeworkProblemCounter}}{}\nobreak{}

%
% Homework Problem Environment
%
% This environment takes an optional argument. When given, it will adjust the
% problem counter. This is useful for when the problems given for your
% assignment aren't sequential. See the last 3 problems of this template for an
% example.
%
\newenvironment{homeworkProblem}[1][-1]{
    \ifnum#1>0
        \setcounter{homeworkProblemCounter}{#1}
    \fi
    \section{Problem \arabic{homeworkProblemCounter}}
    \setcounter{partCounter}{1}
    \enterProblemHeader{homeworkProblemCounter}
}{
    \exitProblemHeader{homeworkProblemCounter}
}

%
% Homework Details
%   - Title
%   - Due date
%   - Class
%   - Section/Time
%   - Instructor
%   - Author
%

\newcommand{\hmwkTitle}{Homework\ \#1}
\newcommand{\hmwkDueDate}{Apr 12, 2024}
\newcommand{\hmwkClass}{MATH 180C}
\newcommand{\hmwkClassInstructor}{Professor Carfagnini}
\newcommand{\hmwkAuthorName}{\textbf{Ray Tsai}}
\newcommand{\hmwkPID}{A16848188}

%
% Title Page
%

\title{
    \vspace{2in}
    \textmd{\textbf{\hmwkClass:\ \hmwkTitle}}\\
    \normalsize\vspace{0.1in}\small{Due\ on\ \hmwkDueDate\ at 23:59pm}\\
    \vspace{0.1in}\large{\textit{\hmwkClassInstructor}} \\
    \vspace{3in}
}

\author{
  \hmwkAuthorName \\
  \vspace{0.1in}\small\hmwkPID
}
\date{}

\renewcommand{\part}[1]{\textbf{\large Part \Alph{partCounter}}\stepcounter{partCounter}\\}

%
% Various Helper Commands
%

% Useful for algorithms
\newcommand{\alg}[1]{\textsc{\bfseries \footnotesize #1}}

% For derivatives
\newcommand{\deriv}[1]{\frac{\mathrm{d}}{\mathrm{d}x} (#1)}

% For partial derivatives
\newcommand{\pderiv}[2]{\frac{\partial}{\partial #1} (#2)}

% Integral dx
\newcommand{\dx}{\mathrm{d}x}

% Probability commands: Expectation, Variance, Covariance, Bias
\newcommand{\Var}{\mathrm{Var}}
\newcommand{\Cov}{\mathrm{Cov}}
\newcommand{\Bias}{\mathrm{Bias}}
\newcommand*{\Z}{\mathbb{Z}}
\newcommand*{\Q}{\mathbb{Q}}
\newcommand*{\R}{\mathbb{R}}
\newcommand*{\C}{\mathbb{C}}
\newcommand*{\N}{\mathbb{N}}
\newcommand*{\p}{\mathds{P}}
\newcommand*{\E}{\mathds{E}}

\begin{document}

\maketitle

\pagebreak

\begin{homeworkProblem}
  A pure birth process starting from $X(0) = 0$ has birth parameters $\lambda_0 = 1$, $\lambda_1 =
  3$, $\lambda_2 = 2$, and $\lambda_3 = 5$. Let $W_3$ be the random time that it takes the process
  to reach state 3.

  \begin{enumerate}[(a)]
      \item Write $W_3$ as a sum of sojourn times and thereby deduce that the mean time is $E[W_3] =
      \frac{11}{6}$.
      \begin{proof}
        We write $W_k = \sum_{i = 0}^{k - 1} S_i$, where $S_i$'s are the sojourn times, with $S_i
        \sim \text{Exp}(\lambda_i)$. Hence,
        \[
          E[W_3] = \sum_{i = 0}^{2} E[S_i] = \sum_{i = 0}^{2} \frac{1}{\lambda_i} = \frac{11}{6}.
        \]
      \end{proof}
      \item Determine the mean of $W_1 + W_2 + W_3$.
      \begin{proof}
        \[
          E[W_1 + W_2 + W_3] = E[W_1] + E[W_2] + E[W_3] = \frac{1}{\lambda_2} + \frac{2}{\lambda_1} + \frac{3}{\lambda_0} = \frac{25}{6}.
        \]
      \end{proof}
      \item What is the variance of $W_3$?
      \begin{proof}
        \[
          Var(W_3) = Var(S_0) + Var(S_1) + Var(S_2) = \sum_{i = 0}^{2} \frac{1}{\lambda_i^2} = \frac{49}{36}.
        \]
      \end{proof}
  \end{enumerate}
\end{homeworkProblem}

\newpage

\begin{homeworkProblem}
  Consider an experiment in which a certain event will occur with probability $\alpha h$ and will
  not occur with probability $1 - \alpha h$, where $\alpha$ is a fixed positive parameter and $h$ is
  a small $(h < \frac{1}{\alpha})$ positive variable. Suppose that $n$ independent trials of the
  experiment are carried out, and the total number of times that the event occurs is noted. Show
  that

  \begin{enumerate}[(a)]
    \item The probability that the event never occurs during the $n$ trials is $1 - n\alpha h +
    o(h)$;
    \begin{proof}
      \begin{align*}
        P\{\text{never occurs during the $n$ trials}\}
        &= \prod_{i = 1}^n P\{\text{doesn't occur in the $i$th trial}\} \\
        &= (1 - \alpha h)^n \\
        &= \sum_{k = 0}^n {n \choose k} (-\alpha h)^k \\
        &= 1 - n\alpha h + h^2\sum_{k = 2}^n {n \choose k} (-\alpha)^kh^{k - 2} \\
        &= 1 - n\alpha h + o(h).
      \end{align*}
    \end{proof}
    \item The probability that the event occurs exactly once is $n\alpha h + o(h)$;
    \begin{proof}
      \begin{align*}
        P\{\text{occurs exactly once in $n$ trials}\}
        &= \sum_{i = 1}^n P\{\text{occurs only in the $i$th trial}\} \\
        &= \sum_{i = 1}^n \alpha h(1 - \alpha h)^{n - 1} \\
        &= n\alpha h\left(1 - h\sum_{k = 1}^{n - 1} {n \choose k} (-\alpha)^kh^{k - 1}\right)^{n - 1} \\
        &= n\alpha \left(h^{\frac{1}{n - 1}} - h^{\frac{n}{n - 1}}\sum_{k = 1}^{n - 1} {n \choose k} (-\alpha)^kh^{k - 1}\right)^{n - 1} \\
        &= n\alpha (h^{\frac{1}{n - 1}} + o(h))^{n - 1} \\
        &= n\alpha(h + o(h)) \\
        &= n\alpha h + o(h).
      \end{align*}
    \end{proof}
    \item The probability that the event occurs twice or more is $o(h)$.
    \begin{proof}
      \begin{align*}
        P\{\text{occurs more than once}\}
        &= 1 - (P\{\text{never occurs in $n$ trials}\} + P\{\text{occurs exactly once in $n$ trials}\}) \\
        &= 1 - (1 - n\alpha h + n\alpha h + o(h)) = o(h).
      \end{align*}
    \end{proof}
  \end{enumerate}
\end{homeworkProblem}

\newpage

\begin{homeworkProblem}
  Let $W_k$ be the time to the $k$th birth in a pure birth process starting from $X(0) = 0$.
  Establish the equivalence

  \[ 
    \Pr\{W_1 > t, W_2 > t + s\} = P_0(t)[P_0(s) + P_1(s)]. 
  \]

  From this relation together with equation (6.7), determine the joint density for $W_1$ and $W_2$,
  and then the joint density of $S_0 = W_1$ and $S_1 = W_2 - W_1$.

  \begin{proof}
    \begin{align*}
      \Pr\{W_1 > t, W_2 > t + s\} 
      &= \Pr\{W_2 > t + s | W_1 > t\}P_0(t) \\
      &= (\Pr\{W_2 > t + s | W_1 > t, W_1 < t + s\}\Pr\{S_1 > s | W_1 > t\} + \Pr\{W_1 \geq t + s | W_1 > t\})P_0(t) \\
      &= P_1(s)(1 - P_0(s)) + P_0(s)P_0(t)
    \end{align*}
  \end{proof}
\end{homeworkProblem}
\end{document}