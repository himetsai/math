\documentclass{article}

\usepackage{fancyhdr}
\usepackage{extramarks}
\usepackage{amsmath}
\usepackage{amsthm}
\usepackage{amsfonts}
\usepackage{tikz}
\usepackage[plain]{algorithm}
\usepackage{algpseudocode}
\usepackage{enumerate}
\usepackage{amssymb}
\usepackage{dsfont}

\usetikzlibrary{automata,positioning}

%
% Basic Document Settings
%

\topmargin=-0.45in
\evensidemargin=0in
\oddsidemargin=0in
\textwidth=6.5in
\textheight=9.0in
\headsep=0.25in

\linespread{1.1}

\pagestyle{fancy}
\lhead{\hmwkAuthorName}
\chead{\hmwkClass:\ \hmwkTitle}
\rhead{\firstxmark}
\lfoot{\lastxmark}
\cfoot{\thepage}

\renewcommand\headrulewidth{0.4pt}
\renewcommand\footrulewidth{0.4pt}

\setlength\parindent{0pt}
\setlength{\parskip}{5pt}

%
% Create Problem Sections
%

\newcommand{\enterProblemHeader}[1]{
    \nobreak\extramarks{}{Problem \arabic{#1} continued on next page\ldots}\nobreak{}
    \nobreak\extramarks{Problem \arabic{#1} (continued)}{Problem \arabic{#1} continued on next page\ldots}\nobreak{}
}

\newcommand{\exitProblemHeader}[1]{
    \nobreak\extramarks{Problem \arabic{#1} (continued)}{Problem \arabic{#1} continued on next page\ldots}\nobreak{}
    \stepcounter{#1}
    \nobreak\extramarks{Problem \arabic{#1}}{}\nobreak{}
}

\setcounter{secnumdepth}{0}
\newcounter{partCounter}
\newcounter{homeworkProblemCounter}
\setcounter{homeworkProblemCounter}{1}
\nobreak\extramarks{Problem \arabic{homeworkProblemCounter}}{}\nobreak{}

%
% Homework Problem Environment
%
% This environment takes an optional argument. When given, it will adjust the
% problem counter. This is useful for when the problems given for your
% assignment aren't sequential. See the last 3 problems of this template for an
% example.
%
\newenvironment{homeworkProblem}[1][-1]{
    \ifnum#1>0
        \setcounter{homeworkProblemCounter}{#1}
    \fi
    \section{Problem \arabic{homeworkProblemCounter}}
    \setcounter{partCounter}{1}
    \enterProblemHeader{homeworkProblemCounter}
}{
    \exitProblemHeader{homeworkProblemCounter}
}

%
% Homework Details
%   - Title
%   - Due date
%   - Class
%   - Section/Time
%   - Instructor
%   - Author
%

\newcommand{\hmwkTitle}{Homework\ \#7}
\newcommand{\hmwkDueDate}{Mar 15, 2024}
\newcommand{\hmwkClass}{MATH 180B}
\newcommand{\hmwkClassInstructor}{Professor Carfagnini}
\newcommand{\hmwkAuthorName}{\textbf{Ray Tsai}}
\newcommand{\hmwkPID}{A16848188}

%
% Title Page
%

\title{
    \vspace{2in}
    \textmd{\textbf{\hmwkClass:\ \hmwkTitle}}\\
    \normalsize\vspace{0.1in}\small{Due\ on\ \hmwkDueDate\ at 23:59pm}\\
    \vspace{0.1in}\large{\textit{\hmwkClassInstructor}} \\
    \vspace{3in}
}

\author{
  \hmwkAuthorName \\
  \vspace{0.1in}\small\hmwkPID
}
\date{}

\renewcommand{\part}[1]{\textbf{\large Part \Alph{partCounter}}\stepcounter{partCounter}\\}

%
% Various Helper Commands
%

% Useful for algorithms
\newcommand{\alg}[1]{\textsc{\bfseries \footnotesize #1}}

% For derivatives
\newcommand{\deriv}[1]{\frac{\mathrm{d}}{\mathrm{d}x} (#1)}

% For partial derivatives
\newcommand{\pderiv}[2]{\frac{\partial}{\partial #1} (#2)}

% Integral dx
\newcommand{\dx}{\mathrm{d}x}

% Probability commands: Expectation, Variance, Covariance, Bias
\newcommand{\Var}{\mathrm{Var}}
\newcommand{\Cov}{\mathrm{Cov}}
\newcommand{\Bias}{\mathrm{Bias}}
\newcommand*{\Z}{\mathbb{Z}}
\newcommand*{\Q}{\mathbb{Q}}
\newcommand*{\R}{\mathbb{R}}
\newcommand*{\C}{\mathbb{C}}
\newcommand*{\N}{\mathbb{N}}
\newcommand*{\p}{\mathds{P}}
\newcommand*{\E}{\mathds{E}}

\begin{document}

\maketitle

\pagebreak

\begin{homeworkProblem}
  Suppose that customers arrive at a facility according to a Poisson process having rate $\lambda = 2$. Let $X(t)$ be the number of customers that have arrived up to time $t$. Determine the following probabilities and conditional probabilities:
  \begin{enumerate}[(a)]
    \item $\Pr\{X(1) = 2\}$.
    \begin{proof}
      \begin{align*}
        \Pr\{X_1 = 2\}
        &= \frac{\lambda^2 e^{-\lambda}}{2!} = 2e^{-2}.
      \end{align*}
    \end{proof}
    \item $\Pr\{X(1) = 2 \text{ and } X(3) = 6\}$.
    \begin{proof}
      \begin{align*}
        \Pr\{X_1 = 2 \text{ and } X_3 = 6\}
        &= \Pr\{X_1 = 2 \text{ and } X_3 - X_1 = 4\} \\
        &= \Pr\{X_1 = 2\}\Pr\{X_3 - X_1 = 4\} \\
        &= 2e^{-2} \cdot \frac{(2\lambda)^4e^{-2\lambda}}{4!} \\
        &= 2e^{-2} \cdot \frac{32e^{-4}}{3} \\
        &= \frac{64e^{-6}}{3}.
      \end{align*}
    \end{proof}
    \item $\Pr\{X(1) = 2 | X(3) = 6\}$.
    \begin{proof}
      By Theorem 5.6,
      \begin{align*}
        \Pr\{X_1 = 2 | X_3 = 6\}
        &= {6 \choose 2}\left(\frac{1}{3}\right)^2\left(1 - \frac{1}{3}\right)^4 \\
        &= 15 \cdot \frac{1}{9} \cdot \frac{16}{81} \\
        &= \frac{80}{243}.
      \end{align*}
    \end{proof}
    \item $\Pr\{X(3) = 6 | X(1) = 2\}$.
    \begin{proof}
      \begin{align*}
        \Pr\{X_3 = 6 | X_1 = 2\}
        &= \frac{\Pr\{X_3 - X_1 = 4\}\Pr\{X_1 = 2\}}{\Pr\{X_1 = 2\}} \\
        &= \Pr\{X_3 - X_1 = 4\} \\
        &= \frac{32e^{-4}}{3}.
      \end{align*}
    \end{proof}
  \end{enumerate}
\end{homeworkProblem}

\newpage

\begin{homeworkProblem}
  Shocks occur to a system according to a Poisson process of rate $\lambda$. Suppose that the system survives each shock with probability $\alpha$, independently of other shocks, so that its probability of surviving $k$ shocks is $\alpha^k$. What is the probability that the system is surviving at time $t$?

  \begin{proof}
    Let $X_t$ denote the number of shocks at time $t$ and $S_t$ be the event that the system survives at time $t$. Then,
    \begin{align*}
      \Pr(S_t)
      &= \sum_{k = 0}^{\infty} \Pr\{S_t \mid X_t = k\}\Pr\{X_t = k\} \\
      &= \sum_{k = 0}^{\infty} \alpha^k \cdot \frac{(t\lambda)^ke^{-t\lambda}}{k!} \\
      &= e^{-t\lambda}\sum_{k = 0}^{\infty} \frac{(\alpha t\lambda)^k}{k!} \\
      &= e^{-t\lambda}e^{-\alpha t\lambda} = e^{(\alpha - 1)t\lambda}.
    \end{align*}
  \end{proof}
\end{homeworkProblem}

\newpage

\begin{homeworkProblem}
  Determine numerical values to three decimal places for $\Pr\{X = k\}$, $k = 0, 1, 2$, when
  \begin{enumerate}[(a)]
    \item $X$ has a binomial distribution with parameters $n = 20$ and $p = 0.06$.
    \begin{proof}
      \[
        \Pr\{X = k\} = {20 \choose k}(0.06)^k(0.94)^{20 - k}.
      \]
      Hence,
      \[
        \Pr\{X = 0\} = 0.290, \quad \Pr\{X = 1\} = 0.370, \quad \Pr\{X = 2\} = 0.225.
      \]
    \end{proof}
    \item $X$ has a binomial distribution with parameters $n = 40$ and $p = 0.03$.
    \begin{proof}
      \[
        \Pr\{X = k\} = {40 \choose k}(0.03)^k(0.97)^{40 - k}.
      \]
      Hence,
      \[
        \Pr\{X = 0\} = 0.296, \quad \Pr\{X = 1\} = 0.366, \quad \Pr\{X = 2\} = 0.221.
      \]
    \end{proof}
    \item $X$ has a Poisson distribution with parameter $\lambda = 1.2$.
    \begin{proof}
      \[
        \Pr\{X = k\} = \frac{(1.2)^ke^{-1.2}}{k!}.
      \]
      Hence,
      \[
        \Pr\{X = 0\} = 0.301, \quad \Pr\{X = 1\} = 0.361, \quad \Pr\{X = 2\} = 0.217.
      \]
    \end{proof}
  \end{enumerate}
\end{homeworkProblem}

\newpage

\begin{homeworkProblem}
  For $i = 1, \ldots, n$, let $\{X_i(t); t \geq 0\}$ be independent Poisson processes, each with the same parameter $\lambda$. Find the distribution of the first time that at least one event has occurred in every process.

  \begin{proof}
    Let $T$ denote the first time that at least one event has occurred in every process. Then, the CDF of the desired distribution is
    \begin{align*}
      \Pr\{T < t\}
      &= \prod_{i = 1}^n \Pr\{X_t \geq 1\} \\
      &= \prod_{i = 1}^n 1 - (t\lambda)^0e^{-t\lambda} \\
      &= \left(1 - e^{-t\lambda}\right)^n
    \end{align*}
  \end{proof}
\end{homeworkProblem}

\newpage

\begin{homeworkProblem}
  Customers arrive at a certain facility according to a Poisson process of rate $\lambda$. Suppose that it is known that five customers arrived in the first hour. Determine the mean total waiting time $E[W_1 + W_2 + \ldots + W_5]$.

  \begin{proof}
    We give each customer a label. Let $U_k$ denote the time customer $k$ arrive at the facility. By Theorem 5.7, $U_k$ is of a uniform distribution. Hence, 
    \[
      E[W_1 + W_2 + \ldots + W_5] = E[U_1 + U_2 + \ldots + U_5] = 5E[U_1] = \frac{5}{2}.
    \]
  \end{proof}
\end{homeworkProblem}

\newpage

\begin{homeworkProblem}
  Let $W_1, W_2, \ldots$ be the event times in a Poisson process $\{X(t); t \geq 0\}$ of rate $\lambda$. Suppose it is known that $X(1) = n$. For $k < n$, what is the conditional density function of $W_1, \ldots, W_{k-1}, W_{k+1}, \ldots, W_n$, given that $W_k = w$?

  \begin{proof}
    \begin{align*}
      f_{W_1, \ldots, W_{k-1}, W_{k+1}, \ldots, W_n|W_k, X_1 = n}&(w_1, \dots, w_n|w) \\
      &= f_{W_1, \ldots, W_{k-1}|X_{w} = k - 1}(w_1, \dots, w_{k - 1})f_{W_1, \ldots, W_{n - k}|X_{1 - w} = n - k}(w_1, \dots, w_{n - k}) \\
      &= \frac{(k - 1)!}{w^{1 - k}} \cdot \frac{(n - k)!}{(1 - w)^{k - n}}
    \end{align*}
  \end{proof}
\end{homeworkProblem}

\newpage

\begin{homeworkProblem}
  Let $W_1, W_2, \ldots$ be the event times in a Poisson process $\{X(t); t \geq 0\}$ of rate $\lambda$, and let $f(w)$ be an arbitrary function. Verify that
  \[
    E\left[\sum_{i=1}^{X(t)} f(W_i)\right] = \lambda \int_{0}^{t} f(w)dw.
  \]

  \begin{proof}
    Let $U_1, \ldots, U_n$ denote independent random variables that are uniformly distributed in $(0, t]$. By Theorem 5.7,
    \begin{align*}
      E\left[\sum_{i=1}^{X(t)} f(W_i)\right]
      &= \sum_{n = 1}^{\infty} E\left[\sum_{i=1}^{n} f(W_i) | X(t) = n\right]\Pr\{X(t) = n\} \\
      &= \sum_{n = 1}^{\infty} E\left[\sum_{i=1}^{n} f(U_i)\right]\Pr\{X(t) = n\} \\
      &= E\left[f(U_1)\right]\sum_{n = 1}^{\infty} n\Pr\{X(t) = n\} \\
      &= t\lambda E\left[f(U_1)\right] \\
      &= t\lambda \int_{0}^t f(w)\Pr\{U_1 = w\} dw \\
      &= \lambda \int_{0}^t f(w) dw.
    \end{align*}
  \end{proof}
\end{homeworkProblem}
\end{document}