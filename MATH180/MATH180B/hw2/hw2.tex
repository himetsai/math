\documentclass{article}

\usepackage{fancyhdr}
\usepackage{extramarks}
\usepackage{amsmath}
\usepackage{amsthm}
\usepackage{amsfonts}
\usepackage{tikz}
\usepackage[plain]{algorithm}
\usepackage{algpseudocode}
\usepackage{enumerate}
\usepackage{amssymb}
\usepackage{dsfont}

\usetikzlibrary{automata,positioning}

%
% Basic Document Settings
%

\topmargin=-0.45in
\evensidemargin=0in
\oddsidemargin=0in
\textwidth=6.5in
\textheight=9.0in
\headsep=0.25in

\linespread{1.1}

\pagestyle{fancy}
\lhead{\hmwkAuthorName}
\chead{\hmwkClass:\ \hmwkTitle}
\rhead{\firstxmark}
\lfoot{\lastxmark}
\cfoot{\thepage}

\renewcommand\headrulewidth{0.4pt}
\renewcommand\footrulewidth{0.4pt}

\setlength\parindent{0pt}
\setlength{\parskip}{5pt}

%
% Create Problem Sections
%

\newcommand{\enterProblemHeader}[1]{
    \nobreak\extramarks{}{Problem \arabic{#1} continued on next page\ldots}\nobreak{}
    \nobreak\extramarks{Problem \arabic{#1} (continued)}{Problem \arabic{#1} continued on next page\ldots}\nobreak{}
}

\newcommand{\exitProblemHeader}[1]{
    \nobreak\extramarks{Problem \arabic{#1} (continued)}{Problem \arabic{#1} continued on next page\ldots}\nobreak{}
    \stepcounter{#1}
    \nobreak\extramarks{Problem \arabic{#1}}{}\nobreak{}
}

\setcounter{secnumdepth}{0}
\newcounter{partCounter}
\newcounter{homeworkProblemCounter}
\setcounter{homeworkProblemCounter}{1}
\nobreak\extramarks{Problem \arabic{homeworkProblemCounter}}{}\nobreak{}

%
% Homework Problem Environment
%
% This environment takes an optional argument. When given, it will adjust the
% problem counter. This is useful for when the problems given for your
% assignment aren't sequential. See the last 3 problems of this template for an
% example.
%
\newenvironment{homeworkProblem}[1][-1]{
    \ifnum#1>0
        \setcounter{homeworkProblemCounter}{#1}
    \fi
    \section{Problem \arabic{homeworkProblemCounter}}
    \setcounter{partCounter}{1}
    \enterProblemHeader{homeworkProblemCounter}
}{
    \exitProblemHeader{homeworkProblemCounter}
}

%
% Homework Details
%   - Title
%   - Due date
%   - Class
%   - Section/Time
%   - Instructor
%   - Author
%

\newcommand{\hmwkTitle}{Homework\ \#2}
\newcommand{\hmwkDueDate}{Jan 26, 2023}
\newcommand{\hmwkClass}{MATH 180B}
\newcommand{\hmwkClassInstructor}{Professor Carfagnini}
\newcommand{\hmwkAuthorName}{\textbf{Ray Tsai}}
\newcommand{\hmwkPID}{A16848188}

%
% Title Page
%

\title{
    \vspace{2in}
    \textmd{\textbf{\hmwkClass:\ \hmwkTitle}}\\
    \normalsize\vspace{0.1in}\small{Due\ on\ \hmwkDueDate\ at 23:59pm}\\
    \vspace{0.1in}\large{\textit{\hmwkClassInstructor}} \\
    \vspace{3in}
}

\author{
  \hmwkAuthorName \\
  \vspace{0.1in}\small\hmwkPID
}
\date{}

\renewcommand{\part}[1]{\textbf{\large Part \Alph{partCounter}}\stepcounter{partCounter}\\}

%
% Various Helper Commands
%

% Useful for algorithms
\newcommand{\alg}[1]{\textsc{\bfseries \footnotesize #1}}

% For derivatives
\newcommand{\deriv}[1]{\frac{\mathrm{d}}{\mathrm{d}x} (#1)}

% For partial derivatives
\newcommand{\pderiv}[2]{\frac{\partial}{\partial #1} (#2)}

% Integral dx
\newcommand{\dx}{\mathrm{d}x}

% Probability commands: Expectation, Variance, Covariance, Bias
\newcommand{\Var}{\mathrm{Var}}
\newcommand{\Cov}{\mathrm{Cov}}
\newcommand{\Bias}{\mathrm{Bias}}
\newcommand*{\Z}{\mathbb{Z}}
\newcommand*{\Q}{\mathbb{Q}}
\newcommand*{\R}{\mathbb{R}}
\newcommand*{\C}{\mathbb{C}}
\newcommand*{\N}{\mathbb{N}}
\newcommand*{\p}{\mathds{P}}
\newcommand*{\E}{\mathds{E}}

\begin{document}

\maketitle

\pagebreak

\begin{homeworkProblem}
  Suppose $U$ and $V$ are independent and follow the geometric distribution
  \[ p(k) = p(1 - p)^k \quad \text{for } k = 0, 1, \ldots. \]
  
  Define the random variable $Z = U + V$.
  \begin{enumerate}[(a)]
      \item Determine the joint probability mass function $p_{U,Z}(u,z) = \p \{U = u, Z = z\}$.
      \begin{proof}
        Since $p_{U,Z}(u,z) = \p \{U = u, Z = z\} = \p \{U = u, V = z - u\}$, we have
        \[
          p_{U,Z}(u,z) = p(1 - p)^up(1 - p)^{z - u} = p^2(1 - p)^{z}.
        \]
      \end{proof}
      \item Determine the conditional probability mass function for $U$ given that $Z = n$.
      \begin{proof}
        Note that $p_Z(z) = \sum_{u = 0}^{z} p_U(u)p_U(z - u) = \sum_{u = 0}^{z} p_U(u)p_v(v) =
        \sum_{u = 0}^{z} p^2(1 - p)^{z} = (z + 1)p^2(1 - p)^{z}$. Thus,
        \begin{align*}
          p_{U|Z}(u|n)
          &= \frac{p_{U,Z}(u,n)}{p_Z(n)} \\
          &= \frac{p^2(1 - p)^{n}}{(n + 1)p^2(1 - p)^{n}} \\
          &= \frac{1}{n + 1}.
        \end{align*}
      \end{proof}
  \end{enumerate}  
\end{homeworkProblem}

\newpage

\begin{homeworkProblem}
  Let $N$ have a Poisson distribution with parameter $\lambda = 1$. Conditioned on $N = n$, let $X$
  have a uniform distribution over the integers $0, 1, \ldots, n+1$. What is the marginal
  distribution for $X$?

  \begin{proof}
    \begin{align*}
      p_X(x) 
      &= \sum_{n = 0}^{\infty} p_{X|N}(x|n)p_N(n) \\
      &= \sum_{n = x - 1}^{\infty} \frac{1}{n + 2} \cdot \frac{1}{n!}e^{-1} \\
      &= e^{-1}\sum_{n = x - 1}^{\infty} \frac{1}{(n + 2)n!} \\
      &= e^{-1}\sum_{n = x - 1}^{\infty} \frac{1}{(n + 1)!} - \frac{1}{(n + 2)!} \\
      &= e^{-1}\left(\sum_{n = x - 2}^{\infty} \frac{1}{(n + 2)!} - \sum_{n = x - 1}^{\infty} \frac{1}{(n + 2)!}\right) \\
      &= e^{-1}\left(\frac{1}{x!} + \sum_{n = x - 1}^{\infty} \frac{1}{(n + 2)!} - \sum_{n = x - 1}^{\infty} \frac{1}{(n + 2)!}\right) \\
      &= \frac{1}{e(x!)}.
    \end{align*}
  \end{proof}
\end{homeworkProblem}

\newpage

\begin{homeworkProblem}
  Suppose that upon striking a plate a single electron is transformed into a number $N$ of
  electrons, where $N$ is a random variable with mean $\mu$ and standard deviation $\sigma$. Suppose
  that each of these electrons strikes a second plate and releases further electrons, independently
  of each other and each with the same probability distribution as $N$. Let $Z$ be the total number
  of electrons emitted from the second plate. Determine the mean and variance of $Z$.

  \begin{proof}
    Note that $Z = \xi_1 + \dots + \xi_N$, where $\xi_k$ is the number of electrons struck by the
    $k$th electron from the first plate. $\xi_k$ shares the same distribution as $N$, for all $k$.
    Thus, $$\E[Z] = \E[N]\E[N] = \mu^2,$$ $$Var(Z) = \E[N]Var(N) + \E[N]^2Var(N) = \mu\sigma^2 +
    \mu^2\sigma^2 = \mu(\mu + 1)\sigma^2.$$
  \end{proof}
\end{homeworkProblem}

\newpage

\begin{homeworkProblem}
  A six-sided die is rolled, and the number $N$ on the uppermost face is recorded. From a jar
  containing 10 tags numbered $1, 2, \ldots, 10$ we then select $N$ tags at random without
  replacement. Let $X$ be the smallest number on the drawn tags. Determine $\p(X = 2)$ and $\E[X]$.

  \begin{proof}
    Note that $\p(X = x| N = n) = \frac{{10 - x \choose n - 1}}{{10 \choose n}}$. Thus,
    \begin{align*}
      \p(X = 2) 
      &= \sum_{n = 1}^6 \p(X = 2 | N = n)\p(N = n) \\
      &= \frac{1}{6}\sum_{n = 1}^6 \p(X = 2 | N = n) \\
      &= \frac{1}{6}\sum_{n = 1}^6 \frac{{8 \choose n - 1}}{{10 \choose n}} \\
      &= \frac{1}{6}\sum_{n = 1}^6 \frac{n}{9} - \frac{n^2}{90} \\
      &= \frac{119}{540},
    \end{align*}
    and
    \begin{align*}
      \E[X]
      &= \E[\E[X | N]] \\
      &= \E\left[\sum_{x = 1}^{11 - N} x\p(X = x| N)\right] \\
      &= \E\left[\sum_{x = 1}^{11 - N} \frac{x{10 - x \choose N - 1}}{{10 \choose N}}\right] \\
      &= \E\left[\frac{1}{{10 \choose N}}\sum_{x = 1}^{11 - N} {x \choose 1}{10 - x \choose N - 1}\right] \\
      &= \E\left[\frac{11}{N + 1}\right] \approx 2.92024
    \end{align*}
  \end{proof}
\end{homeworkProblem}

\newpage

\begin{homeworkProblem}
  Suppose that $\xi_1, \xi_2, \ldots$ are independent and identically distributed with $\p \{\xi_k =
  \pm 1\} = \frac{1}{2}$. Let $N$ be independent of $\xi_1, \xi_2, \ldots$ and follow the geometric
  probability mass function

  \[ p_N(k) = \alpha(1 - \alpha)^k \quad \text{for } k = 0, 1, \ldots, \]

  where $0 < \alpha < 1$. Form the random sum $Z = \xi_1 + \ldots + \xi_N$.
  \begin{enumerate}[(a)]
      \item Determine the mean and variance of $Z$.
      \begin{proof}
        Since the $\E[N] = \frac{1 - \alpha}{\alpha}$ and $Var(N) = \frac{1 - \alpha}{\alpha^2}$, 
        \[
          \E[Z] = \E[N]\E[\xi_1] = 0, \quad Var(Z) = \E[N]Var(\xi_1) + \E[\xi_1]^2Var(N) = \frac{1}{\alpha}.
        \]
      \end{proof}
      \item Evaluate the higher moments $m_3 = \E[Z^3]$ and $m_4 = \E[Z^4]$.
      \begin{proof}
        Note that $\xi_i^2 = 1$ and $\E[\xi_i] = \E[\xi_i\xi_j] = 0$. Hence,
        \[
          \E[\xi_i\xi_j\xi_k] = \E[\xi_i\xi_j\xi_k\xi_m] = \E[\xi_i^2\xi_j\xi_k] = \E[\xi_i^3\xi_j] = 0,
        \]
        so we only need to care about $\E[\xi_i^2\xi_j^2]$ and $\E[\xi_i^4]$.
        We thus get
        \begin{align*}
          \E[Z^3] 
          &= \E\left[\E\left[\sum_i\sum_j\sum_k \xi_i\xi_j\xi_k | N\right]\right] \\
          &= \E\left[N(N - 1)(N - 2)\E[\xi_i\xi_j\xi_k] + N(3N - 2)\E[\xi_i]\right] = 0, \\
          \E[Z^4]
          &= \E\left[\sum_i\sum_j\sum_k\sum_m \E[\xi_i\xi_j\xi_k\xi_m] | N\right] \\
          &= \E\left[3N(N - 1)\E[\xi_i^2\xi_j^2] + N\E[\xi_i^4]\right] \\
          &= \E\left[(3N^2 - 2N)\E[1]\right] \\
          &= \E\left[3N^2 - 2N\right] = 3(Var(N) + \E[N]^2) - 2\E[N] = \frac{6 - 5\alpha}{\alpha^2}.
        \end{align*}
      \end{proof}
  \end{enumerate}
\end{homeworkProblem}

\newpage

\begin{homeworkProblem}
  To form a slightly different random sum, let $\xi_0, \xi_1, \ldots$ be independent identically distributed random variables and let $N$ be a nonnegative integer-valued random variable, independent of $\xi_0, \xi_1, \ldots$. The first two moments are

  \[ \E[\xi_k] = \mu, \quad Var[\xi_k] = \sigma^2, \]
  \[ \E[N] = \nu, \quad Var[N] = \tau^2. \]

  Determine the mean and variance of the random sum $Z = \xi_0 + \ldots + \xi_N$.

  \begin{proof}
    \[
      \E[Z] = \E[\xi_k](\E[N] + 1) = \mu(\nu + 1),
    \]
    \[
      Var(Z) = (\E[N] + 1)Var(\xi_k) + \E[\xi_k]^2Var(N) = (\nu + 1)\sigma^2 + \mu^2\tau^2.
    \]
  \end{proof}
\end{homeworkProblem}
\end{document}