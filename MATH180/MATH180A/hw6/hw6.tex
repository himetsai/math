\documentclass[addpoints, 11pt]{exam}
\setlength{\headsep}{0.25in}
\setlength{\unitlength}{1in}
%
\pagestyle{head}
%
\usepackage[utf8]{inputenc} %use Unicode
\usepackage[T1]{fontenc} %European fonts

\usepackage{%
	amsmath,       %some math tools
	amssymb,       %math symbols
	graphicx,      %enhanced graphics options
	mathtools,     %extension of amsmath
	microtype,     %small typographic effects
	bm,            %bold math symbols
	% todonotes,   %adds the option \todo{...} (use fixme instead)
	stmaryrd,      %some more math symbolswork
	% nicematrix,    %nicer matrix controls
	mathrsfs,      %more math fonts
        dsfont,
}
\usepackage{url}
\usepackage{hyperref}
%
\usepackage{amsthm}
% \newtheorem*{thm11.1.7}{Theorem 11.1.7}
%
\usepackage[shortlabels]{enumitem}
\usepackage{nicematrix}
\usepackage{multicol}
%
\usepackage[normalem]{ulem}
%
\newcommand{\myCourseNumber}{Math 180A}
\newcommand{\myName}{Ray Tsai}
\newcommand{\myID}{A16848188}
\newcommand{\myProfessor}{Professor Carfagnini}
\newcommand{\myHmwkNumber}{6}
% \newcommand{\myExamVersion}{}

\newcommand*{\Z}{\mathbb{Z}}
\newcommand*{\Q}{\mathbb{Q}}
\newcommand*{\R}{\mathbb{R}}
\newcommand*{\C}{\mathbb{C}}
\newcommand*{\N}{\mathbb{N}}
\newcommand*{\prob}{\mathds{P}}
\newcommand*{\E}{\mathds{E}}

\definecolor{crimson}{rgb}{0.86,0.08,0.24}

\newenvironment{question}[1]{\smallskip\noindent\color{crimson}{\bf Question #1.}}{}
\allowdisplaybreaks[1]

%% define absolute value \abs{...}:
\DeclarePairedDelimiterX\abs[1]\lvert\rvert{%
  \ifblank{#1}{\:\cdot\:}{#1}
}
% define norm \norm{...}:
\DeclarePairedDelimiterX\norm[1]\lVert\rVert{%
  \ifblank{#1}{\:\cdot\:}{#1}
}
% define inner product \inner{...}{...}:
\DeclarePairedDelimiterX{\inner}[2]{\langle}{\rangle}{%
  \ifblank{#1}{\:\cdot\:}{#1},\ifblank{#2}{\:\cdot\:}{#2}
}

% define \set{...} to write sets and \given to write \set{... \given ...} for {...|...}
\newcommand*\setSymbol[1][]{
  \nonscript\:#1\vert\allowbreak\nonscript\:\mathopen{}
}
\providecommand\given{}
\DeclarePairedDelimiterX\set[1]{\lbrace}{\rbrace}{
  \renewcommand*\given{\setSymbol[\delimsize]}
  #1
}

% free group geneated by ... \free{...} or \free{... \given ...}
\DeclarePairedDelimiterX\free[1]{\langle}{\rangle}{
  \renewcommand\given{\nonscript\:\delimsize\vert\nonscript\:
    \mathopen{}}
  #1}

% define \lopen{...}{...}, \ropen{...}{...}, \open{...}{...}, \closed{...}{...} for intervals
\DeclarePairedDelimiterX\open[2](){#1,#2}
\DeclarePairedDelimiterX\lopen[2](]{#1,#2}
\DeclarePairedDelimiterX\ropen[2][){#1,#2}
\DeclarePairedDelimiterX\closed[2][]{#1,#2}

\NiceMatrixOptions{cell-space-limits = 1pt}
\newcommand*{\pmat}[1]{\begin{pNiceMatrix} #1 \end{pNiceMatrix}}
\newcommand*{\dfdx}[2]{\frac{\partial #1}{\partial #2}}

\DeclareMathOperator{\vol}{vol}
%
\pointsinmargin
\pointpoints{\thinspace point}{points}
\marginpointname{ \points}
%
\begin{document}
%
\firstpageheader{\bfseries \myCourseNumber}{\bfseries Homework \myHmwkNumber}{\bfseries \myName \\ \myID \\ \myProfessor}
%
\runningheader{}{(page \textit{\thepage}\ of \textit{\numpages})}{}
%
%

\begin{question}{1}
    Suppose that a class of students is star-gazing on top of the local mathematics building from the hours of 11 PM through 3 AM. Suppose further that meteors arrive (i.e. they are seen) according to a Poisson process with intensity $\lambda = 4$ per hour. Find the following.
\end{question}

\begin{enumerate}[(a)]
    \color{crimson}
    \item The probability that the students see more than 2 meteors in the first hour.
    \normalcolor
    
    \begin{proof}[Solution]
        Let $X$ be the number of meteors seen in the first hour, so $X \sim \text{Poisson}(4)$. Then,
        \begin{align*}
            \prob(X > 2)
            &= 1 - \prob(X \leq 2) \\
            &= 1 - e^{-4}\left(1 + 4 + 8\right) \\
            &= 1 - \frac{13}{e^{4}}.
        \end{align*}
    \end{proof}

    \color{crimson}
    \item  The probability that they see zero meteors in the first hour, but at least ten meteors in the final three hours.
    \normalcolor
    
    \begin{proof}[Solution]
        Let $X_1$ be the number of meteors seen in the first hour and $X_2$ be the number of meteors seen in the last 3 hours, so $X_1 \sim \text{Poisson}(4)$ and $X_1 \sim \text{Poisson}(12)$. We know,
        \begin{align*}
            \prob(X_1 = 0) &= e^{-4}, \\
            \prob(X_2 \geq 10)
            &= 1 - \prob(X \leq 9) \\
            &= 1 - e^{-12}\sum_{k = 0}^{9} \frac{12^k}{k!},
        \end{align*}
        and $X_1, X_2$ are independent to each other. Thus, \[
            \prob(X_1 = 0, X_2 \geq 10) = \prob(X_1 = 0)\prob(X_2 \geq 10) = e^{-4} - e^{-16}\sum_{k = 0}^{9} \frac{12^k}{k!}.
        \]
    \end{proof}

    \color{crimson}
    \item Given that there were 13 meteors seen all night, what is the probability there were no meteors seen in the first hour?
    \normalcolor
    
    \begin{proof}[Solution]
        Let $X_1$ be the number of meteors seen in the first hour, $X_2$ be the number of meteors seen in the last 3 hours, and $X$ be the number of meteors seen all night, so $X_1 \sim \text{Poisson}(4)$, $X_2 \sim \text{Poisson}(12)$, and $X \sim \text{Poisson}(16)$. Then,
        \begin{align*}
            \prob(X_1 = 0 | X = 13)
            &= \frac{\prob(X_1 = 0, X = 13)}{\prob(X = 13)} \\
            &= \frac{\prob(X_1 = 0)\prob(X_2 = 13)}{\prob(X = 13)} \\
            &= \frac{e^{-4}e^{-12}\cdot\frac{12^{13}}{13!}}{e^{-16}\frac{16^{13}}{13!}} \\
            &= \left(\frac{3}{4}\right)^{13}.
        \end{align*}
    \end{proof}
\end{enumerate}

\newpage

\begin{question}{2}
    Let $\{N_t\}_{t>0}$ be a Poisson process with rate $\lambda$, that is, for each $t > 0$, $N_t = N((0, t]) \sim Pois(t\lambda)$. Let $X_1$ be the first arrival time of $N_t$, that is the first time a customer arrives (a car passes by etc). Show that
    \[
        \prob(X_1 \leq x \, | \, N(t) = 1) = \frac{x}{t},
    \]
    for $0 \leq x \leq t$. That is, show that given $N(t) = 1$, then $X_1$ is uniformly distributed in $(0, t]$.
\end{question}

\begin{proof}
    \begin{align*}
        \prob(X_1 \leq x \, | \, N(t) = 1)
        &= \frac{\prob(X_1 \leq x, N(t) = 1)}{\prob(N(t) = 1)} \\
        &= \frac{\prob(N([0,x]) = 1)\prob(N((x,t]) = 0)}{\prob(N(t) = 1)} \\
        &= \frac{\lambda xe^{-\lambda x}e^{\lambda(x - t)}}{\lambda te^{-\lambda t}} \\
        &= \frac{xe^{\lambda(-x + x - t)}}{te^{-\lambda t}} \\
        &= \frac{x}{t}.
    \end{align*}
\end{proof}

\newpage

\begin{question}{3}
     Suppose that the random variable X has a density function
     \[
        f(x) \begin{cases}
            \frac{1}{2}x^2e^{-x} & x \geq 0 \\
            0 & x < 0
        \end{cases}.
     \]
     Find the moment generating function $M(t)$ of $X$.
\end{question}

\begin{proof}[Solution]
    \begin{align*}
        M(t)
        &= \E[e^{tX}] \\
        &= \int^{\infty}_{-\infty} e^{tx}f(x) dx \\
        &= \int^{\infty}_{0} \frac{1}{2}x^2e^{(t-1)x} dx.
    \end{align*}
    For $t = 1$,
    \[
        \int^{\infty}_{0} \frac{1}{2}x^2e^{(t-1)x} dx
        = \int^{\infty}_{0} \frac{1}{2}x^2 dx \rightarrow \infty.
    \]
    For $t \neq 1$,
    \begin{align*}
        \int^{\infty}_{0} \frac{1}{2}x^2e^{(t-1)x} dx
        &= \left.\left(\frac{x^2}{2(t - 1)} - \frac{x}{(t - 1)^2} + \frac{1}{(t - 1)^3} \right)e^{(t - 1)x}\right|^{\infty}_0
    \end{align*}
    Since
    \begin{align*}
        \lim_{x\to\infty} \left(\frac{x^2}{2(t - 1)} - \frac{x}{(t - 1)^2} + \frac{1}{(t - 1)^3} \right)e^{(t - 1)x}
        &= \lim_{x\to\infty} \frac{\frac{x^2}{2(t - 1)} - \frac{x}{(t - 1)^2} + \frac{1}{(t - 1)^3}}{e^{(1 - t)x}} \\
        &= \lim_{x\to\infty} \frac{\frac{x}{(t - 1)} - \frac{1}{(t - 1)^2}}{(1 - t)e^{(1 - t)x}} \\
        &= \lim_{x\to\infty} \frac{e^{(t - 1)x}}{(t - 1)^3},
    \end{align*}
    we have $\lim_{x\to\infty} \frac{e^{(t - 1)x}}{(t - 1)^3} \rightarrow \infty$, for $t > 1$, and $\lim_{x\to\infty} \frac{e^{(t - 1)x}}{(t - 1)^3} = 0$, for $t < 1$. Therefore, 
    \[
        M(t) = \begin{cases}
            \infty & t \geq 1 \\
            \frac{1}{(1 - t)^3} & t < 1
        \end{cases}.
    \]
\end{proof}

\newpage

\begin{question}{4}
    Let $X$ be a random variable with moment generating function $M_X(t)$. Let us consider a new random variable $Y = aX + b$, for some real numbers $a, b$. Write the moment-generating function $M_Y(t)$ of $Y$ in terms of $M_X(t)$.
\end{question}

\begin{proof}[Solution]
    \begin{align*}
        M_Y(t)
        &= \E[e^{tY}] \\
        &= \E[e^{atX + bt}] \\
        &= e^{bt}\E[e^{atX}] \\
        &= e^{bt}M_X(at).
    \end{align*}
\end{proof}

\newpage

\begin{question}{5}
    Suppose that $U \sim Unif[0, 1]$. Let $Y = e^{\frac{U}{1 - U}}$. Find the probability density function of $Y$.
\end{question}

\begin{proof}[Solution]
    Let $F_X$ be the cumulative distribution function of $U$, and let $f_Y$ be the probability density function of $Y$. Since $U \sim Unif[0, 1]$, $F_X(x) = \prob(U \leq x) = x$, for $x \in [0, 1]$. Since $e^{\frac{U}{1 - U}} > 0$, $Y$ only takes positive values, and so $F_Y(t) = 0$ for $t \leq 0$. Therefore, for $t > 0$, the cumulative distribution function of $Y$ is
    \begin{align*}
        F_Y(t)
        &= \prob(Y \leq t) \\
        &= \prob(e^{\frac{U}{1 - U}} \leq t) \\
        &= \prob\left(\frac{U}{1 - U} \leq \ln{t}\right) \\
        &= \prob\left(\frac{1}{U} - 1 \geq \frac{1}{\ln{t}}\right) \\
        &= \prob\left(U \leq \frac{\ln{t}}{\ln{t} + 1}\right).
    \end{align*}
    For $t < 1$, $\frac{\ln{t}}{\ln{t} + 1} \notin [0, 1]$, and so $f_Y(t) = 0$. For $t \geq 1$, $F_Y(t) = \prob\left(U \leq \frac{\ln{t}}{\ln{t} + 1}\right) = \frac{\ln{t}}{\ln{t} + 1}$, and so $f_Y(t) = F'_Y(t) = \frac{1}{t(\ln{t} + 1)^2}$. Therefore,
    \[
        f_Y(t) = \begin{cases}
            \frac{1}{t(\ln{t} + 1)^2} & t \geq 1 \\
            0 & t < 1
        \end{cases}.
    \]
\end{proof}

\newpage

\begin{question}{6}
    Let $X$ be a random variable with moment generating 
    \[
        M_X(t) = e^{2(e^{2t} - 1)}.
    \]
    Compute $\E[X^3]$.
\end{question}

\begin{proof}[Solution]
    \begin{align*}
        M_X(t) &= e^{2(e^{2t} - 1)}, \\
        M'_X(t) &= 4e^{2(e^{2t} + t - 1)}, \\
        M''_X(t) &= 8(2e^{2t} + 1)e^{2(e^{2t} + t - 1)}, \\
        M'''_X(t) &= 16(4e^{6t} + 6e^{4t} + e^{2t})e^{2e^{2t} - 2}.
    \end{align*}
    Thus, 
    \[
        \E[X^3] =  M'''_X(0) = 176.
    \]
\end{proof}

\newpage

\begin{question}{7}
    Suppose that $X$ is uniform on $[-2, 3]$ and let $Y = |X - 1|$. Find the density function of $Y$.
\end{question}

\begin{proof}[Solution]
    We first note that $Y$ only takes positive values. Thus, for $t < 0$, $\prob(Y < t) = 0$. For $t > 3$, since $[-2, 3] \subset [1 - t, 1 + t]$, $\prob(Y \leq t) = \prob(X \in [1 - t, 1 + t]) = 1$. For $t \leq 2$, $\prob(Y \leq t) = \prob(X \in [1 - t, 1 + t]) = \frac{2t}{5}$. For $2 < t \leq 3$, $\prob(Y \leq t) = \prob(X \in [1 - t, 1 + t]) = \frac{t + 2}{5}$. Thus, the cumulative distribution function of $Y$ is 
    \[
        F_Y(t) = \begin{cases}
            0 & t < 0 \\
            \frac{2t}{5} & 0 \leq t \leq 2 \\
            \frac{t + 2}{5} & 2 < t \leq 3 \\
            1 & t > 3
        \end{cases}.
    \]
    Therefore, the density function of $Y$ is 
    \[
        f_Y(t) = F'_Y(t) = \begin{cases}
            \frac{2}{5} & 0 \leq t \leq 2 \\
            \frac{1}{5} & 2 < t \leq 3 \\ 
            0 & \text{otherwise}
        \end{cases}.
    \]
\end{proof}

\end{document}
