\documentclass[a4paper]{article}

\usepackage[english]{babel}
\usepackage[utf8]{inputenc}
\usepackage{amsmath}
\usepackage{amsthm}
\usepackage{amsfonts}
\usepackage{graphicx}
\usepackage{xcolor}
\usepackage{enumerate}
\usepackage[colorinlistoftodos]{todonotes}

\graphicspath{ {./images} }

\newtheorem{theorem}{Theorem}[section]
\newtheorem{corollary}{Corollary}[theorem]
\newtheorem{lemma}[theorem]{Lemma}

\title{\textsc{General regularity}}

\author{Russell Impagliazzo}

\date{}

\begin{document}

\maketitle

The energy boosting algorithm was ``reverse engineered'' from proofs of the Szemeredi Regularity
Lemma.  Here, we will show how the algorithm gives a very general version of this lemma, and then
show how to improve it for sparse and pseudo-random structures using the reduction to the dense
model theorem.  

Regularity expresses the intuition that a biased function is otherwise random-looking. We say that
$f$ on $U$ is $\rho$-regular with respect to a class of Boolean functions $H$ if for every $b_1,b_2,
h \in H$, $|Pr_{x \in U} [ f(x)=b_1 \land h(x)=b_2] - Prob[f(x)=b_1]*Prob[h(x)=b_2] | \le \rho$.

The following lemma gives a variety of characterizations of regularity up to universal
multiplicative constants. For a class of functions $H$ on universe $U$,  let $U^{+bit} = U \times
\{-1,1\}$, where we think of the last bit as being chosen uniformly, and $H^{+ bit}$ be the class of
functions of the form $h(x,b)= h_b (x)$ for some $h_1, h_{-1} \in H$.  
Let $b_U$ be the majority bit of $f$ on $U$, let $\delta = Prob_x [f(x) = - b_U]$, and as in the
energy boosting algorithm at the start, let $\mu(x) = \delta/(1-\delta)$ if $f(x) = b_U$, and $1$
otherwise.  Let $D_{mu}$ be the corresponding distribution, e.g., pick $b$ at random, then pick a
random element of $f^{-1}(b)$.  

\begin{lemma}
Let $\rho > 0, \rho_1 = 2 \rho, \rho_2 =  2 \rho_1, \rho_3 = \rho_2, \rho_4 =1/2 \rho_3, \rho_5=
\rho_4 = 2 \rho$ Then each of the following implies the next:
\begin{enumerate}
\item $f$ is $\rho$-regular on $U$ for $H$
\item $(x, f(x))$ is $\rho_1$-indistinguishable from $(x,b)$ for $b$ an independent coin with
probability $1-\delta$ of being $b_U$, with respect to $H^{+bit}$.  
\item $f^{-1} (b_U)$ is $\rho_2$-indistinguishable from $U$ for $H$.
\item $f^{-1} (b_U)$ is $\rho_3/\delta$-indistinguishable from $f^{-1} (-b_U)$ for $H$
\item $f$ is $\rho_4/\delta$-hard-core on $D_{\mu}$ for $H$.
\item $f$ is $\rho_5$ -regular.

\end{enumerate}
\end{lemma}
\begin{proof}
\begin{description}
\item[$1 \implies  2$]
Assume $f$ is $\rho$ regular on $U$ for $H$.  
Let $h_{-1}, h_1 \in H$, and $h(x,b)= h_b (x)$. Let $b$ have probability $1- \delta$ of being $b_U$
and otherwise $-b_U$. Then \( |Prob[ h((x,f(x))=1] - Prob[h(x,b)=1] |= | Prob[f(x)=1] Prob[h_{1}(x)
=1 | f(x)=1] + Prob[f(x)=-1] Prob[h_{-1}(x) = 1 | f(x) = -1] - Prob[f(x)=1] Prob[h_1(x) =1] -
Prob[f(x)=-1] Prob[h_{-1} (x)=1] | \le | Prob[f(x)=1 \land h_1 (x)=1] - Prob[f(x)=1] Prob[h_1(x)
=1]| + | Prob[f(x)=-1 \land h_{-1}(x) = 1 ] - Prob[f(x)=-1] Prob[h_{-1} (x)=1] | \le 2 \rho = \rho_1
\).

\item[$2 \implies 3$]
For $h_1 \in H$, consider the function $h ((x,b))=  (b=b_U) \land h(x)=1$. Then $|Prob [h_1(x) =1 |
f(x) = b_U] - Prob[h_1(x) =1] | =  
|Prob[h((x,f(x))=1]/Prob[f(x)=b_U] - Prob[h_1(x)=1] | = 1/Prob[f(x)=b_U] |Prob[h(x,f(x))=1] -
Prob[h(x,b)=1] \le 2 rho_1= \rho_2$, since $Prob[f(x)=b_U] \ge 1/2$.  

\item[$3 \implies 4$]
$\delta * |Prob[h(x) = 1 | f(x) = b_U] - Prob[ h(x) =1 | f(x) = - b_U] | = |\delta Prob[h(x) = 1 |
f(x) = b_U] - \delta Prob[ h(x) =1 | f(x) = - b_U] + Prob[h(x)=1]-Prob[h(x)=1] | =  
|\delta Prob[h(x) = 1 | f(x) = b_U] - \delta Prob[ h(x) =1 | f(x) = - b_U] + (1-\delta) Prob[h(x)=1
| f(x)=b_U] + \delta Prob[h(x)=1 | f(x)=-b_U] ] -Prob[h(x)=1] | =  |Prob[h(x)=1 | f(x)=b_U ]
-Prob[h(x)=1]  \le rho_2$; the claim then follows by dividing through by $\delta$.  

\item[$4 \implies 5$]
$|Prob_{x \in_{D_\mu} U} [h(x)=f(x)] -1/2 | = |1/2 Prob[h(x)= b_U | f(x)=b_U] +1/2 [Prob[h(x)=-b_U |
f(x) = -b_U] -1/2 = | 1/2 Prob[h(x)=b_U | f(x)= b_U] +1/2 (1 - Prob[h(x)=b_U | f(x) = -b_U]) -1/2| =
1/2 (Prob[h(x)=b_U | f(x)= b_U]- Prob[h(x)=b_U | f(x)= -b_U]) \le 1/2 \rho_3/\delta$.
\item[$5 \implies 6$]
$|Prob[ h(x)=b_1 \land f(x)=b_2] - Prob[ h(x) =b_1] Prob[f(x)=b_2] | = |Prob [h(x)= b_1
|f(x)=b_2]Prob[f(x)=b_2]  - Prob[h(x)=b_1 ] Prob[f(x)=b_2]|= Prob[f(x)=b_2]  |Prob[h(x)=b_1 |
f(x)=b_2] - Prob[h(x)= b_1]| = Prob[f(x)=b_2] |Prob[h(x)=b_1 | f(x) =b_2] - (Prob[f(x)=b_2])
Prob[h(x)=b_1 | f(x) =b_2] + (1-Prob[f(x) = b_2]) Prob[h(x)=b_1 | f(x) = -b_2]| = Prob[f(x)=b_2] (1
- Prob[f(x)=b_2) | Prob[h(x)=b_1 | f(x)=b_2 ] - Prob[h(x)=b_1|f(x)= -b_2] \le \delta (1 - \delta)
(\rho_4/\delta) < \rho_4 $.
\end{description}
\end{proof}

\end{document}