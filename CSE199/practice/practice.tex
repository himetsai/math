\documentclass[a4paper]{article}

\usepackage[english]{babel}
\usepackage[utf8]{inputenc}
\usepackage{amsmath}
\usepackage{amsthm}
\usepackage{amsfonts}
\usepackage{graphicx}
\usepackage{xcolor}
\usepackage{enumerate}
\usepackage[colorinlistoftodos]{todonotes}

\graphicspath{ {./images} }

\newtheorem{theorem}{Theorem}[section]
\newtheorem{corollary}{Corollary}[theorem]
\newtheorem{lemma}[theorem]{Lemma}

\title{\textsc{Complexity Theory Exercies}}

\author{Ray Tsai}

\date{}

\begin{document}

\maketitle

\section*{Diagonolization}

\subsection*{Problem 3.1} 

Prove that $\textbf{SPACE}(n) \neq \textbf{NP}$.

\begin{proof}
  We prove that \textbf{NP} is closed under log-space reduction but $\textbf{SPACE}(n)$ is not. 

  Given a log-space reduction from $L_1$ to $L_2$, there is a log-space turing machine $R$ which
  performs the reduction. Since $\textbf{SPACE}(\log n) \subset P$, both the runtime and the output
  length of $R$ is bounded by a polynomial. If $L_2 \in \textbf{NP}$, then there exists a
  polynomial-time non-deterministic turning machine $M$ which decides $L_2$. Hence, a
  non-deterministic turning machine which runs $R$ then $M$ decides $L_1$ in polynomial-time, and so
  $L_1 \in \textbf{NP}$. 

  We now show that $\textbf{SPACE}(n)$ is not closed under log-space reduction. Let $R$ be a
  reduction which pads $n^2 - n$ $1$'s after an input of length $n$. Note that $R$ uses $O(\log n)$
  space. Pick $L_1 \in \textbf{SPACE}(n^2) \backslash \textbf{SPACE}(n)$, and let $L_2$ be the
  padded version of $L_1$. Consider a turing machine $M$ which checks the (1) input is of a square
  number length, say $|x| = n^2$, (2) checks the first $n$ symbols are in $L_1$, and (3) checks the
  remaining $n^2 - n$ symbols are all $1$'s. $M$ decides $L_2$. Since step (1) and (3) takes $O(\log
  n)$ space, and (2) takes $O(|x|) = O(n^2)$ space, $M$ only uses linear space. But then $L_2 \in
  \textbf{SPACE}(n)$ and $L_1 \notin \textbf{SPACE}(n)$. 
\end{proof}

\section*{Polynomial Hierarchy}

\subsection*{Problem 5.3}

Show that if \texttt{3SAT} is polynomial-time reducible to $\overline{\texttt{3SAT}}$, then
\textbf{PH} = \textbf{NP}.

\begin{proof}
  We show that $\Sigma_i^p, \Pi_i^p \subseteq NP$ for all $i \geq 1$, by induction on $i$. Since
  \texttt{3SAT} is \textbf{NP}-complete, $\textbf{NP} \subseteq \textbf{coNP}$ by assumption. Let $L
  \in \textbf{coNP}$. Since $\texttt{3SAT} \in \textbf{coNP}$, $\overline{\texttt{3SAT}} \in
  \textbf{NP}$. But then $\overline{\texttt{3SAT}}$ is $\textbf{coNP}$-complete, and thus
  $\textbf{coNP} \subseteq \textbf{NP}$. Hence, $\Pi_1^p = \textbf{NP}$. Suppose $i \geq 2$. Let $L
  \in \Sigma_i^p$. There exists a polynomial-time turing maching $M$ and a polynomial $q$ such that
  $x \in L$ if and only if
  \[
    \exists u_1 \in \{0, 1\}^{q(|x|)} \forall u_2 \in \{0, 1\}^{q(|x|)} \cdots Q_iu_i \in \{0, 1\}^{q(|x|)}, M(x, u_1, u_2, \ldots, u_i) = 1.
  \]
  Define language $L'$ such that $\langle x, u_1 \rangle \in L'$ if and only if
  \[
    \forall u_2 \in \{0, 1\}^{q(|x|)} \cdots Q_iu_i \in \{0, 1\}^{q(|x|)}, M(x, u_1, u_2, \ldots, u_i) = 1.
  \]
  By induction, $L' \in \Pi_{i - 1}^p \subseteq \textbf{NP}$, so there exists a polynomial-time
  turing machine $M'$ which verifies $L'$. Combining it into $L$, we get $x \in L$ if and only if
  \[
    \exists (u_1, u_2) \in \{0, 1\}^{2q(|x|)}, M'(x, u_1, u_2) = 1.
  \]
  Hence, $L \in \textbf{NP}$. By the same argument, we may also show that $\Pi_i^p \subseteq
  \textbf{coNP}$. But then by the base case, $\textbf{coNP} = \textbf{NP}$, and this completes the
  induction. 
\end{proof}

\subsection*{Problem 5.11}

Show that $\texttt{SUCCINCT SET-COVER} \in \Sigma_2^p$. 

\begin{proof}
  Let $S = \{\varphi_1, \ldots, \varphi_m\}$ be a set of 3-DNF formulae on $n$ variables $V = \{v_1,
  \ldots, v_n\}$. $\langle S, k \rangle \in \texttt{SUCCINCT SET-COVER}$ if and only if there exists
  $I \subseteq [m]$ such that $|I| \leq k$ and for all assignments to $\bigvee_{i \in I} \varphi_i$
  results in $1$. Hence, there exists a turing machine $M$ such that $\langle S, k \rangle \in
  \texttt{SUCCINCT SET-COVER}$ if and only if
  \[
    \exists I \subseteq [m] \; \forall f: V \to \{0, 1\}, M(S, k, I, f) = 1.
  \]
  Since calculating each $\varphi_i$ with assignment $f$ takes polynomial time and there are at most
  $m$ of them, $M$ runs in polynomial time. Hence, $\texttt{SUCCINCT SET-COVER} \in \Sigma_2^p$. 
\end{proof}

\subsection*{Problem 5.13}

This problem studies the Vapnik-Chervonenkis (VC) dimensions, an important concept in machine
learning. If $\mathcal{S} = \{S_1, S_2, \ldots, S_m\}$ is a collection of subsets of a finite set
$U$, the \textit{VC dimension} of $\mathcal{S}$, denoted $VC(S)$, is the size of the largest set $X
\subseteq U$ such that for every $X' \subseteq X$, there is an $i$ for which $S_i \cap X = X'$. (We
say that $X$ is shattered by $S$.)

A Boolean circuit $C$ succinctly represents collection $\mathcal{S}$ if $S_i$ consists of exactly
those elements $x \in U$ for which $C(i, x) = 1$. Finally, the
\[
  \text{VC-DIMENSION} = \{\langle C, k \rangle : C \text{ represents a collection } \mathcal{S} \text{ s.t. } V(S) \geq k\}
\]
\begin{enumerate}[(a)]
  \item Show that VC-DIMENSION $\in \Sigma_3^p$.
  \item Show that VC-DIMENSION is $\Sigma_3^p$-complete.
\end{enumerate}
\end{document}