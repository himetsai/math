\documentclass[a4paper]{article}

\usepackage[english]{babel}
\usepackage[utf8]{inputenc}
\usepackage{amsmath}
\usepackage{amsthm}
\usepackage{amsfonts}
\usepackage{graphicx}
\usepackage{enumerate}
\usepackage[colorinlistoftodos]{todonotes}

\newtheorem{theorem}{Theorem}[section]
\newtheorem{corollary}{Corollary}[theorem]
\newtheorem{lemma}[theorem]{Lemma}

\title{\textsc{Superimposed Extremal Graphs}}

\author{Ray Tsai}

\date{June 21, 2024}

\begin{document}

\maketitle
                                                                                                                                
\section{Introduction}

In this note we talk about \textit{superimposed graphs}. Given graph $G$ with $n$ vertices, let
$G_1, \ldots, G_m$ be subgraphs of $G$. Let $F$ be a graph with at least one edge. Our goal is to
determine the maximum sum of the number of edges in each $G_i$, i.e. $\sum_{i = 1}^m e(G_i)$, with
the constraint of $E(G_i) \cap E(G_j)$ not including $F$ for all distinct $i, j$. 

\section{Objectives}

\begin{itemize}
  \item Examine the case where $G_1, \ldots, G_m$ are induced
  \begin{itemize}
    \item The case $F = K_3$.
    \item Generalize to any $F$.
  \end{itemize}
  \item Examine the non-induced case
  \begin{itemize}
    \item If $F = K_3$, what happens when $m = 3$?
  \end{itemize}
\end{itemize}


\section{Induced Case}

In this section, we assume that $G_1, \ldots, G_m$ are induced subgraphs of $G$.

\subsection{Triangle-Free Case}

\begin{theorem}
  Suppose that $E(G_i) \cap E(G_j)$ does not include $K_3$ for distinct $i, j$. For $m \geq 2$,
  \[
    \sum_{i = 1}^m e(G_i) \leq m\left\lfloor\frac{n^2}{4}\right\rfloor,
  \]
  with equality if and only if $G_1 = G_2 = \cdots = G_m = K_{\left\lceil\frac{n}{2}\right\rceil,
  \left\lfloor\frac{n}{2}\right\rfloor}$.
\end{theorem}

We claim that it suffices to show for the case $m = 2$. Suppose the theorem holds for $m = 2$. Put
$G_{m + 1} = G_1$ and we have
\[
  \sum_{i = 1}^m e(G_i) = \frac{1}{2}\sum_{i = 1}^m (e(G_i) + e(G_{i + 1})) \leq \frac{1}{2}\sum_{i = 1}^m 2\left\lfloor\frac{n^2}{4}\right\rfloor = m\left\lfloor\frac{n^2}{4}\right\rfloor,
\]
with equality only if $G_i = G_{i + 1} = K_{\left\lceil\frac{n}{2}\right\rceil,
  \left\lfloor\frac{n}{2}\right\rfloor}$ for all $i$. That is, $G_1 = G_2 = \cdots = G_m =
  K_{\left\lceil\frac{n}{2}\right\rceil, \left\lfloor\frac{n}{2}\right\rfloor}$.

\begin{proof}[Proof for m = 2]
  Let $C = V(G_1) \cap V(G_2)$, the set of vertices in both $G_1$ and $G_2$. Let $A = V(G_1)
  \backslash C$, and let $B = V(G_2) \backslash C$. For simplicity, put $a = |A|$, $b = |B|$, and $c
  = |C|$. We may assume that $a + b + c = n$. 

  We now find an upper bound of $e(G_1) + e(G_2)$ with respect to $a, b, c$. Obviously, $e(G_1[A])
  \leq \binom{a}{2}$ and $e(G_2[B]) \leq \binom{b}{2}$. There are at most $ac$ edges in $G_1$
  between $A$ and $C$, and at most $bc$ edges in $G_2$ between $B$ and $C$. Now consider the edges
  in $C$. Since $G_1, G_2$ are induced graphs, we have $\{u, v\} \in E(G_1)$ if and only if $\{u,
  v\} \in E(G_2)$, for $u, v \in C$. This implies the subgraph of $G_1$ induced by $C$ is identical
  to the subgraph of $G_2$ induced by $C$. In other words, $E(G_1[C]) = E(G_2[C]) = E(G_i) \cap
  E(G_j)$, which is triangle-free. By Mantel's Theorem, $e(G_1[C]) \leq
  \left\lfloor\frac{c^2}{4}\right\rfloor$, with equality if and only if $G_1[C] =
  K_{\left\lceil\frac{c}{2}\right\rceil, \left\lfloor\frac{c}{2}\right\rfloor}$. Hence, 
  \begin{gather}
    e(G_1) + e(G_2) \leq \binom{a}{2} + \binom{b}{2} + ac + bc + 2\left\lfloor\frac{c^2}{4}\right\rfloor.
  \end{gather}

  Define real differentiable function $f(x, y, z) = \binom{x}{2} + \binom{y}{2} + xz + yz +
  \frac{z^2}{2}$, with real variables $x, y, z \geq 0$. Given the constraints $x + y + z = n$, we
  apply the Lagrange multiplier and get
  \[
    x + z - \frac{1}{2} = y + z - \frac{1}{2} = x + y + z = n.
  \]
  Solving it yields $x = y = -\frac{1}{2}$ and $z = n + 1$, which is out of the boundary. Hence,
  there is no local maximum for $x, y, x > 0$. By compararing the boundary conditions, we conclude
  that $f$ attains global maximum at $(0, 0, n)$. It now follows that 
  \[
    e(G_1) + e(G_2) \leq 2\left\lfloor\frac{n^2}{4}\right\rfloor,
  \]
  with equality if and only if $G_1 = G_2 = K_{\left\lceil\frac{n}{2}\right\rceil, \left\lfloor\frac{n}{2}\right\rfloor}$.
\end{proof}

\subsection{Generalization to any $F$}



\end{document}