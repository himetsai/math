\documentclass{article}

\usepackage{fancyhdr}
\usepackage{extramarks}
\usepackage{amsmath}
\usepackage{amsthm}
\usepackage{amsfonts}
\usepackage{tikz}
\usepackage[plain]{algorithm}
\usepackage{algpseudocode}

\usetikzlibrary{automata,positioning}

%
% Basic Document Settings
%

\topmargin=-0.45in
\evensidemargin=0in
\oddsidemargin=0in
\textwidth=6.5in
\textheight=9.0in
\headsep=0.25in

\linespread{1.1}

\pagestyle{fancy}
\lhead{\hmwkAuthorName}
\chead{\hmwkClass:\ \hmwkTitle}
\rhead{\firstxmark}
\lfoot{\lastxmark}
\cfoot{\thepage}

\renewcommand\headrulewidth{0.4pt}
\renewcommand\footrulewidth{0.4pt}

\setlength\parindent{0pt}
\setlength{\parskip}{5pt}

%
% Create Problem Sections
%

\newcommand{\enterProblemHeader}[1]{
    \nobreak\extramarks{}{Problem \arabic{#1} continued on next page\ldots}\nobreak{}
    \nobreak\extramarks{Problem \arabic{#1} (continued)}{Problem \arabic{#1} continued on next page\ldots}\nobreak{}
}

\newcommand{\exitProblemHeader}[1]{
    \nobreak\extramarks{Problem \arabic{#1} (continued)}{Problem \arabic{#1} continued on next page\ldots}\nobreak{}
    \stepcounter{#1}
    \nobreak\extramarks{Problem \arabic{#1}}{}\nobreak{}
}

\setcounter{secnumdepth}{0}
\newcounter{partCounter}
\newcounter{homeworkProblemCounter}
\setcounter{homeworkProblemCounter}{1}
\nobreak\extramarks{Problem \arabic{homeworkProblemCounter}}{}\nobreak{}

%
% Homework Problem Environment
%
% This environment takes an optional argument. When given, it will adjust the
% problem counter. This is useful for when the problems given for your
% assignment aren't sequential. See the last 3 problems of this template for an
% example.
%
\newenvironment{homeworkProblem}[1][-1]{
    \ifnum#1>0
        \setcounter{homeworkProblemCounter}{#1}
    \fi
    \section{Problem \arabic{homeworkProblemCounter}}
    \setcounter{partCounter}{1}
    \enterProblemHeader{homeworkProblemCounter}
}{
    \exitProblemHeader{homeworkProblemCounter}
}

%
% Homework Details
%   - Title
%   - Due date
%   - Class
%   - Section/Time
%   - Instructor
%   - Author
%

\newcommand{\hmwkTitle}{Homework\ \#2}
\newcommand{\hmwkDueDate}{October 13, 2023}
\newcommand{\hmwkClass}{MATH 140A}
\newcommand{\hmwkClassTime}{Section A02}
\newcommand{\hmwkClassInstructor}{Professor Mohammadi}
\newcommand{\hmwkAuthorName}{\textbf{Ray Tsai}}
\newcommand{\hmwkPID}{A16848188}

%
% Title Page
%

\title{
    \vspace{2in}
    \textmd{\textbf{\hmwkClass:\ \hmwkTitle}}\\
    \normalsize\vspace{0.1in}\small{Due\ on\ \hmwkDueDate\ at 11:00pm}\\
    \vspace{0.1in}\large{\textit{\hmwkClassInstructor}} \\
    \vspace{0.1in}\small\hmwkClassTime \\
    \vspace{3in}
}

\author{
  \hmwkAuthorName \\
  \vspace{0.1in}\small\hmwkPID
}
\date{}

\renewcommand{\part}[1]{\textbf{\large Part \Alph{partCounter}}\stepcounter{partCounter}\\}

%
% Various Helper Commands
%

% Useful for algorithms
\newcommand{\alg}[1]{\textsc{\bfseries \footnotesize #1}}

% For derivatives
\newcommand{\deriv}[1]{\frac{\mathrm{d}}{\mathrm{d}x} (#1)}

% For partial derivatives
\newcommand{\pderiv}[2]{\frac{\partial}{\partial #1} (#2)}

% Integral dx
\newcommand{\dx}{\mathrm{d}x}

% Probability commands: Expectation, Variance, Covariance, Bias
\newcommand{\Var}{\mathrm{Var}}
\newcommand{\Cov}{\mathrm{Cov}}
\newcommand{\Bias}{\mathrm{Bias}}
\newcommand*{\Z}{\mathbb{Z}}
\newcommand*{\Q}{\mathbb{Q}}
\newcommand*{\R}{\mathbb{R}}
\newcommand*{\C}{\mathbb{C}}
\newcommand*{\N}{\mathbb{N}}
\newcommand*{\prob}{\mathds{P}}
\newcommand*{\E}{\mathds{E}}

\begin{document}

\maketitle

\pagebreak

\begin{homeworkProblem}
    Let $A \subset R$ be a non-empty subset which satisfies the following two properties

    \begin{enumerate}
        \item $A = A + A$
        \item For every $\epsilon > 0$, there exists some $a \in A$ so that $0 < a < \epsilon$.
    \end{enumerate}

    Prove that  for every $x \in R^{>0}$, there exists some $a \in A$ so that
    \[
        0 < x - a < \epsilon.
    \]

    \begin{proof}
      Let $a \in A$. We first show that for $n \in \N$, $na \in A$ by induction on $n$. We already know $a \in A$. For $n > 1$, since $(n - 1)a \in A$, we know $na = a + (n - 1)a \in A$ by rule 1. Thus, $na \in A$, for all $n \in \N$.
      
      Since $\epsilon > 0$, there exists $a \in A$ such that $0 < a < \epsilon$, by rule 2.    
      We assume that $\epsilon < x$, otherwise we are done. Now we show that there exists $n \in \N$ such that $0 < x - an < \epsilon$. Let $0 < \frac{x - \epsilon}{a} < n < \frac{x}{a}$. By the Archimedean Property, we know there exists $n > \frac{x - \epsilon}{a}$. Since $\epsilon > a$, the gap $\frac{x}{a} - \frac{x - \epsilon}{a} = \frac{\epsilon}{a} > 1$, and so there exists such natural number $n$ within the interval. Thus, we get 
      \[
        0 = x - a \cdot \frac{x}{a} < x - an < x - a \cdot \frac{x - \epsilon}{a} = \epsilon.
      \]
    \end{proof}
\end{homeworkProblem}

\newpage

\begin{homeworkProblem}

    Let $a, b, c, d \in \R$ and assume $a < b$ and $c < d$. Give an explicit one-to-one correspondence between

    \begin{enumerate}
        \item The points of the two open intervals $(a, b)$ and $(c, d)$.
        
        \begin{proof}
            Define $f: (a, b) \rightarrow (c, d)$ to be $f(x) = \frac{(d - c)x + (cb - ad)}{b - a}$. Let $l, m \in (a, b)$. Since $a < l < b$, 
            \begin{gather*}
                \frac{d - c}{b - a}a < \frac{d - c}{b - a}l < \frac{d - c}{b - a}b \\
                \frac{(d - c)a + (cb - ad)}{b - a} < \frac{(d - c)l + (cb - ad)}{b - a} < \frac{(d - c)b + (cb - ad)}{b - a} \\
                c < f(l) < d.
            \end{gather*}
            Suppose that $l = m$. Then 
            \[
                \frac{(d - c)l + (cb - ad)}{b - a} = f(l) = f(m) = \frac{(d - c)m + (cb - ad)}{b - a},
            \]
            and so $f$ is well defined.

            Suppose $f(l) = f(m)$. Then, 
            \begin{align*}
                \frac{(d - c)l + (cb - ad)}{b - a} &= \frac{(d - c)m + (cb - ad)}{b - a} \\
                (d - c)l + (db - ad) &= (d - c)m + (db - ad) \\
                (d - c)l &= (d - c)m \\
                l &= m.
            \end{align*}
            Thus, $f$ is injective.

            Let $y \in (c, d)$. There exists $x = \frac{(b - a)y - (cb - ad)}{d - c} \in (a, b)$ such that $f(x) = y$, and so $f$ is surjective.

            Thus, $f$ is an one-to-one correspondence.
        \end{proof}

        \item The points of the two closed intervals $[a, b]$ and $[c, d]$.
        
        \begin{proof}
            Define $f: [a, b] \rightarrow [c, d]$ to be $f(x) = \frac{(d - c)x + (cb - ad)}{b - a}$. Let $l, m \in [a, b]$. Since $a \leq l \leq b$, 
            \begin{gather*}
                \frac{d - c}{b - a}a \leq \frac{d - c}{b - a}l \leq \frac{d - c}{b - a}b \\
                \frac{(d - c)a + (cb - ad)}{b - a} \leq \frac{(d - c)l + (cb - ad)}{b - a} \leq \frac{(d - c)b + (cb - ad)}{b - a} \\
                c \leq f(l) \leq d.
            \end{gather*}
            Suppose that $l = m$. Then 
            \[
                \frac{(d - c)l + (cb - ad)}{b - a} = f(l) = f(m) = \frac{(d - c)m + (cb - ad)}{b - a},
            \]
            and so $f$ is well defined.

            Suppose $f(l) = f(m)$. Then, 
            \begin{align*}
                \frac{(d - c)l + (cb - ad)}{b - a} &= \frac{(d - c)m + (cb - ad)}{b - a} \\
                (d - c)l + (db - ad) &= (d - c)m + (db - ad) \\
                (d - c)l &= (d - c)m \\
                l &= m.
            \end{align*}
            Thus, $f$ is injective.

            Let $y \in [c, d]$. There exists $x = \frac{(b - a)y - (cb - ad)}{d - c} \in [a, b]$ such that $f(x) = y$, and so $f$ is surjective.

            Thus, $f$ is an one-to-one correspondence.
        \end{proof}

        \item The points of the closed interval $[a, b]$ and the open interval $(c, d)$.
        
        \begin{proof}
            Define $f: [a, b] \rightarrow (c, d)$ to be
            \[
                f(x) = \begin{cases}
                    c + \frac{d - c}{n + 2}, &x = a + \frac{b - a}{n}, n \in \N \\
                    \frac{c + d}{2}, &x = a \\
                    \frac{(d - c)x + (cb - ad)}{b - a}, & \text{otherwise}.
                \end{cases}
            \]
            Note that the product of $f$ of different cases would not be equal.

            Obviously, $f(x) \in (c, d)$ for all $x \in [a, b]$. Let $k, m \in [a, b]$. If $k = m = a$, then $f(k) = f(m) = \frac{c + d}{2}$. If $k = m = a + \frac{b - a}{n}$, for some $n \in \N$, then $f(k) = f(m) = c + \frac{d - c}{n + 2}$. Otherwise, $\frac{(d - c)k + (cb - ad)}{b - a} = \frac{(d - c)m + (cb - ad)}{b - a}$, which implies that $f(k) = f(m)$. Therefore, $f$ is well defined.

            Suppose that $f(k) = f(m)$. If $f(k) = f(m) = \frac{c + d}{2}$, then $k = m = a$. If $f(k) = f(m) = c + \frac{d - c}{n + 2}$ for some $n \in \N$, then $k = m = a + \frac{b - a}{n}$. If $f(k) = \frac{(d - c)k + (cb - ad)}{b - a} = \frac{(d - c)m + (cb - ad)}{b - a} = f(m)$, Then $k = m$, by the results we obtained from previous parts. Thus, $f$ is injective.

            Let $y \in (c, d)$. There exists
            \[
                x = \begin{cases}
                    a + \frac{b - a}{n}, &y = c + \frac{d - c}{n + 2}, n \in \N \\
                    a, &y = \frac{c + d}{2} \\
                    \frac{(b - a)x + (ad - cb)}{d - c}, & \text{otherwise},
                \end{cases}
            \]
            such that $f(x) = y$. Thus, $f$ is surjective.

            Therefore, $f$ is bijective.
        \end{proof}

        \item The points of the closed interval $[a, b]$ and $\R$
        
        \begin{proof}
            Consider $\tan: (-\frac{\pi}{2}, \frac{\pi}{2}) \rightarrow \R$ and $\tan^{-1}: \R \rightarrow (-\frac{\pi}{2}, \frac{\pi}{2})$. Since $\tan$ and $\tan^{-1}$ are inverses of each other, they are bijective. We can then use the function $f$ we defined in part 3 to get a bijective mapping from $[a, b]$ to $(-\frac{\pi}{2}, \frac{\pi}{2})$. Thus, we get a bijection $(\tan \circ f): [a, b] \rightarrow \R$,
            \[
                (\tan \circ f)(x) = \begin{cases}
                    \tan(-\frac{\pi}{2} + \frac{\pi}{n + 2}), &x = a + \frac{b - a}{n}, n \in \N \\
                    0, &x = a \\
                    \tan(\frac{2\pi x - \pi(b + a)}{2(b - a)}), & \text{otherwise}.
                \end{cases}
            \]
        \end{proof}
    \end{enumerate}
\end{homeworkProblem}

\newpage

\begin{homeworkProblem}
    Fix $b > 1, y > 0$, and prove that there is a unique real $x$ such that $b^x = y$.
    
    \begin{proof}
        We first show that for any positive integer $n$, $b^n - 1 \geq n(b - 1)$. We show that $b^n > 1$ by induction on $n$. We already know $b > 1$. For $n > 1$, $b^n = b \cdot b^{n - 1} > 1$, since $b^{n - 1} > 1$ by induction. Thus, 
        \begin{gather}
            b^n - 1 = (b - 1)(b^{n-1} + \dots + b + 1) \geq (b - 1)n.
        \end{gather}
        By Theorem $1.21$, we know that there exists a unique $a \in \R^+$ such that $a^n = b$. Suppose that $a \leq 1$. We show that $a^n \leq 1$ by induction on $n$. For $n > 1$, we know that $a^n = a \cdot a^{n - 1} \leq 1$, since $a^{n - 1} \leq 1$ by induction. Thus, $a$ must be greater than $1$. Then, by (1), we know that $b - 1= a^n - 1 \geq (a - 1)n = (b^{\frac{1}{n}} - 1)n$. 
        
        Let $t > 1$. Suppose that $n > \frac{b - 1}{t - 1}$, then $nt - n > b - 1$. Note that we know there exists $n > \frac{b - 1}{t - 1}$ by the Archimedean Property. Since $n \geq 1$, we know $t > b$. Note that since $a^n > 1$ for all $n \in \N$, $b = b^{\frac{1}{n}} \cdot a^{n - 1} \geq b^{\frac{1}{n}}$. Thus, we get
        \begin{gather}
            t > b \geq b^{\frac{1}{n}}.
        \end{gather}

        Let $w \in \R$. Suppose that $b^w < y$. Let $t = y \cdot b^{-w} > b^w \cdot b^{-w} = 1$. By (2), there exists $n > \frac{b - 1}{t - 1}$, such that $t = y \cdot b^{-w} > b^{\frac{1}{n}}$, and so $y > b^{w + \frac{1}{n}}$. Suppose that $b^w < y$. Let $t = b^wy^{-1}$. Similarly, there exists $n > \frac{b - 1}{t - 1}$, such that $t = b^wy^{-1} > b^{\frac{1}{n}}$, and so $b^{w-\frac{1}{n}} > y$.

        Let $A$ be the set of all $w$ such that $b^w < y$. We will show that $x = \sup A$ satisfies $b^x = y$. Suppose for the sake of contradiction that $b^x < y$. Then, by the result we obtained above, we know there exists a large enough $n \in \N$, such that $b^x < b^{x + \frac{1}{n}} < n$. This implies that there exists $x + \frac{1}{n} \in A$, which contradicts that $x = \sup A$. Suppose for the sake of contradiction that $b^x > y$. Then, by the result we obtained above, there exists a large enough $n \in \N$, such that $b^x > b^{x - \frac{1}{n}} > y$, contradicting the fact that $x = \sup A$. Thus, $b^{x} = y$.

        Suppose that $b^z = b^x = y$. $x \not> z$, otherwise $b^z < b^x$, contradiction. Similarly, we also know $x \not< z$. Therefore, $x$ is unique.
    \end{proof}
\end{homeworkProblem}

\newpage

\begin{homeworkProblem}
    If $x, y$ are complex, prove that
    \[
        ||x| - |y|| \leq |x - y|.
    \]

    \begin{proof}
        We sqaure both sides. On the right-hand-side, we have
        \begin{align*}
            |x - y|^2 
            &= (x - y)\overline{(x - y)} \\
            &= |x|^2 + |y|^2 - y\overline x - x\overline y
        \end{align*}
        Note that $\overline{x\overline y} = y\overline x$, so $y\overline x + x\overline y = 2\operatorname{Re}(x\overline y)$. On the left-hand-side, we have 
        \begin{align*}
            (|x| - |y|)^2 
            &= |x|^2 + |y|^2 - 2|x||y| \\
            &= |x|^2 + |y|^2 - 2|x||\overline y| \\
            &= |x|^2 + |y|^2 - 2|x \overline y|
        \end{align*}
        Since $\operatorname{Re}(x\overline y) \leq |x \overline y|$,
        \begin{align*}
            |x|^2 + |y|^2 - 2|x \overline y| 
            &\leq |x|^2 + |y|^2 -2\operatorname{Re}(x\overline y) \\
            &= |x|^2 + |y|^2 - y\overline x - x\overline y,
        \end{align*}
        and thus $||x| - |y|| \leq |x - y|$.
    \end{proof}
\end{homeworkProblem}

\newpage

\begin{homeworkProblem}
    If z is a complex number such that $|z| = 1$, that is, such that $z\overline z= 1$, compute
    \[
        |1 + z|^2 + |1 - z|^2
    \]

    \begin{proof}
        \begin{align*}
            |1 + z|^2 + |1 - z|^2
            &= (1 + z)\overline{(1 + z)} + (1 - z)\overline{(1 - z)} \\
            &= 1 + z + \overline z + z\overline z + 1 - z - \overline z + z\overline z \\
            &= 4.
        \end{align*}
    \end{proof}
\end{homeworkProblem}

\newpage

\begin{homeworkProblem}
    Prove that
    \[
        |x + y|^2 + |x - y|^2 = 2|x|^2 + 2|y|^2
    \]
    if $x \in \R^k$ and $y \in \R^k$. Interpret this geometrically, as a statement about parallelograms.

    \begin{proof}
        \begin{align*}
            |x + y|^2 + |x - y|^2
            &= |x|^2 + |y|^2 + 2x \cdot y + |x|^2 + |y|^2 - 2x \cdot y \\
            &= 2|x|^2 + 2|y|^2.
        \end{align*}
        Interpreting geometrically, if $x, y$ were the neighboring sides of a parallelogram, then $x + y$ and $x - y$ are its diagonals. Thus, the equation suggests that the sum of the squares of the sides is equal to the sum of the square of the diagonals.
    \end{proof}
\end{homeworkProblem}
\end{document}