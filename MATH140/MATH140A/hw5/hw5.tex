\documentclass{article}

\usepackage{fancyhdr}
\usepackage{extramarks}
\usepackage{amsmath}
\usepackage{amsthm}
\usepackage{amsfonts}
\usepackage{tikz}
\usepackage[plain]{algorithm}
\usepackage{algpseudocode}
\usepackage{enumerate}
\usepackage{amssymb}

\usetikzlibrary{automata,positioning}

%
% Basic Document Settings
%

\topmargin=-0.45in
\evensidemargin=0in
\oddsidemargin=0in
\textwidth=6.5in
\textheight=9.0in
\headsep=0.25in

\linespread{1.1}

\pagestyle{fancy}
\lhead{\hmwkAuthorName}
\chead{\hmwkClass:\ \hmwkTitle}
\rhead{\firstxmark}
\lfoot{\lastxmark}
\cfoot{\thepage}

\renewcommand\headrulewidth{0.4pt}
\renewcommand\footrulewidth{0.4pt}

\setlength\parindent{0pt}
\setlength{\parskip}{5pt}

%
% Create Problem Sections
%

\newcommand{\enterProblemHeader}[1]{
    \nobreak\extramarks{}{Problem \arabic{#1} continued on next page\ldots}\nobreak{}
    \nobreak\extramarks{Problem \arabic{#1} (continued)}{Problem \arabic{#1} continued on next page\ldots}\nobreak{}
}

\newcommand{\exitProblemHeader}[1]{
    \nobreak\extramarks{Problem \arabic{#1} (continued)}{Problem \arabic{#1} continued on next page\ldots}\nobreak{}
    \stepcounter{#1}
    \nobreak\extramarks{Problem \arabic{#1}}{}\nobreak{}
}

\setcounter{secnumdepth}{0}
\newcounter{partCounter}
\newcounter{homeworkProblemCounter}
\setcounter{homeworkProblemCounter}{1}
\nobreak\extramarks{Problem \arabic{homeworkProblemCounter}}{}\nobreak{}

%
% Homework Problem Environment
%
% This environment takes an optional argument. When given, it will adjust the
% problem counter. This is useful for when the problems given for your
% assignment aren't sequential. See the last 3 problems of this template for an
% example.
%
\newenvironment{homeworkProblem}[1][-1]{
    \ifnum#1>0
        \setcounter{homeworkProblemCounter}{#1}
    \fi
    \section{Problem \arabic{homeworkProblemCounter}}
    \setcounter{partCounter}{1}
    \enterProblemHeader{homeworkProblemCounter}
}{
    \exitProblemHeader{homeworkProblemCounter}
}

%
% Homework Details
%   - Title
%   - Due date
%   - Class
%   - Section/Time
%   - Instructor
%   - Author
%

\newcommand{\hmwkTitle}{Homework\ \#5}
\newcommand{\hmwkDueDate}{Feb 16, 2024}
\newcommand{\hmwkClass}{MATH 140A}
\newcommand{\hmwkClassInstructor}{Professor Seward}
\newcommand{\hmwkAuthorName}{\textbf{Ray Tsai}}
\newcommand{\hmwkPID}{A16848188}

%
% Title Page
%

\title{
    \vspace{2in}
    \textmd{\textbf{\hmwkClass:\ \hmwkTitle}}\\
    \normalsize\vspace{0.1in}\small{Due\ on\ \hmwkDueDate\ at 23:59pm}\\
    \vspace{0.1in}\large{\textit{\hmwkClassInstructor}} \\
    \vspace{3in}
}

\author{
  \hmwkAuthorName \\
  \vspace{0.1in}\small\hmwkPID
}
\date{}

\renewcommand{\part}[1]{\textbf{\large Part \Alph{partCounter}}\stepcounter{partCounter}\\}

%
% Various Helper Commands
%

% Useful for algorithms
\newcommand{\alg}[1]{\textsc{\bfseries \footnotesize #1}}

% For derivatives
\newcommand{\deriv}[1]{\frac{\mathrm{d}}{\mathrm{d}x} (#1)}

% For partial derivatives
\newcommand{\pderiv}[2]{\frac{\partial}{\partial #1} (#2)}

% Integral dx
\newcommand{\dx}{\mathrm{d}x}

% Probability commands: Expectation, Variance, Covariance, Bias
\newcommand{\Var}{\mathrm{Var}}
\newcommand{\Cov}{\mathrm{Cov}}
\newcommand{\Bias}{\mathrm{Bias}}
\newcommand*{\Z}{\mathbb{Z}}
\newcommand*{\Q}{\mathbb{Q}}
\newcommand*{\R}{\mathbb{R}}
\newcommand*{\C}{\mathbb{C}}
\newcommand*{\N}{\mathbb{N}}
\newcommand*{\prob}{\mathds{P}}
\newcommand*{\E}{\mathds{E}}

\begin{document}

\maketitle

\pagebreak

\begin{homeworkProblem}
  Construct a compact set of real numbers whose limit points form a countable set.

  \begin{proof}
    Consider
    \[
      S = \{0\} \cup \left\{\frac{1}{n} \mid n \in \N\right\} \cup \bigcup_{k \in \N} S_k,
    \]
    where $S_k = \left\{\frac{1}{k} + \frac{1}{n} \mid n > k(k - 1), n \in \N\right\}$. Note that
    $S_k$ is bounded below by $\frac{1}{k}$ and bounded above by $\frac{1}{k - 1}$, as $\sup S_k <
    \frac{1}{k} + \frac{1}{k(k - 1)} = \frac{1}{k - 1}$. Thus, $S_i$ and $S_j$ are disjoint if $i
    \neq j$. We claim $S' = \{0\} \cup \left\{\frac{1}{n} \mid n \in \N\right\}$. Since $0$ is a
    limit point of $\left\{\frac{1}{n} \mid n \in \N\right\}$ and $\frac{1}{k}$ is the limit point
    of $S_k$, we only need to prove that $S' \subseteq \left\{\frac{1}{n} \mid n \in \N\right\} \cup
    \{0\}$. It is obvious that $S$ is bounded above by $2$ and below by $0$, so we only need to
    consider points in $[0, 2]$. Let $x \in [0, 2]$ be a limit point of $S$. Suppose for sake of
    contradiction that $x \notin \left\{\frac{1}{n} \mid n \in \N\right\} \cup \{0\}$. If $x > 1$,
    then $x = \epsilon + 1$, for some positive $\epsilon$. But then $B_{\frac{\epsilon}{2}}(x) \cap
    S = B_{\frac{\epsilon}{2}}(x) \cap S_1 = \left\{1 + \frac{1}{n} \mid n < \frac{1}{\epsilon}, n
    \in \N\right\}$ is finite. Hence, we may assume $x < 1$. Then, $\frac{1}{p} < x < \frac{1}{p -
    1}$ for some $p \in \N$, by the archimedean property. This means that $B_{\delta}(x) \cap S =
    B_{\delta}(x) \cap S_p$, for $\delta < \frac{1}{2p(p - 1)}$. But then $x$ is the limit point of
    $S_p$, which is $\frac{1}{p}$, contradiction. Hence, $x \in \{0\} \cup \left\{\frac{1}{n} \mid n
    \in \N\right\} \subset S$, and thus $S = \{0\} \cup \left\{\frac{1}{n} \mid n \in \N\right\}$,
    which is countable. Since $S' \in S$ and $S \subset [0, 2]$, $S$ is compact, by Theorem 2.41.
  \end{proof}
\end{homeworkProblem}

\newpage

\begin{homeworkProblem}
  Regard $\Q$, the set of all rational numbers, as a metric space, with $d(p, q) = |p - q|$. Let $E$
  be the set of all $p \in \Q$ such that $2 < p^2 < 3$. Show that $E$ is closed and bounded in $\Q$,
  but that $E$ is not compact. Is $E$ open in $\Q$?

  \begin{proof}
    $E$ is obviously bounded above by $3$ and below by $-3$, otherwise there exists $p \in E$ such
    that $p^2 > 3^2 > 3$. 
    
    We show that $E$ is closed. Let $x \in E^c$. Then, either $x^2 \leq 2$ or $x^2 \geq 3$.
    Suppose that $x^2 \leq 2$. Then, $-\sqrt{2} \leq x \leq \sqrt{2}$. Pick $\epsilon <
    \min(\sqrt{2} - x, x + \sqrt{2})$. Since $x + \epsilon < x + (\sqrt{2} - x) = \sqrt{2}$ and $x -
    \epsilon > x - (x + \sqrt{2}) = -\sqrt{2}$, $B_{\epsilon}(x)$ is bounded above by $\sqrt{2}$ and
    below by $-\sqrt{2}$. Thus, $B_{\epsilon}(x) \subset [-\sqrt{2}, \sqrt{2}] \cap \Q \subset E^c$,
    $x$ is an interior point of $E^c$. Suppose that $x^2 \geq 3$. Then, either $x \geq \sqrt{3}$ or
    $x \leq -\sqrt{3}$. If $x \geq \sqrt{3}$, then $B_{x - \sqrt{3}}(x)$ is bounded below by
    $\sqrt{3}$, so $B_{x - \sqrt{3}}(x) \subset E^c$. Otherwise, $B_{-\sqrt{3} - x}(x)$ is bounded
    above by $-\sqrt{3}$, and thus $B_{-\sqrt{3} - x}(x) \subset E^c$. Hence, $x$ is an interior
    point of $E^c$. It follows that $E^c$ is open, and thus $E$ is closed. 

    We now show that $E$ is not compact. Consider the set $S = \{2 < r^2 < 3\}$ under $\R$. Let $r
    \in S$. Since $2 < r^2 < 3$, either $\sqrt{2} < r < \sqrt{3}$ or $-\sqrt{3} < r < -\sqrt{2}$.
    Hence, $S = (-\sqrt{3}, -\sqrt{2}) \cup (\sqrt{2}, \sqrt{3})$, which is open by Theorem 2.24. By
    Theorem 2.34, $S$ is not compact in $\R$. By Theorem 2.33, $S$ is not compact relative to $\Q
    \subset \R$. But then $E = S \cap \Q$, so $E$ is not compact.

    By Theorem 2.30, we also know $E = S \cap \Q$ is open in $\Q$.
  \end{proof}
\end{homeworkProblem}

\newpage

\begin{homeworkProblem}
  Let $E$ be the set of all $x \in [0, 1]$ whose decimal expansion contains only the digits 4 and 7.
  Is $E$ countable? Is $E$ dense in $[0, 1]$? Is $E$ compact? Is $E$ perfect?

  \begin{proof}
    We show by Cantor's diagonalization argument that $E$ is uncountable. Let $C$ be a countable set
    of $E$. We associate each number in $C$ a unique index, say $a_1, a_2, \ldots \in C$. Define $f:
    [0, 1] \times \N \rightarrow \Z_{\geq 0}$ that maps $(a, n)$ to the $n$th decimal digit of $a$.
    Let $k \in [0, 1]$ such that $f(k, n) = 4$ if $f(a_n, n) = 7$ and $f(k, n) = 7$ otherwise, for
    $n \in \N$. Since the decimal expansion is of $k$ contains only 4 and 7, $k$ is in $E$. However,
    $k \neq a_i$, for all $a_i \in C$. Hence, $E$ is uncountable.

    Note that $E \subset [0.4, 0.8]$. Since there does not exist $a \in E$ such that $0.8 < a < 1$,
    $E$ is not dense in $[0, 1]$.

    For compactness, we already know $E$ is bounded, so it suffices to show that $E$ is closed, by
    Theorem 2.41. 
    
    Let $x \in E^c$. $x$ contains a decimal digit other than 4 and 7. Consider the first such digit,
    say the $n$th digit. Let $\delta > 0$ such that $\delta < 10^{-n}$. Then, $N_{\delta}(x)$ does
    not contain any point in $E$, and thus $x$ cannot be a limit point of $E$. Therefore, $E'
    \subset E$, so $E$ is closed.

    However, $E$ is not perfect. Consider $0.4 \in E$. $N_{0.01}(0.4) \cap E = \emptyset$ and thus
    $0.4$ is not a limit point of $E$.
  \end{proof}
\end{homeworkProblem}

\newpage

\begin{homeworkProblem}
  Is there a nonempty perfect set in $\R^1$ which contains no rational number?

  \begin{proof}
    Yes. Let $E_0 = [r, s]$, where $r, s$ are two irrational numbers. We inductively remove all
    rational numbers from $E_0$. Since $\Q$ is countable, associate an index to each rational
    number, say $a_1, a_2, \ldots \in E_0$. We construct $E_n$ by removing the segment $(r_n, s_n)$
    from $E_{n - 1}$, for some irrational $r_n, s_n$ such that $r_n < a_n < s_n$, and make it $E_n$.
    Note that $E_n$ union of intervals and thus $E_0 \supset E_1 \supset \cdots$ is a chain of
    compact sets, by Theorem 2.41. Let $E = \bigcap_{i = 0}^{\infty} E_i$. $E$ is closed and
    nonempty, by Theorem 2.24 and 2.36. For rational $q \in [r, s]$, $q \notin E_n$, and thus $q
    \notin E$. Hence, $E$ does not contain any rational numbers. It remains to show that every point
    in $E$ is a limit point. Notice that $E$ does contain any segments, as $\Q$ is dense in $\R$ so
    any segment $(a, b) \subset \R$ contains a rational number. Let $x \in E$ and let $\epsilon >
    0$. Let $I_n$ be an interval of $E_n$ which contains $x$. Since the open set $N_{\epsilon}(x)$
    is not contained in $E$, there exists large enough $n$ such that $I_n \subset N_{\epsilon}(x)$,
    and thus $N_{\epsilon}(x)$ contains the end point of $I_n$. It follows that $x$ is a limit point
    of $E$, as $(N_{\epsilon}(x) \backslash \{x\}) \cap E \neq \emptyset$, and this completes the
    proof.
  \end{proof}
\end{homeworkProblem}

\newpage

\begin{homeworkProblem}
  \begin{enumerate}[(a)]
    \item If $A$ and $B$ are disjoint closed sets in some metric space $X$, prove that they are
    separated.
    \begin{proof}
      Since $A, B$ are closed, $A = \overline{A}$ and $B = \overline{B}$. Since $\overline{A} \cap B
      = A \cap \overline{B} = A \cap B = \emptyset$, the result follows.
    \end{proof}
    \item Prove the same for disjoint open sets.
    \begin{proof}
      It suffices to show that $A' \cap B = A \cap B' = \emptyset$. Let $a \in A$. Since there
      exists a neighborhood $N$ of $a$ such that $N \subset A$, $N$ contains not point of $B$, and
      thus $A \cap B' = \emptyset$. By symmetry, we also know $B' \cap A = \emptyset$, and this
      completes to proof.
    \end{proof}
    \item Fix $p \in X, \delta > 0$, define $A$ to be the set of all $q \in X$ for which $d(p, q) <
    \delta$, define $B$ similarly, with $>$ in place of $<$. Prove that $A$ and $B$ are separated.
    \begin{proof}
      Since $A = N_{\delta}(p)$, $A$ is obviously an open set. Moreover, $B$ is the complement of
      $\overline{A}$, so $B$ is an open set disjoint to $A$. The result now follows from $(b)$.
    \end{proof}
    \item Prove that every connected metric space with at least two points is uncountable.
    \textit{Hint:} Use (c).
    \begin{proof}
      Let $X$ be a metric space and $p, q \in X$ such that $p < q$. Then, $d(p, q) > 0$. Since $X$
      is connected, there exists $m \in X$ such that $d(p, m) = \delta$, for every $\delta \in [0,
      d(p, q)]$, otherwise $X$ is separated by $(c)$. Hence, there exists an surjective mapping $X
      \rightarrow [0, d(p, q)]$ that maps $r$ to $\epsilon$, for some $d(p ,r) = \epsilon$. However,
      $[0, d(p, q)]$ is uncountable, and thus $X$ is uncountable.
    \end{proof}
  \end{enumerate}
\end{homeworkProblem}

\newpage

\begin{homeworkProblem}
  Are closures and interiors of connected sets always connected? (Look at subsets of $\R^2$.)

  \begin{proof}
    Closures of connected sets are connected. Let $X$ be a nonempty connected set. Suppose for the
    sake of contradiction that $\overline{X}$ is not connected. Then, $\overline{X} = A \cup B$,
    where $\overline{A} \cap B = \overline{B} \cap A = \emptyset$. We know $X \not\subset A$,
    otherwise  $B$ contains limit points of $A$, which forces $\overline{A} \cap B \neq \emptyset$.
    Similarly, $X \not\subset B$, so $X \cap A$ and $X \cap B$ are both nonempty. Hence, $X$ is the
    union of disjoint sets $X \cap A$ and $X \cap B$. However, since $\overline{A} \cap B =
    \overline{B} \cap A = \emptyset$, we have $(\overline{X \cap A}) \cap (X \cap B) = (\overline{X
    \cap B}) \cap (X \cap A) = \emptyset$, contradicting that $X$ is connected. 

    However, interiors of connected sets are not always connected. Let $p, q \in \R^2$, $p \neq q$.
    Let $A = \{a \in \R^2 \mid d(p, a) \leq \frac{1}{2}d(p, q)\}$ and $B = \{b \in \R^2 \mid d(q, b)
    \leq \frac{1}{2}d(p, q)\}$, and let $E = A \cup B$. Note that $A \cap B \neq \emptyset$. Suppose
    for the sake of contradiction tha $E$ is not connected. Then, $E$ can be partitioned into two
    nonempty sets $G, H$, such that $\overline{G} \cap H = \overline{H} \cap G = \emptyset$. Let $x
    \in A \cap B$. Suppose WLOG that $x \in G$. Let $y \in H$. Since $y \in E$, we know $y$ is in
    $A$ or $B$. Say that $y \in A$. Then, $x \in A \cap G$ and $y \in A \cap H$. However, since $G,
    H$ are disjoint, $A$ can be separated into two disjoint sets $A \cap G$ and $A \cap H$,
    contradiction. Hence, $E$ is connected. The interior points of $E$ is $N_{\frac{1}{2}d(p, q)}(p)
    \cup N_{\frac{1}{2}d(p, q)}(q)$, which is the union of two disjoint open sets. The result now
    follows from Problem 5 (b).
  \end{proof}
\end{homeworkProblem}

\newpage

\begin{homeworkProblem}
  Let $A$ and $B$ be separated subsets of some $\mathbb{R}^k$, suppose $a \in A$, $b \in B$, and
  define
  \[ p(t) = (1 - t)a + tb \] for $t \in \mathbb{R}^1$. Put $A_0 = p^{-1}(A)$, $B_0 = p^{-1}(B)$.
  [Thus $t \in A_0$ if and only if $p(t) \in A$.]
  \begin{enumerate}[(a)]
      \item Prove that $A_0$ and $B_0$ are separated subsets of $\mathbb{R}^1$.
      \begin{proof}
        $A_0$ and $B_0$ are disjoint, otherwise there exists $x \in A_0 \cap B_0$ such that $p(x)
        \in A \cap B$. Let $k$ be a limit point of $A_0$. Suppose for the sake of contradiction that
        $k \in B_0$. Then, for $\epsilon > 0$, there exists $m \in N_{\frac{\epsilon}{|b - a|}}(k)
        \cap A_0$. Hence, $d(p(k), p(m)) = |[(1 - k)a + kb] - [(1 - m)a + mb]| = (k - m)|b - a| <
        \epsilon$. Since $\epsilon$ is arbitrary and $p(m) \in A$, $p(k)$ is a limit point of $A$.
        But then $p(k) \in B_0 \cap \overline{A_0}$, contradiction. Hence, $k \notin B_0$. By
        symmetry, we also know that $A_0 \cap \overline{B_0} = \emptyset$, and thus $A_0$ and $B_0$
        are separated.
      \end{proof}
      \item Prove that there exists $t_0 \in (0, 1)$ such that $p(t_0) \not\in A \cup B$.
      \begin{proof}
        Note that $0 \in A_0$ and $1 \in B_0$. Let $t_0 = \sup(A \cap [0, 1])$. By Theorem 2.28,
        $t_0 \in \overline{A_0}$, and thus $t_0 \notin B_0$, as $A_0, B_0$ are separated. In
        particular, $0 \leq t_0 < 1$. If $t_0 \notin A_0$, it follows that $0 < t_0 < 1$ and $t_0
        \notin A_0 \cup B_0$. If $t_0 \in A_0$, then $t_0 \notin \overline{B_0}$, and thus there
        exists $t_0' \notin B_0$ such that $t_0 < t_0' < 1$. Hence, $0 < t_0' < 1$ and $t_0 \notin
        A_0 \cup B_0$. Since there is a point $t_0 \in (0, 1) \backepsilon (A_0 \cup B_0)$, $p(t_0)
        \notin A \cup B$.
      \end{proof}
      \item Prove that every convex subset of $\mathbb{R}^k$ is connected.
      \begin{proof}
        Let $S \subset \R^k$ be not connected. Then, $S = A \cup B$, for separated $A, B$. But then
        there exists $t \in (0, 1)$ such that $(1 - t)a + tb \notin A \cup B$, by (b), and thus $S$
        is not convex. The result now follows from the contrapositive.
      \end{proof}
  \end{enumerate}
\end{homeworkProblem}
\end{document}