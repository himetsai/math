\documentclass{article}

\usepackage{fancyhdr}
\usepackage{extramarks}
\usepackage{amsmath}
\usepackage{amsthm}
\usepackage{amsfonts}
\usepackage{tikz}
\usepackage[plain]{algorithm}
\usepackage{algpseudocode}
\usepackage{enumerate}
\usepackage{amssymb}

\usetikzlibrary{automata,positioning}

%
% Basic Document Settings
%

\topmargin=-0.45in
\evensidemargin=0in
\oddsidemargin=0in
\textwidth=6.5in
\textheight=9.0in
\headsep=0.25in

\linespread{1.1}

\pagestyle{fancy}
\lhead{\hmwkAuthorName}
\chead{\hmwkClass:\ \hmwkTitle}
\rhead{\firstxmark}
\lfoot{\lastxmark}
\cfoot{\thepage}

\renewcommand\headrulewidth{0.4pt}
\renewcommand\footrulewidth{0.4pt}

\setlength\parindent{0pt}
\setlength{\parskip}{5pt}

%
% Create Problem Sections
%

\newcommand{\enterProblemHeader}[1]{
    \nobreak\extramarks{}{Problem \arabic{#1} continued on next page\ldots}\nobreak{}
    \nobreak\extramarks{Problem \arabic{#1} (continued)}{Problem \arabic{#1} continued on next page\ldots}\nobreak{}
}

\newcommand{\exitProblemHeader}[1]{
    \nobreak\extramarks{Problem \arabic{#1} (continued)}{Problem \arabic{#1} continued on next page\ldots}\nobreak{}
    \stepcounter{#1}
    \nobreak\extramarks{Problem \arabic{#1}}{}\nobreak{}
}

\setcounter{secnumdepth}{0}
\newcounter{partCounter}
\newcounter{homeworkProblemCounter}
\setcounter{homeworkProblemCounter}{1}
\nobreak\extramarks{Problem \arabic{homeworkProblemCounter}}{}\nobreak{}

%
% Homework Problem Environment
%
% This environment takes an optional argument. When given, it will adjust the
% problem counter. This is useful for when the problems given for your
% assignment aren't sequential. See the last 3 problems of this template for an
% example.
%
\newenvironment{homeworkProblem}[1][-1]{
    \ifnum#1>0
        \setcounter{homeworkProblemCounter}{#1}
    \fi
    \section{Problem \arabic{homeworkProblemCounter}}
    \setcounter{partCounter}{1}
    \enterProblemHeader{homeworkProblemCounter}
}{
    \exitProblemHeader{homeworkProblemCounter}
}

%
% Homework Details
%   - Title
%   - Due date
%   - Class
%   - Section/Time
%   - Instructor
%   - Author
%

\newcommand{\hmwkTitle}{Homework\ \#3}
\newcommand{\hmwkDueDate}{Feb 5, 2024}
\newcommand{\hmwkClass}{MATH 140A}
\newcommand{\hmwkClassInstructor}{Professor Seward}
\newcommand{\hmwkAuthorName}{\textbf{Ray Tsai}}
\newcommand{\hmwkPID}{A16848188}

%
% Title Page
%

\title{
    \vspace{2in}
    \textmd{\textbf{\hmwkClass:\ \hmwkTitle}}\\
    \normalsize\vspace{0.1in}\small{Due\ on\ \hmwkDueDate\ at 23:59pm}\\
    \vspace{0.1in}\large{\textit{\hmwkClassInstructor}} \\
    \vspace{3in}
}

\author{
  \hmwkAuthorName \\
  \vspace{0.1in}\small\hmwkPID
}
\date{}

\renewcommand{\part}[1]{\textbf{\large Part \Alph{partCounter}}\stepcounter{partCounter}\\}

%
% Various Helper Commands
%

% Useful for algorithms
\newcommand{\alg}[1]{\textsc{\bfseries \footnotesize #1}}

% For derivatives
\newcommand{\deriv}[1]{\frac{\mathrm{d}}{\mathrm{d}x} (#1)}

% For partial derivatives
\newcommand{\pderiv}[2]{\frac{\partial}{\partial #1} (#2)}

% Integral dx
\newcommand{\dx}{\mathrm{d}x}

% Probability commands: Expectation, Variance, Covariance, Bias
\newcommand{\Var}{\mathrm{Var}}
\newcommand{\Cov}{\mathrm{Cov}}
\newcommand{\Bias}{\mathrm{Bias}}
\newcommand*{\Z}{\mathbb{Z}}
\newcommand*{\Q}{\mathbb{Q}}
\newcommand*{\R}{\mathbb{R}}
\newcommand*{\C}{\mathbb{C}}
\newcommand*{\N}{\mathbb{N}}
\newcommand*{\prob}{\mathds{P}}
\newcommand*{\E}{\mathds{E}}

\begin{document}

\maketitle

\pagebreak

\begin{homeworkProblem}
  A complex number $z$ is said to be \textit{algebraic} if there are integers $a_0, \ldots, a_n$,
  not all zero, such that
  \[
    a_0z^n + a_1z^{n-1} + \cdots + a_{n-1}z + a_n = 0.
  \]
  Prove that the set of all algebraic numbers is countable. \textit{Hint}: For every positive
  integer $N$ there are only finitely many equations with
  \[
    n + |a_0| + |a_1| + \cdots + |a_n| = N.
  \]

  \begin{proof}
    Let $p$ be a $n$-degree polynomial of integer coefficients. By the Fundamental Theorem of
    Algebra, $p$ has $n$ complex roots. Notice that since $\Z^i$ is countable for all $i > 0$, $S =
    \bigcup_{i = 1}^{\infty} \{i\} \times \Z^i$ is countable, by Theorem 2.12. This follows that for
    $m \in \N$, each $(m, a_0, a_1, \ldots, a_m) \in S$, gives $m$ algebraic numbers and $S$
    contains all possible tuples of integer coefficients, so the set
   
    \[
      \bigcup_{(n, a_0, a_1, \ldots, a_n) \in S} \{z \mid a_0z^n + a_1z^{n-1} + \cdots + a_{n-1}z + a_n = 0\}
    \]
    contains all algebraic numbers and it is countable.
  \end{proof}
\end{homeworkProblem}

\newpage

\begin{homeworkProblem}
  Let $E'$ be the set of all limit points of a set $E$. Prove that $E'$ is closed. Prove that $E$
  and $\overline{E}$ have the same limit points. Do $E$ and $E'$ always have the same limit points?

  \begin{proof}
    Let $p$ be a limit point of $E'$. It suffices to show that there exists some $k \in E$ such that
    $d(p, k) < r$, for all $r > 0$. Since $p$ is a limit point, there exists $q \in E'$ such that
    $d(p, q) < \frac{r}{2}$. However, as $q \in E'$, $q$ is a limit point of $E$, so there exists $k
    \in E$ such that $d(q, k) < \frac{r}{2}$. Hence, $d(p, k) < d(p, q) + d(q, k) < r$, so $p$ is a
    limit point of $E$. It follows that $p \in E'$ so $E'$ is closed. 

    We prove that $E$ and $\overline{E}$ have the same limit points. $E'$ is obviously contained in
    the set of limit points of $\overline{E}$, so it suffices to show the converse. Let $x$ be a
    limit point of $\overline{E}$. We show that $x \in E'$. Since $\overline{E}$ is closed, $x \in
    \overline{E} = E \cup E'$. We may assume that $x \in E$, otherwise we are done. For $r > 0$, we
    know that there exists $y \in \overline{E}$ such that $d(x, y) < \frac{r}{2}$. If $y \notin E$,
    then $y$ is a limit point of $E$, so there exists $z \in E$ such that $d(y, z) < \frac{r}{2}$.
    But then $d(x, z) < d(x, y) + d(y, z) < r$. Hence, there exists some elements in $E$ such that
    its in $N_r(x)$, for any $r > 0$. Thus, $x$ is a limit point of $E$, so $x \in E'$.

    To see that $E$ and $E'$ do not always share the same limit points, consider $E = \{0, 1,
    \frac{1}{2}, \ldots\}$. Since $E' = \{0\}$, $E'$ does not have any limit points.
  \end{proof}
\end{homeworkProblem}

\newpage

\begin{homeworkProblem}
  Let \( A_1, A_2, A_3, \ldots \) be subsets of a metric space.

  \begin{enumerate}[(a)]
      \item If \( B_n = \bigcup_{i=1}^{n} A_i \), prove that \( \overline{B}_n = \bigcup_{i=1}^{n}
      \overline{A}_i \) for \( n = 1,2,3,\ldots \).
      \begin{proof}
        We first show that $M' \cup N' = (M \cup N)'$, for subsets $M, N$. Since $x \in M' \cup N'$
       is a limit point of $M$ or $N$, we get $x \in (M \cup N)'$. Hence, it just need to show that
       $(M \cup N)' \subseteq M' \cup N'$. Suppose $y \notin M' \cup N'$. Then, there exists $r, s >
       0$ such that $N_r(y)$ does not contain any points in $M$ and $N_s(y)$ does not contain any
       points in $N$. Hence, $N_{\min(r, s)}(y)$ does not contain any points in $M \cup N$, and thus
       $y \notin (M \cup N)'$. By the contrapositive of the statement, we get $(M \cup N)' \subseteq
       M' \cup N'$. Now that we have shown $M' \cup N' = (M \cup N)'$, we get $\overline{M} \cup
       \overline{N} = \overline{M \cup N}$.

       We may now prove $\overline{B}_n = \bigcup_{i=1}^{n} \overline{A}_i$ by induction on $n$. The
       base case is trivial. For $n > 1$, 
       \begin{align*}
        \overline{B}_n 
        &= \overline{\left(\bigcup_{i=1}^{n} A_i\right)} \\
        &= \overline{\left(A_n \cup \bigcup_{i=1}^{n - 1} A_i\right)} \\
        &= \overline{A_n} \cup \overline{\left(\bigcup_{i=1}^{n - 1} A_i\right)}.
       \end{align*}
       Hence, $\overline{B}_n = \overline{A_n} \cup \overline{\left(\bigcup_{i=1}^{n - 1}
       A_i\right)} = \overline{A_n} \cup \bigcup_{i=1}^{n - 1} \overline{A}_i =  \bigcup_{i=1}^{n}
       \overline{A}_i$, by induction.
      \end{proof}
      \item If \( B = \bigcup_{i=1}^{\infty} A_i \), prove that \( \overline{B} \supset
      \bigcup_{i=1}^{\infty} \overline{A}_i \). Show, by an example, that this inclusion can be
      proper.

      \begin{proof}
        Let $x \in \bigcup_{i=1}^{\infty} \overline{A}_i$. Then, $x \in A_i \cup A_i'$, for some $i
        \in \N$. Hence, we may assume that $x$ is the limit point of some $A_i$, otherwise $x \in
        A_i \subset B \subset \overline{B}$ and we are done. However, $N_r(x)$ contains a point in
        $A_i \subset B$ for $r > 0$, so $x$ is also a limit point of $B$, and thus $x \in
        \overline{B}$.

        Let $A_i = \{\frac{1}{i}\}$, for $i \in \N$. Note that $A_i$ does not have a limit point.
        But then $B = \{\frac{1}{k} \mid k \in \N\}$ has a limit point $0$. Therefore, $0 \in
        \overline{B} \backslash \bigcup_{i=1}^{\infty} \overline{A}_i$.
      \end{proof}
  \end{enumerate}
\end{homeworkProblem}

\newpage

\begin{homeworkProblem}
  Is every point of every open set \( E \subseteq \mathbb{R}^2 \) a limit point of \( E \)? Answer
  the same question for closed sets in \( \mathbb{R}^2 \).

  \begin{proof}
    This is true. Let $x = (x_1, x_2) \in E$. Since $x$ is an interior point in $E$, there exists $r
    > 0$ such that $N_r(x) \subseteq E$. Since $x \in \R^2$, there exists $k = (x_1 - \frac{r}{2},
    x_2 - \frac{r}{2}) \in \R^2$ such that $d(x, k) = \sqrt{(x_1 - (x_1 - \frac{r}{2}))^2 + (x_2 -
    (x_2 - \frac{r}{2}))^2} = \frac{r}{\sqrt{2}} < r$, so $N_r(x)$ is not empty. Hence, for any $t >
    0$, if $t > r$ we can find $k \in N_r(x)$ such that $d(x, y) < r < t$. Otherwise, since $x \in
    \R^2$, there exists $s = (x_1 - \frac{t}{2}, x_2 - \frac{t}{2}) \in \R^2$ such that $d(x, s) =
    \frac{t}{\sqrt{2}} < t \leq r$. But then $s\in N_r(x)$. Therefore, $x$ is a limit point in $E$.
    
    However, this does not hold true for closed sets. Consider any non-empty finite set $S$ in
    $\R^2$. $S$ does not have any limit points.
  \end{proof}
\end{homeworkProblem}

\newpage

\begin{homeworkProblem}
  Let \( X \) be an infinite set. For \( p \in X \) and \( q \in X \), define
  \[
  d(p, q) = 
  \begin{cases} 
  1 & \text{if } p \neq q, \\
  0 & \text{if } p = q.
  \end{cases}
  \]
  Prove that this is a metric. Which subsets of the resulting metric space are open? Which are
  closed?

  \begin{proof}
    We first check that $d$ is a valid metric. By definition, we already know $d(p, q) = d(q, p)$ is
    positive than $p \neq q$, otherwise it is $0$. Let $r \in X$. We show that $d(p, q) \leq d(p, r)
    + d(r, q)$ holds. Since $d$ is nonnegative, we may assume that $p \neq q$, otherwise we are
    done. Then, $r$ canot be equal to both $p$ and $q$, so at least one of $d(p, r)$, $d(r, q)$ is
    $1$. Therefore, $d(p, q) \leq 1 \leq d(p, r) + d(r, q)$, and thus $d$ is a metric.

    Let $E \subset X$ be finite and non-empty. Since for $e \in E$, $N_{\frac{1}{\pi}}(e) = \{e\}
    \subset E$, so every point in $E$ is an interior point, which makes $E$ an open set. Since any
    set in $X$ is an union of finite sets, all sets in $X$ is thus an open set. However, any set in
    $X$ is also the complement of a set, so any set in $X$ is also closed.
  \end{proof}
\end{homeworkProblem}

\newpage

\begin{homeworkProblem}
  For \( x \in \mathbb{R}^1 \) and \( y \in \mathbb{R}^1 \), define
  \begin{align*}
    d_1 (x, y) &= (x - y)^2, \\
    d_2 (x, y) &= \sqrt{\lvert x - y \rvert}, \\
    d_3 (x, y) &= \lvert x^2 - y^2 \rvert, \\
    d_4 (x, y) &= \lvert x - 2y \rvert, \\
    d_5 (x, y) &= \frac{\lvert x - y \rvert}{1 + \lvert x - y \rvert}.
  \end{align*}
  Determine, for each of these, whether it is a metric or not.

  \begin{proof}
    We first note that $d_i(x, x) = 0$ and $d_i(x, y) = d_i(y, x)$, for $i \in \{1, 2, 5\}$. $d_3$
    is not a metric as $d(1, -1) = 0$. $d_4$ is not a metric as $d_4(1, 1) \neq 0$. Hence, we only
    need to check the triangle inequality for each $d_i$. Let $z \in \R$.

    For $d_1$, choose $x = 1$, $y = 0$, and $z = \frac{1}{2}$. Since $(x - y)^2 = 1 \geq \frac{1}{4}
    = (x - z)^2 + (z - y)^2$, $d_1$ is not a matric.

    For $d_2$, since $|x - y| \leq |x - z| + |z - y|$ and $2\sqrt{|x - z||z - y|} \geq 0$, we get
    \[
      |x - y| \leq |x - z| + |z - y| + 2\sqrt{|x - z||z - y|} = (\sqrt{\lvert x - y \rvert} + \sqrt{\lvert y - z \rvert})^2,
    \]
    and thus the triangle equality is met by taking the square roots of both sides. Hence, $d_2$ is
    a metric. 

    For $d_5$, we show that $\frac{\lvert x - y \rvert}{1 + \lvert x - y \rvert} \leq \frac{\lvert x
    - z \rvert}{1 + \lvert x - z \rvert} + \frac{\lvert y - z \rvert}{1 + \lvert y - z \rvert}$. By
    multiplying both sides by the denominators and clearing the repeated terms on both sides, we get
    \begin{align*}
      |x - y| \leq |x - z| + |z - y| + 2|x - z||z - y| + 2|x - y||x - z||z - y|.
    \end{align*}
    Since $|x - y| \leq |x - z| + |z - y|$, the above inequality holds, and thus $d_5$ is a metric.
  \end{proof}
\end{homeworkProblem}

\newpage

\begin{homeworkProblem}
  Prove that the set of all injections from the set of natural numbers to itself is uncountable.

  \begin{proof}
    Let $S$ be a countable set of injections from $\N$ to $\N$, and we index each function in $S$,
    say $s_1, s_2, \ldots$. Note that we may view an injection from $\N$ to $\N$ as an infinite
    sequence that does not have repeated numbers. We wish to construct an injection not already in
    $S$. We start with some injection $f: \N \rightarrow \N$. Whenever $f(2k) = s_k(2k)$, we update
    $f$ by swapping $f(2k)$ with $f(2k + 1)$, as $f(2k) \neq f(2k + 1)$. Note that we merely changed
    the ordering of $f$, so $f$ remains to be an injection. Hence, $f(2k) \neq s_k(2k)$ for all $s_k
    \in S$, so $f$ is an injection not in $S$. The result then follows.
  \end{proof}
\end{homeworkProblem}
\end{document}