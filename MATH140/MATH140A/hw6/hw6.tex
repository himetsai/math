\documentclass{article}

\usepackage{fancyhdr}
\usepackage{extramarks}
\usepackage{amsmath}
\usepackage{amsthm}
\usepackage{amsfonts}
\usepackage{tikz}
\usepackage[plain]{algorithm}
\usepackage{algpseudocode}
\usepackage{enumerate}
\usepackage{amssymb}

\usetikzlibrary{automata,positioning}

%
% Basic Document Settings
%

\topmargin=-0.45in
\evensidemargin=0in
\oddsidemargin=0in
\textwidth=6.5in
\textheight=9.0in
\headsep=0.25in

\linespread{1.1}

\pagestyle{fancy}
\lhead{\hmwkAuthorName}
\chead{\hmwkClass:\ \hmwkTitle}
\rhead{\firstxmark}
\lfoot{\lastxmark}
\cfoot{\thepage}

\renewcommand\headrulewidth{0.4pt}
\renewcommand\footrulewidth{0.4pt}

\setlength\parindent{0pt}
\setlength{\parskip}{5pt}

%
% Create Problem Sections
%

\newcommand{\enterProblemHeader}[1]{
    \nobreak\extramarks{}{Problem \arabic{#1} continued on next page\ldots}\nobreak{}
    \nobreak\extramarks{Problem \arabic{#1} (continued)}{Problem \arabic{#1} continued on next page\ldots}\nobreak{}
}

\newcommand{\exitProblemHeader}[1]{
    \nobreak\extramarks{Problem \arabic{#1} (continued)}{Problem \arabic{#1} continued on next page\ldots}\nobreak{}
    \stepcounter{#1}
    \nobreak\extramarks{Problem \arabic{#1}}{}\nobreak{}
}

\setcounter{secnumdepth}{0}
\newcounter{partCounter}
\newcounter{homeworkProblemCounter}
\setcounter{homeworkProblemCounter}{1}
\nobreak\extramarks{Problem \arabic{homeworkProblemCounter}}{}\nobreak{}

%
% Homework Problem Environment
%
% This environment takes an optional argument. When given, it will adjust the
% problem counter. This is useful for when the problems given for your
% assignment aren't sequential. See the last 3 problems of this template for an
% example.
%
\newenvironment{homeworkProblem}[1][-1]{
    \ifnum#1>0
        \setcounter{homeworkProblemCounter}{#1}
    \fi
    \section{Problem \arabic{homeworkProblemCounter}}
    \setcounter{partCounter}{1}
    \enterProblemHeader{homeworkProblemCounter}
}{
    \exitProblemHeader{homeworkProblemCounter}
}

%
% Homework Details
%   - Title
%   - Due date
%   - Class
%   - Section/Time
%   - Instructor
%   - Author
%

\newcommand{\hmwkTitle}{Homework\ \#6}
\newcommand{\hmwkDueDate}{Feb 23, 2024}
\newcommand{\hmwkClass}{MATH 140A}
\newcommand{\hmwkClassInstructor}{Professor Seward}
\newcommand{\hmwkAuthorName}{\textbf{Ray Tsai}}
\newcommand{\hmwkPID}{A16848188}

%
% Title Page
%

\title{
    \vspace{2in}
    \textmd{\textbf{\hmwkClass:\ \hmwkTitle}}\\
    \normalsize\vspace{0.1in}\small{Due\ on\ \hmwkDueDate\ at 23:59pm}\\
    \vspace{0.1in}\large{\textit{\hmwkClassInstructor}} \\
    \vspace{3in}
}

\author{
  \hmwkAuthorName \\
  \vspace{0.1in}\small\hmwkPID
}
\date{}

\renewcommand{\part}[1]{\textbf{\large Part \Alph{partCounter}}\stepcounter{partCounter}\\}

%
% Various Helper Commands
%

% Useful for algorithms
\newcommand{\alg}[1]{\textsc{\bfseries \footnotesize #1}}

% For derivatives
\newcommand{\deriv}[1]{\frac{\mathrm{d}}{\mathrm{d}x} (#1)}

% For partial derivatives
\newcommand{\pderiv}[2]{\frac{\partial}{\partial #1} (#2)}

% Integral dx
\newcommand{\dx}{\mathrm{d}x}

% Probability commands: Expectation, Variance, Covariance, Bias
\newcommand{\Var}{\mathrm{Var}}
\newcommand{\Cov}{\mathrm{Cov}}
\newcommand{\Bias}{\mathrm{Bias}}
\newcommand*{\Z}{\mathbb{Z}}
\newcommand*{\Q}{\mathbb{Q}}
\newcommand*{\R}{\mathbb{R}}
\newcommand*{\C}{\mathbb{C}}
\newcommand*{\N}{\mathbb{N}}
\newcommand*{\prob}{\mathds{P}}
\newcommand*{\E}{\mathds{E}}

\begin{document}

\maketitle

\pagebreak

\begin{homeworkProblem}
  Calculate $\lim_{{n \to \infty}} \left( \sqrt{n^2 + n} - n \right)$.

  \begin{proof}
    We show that the limit is $\frac{1}{2}$. Since $\lim_{{n \to \infty}} \frac{\sqrt{n^2 + n} + n}{\sqrt{n^2 + n} + n} = 1$, we have
    \begin{align*}
      \lim_{{n \to \infty}} \left( \sqrt{n^2 + n} - n \right) 
      &= \lim_{{n \to \infty}} \left( \sqrt{n^2 + n} - n \right)\left(\frac{\sqrt{n^2 + n} + n}{\sqrt{n^2 + n} + n}\right) \\
      &= \lim_{{n \to \infty}} \left(\frac{n}{\sqrt{n^2 + n} + n}\right) \\
      &= \lim_{{n \to \infty}} \left(\frac{1}{\sqrt{1 + \frac{1}{n}} + 1}\right),
    \end{align*}
    by Theorem 3.3. Note that
    \[
      \frac{1}{1 + \frac{1}{n} + 1} = \frac{1}{\frac{1}{n} + 2} < \frac{1}{\sqrt{1 + \frac{1}{n}} + 1} < \frac{1}{1 + 1} = \frac{1}{2}.
    \]
    Since $\frac{1}{\frac{1}{n} + 2} \to \frac{1}{2}$, the result follows from Theorem 3.19.
  \end{proof}
\end{homeworkProblem}

\newpage

\begin{homeworkProblem}
  Find the upper and lower limits of the sequence $(s_n)$ defined by
  \[
    s_1 = 0; \quad s_{2m} = \frac{s_{2m-1}}{2}, \quad s_{2m+1} = \frac{1}{2} + s_{2m}.
  \]

  \begin{proof}
    We first show that $s_{2m + 1} = 1 - 2^{-m}$ by induction on $m$. If $m = 0$, $s_1 = 1 - 2^{0} = 0$. Suppose $m > 0$. We know $s_{2m + 1} = s_{2m} + \frac{1}{2} = \frac{s_{2(m - 1) + 1}}{2} + \frac{1}{2}$. It follows that
    \[
      \frac{s_{2(m - 1) + 1}}{2} + \frac{1}{2} = \frac{1 - 2^{-(m - 1)}}{2} + \frac{1}{2} = 1 - 2^{-m},
    \]
    by induction. Hence $s_{2m + 1} = 1 - 2^{-m}$, and thus $s_{2m} = s_{2m+1} - \frac{1}{2} = \frac{1}{2} - 2^{-m}$. By Theorem $3.20$,
    \[
      \lim_{m \to \infty} s_{2m + 1} = \lim_{m \to \infty} (1 - 2^{-m}) = 1,
    \]
    \[
      \lim_{m \to \infty} s_{2m} = \lim_{m \to \infty} \left(\frac{1}{2} - 2^{-m}\right) = \frac{1}{2}.
    \]
    Since subsequences of $s_n$ contains either a subsequence of $s_{2m}$ or a subsequence of $s_{2m + 1}$, any convergence sequence converges to either $1$ or $\frac{1}{2}$. Therefore, the upper limit and lower limit of $(s_n)$ are $1$ and $\frac{1}{2}$, respectively.
  \end{proof}
\end{homeworkProblem}

\newpage

\begin{homeworkProblem}
  For any two real sequences $(a_n), (b_n)$, prove that
  \[
    \limsup_{n \to \infty} (a_n + b_n) \leq \limsup_{n \to \infty} a_n + \limsup_{n \to \infty} b_n,
  \]
  provided the sum on the right is not of the form $\infty - \infty$.

  \begin{proof}
    The inequality obviously holds for the case $\limsup_{n \to \infty} a_n = \infty$ and $\limsup_{n \to \infty} b_n > -\infty$. 
    
    Suppose $\limsup_{n \to \infty} a_n = -\infty$ and $\limsup_{n \to \infty} b_n < \infty$. Then, there are no subsequential limits for $a_n$ and $b_n$ is bounded above by some $b$. Consider subequence $(a_{n_k} + b_{n_k})$. Suppose for the sake of contradiction that $(a_{n_k} + b_{n_k})$ converges at some point $p$. Let $r > 0$. Since $a_n$ has no subsequential limits, there are only at most finitely many values of $n$ such that $a_n > p - r - b$. It follows that the neighborhood $N_{r}(p)$ only contains at most finitely many values of $n$ such that $a_{n} + b_{n} \in N_{r}(p)$, contradiction. Hence, $$\limsup_{n \to \infty} (a_n + b_n) = \limsup_{n \to \infty} a_n + \limsup_{n \to \infty} b_n = -\infty,$$ and the inequality holds.

    It remains to show the case for $\limsup_{n \to \infty} a_n = p$ and $\limsup_{n \to \infty} b_n = q$, for some $p, q \in \R$. Since both $a_n$ and $b_n$ have subsequential limits, $a_n$ and $b_n$ are bounded. It follows that $(a_n + b_n)$ are also bounded, so $\limsup_{n \to \infty} (a_n + b_n) = r$, for some $r \in \R$, by Theorem 3.6. Theorem 3.7 shows that there exists subsequence $(a_{n_k} + b_{n_k})$ such that $a_{n_k} + b_{n_k} \to r$. Since $a_{n_k}$ is bounded, there exists subsequence $a_{n_{k_p}}$ of $a_{n_k}$ such that $a_{n_{k_p}} \to \limsup_{k \to \infty} a_{n_k}$. The subsequence $(a_{n_{k_p}} + b_{n_{k_p}})$ of $(a_{n_k} + b_{n_k})$ also converges to $r$. By Theorem 3.3, $\lim_{p \to \infty} a_{n_{k_p}} + \lim_{p \to \infty} b_{n_{k_p}} = \lim_{p \to \infty} (a_{n_{k_p}} + b_{n_{k_p}})$, and so $b_{n_{k_p}}$ is also a convergence sequence. Hence, we have shown the existence of convergence subsequences $a_{n_{k_p}}$ and $b_{n_{k_p}}$. It immediately follows that
    \[
      r = \lim_{n \to \infty} (a_{n_k} + b_{n_k}) = \lim_{p \to \infty} (a_{n_{k_p}} + b_{n_{k_p}}) =  \lim_{p \to \infty} a_{n_{k_p}} + \lim_{p \to \infty} b_{n_{k_p}} \leq p + q,
    \]
    and this completes the proof.
  \end{proof}
\end{homeworkProblem}

\newpage

\begin{homeworkProblem}
  If $(s_n)$ is a complex sequence, define its arithmetic means $\sigma_n$ by
  \[
    \sigma_n = \frac{s_0 + s_1 + \ldots + s_n}{n + 1} \quad (n = 0,1,2,\ldots).
  \]
  \begin{enumerate}[(a)]
    \item If $\lim s_n = s$, prove that $\lim \sigma_n = s$.
    \begin{proof}
      Fix $\epsilon > 0$. There exsits $N$ such that for all $n \geq N$, $|s - s_n| < \frac{\epsilon}{2}$. Pick integer $N'$ such that $\frac{\epsilon}{2} N' > \sum_{i = 0}^{N - 1} |s - s_i|$. Then for $n \geq \max(N, N')$,
      \begin{align*}
        |s - \sigma_n|
        &= \left|s - \frac{1}{n + 1}\sum_{i = 0}^n s_i\right| \\
        &\leq \frac{1}{n + 1}\sum_{i = 0}^n |s - s_i| \\
        &= \frac{1}{n + 1}\left(\sum_{i = 0}^{N - 1} |s - s_i| + \sum_{i = N}^n |s - s_i|\right) \\
        &< \frac{1}{n + 1}\left(\sum_{i = 0}^{N - 1} |s - s_i| + (n - N + 1)\frac{\epsilon}{2} \right) \\
        &= \frac{\sum_{i = 0}^{N - 1} |s - s_i|}{n + 1} + \frac{n - N + 1}{n + 1} \cdot \frac{\epsilon}{2} \\
        &\leq \frac{\epsilon}{2} + \frac{\epsilon}{2} = \epsilon.
      \end{align*}
      Hence, $\sigma_n \to s$.
    \end{proof}
    \item Construct a sequence $(s_n)$ which does not converge, although $\lim \sigma_n = 0$.
    \begin{proof}
      Consider $(s_n)$, with $s_1 = 1$, $s_{2k} = -1$, and $s_{2k + 1} = 1$. $s_n$ obviously does not converge. Since
      \begin{align*}
        \sigma_n 
        &= \begin{cases}
          \frac{1}{n} \left(\sum_{i = 1}^k 1 +  \sum_{i = 1}^k -1\right) & n = 2k, $ for some $k \in \N \\
          \frac{1}{n} \left(1 + \sum_{i = 1}^k 1 +  \sum_{i = 1}^k -1\right) & n = 2k + 1, $ for some $k \in \N
        \end{cases} \\
        &= \begin{cases}
          0 & n = 2k, $ for some $k \in \N \\
          \frac{1}{n} & n = 2k + 1, $ for some $k \in \N
        \end{cases},
      \end{align*}
      we get $\sigma_n \to 0$.
    \end{proof}
  \end{enumerate}
\end{homeworkProblem}

\newpage

\begin{homeworkProblem}
  Fix a positive number $\alpha$. Choose $x_1 > \sqrt{\alpha}$, and define $x_2, x_3, x_4, \ldots$ by the recursion formula
  \[
    x_{n+1} = \frac{1}{2} \left( x_n + \frac{\alpha}{x_n} \right).
  \]
  Prove that $(x_n)$ decreases monotonically and that $\lim x_n = \sqrt{\alpha}$.

  \begin{proof}
    We show that $x_n > \sqrt{\alpha}$ by induction on $n$. $x_1 > \sqrt{\alpha}$, obviously. Suppose $n > 1$. By induction, $x_{n - 1} > \sqrt{\alpha}$, and the induction result then follows from
    \[
      \frac{(x_{n - 1} - \sqrt{\alpha})^2}{2x_{n - 1}} = \frac{1}{2}\left(x_{n - 1} + \frac{\alpha}{x_{n - 1}}\right) - \sqrt{\alpha} = x_{n} - \sqrt{\alpha} > 0.
    \]
    Notice that since $x_n^2 > \alpha$, we substitute $\alpha$ from the recursion formula and get $x_{n + 1} < x_n$, and thus $x_n$ is monotonically decreasing. It remains to show $x_n \to \sqrt{\alpha}$. Note that $\lim x_n = \lim x_{n + 1} = a$, for some $a \geq \sqrt{\alpha}$. But then $a = \frac{1}{2}\left( a + \frac{\alpha}{a} \right)$, the solving the equation gives us $a = \sqrt{\alpha}$, and we are done.
  \end{proof}
\end{homeworkProblem}

\newpage

\begin{homeworkProblem}
  Fix $\alpha > 1$. Take $x_1 > \sqrt{\alpha}$ and define
  \[
    x_{n+1} = \frac{\alpha + x_n}{1 + x_n} = x_n + \frac{\alpha - x_n^2}{1 + x_n}.
  \]
  \begin{enumerate}[(a)]
    \item Prove that $x_1 > x_3 > x_5 > \ldots$.
    \begin{proof}
      We first note that
      \begin{gather}
        x_{n + 1} = \frac{\alpha + x_n}{1 + x_n} = \frac{\alpha + \left(\frac{\alpha + x_{n - 1}}{1 + x_{n - 1}}\right)}{1 + \left(\frac{\alpha + x_{n - 1}}{1 + x_{n - 1}}\right)} = \frac{2\alpha + (1 + \alpha)x_{n - 1}}{(1 + \alpha) + 2x_{n - 1}} = x_{n - 1} + \delta_n,
      \end{gather}
      where $\delta_n = \frac{\alpha - x_{n - 1}^2}{\frac{1}{2}(1 + \alpha) + x_{n - 1}}$. Hence, if $x_{n - 1} > \sqrt{\alpha}$, then $\delta_n < 0$ and thus $x_{n + 1} < x_{n - 1}$. Otherwise, we have $\delta_n > 0$, and so $x_{n + 1} > x_{n - 1}$.

      Let $a_m = x_{2m - 1}$, for $m \geq 1$. We now show that $a_{m} > \sqrt{\alpha}$ by induction on $m$. The base case is clear. Suppose $m > 1$. By induction, $a_{m - 1} > \sqrt{\alpha}$, and so $a_{m} - a_{m - 1} = \delta_m < 0$. Hence, $a_m$ is monotonically decreasing.
    \end{proof}
    \item Prove that $x_2 < x_4 < x_6 < \ldots$.
    \begin{proof}
      Similar to (a), we show that $b_m = x_{2m} < \sqrt{\alpha}$ by induction on $m$. We first prove the base case $m = 1$. Let $\epsilon = x_1 - \sqrt{\alpha} > 0$. Then,
      \[
        x_2 = x_1 + \frac{\alpha - x_1^2}{1 + x_1} = x_1 + \frac{(\sqrt{\alpha} - x_1)(\sqrt{\alpha} + x_1)}{1 + x_1} = x_1 - \frac{\sqrt{\alpha} + x_1}{1 + x_1} \cdot \epsilon.
      \]
      It follows that $\frac{\sqrt{\alpha} + x_1}{1 + x_1} > 1$, so $x_2 < x_1 - \epsilon = \sqrt{\alpha}$. Suppose $m > 1$. Define $\delta_m$ the way we did in (1). By induction, $b_{m - 1} < \sqrt{\alpha}$, and so $b_{m} - b_{m - 1} = \delta_m > 0$. Hence, $b_m$ is monotonically increasing.
    \end{proof}
    \item Prove that $\lim x_n = \sqrt{\alpha}$.
    \begin{proof}
      We show that both subsequences $a_n$ and $b_n$ converge to $\sqrt{\alpha}$. Since both $a_n$ and $b_n$ are bounded and monotomic, by Theorem 3.14, $a_n \to a$ and $b_n \to b$, where $a \geq \sqrt{\alpha} \geq b$. Notice that $\lim a_n = \lim a_{n + 1} = a$ and $\lim b_n = \lim b_{n + 1} = b$. By (1),
      \[
        a = a + \lim \delta_m = a + \frac{\alpha - a^2}{\frac{1}{2}(1 + \alpha) + a},
      \]
      \[
        b = b + \lim \delta_m = b + \frac{\alpha - b^2}{\frac{1}{2}(1 + \alpha) + b},
      \]
      and solving the equations gives us $a = b = \sqrt{\alpha}$. Take $\gamma > 0$. There exists $m_a$ and $m_b$ such that $|a_k - \sqrt{\alpha}|, |b_l - \sqrt{\alpha}| < \gamma$, for all $k > m_a$ and $l > m_b$. Hence, for all $n \geq \max(m_a, m_b)$, we have $|x_n - \sqrt{\alpha}| < \gamma$, and the result follows.
    \end{proof}
  \end{enumerate}
\end{homeworkProblem}

\newpage

\begin{homeworkProblem}
  Suppose $(p_n)$ is a Cauchy sequence in a metric space $X$, and some subsequence $(p_{n_i})$ converges to a point $p \in X$. Prove that the full sequence $(p_n)$ converges to $p$.

  \begin{proof}
    Fix $\epsilon > 0$. There exists integer $N$ such that $d(p_n, p_m) < \frac{\epsilon}{2}$, for all $m ,n \geq N$. Since $(p_{n_i})$ converges, there exists $N'$ such that $d(p_{n_i}, p) < \frac{\epsilon}{2}$, for all $i \geq N'$. Hence, for all $n \geq N$, pick $i > N'$ such that $n_i \geq N$ and we have
    \[
      d(p_n, p) \leq d(p_n, p_{n_1}) + d(p_{n_i}, p) < \frac{\epsilon}{2} + \frac{\epsilon}{2} = \epsilon,
    \]
    and the result follows.
  \end{proof}
\end{homeworkProblem}
\end{document}