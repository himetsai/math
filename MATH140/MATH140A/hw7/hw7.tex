\documentclass{article}

\usepackage{fancyhdr}
\usepackage{extramarks}
\usepackage{amsmath}
\usepackage{amsthm}
\usepackage{amsfonts}
\usepackage{tikz}
\usepackage[plain]{algorithm}
\usepackage{algpseudocode}
\usepackage{enumerate}
\usepackage{amssymb}

\usetikzlibrary{automata,positioning}

%
% Basic Document Settings
%

\topmargin=-0.45in
\evensidemargin=0in
\oddsidemargin=0in
\textwidth=6.5in
\textheight=9.0in
\headsep=0.25in

\linespread{1.1}

\pagestyle{fancy}
\lhead{\hmwkAuthorName}
\chead{\hmwkClass:\ \hmwkTitle}
\rhead{\firstxmark}
\lfoot{\lastxmark}
\cfoot{\thepage}

\renewcommand\headrulewidth{0.4pt}
\renewcommand\footrulewidth{0.4pt}

\setlength\parindent{0pt}
\setlength{\parskip}{5pt}

%
% Create Problem Sections
%

\newcommand{\enterProblemHeader}[1]{
    \nobreak\extramarks{}{Problem \arabic{#1} continued on next page\ldots}\nobreak{}
    \nobreak\extramarks{Problem \arabic{#1} (continued)}{Problem \arabic{#1} continued on next page\ldots}\nobreak{}
}

\newcommand{\exitProblemHeader}[1]{
    \nobreak\extramarks{Problem \arabic{#1} (continued)}{Problem \arabic{#1} continued on next page\ldots}\nobreak{}
    \stepcounter{#1}
    \nobreak\extramarks{Problem \arabic{#1}}{}\nobreak{}
}

\setcounter{secnumdepth}{0}
\newcounter{partCounter}
\newcounter{homeworkProblemCounter}
\setcounter{homeworkProblemCounter}{1}
\nobreak\extramarks{Problem \arabic{homeworkProblemCounter}}{}\nobreak{}

%
% Homework Problem Environment
%
% This environment takes an optional argument. When given, it will adjust the
% problem counter. This is useful for when the problems given for your
% assignment aren't sequential. See the last 3 problems of this template for an
% example.
%
\newenvironment{homeworkProblem}[1][-1]{
    \ifnum#1>0
        \setcounter{homeworkProblemCounter}{#1}
    \fi
    \section{Problem \arabic{homeworkProblemCounter}}
    \setcounter{partCounter}{1}
    \enterProblemHeader{homeworkProblemCounter}
}{
    \exitProblemHeader{homeworkProblemCounter}
}

%
% Homework Details
%   - Title
%   - Due date
%   - Class
%   - Section/Time
%   - Instructor
%   - Author
%

\newcommand{\hmwkTitle}{Homework\ \#7}
\newcommand{\hmwkDueDate}{Mar 4, 2024}
\newcommand{\hmwkClass}{MATH 140A}
\newcommand{\hmwkClassInstructor}{Professor Seward}
\newcommand{\hmwkAuthorName}{\textbf{Ray Tsai}}
\newcommand{\hmwkPID}{A16848188}

%
% Title Page
%

\title{
    \vspace{2in}
    \textmd{\textbf{\hmwkClass:\ \hmwkTitle}}\\
    \normalsize\vspace{0.1in}\small{Due\ on\ \hmwkDueDate\ at 23:59pm}\\
    \vspace{0.1in}\large{\textit{\hmwkClassInstructor}} \\
    \vspace{3in}
}

\author{
  \hmwkAuthorName \\
  \vspace{0.1in}\small\hmwkPID
}
\date{}

\renewcommand{\part}[1]{\textbf{\large Part \Alph{partCounter}}\stepcounter{partCounter}\\}

%
% Various Helper Commands
%

% Useful for algorithms
\newcommand{\alg}[1]{\textsc{\bfseries \footnotesize #1}}

% For derivatives
\newcommand{\deriv}[1]{\frac{\mathrm{d}}{\mathrm{d}x} (#1)}

% For partial derivatives
\newcommand{\pderiv}[2]{\frac{\partial}{\partial #1} (#2)}

% Integral dx
\newcommand{\dx}{\mathrm{d}x}

% Probability commands: Expectation, Variance, Covariance, Bias
\newcommand{\Var}{\mathrm{Var}}
\newcommand{\Cov}{\mathrm{Cov}}
\newcommand{\Bias}{\mathrm{Bias}}
\newcommand*{\Z}{\mathbb{Z}}
\newcommand*{\Q}{\mathbb{Q}}
\newcommand*{\R}{\mathbb{R}}
\newcommand*{\C}{\mathbb{C}}
\newcommand*{\N}{\mathbb{N}}
\newcommand*{\prob}{\mathds{P}}
\newcommand*{\E}{\mathds{E}}

\begin{document}

\maketitle

\pagebreak

\begin{homeworkProblem}
  Investigate the behavior (convergence or divergence) of $\sum a_n$ if

  \begin{enumerate}[(a)]
      \item $a_n = \sqrt{n+1} - \sqrt{n}$;
      \begin{proof}
        \begin{align*}
          \sum^{\infty}_{n = 1} a_n 
          &=  \sqrt{2} - \sqrt{1} + \sqrt{3} - \sqrt{2} + \dots + \sqrt{n} - \sqrt{n - 1} + \sqrt{n + 1} - \sqrt{n} \\
          &= \sqrt{n + 1} - 1.
        \end{align*}
        Hence, $\lim_{n \to \infty} (\sqrt{n + 1} - 1) = \infty$.
      \end{proof}
      \item $a_n = \left(\sqrt{n+1} - \sqrt{n}\right) / n$;
      \begin{proof}
        Notice $$a_n = \frac{\sqrt{n + 1} - \sqrt{n}}{n} = \frac{1}{n\sqrt{n + 1} + \sqrt{n}} <
        \frac{1}{n\sqrt{n}} = \frac{1}{n^{\frac{3}{2}}}.$$ By Theorem 3.28, $\sum
        \frac{1}{n^{\frac{3}{2}}}$ converges, and thus $\sum a_n$ converges by the compatison test.
      \end{proof}
      \item $a_n = \left( \sqrt[n]{n} - 1 \right)^n$;
      \begin{proof}
        Since
        \[
          \lim_{n \to \infty} \sqrt[n]{|a_n|} = \lim_{n \to \infty} \sqrt[n]{n} - 1 = 1 - 1 = 0, 
        \]
        $\sum a_n$ converges by the root test.
      \end{proof}
      \item $a_n = 1 / (1 + z^n)$, for complex values of $z$.
      \begin{proof}
        Suppose $|z| \leq 1$. Since
        \[
          |a_n| = \left|\frac{1}{1 + z^n}\right| \geq \frac{1}{1 + |z|^n} \geq \frac{1}{2},
        \]
        $a_n$ does not converge to $0$, and thus $\sum a_n$ diverges.

        Suppose $|z| > 1$. Notice
        \[
          |a_n| = \left|\frac{1}{1 + z^n}\right| \leq \frac{1}{|z^n|} = \left|\frac{1}{z}\right|^n.
        \]
        Since $\lim_{n \to \infty} \sqrt[n]{\left|\frac{1}{z}\right|^n} = \frac{1}{|z|} < 1$, $\sum
        \left|\frac{1}{z}\right|^n$ converges by the root test, and thus $a_n$ converges by the
        comparison test.
      \end{proof}
  \end{enumerate}
\end{homeworkProblem}

\newpage

\begin{homeworkProblem}
  Prove that the convergence of $\sum a_n$ implies the convergence of $\sum \frac{\sqrt{a_n}}{n}$,
  if $a_n \geq 0$.

  \begin{proof}
    Note that both $\sum a_n$ and $\sum \frac{1}{n^2}$ converges absolutely. By the Cauchy-Schwarz
    inequality, 
    \[
      \left(\sum \frac{\sqrt{a_n}}{n}\right)^2 \leq \sum a_n \sum \frac{1}{n^2}.
    \]
    Since $\sum a_n \sum \frac{1}{n^2}$ converges, $\sum \frac{\sqrt{a_n}}{n}$ is bounded. But then
    $\sum \frac{\sqrt{a_n}}{n}$ is a series of nonnegative terms, so it converges.
  \end{proof}
\end{homeworkProblem}

\newpage

\begin{homeworkProblem}
  If $\sum a_n$ converges and if $(b_n)$ is monotonic and bounded, prove that $\sum a_n b_n$
  converges.

  \begin{proof}
    Since $(b_n)$ is monotonic and bounded, $(b_n)$ converges. Put $A = \sum^{\infty}_{n = 1} a_n$
    and $B = \lim b_n$. Let $c_n = b_n - B$ if $b_n$ monotonically decreases. Otherwise, let $c_n =
    B - b_n$. In this way, we guarantee $c_n$ is monotonically decreasing and $c_n \to 0$. Then,
    \[
      \sum^{\infty}_{n = 1} a_nb_n = \sum^{\infty}_{n = 1} a_n(b_n - B + B),
    \]
    so $\sum^{\infty}_{n = 1} a_nb_n = \sum^{\infty}_{n = 1} a_nc_n \pm AB$, depending on whether
    $b_n$ monotonically increases or decreases. Since $\sum a_n$ converges, it follows from Theorem
    3.42 that $\sum a_nc_n$ converges, and thus $\sum a_nb_n$ converges.
  \end{proof}
\end{homeworkProblem}

\newpage

\begin{homeworkProblem}
  Find the radius of convergence of each of the following power series:
  \begin{enumerate}[(a)]
    \item $\sum n^3 z^n$
    \begin{proof}
      Since
      \begin{align*}
        \limsup_{n \to \infty} \sqrt[n]{n^3} = \limsup_{n \to \infty} {(\sqrt[n]{n})^3} = 1,
      \end{align*}
      the radius of convergence is $1$.
    \end{proof}
    \item $\sum \frac{2^n}{n!} z^n$
    \begin{proof}
      Since
      \begin{align*}
        \limsup_{n \to \infty} \left|\frac{a_{n + 1}}{a_n}\right| = \limsup_{n \to \infty} \left|\frac{2}{n + 1}\right| = 0
      \end{align*}
      The radius of convergence is $\infty$.
    \end{proof}
    \item $\sum \frac{2^n}{n^2} z^n$
    \begin{proof}
      Since
      \begin{align*}
        \limsup_{n \to \infty} \left|\frac{a_{n + 1}}{a_n}\right| = \limsup_{n \to \infty} 2\left(\frac{n}{n + 1}\right)^2 = 2
      \end{align*}
      The radius of convergence is $\frac{1}{2}$.
    \end{proof}
    \item $\sum \frac{n^3}{3^n} z^n$.
    \begin{proof}
      Since
      \begin{align*}
        \limsup_{n \to \infty} \left|\frac{a_{n + 1}}{a_n}\right| = \limsup_{n \to \infty} \frac{1}{3}\left(\frac{n + 1}{n}\right)^2 = \frac{1}{3}
      \end{align*}
      The radius of convergence is $3$.
    \end{proof}
  \end{enumerate}
\end{homeworkProblem}

\newpage

\begin{homeworkProblem}
  Suppose that the coefficients of the power series $\sum a_n z^n$ are integers, infinitely many of
  which are distinct from zero. Prove that the radius of convergence is at most 1.

  \begin{proof}
    Since infinitely many of $a_n$ are distinct from zero, $\limsup_{n \to \infty} |a_n z^n| \geq \limsup_{n \to \infty} |z|^n \geq 1$
    when $z > 0$. But then $|a_n z^n|$ does not converge to $0$, so $\sum a_n z^n$ diverges when
    $|z| \geq 1$.
  \end{proof}
\end{homeworkProblem}

\newpage

\begin{homeworkProblem}
  Prove the following analogue of Theorem 3.10(b): If $(E_n)$ is a sequence of closed, nonempty, and
  bounded sets in a \textit{complete} metric space $X$, if $E_n \supset E_{n+1}$, and if
  \[
    \lim_{n \to \infty} \text{diam } E_n = 0,
  \]
  then $\bigcap_{n=1}^{\infty} E_n$ consists of exactly one point.

  \begin{proof}
    Since $E_n$ is nonempty, let $(p_n)$ be a sequence such that $a_n \in E_n$ for all $n$. Let
    $K_N$ contain the points $p_n, p_{N + 1}, \ldots$. Since $K_N \subset E_N$ and $E_N$ is bounded, 
    \[
      \lim_{n \to \infty} \text{diam } K_n \leq \lim_{n \to \infty} \text{diam } E_n = 0,
    \]
    and so $(p_n)$ is a Cauchy sequence. Since $X$ is complete, $p_n$ converges to some point $p \in
    X$. Note that $p$ is a limit point of every $E_n$. But then $E_n$ is closed, so $p \in E_n$ for
    all $n$, that is, $p \in \bigcap_{n=1}^{\infty} E_n$. Suppose for the sake of contradiction that
    $\bigcap_{k=1}^{\infty} E_k$ contains two distinct points $p, q$. But since
    $\bigcap_{k=1}^{\infty} E_k \subset E_n$ for all $n$, 
    \[
      0 < \text{diam} \bigcap_{k=1}^{\infty} E_k \leq \text{diam } E_n,
    \]
    and so $\text{diam } E_n$ does not converge to $0$, contradiction.
  \end{proof}
\end{homeworkProblem}

\newpage

\begin{homeworkProblem}
  Suppose $X$ is a nonempty complete metric space, and $(G_n)$ is a sequence of dense open subsets
  of $X$. Prove Baire's theorem, namely, that $\bigcap_{n=1}^{\infty} G_n$ is not empty (In fact, it
  is dense in $X$). \textit{Hint}: Find a shrinking sequence of neighborhoods $E_n$ such that
  $\overline{E}_n \subset G_n$, and apply Exercise 3.21.

  \begin{proof}
    We inductively construct sequence of open sets $(E_n)$. Since $G_1$ is open and nonempty, let
    $x_1 \in G_1$. There exists small enough $\epsilon_1 > 0$ such that $\text{diam }
    N_{\epsilon_1}(x_1) < 1$ and $\overline{N_{\epsilon_1}(x_1)} \subset G_1$. Put $E_1 =
    N_{\epsilon_1}(x_1)$. 
    
    Suppose that $E_n$ is constructed. Since $G_{n + 1}$ is dense and open, $E_n \cap G_{n + 1}$ is
    nonempty and open. Let $x_{n + 1} \in E_n \cap G_{n + 1}$. There exists small enough
    $\epsilon_{n + 1}$ such that $\text{diam } N_{\epsilon_{n + 1}}(x_{n + 1}) < \frac{1}{n + 1}$
    and $\overline{N_{\epsilon_{n + 1}}(x_{n + 1})} \subset E_n \cap G_{n + 1}$. Put $E_{n + 1} =
    N_{\epsilon_{n + 1}}(x_{n + 1})$. 

    Thus, we have constructed a sequence of closed, nonempty, and bounded sets $\overline{E_n}$.
    Since $G_n \supset \overline{E_n} \supset \overline{E_{n + 1}}$ and
    \[
      \lim_{n \to \infty} \text{diam } \overline{E_n} = \lim_{n \to \infty} \text{diam } E_n \leq \lim_{n \to \infty} \frac{1}{n} = 0,
    \]
    it follows from Exercise 3.21 that $\bigcap_{n = 1}^{\infty} G_n \supset \bigcap_{n =
    1}^{\infty} \overline{E_n} \neq \emptyset$.
  \end{proof}
\end{homeworkProblem}
\end{document}