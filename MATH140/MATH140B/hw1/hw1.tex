\documentclass{article}

\usepackage{fancyhdr}
\usepackage{extramarks}
\usepackage{amsmath}
\usepackage{amsthm}
\usepackage{amsfonts}
\usepackage{tikz}
\usepackage[plain]{algorithm}
\usepackage{algpseudocode}
\usepackage{enumerate}
\usepackage{amssymb}

\usetikzlibrary{automata,positioning}

%
% Basic Document Settings
%

\topmargin=-0.45in
\evensidemargin=0in
\oddsidemargin=0in
\textwidth=6.5in
\textheight=9.0in
\headsep=0.25in

\linespread{1.1}

\pagestyle{fancy}
\lhead{\hmwkAuthorName}
\chead{\hmwkClass:\ \hmwkTitle}
\rhead{\firstxmark}
\lfoot{\lastxmark}
\cfoot{\thepage}

\renewcommand\headrulewidth{0.4pt}
\renewcommand\footrulewidth{0.4pt}

\setlength\parindent{0pt}
\setlength{\parskip}{5pt}

%
% Create Problem Sections
%

\newcommand{\enterProblemHeader}[1]{
    \nobreak\extramarks{}{Problem \arabic{#1} continued on next page\ldots}\nobreak{}
    \nobreak\extramarks{Problem \arabic{#1} (continued)}{Problem \arabic{#1} continued on next page\ldots}\nobreak{}
}

\newcommand{\exitProblemHeader}[1]{
    \nobreak\extramarks{Problem \arabic{#1} (continued)}{Problem \arabic{#1} continued on next page\ldots}\nobreak{}
    \stepcounter{#1}
    \nobreak\extramarks{Problem \arabic{#1}}{}\nobreak{}
}

\setcounter{secnumdepth}{0}
\newcounter{partCounter}
\newcounter{homeworkProblemCounter}
\setcounter{homeworkProblemCounter}{1}
\nobreak\extramarks{Problem \arabic{homeworkProblemCounter}}{}\nobreak{}

%
% Homework Problem Environment
%
% This environment takes an optional argument. When given, it will adjust the
% problem counter. This is useful for when the problems given for your
% assignment aren't sequential. See the last 3 problems of this template for an
% example.
%
\newenvironment{homeworkProblem}[1][-1]{
    \ifnum#1>0
        \setcounter{homeworkProblemCounter}{#1}
    \fi
    \section{Problem \arabic{homeworkProblemCounter}}
    \setcounter{partCounter}{1}
    \enterProblemHeader{homeworkProblemCounter}
}{
    \exitProblemHeader{homeworkProblemCounter}
}

%
% Homework Details
%   - Title
%   - Due date
%   - Class
%   - Section/Time
%   - Instructor
%   - Author
%

\newcommand{\hmwkTitle}{Homework\ \#1}
\newcommand{\hmwkDueDate}{Apr 12, 2024}
\newcommand{\hmwkClass}{MATH 140B}
\newcommand{\hmwkClassInstructor}{Professor Seward}
\newcommand{\hmwkAuthorName}{\textbf{Ray Tsai}}
\newcommand{\hmwkPID}{A16848188}

%
% Title Page
%

\title{
    \vspace{2in}
    \textmd{\textbf{\hmwkClass:\ \hmwkTitle}}\\
    \normalsize\vspace{0.1in}\small{Due\ on\ \hmwkDueDate\ at 23:59pm}\\
    \vspace{0.1in}\large{\textit{\hmwkClassInstructor}} \\
    \vspace{3in}
}

\author{
  \hmwkAuthorName \\
  \vspace{0.1in}\small\hmwkPID
}
\date{}

\renewcommand{\part}[1]{\textbf{\large Part \Alph{partCounter}}\stepcounter{partCounter}\\}

%
% Various Helper Commands
%

% Useful for algorithms
\newcommand{\alg}[1]{\textsc{\bfseries \footnotesize #1}}

% For derivatives
\newcommand{\deriv}[1]{\frac{\mathrm{d}}{\mathrm{d}x} (#1)}

% For partial derivatives
\newcommand{\pderiv}[2]{\frac{\partial}{\partial #1} (#2)}

% Integral dx
\newcommand{\dx}{\mathrm{d}x}

% Probability commands: Expectation, Variance, Covariance, Bias
\newcommand{\Var}{\mathrm{Var}}
\newcommand{\Cov}{\mathrm{Cov}}
\newcommand{\Bias}{\mathrm{Bias}}
\newcommand*{\Z}{\mathbb{Z}}
\newcommand*{\Q}{\mathbb{Q}}
\newcommand*{\R}{\mathbb{R}}
\newcommand*{\C}{\mathbb{C}}
\newcommand*{\N}{\mathbb{N}}
\newcommand*{\prob}{\mathds{P}}
\newcommand*{\E}{\mathds{E}}

\begin{document}

\maketitle

\pagebreak

\begin{homeworkProblem}
  Suppose $f'(x) > 0$ in $(a, b)$. Prove that $f$ is strictly increasing on $(a, b)$, and let $g$ be its inverse function. Prove that $g$ is differentiable and that
  \[
  g'(f(x)) = \frac{1}{f'(x)} \quad (a < x < b).
  \]


  \begin{proof}
    Suppose for contradiction that there exists $x, y \in (a, b)$ such that $y > x$ but $f(y) <
    f(x)$. Since $f$ is differentiable in $(x, y)$, there exists $w \in (x, y)$ such that $(y -
    x)f'(w) = f(y) - f(x)$, by the Mean Value Theorem. But then $f'(w) < 0$, contradiction.

    Pick any arbitrary closed set $S \subset (a, b)$. By Theorem 2.41, $S$ is compact. By Theorem
    4.14, $f(S)$ is compact and thus closed in the domain of $g$. By Theorem 4.8, $g$ is continuous.
    Let $y = f(x)$ and $s = f(t)$, such that $s \neq y$. Since $g$ is continuous, $t \to x$ as $s
    \to y$. Note that $\frac{1}{f'(x)}$ exists. Hence,
    \[
      g'(f(x)) = \lim_{s \to y} \frac{g(s) - g(y)}{s - y} = \lim_{t \to x} \frac{1}{\frac{f(t) - f(x)}{t - x}} = \frac{1}{\lim\limits_{t \to x} \frac{f(t) - f(x)}{t - x}} = \frac{1}{f'(x)},
    \]
    and $g$ is differentiable, as $f$ is strictly increasing and thus injective.
  \end{proof}
\end{homeworkProblem}

\newpage

\begin{homeworkProblem}
  If 
  \[
    C_0 + \frac{C_1}{2} + \ldots + \frac{C_{n-1}}{n} + \frac{C_n}{n+1} = 0,
  \]
  where $C_0, \ldots, C_n$ are real constants, prove that the equation
  \[
    C_0 + C_1 x + \ldots + C_{n-1}x^{n-1} + C_n x^n = 0
  \]
  has at least one real root between 0 and 1.

  \begin{proof}
    Define $f(x) = \sum_{k = 0}^n \frac{C_kx^{k + 1}}{k + 1}$. $f$ is differentiable as it is a real
    polynomial, and $f'(x) = C_0 + C_1 x + \ldots + C_{n-1}x^{n-1} + C_n x^n$. But then $f(0) = f(1)
    = 0$, and the result now follows from the mean value theorem.
  \end{proof}
\end{homeworkProblem}

\newpage

\begin{homeworkProblem}
  Suppose
  \begin{enumerate}[(a)]
    \item $f$ is continuous for $x \geq 0$,
    \item $f'(x)$ exists for $x > 0$,
    \item $f(0) = 0$,
    \item $f'$ is monotonically increasing.
  \end{enumerate}
  Put 
  \[ 
    g(x) = \frac{f(x)}{x} \quad (x > 0) 
  \]
  and prove that $g$ is monotonically increasing.

  \begin{proof}
    Notice that $g(x) = \frac{f(x) - f(0)}{x - 0}$. By the mean value theorem, there exists $w \in
    (0, x)$ such that $f'(w) = g(x)$. Since $f'$ is monotonically increasing and $x > w$, $f'(x)
    \geq f'(w) = \frac{f(x)}{x}$, and so $xf'(x) - f(x) \geq 0$. But then $g$ is differentiable and
    $g'(x) = \frac{xf'(x) - f(x)}{x^2} \geq 0$, by Theorem 5.3. Pick any $a, b$ such that $b > a >
    0$. By the mean value theorem, $g(b) - g(a) = (b - a)g'(p) \geq 0$, for some $p > 0$, and the
    result follows.
  \end{proof}
\end{homeworkProblem}

\newpage

\begin{homeworkProblem}
  Suppose $f'$ is continuous on $[a, b]$ and $\varepsilon > 0$. Prove that there exists $\delta > 0$
  such that
  \[
    \left| \frac{f(t) - f(x)}{t - x} - f'(x) \right| < \varepsilon
  \]
  whenever $0 < |t - x| < \delta$, $a \leq x \leq b$, $a \leq t \leq b$. (This could be expressed by
  saying that $f$ is uniformly differentiable on $[a, b]$ if $f'$ is continuous on $[a, b]$.) Does
  this hold for vector-valued functions too?

  \begin{proof}
    Since $f'$ is continuous on a compact set, $f'$ is uniformly continuous, by Theorem 4.19. That
    is, there exists $\delta > 0$ such that $|f'(y) - f'(x)| < \epsilon$, for all $|y - x| < \delta$,
    $y, x \in [a, b]$. By the mean value theorem, for any $x, t \in [a, b]$ such that $0 < |x - t| <
    \delta$, 
    \[
      \left| \frac{f(t) - f(x)}{t - x} - f'(x) \right| = \left| f'(w) - f'(x) \right| < \epsilon,
    \]
    for some $w$ between $x$ and $t$, as $|w - x| < |t - x| < \delta$.

    Since this holds for all components, it also holds for vector-valued functions, by Theorem 3.4.
  \end{proof}
\end{homeworkProblem}

\newpage

\begin{homeworkProblem}
  Let $f$ be a continuous real function on $\mathbb{R}^1$, of which it is known that $f'(x)$ exists
  for all $x \neq 0$ and that $f'(x) \rightarrow 3$ as $x \rightarrow 0$. Does it follow that
  $f'(0)$ exists?

  \begin{proof}
    Since $f$ is continuous and differentiable on $\R \backslash 0$, for all $x \neq 0$ we have
    \[
      \frac{f(x) - f(0)}{x} = f'(w),
    \]
    for some $w$ between $0$ and $x$. But then 
    \[
      f'(0) = \lim_{x \to 0} \frac{f(x) - f(0)}{x} = \lim_{x \to 0} f'(w) = 3,
    \]
    since $w \to 0$ as $x \to 0$. The result now follows.
  \end{proof}
\end{homeworkProblem}

\newpage

\begin{homeworkProblem}
  Suppose $f$ is a real, three times differentiable function on $[-1,1]$, such that
  \[
    f(-1) = 0, \quad f(0) = 0, \quad f(1) = 1, \quad f'(0) = 0.
  \]
  Prove that $f^{(3)}(x) \geq 3$ for some $x \in (-1,1)$. Note that equality holds for $\frac{x^3 +
  x^2}{2}$. \textit{Hint:} Use Theorem 5.15, with $\alpha = 0$ and $\beta = \pm1$, to show that
  there exists $s \in (0,1)$ and $t \in (-1,0)$ such that
  \begin{gather}
    f^{(3)}(s) + f^{(3)}(t) = 6.
  \end{gather}

  \begin{proof}
    Define function $P$ over $[-1, 1]$ as
    \[
      P(x) = \sum_{k = 0}^{2} \frac{f^{(k)}(0)}{k!} \cdot x^k = \frac{f''(0)}{2} \cdot x^2.
    \]
    By Taylor's Theorem, there exists $s \in (0, 1)$ and $t \in (-1, 0)$ such that 
    \[
      f(1) = P(1) + \frac{f^{(3)}(s)}{6}, \quad \text{and} \quad f(-1) = P(-1) - \frac{f^{(3)}(t)}{6}.
    \]
    Note that $P(1) = P(-1)$. Combining both equations, we get 
    \[
      f(1) - f(-1) = 1 = \frac{f^{(3)}(s)}{6} + \frac{f^{(3)}(t)}{6},
    \]
    and (1) follows. Since the average of $f^{(3)}(s)$ and $f^{(3)}(t)$ is 3, one of them must be at
    least 3.
  \end{proof}
\end{homeworkProblem}

\newpage

\begin{homeworkProblem}
  Suppose $f$ is differentiable on $[a, b]$, $f(a) = 0$, and there is a real number $A$ such that
  $|f'(x)| \leq A |f(x)|$ on $[a, b]$. Prove that $f(x) = 0$ for all $x \in [a, b]$.

  \begin{proof}
    Fix $x_0 \in [a, b]$, and let
    \[
      M_0 = \sup |f(x)|, \quad M_1 = \sup |f'(x)|
    \]
    for $a \leq x \leq x_0$. For any such $x$,
    \[
      |f(x)| \leq M_1 (x_0 - a) \leq A(x_0 - a)M_0.
    \]
    Hence $M_0 = 0$ if $A(x_0 - a) < 1$. That is, $f = 0$ on $[a, x_0]$. To achieve this, we may
    pick $x_0 = a + \frac{1}{2A}$. Hence, it remains to show that $f = 0$ on $[x_0, b]$. Again, we
    may pick $x_1 = x_0 + \frac{1}{2A} = a + 2 \cdot \frac{1}{2A}$. Then, $A(x_1 - x_0) =
    \frac{1}{2} < 1$, and thus it remains to show that $f = 0$ on $[x_1, b]$, and so on. Let natural
    number $n > 2A(b - a)$. Since $x_n = a + n \cdot \frac{1}{2A} > b$, the above process would
    reach $b$ and terminate after $n$ steps, which concludes that $f(x) = 0$ for all $x \in [a, b]$. 
  \end{proof}
\end{homeworkProblem}
\end{document}