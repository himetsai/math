\documentclass{article}

\usepackage{fancyhdr}
\usepackage{extramarks}
\usepackage{amsmath}
\usepackage{amsthm}
\usepackage{amsfonts}
\usepackage{tikz}
\usepackage[plain]{algorithm}
\usepackage{algpseudocode}
\usepackage{enumerate}
\usepackage{amssymb}
\usepackage{mathrsfs}
\usepackage{mathtools}
\usepackage{amsmath}

\usetikzlibrary{automata,positioning}

%
% Basic Document Settings
%

\topmargin=-0.45in
\evensidemargin=0in
\oddsidemargin=0in
\textwidth=6.5in
\textheight=9.0in
\headsep=0.25in

\linespread{1.1}

\pagestyle{fancy}
\lhead{\hmwkAuthorName}
\chead{\hmwkClass:\ \hmwkTitle}
\rhead{\firstxmark}
\lfoot{\lastxmark}
\cfoot{\thepage}

\renewcommand\headrulewidth{0.4pt}
\renewcommand\footrulewidth{0.4pt}

\setlength\parindent{0pt}
\setlength{\parskip}{5pt}

%
% Create Problem Sections
%

\newcommand{\enterProblemHeader}[1]{
    \nobreak\extramarks{}{Problem \arabic{#1} continued on next page\ldots}\nobreak{}
    \nobreak\extramarks{Problem \arabic{#1} (continued)}{Problem \arabic{#1} continued on next page\ldots}\nobreak{}
}

\newcommand{\exitProblemHeader}[1]{
    \nobreak\extramarks{Problem \arabic{#1} (continued)}{Problem \arabic{#1} continued on next page\ldots}\nobreak{}
    \stepcounter{#1}
    \nobreak\extramarks{Problem \arabic{#1}}{}\nobreak{}
}

\setcounter{secnumdepth}{0}
\newcounter{partCounter}
\newcounter{homeworkProblemCounter}
\setcounter{homeworkProblemCounter}{1}
\nobreak\extramarks{Problem \arabic{homeworkProblemCounter}}{}\nobreak{}

%
% Homework Problem Environment
%
% This environment takes an optional argument. When given, it will adjust the
% problem counter. This is useful for when the problems given for your
% assignment aren't sequential. See the last 3 problems of this template for an
% example.
%
\newenvironment{homeworkProblem}[1][-1]{
    \ifnum#1>0
        \setcounter{homeworkProblemCounter}{#1}
    \fi
    \section{Problem \arabic{homeworkProblemCounter}}
    \setcounter{partCounter}{1}
    \enterProblemHeader{homeworkProblemCounter}
}{
    \exitProblemHeader{homeworkProblemCounter}
}

%
% Homework Details
%   - Title
%   - Due date
%   - Class
%   - Section/Time
%   - Instructor
%   - Author
%

\newcommand{\hmwkTitle}{Homework\ \#6}
\newcommand{\hmwkDueDate}{May 17, 2024}
\newcommand{\hmwkClass}{MATH 140B}
\newcommand{\hmwkClassInstructor}{Professor Seward}
\newcommand{\hmwkAuthorName}{\textbf{Ray Tsai}}
\newcommand{\hmwkPID}{A16848188}

%
% Title Page
%

\title{
    \vspace{2in}
    \textmd{\textbf{\hmwkClass:\ \hmwkTitle}}\\
    \normalsize\vspace{0.1in}\small{Due\ on\ \hmwkDueDate\ at 23:59pm}\\
    \vspace{0.1in}\large{\textit{\hmwkClassInstructor}} \\
    \vspace{3in}
}

\author{
  \hmwkAuthorName \\
  \vspace{0.1in}\small\hmwkPID
}
\date{}

\renewcommand{\part}[1]{\textbf{\large Part \Alph{partCounter}}\stepcounter{partCounter}\\}

%
% Various Helper Commands
%

% define norm \norm{...}:
\DeclarePairedDelimiterX\norm[1]\lVert\rVert{{#1}}

% Useful for algorithms
\newcommand{\alg}[1]{\textsc{\bfseries \footnotesize #1}}

% For derivatives
\newcommand{\deriv}[1]{\frac{\mathrm{d}}{\mathrm{d}x} (#1)}

% For partial derivatives
\newcommand{\pderiv}[2]{\frac{\partial}{\partial #1} (#2)}

% Integral dx
\newcommand{\dx}{\mathrm{d}x}

% Probability commands: Expectation, Variance, Covariance, Bias
\newcommand{\Var}{\mathrm{Var}}
\newcommand{\Cov}{\mathrm{Cov}}
\newcommand{\Bias}{\mathrm{Bias}}
\newcommand*{\Z}{\mathbb{Z}}
\newcommand*{\Q}{\mathbb{Q}}
\newcommand*{\R}{\mathbb{R}}
\newcommand*{\C}{\mathbb{C}}
\newcommand*{\N}{\mathbb{N}}
\newcommand*{\prob}{\mathds{P}}
\newcommand*{\E}{\mathds{E}}

\begin{document}

\maketitle

\pagebreak

\begin{homeworkProblem}
  Suppose $g$ and $f_n$ ($n = 1, 2, 3, \ldots$) are defined on $(0, \infty)$, are Riemann-integrable
  on $[t, T]$ whenever $0 < t < T < \infty$, $|f_n| \leq g$, $f_n \rightarrow f$ uniformly on every
  compact subset of $(0, \infty)$, and
  \[
    \int_0^\infty g(x) \, dx < \infty.
  \]
  Prove that
  \[
    \lim_{n \to \infty} \int_0^\infty f_n(x) \, dx = \int_0^\infty f(x) \, dx.
  \]
  \begin{proof}
    Since $|f_n| \leq g$ and $\int_t^\infty g(x) \, dx < \infty$, we know $\int_t^\infty f_n(x) \,
    dx$ exists. Define $h_k = \int_{1/k}^\infty f(x) \, dx$. Pick $\epsilon > 0$. Since
    $\int_0^\infty g(x) \, dx$ exists, there exists $N$ such that $\int_0^{1/n} g(x) \, dx <
    \epsilon/2$ for all $n \geq N$. Let $\beta > \alpha \geq N$. There exists large enough $n$ such
    that $|f(x) - f_n(x)| < \epsilon(\beta - \alpha)/2\alpha\beta$ for all $x \in [1/\beta,
    1/\alpha]$. But then,
    \begin{align*}
      |h_\beta - h_\alpha|
      &= \left|\int_{1/\beta}^{1/\alpha} f(x) \, dx\right| \\
      &\leq \int_{1/\beta}^{1/\alpha} |f(x) - f_n(x)| + |f_n(x)| \, dx \\
      &\leq \int_{1/\beta}^{1/\alpha} |f(x) - f_n(x)| \, dx +  \int_{1/\beta}^{1/\alpha} g(x) \, dx \\
      &< \epsilon/2 + \epsilon/2 = \epsilon,
    \end{align*}
    Thus, $h_k$ converges uniformly by Cauchy criterion, and so $\int_t^\infty f(x) \, dx$ exists
    for $t \in (0, \infty)$.
    
    Let $I_n(t) = \int_t^\infty f_n(x) \, dx$ and let $I(t) = \int_t^\infty f(x) \, dx$. We show
    that $I_n \to I$ uniformly on $(0, \infty)$. Again, pick $\epsilon > 0$. There exists $t_1, t_2
    \in (0, \infty)$, $t_2 > t_1$, such that $\int_0^{t_1} g(x) \, dx < \epsilon/6$ and
    $\int_{t_2}^{\infty} g(x) \, dx < \epsilon/6$. Since $f_n$ converges to $f$ uniformly, there
    exists $N$ such that $|f_n(x) - f(x)| < \epsilon/3(t_2 - t_1)$ for all $n \geq N$ and $x \in
    [t_1, t_2]$. Hence, for all $n \geq N$ and $t \in (0, \infty)$, let $t' \in (0, \min(t, t_1))$
    and we have
    \begin{align*}
      \left|I(t) - I_n(t)\right|
      &\leq \int_{t'}^{\infty} |f(x) - f_n(x)| \, dx \\
      &= \int_{t'}^{t_1} |f(x) - f_n(x)| \, dx + \int_{t_1}^{t_2} |f(x) - f_n(x)| \, dx + \int_{t_2}^{\infty} |f(x) - f_n(x)| \, dx \\
      &\leq 2\int_{0}^{t_1} g(x) \, dx + \int_{t_1}^{t_2} |f(x) - f_n(x)| \, dx + 2\int_{t_2}^{\infty} g(x) \, dx \\
      &< \epsilon/3 + \epsilon/3 + \epsilon/3 = \epsilon.
    \end{align*}
    Hence, $I_n \to I$ uniformly on $(0, \infty)$. By Theorem 7.11, 
    \[
      \lim_{n \to \infty} \lim_{t \to 0} I_n(t) = \lim_{t \to 0} I(t).
    \]
  \end{proof}
\end{homeworkProblem}

\newpage

\begin{homeworkProblem}
  Assume that $(f_n)$ is a sequence of monotonically increasing functions on $\mathbb{R}$ with $0
  \leq f_n(x) \leq 1$ for all $x$ and all $n$. 

  Prove that there is a function $f$ and a sequence $(n_k)$ such that
  \[
    f(x) = \lim_{k \to \infty} f_{n_k}(x)
  \]
  for every $x \in \mathbb{R}^1$. (The existence of such a pointwise convergent subsequence is
  usually called \textit{Helly's selection theorem}.) 

  \begin{proof}
    
  \end{proof}
\end{homeworkProblem}

\newpage

\begin{homeworkProblem}
  Suppose $f$ is a real continuous function on $\mathbb{R}$, $f_n(t) = f(nt)$ for $n = 1, 2, 3,
  \ldots$, and $(f_n)$ is equicontinuous on $[0, 1]$. What conclusion can you draw about $f$?

  \begin{proof}
    Pick $\epsilon > 0$. Since $(f_n)$ is equicontinuous on $[0, 1]$, there exists $\delta > 0$ such
    that
    \[
      |f(nx) - f(ny)| < \epsilon,
    \]
    whenever $|x - y| < \delta$, for all $n$ and $x, y \in [0, 1]$. Let $s, t \in [0, \infty]$. Put
    integer $n > \max(|s - t|/\delta, s, t)$. Since $|\frac{s}{n} - \frac{t}{n}| = |s - t|/n <
    \delta$ and $\frac{s}{n}, \frac{t}{n} \in [0, 1]$, we have 
    \[
      |f(s) - f(t)| < \epsilon.
    \]
    But then $\epsilon$ is arbitrary, so $f$ is constant on $[0, \infty]$.
  \end{proof}
\end{homeworkProblem}

\newpage

\begin{homeworkProblem}
  Let $(f_n)$ be a uniformly bounded sequence of functions which are Riemann-integrable on $[a, b]$,
  and put
  \[
  F_n(x) = \int_a^x f_n(t) \, dt \quad (a \leq x \leq b).
  \]
  Prove that there exists a subsequence $(F_{n_k})$ which converges uniformly on $[a, b]$.

  \begin{proof}
    We show that $(F_{n})$ is point-wise bounded and equicontinuous on $[a, b]$. 
    
    Since $(f_n)$ is uniformly bounded, there exists $K$ such that $|f_n(x)| < K$ for all $x$ and
    $n$. Hence,
    \[
      |F_n(x)| \leq \int_a^x |f_n(t)| \, dt < K(x - a),
    \]
    and so $F_n$ is point-wise bounded. 

    Pick $\epsilon > 0$. Let $\delta = \epsilon/K(b - a)$. Then,
    \[
      |F_n(x) - F_n(y)| \leq \int_{x}^y |f_n(t)| \, dt \leq K(y - x) < \epsilon,
    \]
    for all $n \in \N$ and $x, y \in [a, b]$ such that $\delta > y - x > 0$. Hence, $(F_n)$ is
    equicontinuous. 
    
    The result now follows from Theorem 7.25.
  \end{proof}
\end{homeworkProblem}

\newpage

\begin{homeworkProblem}
  Let $K$ be a compact metric space, let $S$ be a subset of $\mathscr{C}(K)$. Prove that $S$ is
  compact (with respect to the metric defined in Definition 7.14) if and only if $S$ is uniformly
  closed, pointwise bounded, and equicontinuous. (If $S$ is not equicontinuous, then $S$ contains a
  sequence which has no equicontinuous subsequence, hence has no subsequence that converges
  uniformly on $K$.)

  \begin{proof}
    Suppose $S$ is compact. $S$ is closed so $S$ is uniformly closed by definition. 
    
    Pick $\epsilon > 0$ and pick $\nu \in (0, \epsilon/2)$. Consider the open cover $\{B_{\nu}
    (f)\}_{f \in S}$. There exists $\{f_i\}_{i = 1}^n$ such that $\bigcup_{i = 1}^n B_{\nu} (f_i)
    \supset S$. Since $K$ is compact, $f$ is uniformly continuous for all $f \in \mathscr{C}(K)$.
    Hence, for each $f_i$, there exists $\delta_i$ such that $|f_i(x) - f_i(y)| < \epsilon - 2\nu$
    for all $x, y \in K$ such that $d(x, y) < \delta_i$. Put $\delta = \min_{1 \leq i \leq n}
    \delta_i$. Let $g \in S$. Then $\sup_{x \in K} |g(x) - f_i(x)| < \nu$ for some $f_i$. But then 
    \begin{align*}
      |g(x) - g(y)| 
      &\leq |g(x) - f_i(x)| + |f_i(x) - f_i(y)| + |f_i(y) - g(y)| \\
      &< \nu + \epsilon - 2\nu + \nu = \epsilon.
    \end{align*}
    Define $\Phi(x) = \nu + \max_{1 \leq i \leq n} |f_i(x)|$. Then, $f(x) < \Phi(x)$ for all $f \in
    S$, so $S$ is pointwise bounded. 

    Now suppose that $S$ is uniformly closed, pointwise bounded, and equicontinuous. Let $T \subset
    S$ be an infinite subset of $S$. By Theorem 7.25, $T$ contains a uniformly convergent sequence,
    which converges to some $f \in S$ as $S$ is uniformly closed. Hence, every infinite subset of
    $S$ has a limit point in $S$, and thus $S$ is compact.
  \end{proof}
\end{homeworkProblem}

\newpage

\begin{homeworkProblem}
  If $f$ is continuous on $[0,1]$ and if
  \[
    \int_0^1 f(x) x^n \, dx = 0 \quad (n = 0, 1, 2, \ldots),
  \]
  prove that $f(x) = 0$ on $[0,1]$. 

  \begin{proof}
    By the Weierstrass Theorem, there exists a sequence of polynomials $P_n$ which converges to $f$
    uniformly on $[0, 1]$. By the given identity,
    \[
      \int_0^1 f(x) P_n(x) \, dx = 0.
    \]
    But then
    \begin{align*}
      \int_0^1 f^2(x) \, dx = \lim_{n \to \infty} \int_0^1 f(x) P_n(x) \, dx = 0,
    \end{align*}
    and the result now follows.
  \end{proof}
\end{homeworkProblem}
\end{document}