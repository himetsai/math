\documentclass{article}

\usepackage{fancyhdr}
\usepackage{extramarks}
\usepackage{amsmath}
\usepackage{amsthm}
\usepackage{amsfonts}
\usepackage{tikz}
\usepackage[plain]{algorithm}
\usepackage{algpseudocode}
\usepackage{enumerate}
\usepackage{amssymb}
\usepackage{mathrsfs}
\usepackage{mathtools}
\usepackage{amsmath}

\usetikzlibrary{automata,positioning}

%
% Basic Document Settings
%

\topmargin=-0.45in
\evensidemargin=0in
\oddsidemargin=0in
\textwidth=6.5in
\textheight=9.0in
\headsep=0.25in

\linespread{1.1}

\pagestyle{fancy}
\lhead{\hmwkAuthorName}
\chead{\hmwkClass:\ \hmwkTitle}
\rhead{\firstxmark}
\lfoot{\lastxmark}
\cfoot{\thepage}

\renewcommand\headrulewidth{0.4pt}
\renewcommand\footrulewidth{0.4pt}

\setlength\parindent{0pt}
\setlength{\parskip}{5pt}

%
% Create Problem Sections
%

\newcommand{\enterProblemHeader}[1]{
    \nobreak\extramarks{}{Problem \arabic{#1} continued on next page\ldots}\nobreak{}
    \nobreak\extramarks{Problem \arabic{#1} (continued)}{Problem \arabic{#1} continued on next page\ldots}\nobreak{}
}

\newcommand{\exitProblemHeader}[1]{
    \nobreak\extramarks{Problem \arabic{#1} (continued)}{Problem \arabic{#1} continued on next page\ldots}\nobreak{}
    \stepcounter{#1}
    \nobreak\extramarks{Problem \arabic{#1}}{}\nobreak{}
}

\setcounter{secnumdepth}{0}
\newcounter{partCounter}
\newcounter{homeworkProblemCounter}
\setcounter{homeworkProblemCounter}{1}
\nobreak\extramarks{Problem \arabic{homeworkProblemCounter}}{}\nobreak{}

%
% Homework Problem Environment
%
% This environment takes an optional argument. When given, it will adjust the
% problem counter. This is useful for when the problems given for your
% assignment aren't sequential. See the last 3 problems of this template for an
% example.
%
\newenvironment{homeworkProblem}[1][-1]{
    \ifnum#1>0
        \setcounter{homeworkProblemCounter}{#1}
    \fi
    \section{Problem \arabic{homeworkProblemCounter}}
    \setcounter{partCounter}{1}
    \enterProblemHeader{homeworkProblemCounter}
}{
    \exitProblemHeader{homeworkProblemCounter}
}

%
% Homework Details
%   - Title
%   - Due date
%   - Class
%   - Section/Time
%   - Instructor
%   - Author
%

\newcommand{\hmwkTitle}{Homework\ \#6}
\newcommand{\hmwkDueDate}{May 17, 2024}
\newcommand{\hmwkClass}{MATH 140B}
\newcommand{\hmwkClassInstructor}{Professor Seward}
\newcommand{\hmwkAuthorName}{\textbf{Ray Tsai}}
\newcommand{\hmwkPID}{A16848188}

%
% Title Page
%

\title{
    \vspace{2in}
    \textmd{\textbf{\hmwkClass:\ \hmwkTitle}}\\
    \normalsize\vspace{0.1in}\small{Due\ on\ \hmwkDueDate\ at 23:59pm}\\
    \vspace{0.1in}\large{\textit{\hmwkClassInstructor}} \\
    \vspace{3in}
}

\author{
  \hmwkAuthorName \\
  \vspace{0.1in}\small\hmwkPID
}
\date{}

\renewcommand{\part}[1]{\textbf{\large Part \Alph{partCounter}}\stepcounter{partCounter}\\}

%
% Various Helper Commands
%

% define norm \norm{...}:
\DeclarePairedDelimiterX\norm[1]\lVert\rVert{{#1}}

% Useful for algorithms
\newcommand{\alg}[1]{\textsc{\bfseries \footnotesize #1}}

% For derivatives
\newcommand{\deriv}[1]{\frac{\mathrm{d}}{\mathrm{d}x} (#1)}

% For partial derivatives
\newcommand{\pderiv}[2]{\frac{\partial}{\partial #1} (#2)}

% Integral dx
\newcommand{\dx}{\mathrm{d}x}

% Probability commands: Expectation, Variance, Covariance, Bias
\newcommand{\Var}{\mathrm{Var}}
\newcommand{\Cov}{\mathrm{Cov}}
\newcommand{\Bias}{\mathrm{Bias}}
\newcommand*{\Z}{\mathbb{Z}}
\newcommand*{\Q}{\mathbb{Q}}
\newcommand*{\R}{\mathbb{R}}
\newcommand*{\C}{\mathbb{C}}
\newcommand*{\N}{\mathbb{N}}
\newcommand*{\prob}{\mathds{P}}
\newcommand*{\E}{\mathds{E}}

\begin{document}

\maketitle

\pagebreak

\begin{homeworkProblem}
  Suppose $g$ and $f_n$ ($n = 1, 2, 3, \ldots$) are defined on $(0, \infty)$, are Riemann-integrable
  on $[t, T]$ whenever $0 < t < T < \infty$, $|f_n| \leq g$, $f_n \rightarrow f$ uniformly on every
  compact subset of $(0, \infty)$, and
  \[
    \int_0^\infty g(x) \, dx < \infty.
  \]
  Prove that
  \[
    \lim_{n \to \infty} \int_0^\infty f_n(x) \, dx = \int_0^\infty f(x) \, dx.
  \]
  \begin{proof}
    Since $|f_n| \leq g$ and $\int_t^\infty g(x) \, dx < \infty$, we know $\int_t^\infty f_n(x) \,
    dx$ exists. Define $h_k = \int_{1/k}^\infty f(x) \, dx$. Pick $\epsilon > 0$. Since
    $\int_0^\infty g(x) \, dx$ exists, there exists $N$ such that $\int_0^{1/n} g(x) \, dx <
    \epsilon/2$ for all $n \geq N$. Let $\beta > \alpha \geq N$. There exists large enough $n$ such
    that $|f(x) - f_n(x)| < \epsilon(\beta - \alpha)/2\alpha\beta$ for all $x \in [1/\beta,
    1/\alpha]$. But then,
    \begin{align*}
      |h_\beta - h_\alpha|
      &= \left|\int_{1/\beta}^{1/\alpha} f(x) \, dx\right| \\
      &\leq \int_{1/\beta}^{1/\alpha} |f(x) - f_n(x)| + |f_n(x)| \, dx \\
      &\leq \int_{1/\beta}^{1/\alpha} |f(x) - f_n(x)| \, dx +  \int_{1/\beta}^{1/\alpha} g(x) \, dx \\
      &< \epsilon/2 + \epsilon/2 = \epsilon,
    \end{align*}
    and thus $h_k$ converges uniformly by Cauchy criterion. Let $I_n = \int_0^\infty f_n(x) \, dx$
    and let $I = \int_0^\infty f_n(x) \, dx$. 
    
    It remains to show that $\int_0^\infty
    f(x) \, dx$ converges to $\lim_{n \to \infty} \int_0^\infty f_n(x) \, dx$.
    \[
      \left|\int_0^\infty f_n(x) - f(x) \, dx\right| < \epsilon,
    \]
    for large enough $n$.
  \end{proof}
\end{homeworkProblem}
\end{document}