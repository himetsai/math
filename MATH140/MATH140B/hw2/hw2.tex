\documentclass{article}

\usepackage{fancyhdr}
\usepackage{extramarks}
\usepackage{amsmath}
\usepackage{amsthm}
\usepackage{amsfonts}
\usepackage{tikz}
\usepackage[plain]{algorithm}
\usepackage{algpseudocode}
\usepackage{enumerate}
\usepackage{amssymb}

\usetikzlibrary{automata,positioning}

%
% Basic Document Settings
%

\topmargin=-0.45in
\evensidemargin=0in
\oddsidemargin=0in
\textwidth=6.5in
\textheight=9.0in
\headsep=0.25in

\linespread{1.1}

\pagestyle{fancy}
\lhead{\hmwkAuthorName}
\chead{\hmwkClass:\ \hmwkTitle}
\rhead{\firstxmark}
\lfoot{\lastxmark}
\cfoot{\thepage}

\renewcommand\headrulewidth{0.4pt}
\renewcommand\footrulewidth{0.4pt}

\setlength\parindent{0pt}
\setlength{\parskip}{5pt}

%
% Create Problem Sections
%

\newcommand{\enterProblemHeader}[1]{
    \nobreak\extramarks{}{Problem \arabic{#1} continued on next page\ldots}\nobreak{}
    \nobreak\extramarks{Problem \arabic{#1} (continued)}{Problem \arabic{#1} continued on next page\ldots}\nobreak{}
}

\newcommand{\exitProblemHeader}[1]{
    \nobreak\extramarks{Problem \arabic{#1} (continued)}{Problem \arabic{#1} continued on next page\ldots}\nobreak{}
    \stepcounter{#1}
    \nobreak\extramarks{Problem \arabic{#1}}{}\nobreak{}
}

\setcounter{secnumdepth}{0}
\newcounter{partCounter}
\newcounter{homeworkProblemCounter}
\setcounter{homeworkProblemCounter}{1}
\nobreak\extramarks{Problem \arabic{homeworkProblemCounter}}{}\nobreak{}

%
% Homework Problem Environment
%
% This environment takes an optional argument. When given, it will adjust the
% problem counter. This is useful for when the problems given for your
% assignment aren't sequential. See the last 3 problems of this template for an
% example.
%
\newenvironment{homeworkProblem}[1][-1]{
    \ifnum#1>0
        \setcounter{homeworkProblemCounter}{#1}
    \fi
    \section{Problem \arabic{homeworkProblemCounter}}
    \setcounter{partCounter}{1}
    \enterProblemHeader{homeworkProblemCounter}
}{
    \exitProblemHeader{homeworkProblemCounter}
}

%
% Homework Details
%   - Title
%   - Due date
%   - Class
%   - Section/Time
%   - Instructor
%   - Author
%

\newcommand{\hmwkTitle}{Homework\ \#2}
\newcommand{\hmwkDueDate}{Apr 19, 2024}
\newcommand{\hmwkClass}{MATH 140B}
\newcommand{\hmwkClassInstructor}{Professor Seward}
\newcommand{\hmwkAuthorName}{\textbf{Ray Tsai}}
\newcommand{\hmwkPID}{A16848188}

%
% Title Page
%

\title{
    \vspace{2in}
    \textmd{\textbf{\hmwkClass:\ \hmwkTitle}}\\
    \normalsize\vspace{0.1in}\small{Due\ on\ \hmwkDueDate\ at 23:59pm}\\
    \vspace{0.1in}\large{\textit{\hmwkClassInstructor}} \\
    \vspace{3in}
}

\author{
  \hmwkAuthorName \\
  \vspace{0.1in}\small\hmwkPID
}
\date{}

\renewcommand{\part}[1]{\textbf{\large Part \Alph{partCounter}}\stepcounter{partCounter}\\}

%
% Various Helper Commands
%

% Useful for algorithms
\newcommand{\alg}[1]{\textsc{\bfseries \footnotesize #1}}

% For derivatives
\newcommand{\deriv}[1]{\frac{\mathrm{d}}{\mathrm{d}x} (#1)}

% For partial derivatives
\newcommand{\pderiv}[2]{\frac{\partial}{\partial #1} (#2)}

% Integral dx
\newcommand{\dx}{\mathrm{d}x}

% Probability commands: Expectation, Variance, Covariance, Bias
\newcommand{\Var}{\mathrm{Var}}
\newcommand{\Cov}{\mathrm{Cov}}
\newcommand{\Bias}{\mathrm{Bias}}
\newcommand*{\Z}{\mathbb{Z}}
\newcommand*{\Q}{\mathbb{Q}}
\newcommand*{\R}{\mathbb{R}}
\newcommand*{\C}{\mathbb{C}}
\newcommand*{\N}{\mathbb{N}}
\newcommand*{\prob}{\mathds{P}}
\newcommand*{\E}{\mathds{E}}

\begin{document}

\maketitle

\pagebreak

\begin{homeworkProblem}
  Suppose $f$ is defined in a neighborhood of $x$, and suppose $f''(x)$ exists. Show that
  \[
  \lim_{{h \to 0}} \frac{f(x + h) + f(x - h) - 2f(x)}{h^2} = f''(x).
  \]
  Show by example that the limit may exist even if $f''(x)$ does not. 

  \begin{proof}
    Put $g(h) = f(x + h) + f(x - h) - 2f(x)$. Since $g$ is differentiable in a neighborhood of $x$
    and $g(h) \to 0$ as $h \to 0$, we may apply the L'Hospotal's Rule and get
    \begin{align*}
      \lim_{{h \to 0}} \frac{g(h)}{h^2} 
      &= \lim_{{h \to 0}} \frac{f'(x + h) - f'(x - h)}{2h} \\
      &= \lim_{{h \to 0}} \frac{f'(x + h) - f'(x)}{2h} - \lim_{{h \to 0}} \frac{f'(x - h) - f'(x)}{2h} \\
      &= \lim_{{h \to 0}} \frac{f'(x + h) - f'(x)}{2h} - \lim_{{k \to 0}} \frac{f'(x + k) - f'(x)}{-2k} \\
      &= \frac{f''(x)}{2} + \frac{f''(x)}{2} = f''(x).
    \end{align*}

    Consider $f(x) = \begin{cases}
      1 & x > 0 \\
      0 & x = 0 \\
      -1 & x < 0 \end{cases}$. $f$ is not continuous at $0$, so $f''(0)$ does not exist. But then
    $f(h) + f(-h) - 2f(0) = 0$ for all $h > 0$, so $\lim_{{h \to 0}} \frac{f(x + h) + f(x - h) -
    2f(x)}{h^2}$ exists.
  \end{proof}
\end{homeworkProblem}

\newpage

\begin{homeworkProblem}
  Suppose $a \in \mathbb{R}^1$, $f$ is a twice-differentiable real function on $(a, \infty)$, and
  $M_0$, $M_1$, $M_2$ are the least upper bounds of $|f(x)|$, $|f'(x)|$, $|f''(x)|$, respectively,
  on $(a, \infty)$. Prove that
  \[
    M_1^2 \leq 4M_0M_2.
  \]
  \textit{Hint}: If $h > 0$, Taylor's theorem shows that
  \[
    f'(x) = \frac{1}{2h} [f(x + 2h) - f(x)] - hf''(\xi)
  \]
  for some $\xi$ in $(x, x + 2h)$. Hence
  \[
    |f'(x)| \leq hM_2 + \frac{M_0}{h}.
  \]
  To show that $M_1^2 = 4M_0M_2$ can actually happen, take $a = -1$, define
  \[
    f(x) = \begin{cases} 
      2x^2 - 1 & (-1 < x < 0),\\
      \frac{x^2 - 1}{x^2 + 1} & (0 \leq x < \infty),
    \end{cases}
  \]
  and show that $M_0 = 1$, $M_1 = 4$, and $M_2 = 4$. Does $M_1^2 \leq 4M_0M_2$ hold for
  vector-valued functions too?

  \begin{proof}
    
  \end{proof}
\end{homeworkProblem}

\newpage

\begin{homeworkProblem}
  Suppose $f$ is a real function on $(-\infty, \infty)$. Call $x$ a fixed point of $f$ if $f(x) =
  x$.

  \begin{enumerate}[(a)]
    \item If $f$ is differentiable and $f'(t) \neq 1$ for every real $t$, prove that $f$ has at most
    one fixed point.
    \item Show that the function $f$ defined by
    \[
    f'(t) = t + (1 + e^t)^{-1}
    \]
    has no fixed point, although $0 < f'(t) < 1$ for all real $t$.
    \item However, if there is a constant $A < 1$ such that $|f'(t)| \leq A$ for all real $t$, prove
    that a fixed point of $f$ exists, and that $x = \lim_{n \to \infty} x_n$, where $x_1$ is an
    arbitrary real number and
    \[
    x_{n+1} = f(x_n)
    \]
    for $n = 1, 2, 3, \ldots$.
    \item Show that the process described in (c) can be visualized by the zig-zag path
    \[
    (x_1, x_2) \to (x_2, x_2) \to (x_2, x_3) \to (x_3, x_3) \to (x_3, x_4) \to \ldots .
    \]
  \end{enumerate}
\end{homeworkProblem}

\newpage

\begin{homeworkProblem}
  Suppose $\alpha$ increases on $[a, b]$, $a \leq x_0 \leq b$, $\alpha$ is continuous at $x_0$,
  $f(x_0) = 1$, and $f(x) = 0$ if $x \neq x_0$. Prove that $f \in \mathcal{C}(\alpha)$ and that
  $\int f \, d\alpha = 0$.

  \begin{proof}
    
  \end{proof}
\end{homeworkProblem}
\end{document}