\documentclass{article}

\usepackage{fancyhdr}
\usepackage{extramarks}
\usepackage{amsmath}
\usepackage{amsthm}
\usepackage{amsfonts}
\usepackage{tikz}
\usepackage[plain]{algorithm}
\usepackage{algpseudocode}
\usepackage{enumerate}
\usepackage{amssymb}
\usepackage{mathrsfs}
\usepackage{mathtools}
\usepackage{amsmath}

\usetikzlibrary{automata,positioning}

%
% Basic Document Settings
%

\topmargin=-0.45in
\evensidemargin=0in
\oddsidemargin=0in
\textwidth=6.5in
\textheight=9.0in
\headsep=0.25in

\linespread{1.1}

\pagestyle{fancy}
\lhead{\hmwkAuthorName}
\chead{\hmwkClass:\ \hmwkTitle}
\rhead{\firstxmark}
\lfoot{\lastxmark}
\cfoot{\thepage}

\renewcommand\headrulewidth{0.4pt}
\renewcommand\footrulewidth{0.4pt}

\setlength\parindent{0pt}
\setlength{\parskip}{5pt}

%
% Create Problem Sections
%

\newcommand{\enterProblemHeader}[1]{
    \nobreak\extramarks{}{Problem \arabic{#1} continued on next page\ldots}\nobreak{}
    \nobreak\extramarks{Problem \arabic{#1} (continued)}{Problem \arabic{#1} continued on next page\ldots}\nobreak{}
}

\newcommand{\exitProblemHeader}[1]{
    \nobreak\extramarks{Problem \arabic{#1} (continued)}{Problem \arabic{#1} continued on next page\ldots}\nobreak{}
    \stepcounter{#1}
    \nobreak\extramarks{Problem \arabic{#1}}{}\nobreak{}
}

\setcounter{secnumdepth}{0}
\newcounter{partCounter}
\newcounter{homeworkProblemCounter}
\setcounter{homeworkProblemCounter}{1}
\nobreak\extramarks{Problem \arabic{homeworkProblemCounter}}{}\nobreak{}

%
% Homework Problem Environment
%
% This environment takes an optional argument. When given, it will adjust the
% problem counter. This is useful for when the problems given for your
% assignment aren't sequential. See the last 3 problems of this template for an
% example.
%
\newenvironment{homeworkProblem}[1][-1]{
    \ifnum#1>0
        \setcounter{homeworkProblemCounter}{#1}
    \fi
    \section{Problem \arabic{homeworkProblemCounter}}
    \setcounter{partCounter}{1}
    \enterProblemHeader{homeworkProblemCounter}
}{
    \exitProblemHeader{homeworkProblemCounter}
}

%
% Homework Details
%   - Title
%   - Due date
%   - Class
%   - Section/Time
%   - Instructor
%   - Author
%

\newcommand{\hmwkTitle}{Homework\ \#8}
\newcommand{\hmwkDueDate}{Jun 3, 2024}
\newcommand{\hmwkClass}{MATH 140B}
\newcommand{\hmwkClassInstructor}{Professor Seward}
\newcommand{\hmwkAuthorName}{\textbf{Ray Tsai}}
\newcommand{\hmwkPID}{A16848188}

%
% Title Page
%

\title{
    \vspace{2in}
    \textmd{\textbf{\hmwkClass:\ \hmwkTitle}}\\
    \normalsize\vspace{0.1in}\small{Due\ on\ \hmwkDueDate\ at 23:59pm}\\
    \vspace{0.1in}\large{\textit{\hmwkClassInstructor}} \\
    \vspace{3in}
}

\author{
  \hmwkAuthorName \\
  \vspace{0.1in}\small\hmwkPID
}
\date{}

\renewcommand{\part}[1]{\textbf{\large Part \Alph{partCounter}}\stepcounter{partCounter}\\}

%
% Various Helper Commands
%

% define norm \norm{...}:
\DeclarePairedDelimiterX\norm[1]\lVert\rVert{{#1}}

% Useful for algorithms
\newcommand{\alg}[1]{\textsc{\bfseries \footnotesize #1}}

% For derivatives
\newcommand{\deriv}[1]{\frac{\mathrm{d}}{\mathrm{d}x} (#1)}

% For partial derivatives
\newcommand{\pderiv}[2]{\frac{\partial}{\partial #1} (#2)}

% Integral dx
\newcommand{\dx}{\mathrm{d}x}

% Probability commands: Expectation, Variance, Covariance, Bias
\newcommand{\Var}{\mathrm{Var}}
\newcommand{\Cov}{\mathrm{Cov}}
\newcommand{\Bias}{\mathrm{Bias}}
\newcommand*{\Z}{\mathbb{Z}}
\newcommand*{\Q}{\mathbb{Q}}
\newcommand*{\R}{\mathbb{R}}
\newcommand*{\C}{\mathbb{C}}
\newcommand*{\N}{\mathbb{N}}
\newcommand*{\prob}{\mathds{P}}
\newcommand*{\E}{\mathds{E}}

\begin{document}

\maketitle

\pagebreak

\begin{homeworkProblem}
  If $0 < x < \frac{\pi}{2}$, prove that
  \[
    \frac{2}{\pi} < \frac{\sin x}{x} < 1.
  \]
  \begin{proof}
    Consider the function $f(x) = x - \sin x$. Since $\cos x < 1 = (x)'$ in $(0, \pi/2)$, $f'(x) = x
    - \cos x > 0$ in $(0, \pi/2)$, so $f$ is strictly increasing in $(0, \pi/2)$. But then $f(x) >
    f(0) = 0$ for all $x \in (0, \pi/2)$. It now follows that $\frac{\sin x}{x} < 1$.

    Now consider $g(x) = \frac{\sin x}{x}$. $g'(x) = \frac{x\cos x - \sin x}{x^2}$. We now show that
    $x < \tan x = \frac{\sin x}{\cos x}$ in $(0, \pi/2)$. Put $h(x) = \tan x - x$. Since $|\cos x| <
    1$ in $(0, \pi/2)$, $h'(x) = \frac{\cos^2 x + \sin^2 x}{\cos^2 x} - 1 = \frac{1}{\cos^2 x} - 1 >
    1$. But then $h(0) = 0$ and $h$ is strictly increasing, so $\tan x - x > 0$ in $(0, \pi/2)$. It
    now follows that $g'(x) < \frac{\tan x\cos x - \sin x}{x^2} = 0$ and $g(\pi/2) = \frac{2}{\pi}$,
    and thus $\frac{2}{\pi} < \frac{\sin x}{x}$ for all $0 < x < \frac{\pi}{2}$.
  \end{proof}
\end{homeworkProblem}

\newpage

\begin{homeworkProblem}
  For $n = 0, 1, 2, \ldots$ and $x$ real, prove that
  \[
    |\sin nx| \leq n |\sin x|.
  \]
  \begin{proof}
    We proceed by indcution on $n$. The base case $n = 0$ is trivial. Suppose $n \geq 1$.
    \begin{align*}
      |\sin nx|
      &= \left|\frac{1}{2i}(e^{nix} - e^{-nix})\right| \\
      &= \left|\frac{1}{2i}[(e^{(n - 1)ix} - e^{-(n - 1)ix})(e^{ix} + e^{-ix}) + (e^{(n - 1)ix} + e^{-(n - 1)ix})(e^{ix} - e^{-ix})]\right| \\
      &= [\sin (n - 1)x \cdot \cos x + \cos (n - 1)x \cdot \sin x] \\
      &\leq |\sin (n - 1)x \cdot \cos x| + |\cos (n - 1)x \cdot \sin x| \\
      &\leq |\sin(n - 1)x| + |\sin x|
    \end{align*}
    By induction,
    \begin{align*}
      |\sin nx| = |\sin(n - 1)x| + |\sin x| \leq (n - 1)|\sin x| + |\sin x| = n|\sin x|.
    \end{align*}
  \end{proof}
\end{homeworkProblem}

\newpage

\begin{homeworkProblem}
  Put $s_N = 1 + \left(\frac{1}{2}\right) + \cdots + \left(\frac{1}{N}\right)$. Prove that
  \[
  \lim_{N \to \infty} (s_N - \log N)
  \]
  exists. (The limit, often denoted by $\gamma$, is called Euler's constant. Its numerical value
  is 0.5772.... It is not known whether $\gamma$ is rational or not.)
  \begin{proof}
    Let $f_n = s_n - \log n$. Since $\frac{1}{x}$ is a decreasing function, $\int_{n}^{n + 1} \frac{1}{x} \, \dx \geq
    \frac{1}{n + 1}$. Thus,
    \[
      f_{n + 1} - f_n = \frac{1}{n + 1} - (\log (n + 1) - \log n) = \frac{1}{n + 1} - \int_{n}^{n + 1} \frac{1}{x} \, \dx \leq 0,
    \]
    and so $\{f_n\}$ is a monotonically decreasing sequence. But then $\int_{1}^n \frac{1}{x} \,
    dx \leq \sum_{k = 1}^{n - 1} \frac{1}{k}$. Hence,
    \[
      f_n = \sum_{k = 1}^n \frac{1}{k} - \int_{1}^n \frac{1}{x} \, dx \geq \frac{1}{n} > 0,
    \]
    so $f_n$ is bounded below. The result now follows from Theorem 3.14.
  \end{proof}
\end{homeworkProblem}

\newpage

\begin{homeworkProblem}
  Prove that $\sum 1/p$ diverges; the sum extends over all primes.

  \begin{proof}
    Given $N$, let $p_1, \ldots , p_k$ be those primes that divide at least one integer at most $N$.
    Each $n \leq N$ is a product of powers of $p_j$'s. Since $\prod_{j = 1}^k \left(1 +
    \frac{1}{p_j} + \frac{1}{p_j^2} + \cdots\right)$ is the sum of all inverses of numbers whose
    factorization consists of only powers of $p_j$'s, 
    \begin{align*}
      \sum_{n = 1}^N \frac{1}{n}
      &= \sum_{n = 1}^N \frac{1}{p_1^{l_1}p_2^{l_2}\cdots p_k^{l_k}} \\
      &\leq \prod_{j = 1}^k \left(1 + \frac{1}{p_j} + \frac{1}{p^2} + \cdots\right) \\
      &= \prod_{j = 1}^k \left(1 - \frac{1}{p_j}\right)^{-1}.
    \end{align*}
    We now show that $e^{2x} \geq (1 - x)^{-1}$ for $x \in (0, 1/2)$. Put $f(x) = (1 - x)e^{2x}$.
    Since $f'(x) = (1 - 2x)e^{2x} > 0$ for $x \in (0, 1/2)$ and $f(0) = 1$, we have $f(x) \geq 1$ in
    $(0, 1/2)$, and thus $e^{2x} \geq (1 - x)^{-1}$. Hence, we have
    \[
      \prod_{j = 1}^k \left(1 - \frac{1}{p_j}\right)^{-1} \leq \exp \sum_{j = 1}^k \frac{2}{p_j}.
    \]
    The logarithmic function is monotonically increasing, so we get
    \[
      \frac{1}{2}\log \left(\sum_{n = 1}^N \frac{1}{n}\right) \leq \sum_{j = 1}^k \frac{1}{p_j}.
    \]
    Since $k \to \infty$ as $N \to \infty$ and $\sum_{n = 1}^{\infty} \frac{1}{n}$ diverges,
    $\sum_{j = 1}^{\infty} \frac{1}{p_j}$ diverges, by comparison test.
  \end{proof}
\end{homeworkProblem}

\newpage

\begin{homeworkProblem}
  Suppose $f \in \mathscr{R}$ on $[0, A]$ for all $A < \infty$, and $f(x) \to 1$ as $x \to \infty$.
  Prove that
  \[
    \lim_{t \to 0} t \int_0^\infty e^{-tx} f(x) \, dx = 1 \quad (t > 0).
  \]
  \begin{proof}
    Pick $\epsilon > 0$. There exists $A$ such that $|f(x) - 1| < \epsilon$ for all $x \geq A$.
    Since $|e^{-tx}| < 1$ for all $t > 0$, 
    \[
      \lim_{t \to 0^+} t \left|\int_0^{A} e^{-tx} f(x) \, dx\right| \leq \lim_{t \to 0^+} t\int_0^A |f(x)| \, dx = 0.
    \]
    On the other hand, for $t > 0$,
    \[
      e^{-At}(1 - \epsilon) \leq \left|\int_A^{\infty} te^{-tx}(1 - \epsilon) \, dx\right| \leq t \left|\int_A^{\infty} e^{-tx} f(x) \, dx\right| \leq \left|\int_A^{\infty} te^{-tx}(1 + \epsilon) \, dx\right| \leq e^{-At}(1 + \epsilon).
    \]
    Thus, $t \left|\int_A^{\infty} e^{-tx} f(x) \, dx\right| = e^{-At}$, as $\epsilon$ is
    arbitrary. It now follows that
    \begin{align*}
      \lim_{t \to 0^+} t \left|\int_0^{\infty} e^{-tx} f(x) \, dx\right|
      &= \lim_{t \to 0^+} t \left|\int_0^{A} e^{-tx} f(x) \, dx + \int_0^{\infty} e^{-tx} f(x) \, dx\right| \\
      &\leq \lim_{t \to 0^+} t \left|\int_0^{A} e^{-tx} f(x) \, dx\right| + \lim_{t \to 0^+} t \left|\int_A^{\infty} e^{-tx} f(x) \, dx\right| \\
      &= \lim_{t \to 0^+} t \left|\int_A^{\infty} e^{-tx} f(x) \, dx\right| \\
      &= \lim_{t \to 0^+} e^{-At} \\
      &= 1.
    \end{align*}
  \end{proof}
\end{homeworkProblem}

\newpage

\begin{homeworkProblem}
  If $\alpha$ is real and $-1 < x < 1$, prove Newton's binomial theorem
  \[
    (1 + x)^\alpha = 1 + \sum_{n=1}^\infty \frac{\alpha(\alpha-1) \cdots (\alpha-n+1)}{n!} x^n.
  \]
  \begin{proof}
    Since
    \[
      \lim_{n \to \infty} \left|\frac{\frac{\alpha(\alpha-1) \cdots (\alpha-n)}{(n + 1)!} x^{n + 1}}{\frac{\alpha(\alpha-1) \cdots (\alpha-n+1)}{n!} x^n}\right| = \lim_{n \to \infty} \left|\frac{n - \alpha}{n + 1}\right||x| < 1,
    \]
    the series on the right converges in $(-1, 1)$ by the ratio test. Let $f(x)$ denote the function
    on the right-hand side. By Theorem 8.1, $f(x)$ is differentiable. Note that
    \[
      f'(x) = \sum_{n=1}^\infty \frac{\alpha(\alpha-1) \cdots (\alpha-n+1)}{(n - 1)!} x^{n - 1} = \sum_{n=0}^\infty \frac{\alpha(\alpha-1) \cdots (\alpha-n)}{n!} x^{n}.
    \]
    Hence, we have
    \begin{align*}
      (1 + x)f'(x)
      &= \sum_{n=0}^\infty \frac{\alpha(\alpha-1) \cdots (\alpha-n)}{n!} x^{n} + \sum_{n=0}^\infty \frac{\alpha(\alpha-1) \cdots (\alpha-n)}{n!} x^{n + 1} \\
      &= \alpha + \sum_{n=1}^\infty \left(\frac{\alpha(\alpha-1) \cdots (\alpha-n)}{n!} + \frac{\alpha(\alpha-1) \cdots (\alpha-n + 1)}{(n - 1)!}\right) x^{n} \\
      &= \alpha + \sum_{n=1}^\infty (n + \alpha - n)\frac{\alpha(\alpha-1) \cdots (\alpha-n + 1)}{n!} x^{n} \\
      &= \alpha + \alpha\sum_{n=1}^\infty \frac{\alpha(\alpha-1) \cdots (\alpha-n + 1)}{n!} x^{n} \\
      &= \alpha f(x).
    \end{align*}
    Since $f(0) = 1$ and $f$ is continuous, there exists $R \in (0, 1)$ such that $f(x) > 0$ in
    $(-R, R)$. Hence, $(\log f(x))' = \frac{f'(x)}{f(x)} = \frac{\alpha}{1 + x}$ in $(-R, R)$, which
    shares the same derivative with $\log (1 + x)^{\alpha}$. But then for $x \in (-R, R)$,
    \[
      \log f(x) = \log f(x) - \log f(0) = \int_{0}^x \frac{\alpha}{1 + t} \, dt = \alpha \log(1 + x) = \log(1 + x)^{\alpha},
    \]
    and so $f(x) = \exp(\log f(x)) = \exp(\log (1 + x)^{\alpha}) = (1 + x)^{\alpha}$. Now let $S =
    \{K \in (0, 1) \mid f(x) > 0 \text{ if } x \in [-K, K]\}$. Suppose for contradiction that
    $A = \sup S < 1$. We know $f(x) = (1 + x)^{\alpha}$ in $(-A, A)$. But then 
    \[
      \lim_{x \to A} f(x) = (1 + A)^{\alpha} > 0 \text{ and } \lim_{x \to -A} f(x) = (1 + -A)^{\alpha} > 0.
    \]
    By continuity, there exists $\delta$ such that $f(x) > 0$ in $(-A - \delta, A + \delta)$,
    contradiction. Hence, $f(x) = (1 + x)^{\alpha}$ in $(-1, 1)$. 
  \end{proof}
\end{homeworkProblem}
\end{document}