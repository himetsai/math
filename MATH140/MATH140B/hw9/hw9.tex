\documentclass{article}

\usepackage{fancyhdr}
\usepackage{extramarks}
\usepackage{amsmath}
\usepackage{amsthm}
\usepackage{amsfonts}
\usepackage{tikz}
\usepackage[plain]{algorithm}
\usepackage{algpseudocode}
\usepackage{enumerate}
\usepackage{amssymb}
\usepackage{mathrsfs}
\usepackage{mathtools}
\usepackage{amsmath}

\usetikzlibrary{automata,positioning}

%
% Basic Document Settings
%

\topmargin=-0.45in
\evensidemargin=0in
\oddsidemargin=0in
\textwidth=6.5in
\textheight=9.0in
\headsep=0.25in

\linespread{1.1}

\pagestyle{fancy}
\lhead{\hmwkAuthorName}
\chead{\hmwkClass:\ \hmwkTitle}
\rhead{\firstxmark}
\lfoot{\lastxmark}
\cfoot{\thepage}

\renewcommand\headrulewidth{0.4pt}
\renewcommand\footrulewidth{0.4pt}

\setlength\parindent{0pt}
\setlength{\parskip}{5pt}

%
% Create Problem Sections
%

\newcommand{\enterProblemHeader}[1]{
    \nobreak\extramarks{}{Problem \arabic{#1} continued on next page\ldots}\nobreak{}
    \nobreak\extramarks{Problem \arabic{#1} (continued)}{Problem \arabic{#1} continued on next page\ldots}\nobreak{}
}

\newcommand{\exitProblemHeader}[1]{
    \nobreak\extramarks{Problem \arabic{#1} (continued)}{Problem \arabic{#1} continued on next page\ldots}\nobreak{}
    \stepcounter{#1}
    \nobreak\extramarks{Problem \arabic{#1}}{}\nobreak{}
}

\setcounter{secnumdepth}{0}
\newcounter{partCounter}
\newcounter{homeworkProblemCounter}
\setcounter{homeworkProblemCounter}{1}
\nobreak\extramarks{Problem \arabic{homeworkProblemCounter}}{}\nobreak{}

%
% Homework Problem Environment
%
% This environment takes an optional argument. When given, it will adjust the
% problem counter. This is useful for when the problems given for your
% assignment aren't sequential. See the last 3 problems of this template for an
% example.
%
\newenvironment{homeworkProblem}[1][-1]{
    \ifnum#1>0
        \setcounter{homeworkProblemCounter}{#1}
    \fi
    \section{Problem \arabic{homeworkProblemCounter}}
    \setcounter{partCounter}{1}
    \enterProblemHeader{homeworkProblemCounter}
}{
    \exitProblemHeader{homeworkProblemCounter}
}

%
% Homework Details
%   - Title
%   - Due date
%   - Class
%   - Section/Time
%   - Instructor
%   - Author
%

\newcommand{\hmwkTitle}{Homework\ \#9}
\newcommand{\hmwkDueDate}{Jun 7, 2024}
\newcommand{\hmwkClass}{MATH 140B}
\newcommand{\hmwkClassInstructor}{Professor Seward}
\newcommand{\hmwkAuthorName}{\textbf{Ray Tsai}}
\newcommand{\hmwkPID}{A16848188}

%
% Title Page
%

\title{
    \vspace{2in}
    \textmd{\textbf{\hmwkClass:\ \hmwkTitle}}\\
    \normalsize\vspace{0.1in}\small{Due\ on\ \hmwkDueDate\ at 23:59pm}\\
    \vspace{0.1in}\large{\textit{\hmwkClassInstructor}} \\
    \vspace{3in}
}

\author{
  \hmwkAuthorName \\
  \vspace{0.1in}\small\hmwkPID
}
\date{}

\renewcommand{\part}[1]{\textbf{\large Part \Alph{partCounter}}\stepcounter{partCounter}\\}

%
% Various Helper Commands
%

% define norm \norm{...}:
\DeclarePairedDelimiterX\norm[1]\lVert\rVert{{#1}}

% Useful for algorithms
\newcommand{\alg}[1]{\textsc{\bfseries \footnotesize #1}}

% For derivatives
\newcommand{\deriv}[1]{\frac{\mathrm{d}}{\mathrm{d}x} (#1)}

% For partial derivatives
\newcommand{\pderiv}[2]{\frac{\partial}{\partial #1} (#2)}

% Integral dx
\newcommand{\dx}{\mathrm{d}x}

% Probability commands: Expectation, Variance, Covariance, Bias
\newcommand{\Var}{\mathrm{Var}}
\newcommand{\Cov}{\mathrm{Cov}}
\newcommand{\Bias}{\mathrm{Bias}}
\newcommand*{\Z}{\mathbb{Z}}
\newcommand*{\Q}{\mathbb{Q}}
\newcommand*{\R}{\mathbb{R}}
\newcommand*{\C}{\mathbb{C}}
\newcommand*{\N}{\mathbb{N}}
\newcommand*{\prob}{\mathds{P}}
\newcommand*{\E}{\mathds{E}}

\begin{document}

\maketitle

\pagebreak

\begin{homeworkProblem}
  Suppose $0 < \delta < \pi$, $f(x) = 1$ if $|x| \leq \delta$, $f(x) = 0$ if $\delta < |x| \leq
  \pi$, and $f(x + 2\pi) = f(x)$ for all $x$.

  \begin{enumerate}[(a)]
    \item Compute the Fourier coefficients of $f$.
    \begin{proof}
      Let $c_n$ denote the $n$th fourier coefficient of $f$. We first note that
      \[
        c_0 = \frac{1}{2\pi}\int_{-\pi}^{\pi} f(x) \, dx = \frac{\delta}{\pi}.
      \]
      For $n \neq 0$,
      \[
        c_n = \frac{1}{2\pi}\int_{-\pi}^{\pi} f(x)e^{-inx} \, dx = \frac{1}{2\pi}\int_{-\delta}^{\delta} e^{-inx} \, dx = \frac{1}{2in\pi}(e^{in\delta} - e^{-in\delta}) = \frac{\sin (n\delta)}{n\pi}.
      \]
    \end{proof}
    \item Conclude that
    \[
    \sum_{n=1}^\infty \frac{\sin(n\delta)}{n} = \frac{\pi - \delta}{2} \quad (0 < \delta < \pi).
    \]
    \begin{proof}
      Since $f(t) = 1$ for all $t \in (-\delta, \delta)$, it follows from Theorem 8.14 that 
      \[
        \sum_{-\infty}^\infty c_n = f(0) = 1.
      \]
      Since $\frac{\sin(-n\delta)}{-n\pi} = \frac{\sin(n\delta)}{n\pi}$,
      \[
        \pi = \delta + \sum_{n \neq 0} \frac{\sin(n\delta)}{n} = \delta + 2\sum_{n=1}^\infty \frac{\sin(n\delta)}{n},
      \]
      and the result now follows from rearranging the equation.
    \end{proof}
    \item Deduce from Parseval's theorem that
    \[
    \sum_{n=1}^\infty \frac{\sin^2(n\delta)}{n^2\delta} = \frac{\pi - \delta}{2}.
    \]
    \begin{proof}
      Note that $\frac{\sin^2(n\delta)}{(n\pi)^2}$ is an even function with respect to $n$. By
      Parseval's theorem
      \[
        \frac{\delta^2}{\pi^2} + 2\sum_{n=1}^\infty \frac{\sin^2(n\delta)}{(n\pi)^2} = \sum_{-\infty}^\infty |c_n|^2 = \frac{1}{2\pi} \int_{-\pi}^{\pi} |f(x)|^2 \, dx = \frac{\delta}{\pi}.
      \]
      The result now follows from rearranging the equation.
    \end{proof}

    \break

    \item Let $\delta \rightarrow 0$ and prove that
    \[
    \int_0^\infty \left(\frac{\sin x}{x}\right)^2 \, dx = \frac{\pi}{2}.
    \]
    \begin{proof}
      We first show that the improper integral exists. Pick $\epsilon > 0$. By L'Hopital's rule,
      \[
        \lim_{x \to 0} \frac{\sin x}{x} = \lim_{x \to 0} \cos x = 1,
      \]
      and thus there exists $\nu > 0$ such that $\left|(\frac{\sin x}{x})^2 - 1\right| < \epsilon$
      whenever $|x| < \nu$. Hence,
      \[
        \nu(1 - \epsilon) \leq \int_0^{\nu} \left(\frac{\sin x}{x}\right)^2 \, dx \leq \nu(1 + \epsilon),
      \]
      and so the the improper integral $\int_0^{1} \left(\frac{\sin x}{x}\right)^2 \, dx$ exists. On
      the other hand, 
      \[
        \left|\int_{1}^n \left(\frac{\sin x}{x}\right)^2 \, dx\right| \leq \int_{1}^n \frac{1}{x^2} \, dx = 1 - \frac{1}{n},
      \] 
      and thus $\int_{1}^n \left(\frac{\sin x}{x}\right)^2 \, dx \to 1$ as $n \to \infty$.

      Pick $\epsilon > 0$. Since the improper integral exists, there exists $A > 0, B > \max(A,
      3/\epsilon)$, such that
      \[
        \left|\int_{0}^{\infty} \left(\frac{\sin x}{x}\right)^2 \, dx - \int_{a}^{b} \left(\frac{\sin x}{x}\right)^2 \, dx\right| < \epsilon/3,
      \]
      for all $a \in (0, A]$ and $b \geq B$. 

      We now prove 2 lemmas:

      \textbf{Lemma 1 } Let $f: [a, b] \to \R$ be continuous. For all $\epsilon > 0$, there exists
      $\delta_1 > 0$ such that, for any partition $P = \{x_1, \dots, x_r\}$ on $[a, b]$ with $\max_i
      \Delta x_i < \delta_1$, we have 
      \begin{itemize}
        \item[(i)] $U(P, f) - L(P, f) < \epsilon$ \\
        \item[(ii)] $\left|\int_a^b f - \sum_{i = 1}^n f(s_i) \Delta x_i\right| < \epsilon$, for any
        $s_i \in [x_{i - 1}, x_i]$.
      \end{itemize}

      \begin{proof}
        (a) immediately follows from the proof of Theorem 6.8. (b) is by Theorem 6.7(c).
      \end{proof}

      \textbf{Lemma 2 } With the setups of Lemma, let $[c, d] \subseteq [a, b]$. For any partition
      $Q = \{y_1, \ldots, y_m\}$ on $[c, d]$ with $\max_i \Delta y_i < \delta_1$,
      \begin{enumerate}[(i)]
        \item $U(Q, f|_{[c, d]}) - L(Q, f|_{[c, d]}) < \epsilon$ \\
        \item $\left|\int_c^d f - \sum_{i = 1}^n f(s_i) \Delta y_i\right| < \epsilon$, for any
        $s_i \in [y_{i - 1}, y_i]$.
      \end{enumerate}
      \begin{proof}
        Consider the partition $Q^+ = Q \cup Q'$ on $[a, b]$, where the length of every interval in
        $Q'$ is also lesser than $\delta_1$. By Lemma 1, $U(Q, f|_{[c, d]}) - L(Q, f|_{[c, d]}) \leq
        U(Q^+, f) - L(Q^+, f) < \epsilon$. (b) again follows from Theorem 6.7(c).
      \end{proof}

      Now consider $g:[0, B + 1] \to \R$ with $g(x) = \begin{cases}
        \left(\frac{\sin x}{x}\right)^2 & x > 0 \\
        1 & x = 0
      \end{cases}$. Note that $g$ is continuous on $[0, B + 1]$. For $T \in (0, B +
      1]$, we have
      \begin{align*}
        \left|\int_0^T g(x) \, dx - \int^T_0 \left(\frac{\sin x}{x}\right)^2 \, dx\right|
        &\leq \int_0^T \left|g(x) - \left(\frac{\sin x}{x}\right)^2\right| \, dx \\
        &= \int_0^{\eta} \left|g(x) - \left(\frac{\sin x}{x}\right)^2\right| \, dx
      \end{align*}
      for arbitrary $\eta > 0$. Since $\left(\frac{\sin x}{x}\right)^2$ is bounded, we have
      $\int_0^T g(x) \, dx = \int^T_0 \left(\frac{\sin x}{x}\right)^2 \, dx$ for all $T \in (0, B +
      1]$. Applying Lemma 1 with $\epsilon/3$ gives a $\delta_1$ for $g$. Let $\delta_0 = \min(A,
      \delta_1, 1)$. 
      
      Fix arbitrary $\delta \in (0, \delta_0)$. Let $N = \lfloor \frac{B}{\delta} \rfloor$. Note
      that $\frac{B}{\delta} \leq N < \frac{B}{\delta} + 1$, and so $B \leq N \delta < B + \delta
      \leq B + 1$. 

      Since $\delta < \delta_0 \leq A$ and $N \delta \geq B$, we have
      \[
        \left|\int_{0}^{\infty} \left(\frac{\sin x}{x}\right)^2 \, dx - \int_{\delta}^{N\delta} \left(\frac{\sin x}{x}\right)^2 \, dx\right| < \epsilon/3.
      \]
      Since $[\delta, N\delta] \subseteq [0, B + 1]$, Lemma 2 tells us
      \[
        \left|\int_{\delta}^{N\delta} \left(\frac{\sin x}{x}\right)^2 \, dx - \sum_{n = 1}^N \frac{\sin(n\delta)}{(n\delta)^2}\delta\right| < \epsilon/3.
      \]
      Since
      \begin{align*}
        \left|\sum_{n = 1}^N \frac{\sin(n\delta)}{(n\delta)^2}\delta - \sum_{n = 1}^{\infty} \frac{\sin(n\delta)}{n^2\delta}\right|
        &\leq \sum_{n = N + 1}^{\infty} \frac{|\sin(n\delta)|}{n^2\delta} \\
        &\leq \frac{1}{\delta}\sum_{n = N + 1}^{\infty} \frac{1}{n^2} \\
        &\leq \frac{1}{\delta} \int_{N}^{\infty} \frac{1}{x^2} \, dx = \frac{1}{N\delta} \leq \frac{1}{B} < \epsilon/3.
      \end{align*}
      Therefore,
      \begin{align*}
        &\left|\int_0^\infty \left(\frac{\sin x}{x}\right)^2 \, dx - \frac{\pi - \delta}{2}\right| \\
        &\leq \left|\int_{0}^{\infty} \left(\frac{\sin x}{x}\right)^2 \, dx - \int_{\delta}^{N\delta} \left(\frac{\sin x}{x}\right)^2 \, dx\right| \\
        &\quad + \left|\int_{\delta}^{N\delta} \left(\frac{\sin x}{x}\right)^2 \, dx - \sum_{n = 1}^N \frac{\sin(n\delta)}{(n\delta)^2}\delta\right| + \left|\sum_{n = 1}^N \frac{\sin(n\delta)}{(n\delta)^2}\delta - \sum_{n = 1}^{\infty} \frac{\sin(n\delta)}{n^2\delta}\right| < \epsilon.
      \end{align*}
      But then $\delta$ is arbitrary, so $\int_0^\infty \left(\frac{\sin x}{x}\right)^2 \, dx =
      \frac{\pi}{2}$.
    \end{proof}
    \item Put $\delta = \frac{\pi}{2}$ in (c). What do you get?
    \begin{proof}
      \[
      \sum_{n=1}^\infty \frac{\sin^2(n\pi/2)}{n^2\pi/2} = \frac{2}{\pi}\sum_{k=1}^\infty \frac{1}{(2k + 1)^2} = \frac{\pi}{4},
      \]
      and thus
      \[
        \sum_{n \text{ odd}} \frac{1}{n^2} = \frac{\pi^2}{8}.
      \]
    \end{proof}
  \end{enumerate}
\end{homeworkProblem}

\newpage

\begin{homeworkProblem}
  Put $f(x) = x$ if $0 \leq x < 2\pi$, and apply Parseval's theorem to conclude that
  \[
    \sum_{n=1}^\infty \frac{1}{n^2} = \frac{\pi^2}{6}.
  \]
  \begin{proof}
    For $x \in \R$, define $f(x + 2\pi) = f(x)$. 
    \[
      c_0 = \frac{1}{2\pi} \int_{-\pi}^{\pi} x \, dx = \pi
    \]
    \[
      c_n = \frac{1}{2\pi} \int_0^{2\pi} xe^{-inx} \, dx = -\frac{1}{in}e^{-2\pi i n} - \frac{1}{2\pi(in)^2}(e^{-in2\pi} - 1) = \frac{i}{n}.
    \]
    By Parseval's theorem,
    \begin{align*}
      \frac{1}{2\pi} \int_{-\pi}^{\pi} |f(x)|^2 \, dx = \sum_{-\infty}^{\infty} |c_n|^2.
    \end{align*}
    On the left-hand-side, we have
    \[
      \frac{1}{2\pi} \int_{-\pi}^{\pi} |f(x)|^2 \, dx = \frac{1}{2\pi} \int_{0}^{2\pi} |f(x)|^2 \, dx = \frac{4\pi^2}{3}.
    \]
    On the right-hand-side, since $|c_n|^2 = \frac{1}{n^2} = |c_{-n}|^2$,
    \[
      \sum_{-\infty}^{\infty} |c_n|^2 = \pi^2 + 2\sum_{n = 1}^{\infty} \frac{1}{n^2}.
    \]
    Hence, we get
    \[
      \sum_{n = 1}^{\infty} \frac{1}{n^2} = \frac{1}{6} \left(\frac{4\pi^2}{3} - \pi^2\right) = \frac{\pi^2}{6}.
    \]
  \end{proof}
\end{homeworkProblem}

\newpage

\begin{homeworkProblem}
  If $f(x) = (\pi - |x|)^2$ on $[-\pi, \pi]$, prove that
  \[
  f(x) = \frac{\pi^2}{3} + \sum_{n=1}^\infty \frac{4}{n^2} \cos nx
  \]
  and deduce that
  \[
  \sum_{n=1}^\infty \frac{1}{n^2} = \frac{\pi^2}{6}, \quad \sum_{n=1}^\infty \frac{1}{n^4} = \frac{\pi^4}{90}.
  \]
  \begin{proof}
    Let $x, t \in [-\pi, \pi]$. By MVT, 
    \[
      |f(x + t) - f(x)| = |f'(s)||t| = 2|t||\pi - |s|| \leq 2\pi|t|,
    \]
    for some $s \in (-\pi, \pi)$. Hence, the Fourier series converges to $f$ for all $x$, by Theorem
    8.14. The Fourier coefficients of $f(x)$ for $n = 0$ is
    \[
      c_0 = \frac{1}{2\pi}\int_{-\pi}^{\pi} (\pi - |x|)^2 \, dx = \frac{1}{\pi}\int_{-\pi}^{0} x^2 \, dx = \frac{\pi^2}{3},
    \]
    and for $n \neq 0$ is
    \[
      c_n = \frac{1}{2\pi}\int_{-\pi}^{\pi} (\pi - |x|)^2e^{-inx} \, dx = \frac{1}{2\pi}\left(\pi^2\int_{-\pi}^{\pi} e^{-inx} \, dx - 2\pi\int_{-\pi}^{\pi} |x|e^{-inx} \, dx + \int_{-\pi}^{\pi} x^2e^{-inx} \, dx\right).
    \]
    We calculate each integral separately:
    \begin{gather*}
      \int_{-\pi}^{\pi} e^{-inx} \, dx = \frac{1}{in}(e^{in\pi} - e^{-in\pi}) = \frac{2}{n}\sin(n\pi) = 0. \\
      \int_{-\pi}^{\pi} |x|e^{-inx} \, dx = \int_{-\pi}^{\pi} |x|\cos(nx) \, dx - i\int_{-\pi}^{\pi} |x|\sin(nx) \, dx = 2\int_{0}^{\pi} x\cos(nx) \, dx = \frac{2(\cos(nx) - 1)}{n^2}. \\
      \int_{-\pi}^{\pi} x^2e^{-inx} \, dx = \int_{-\pi}^{\pi} x^2\cos(nx) \, dx - i\int_{-\pi}^{\pi} x^2\sin(nx) \, dx = 2\int_{0}^{\pi} x^2\cos(nx) \, dx = \frac{4\pi\cos(nx)}{n^2}.
    \end{gather*}
    Hence, $c_n = \frac{2 - 2\cos(nx)}{n^2} + \frac{2\cos(nx)}{n^2} = \frac{2}{n^2}$. It now follows
    that 
    \[
      f(x) = \frac{\pi^2}{3} + \sum_{n \neq 0} \frac{2}{n^2}e^{-inx} = \frac{\pi^2}{3} + \sum_{n = 1}^{\infty} \frac{2}{n^2}(e^{-inx} + e^{-inx}) = \frac{\pi^2}{3} + \sum_{n=1}^\infty \frac{4}{n^2} \cos nx.
    \]
    When $x = 0$, we have
    \[
      \pi^2 = \frac{\pi^2}{3} + 4\sum_{n=1}^\infty \frac{1}{n^2} \Rightarrow \sum_{n=1}^\infty \frac{1}{n^2} = \frac{\pi^2}{6}.
    \]
    By Parseval's Theorem, 
    \[
      \frac{1}{2\pi}\int_{-\pi}^{\pi} |f(x)|^2 \, dx = \sum_{-\infty}^{\infty} |c_n|^2,
    \]
    and thus
    \[
      \frac{1}{\pi} \int_{0}^{\pi} (x - \pi)^4 \, dx = \frac{\pi^4}{5} = \frac{\pi^4}{9} + 8\sum_{n = 1}^{\infty} \frac{1}{n^4} \Rightarrow \sum_{n=1}^\infty \frac{1}{n^4} = \frac{\pi^4}{90}.
    \]
  \end{proof}
\end{homeworkProblem}

\newpage

\begin{homeworkProblem}
  Let $\gamma$ be a continuously differentiable closed curve in the complex plane, with parameter
  interval $[a, b]$, and assume that $\gamma(t) \neq 0$ for every $t \in [a, b]$. Define the
  \textit{index} of $\gamma$ to be
  \[
  \text{Ind}(\gamma) = \frac{1}{2\pi i} \int_a^b \frac{\gamma'(t)}{\gamma(t)} \, dt.
  \]

  Prove that $\text{Ind}(\gamma)$ is always an integer. Compute $\text{Ind}(\gamma)$ when $\gamma(t)
  = e^{int}$, $a = 0$, $b = 2\pi$. Explain why $\text{Ind}(\gamma)$ is often called the winding
  number of $\gamma$ around $0$.

  \begin{proof}
    Define $\phi(x) = \int_a^x \frac{\gamma'(t)}{\gamma(t)} \, dt$. Since $\phi' =
    \frac{\gamma'}{\gamma}$ and $\phi(a) = 0$, 
    \[
      (\gamma(x) \exp(-\phi(x)))' = \gamma'(x) \exp(-\phi(x)) - \gamma(x)\phi'(x) \exp(-\phi(x)) = 0,
    \]
    and thus $\gamma(x) \exp(-\phi(x)) = \gamma(a)$. Since $\gamma(a) = \gamma(b)$, we have $\exp
    \phi(b) = \frac{\gamma(b)}{\gamma(a)} = 1$, and so $\phi(b) = 2n\pi i$. But then $\phi(b) =
    \int_a^b \frac{\gamma'(t)}{\gamma(t)} \, dt = 2\pi i\text{Ind}(\gamma)$, an thus
    $\text{Ind}(\gamma) = n$ for some integer $n$.

    Now consider $\text{Ind}(\gamma)$ when $\gamma(t) = e^{int}$, $a = 0$, $b = 2\pi$. 
    \[
      \text{Ind}(\gamma) = \frac{1}{2\pi i} \int_0^{2\pi} \frac{ine^{int}}{e^{int}} \, dt = \frac{1}{2\pi i} \int_0^{2\pi} in \, dt = n.
    \]
    Since $\text{Ind}(\gamma)$ represents the number of rotations $\gamma(t)$ goes around 0, it
    makes sense to call $\text{Ind}(\gamma)$ the winding number of $\gamma$ around 0.
  \end{proof}
\end{homeworkProblem}

\newpage

\begin{homeworkProblem}
  Let $\gamma$ be as in Exercise 8.23, and assume in addition that the range of $\gamma$ does not
  intersect the negative real axis. Prove that $\text{Ind}(\gamma) = 0$. 

  \begin{proof}
    For any $c \geq 0$, define $\gamma_c(t) = \gamma(t) + c$. Consider the function
    \[
    f(c) = \text{Ind}(\gamma_c) = \frac{1}{2\pi i} \int_a^b \frac{\gamma'(t)}{\gamma(t) + c} \, dt
    \]
    on $[0, \infty)$. We show that $f(c)$ is continuous and integer-valued. Since $\gamma'$ is
    continuous on $[-\pi, \pi]$, $|\gamma'| < M$ for some $M$. Since $\gamma$ does not intersect the
    negative real axis, $\gamma_c \neq 0$. But then $|\gamma_c|$ is continuous on compact set, so
    $\min_t |\gamma_c(t)|$ exists for all $c \geq 0$. Consider $\min_{c, t} |\gamma_c(t)| =
    |\gamma(t) + c|$. If $\text{Re}(\gamma(t)) > 0$, then $|\gamma_c(t)| = |(\text{Re}(\gamma(t)) +
    c) + i\text{Im}(\gamma(t))| \geq |\gamma(t)| > 0$. If $\text{Re}(\gamma(t)) \leq 0$, then
    $\text{Im}(\gamma(t)) \neq 0$ by assumption, and thus $|\gamma_c(t)| > 0$. Hence, we know
    $\min_{c, t} |\gamma_c(t)| \geq m > 0$ for some $m$. Pick $\epsilon > 0$. Let $\delta =
    \frac{2\pi\epsilon m^2}{(b - a)M}$. Then,
    \begin{align*}
      |f(x) - f(y)| 
      &\leq \frac{1}{2\pi}\int_a^b \left|\frac{\gamma'(t)}{\gamma(t) + x} - \frac{\gamma'(t)}{\gamma(t) + y}\right| \, dt \\
      &\leq \frac{1}{2\pi}\int_a^b \frac{M|y - x|}{|\gamma(t) + x||\gamma(t) + y|} \, dt \\
      &\leq \frac{(b - a)M|y - x|}{2\pi m^2} < \epsilon,
    \end{align*}
    whenever $x, y \geq 0$ and $|x - y| < \delta$, and so $f$ is continuous. 

    Put $\phi_c(x) = \int_a^x \frac{\gamma'(t)}{\gamma_c(t)} \, dt$. With the exact same argument
    as in Exercise 8.23, we get that $f(c) = \text{Ind}(\gamma_c)$ is integer valued.

    Given sequence of function $\left\{\frac{\gamma'}{\gamma_n}\right\}_{n \in \N}$, for all $n \geq
    \frac{M}{\epsilon} - m$, we have
    \[
      \left|\frac{\gamma'(t)}{\gamma_n(t)}\right| = \frac{M}{m + n} < \epsilon,
    \]
    for all $t \in [a, b]$. Hence, $\frac{\gamma'}{\gamma_n} \to 0$ uniformly. By Theorem 7.16, 
    \[
      \lim_{c \to \infty} \text{Ind}(\gamma_c) = \frac{1}{2\pi i} \int_a^b \lim_{c \to \infty} \frac{\gamma'(t)}{\gamma(t) + c} \, dt = 0.
    \]
    Since $\text{Ind}(\gamma_c)$ is integer-valued and continuous, it now follows that
    $\text{Ind}(\gamma_c) = 0$, for all $c$, which includes $c = 0$.
  \end{proof}
\end{homeworkProblem}

\newpage

\begin{homeworkProblem}
  Suppose $\gamma_1$ and $\gamma_2$ are curves as in Exercise 8.23, and $|\gamma_1(t) - \gamma_2(t)|
  < |\gamma_1(t)|$ for $a \leq t \leq b$. Prove that $\text{Ind}(\gamma_1) = \text{Ind}(\gamma_2)$.

  \begin{proof}
    Put $\gamma = \frac{\gamma_2}{\gamma_1}$. Note that $\gamma$ is well-defined, as $\gamma_1 \neq
    0$. Then $|1 - \gamma| = \frac{|\gamma_1(t) - \gamma_2(t)|}{|\gamma_1(t)|} < 1$, and so $\gamma$
    does not intersect with the negative real axis. By Exercise 8.24, $\text{Ind}(\gamma) = 0$.
    Also,
    \[
    \frac{\gamma'}{\gamma} = \frac{\frac{\gamma'_2\gamma_1 - \gamma'_1\gamma_2}{\gamma_1^2}}{\frac{\gamma_2}{\gamma_1}} = \frac{\gamma'_2\gamma_1 - \gamma'_1\gamma_2}{\gamma_1\gamma_2} = \frac{\gamma'_2}{\gamma_2} - \frac{\gamma'_1}{\gamma_1}.
    \]
    Hence,
    \[
      \text{Ind}(\gamma) = \frac{1}{2\pi i}\int_a^b \frac{\gamma'(t)}{\gamma(t)} \, dt = \frac{1}{2\pi i}\int_a^b \frac{\gamma'_2(t)}{\gamma_2(t)} \, dt - \frac{1}{2\pi i}\int_a^b \frac{\gamma'_1(t)}{\gamma_1(t)} \, dt = 0,
    \]
    and thus
    \[
      \text{Ind}(\gamma_1) = \frac{1}{2\pi i}\int_a^b \frac{\gamma'_2(t)}{\gamma_2(t)} \, dt = \frac{1}{2\pi i}\int_a^b \frac{\gamma'_1(t)}{\gamma_1(t)} \, dt = \text{Ind}(\gamma_2).
    \]
  \end{proof}
\end{homeworkProblem}
\end{document}