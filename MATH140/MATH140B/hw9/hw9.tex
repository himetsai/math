\documentclass{article}

\usepackage{fancyhdr}
\usepackage{extramarks}
\usepackage{amsmath}
\usepackage{amsthm}
\usepackage{amsfonts}
\usepackage{tikz}
\usepackage[plain]{algorithm}
\usepackage{algpseudocode}
\usepackage{enumerate}
\usepackage{amssymb}
\usepackage{mathrsfs}
\usepackage{mathtools}
\usepackage{amsmath}

\usetikzlibrary{automata,positioning}

%
% Basic Document Settings
%

\topmargin=-0.45in
\evensidemargin=0in
\oddsidemargin=0in
\textwidth=6.5in
\textheight=9.0in
\headsep=0.25in

\linespread{1.1}

\pagestyle{fancy}
\lhead{\hmwkAuthorName}
\chead{\hmwkClass:\ \hmwkTitle}
\rhead{\firstxmark}
\lfoot{\lastxmark}
\cfoot{\thepage}

\renewcommand\headrulewidth{0.4pt}
\renewcommand\footrulewidth{0.4pt}

\setlength\parindent{0pt}
\setlength{\parskip}{5pt}

%
% Create Problem Sections
%

\newcommand{\enterProblemHeader}[1]{
    \nobreak\extramarks{}{Problem \arabic{#1} continued on next page\ldots}\nobreak{}
    \nobreak\extramarks{Problem \arabic{#1} (continued)}{Problem \arabic{#1} continued on next page\ldots}\nobreak{}
}

\newcommand{\exitProblemHeader}[1]{
    \nobreak\extramarks{Problem \arabic{#1} (continued)}{Problem \arabic{#1} continued on next page\ldots}\nobreak{}
    \stepcounter{#1}
    \nobreak\extramarks{Problem \arabic{#1}}{}\nobreak{}
}

\setcounter{secnumdepth}{0}
\newcounter{partCounter}
\newcounter{homeworkProblemCounter}
\setcounter{homeworkProblemCounter}{1}
\nobreak\extramarks{Problem \arabic{homeworkProblemCounter}}{}\nobreak{}

%
% Homework Problem Environment
%
% This environment takes an optional argument. When given, it will adjust the
% problem counter. This is useful for when the problems given for your
% assignment aren't sequential. See the last 3 problems of this template for an
% example.
%
\newenvironment{homeworkProblem}[1][-1]{
    \ifnum#1>0
        \setcounter{homeworkProblemCounter}{#1}
    \fi
    \section{Problem \arabic{homeworkProblemCounter}}
    \setcounter{partCounter}{1}
    \enterProblemHeader{homeworkProblemCounter}
}{
    \exitProblemHeader{homeworkProblemCounter}
}

%
% Homework Details
%   - Title
%   - Due date
%   - Class
%   - Section/Time
%   - Instructor
%   - Author
%

\newcommand{\hmwkTitle}{Homework\ \#9}
\newcommand{\hmwkDueDate}{Jun 7, 2024}
\newcommand{\hmwkClass}{MATH 140B}
\newcommand{\hmwkClassInstructor}{Professor Seward}
\newcommand{\hmwkAuthorName}{\textbf{Ray Tsai}}
\newcommand{\hmwkPID}{A16848188}

%
% Title Page
%

\title{
    \vspace{2in}
    \textmd{\textbf{\hmwkClass:\ \hmwkTitle}}\\
    \normalsize\vspace{0.1in}\small{Due\ on\ \hmwkDueDate\ at 23:59pm}\\
    \vspace{0.1in}\large{\textit{\hmwkClassInstructor}} \\
    \vspace{3in}
}

\author{
  \hmwkAuthorName \\
  \vspace{0.1in}\small\hmwkPID
}
\date{}

\renewcommand{\part}[1]{\textbf{\large Part \Alph{partCounter}}\stepcounter{partCounter}\\}

%
% Various Helper Commands
%

% define norm \norm{...}:
\DeclarePairedDelimiterX\norm[1]\lVert\rVert{{#1}}

% Useful for algorithms
\newcommand{\alg}[1]{\textsc{\bfseries \footnotesize #1}}

% For derivatives
\newcommand{\deriv}[1]{\frac{\mathrm{d}}{\mathrm{d}x} (#1)}

% For partial derivatives
\newcommand{\pderiv}[2]{\frac{\partial}{\partial #1} (#2)}

% Integral dx
\newcommand{\dx}{\mathrm{d}x}

% Probability commands: Expectation, Variance, Covariance, Bias
\newcommand{\Var}{\mathrm{Var}}
\newcommand{\Cov}{\mathrm{Cov}}
\newcommand{\Bias}{\mathrm{Bias}}
\newcommand*{\Z}{\mathbb{Z}}
\newcommand*{\Q}{\mathbb{Q}}
\newcommand*{\R}{\mathbb{R}}
\newcommand*{\C}{\mathbb{C}}
\newcommand*{\N}{\mathbb{N}}
\newcommand*{\prob}{\mathds{P}}
\newcommand*{\E}{\mathds{E}}

\begin{document}

\maketitle

\pagebreak

\begin{homeworkProblem}
  Suppose $0 < \delta < \pi$, $f(x) = 1$ if $|x| \leq \delta$, $f(x) = 0$ if $\delta < |x| \leq
  \pi$, and $f(x + 2\pi) = f(x)$ for all $x$.

  \begin{enumerate}[(a)]
    \item Compute the Fourier coefficients of $f$.
    \begin{proof}
      Let $c_n$ denote the $n$th fourier coefficient of $f$. We first note that
      \[
        c_0 = \frac{1}{2\pi}\int_{-\pi}^{\pi} f(x) \, dx = \frac{\delta}{\pi}.
      \]
      For $n \neq 0$,
      \[
        c_n = \frac{1}{2\pi}\int_{-\pi}^{\pi} f(x)e^{-inx} \, dx = \frac{1}{2\pi}\int_{-\delta}^{\delta} e^{-inx} \, dx = \frac{1}{2in\pi}(e^{in\delta} - e^{-in\delta}) = \frac{\sin (n\delta)}{n\pi}.
      \]
    \end{proof}
    \item Conclude that
    \[
    \sum_{n=1}^\infty \frac{\sin(n\delta)}{n} = \frac{\pi - \delta}{2} \quad (0 < \delta < \pi).
    \]
    \begin{proof}
      Since $f(t) = 1$ for all $t \in (-\delta, \delta)$, it follows from Theorem 8.14 that 
      \[
        \sum_{-\infty}^\infty c_n = f(0) = 1.
      \]
      Since $\frac{\sin(-n\delta)}{-n\pi} = \frac{\sin(n\delta)}{n\pi}$,
      \[
        \pi = \delta + \sum_{n \neq 0} \frac{\sin(n\delta)}{n} = \delta + 2\sum_{n=1}^\infty \frac{\sin(n\delta)}{n},
      \]
      and the result now follows from rearranging the equation.
    \end{proof}
    \item Deduce from Parseval's theorem that
    \[
    \sum_{n=1}^\infty \frac{\sin^2(n\delta)}{n^2\delta} = \frac{\pi - \delta}{2}.
    \]
    \begin{proof}
      Note that $\frac{\sin^2(n\delta)}{(n\pi)^2}$ is an even function with respect to $n$. By
      Parseval's theorem
      \[
        \frac{\delta^2}{\pi^2} + 2\sum_{n=1}^\infty \frac{\sin^2(n\delta)}{(n\pi)^2} = \sum_{-\infty}^\infty |c_n|^2 = \frac{1}{2\pi} \int_{-\pi}^{\pi} |f(x)|^2 \, dx = \frac{\delta}{\pi}.
      \]
      The result now follows from rearranging the equation.
    \end{proof}

    \break

    \item Let $\delta \rightarrow 0$ and prove that
    \[
    \int_0^\infty \left(\frac{\sin x}{x}\right)^2 \, dx = \frac{\pi}{2}.
    \]
    \begin{proof}
      We first show that the improper integral exists. Pick $\epsilon > 0$. By L'Hopital's rule,
      \[
        \lim_{x \to 0} \frac{\sin x}{x} = \lim_{x \to 0} \cos x = 1,
      \]
      and thus there exists $\nu > 0$ such that $\left|(\frac{\sin x}{x})^2 - 1\right| < \epsilon$
      whenever $|x| < \nu$. Hence,
      \[
        \nu(1 - \epsilon) \leq \int_0^{\nu} \left(\frac{\sin x}{x}\right)^2 \, dx \leq \nu(1 + \epsilon),
      \]
      and so the the improper integral $\int_0^{A} \left(\frac{\sin x}{x}\right)^2 \, dx$ exists. On
      the other hand, 
      \[
        \left|\int_{A}^n \left(\frac{\sin x}{x}\right)^2 \, dx\right| \leq \int_{A}^n \frac{1}{x^2} \, dx = \frac{1}{A} - \frac{1}{n},
      \] 
      and thus $\int_{A}^n \left(\frac{\sin x}{x}\right)^2 \, dx \to \frac{1}{A}$ as $n \to \infty$.

      Since the improper integral exists, there exists large enough $A$, such that
      \[
        \left|\int_{A}^{\infty} \left(\frac{\sin x}{x}\right)^2 \, dx\right| < \epsilon/3.
      \]
      There exists small enough $\delta' \in (0, \delta)$ such that $A/\delta' \in \Z^+$ and the
      partition $P = \{n\delta' \mid n \in \Z, 0 \leq n \leq A/\delta'\}$, yields 
      
      By (c), given $\delta > 0$, there exists large enough $N_1$ such that for all $N \geq N_1$,
      \[
        \left|\sum_{n=1}^{N} \frac{\sin^2(n\delta)}{n^2\delta} - \frac{\pi - \delta}{2}\right| < \epsilon/3.
      \]
      Since the improper integral exists, there exists large enough $N_2$, such that
      \[
        \left|\int_{N_2}^{\infty} \left(\frac{\sin x}{x}\right)^2 \, dx\right| < \epsilon/3.
      \]
      Since the improper integral exists, there exists large enough $N_3$ such that for all $N \geq
      N_3$, the paritition $P = \{n\delta \mid n \in \Z, 0 \leq n \leq N\}$ on $[0, N_2]$ yields
      \[
        \left|\int_{0}^{A} \left(\frac{\sin x}{x}\right)^2 \, dx - \sum_{n=1}^{N} \frac{\sin^2(n\delta)}{n^2\delta}\right| < \epsilon/3.
      \]
      Put $N = \max(N_1, N_2)$, and 
      
    \end{proof}
    \item Put $\delta = \frac{\pi}{2}$ in (c). What do you get?
    \begin{proof}
      \[
      \sum_{n=1}^\infty \frac{\sin^2(n\pi/2)}{n^2\pi/2} = \frac{2}{\pi}\sum_{k=1}^\infty \frac{1}{(2k + 1)^2} = \frac{\pi}{4},
      \]
      and thus
      \[
        \sum_{n \text{ odd}} \frac{1}{n^2} = \frac{\pi^2}{8}.
      \]
    \end{proof}
  \end{enumerate}
\end{homeworkProblem}

\newpage

\begin{homeworkProblem}
  Put $f(x) = x$ if $0 \leq x < 2\pi$, and apply Parseval's theorem to conclude that
  \[
    \sum_{n=1}^\infty \frac{1}{n^2} = \frac{\pi^2}{6}.
  \]
  \begin{proof}
    For $x \in \R$, define $f(x + 2\pi) = f(x)$. 
    \[
      c_0 = \frac{1}{2\pi} \int_{-\pi}^{\pi} x \, dx = \pi
    \]
    \[
      c_n = \frac{1}{2\pi} \int_0^{2\pi} xe^{-inx} \, dx = -\frac{1}{in}e^{-2\pi i n} - \frac{1}{2\pi(in)^2}(e^{-in2\pi} - 1) = \frac{i}{n}.
    \]
    By Parseval's theorem,
    \begin{align*}
      \frac{1}{2\pi} \int_{-\pi}^{\pi} |f(x)|^2 \, dx = \sum_{-\infty}^{\infty} |c_n|^2.
    \end{align*}
    On the left-hand-side, we have
    \[
      \frac{1}{2\pi} \int_{-\pi}^{\pi} |f(x)|^2 \, dx = \frac{1}{2\pi} \int_{0}^{2\pi} |f(x)|^2 \, dx = \frac{4\pi^2}{3}.
    \]
    On the right-hand-side, since $|c_n|^2 = \frac{1}{n^2} = |c_{-n}|^2$,
    \[
      \sum_{-\infty}^{\infty} |c_n|^2 = \pi^2 + 2\sum_{n = 1}^{\infty} \frac{1}{n^2}.
    \]
    Hence, we get
    \[
      \sum_{n = 1}^{\infty} \frac{1}{n^2} = \frac{1}{6} \left(\frac{4\pi^2}{3} - \pi^2\right) = \frac{\pi^2}{6}.
    \]
  \end{proof}
\end{homeworkProblem}
\end{document}