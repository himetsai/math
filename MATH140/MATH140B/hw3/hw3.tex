\documentclass{article}

\usepackage{fancyhdr}
\usepackage{extramarks}
\usepackage{amsmath}
\usepackage{amsthm}
\usepackage{amsfonts}
\usepackage{tikz}
\usepackage[plain]{algorithm}
\usepackage{algpseudocode}
\usepackage{enumerate}
\usepackage{amssymb}
\usepackage{mathrsfs}
\usepackage{mathtools}
\usepackage{amsmath}

\usetikzlibrary{automata,positioning}

%
% Basic Document Settings
%

\topmargin=-0.45in
\evensidemargin=0in
\oddsidemargin=0in
\textwidth=6.5in
\textheight=9.0in
\headsep=0.25in

\linespread{1.1}

\pagestyle{fancy}
\lhead{\hmwkAuthorName}
\chead{\hmwkClass:\ \hmwkTitle}
\rhead{\firstxmark}
\lfoot{\lastxmark}
\cfoot{\thepage}

\renewcommand\headrulewidth{0.4pt}
\renewcommand\footrulewidth{0.4pt}

\setlength\parindent{0pt}
\setlength{\parskip}{5pt}

%
% Create Problem Sections
%

\newcommand{\enterProblemHeader}[1]{ \nobreak\extramarks{}{Problem \arabic{#1} continued on next
    page\ldots}\nobreak{} \nobreak\extramarks{Problem \arabic{#1} (continued)}{Problem \arabic{#1}
    continued on next page\ldots}\nobreak{} }

\newcommand{\exitProblemHeader}[1]{ \nobreak\extramarks{Problem \arabic{#1} (continued)}{Problem
    \arabic{#1} continued on next page\ldots}\nobreak{}
    \stepcounter{#1}
    \nobreak\extramarks{Problem \arabic{#1}}{}\nobreak{}
}

\setcounter{secnumdepth}{0}
\newcounter{partCounter}
\newcounter{homeworkProblemCounter}
\setcounter{homeworkProblemCounter}{1}
\nobreak\extramarks{Problem \arabic{homeworkProblemCounter}}{}\nobreak{}

%
% Homework Problem Environment
%
% This environment takes an optional argument. When given, it will adjust the
% problem counter. This is useful for when the problems given for your
% assignment aren't sequential. See the last 3 problems of this template for an
% example.
%
\newenvironment{homeworkProblem}[1][-1]{
    \ifnum#1>0
        \setcounter{homeworkProblemCounter}{#1}
    \fi
    \section{Problem \arabic{homeworkProblemCounter}}
    \setcounter{partCounter}{1}
    \enterProblemHeader{homeworkProblemCounter}
}{
    \exitProblemHeader{homeworkProblemCounter}
}

%
% Homework Details
%   - Title
%   - Due date
%   - Class
%   - Section/Time
%   - Instructor
%   - Author
%

\newcommand{\hmwkTitle}{Homework\ \#3}
\newcommand{\hmwkDueDate}{Apr 29, 2024}
\newcommand{\hmwkClass}{MATH 140B}
\newcommand{\hmwkClassInstructor}{Professor Seward}
\newcommand{\hmwkAuthorName}{\textbf{Ray Tsai}}
\newcommand{\hmwkPID}{A16848188}

%
% Title Page
%

\title{
    \vspace{2in}
    \textmd{\textbf{\hmwkClass:\ \hmwkTitle}}\\
    \normalsize\vspace{0.1in}\small{Due\ on\ \hmwkDueDate\ at 23:59pm}\\
    \vspace{0.1in}\large{\textit{\hmwkClassInstructor}} \\
    \vspace{3in}
}

\author{
  \hmwkAuthorName \\
  \vspace{0.1in}\small\hmwkPID
}
\date{}

\renewcommand{\part}[1]{\textbf{\large Part \Alph{partCounter}}\stepcounter{partCounter}\\}

%
% Various Helper Commands
%

% define norm \norm{...}:
\DeclarePairedDelimiterX\norm[1]\lVert\rVert{{#1}}

% Useful for algorithms
\newcommand{\alg}[1]{\textsc{\bfseries \footnotesize #1}}

% For derivatives
\newcommand{\deriv}[1]{\frac{\mathrm{d}}{\mathrm{d}x} (#1)}

% For partial derivatives
\newcommand{\pderiv}[2]{\frac{\partial}{\partial #1} (#2)}

% Integral dx
\newcommand{\dx}{\mathrm{d}x}

% Probability commands: Expectation, Variance, Covariance, Bias
\newcommand{\Var}{\mathrm{Var}}
\newcommand{\Cov}{\mathrm{Cov}}
\newcommand{\Bias}{\mathrm{Bias}}
\newcommand*{\Z}{\mathbb{Z}}
\newcommand*{\Q}{\mathbb{Q}}
\newcommand*{\R}{\mathbb{R}}
\newcommand*{\C}{\mathbb{C}}
\newcommand*{\N}{\mathbb{N}}
\newcommand*{\prob}{\mathds{P}}
\newcommand*{\E}{\mathds{E}}

\begin{document}

\maketitle

\pagebreak

\begin{homeworkProblem}
  Suppose $f$ is a bounded real function on $[a, b]$ and $f^2 \in \mathscr{R}$ on $[a, b]$. Does it
  follow that $f \in \mathscr{R}$? Does the answer change if we assume that $f^3 \in \mathscr{R}$?

  \begin{proof}
    $f$ is not necessarily integrable. Consider
    \[
      f(x) = \begin{cases}
        1, & x \in \Q \\
        -1, & x \notin \Q
      \end{cases}.
    \]
    $f^2(x) = 1$ is obviously continuous, as it is constant. Since both rationals and irrationals
    are dense in $\R$,
    \[
      U(P, f) = \sum_{i = 0}^n \Delta x_i = b - a, \quad L(P, f) = \sum_{i = 0}^n -\Delta x_i = a - b,
    \]
    for any partition $P$. But then $U(P, f) - L(P, f) = 2(b - a)$, and thus $f \notin \mathscr{R}$.

    Suppose $f^3 \in \mathscr{R}$. Since $f$ is bounded, we may assume $|f| < M$. Define $\phi(x) =
     \sqrt[3]{x}$. Note that $\phi(f^3(x)) = f(x)$. Since $x^3$ is a continuous 1-1 mapping on
     $[-M^{1/3}, M^{1/3}]$, its inverse $\phi = \sqrt[3]{x}$ is continuous on $[-M, M]$, by Theorem
     4.17. But then by Theorem 6.11, $f(x) = \phi(f^3(x)) \in \mathscr{R}$ on $[a, b]$. 
  \end{proof}
\end{homeworkProblem}

\newpage

\begin{homeworkProblem}
  Let $P$ be the Cantor set constructed in Theorem 2.44. Let $f$ be a bounded real function on $[0,
  1]$ which is continuous at every point outside $P$. Prove that $f \in \mathscr{R}$ on $[0, 1]$.

  \begin{proof}
    We first show that $P$ can be covered by finitely many segments whose total length can be made
    as small as desired. Pick $\epsilon > 0$. Note that $P = \bigcap_{n = 1}^{\infty} E_n$, where
    $E_n$ is the union of $2^n$ intervals, each of length $3^{-n}$. Pick $n$ large enough such that
    $\frac{2^n}{3^n} < \epsilon$. We know $P$ can be covered by $E_n$. Put $\nu \in (0, \epsilon -
    \frac{2^n}{3^n})$. Let $C$ be the union of segments, where each segment corresponds to an
    interval in $E_n$ with both endpoints extended by $\frac{\nu}{2^{n + 1}}$. Then, $C$ is a open
    cover of $P$ and the total length of all segments in $C$ is $\frac{2^n}{3^n} + \nu < \epsilon$. 
    
    We may assume that $C = \bigcup_{i = 1}^{2^n} (u_i, v_i)$ and the intervals $[u_i, v_i]$ are
    pairwise disjoint. Let $M = \sup |f(x)|$. Put $K = [0, 1] \backslash C$. Since $K$ is compact,
    $f$ is uniformly continuous on $K$. Hence, there exists $\delta > 0$ such that $|f(s) - f(t)| <
    \epsilon$ whenever $s, t \in K$ and $|s - t| < \delta$. 
    
    Now consider a partition $\rho = \{x_0, x_1, \dots, x_k\}$ of $[0, 1]$ such that each $u_i, v_i$
    occurs in $\rho$ and no point of any segment $(u_i, v_i)$ occurs in $\rho$. Additionally, if
    $x_{i - 1}$ is not one of the $u_j$, then $\Delta x_i < \delta$. 

    Note that $M_i - m_i \leq 2M$ for every $i$, and that $M_i - m_i \leq \epsilon$ unless $x_{i -
    1}$ is one of the $u_j$. Hence, 
    \begin{align*}
      U(\rho, f) - L(\rho, f) 
      &= \sum_{i = 1}^k (M_i - m_i) \Delta x_i \\
      &= \sum_{x_{i - 1} = u_j}^k (M_i - m_i) \Delta x_i + \sum_{x_{i - 1} \neq u_j}^k (M_i - m_i) \Delta x_i \\
      &< 2M\epsilon + \epsilon = (2M + 1)\epsilon, \\
    \end{align*}
    and the result follows from Theorem 6.6.
  \end{proof}
\end{homeworkProblem}

\newpage 

\begin{homeworkProblem}
  Suppose $f$ is a real function on $(0,1]$ and $f \in \mathscr{R}$ on $[c,1]$ for every $c > 0$.
  Define
  \[
    \int_{0}^{1} f(x) \, dx = \lim_{c \to 0} \int_{c}^{1} f(x) \, dx
  \]
  if this limit exists (and is finite).
  
  \begin{enumerate}[(a)]
    \item If $f \in \mathscr{R}$ on $[0,1]$, show that this definition of the integral agrees with
    the old one.
    \begin{proof}
      \[
        \lim_{c \to 0} \int_{c}^{1} f(x) \, dx = \int_{0}^{1} f(x) \, dx - \lim_{c \to 0} \int_{0}^{c} f(x) \, dx,
      \]
      so it remains to show that $\lim_{c \to 0} \int_{0}^{c} f(x) \, dx = 0$. Since $f \in
      \mathscr{R}$, we may assume $|f(x)| \leq M$ for $x \in [0, 1]$. Pick $\epsilon > 0$. Then,
      given any partition $P= \{x_0, \dots, x_n\}$, we have $\delta = \frac{\epsilon}{nM}$ such
      that, 
      \[
        U(P, f) = \sum_{i = 1}^n M_i \Delta x_i \leq nMc < \epsilon, \quad L(P, f) = \sum_{i = 1}^n m_i \Delta x_i > -nMc > -\epsilon,
      \]
      for all $c \in (0, \delta)$. But then
      \[
        \left|\int_{0}^{c} f(x)\right| < \epsilon,
      \]
      and the result follows.
    \end{proof}
    \item Construct a function $f$ such that the above limit exists, although it fails to exist with
    $|f|$ in place of $f$.
    \begin{proof}
      Define $f(x) = (-1)^n(n + 1)$ if $x \in (\frac{1}{n + 1}, \frac{1}{n}]$, for $n \in \N$.
      Suppose $c = \frac{1}{n + 1}$. Then, 
      \[
        \int_c^1 f(x) \, dx = \sum_{k = 1}^n \frac{(-1)^k}{k}, \quad \int_c^1 |f(x)| \, dx = \sum_{k = 1}^n \frac{1}{k}.
      \]
      As $c \to 0$, $n \to \infty$, and thus $\int_c^1 f(x) \, dx$ converges but not $\int_c^1
      |f(x)| \, dx$.
    \end{proof}
  \end{enumerate}
  \end{homeworkProblem}

  \newpage

  \begin{homeworkProblem}
    Suppose $f \in \mathscr{R}$ on $[a, b]$ for every $b > a$ where $a$ is fixed. Define
    \[
      \int_{a}^{\infty} f(x) \, dx = \lim_{b \to \infty} \int_{a}^{b} f(x) \, dx
    \]
    if this limit exists (and is finite). In that case, we say that the integral on the left
    \emph{converges}. If it also converges after $f$ has been replaced by $|f|$, it is said to
    converge \emph{absolutely}.
    
    Assume that $f(x) \geq 0$ and that $f$ decreases monotonically on $[1, \infty)$. Prove that
    \[
      \int_{1}^{\infty} f(x) \, dx
    \]
    converges if and only if
    \[
      \sum_{n=1}^{\infty} f(n)
    \]
    converges. (This is the so-called ``integral test'' for convergence of series.)

    \begin{proof}
      Consider the partition $P = \{1, \dots, n\}$. Note that as $f \geq 0$, both the $\int_1^n f(x)
      \, dx$ and $\sum_{k = 1}^n f(k)$ are monotonically increasing with respect to $n$. Hence, it
      suffices to show that the integral and summation are bounded together. Since $f$ is
      monotonically decreasing,
      \[
        U(P, f) = \sum_{k = 1}^{n - 1} f(k), \quad L(P, f) = \sum_{k = 2}^n f(k).
      \]
      Note that since $f$ is at least 0 and monotonically decreasing, $\lim_{n \to \infty} f(n) \in
      \R$. We then get
      \[
        f(n) + \sum_{k = 2}^n f(k) \leq f(n) + \int_{1}^n f(x) \, dx \leq \sum_{k = 1}^{n} f(k) \leq f(1) + \int^n_2 f(x) \, dx.
      \]
      But then $\int_{1}^{\infty} f(x) \, dx$ and $\sum_{n=1}^{\infty} f(n)$ are bounded together,
      and the result follows.
    \end{proof}
  \end{homeworkProblem}
    
  \newpage

  \begin{homeworkProblem}
    Let $p$ and $q$ be positive real numbers such that
    \[
      \frac{1}{p} + \frac{1}{q} = 1.
    \]
    Prove the following statements.
    
    \begin{enumerate}[(a)]
    \item If $u \geq 0$ and $v \geq 0$, then
    \[
      uv \leq \frac{u^p}{p} + \frac{v^q}{q}.
    \]
    Equality holds if and only if $u^p = v^q$.

    \begin{proof}
      Fix $u$. Define $f(v) = \frac{u^p}{p} + \frac{v^q}{q} - uv$. Then, $f'(v) = v^{q - 1} - u,
      f''(v) = (q - 1)v^{q - 2} \geq 0$, for $v \geq 0$. Hence, $f(v)$ reaches minimum at $v =
      u^{\frac{1}{q - 1}}$. Note that $p = \frac{q}{q - 1}$. But then
      \begin{align*}
        f(v)
        &= \frac{u^p}{p} + \frac{v^q}{q} - uv \\
        &\geq \frac{u^p}{p} + \frac{u^{\frac{q}{q - 1}}}{q} - u^{\frac{q}{q - 1}} \\
        &= \left(\frac{1}{p} + \frac{1}{q} - 1\right)u^p = 0,
      \end{align*}
      and the result follows.
    \end{proof}
    
    \item If $f \in \mathscr{R}(\alpha)$, $g \in \mathscr{R}(\alpha)$, $f \geq 0$, $g \geq 0$, and
    \[
      \int_{a}^{b} f^p d\alpha = 1 = \int_{a}^{b} g^q d\alpha,
    \]
    then
    \[
      \int_{a}^{b} fg d\alpha \leq 1.
    \]

    \begin{proof}
      By (a),
      \[
        \int_{a}^{b} \frac{f^p}{p} d\alpha + \int_{a}^{b} \frac{g^q}{q} d\alpha = \int_{a}^{b} \frac{f^p}{p} + \frac{g^q}{q} d\alpha \geq \int_{a}^{b} fg d\alpha.
      \]
      But then
      \[
        1 = \frac{1}{p} + \frac{1}{q} = \int_{a}^{b} fg d\alpha.
      \]
    \end{proof}
    
    \item If $f$ and $g$ are complex functions in $\mathscr{R}(\alpha)$, then
    \[
      \left| \int_{a}^{b} fg d\alpha \right| \leq \left( \int_{a}^{b} |f|^p d\alpha \right)^{1/p} \left( \int_{a}^{b} |g|^q d\alpha \right)^{1/q}.
    \]
    This is H\"older's inequality. When $p = q = 2$, it is usually called the Schwarz inequality. 
    (Note that Theorem 1.35 is a very special case of this.)

    \begin{proof}
      Put $F = \int_{a}^{b} |f|^p d\alpha$, $G = \int_{a}^{b} |g|^p d\alpha$. Since $f, g \in
      \mathscr{R}(\alpha)$, $|f|, |g| < M \in \R$. Note that $F = 0$ implies $\int_{a}^{b} |f|
      d\alpha = 0$. Thus
      \[
        0 = M\int_{a}^{b} |f| \, d\alpha \geq \int_{a}^{b} |f||g| \, d\alpha \geq  \left| \int_{a}^{b} fg d\alpha \right|,
      \]
      and the inequality holds.

      Hence, we may assume $F, G > 0$. Substituting $f$ as $\frac{|f|}{F^{1/p}}$ and $g$ as
      $\frac{|g|}{G^{1/q}}$, we get
      \[
        \int^{b}_a \frac{|f||g|}{F^{1/p}G^{1/q}} \, d\alpha \leq= 1.
      \]
      But then
      \[
        \left|\int^{b}_a |fg| \, d\alpha\right|\leq \int^{b}_a |f||g| \, d\alpha \leq F^{1/p}G^{1/q}.
      \]
    \end{proof}
    
    \item Show that H\"older's inequality is also true for the ``improper'' integrals described in
    Exercises 6.7 and 6.8.

    \begin{proof}
      Since the equality holds for any finite interval, and thus the inequality also holds if the
      improper integrals converge. 

      Suppose the improper integral of $f$ or $g$ diverge, the right-hand side of the inequality
      tends to infinity, and thus the inequality still holds.
    \end{proof}
    \end{enumerate}
  \end{homeworkProblem}      
\end{document}