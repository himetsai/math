\documentclass{article}

\usepackage{fancyhdr}
\usepackage{extramarks}
\usepackage{amsmath}
\usepackage{amsthm}
\usepackage{amsfonts}
\usepackage{tikz}
\usepackage[plain]{algorithm}
\usepackage{algpseudocode}
\usepackage{enumerate}
\usepackage{amssymb}
\usepackage{mathrsfs}
\usepackage{mathtools}
\usepackage{amsmath}

\usetikzlibrary{automata,positioning}

%
% Basic Document Settings
%

\topmargin=-0.45in
\evensidemargin=0in
\oddsidemargin=0in
\textwidth=6.5in
\textheight=9.0in
\headsep=0.25in

\linespread{1.1}

\pagestyle{fancy}
\lhead{\hmwkAuthorName}
\chead{\hmwkClass:\ \hmwkTitle}
\rhead{\firstxmark}
\lfoot{\lastxmark}
\cfoot{\thepage}

\renewcommand\headrulewidth{0.4pt}
\renewcommand\footrulewidth{0.4pt}

\setlength\parindent{0pt}
\setlength{\parskip}{5pt}

%
% Create Problem Sections
%

\newcommand{\enterProblemHeader}[1]{
    \nobreak\extramarks{}{Problem \arabic{#1} continued on next page\ldots}\nobreak{}
    \nobreak\extramarks{Problem \arabic{#1} (continued)}{Problem \arabic{#1} continued on next page\ldots}\nobreak{}
}

\newcommand{\exitProblemHeader}[1]{
    \nobreak\extramarks{Problem \arabic{#1} (continued)}{Problem \arabic{#1} continued on next page\ldots}\nobreak{}
    \stepcounter{#1}
    \nobreak\extramarks{Problem \arabic{#1}}{}\nobreak{}
}

\setcounter{secnumdepth}{0}
\newcounter{partCounter}
\newcounter{homeworkProblemCounter}
\setcounter{homeworkProblemCounter}{1}
\nobreak\extramarks{Problem \arabic{homeworkProblemCounter}}{}\nobreak{}

%
% Homework Problem Environment
%
% This environment takes an optional argument. When given, it will adjust the
% problem counter. This is useful for when the problems given for your
% assignment aren't sequential. See the last 3 problems of this template for an
% example.
%
\newenvironment{homeworkProblem}[1][-1]{
    \ifnum#1>0
        \setcounter{homeworkProblemCounter}{#1}
    \fi
    \section{Problem \arabic{homeworkProblemCounter}}
    \setcounter{partCounter}{1}
    \enterProblemHeader{homeworkProblemCounter}
}{
    \exitProblemHeader{homeworkProblemCounter}
}

%
% Homework Details
%   - Title
%   - Due date
%   - Class
%   - Section/Time
%   - Instructor
%   - Author
%

\newcommand{\hmwkTitle}{Homework\ \#4}
\newcommand{\hmwkDueDate}{May 3, 2024}
\newcommand{\hmwkClass}{MATH 140B}
\newcommand{\hmwkClassInstructor}{Professor Seward}
\newcommand{\hmwkAuthorName}{\textbf{Ray Tsai}}
\newcommand{\hmwkPID}{A16848188}

%
% Title Page
%

\title{
    \vspace{2in}
    \textmd{\textbf{\hmwkClass:\ \hmwkTitle}}\\
    \normalsize\vspace{0.1in}\small{Due\ on\ \hmwkDueDate\ at 23:59pm}\\
    \vspace{0.1in}\large{\textit{\hmwkClassInstructor}} \\
    \vspace{3in}
}

\author{
  \hmwkAuthorName \\
  \vspace{0.1in}\small\hmwkPID
}
\date{}

\renewcommand{\part}[1]{\textbf{\large Part \Alph{partCounter}}\stepcounter{partCounter}\\}

%
% Various Helper Commands
%

% define norm \norm{...}:
\DeclarePairedDelimiterX\norm[1]\lVert\rVert{{#1}}

% Useful for algorithms
\newcommand{\alg}[1]{\textsc{\bfseries \footnotesize #1}}

% For derivatives
\newcommand{\deriv}[1]{\frac{\mathrm{d}}{\mathrm{d}x} (#1)}

% For partial derivatives
\newcommand{\pderiv}[2]{\frac{\partial}{\partial #1} (#2)}

% Integral dx
\newcommand{\dx}{\mathrm{d}x}

% Probability commands: Expectation, Variance, Covariance, Bias
\newcommand{\Var}{\mathrm{Var}}
\newcommand{\Cov}{\mathrm{Cov}}
\newcommand{\Bias}{\mathrm{Bias}}
\newcommand*{\Z}{\mathbb{Z}}
\newcommand*{\Q}{\mathbb{Q}}
\newcommand*{\R}{\mathbb{R}}
\newcommand*{\C}{\mathbb{C}}
\newcommand*{\N}{\mathbb{N}}
\newcommand*{\prob}{\mathds{P}}
\newcommand*{\E}{\mathds{E}}

\begin{document}

\maketitle

\pagebreak

\begin{homeworkProblem}
  Show that integration by parts can sometimes be applied to the ``improper'' integrals defined in
  Exercises 6.7 and 6.8. (State the appropriate hypotheses, formulate a theorem, and prove it.) For
  instance, show that
  \[
    \int_0^{\infty} \frac{\cos x}{1 + x} \, dx = \int_0^{\infty} \frac{\sin x}{(1 + x)^2} \, dx.
  \]
  Show that one of these integrals converges absolutely, but that the other does not.

  \textbf{Theorem } Let $F, G$ be differentiable functions on $[a, \infty)$, where $F' = f \in
  \mathscr{R}$ and $G' = g \in \mathscr{R}$. Suppose both $\lim_{x \to \infty} F(x)G(x)$ and
  $\int_{a}^{\infty} f(x)G(x) \, dx$ exist. Then
  \[
    \int_{a}^{\infty} F(x)g(x) \, dx = \lim_{x \to \infty} F(x)G(x) - F(a)G(a) - \int_{a}^{\infty} f(x)G(x) \, dx.
  \] 

  \begin{proof}
    Put $H(x) = F(x)G(x)$. By Theorem 6.13, we know $H' \in \mathscr{R}$. For finite $b > a$,
    applying Theorem 6.21 to $H$ and its derivative yields
    \[
      H(b) - H(a) = \int_a^b F(x)g(x) + f(x)G(x) \, dx,
    \]
    that is,
    \[
      \int_a^b F(x)g(x) \, dx = F(b)G(b) - F(a)G(a) - \int_a^b f(x)G(x) \, dx.
    \]
    But then by assumption, $\lim_{x \to \infty} F(x)G(x)$ and $\int_{a}^{\infty} f(x)G(x) \, dx$
    exist, and thus $\int_a^{\infty} F(x)g(x) \, dx$ also converges.
  \end{proof}

  Put $F(x) = \frac{1}{1 + x}$ and $G(x) = \sin x$. We know $f(x) = -\frac{1}{(1 + x)^2} \in
  \mathscr{R}$, $g(x) = \cos x$. Note that
  \[
    \lim_{x \to \infty} |F(x)G(x)| = \lim_{x \to \infty} \left|\frac{\sin x}{1 + x}\right| \leq \lim_{x \to \infty} \left|\frac{1}{1 + x}\right| = 0 = F(0)G(0).
  \]
  By exercise 6.8, we know that $\int_0^{\infty} \left|\frac{\sin x}{(1 + x)^2}\right| \, dx$
  converges as $\sum_{n = 0}^{\infty} \frac{|\sin x|}{(1 + n)^2}$ converges by comparison test with
  $\sum_{n = 0}^{\infty} \frac{1}{(1 + n)^2}$. Hence, again by exercise 6.8, $\int_0^{\infty}
  \frac{\sin x}{(1 + x)^2} \, dx$ also converges, as $\sum_{n = 0}^{\infty} \frac{\sin x}{(1 +
  n)^2}$ converges absolutely. Since the hypothesis holds, we may apply our theorem stated above and
  get
  \[
    \int_0^{\infty} \frac{\cos x}{1 + x} \, dx = \lim_{x \to \infty} \frac{\sin x}{1 + x} - \frac{\sin 0}{1} + \int_0^{\infty}
    \frac{\sin x}{(1 + x)^2} \, dx = \int_0^{\infty} \frac{\sin x}{(1 + x)^2} \, dx.
  \]

  To see that $\int_0^{\infty} \frac{\cos x}{1 + x} \, dx$ does not converge absolutely, we again
  apply exercise 6.8. Since
  \[
    \sum_{n \geq 0} \left|\frac{\cos x}{1 + x}\right| \geq \sum_{n \geq 0} \frac{1}{1 + x}
  \]
  diverges, $\int_0^{\infty} \left|\frac{\cos x}{1 + x}\right| \, dx$ also diverges.
\end{homeworkProblem}

\newpage

\begin{homeworkProblem}
  Let $\alpha$ be a fixed increasing function on $[a, b]$. For $u \in \mathscr{R}(\alpha)$, define
  \[
    \|u\|_2 = \left(\int_a^b |u|^2 \right)^{1/2}.
  \]
  Suppose $f, g, h \in \mathscr{R}(\alpha)$, and prove the triangle inequality
  \[
    \|f-h\|_2 \leq \|f-g\|_2 + \|g-h\|_2
  \]
  as a consequence of the Schwarz inequality, as in the proof of Theorem 1.37.

  \begin{proof}
    \begin{align*}
      \|f-h\|_2
      &= \|f - g + g - h\|_2 \\
      &= \left(\int_a^b |f - g + g - h|^2\right)^{1/2} \\
      &= \left(\int_a^b |f - g|^2 + 2\int_a^b |(f - g)(g - h)| + \int_a^b |g - h|^2\right)^{1/2} \\
      &\leq \left(\int_a^b |f - g|^2 + 2\int_a^b |f - g| \int_a^b|g - h| + \int_a^b |g - h|^2\right)^{1/2} \\
      &= \left(\int_a^b |f - g|^2\right)^{1/2} + \left(\int_a^b |g - h|^2\right)^{1/2} \\
      &= \|f-g\|_2 + \|g-h\|_2.
    \end{align*}
  \end{proof}
\end{homeworkProblem}

\newpage

\begin{homeworkProblem}
  With the notations of Exercise 6.11, suppose $f \in \mathscr{R}(\alpha)$ and $\epsilon > 0$. Prove
  that there exists a continuous function $g$ on $[a, b]$ such that $\|f - g\|_2 < \epsilon$.

  \begin{proof}
    Pick $\epsilon > 0$. Since $f \in \mathscr{R}(\alpha)$, there exists a partition $P = \{x_0,
    \ldots, x_n\}$ on $[a, b]$ such that
    \[
      U(P, f, \alpha) - L(P, f, \alpha) < \epsilon^2/2M.
    \]
    Suppose $|f| < M$. Define 
    \[
      g(t) = \frac{x_i - t}{\Delta x_i} f(x_{i-1}) + \frac{t - x_{i-1}}{\Delta x_i} f(x_i).
    \]
    if $x_{i-1} \leq t \leq x_i$. Note that $g$ is defined to be linear on every interval $[x_i,
    x_{i + 1}]$, and $g$ remains continuous between neighboring intervals. Hence, $g$ is continuous
    on $[a, b]$. In addition, on every interval $[x_1, x_{i + 1}]$, since $g(t)$ is between $f(x_1)$
    and $f(x_{i + 1})$, we have $m_i \leq g(t) \leq M_i$ for all $t \in [x_1, x_{i + 1}]$. But then
    \begin{align*}
      \|f - g\|_2^2
      &= \int_a^b |f - g|^2 \\
      &\leq U(P, |f - g|^2, \alpha) \\
      &= \sum_{i = 1}^n \sup_{x \in [x_i, x_{i + 1}]} (f(x) - g(x))^2 \Delta \alpha_i \\
      &\leq \sum_{i = 1}^n (M_i - m_i)^2 \Delta \alpha_i \\
      &\leq 2M[U(P, f, \alpha) - L(P, f, \alpha)] < \epsilon^2,
    \end{align*}
    and thus $\|f - g\|_2 < \epsilon$.
  \end{proof}
\end{homeworkProblem}

\newpage

\begin{homeworkProblem}
  Define
  \[
    f(x) = \int_x^{x+1} \sin(t^2) \, dt.
  \]
  \begin{enumerate}[(a)]
    \item Prove that $|f(x)| < \frac{1}{x}$ if $x > 0$.

    \begin{proof}
      By Theorem 6.17 and 6.19, we may substitute $t^2$ by $u$ and get 
      \[
        f(x) = \int_{x^2}^{(x+1)^2} \sin(u) \, du^{1/2} = \int_{x^2}^{(x+1)^2} \frac{\sin(u)}{2u^{1/2}} \, du.
      \]
      Put $F(x) = \frac{1}{2u^{1/2}}$ and $G(x) = -\cos(x)$. Applying Theorem 6.22 then yields
      \[
        f(x) = \frac{\cos(x^2)}{2x} - \frac{\cos[(x + 1)^2]}{2(x + 1)} - \frac{1}{2}\int_{x^2}^{(x+1)^2} \frac{\cos u}{2u^{3/2}} \, du.
      \]
      But then notice 
      \[
        \left|\int_{x^2}^{(x+1)^2} \frac{\cos u}{2u^{3/2}} \, du\right| \leq \int_{x^2}^{(x+1)^2}
        \left|\frac{\cos u}{2u^{3/2}}\right| \, du < \int_{x^2}^{(x+1)^2} \frac{1}{2u^{3/2}} =
        \frac{1}{x} - \frac{1}{x + 1}.
      \]
      Hence,
      \begin{align*}
        |f(x)| 
        &\leq  \left|\frac{\cos(x^2)}{2x} \right| + \left|\frac{\cos[(x + 1)^2]}{2(x + 1)}\right| + \left|\frac{1}{2}\int_{x^2}^{(x+1)^2} \frac{\cos u}{2u^{3/2}} \, du\right| \\
        &< \frac{1}{2x} + \frac{1}{2(x + 1)} + \frac{1}{2}\left(\frac{1}{x} - \frac{1}{x + 1}\right) = \frac{1}{x}.
      \end{align*}
    \end{proof}

    \item Prove that
    \[
      2xf(x) = \cos(x^2) - \cos[(x + 1)^2] + r(x)
    \]
    where $|r(x)| < \frac{c}{x}$ and $c$ is a constant.

    \begin{proof}
      By (a),
      \begin{align*}
        2xf(x) 
        &= \left(\frac{\cos(x^2)}{2x} - \frac{\cos[(x + 1)^2]}{2(x + 1)} - \frac{1}{2}\int_{x^2}^{(x+1)^2} \frac{\cos u}{2u^{3/2}} \, du\right) \\
        &= \cos(x^2) - \frac{x}{(x + 1)} \cdot \cos[(x + 1)^2] - x\int_{x^2}^{(x+1)^2} \frac{\cos u}{2u^{3/2}} \, du \\
        &< \cos(x^2) - \cos[(x + 1)^2] + \frac{\cos[(x + 1)^2]}{x + 1} + \frac{1}{x + 1},
      \end{align*}
      and thus
      \[
        |r(x)| < \left|\frac{\cos[(x + 1)^2] + 1}{x + 1}\right| < \frac{2}{x}.
      \]
    \end{proof}

    \break

    \item Does $\int_0^\infty \sin(t^2) \, dt$ converge?
    \begin{proof}
      \begin{align*}
        \int_0^\infty \sin(t^2) \, dt
        &= \sum_{x = 0}^{\infty} f(x) \\
        &= f(0) + \sum_{x = 1}^{\infty} \frac{\cos(x^2)}{2x} - \sum_{x = 1}^{\infty} \frac{\cos[(x + 1)^2]}{2x} + \sum_{x = 1}^{\infty} \frac{r(x)}{2x} \\
        &= f(0) + \sum_{x = 1}^{\infty} \frac{\cos(x^2)}{2x} - \sum_{x = 1}^{\infty} \frac{x + 1}{x} \cdot \frac{\cos[(x + 1)^2]}{2(x + 1)} + \sum_{x = 1}^{\infty} \frac{r(x)}{2x} \\
        &= f(0) + \sum_{x = 1}^{\infty} \frac{\cos(x^2)}{2x} - \sum_{x = 2}^{\infty} \frac{x}{x - 1} \cdot \frac{\cos(x^2)}{2x} + \sum_{x = 1}^{\infty} \frac{r(x)}{2x} \\
        &= f(0) + \frac{\cos 1}{2} + \sum_{x = 2}^{\infty} \frac{\cos(x^2)}{2x(1 - x)} + \sum_{x = 1}^{\infty} \frac{r(x)}{2x}.
      \end{align*}
      But then 
      \[
        \sum_{x = 1}^{\infty} \frac{|r(x)|}{2x} < \sum_{x = 1}^{\infty} \frac{1}{x^2},
      \]
      \begin{align*}
        \sum_{x = 2}^{\infty} \left| \frac{\cos(x^2)}{2x(1 - x)} \right| 
        &< \sum_{x = 2}^{\infty} \left| \frac{1}{2x(1 - x)} \right| \\
        &= \frac{1}{2}\sum_{x = 2}^{\infty} \frac{1}{x(x - 1)}\\
        &< \frac{1}{2}\sum_{x = 1}^{\infty} \frac{1}{x^2},
      \end{align*}
      and thus both series converge by comparison test. Since all terms of $f(0) + \frac{\cos 1}{2}
      + \sum_{x = 2}^{\infty} \frac{\cos(x^2)}{2x(1 - x)} + \sum_{x = 1}^{\infty} \frac{r(x)}{2x}$
      converge, $\int_0^\infty \sin(t^2) \, dt$ converges.
    \end{proof}
  \end{enumerate}
\end{homeworkProblem}

\newpage

\begin{homeworkProblem}
  Suppose $f$ is a real, continuously differentiable function on $[a, b]$, $f(a) = f(b) = 0$, and
  \[
    \int_a^b f^2(x) \, dx = 1.
  \]
  Prove that
  \[
    \int_a^b xf(x)f'(x) \, dx = -\frac{1}{2}
  \]
  and that
  \[
    \int_a^b [f'(x)]^2 \, dx \cdot \int_a^b x^2f^2(x) \, dx \geq \frac{1}{4}.
  \]

  \begin{proof}
    By Theorem 6.22,
    \begin{align*}
      \int_a^b xf(x)f'(x) \, dx
      &= bf(b) - af(a) - \int_a^b f(x)(f(x) + xf'(x)) \\
      &= bf(b) - af(a) - \int_a^b f^2(x) \, dx - \int_a^bxf(x)f'(x) \, dx.
    \end{align*}
    But then
    \[
      2\int_a^b xf(x)f'(x) \, dx = bf(b) - af(a) - 1 = -1,
    \]
    and the result follows.

    It now follows from Hölder's inequality that
    \begin{align*}
      \int_a^b [f'(x)]^2 \, dx \cdot \int_a^b x^2f^2(x) \, dx
      &\geq \left(\int_a^b xf(x)f'(x) \, dx\right)^2 = \frac{1}{4}.
    \end{align*}
  \end{proof}
\end{homeworkProblem}

\newpage

\begin{homeworkProblem}
  Suppose $\alpha$ increases monotonically on $[a, b]$, $g$ is continuous, and $g(x) = G'(x)$ for $a
  \leq x \leq b$. Prove that
  \[
  \int_a^b \alpha(x)g(x) \, dx = G(b)\alpha(b) - G(a)\alpha(a) - \int_a^b G \, d\alpha.
  \]

  \begin{proof}
    Let $P = \{x_0, x_1, \ldots, x_n\}$ be any partition on $[a, b]$. For each segment $(x_{i-1},
    x_i)$, the mean value theorem furnishes some $t \in (x_{i-1}, x_i)$ such that $g(t_i)\Delta x_i
    = G(x_i) - G(x_{i-1})$. Since $\alpha$ increases monotonically,
    \begin{align*}
      \sum_{i=1}^n \alpha(x_i)g(t_i) \Delta x_i
      &= \sum_{i=1}^n \alpha(x_i)(G(x_i) - G(x_{i-1})) \\
      &= G(b)\alpha(b) - G(a)\alpha(a) + \sum_{i=1}^n \alpha(x_{i - 1})G(x_{i - 1}) - \sum_{i=1}^n \alpha(x_i)G(x_{i - 1}) \\
      &= G(b)\alpha(b) - G(a)\alpha(a) - \sum_{i=1}^n G(x_{i-1}) \Delta \alpha_i,
    \end{align*}
    for any partition $P$, and thus both sides converge to the same integral.
  \end{proof}
\end{homeworkProblem}

\newpage

\begin{homeworkProblem}
  Let $\gamma_1$ be a curve in $\mathbb{R}^k$ defined on $[a, b]$, let $\phi$ be a continuous 1-1
  mapping of $[c, d]$ into $[a, b]$ such that $\phi(c) = a$; and define $\gamma_2(s) =
  \gamma_1(\phi(s))$. Prove that $\gamma_2$ is an arc, a closed curve, or a rectifiable curve if and
  only if the same is true of $\gamma_1$. Prove that $\gamma_2$ and $\gamma_1$ have the same length.

  \begin{proof}
    Since $\phi$ is a continuous 1-1 mapping on a compact space, its inverse mapping $\psi$ from
    $[a, b]$ into $[c, d]$ is also a continuous mapping, by Theorem 4.17. But then $\gamma_2(s) =
    \gamma_1(\phi(s))$ and $\gamma_2(\psi(t)) = \gamma_1(t)$, so $\gamma_1$ is 1-1 if and only if
    $\gamma_2$ is. Additionally, since $\phi$ is a continuous bijection with $\phi(a) = c$, we know
    $\phi(d) = b$, and thus $\gamma_2(c) = \gamma_1(\phi(c)) = \gamma_1(\phi(d)) = \gamma_2(d)$ if
    and only if $\gamma_1(a) = \gamma_1(b)$. Given a partition $P = \{x_0, \dots, x_n\}$ on $[c,
    d]$, $\gamma_1$ yields a partition $P' = \{y_0, \dots, y_n\}$ on $[a, b]$, with $[x_i, x_{i +
    1}]$ corresponding to $[y_i, y_{i + 1}]$ for all $i$. But then,
    \[
      \Lambda(P, \gamma_1) = \sum_{i = 1}^n |\gamma_1(x_i) - \gamma_1(x_{i - 1})| = \sum_{i = 1}^n |\gamma_2(\psi(x_i)) - \gamma_2(\psi(x_{i - 1}))| = \sum_{i = 1}^n |\gamma_2(y_i) - \gamma_2(y_{i - 1})| = \Lambda(P', \gamma_2).
    \]
    Therefore, $\gamma_1$ is a rectifiable curve if and only if $\gamma_2$ is, and 
    \[
      \Lambda(\gamma_1) = \sup \Lambda(P, \gamma_1) = \sup \Lambda(P', \gamma_2) = \Lambda(\gamma_2). 
    \]
  \end{proof}
\end{homeworkProblem}
\end{document}