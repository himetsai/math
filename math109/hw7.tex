\documentclass{article}
\usepackage{amsfonts, amsmath, amssymb, amsthm} % Math notations imported
\usepackage{enumitem}

\newtheorem{thm}{Theorem}
\newtheorem{prop}[thm]{Proposition}
\newtheorem{cor}[thm]{Corollary}

% title information
\title{Math 109 HW 7}
\author{Ray Tsai}
\date{11/9/2022}

% main content
\begin{document} 

% placing title information; comment out if using fancyhdr
\maketitle 

\begin{enumerate}
% Q1
\item 
\begin{prop}
    $P(n)$ is true for all $n \geq 1$.
\end{prop}
\begin{proof}
    We will show by strong induction on $n$ that for all $n \geq 1$, we have $P(n)$ is true.

    Base Case: If $n = 1$, then we have $P(1)$, which is true. 

    Induction Step: Let $k \geq 1$, $k \in \mathbb{Z}$. Suppose that for all $m \in \mathbb{Z}$, $1 \leq m \leq k$, we have $P(m)$ is true. We will show that $P(k + 1)$ is also true.

    We know that if $P(k)$ is true, then $P(k + 1) is true$. By the induction hypothesis, we know that $P(k)$ is true, and thus $P(k + 1)$ is true. Hence, if $P(1), P(2), \ldots , P(k)$ are all true, then $P(k + 1)$ is true.

    Therefore, for all $n \geq 1$, we have $P(n)$ is true.
\end{proof}

% Q2
\item 
\begin{prop}
    $Q(n)$ is true for all $n \geq 1$.
\end{prop}
\begin{proof}
    Let $Q(n)$ be the statement $``P(1), P(2), \ldots , P(n)$ are all true."
    We will show induction on $n$ that for all $n \geq 1$, we have $Q(n)$ is true, which shows that $P(n)$ is true.

    Suppose that $Q(1)$ is true and for all $k \geq 1$, if $Q(k)$ is true, then $Q(k + 1)$ is also true.

    Base Case: If $n = 1$, we have $Q(1)$ is true.

    Induction Step: Assume that $Q(k)$ is true for some $k \in \mathbb{N}$. By the Induction Hypothesis, $Q(k + 1)$ is true, as $Q(k)$ is true. Since $Q(k + 1)$ is true, $P(1), P(2), \ldots , P(n + 1)$ are all true, which shows that $P(n + 1)$ is true. Thus, if $Q(k)$ is true for some $k$, $P(n + 1)$ is true.

    Therefore, for all $n \geq 1$, $P(n)$ is true.
\end{proof}

% Q3
\item 
\begin{prop}
     $gcd(115, 29) = 1 = -1 \cdot 115 + 4 \cdot 29$.
\end{prop}
\begin{proof}
Let $gcd(115, 29) = 29m + 115n$, for some $m, n \in \mathbb{Z}$.
\begin{align}
    115 &= 29 \cdot 3  + 28 \\
    29  &= 28 \cdot 1 + 1 \\
    28  &= 1 \cdot 28 + 0 \\
\end{align}
Therefore, $gcd(115, 29) = 1$. 

We then have
\begin{align}
    1 &= 29 - 1 \cdot 28 \\
      &= 29 - 1 \cdot (115 - 3 \cdot 29) \\
      &= 29 - 115 + 3 \cdot 29 \\
      &= -1 \cdot 115 + 4 \cdot 29
\end{align}
Therefore, $m = 4, n = -1$.
\end{proof}


% Q4
\item 
\begin{prop}
     $gcd(1001, 182) = 91 = -5 \cdot 182 + 1 \cdot 1001$.
\end{prop}
\begin{proof}
Let $gcd(1001, 182) = 182m + 1001n$, for some $m, n \in \mathbb{Z}$.
\begin{align}
    1001 &= 182 \cdot 5  + 91 \\
    182  &= 91 \cdot 2 + 0
\end{align}
Therefore, $gcd(1001, 182) = 91$. 

We then have
\begin{align}
    91 &= 1001 - 182 \cdot 5 \\
       &= -5 \cdot 182 + 1 \cdot 1001
\end{align}
Therefore, $m = -5, n = 1$.
\end{proof}

% Q5
\item \begin{enumerate}
    
    \item \begin{prop}
        $\sim$ is reflexive and symmetric but not transitive.
    \end{prop}
    \begin{proof}
        Define $R \subseteq \mathbb{R}^2$ by
        \begin{gather}
            (x, y) \in R \text{ if } |x - y| \leq 2.
        \end{gather}
        Reflexive: Let $x \in \mathbb{R}$. We have $|x - x| = 0 \leq 2$. Thus, for all $x \in \mathbb{R}$, $(x,x) \in R$, and so $\sim$ is reflexive.

        Symmetric: Let $(x, y) \in R$. We have $|y - x| = |x - y| \leq 2$. Thus, for all $(x, y) \in R$, $(y,x) \in R$, and so $\sim$ is symmetric.

        Transitive: Consider the case $(4, 2), (2, 0) \in R$. $(4, 0) \notin R$ because $|4 - 0| = 4 > 2$, and so $\sim$ is not transitive.
    \end{proof}

    \item \begin{prop}
        $\sim$ is reflexive and transitive but not symmetric.
    \end{prop}
    \begin{proof}
        Define $R \subseteq \mathbb{Z}^2$ by
        \begin{gather}
            (m, n) \in R \text{ if } m|n.
        \end{gather}
        Reflexive: Let $m \in \mathbb{Z}$. We have $m|m$. Thus, for all $m \in \mathbb{R}$, $(m,m) \in R$, and so $\sim$ is reflexive.

        Symmetric: Consider the case $(1, 3) \in R$. $(3, 1) \notin R$ because $3 \nmid 1$, and so $\sim$ is not symmetric.

        Transitive: Let $(m, n), (n, k) \in R$. We have $mq = n$ and $np = k$, $q, p \in \mathbb{Z}$. We then have $k = (mq)p = (qp)m$. Since $qp$ is an integer, we have $m|k$. Thus, for all $(m, n), (n, k) \in R$, $(m,k) \in R$, and so $\sim$ is transitive.
    \end{proof}

    \item \begin{prop}
        $\sim$ is reflexive, symmetric and transitive.
    \end{prop}
    \begin{proof}
        Define $R \subseteq \mathbb{R}^2 \times \mathbb{R}^2$ by
        \begin{gather}
            ((x_1, y_1), (x_2, y_2)) \in R \text{ if } x_1 + 2y_1 = x_2 + 2y_2.
        \end{gather}
        Reflexive: Let $(x_1, y_1) \in \mathbb{R}^2$. We have $x_1 + 2y_1 = x_1 + 2y_1$. Thus, for all $(x_1, y_1) \in \mathbb{R}^2$, $((x_1, y_1), (x_1, y_1)) \in R$, and so $\sim$ is reflexive.

        Symmetric: Let $((x_1, y_1), (x_2, y_2)) \in R$. We have $x_2 + 2y_2 = x_1+ 2y_1$, . Thus, for all $((x_1, y_1), (x_2, y_2)) \in R$, $((x_2, y_2), (x_1, y_1)) \in R$, and so $\sim$ is symmetric.

        Transitive: Let $((x_1, y_1), (x_2, y_2)), ((x_2, y_2), (x_3, y_3)) \in R$. We have $x_1 + 2y_1 = x_2 + 2y_2 = x_3 + 2y_3$. \\
        Thus, for all $((x_1, y_1), (x_2, y_2)), ((x_2, y_2), (x_3, y_3)) \in R$, $((x_1, y_1), (x_3, y_3)) \in R$, and so $\sim$ is transitive.
    \end{proof}

    \item \begin{prop}
        $\sim$ symmetric and transitive but not reflexive.
    \end{prop}
    \begin{proof}
        Define $R \subseteq \mathbb{R}^2$ by
        \begin{gather}
            (x, y) \in R \text{ if } \frac{x}{y} = 1.
        \end{gather}
        Reflexive: Consider $0 \in \mathbb{R}$. We have $\frac{0}{0}$, which is undefined, and so $\sim$ is not reflexive.

        Symmetric: Let $(x, y) \in R$. Since $\frac{x}{y} = 1$, we know that $x = y$. We then have $\frac{y}{x} = 1$. Thus, for all $(x, y) \in R$, $(y, x) \in R$, and so $\sim$ is symmetric.

        Transitive: Let $(x, y), (y, z) \in R$. Since $\frac{x}{y} = \frac{y}{z} = 1$, we know that $x = y$ and $y = z$, so $x = z$. We then have $\frac{x}{z} = 1$. Thus, for all $(x, y), (y, z) \in R$, $(x, z) \in R$, and so $\sim$ is transitive.
    \end{proof}
\end{enumerate}


% Q6
\item 
\begin{enumerate}
    \item 
    $x \sim y \text{ if } |x - y| = 2$, $x, y \in \mathbb{R}$.
    \item
    $x \sim y \text{ if } x|y$, $x, y \in \mathbb{Z}$.
    \item
    $x \sim y \text{ if } \frac{x}{y}$, $x, y \in \mathbb{R}$.
\end{enumerate}

% Q7
\item 
The proof is assuming that for all $a \in S$, there exists $b \in S$ such that $a \sim b$. However, $a \in S$ does not imply there exists $b \in S$ such that $a \sim b$ because there could still be elements in $S$ that does not have a relation with any other elements.
    
\end{enumerate}
\end{document}