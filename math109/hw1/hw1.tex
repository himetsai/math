\documentclass{article}
\usepackage{amsfonts, amsmath, amssymb, amsthm} % Math notations imported
\usepackage{enumitem}

% If you want to use another style of headers, uncomment (hotkey: ctrl + /) these 7 lines, and comment out "\maketitle" below
% \usepackage{fancyhdr} % Import package
% \pagestyle{fancy} % using the "fancy" pagestyle
% \fancyhf{} % clear out original headers and footers
% \lhead{Math 109 HW 1} % left header, rest is self-explanatory
% \rhead{(your name)}
% \lfoot{(date when hw is due)}
% \rfoot{Page \thepage}

% Some basic theorem environments set up
\newtheorem{thm}{Theorem}
\newtheorem{prop}[thm]{Proposition}
\newtheorem{cor}[thm]{Corollary}

% title information
\title{Math 109 HW 1}
\author{Ray Tsai}
\date{10/3/2022}

% main content
\begin{document}

% placing title information; comment out if using fancyhdr
\maketitle

\begin{enumerate}
% Q1
\item \begin{prop}
$x = -1, y = -2$ is a counter example to the statement``For all real numbers $x, y$ with $x \geq y$, we have $x^2 \leq y^2$.”
\end{prop}
\begin{proof}
$x > y$, but $x^2 = 1 > y^2 = 4$, which contradicts the given statement.
\end{proof}

% Q2
\item 
\begin{enumerate}
    \item $\neg(P \vee (\neg Q)) \equiv (\neg P) \wedge Q$
    \item $\neg(P \rightarrow (\neg Q)) \equiv \neg((\neg P) \vee (\neg Q)) \equiv P \wedge Q$
    \item $\neg$($\forall x$ ($\exists y$ (S(x,y) is true.))) $\equiv$ $\exists x$($\forall y$ (S(x,y) is false.))
    \item $\neg$($\exists x$ ($\forall y$ (S(x,y) is true.))) $\equiv$ $\forall x$($\exists y$ (S(x,y) is false.))
\end{enumerate}

% Q3
\item 
The negation of ``for every real number $x$, there exists a real number $y$ such that $x + y = 0$” is``there exists a real number $x$, such that $x + y \neq 0$ for every real number $y$."


% Q4
\item 
The negation of ``if $z$ is a real number such that there is a real number $x$ with $xz = 0$, then $z = 0$" is ``$z$ is a real number such that there is a real number $x$ with $xz = 0$ and $z \neq 0$."

% Q5
\item 
\begin{prop}
``if a person gets at least an 98\% in the class, then they get an A+"
\end{prop}
\begin{enumerate}
    \item 
    \begin{enumerate}
    \item{converse:}
    ``If a person gets an A+, then they get at least an 98\% in the class."
    \item{contrapositive:}
    ``If a person does not get an A+, then they get lesser than 98\% in the class."
    \item{negation:}
    ``A person gets at least an 98\% in the class and they does not get an A+".
    \end{enumerate}
    \item 
    If a person gets an 97\% in the class and gets an A+, then this scenario makes (iii) and (ii) true but (i) false. \\ \\
    If a person does not gets an A+ and gets an 98\% in the class, then this scenario makes (iii) true but (ii) false. \\ \\
    Therefore, the three statements written above are different.
    \item 
    Below is a truth table of the original statement and statement (ii).
\begin{center}
    \begin{tabular}{|c|c|c|c|}
    \hline
    gets at least an 98\% & gets an A+ & original & ii. \\ \hline
    T & T & T & T\\ \hline
    T & F & F & F\\ \hline
    F & T & T & T\\ \hline
    F & F & T & T\\ \hline
\end{tabular}
\end{center}
Therefore, according to the truth table, the original statement is equivalent to statement (ii).
\end{enumerate}

% Q6
\item
\begin{enumerate}
    \item 
    There is not enough information to tell because $P \rightarrow Q$ does not imply whether $P$ is true or false.
    \item 
    There is not enough information to tell because $P \rightarrow Q$ does not imply whether $Q$ is true or false.
    \item 
    There is not enough information to tell because $P \rightarrow Q$ does not imply $P \wedge Q$.
    \item 
    This statement is false because $P \wedge (\neg Q) \equiv \neg ((\neg P) \vee Q) \equiv \neg(P \rightarrow Q)$.
    \item 
    There is not enough information to tell because $P \rightarrow Q$ does not imply $Q \rightarrow P$.
    \item 
    This statement is true because a statement is equivalent to it's contrapositive. 
    \item 
    This statement is false because a statement is opposite to it's negation. 
\end{enumerate}
    
\end{enumerate}
\end{document}