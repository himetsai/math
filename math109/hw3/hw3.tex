\documentclass{article}
\usepackage{amsfonts, amsmath, amssymb, amsthm} % Math notations imported
\usepackage{enumitem}

% If you want to use another style of headers, uncomment (hotkey: ctrl + /) these 7 lines, and comment out "\maketitle" below
% \usepackage{fancyhdr} % Import package
% \pagestyle{fancy} % using the "fancy" pagestyle
% \fancyhf{} % clear out original headers and footers
% \lhead{Math 109 HW 1} % left header, rest is self-explanatory
% \rhead{(your name)}
% \lfoot{(date when hw is due)}
% \rfoot{Page \thepage}

% Some basic theorem environments set up
\newtheorem{thm}{Theorem}
\newtheorem{prop}[thm]{Proposition}
\newtheorem{cor}[thm]{Corollary}

% title information
\title{Math 109 HW 3}
\author{Ray Tsai}
\date{10/17/2022}

% main content
\begin{document} 

% placing title information; comment out if using fancyhdr
\maketitle 

\begin{enumerate}
% Q1
\item 
\begin{prop}
    The additive inverse of integer $n$ is unique.
\end{prop}
\begin{proof}
Let $b, c$ be some integers such that $n + b = 0, n + c = 0$. We will show that the additive inverse of integer $n$ is unique. 
\begin{gather}
    n + b = 0\quad n + c = 0 \\
    n + b + (-n) = -n\quad n + c + (-n) = -n \\
    b = -n = c
\end{gather}
Thus, the additive inverse of integer $n$ is unique.
\end{proof}

% Q2
\item \begin{enumerate}
    \item \begin{prop}
     For all real number $x$, there is a 2 × 2 matrix over $\mathbb{R}$ such that its determinant is $x$.
    \end{prop}
    \begin{proof}
    Let det$
        \begin{pmatrix}
            a & b \\
            c & d \\
        \end{pmatrix} = ad - bc$, where $a, b, c ,d$ are real numbers. We will show that, for all real number $x$, there is a 2 × 2 matrix over $\mathbb{R}$ such that its determinant is $x$. \\\\
        Let $ad = k, bc = l$ where $k, l$ are real numbers.
        \begin{align}
            ad - bc = k - l
        \end{align}
        Let $k - l$ be a real number $x$.
        \begin{align}
            det\begin{pmatrix}
            a & b \\
            c & d \\
        \end{pmatrix} &= ad - bc \\ &= k - l \\ &= x
        \end{align}
        Thus, the determinant of a 2 x 2 matrix over $\mathbb{R}$ can be any real number $x$.
    \end{proof}
    
    \item \begin{prop}
    There exist a different 2 x 2 matrix over $\mathbb{R}$ such that its determinant is the same as the matrix in part(a).
    \end{prop}
    \begin{proof}
    Let $A = \begin{pmatrix}
            a & b \\
            c & d \\
        \end{pmatrix}$ and 
        $B = \begin{pmatrix}
            d & b \\
            c & a \\
        \end{pmatrix}$, where $a, b ,c, d$ are real numbers. We will show that $B$ and $A$ have the same determinant.
        \begin{gather}
            det(A) = ad - bc = da - bc = det(B)
        \end{gather}
        Thus, $A$ is not a unique matrix that has its determinant.
    \end{proof}
\end{enumerate}

% Q3
\item \begin{prop}
    For all $x \in \mathbb{R}$, we have $x^2 \geq 0$.
\end{prop}
\begin{proof}
Let $x = |x|$ when $x \geq 0$ and $x = -|x|$ when $x < 0$, according to the definition provided above. We will show that $x^2 \geq 0$. 


We can separate $x^2$ into two cases where $x \geq 0$ or $x < 0$.
\begin{gather}
      x^2 =  
\begin{cases}
    |x|^2,& \text{if } x\geq 0\\
    (-|x|)^2,& \text{if } x < 0
\end{cases}
\end{gather}
Since $|x| \geq 0$, 
\begin{gather}
    |x|\cdot |x| \geq 0 \cdot |x| \\
    |x|^2 \geq 0
\end{gather}
Hence, $x^2 = |x|^2 \geq 0$ when $x \geq 0$. \\
When $x < 0$,
\begin{align}
    x^2 &= (-|x|)^2 \\ 
    &= (-1)^2|x|^2 \\
    &= |x|^2 \geq 0
\end{align}
Thus, for all $x \in \mathbb{R}$, we have $x^2 \geq 0$.
\end{proof}

% Q4
\item \begin{prop}
    For all $x \in \mathbb{R}$, if $x^2 = x$, then $x < 2$.
\end{prop}

\begin{proof}
Let $x \in \mathbb{R}$. We will show that if $x^2 = x$, then $x < 2$.
\begin{gather}
    x^2 = x \\
    x^2 + (-x) = x + (-x) \\
    x(x - 1) = 0
\end{gather}
From the contrapositive of HW 3 Fact 4, we know that if $x(x - 1) = 0$ then $x - 1 = 0$ or $x$ = 0. If $x - 1 = 0$ then $x - 1 + 1 = x = 1$. Thus, $x$ can be $0$ or $1$, both of which are smaller than $2$. Thus, if $x^2 = x$, then $x < 2$.
\end{proof}

% Q5
\item \begin{prop}
     If $n$ is an integer, then $n^2 + 3n + 1$ is odd.
\end{prop}

\begin{proof}
Let $n$ be an integer. We will show that $n^2 + 3n + 1$ is odd.
\begin{gather}
    n^2 + 3n + 1 = n(n + 3) + 1
\end{gather}
From HW3 Fact 3, we know that all integers are even or odd. Thus, we can split $n(n + 3) + 1$ into $2$ cases, $n$ is even and $n$ is odd. \\
If $n$ is even, let $n$ be $2k$ for some integer $k$ by HW3 Fact 1.
\begin{align}
    n(n + 3) + 1 &= 2k(2k + 3) + 1 \\
    &= 2(2k^2 + 3k) + 1
\end{align}
Let $2k^2 + 3k$ be some integer $l$.
\begin{align}
    2(2k^2 + 3k) + 1 &= 2l + 1
\end{align}
Therefore, if $n$ is even, $n^2 + 3n + 1$ is odd by HW3 Fact 2.
If $n$ is odd, let $n$ be $2k + 1$ for some integer $k$ by HW3 Fact 2.
\begin{align}
    n(n + 3) + 1 &= (2k + 1)(2k + 4) + 1 \\
    &= 2(2k + 1)(k + 1) + 1
\end{align}
Let $(2k + 1)(k + 1)$ be some integer $l$.
\begin{align}
    2(2k + 1)(k + 1) + 1 &= 2l + 1
\end{align}
Therefore, if $n$ is odd, $n^2 + 3n + 1$ is odd by HW3 fact 1. \\
Thus, for all integer $n$, $n^2 + 3n + 1$ is odd.
\end{proof}

% Q6
\item \begin{prop}
    For all integer $a, b$.  If $a + b$ is even, then $a - b$ is even.
\end{prop}
\begin{proof}
Let $a, b$ be some integers. We will show that if $a + b$ is even, then $a - b$ is even. \\
By HW3 Fact 1, let $a + b$ be an even integer $2k$ for some integer $k$.
\begin{align}
    a - b &= a + b - 2b \\ 
    &= 2k - 2b \\
    &= 2(k - b)
\end{align}
Let $k - b$ be some integer $l$.
\begin{align}
    2(k - b) &= 2l
\end{align}
Thus, if $a + b$ is even, then $a - b$ is even by HW3 fact 1.
\end{proof}

% Q7
\item \begin{prop}
    Let $a, b$ be integers. If $ab$ is even, then $a$ or $b$ is even.
\end{prop}

\begin{proof}
We will prove by contradiction. Suppose for sake of contradiction that there exist some even integer $ab$ where both $a$ and $b$ are odd. By HW3 Fact 2, let $a$ and $b$ be some odd integers $2k + 1$ and $2l + 1$ for some integers $k, l$.
\begin{align}
    ab &= (2k + 1)(2l + 1) \\
    &= 4kl + 2l + 2k + 1 \\
    &= 2(2kl + l + k) + 1
\end{align}
Let $2kl + l + k$ be some integer $m$.
\begin{align}
    2(2kl + l + k) + 1 &= 2m + 1
\end{align}
By HW3 Fact 2, this shows that if both $a$ and $b$ are odd integers then $ab$ is odd, which contradicts our initial assumption. Thus, if $ab$ is even, then $a$ or $b$ is even.
\end{proof}
    
\end{enumerate}
\end{document}