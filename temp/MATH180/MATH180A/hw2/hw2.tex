\documentclass[addpoints, 11pt]{exam}
\setlength{\headsep}{0.25in}
\setlength{\unitlength}{1in}
%
\pagestyle{head}
%
\usepackage[utf8]{inputenc} %use Unicode
\usepackage[T1]{fontenc} %European fonts

\usepackage{%
	amsmath,       %some math tools
	amssymb,       %math symbols
	graphicx,      %enhanced graphics options
	mathtools,     %extension of amsmath
	microtype,     %small typographic effects
	bm,            %bold math symbols
	% todonotes,   %adds the option \todo{...} (use fixme instead)
	stmaryrd,      %some more math symbolswork
	% nicematrix,    %nicer matrix controls
	mathrsfs,      %more math fonts
        dsfont,
}
\usepackage{url}
\usepackage{hyperref}
%
\usepackage{amsthm}
% \newtheorem*{thm11.1.7}{Theorem 11.1.7}
%
\usepackage[shortlabels]{enumitem}
\usepackage{nicematrix}
\usepackage{multicol}
%
\usepackage[normalem]{ulem}
%
\newcommand{\myCourseNumber}{Math 180A}
\newcommand{\myName}{Ray Tsai}
\newcommand{\myID}{A16848188}
\newcommand{\myProfessor}{Professor Carfagnini}
\newcommand{\myHmwkNumber}{2}
% \newcommand{\myExamVersion}{}

\newcommand*{\Z}{\mathbb{Z}}
\newcommand*{\Q}{\mathbb{Q}}
\newcommand*{\R}{\mathbb{R}}
\newcommand*{\C}{\mathbb{C}}
\newcommand*{\N}{\mathbb{N}}
\newcommand*{\prob}{\mathds{P}}

%% define absolute value \abs{...}:
\DeclarePairedDelimiterX\abs[1]\lvert\rvert{%
  \ifblank{#1}{\:\cdot\:}{#1}
}
% define norm \norm{...}:
\DeclarePairedDelimiterX\norm[1]\lVert\rVert{%
  \ifblank{#1}{\:\cdot\:}{#1}
}
% define inner product \inner{...}{...}:
\DeclarePairedDelimiterX{\inner}[2]{\langle}{\rangle}{%
  \ifblank{#1}{\:\cdot\:}{#1},\ifblank{#2}{\:\cdot\:}{#2}
}

% define \set{...} to write sets and \given to write \set{... \given ...} for {...|...}
\newcommand*\setSymbol[1][]{
  \nonscript\:#1\vert\allowbreak\nonscript\:\mathopen{}
}
\providecommand\given{}
\DeclarePairedDelimiterX\set[1]{\lbrace}{\rbrace}{
  \renewcommand*\given{\setSymbol[\delimsize]}
  #1
}

% free group geneated by ... \free{...} or \free{... \given ...}
\DeclarePairedDelimiterX\free[1]{\langle}{\rangle}{
  \renewcommand\given{\nonscript\:\delimsize\vert\nonscript\:
    \mathopen{}}
  #1}

% define \lopen{...}{...}, \ropen{...}{...}, \open{...}{...}, \closed{...}{...} for intervals
\DeclarePairedDelimiterX\open[2](){#1,#2}
\DeclarePairedDelimiterX\lopen[2](]{#1,#2}
\DeclarePairedDelimiterX\ropen[2][){#1,#2}
\DeclarePairedDelimiterX\closed[2][]{#1,#2}

\NiceMatrixOptions{cell-space-limits = 1pt}
\newcommand*{\pmat}[1]{\begin{pNiceMatrix} #1 \end{pNiceMatrix}}
\newcommand*{\dfdx}[2]{\frac{\partial #1}{\partial #2}}

\DeclareMathOperator{\vol}{vol}
%
\pointsinmargin
\pointpoints{\thinspace point}{points}
\marginpointname{ \points}
%
\begin{document}
%
\firstpageheader{\bfseries \myCourseNumber}{\bfseries Homework \myHmwkNumber}{\bfseries \myName \\ \myID \\ \myProfessor}
%
\runningheader{}{(page \textit{\thepage}\ of \textit{\numpages})}{}
%
%

\begin{description}
    \item[Question 1] A fair coin is flipped three times. What is the probability that the second flip is tails, given that there is at most one tail among the three flips?
    \begin{proof}[Solution]
        The probability space of at most one tail is $\Omega = \{HHH, THH, HTH, HHT\}$.

        Thus, the probability that the second flip is tail given $\Omega$ is $\frac{1}{4}$.
    \end{proof}

\newpage

    \item[Question 2] We have two urns. The first urn contains two balls labeled 1 and 2. The second urn contains three balls labeled 3, 4 and 5. We choose one of the urns at random (with equal probability) and then sample one ball (uniformly at random) from the chosen urn. What is the probability that we picked the ball labeled 5?

    \begin{proof}[Solution]
    Let $X$ be the ball we obtained and $U$ be the urn we chose. Since it's impossible to obtain the ball labeled 5 if we choose the first urn, 
        \[
            \prob(X = 5) = \prob(X = 5| U = 2)\prob(U = 2).
        \]
    Thus, $\prob(X = 5) = \frac{1}{3} \cdot \frac{1}{2} = \frac{1}{6}$.
    \end{proof}

\newpage

    \item[Question 3]  When Alice spends the day with the babysitter, there is a $\frac{3}{5}$ probability that she turns on the TV and watches a show. Her little sister Betty cannot turn the TV on by herself. But once the TV is on, Betty watches with probability $\frac{4}{5}$. Tomorrow the girls spend the day with the babysitter.

    Let $A$ be the event where Alice watches TV, $B$ be the event where Betty watches TV, $AT$ be the event where Alice turns on the TV, and $BT$ be the event where Betty turns on the TV.
    \begin{enumerate}[(a)]
        \item What is the probability that both Alice and Betty watch TV tomorrow?

        \begin{proof}[Solution]
            \begin{align}
                \prob(A \cap B) 
                &= \prob(A \cap B|AT)\prob(AT) + \prob(A \cap B|BT)\prob(BT) \\
                &= \prob(A|AT)\prob(B|AT)\prob(AT) \\
                &= 1 \cdot \frac{4}{5} \cdot \frac{3}{5} = \frac{12}{25}.
            \end{align}
        \end{proof}

        \item What is the probability that Betty watches TV tomorrow?

        \begin{proof}[Solution]
            \begin{align}
            \prob(B) 
            &= \prob(B|AT)\prob(AT) + \prob(B|BT)\prob(BT) \\
            &= \frac{4}{5} \cdot \frac{3}{5} + 0 = \frac{12}{25}.
            \end{align}
        \end{proof}

        \item What is the probability that only Alice watches TV tomorrow?

        \begin{proof}[Solution]
            \begin{align}
            \prob(A \cap B^c) 
            &= \prob(A \cap B^c|AT)\prob(AT) + \prob(A \cap B^c|BT)\prob(BT) \\
            &= 1 \cdot \frac{1}{5} \cdot \frac{3}{5} + 0 = \frac{3}{25}.
            \end{align}
        \end{proof}
    \end{enumerate}

\newpage

    \item[Question 4] I have a bag with 3 fair dice. One is 4-sided, one is 6-sided, and one is 12-sided. I reach into the bag, pick one at random and roll it. The outcome of the roll is 4. What is the probability that I pulled out the 6-sided die?

    \begin{proof}[Solution]
        Let $X$ be the outcome of the roll and $D$ be the die I picked.
        \begin{align}
            \prob(D = 6 | X = 4)
            &= \frac{\prob(D = 6 \cap X = 4)}{\prob(X = 4)} \\
            &= \frac{\prob(D = 6 \cap X = 4)}{\prob(X = 4 \cap D = 4) + \prob(X = 4 \cap D = 6) + \prob(X = 4 \cap D = 12)} \\
            &= \frac{\frac{1}{6} \cdot \frac{1}{3}}{\frac{1}{4} \cdot \frac{1}{3} + \frac{1}{6} \cdot \frac{1}{3} + \frac{1}{12} \cdot \frac{1}{3}} \\
            &= \frac{1}{3}.
        \end{align}
    \end{proof}

\newpage

    \item[Question 5] Incoming students at a certain school take a mathematics placement exam. The possible scores are 1, 2, 3, and 4. From past experience, the school knows that if a particular student’s score is $x \in \{1, 2, 3, 4\}$, then the student will become a mathematics major with probability $\frac{x-1}{x+3}$. Suppose that the incoming class had the following scores: $10\%$ of the students scored a 1, $20\%$ of the students scored a 2, $60\%$ scored a 3, and $10\%$ scored a 4.

    Let $M$ be the event where the selected student would become a math major and $X$ be the score that the selected student got.

    \begin{enumerate}[(a)]
        \item What is the probability that a randomly selected student from the incoming class will become a mathematics major?

        \begin{proof}[Solution]
            \begin{align}
                \prob(M) 
                &= \sum_{x = 1}^4 \prob(M | X = x)\prob(X = x) \\
                &= \sum_{x = 1}^4 \frac{x-1}{x+3} \cdot \prob(X = x) \\
                &= 0 + \frac{1}{5} \cdot 20\% + \frac{1}{3} \cdot 60\% + \frac{3}{7} \cdot 10\% \\
                &= 4\% + 20\% + \frac{30}{7}\% \\
                &= \frac{99}{350}.
            \end{align}
        \end{proof}

        \item Suppose a randomly selected student from the incoming class turns out to be a mathematics major. What is the probability that he or she scored a 4 on the placement exam?

        \begin{proof}[Solution]
            \begin{align}
                \prob(X = 4 | M) 
                &= \frac{\prob(X = 4 \cap M)}{\prob(M)} \\
                &= \frac{\frac{3}{7} \cdot 10\%}{\frac{99}{350}} \\
                &= \frac{5}{33}.
            \end{align}
        \end{proof}
    \end{enumerate}

    \newpage

    \item[Question 6] Urn A contains 2 red and 4 white balls, and urn B contains 1 red and 1 white ball. A ball is randomly chosen from urn $A$ and put into urn $B$, and a ball is then chosen from urn $B$. What is the conditional probability that the transferred ball was white given that a white ball is selected from urn $B$?

    \begin{proof}[Solution]
        Let $W$ be the event where the transferred ball was white and $S$ be the event where a white ball is selected from $B$.

        \begin{align}
            \prob(W|S)
            &= \frac{\prob(W \cap S)}{\prob(S \cap W) + \prob(S \cap W^c)} \\
            &= \frac{\frac{2}{3} \cdot \frac{4}{6}}{\frac{2}{3} \cdot \frac{4}{6} + \frac{1}{3} \cdot \frac{1}{3}} = \frac{4}{5}.
        \end{align}
    \end{proof}

\newpage

    \item[Question 7] Consider the Monty Hall problem. In the case that the door you initially chose hid the valuable prize, suppose that Monty opens one of the other two doors for you not uniformly at random, but rather opens the second door with probability $p$ and the third door with probability $1 - p$, for some $p \in [0, 1]$. How does this affect your chances of getting the prize when switching?

    \begin{proof}[Solution]
        Let $C$ be the event where we chose the correct door initially, $S$ be the case where the second door is open and you chose the wrong door initially, and $T$ be the case where the second door is open and you chose the wrong door initially.
        \begin{align}
            \prob(\text{win by switching})
            &= \prob(\text{win by switching} \, | \, C)\prob(C) + \prob(\text{win by switching} \, | \, C^c)\prob(C^c) \\
            &= \prob(\text{win by switching} \, | \, C^c)\prob(C^c) \\
            &= \prob(\text{win by switching} \, | \, S)\prob(S) + \prob(\text{win by switching} \, | \, T)\prob(T) \\
            &= \prob(S) + \prob(T) \\
            &= \prob(C^c) = \frac{2}{3}.
        \end{align}
        Thus, $p$ does not affect the chances of getting the prize when switching.
    \end{proof}

    \newpage

    \item[Question 8] We choose a number from the set $\{1, 2, 3, . . . , 100\}$ uniformly at random and denote this number by $X$. For each of the following choices decide whether the two events in questions are independent or not?

    \begin{enumerate}[(a)]
        \item  $A$ = $\{\text{X is even}\}$, $B$ = $\{\text{X is divisible by 5}\}$.

        \begin{proof}[Solution]
        Since
            \[
                \prob(AB) = \prob(\text{X is divisible by 10}) = \frac{1}{10} = \prob(A)\prob(B),
            \]
            $A$, $B$ are independent.
        \end{proof}

        \item  $C$ = $\{\text{X is even}\}$, $D$ = $\{\text{X is divisible by 5}\}$, $E$ = $\{\text{X is divisible by 7}\}$.

        \begin{proof}[Solution]
        \begin{gather}
            \prob(DE) = \prob(\text{X is divisible by 35}) = \frac{2}{100} \\
            \prob(D)\prob(E) = \frac{20}{100} \cdot \frac{14}{100} = \frac{2.8}{100}
        \end{gather}
        Thus, $C$, $D$, $E$ are not mutually independent.
        \end{proof}

        \item  $F$ = $\{\text{X has two digits}\}$, $G$ = $\{\text{X is divisible by 3}\}$.

        \begin{proof}[Solution]
        We know $F \cap G = G \backslash \{3, 6, 9\}$. Since
            \begin{gather}
                \prob(FG) = \frac{33 - 3}{100} = \frac{3}{10} \\
                \prob(F)\prob(G) = \frac{9}{10} \cdot \frac{33}{100} = \frac{297}{1000}
            \end{gather}
            $F$, $G$ are not independent.
        \end{proof}

        \item  $H$ = $\{\text{X is prime}\}$, $I$ = $\{\text{X has a digit 5}\}$.

        \begin{proof}[Solution]
        We know $H \cap I = \{5, 53, 59\}$, and $|H| = 25$. Since
            \begin{gather}
                \prob(HI) = \frac{3}{100} \\
                \prob(H)\prob(I) = \frac{1}{4} \cdot \frac{10 + 10 - 1}{100} = \frac{19}{400}
            \end{gather}
            $H$, $I$ are not independent.
        \end{proof}
    \end{enumerate}
    
\end{description}

\end{document}
