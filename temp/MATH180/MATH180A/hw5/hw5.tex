\documentclass[addpoints, 11pt]{exam}
\setlength{\headsep}{0.25in}
\setlength{\unitlength}{1in}
%
\pagestyle{head}
%
\usepackage[utf8]{inputenc} %use Unicode
\usepackage[T1]{fontenc} %European fonts

\usepackage{%
	amsmath,       %some math tools
	amssymb,       %math symbols
	graphicx,      %enhanced graphics options
        amsfonts,
	mathtools,     %extension of amsmath
	microtype,     %small typographic effects
	bm,            %bold math symbols
	% todonotes,   %adds the option \todo{...} (use fixme instead)
	stmaryrd,      %some more math symbolswork
	% nicematrix,    %nicer matrix controls
	mathrsfs,      %more math fonts
        dsfont,
        xcolor,
}
\usepackage{url}
\usepackage{hyperref}
%
\usepackage{amsthm}
% \newtheorem*{thm11.1.7}{Theorem 11.1.7}
%
\usepackage[shortlabels]{enumitem}
\usepackage{nicematrix}
\usepackage{multicol}
%
\usepackage[normalem]{ulem}
% 
\definecolor{crimson}{rgb}{0.86,0.08,0.24}
%
\newcommand{\myCourseNumber}{MATH 180A}
\newcommand{\myName}{Ray Tsai}
\newcommand{\myID}{A16848188}
\newcommand{\myProfessor}{Professor Carfagnini}
\newcommand{\myHmwkNumber}{5}
% \newcommand{\myExamVersion}{}

\newcommand*{\Z}{\mathbb{Z}}
\newcommand*{\Q}{\mathbb{Q}}
\newcommand*{\R}{\mathbb{R}}
\newcommand*{\C}{\mathbb{C}}
\newcommand*{\N}{\mathbb{N}}
\newcommand*{\prob}{\mathds{P}}
\newcommand*{\E}{\mathds{E}}

\newenvironment{question}[1]{\smallskip\noindent\color{crimson}{\bf Question #1.}}{}
\allowdisplaybreaks[1]

%% define absolute value \abs{...}:
\DeclarePairedDelimiterX\abs[1]\lvert\rvert{%
  \ifblank{#1}{\:\cdot\:}{#1}
}
% define norm \norm{...}:
\DeclarePairedDelimiterX\norm[1]\lVert\rVert{%
  \ifblank{#1}{\:\cdot\:}{#1}
}
% define inner product \inner{...}{...}:
\DeclarePairedDelimiterX{\inner}[2]{\langle}{\rangle}{%
  \ifblank{#1}{\:\cdot\:}{#1},\ifblank{#2}{\:\cdot\:}{#2}
}

% define \set{...} to write sets and \given to write \set{... \given ...} for {...|...}
\newcommand*\setSymbol[1][]{
  \nonscript\:#1\vert\allowbreak\nonscript\:\mathopen{}
}
\providecommand\given{}
\DeclarePairedDelimiterX\set[1]{\lbrace}{\rbrace}{
  \renewcommand*\given{\setSymbol[\delimsize]}
  #1
}

% free group geneated by ... \free{...} or \free{... \given ...}
\DeclarePairedDelimiterX\free[1]{\langle}{\rangle}{
  \renewcommand\given{\nonscript\:\delimsize\vert\nonscript\:
    \mathopen{}}
  #1}

% define \lopen{...}{...}, \ropen{...}{...}, \open{...}{...}, \closed{...}{...} for intervals
\DeclarePairedDelimiterX\open[2](){#1,#2}
\DeclarePairedDelimiterX\lopen[2](]{#1,#2}
\DeclarePairedDelimiterX\ropen[2][){#1,#2}
\DeclarePairedDelimiterX\closed[2][]{#1,#2}

\NiceMatrixOptions{cell-space-limits = 1pt}
\newcommand*{\pmat}[1]{\begin{pNiceMatrix} #1 \end{pNiceMatrix}}
\newcommand*{\dfdx}[2]{\frac{\partial #1}{\partial #2}}

\DeclareMathOperator{\vol}{vol}
%
\pointsinmargin
\pointpoints{\thinspace point}{points}
\marginpointname{ \points}
%

\begin{document}
%
\firstpageheader{\bf \myCourseNumber}{\bf Homework \myHmwkNumber}{\bf \myName \\ \myID \\ \myProfessor}
%
\runningheader{}{(page \textit{\thepage}\ of \textit{\numpages})}{}
%
%

\begin{question}{1}
    The probability of getting a single pair in a poker hand of 5 cards is approximately 0.42. Find the approximate probability that out of 1000 poker hands there will be at least 450 with a single pair
\end{question}

\begin{proof}[Solution]
    Let $X$ be the number of hands with a single pair out of 1000 poker hands. Then $X \sim Bin(1000, 0.42)$, and so
    \begin{gather*}
        \E[X] = 420, \\
        \text{Var}(X) = 243.6.
    \end{gather*}
    Let $Z = \frac{X - 420}{\sqrt{243.6}}$.
    Thus, 
    \begin{align*}
        \prob(X \geq 450)
        &= \prob\left(Z \geq \frac{30}{\sqrt{243.6}}\right) \\
        &\approx 1 - \Phi\left(\frac{29}{\sqrt{243.6}}\right) \approx 0.0314.
    \end{align*}
\end{proof}

\newpage

\begin{question}{2}
    Approximate the probability that out of 300 die rolls, we get exactly 100 numbers that are multiples of 3.
\end{question}

\begin{proof}[Solution]
    Let $X$ be the number of multiples of 3 we get from 300 die rolls. We know $X \sim Bin(300, \frac{1}{3})$, and so
    \begin{gather*}
        \E[X] = 100, \\
        \text{Var}(X) = \frac{200}{3}.
    \end{gather*}
    Let $Z = \frac{X - 100}{\sqrt{\frac{200}{3}}}$.
    \begin{align*}
        \prob(99.5 \leq X \leq 100.5) 
        &= \prob\left(\frac{-0.5}{\sqrt{\frac{200}{3}}} \leq Z \leq \frac{0.5}{\sqrt{\frac{200}{3}}}\right) \\
        &\approx 2\Phi\left(\sqrt{\frac{3}{800}}\right) - 1 \\
        &\approx 0.0478.
    \end{align*}
\end{proof}

\newpage

\begin{question}{3}
    We roll a pair of dice 10,000 times. Estimate the probability that the number of times we get snake eyes (two ones) is between 280 and 300.
\end{question}

\begin{proof}[Solution]
    Let $X$ be the times we get snake eyes from rolling a pair of dice 10,000 times. The probability of getting a snake eye is $\frac{1}{36}$. Thus, $X \sim Bin(10000, \frac{1}{36})$, and so
    \begin{gather*}
        \E[X] = \frac{2500}{9}, \\
        \text{Var}(X) = \frac{21875}{81}.
    \end{gather*}
    Let $Z = \frac{X - \frac{2500}{9}}{\sqrt{\frac{21875}{81}}}$.
    \begin{align*}
        \prob(280 \leq X \leq 300)
        &= \prob\left(\frac{20}{\sqrt{21875}} \leq Z \leq \frac{200}{\sqrt{21875}}\right) \\
        &\approx \Phi\left(\frac{200}{\sqrt{21875}}\right) - \Phi\left(\frac{20}{\sqrt{21875}}\right) \\
        &\approx 0.3558.
    \end{align*}
\end{proof}

\newpage

\begin{question}{4}
    On the first 300 pages of a book, you notice that there are, on average, 6 typos per page. What is the probability that there will be at least 4 typos on page 301?
\end{question}

\begin{proof}[Solution]
    Let $X$ be the number of typos on a page. We assume that $X \sim \text{Possion(6)}$. Then, 
    \begin{align*}
        \prob(X \geq 4)
        &= 1 - \prob(X \leq 3) \\
        &\approx 1 - e^{-6}\sum_{k = 0}^{3} \frac{6^k}{k!} \\
        &\approx 1 - e^{-6}\left(1 + 6 + 18 + 36\right) \\
        &\approx 0.8488.
    \end{align*}
\end{proof}

\newpage

\begin{question}{5}
     Let $T \sim Exp(1/3)$.
\end{question}

\begin{enumerate}[(a)]
    \color{crimson}
    \item  Find $\prob(T > 3)$.
    \normalcolor

    \begin{proof}[Solution]
        $\prob(T > 3) = e^{-1}$.
    \end{proof}

    \color{crimson}
    \item  Find $\prob(1 \leq T < 8)$.
    \normalcolor

    \begin{proof}[Solution]
        $\prob(1 \leq T < 8) = \prob(1 \leq T) - \prob(8 \leq T) = e^{-\frac{1}{3}} - e^{-\frac{8}{3}}$.
    \end{proof}

    \color{crimson}
    \item  Find $\prob(T > 4 \, | \, T > 1)$.
    \normalcolor

    \begin{proof}[Solution]
        $\prob(T > 4 \, | \, T > 1) = \prob(T > 3) = e^{-1}$.
    \end{proof}
\end{enumerate}

\newpage

\begin{question}{6}
    Over the course of 365 days, 1 million radioactive atoms of Cesium-137 decayed to 977,287 radioactive atoms. Use the Poisson distribution to estimate the probability that on a given day, 50 radioactive atoms decayed.
\end{question}

\begin{proof}[Solution]
    Let $X$ be the number of radioactive atoms decayed in a day. Since 22,713 radioactive atoms of Celsium-137 decayed over the course of 365 days, we expect $\frac{22713}{365}$ to decay in a day, and so $X \sim \text{Poisson}\left(\frac{22713}{365}\right)$. Therefore, the probability of 50 radioactive atoms decaying is
    \[
        \prob(X = 50) = e^{-\frac{22713}{365}} \cdot \frac{\left(\frac{22713}{365}\right)^{50}}{50!} \approx 0.0155.
    \]
\end{proof}

\newpage

\begin{question}{7}
    Telephone calls enter a college switchboard on an average of two every three minutes. What is the probability of 5 or more calls arriving in a 9-minute period?
\end{question}

\begin{proof}[Solution]
    Let $X$ be the number of calls arriving in a 9-minute period. Since there are two calls arriving every three minutes on average, $\E[X] = 6$. We assume $X \sim \text{Poisson}(6)$. Then,
    the probability of $5$ or more calls arriving in a 9-minute period is
    \begin{align*}
        \prob(X \geq 5)
        &= 1 - \prob(X \leq 4) \\
        &\approx 1 - \sum_{k = 0}^{4} e^{-6} \cdot \frac{6^k}{k!} \\
        &\approx 1 - e^{-6}\left(1 + 6 + 18 + 36 + 54\right) \\
        &\approx 1 - 115e^{-6} = 0.7149.
    \end{align*}
\end{proof}

\end{document}
