\documentclass{article}

\usepackage{fancyhdr}
\usepackage{extramarks}
\usepackage{amsmath}
\usepackage{amsthm}
\usepackage{amsfonts}
\usepackage{tikz}
\usepackage[plain]{algorithm}
\usepackage{algpseudocode}
\usepackage{enumerate}
\usepackage{amssymb}
\usepackage{dsfont}

\usetikzlibrary{automata,positioning}

%
% Basic Document Settings
%

\topmargin=-0.45in
\evensidemargin=0in
\oddsidemargin=0in
\textwidth=6.5in
\textheight=9.0in
\headsep=0.25in

\linespread{1.1}

\pagestyle{fancy}
\lhead{\hmwkAuthorName}
\chead{\hmwkClass:\ \hmwkTitle}
\rhead{\firstxmark}
\lfoot{\lastxmark}
\cfoot{\thepage}

\renewcommand\headrulewidth{0.4pt}
\renewcommand\footrulewidth{0.4pt}

\setlength\parindent{0pt}
\setlength{\parskip}{5pt}

%
% Create Problem Sections
%

\newcommand{\enterProblemHeader}[1]{
    \nobreak\extramarks{}{Problem \arabic{#1} continued on next page\ldots}\nobreak{}
    \nobreak\extramarks{Problem \arabic{#1} (continued)}{Problem \arabic{#1} continued on next page\ldots}\nobreak{}
}

\newcommand{\exitProblemHeader}[1]{
    \nobreak\extramarks{Problem \arabic{#1} (continued)}{Problem \arabic{#1} continued on next page\ldots}\nobreak{}
    \stepcounter{#1}
    \nobreak\extramarks{Problem \arabic{#1}}{}\nobreak{}
}

\setcounter{secnumdepth}{0}
\newcounter{partCounter}
\newcounter{homeworkProblemCounter}
\setcounter{homeworkProblemCounter}{1}
\nobreak\extramarks{Problem \arabic{homeworkProblemCounter}}{}\nobreak{}

%
% Homework Problem Environment
%
% This environment takes an optional argument. When given, it will adjust the
% problem counter. This is useful for when the problems given for your
% assignment aren't sequential. See the last 3 problems of this template for an
% example.
%
\newenvironment{homeworkProblem}[1][-1]{
    \ifnum#1>0
        \setcounter{homeworkProblemCounter}{#1}
    \fi
    \section{Problem \arabic{homeworkProblemCounter}}
    \setcounter{partCounter}{1}
    \enterProblemHeader{homeworkProblemCounter}
}{
    \exitProblemHeader{homeworkProblemCounter}
}

%
% Homework Details
%   - Title
%   - Due date
%   - Class
%   - Section/Time
%   - Instructor
%   - Author
%

\newcommand{\hmwkTitle}{Homework\ \#3}
\newcommand{\hmwkDueDate}{Feb 9, 2023}
\newcommand{\hmwkClass}{MATH 180B}
\newcommand{\hmwkClassInstructor}{Professor Carfagnini}
\newcommand{\hmwkAuthorName}{\textbf{Ray Tsai}}
\newcommand{\hmwkPID}{A16848188}

%
% Title Page
%

\title{
    \vspace{2in}
    \textmd{\textbf{\hmwkClass:\ \hmwkTitle}}\\
    \normalsize\vspace{0.1in}\small{Due\ on\ \hmwkDueDate\ at 23:59pm}\\
    \vspace{0.1in}\large{\textit{\hmwkClassInstructor}} \\
    \vspace{3in}
}

\author{
  \hmwkAuthorName \\
  \vspace{0.1in}\small\hmwkPID
}
\date{}

\renewcommand{\part}[1]{\textbf{\large Part \Alph{partCounter}}\stepcounter{partCounter}\\}

%
% Various Helper Commands
%

% Useful for algorithms
\newcommand{\alg}[1]{\textsc{\bfseries \footnotesize #1}}

% For derivatives
\newcommand{\deriv}[1]{\frac{\mathrm{d}}{\mathrm{d}x} (#1)}

% For partial derivatives
\newcommand{\pderiv}[2]{\frac{\partial}{\partial #1} (#2)}

% Integral dx
\newcommand{\dx}{\mathrm{d}x}

% Probability commands: Expectation, Variance, Covariance, Bias
\newcommand{\Var}{\mathrm{Var}}
\newcommand{\Cov}{\mathrm{Cov}}
\newcommand{\Bias}{\mathrm{Bias}}
\newcommand*{\Z}{\mathbb{Z}}
\newcommand*{\Q}{\mathbb{Q}}
\newcommand*{\R}{\mathbb{R}}
\newcommand*{\C}{\mathbb{C}}
\newcommand*{\N}{\mathbb{N}}
\newcommand*{\p}{\mathds{P}}
\newcommand*{\E}{\mathds{E}}

\begin{document}

\maketitle

\pagebreak

\begin{homeworkProblem}
  A Markov chain \(X_0, X_1, \ldots\) on states \(0, 1, 2\) has the transition probability matrix

  \[
  P = \begin{bmatrix}
  0.1 & 0.2 & 0.7 \\
  0.9 & 0.1 & 0 \\
  0.1 & 0.8 & 0.1 \\
  \end{bmatrix}
  \]
  
  and initial distribution \(p_0 = \Pr\{X_0 = 0\} = 0.3\), \(p_1 = \Pr\{X_0 = 1\} = 0.4\), and \(p_2
  = \Pr\{X_0 = 2\} = 0.3\). Determine \(\Pr\{X_0 = 0, X_1 = 1, X_2 = 2\}\).
  
  \begin{proof}
    \[
      \Pr\{X_0 = 0, X_1 = 1, X_2 = 2\} = p_0P_{0, 1}P_{1, 2} = 0.3 \cdot 0.2 \cdot 0 = 0.
    \]
  \end{proof}
\end{homeworkProblem}

\newpage

\begin{homeworkProblem}
  A Markov chain \(X_0, X_1, X_2, \ldots\) has the transition probability matrix

  \[
  P = \begin{bmatrix}
  0.7 & 0.2 & 0.1 \\
  0 & 0.6 & 0.4 \\
  0.5 & 0 & 0.5 \\
  \end{bmatrix}
  \].

  Determine the conditional probabilities

  \[
  \Pr\{X_2 = 1, X_3 = 1 | X_1 = 0\} \quad \text{and} \quad \Pr\{X_1 = 1, X_2 = 1 | X_0 = 0\}.
  \]

  \begin{proof}
    \[
      \Pr\{X_2 = 1, X_3 = 1 | X_1 = 0\} = \Pr\{X_1 = 1, X_2 = 1 | X_0 = 0\} = P_{0, 1}P_{1, 1} = 0.2 \cdot 0.6 = 0.12.
    \]
  \end{proof}
\end{homeworkProblem}

\newpage

\begin{homeworkProblem}
  A simplified model for the spread of a disease goes this way: The total population size is \( N =
  5 \), of which some are diseased and the remainder are healthy. During any single period of time,
  two people are selected at random from the population and assumed to interact. The selection is
  such that an encounter between any pair of individuals in the population is just as likely as
  between any other pair. If one of these persons is diseased and the other not, with probability \(
  \alpha = 0.1 \) the disease is transmitted to the healthy person. Otherwise, no disease
  transmission takes place. Let \( X_n \) denote the number of diseased persons in the population at
  the end of the \( n \)th period. Specify the transition probability matrix.
  
  \begin{proof}
    Note that $X_n$ can either increment by one or stay the same in a single period of time. $X_n$
    increase by one if and only if one diseased and one healthy people are selected and the disease
    succfully transmitted during the period of time, of which the probability is $\alpha\frac{{X_{n
    - 1} \choose 1}{5 - X_{n - 1} \choose 1}}{{5 \choose 2}} = \frac{X_{n - 1}(5 - X_{n -
    1})}{100}$. Thus, the probability that $X_n$ remain the same is $1 - \frac{X_{n - 1}(5 - X_{n -
    1})}{100}$. Hence, we have the trasition matrix is
    \[
      \begin{bmatrix}
        0.96 & 0.04 & 0 & 0 & 0 \\
        0 & 0.94 & 0.06 & 0 & 0 \\
        0 & 0 & 0.94 & 0.06 & 0 \\
        0 & 0 & 0 & 0.96 & 0.04 \\
        0 & 0 & 0 & 0 & 1
      \end{bmatrix},
    \]
    where the $i$-th row $j$-th column represents the probability that the number of diseased
    transition from $N = i$ to $N = j$.
  \end{proof}
\end{homeworkProblem}

\newpage

\begin{homeworkProblem}
  A particle moves among the states 0, 1, 2 according to a Markov process whose transition
  probability matrix is

  \[
    P = \begin{bmatrix}
      0 & \frac{1}{2} & \frac{1}{2} \\
      \frac{1}{2} & 0 & \frac{1}{2} \\
      \frac{1}{2} & \frac{1}{2} & 0
    \end{bmatrix}.
  \]

  Let \( X_n \) denote the position of the particle at the $n$th move. Calculate \( \Pr\{X_n = 0
  \mid X_0 = 0\} \) for \( n = 0, 1, 2, 3, 4 \).

  \begin{proof}
    Note that
    \begin{gather*}
      P^2 = \begin{bmatrix}
        \frac{1}{2} & \frac{1}{4} & \frac{1}{4} \\
        \frac{1}{4} & \frac{1}{2} & \frac{1}{4} \\
        \frac{1}{4} & \frac{1}{4} & \frac{1}{2} \\
      \end{bmatrix}\quad
      P^3 = \begin{bmatrix}
        \frac{1}{4} & \frac{3}{8} & \frac{3}{8} \\
        \frac{3}{8} & \frac{1}{4} & \frac{3}{8} \\
        \frac{3}{8} & \frac{3}{8} & \frac{1}{4} \\
      \end{bmatrix}\quad
      P^4 = \begin{bmatrix}
        \frac{3}{8} & \frac{5}{16} & \frac{5}{16} \\
        \frac{5}{16} & \frac{3}{8} & \frac{5}{16} \\
        \frac{5}{16} & \frac{5}{16} & \frac{3}{8} \\
      \end{bmatrix}.
    \end{gather*}
    Thus,
    \begin{align*}
      &\Pr\{X_0 = 0\} = 1 \\
      &\Pr\{X_1 = 0 \mid X_0 = 0\} = 0 \\
      &\Pr\{X_2 = 0 \mid X_0 = 0\} = P^2_{0, 0} = \frac{1}{2} \\
      &\Pr\{X_3 = 0 \mid X_0 = 0\} = P^3_{0, 0} = \frac{1}{4} \\
      &\Pr\{X_4 = 0 \mid X_0 = 0\} = P^4_{0, 0} = \frac{3}{8}.
    \end{align*}
  \end{proof}
\end{homeworkProblem}

\newpage

\begin{homeworkProblem}
  Consider the Markov chain whose transition probability matrix is given by

  \[
    P = \begin{bmatrix}
      0.4 & 0.3 & 0.2 & 0.1 \\
      0.1 & 0.4 & 0.3 & 0.2 \\
      0.3 & 0.2 & 0.1 & 0.4 \\
      0.2 & 0.1 & 0.4 & 0.3 \\
    \end{bmatrix}
  \].

  Suppose that the initial distribution is \( p_i = \frac{1}{4} \) for \( i = 0, 1, 2, 3 \). Show
  that \( \Pr\{X_n = k\} = \frac{1}{4} \), \( k = 0, 1, 2, 3 \), for all \( n \). Can you deduce a
  general result from this example?

  \begin{proof}
    Let $v = (\frac{1}{4}, \frac{1}{4}, \frac{1}{4}, \frac{1}{4})^T$. Note that $Pv = v$, and thus
      we may conclude that $P^nv = PP^{n - 1}v = Pv = v$ by induction on $n$. Since $P^n_{ij} =
      \Pr\{X_n = j | X_{0} = 0\}$, we get that
      \begin{align*}
        \Pr\{X_n = k\} 
        &= \sum_{i = 0}^3 \Pr\{X_n = k | X_{0} = i\}\Pr\left\{X_{0} = i\right\} \\
        &= \sum_{i = 0}^3 P_{ik}^np_i\\
        &= P_{k}^nv = v_k = \frac{1}{4},
      \end{align*}
      where $P^n_k$ is the $k$-th row of $P^n$ and $v_k$ is the $k$-th entry of $v$.
  \end{proof}
\end{homeworkProblem}

\newpage

\begin{homeworkProblem}
  Consider two urns \( A \) and \( B \) containing a total of \( N \) balls. An experiment is
  performed in which a ball is selected at random (all selections equally likely) at time \( t \)
  (\( t = 1, 2, \ldots \)) from among the totality of \( N \) balls. Then, an urn is selected at
  random (\( A \) is chosen with probability \( p \) and \( B \) is chosen with probability \( q \))
  and the ball previously drawn is placed in this urn. The state of the system at each trial is
  represented by the number of balls in \( A \). Determine the transition matrix for this Markov
  chain.

  \begin{proof}
    We denote $X_t$ as the number of balls in $A$ at time $t$. We know $B$ contains $N - X_t$ balls
    at time $t$. Since we only move a ball at a time, $|X_{t + 1} - X_t| \leq 1$. Then,
    \begin{align*}
      \p(X_{t + 1} = x + 1 \mid X_t = x) 
      &= \p\{\text{Pick a ball from }A\text{ and pick urn } B\} \\ 
      &= \p\{\text{Pick a ball from }A\}\p\{\text{Pick urn } B\} \\
      &= \frac{x}{N} \cdot q
    \end{align*}
    \begin{align*}
      \p(X_{t + 1} = x - 1 \mid X_t = x) 
      &= \p\{\text{Pick a ball from }B\text{ and pick urn } A\} \\
      &= \p\{\text{Pick a ball from }B\}\p\{\text{Pick urn } A\} \\
      &= \frac{N - x}{N} \cdot p
    \end{align*}
    \begin{align*}
      \p(X_{t + 1} = x \mid X_t = x) 
      &= \p\{\text{Pick a ball from }A\text{ and pick urn } A\} + \p\{\text{Pick a ball from }B\text{ and pick urn } B\} \\
      &= \p\{\text{Pick a ball from }A\}\p\{\text{Pick urn } A\} + \p\{\text{Pick a ball from }B\}\p\{\text{Pick urn } B\} \\
      &= \frac{x}{N} \cdot p + \frac{N - x}{N} \cdot q.
    \end{align*}
    Hence, the entry at the $i$-th row $j$-column of the transition matrix is
    \[
      P_{ij} = \begin{cases}
        \frac{iq}{N} & j = i + 1 \\
        \frac{(N - i)p}{N} & j = i - 1 \\
        \frac{ip + (N - i)q}{N} & j = i \\
        0 & \text{otherwise}
      \end{cases}.
    \]
  \end{proof}
\end{homeworkProblem}
\end{document}