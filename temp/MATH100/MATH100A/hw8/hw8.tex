\documentclass{article}

\usepackage{fancyhdr}
\usepackage{extramarks}
\usepackage{amsmath}
\usepackage{amsthm}
\usepackage{amsfonts}
\usepackage{tikz}
\usepackage[plain]{algorithm}
\usepackage{algpseudocode}
\usepackage{enumerate}
\usepackage{amssymb}

\usetikzlibrary{automata,positioning}

%
% Basic Document Settings
%

\topmargin=-0.45in
\evensidemargin=0in
\oddsidemargin=0in
\textwidth=6.5in
\textheight=9.0in
\headsep=0.25in

\linespread{1.1}

\pagestyle{fancy}
\lhead{\hmwkAuthorName}
\chead{\hmwkClass:\ \hmwkTitle}
\rhead{\firstxmark}
\lfoot{\lastxmark}
\cfoot{\thepage}

\renewcommand\headrulewidth{0.4pt}
\renewcommand\footrulewidth{0.4pt}

\setlength\parindent{0pt}
\setlength{\parskip}{5pt}

%
% Create Problem Sections
%

\newcommand{\enterProblemHeader}[1]{
    \nobreak\extramarks{}{Problem \arabic{#1} continued on next page\ldots}\nobreak{}
    \nobreak\extramarks{Problem \arabic{#1} (continued)}{Problem \arabic{#1} continued on next page\ldots}\nobreak{}
}

\newcommand{\exitProblemHeader}[1]{
    \nobreak\extramarks{Problem \arabic{#1} (continued)}{Problem \arabic{#1} continued on next page\ldots}\nobreak{}
    \stepcounter{#1}
    \nobreak\extramarks{Problem \arabic{#1}}{}\nobreak{}
}

\setcounter{secnumdepth}{0}
\newcounter{partCounter}
\newcounter{homeworkProblemCounter}
\setcounter{homeworkProblemCounter}{1}
\nobreak\extramarks{Problem \arabic{homeworkProblemCounter}}{}\nobreak{}

%
% Homework Problem Environment
%
% This environment takes an optional argument. When given, it will adjust the
% problem counter. This is useful for when the problems given for your
% assignment aren't sequential. See the last 3 problems of this template for an
% example.
%
\newenvironment{homeworkProblem}[1][-1]{
    \ifnum#1>0
        \setcounter{homeworkProblemCounter}{#1}
    \fi
    \section{Problem \arabic{homeworkProblemCounter}}
    \setcounter{partCounter}{1}
    \enterProblemHeader{homeworkProblemCounter}
}{
    \exitProblemHeader{homeworkProblemCounter}
}

%
% Homework Details
%   - Title
%   - Due date
%   - Class
%   - Section/Time
%   - Instructor
%   - Author
%

\newcommand{\hmwkTitle}{Homework\ \#8}
\newcommand{\hmwkDueDate}{November 30, 2023}
\newcommand{\hmwkClass}{MATH 100A}
\newcommand{\hmwkClassTime}{Section A02 5:00PM - 5:50PM}
\newcommand{\hmwkSectionLeader}{Castellano}
\newcommand{\hmwkClassInstructor}{Professor McKernan}
\newcommand{\hmwkSource}{Source Consulted: Textbook, Lecture, Discussion}
\newcommand{\hmwkAuthorName}{\textbf{Ray Tsai}}
\newcommand{\hmwkPID}{A16848188}

%
% Title Page
%

\title{
    \vspace{2in}
    \textmd{\textbf{\hmwkClass:\ \hmwkTitle}}\\
    \normalsize\vspace{0.1in}\small{Due\ on\ \hmwkDueDate\ at 12:00pm}\\
    \vspace{0.1in}\large{\textit{\hmwkClassInstructor}} \\
    \vspace{0.1in}\small\hmwkClassTime \\
    \small Section Leader: \hmwkSectionLeader \\
    \vspace{0.1in}\small\hmwkSource \\
    \vspace{3in}
}

\author{
  \hmwkAuthorName \\
  \vspace{0.1in}\small\hmwkPID
}
\date{}

\renewcommand{\part}[1]{\textbf{\large Part \Alph{partCounter}}\stepcounter{partCounter}\\}

%
% Various Helper Commands
%

% Useful for algorithms
\newcommand{\alg}[1]{\textsc{\bfseries \footnotesize #1}}

% For derivatives
\newcommand{\deriv}[1]{\frac{\mathrm{d}}{\mathrm{d}x} (#1)}

% For partial derivatives
\newcommand{\pderiv}[2]{\frac{\partial}{\partial #1} (#2)}

% Integral dx
\newcommand{\dx}{\mathrm{d}x}

% Probability commands: Expectation, Variance, Covariance, Bias
\newcommand{\Var}{\mathrm{Var}}
\newcommand{\Cov}{\mathrm{Cov}}
\newcommand{\Bias}{\mathrm{Bias}}
\newcommand*{\Z}{\mathbb{Z}}
\newcommand*{\Q}{\mathbb{Q}}
\newcommand*{\R}{\mathbb{R}}
\newcommand*{\C}{\mathbb{C}}
\newcommand*{\N}{\mathbb{N}}
\newcommand*{\prob}{\mathds{P}}
\newcommand*{\E}{\mathds{E}}

\begin{document}

\maketitle

\pagebreak

\begin{homeworkProblem}
    Find the parity of each of permutation.

    \begin{enumerate}[(a)]
        \item $\begin{pmatrix}
            1 & 2 & 3 & 4 & 5 & 6 & 7 & 8 & 9 \\
            2 & 4 & 5 & 1 & 3 & 7 & 8 & 9 & 6
        \end{pmatrix}$.

        \begin{proof}
            Since
            \begin{align*}
                \begin{pmatrix}
                    1 & 2 & 3 & 4 & 5 & 6 & 7 & 8 & 9 \\
                    2 & 4 & 5 & 1 & 3 & 7 & 8 & 9 & 6
                \end{pmatrix}
                &= (1, 2, 4)(3, 5)(6, 7, 8, 9) \\
                &= (1, 2)(1, 4)(3, 5)(6, 7)(6, 8)(6, 9),
            \end{align*}
            the parity is even.
        \end{proof}

        \item $(1, 2, 3, 4, 5, 6)(7, 8, 9)$.
        
        \begin{proof}
            Since $(1, 2, 3, 4, 5, 6)(7, 8, 9) = (1, 2)(1, 3)(1, 4)(1, 5)(1, 6)(7, 8)(7, 9)$,
            the parity is odd.
        \end{proof}

        \item $(1, 2, 3, 4, 5, 6)(1, 2, 3, 4, 5, 7)$.
        
        \begin{proof}
            Since 
            \begin{align*}
                (1, 2, 3, 4, 5, 6)(1, 2, 3, 4, 5, 7) 
                &= (1, 2)(1, 3)(1, 4)(1, 5)(1, 6)(1, 2)(1, 3)(1, 4)(1, 5)(1, 7),
            \end{align*}
            the parity is even.
        \end{proof}

        \item $(1, 2)(1, 2, 3)(4, 5)(5, 6, 8)(1, 7, 9)$.
        
        \begin{proof}
            Since 
            \begin{align*}
                (1, 2)(1, 2, 3)(4, 5)(5, 6, 8)(1, 7, 9)
                &= (2, 3)(4, 5)(5, 6)(5, 8)(1, 7)(1, 9),
            \end{align*}
            the parity is even.
        \end{proof}
    \end{enumerate}
\end{homeworkProblem}

\pagebreak

\begin{homeworkProblem}
    If $\sigma$ is a $k$-cycle, show that $\sigma$ is an odd permutation if $k$ is even, and is an even permutation if $k$ is odd.
    
    \begin{proof}
        Since every $k$-cycle is a product of $k - 1$ transposes, the above statement holds.
    \end{proof}
\end{homeworkProblem}

\pagebreak

\begin{homeworkProblem}
    Prove that $\sigma$ and $\tau^{-1}\sigma\tau$, for any $\sigma$, $\tau \in S_n$, are of the same parity.

    \begin{proof}
        Since $\tau^{-1}\sigma\tau$ is the conjugate of $\sigma$, they are of the same cycle type, and thus they are of the same parity.
    \end{proof}
\end{homeworkProblem}

\pagebreak

\begin{homeworkProblem}
    Suppose that you are told that the permutation 
    \[
        \begin{pmatrix}
            1 & 2 & 3 & 4 & 5 & 6 & 7 & 8 & 9 \\
            3 & 1 & 2 &  &  & 7 & 8 & 9 & 6
        \end{pmatrix},
    \]
    in $S_9$, where the images of 5 and 4 have been lost, is an even permutation.
    What must the images of 5 and 4 be?

    \begin{proof}
        Notice that the permutation contains $(1, 3, 2)$ and $(6, 7, 8, 9)$, and all the numbers that are not classified to a cycle are 4 and 5.
        Since the permutation is even, 4 and 5 must form a transposition, otherwise the permutation can be decomposed into $2 + 3 = 5$ transpositions, which forces it to be odd.
    \end{proof}
\end{homeworkProblem}

\pagebreak

\begin{homeworkProblem}
    If $n \geq 3$, show that every element in $A_n$ is a product of $3$-cycles.

    \begin{proof}
        Note that every element $\sigma \in A_n$ can be decomposed into even number of transpositions.
        Suppose that $\sigma$ is the identity.
        Since the identity permutation can be represented as a$\prod\limits_{1 \leq x < y < z \leq n} (x, y, z)^3$, we may assume $\sigma$ is not the identity.
        Since $(a, b)(c, d) = (a, b)(a, c)(c, a)(c, d) = (a, b, c)(c, a, d)$ for any pair of distinct transpositions $(a, b)(c, d)$, we can pair up consecutive transpositions in $\sigma$ and convert each of them into a product of $3$-cycles, which makes $\sigma$ also a product of $3$-cycles.
    \end{proof}
\end{homeworkProblem}

\pagebreak

\begin{homeworkProblem}
    Show that every element in $A_n$ is a product of $n$-cycles.

    \begin{proof}
        Let $\sigma \in A_n$.
        Since the identity is simply the $n$-th power of any $n$-cycle, we may assume that $\sigma$ is not the identity.
        Let $(a_1, a_2)(b_1, b_2)$ be a pair of consecutive transpositions in $\sigma$.
        Note that $(a_1, a_2)$ and $(b_1, b_2)$ are distinct, otherwise they may cancell each other.
        Thus, we may assume $a_1 \neq b_1$.
        Let $\tau$ be a $(n - 2)$-cycle $\underbrace{(a_1, \dots, b_1)}_{n - 2 \text{ elements}}$ that only excludes $a_2$ and $b_2$.
        Then, 
        \begin{align*}
            \sigma
            &= (a_1, a_2)(b_1, b_2) \\
            &= (a_1, a_2)\tau\tau^{-1}(b_1, b_2) \\
            &= (a_1, a_2)(a_1, \dots, b_1)(b_1, \dots, a_1)(b_1, b_2) \\
            &= \underbrace{(a_1, a_2, \dots, b_1)}_{\text{only excludes }b_2}\underbrace{(b_1, \dots, a_1, b_2)}_{\text{only excludes }a_2} \\
            &= (a_2, \dots, b_1, a_1)(b_2, b_1, \dots, a_1) \\
            &= (a_2, \dots, b_1, a_1)(a_2, b_2)(b_2, a_2)(b_2, b_1, \dots, a_1) \\
            &= \underbrace{(a_2, \dots, b_1, a_1, b_2)}_{n\text{-cycle}} \underbrace{(b_2, a_2, b_1, \dots, a_1)}_{n\text{-cycle}}.
        \end{align*}
        Since $\sigma$ is even, we can pair up consecutive transpositions in $\sigma$ and convert each of them into a product of $n$-cycles, which makes $\sigma$ also a product of $n$-cycles.
    \end{proof}
\end{homeworkProblem}

\pagebreak

\begin{homeworkProblem}
    Find a normal subgroup in $A_4$ of order 4.
    
    \begin{proof}
       $A_4$ only contains even permutations, namely the identity, 3-cycles, and the product of 2 disjoint transpositions.
       3-cycles cannot be in a subgroup of order 4, so the subgroup can only contain the identity and the product of disjoint transpositions.
       There are $\frac{1}{2}{4 \choose 2} = 3$ cycles in $A_4$, so the group we are looking for can only be $S = \{(), (1, 2)(3, 4), (1, 3)(2, 4), (1, 4)(2, 3)\}$.
       Since
       \begin{align*}
            (1, 2)(3, 4)(1, 3)(2, 4) &= (1, 4)(2, 3) \\
            (1, 2)(3, 4)(1, 4)(2, 3) &= (1, 3)(2, 4) \\
            (1, 3)(2, 4)(1, 4)(2, 3) &= (1, 2)(3, 4),
       \end{align*}
       $S$ is a subset of a finite group and is closed under multiplication, so $S$ is a subgroup.
       Since $S$ contains the identity and all products of disjoint transpositions, $S$ is a union of conjugacy class, which makes $S$ a normal subgroup in $A_4$.
    \end{proof}
\end{homeworkProblem}
\end{document}