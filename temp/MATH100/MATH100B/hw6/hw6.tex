\documentclass{article}

\usepackage{fancyhdr}
\usepackage{extramarks}
\usepackage{amsmath}
\usepackage{amsthm}
\usepackage{amsfonts}
\usepackage{tikz}
\usepackage[plain]{algorithm}
\usepackage{algpseudocode}
\usepackage{enumerate}
\usepackage{amssymb}
\usepackage{stmaryrd}

\usetikzlibrary{automata,positioning}

%
% Basic Document Settings
%

\topmargin=-0.45in
\evensidemargin=0in
\oddsidemargin=0in
\textwidth=6.5in
\textheight=9.0in
\headsep=0.25in

\linespread{1.1}

\pagestyle{fancy}
\lhead{\hmwkAuthorName}
\chead{\hmwkClass:\ \hmwkTitle}
\rhead{\firstxmark}
\lfoot{\lastxmark}
\cfoot{\thepage}

\renewcommand\headrulewidth{0.4pt}
\renewcommand\footrulewidth{0.4pt}

\setlength\parindent{0pt}
\setlength{\parskip}{5pt}

%
% Create Problem Sections
%

\newcommand{\enterProblemHeader}[1]{
    \nobreak\extramarks{}{Problem \arabic{#1} continued on next page\ldots}\nobreak{}
    \nobreak\extramarks{Problem \arabic{#1} (continued)}{Problem \arabic{#1} continued on next page\ldots}\nobreak{}
}

\newcommand{\exitProblemHeader}[1]{
    \nobreak\extramarks{Problem \arabic{#1} (continued)}{Problem \arabic{#1} continued on next page\ldots}\nobreak{}
    \stepcounter{#1}
    \nobreak\extramarks{Problem \arabic{#1}}{}\nobreak{}
}

\setcounter{secnumdepth}{0}
\newcounter{partCounter}
\newcounter{homeworkProblemCounter}
\setcounter{homeworkProblemCounter}{1}
\nobreak\extramarks{Problem \arabic{homeworkProblemCounter}}{}\nobreak{}

%
% Homework Problem Environment
%
% This environment takes an optional argument. When given, it will adjust the
% problem counter. This is useful for when the problems given for your
% assignment aren't sequential. See the last 3 problems of this template for an
% example.
%
\newenvironment{homeworkProblem}[1][-1]{
    \ifnum#1>0
        \setcounter{homeworkProblemCounter}{#1}
    \fi
    \section{Problem \arabic{homeworkProblemCounter}}
    \setcounter{partCounter}{1}
    \enterProblemHeader{homeworkProblemCounter}
}{
    \exitProblemHeader{homeworkProblemCounter}
}

%
% Homework Details
%   - Title
%   - Due date
%   - Class
%   - Section/Time
%   - Instructor
%   - Author
%

\newcommand{\hmwkTitle}{Homework\ \#6}
\newcommand{\hmwkDueDate}{Feb 22, 2024}
\newcommand{\hmwkClass}{MATH 100B}
\newcommand{\hmwkClassTime}{Section A02 6:00PM - 6:50PM}
\newcommand{\hmwkSectionLeader}{Castellano-Macías}
\newcommand{\hmwkClassInstructor}{Professor McKernan}
\newcommand{\hmwkSource}{Source Consulted: Textbook, Lecture, Discussion, Office Hour}
\newcommand{\hmwkAuthorName}{\textbf{Ray Tsai}}
\newcommand{\hmwkPID}{A16848188}

%
% Title Page
%

\title{
    \vspace{2in}
    \textmd{\textbf{\hmwkClass:\ \hmwkTitle}}\\
    \normalsize\vspace{0.1in}\small{Due\ on\ \hmwkDueDate\ at 12:00pm}\\
    \vspace{0.1in}\large{\textit{\hmwkClassInstructor}} \\
    \vspace{0.1in}\small\hmwkClassTime \\
    \small Section Leader: \hmwkSectionLeader \\
    \vspace{0.1in}\small\hmwkSource \\
    \vspace{3in}
}

\author{
  \hmwkAuthorName \\
  \vspace{0.1in}\small\hmwkPID
}
\date{}

\renewcommand{\part}[1]{\textbf{\large Part \Alph{partCounter}}\stepcounter{partCounter}\\}

%
% Various Helper Commands
%

% Useful for algorithms
\newcommand{\alg}[1]{\textsc{\bfseries \footnotesize #1}}

% For derivatives
\newcommand{\deriv}[1]{\frac{\mathrm{d}}{\mathrm{d}x} (#1)}

% For partial derivatives
\newcommand{\pderiv}[2]{\frac{\partial}{\partial #1} (#2)}

% Integral dx
\newcommand{\dx}{\mathrm{d}x}

% Probability commands: Expectation, Variance, Covariance, Bias
\newcommand{\Var}{\mathrm{Var}}
\newcommand{\Cov}{\mathrm{Cov}}
\newcommand{\Bias}{\mathrm{Bias}}
\newcommand*{\Z}{\mathbb{Z}}
\newcommand*{\Q}{\mathbb{Q}}
\newcommand*{\R}{\mathbb{R}}
\newcommand*{\C}{\mathbb{C}}
\newcommand*{\N}{\mathbb{N}}
\newcommand*{\F}{\mathbb{F}}
\newcommand*{\prob}{\mathds{P}}
\newcommand*{\E}{\mathds{E}}

\begin{document}

\maketitle

\pagebreak

\begin{homeworkProblem}
  Let $M$ be an $R$-module and let $r \in R$. Show that the map
  \[
    \phi: M \rightarrow M \quad \text{given by} \quad m \mapsto rm
  \]
  is $R$-linear.
  
  \begin{proof}
    Let $m, n \in M$, and let $s \in R$. Since $\phi(m + n) = r(m + n) = rm + rn = \phi(m) +
    \phi(n)$, and $\phi(sm) = rsm = s\phi(m)$, $\phi$ is $R$-linear.
  \end{proof}
\end{homeworkProblem}

\newpage

\begin{homeworkProblem}
  Prove that a subset $N$ of an $R$-module is a submodule if and only if it is non-empty and closed
  under addition and scalar multiplication.

  \begin{proof}
    If $N$ is a submodule, then $N$ is an additive subgroup closed under scalar multiplication, by
    definition. Hence, it suffices to show the converse. Let $m, n \in N$ and $r, s \in R$. Since
    $N$ is closed under scalar multiplication, $-1 \cdot m = -m \in N$. Since $N$ is both closed
    under addition and taking inverses, $N$ is an additive subgroup, and $N$ is obviously abelian as
    it is a subset of an $R$-module. Since $m, n$ are elements in an $R$-module, $1 \cdot m = m$,
    $(rs) \cdot m = r \cdot (s \cdot m)$, $(r + s) \cdot m = r \cdot m + s \cdot m$, and $r \cdot (m
    + n) = r \cdot m + r \cdot n$. However, $N$ is closed under addition and scalar multiplication,
    so $r \cdot (s \cdot m), r \cdot m + s \cdot m, r \cdot m + r \cdot n \in N$. The result now
    follows.
  \end{proof}
\end{homeworkProblem}

\newpage

\begin{homeworkProblem}
  Let $\phi: M \rightarrow N$ be an $R$-linear map between two $R$-modules. Prove that the kernel of
  $\phi$ is a submodule of $M$.

  \begin{proof}
    Let $K$ be the kernel of $\phi$. Note that $\phi$ is a group homomorphism, so $K$ is an additive
   subgroup. It suffices to check that $K$ is closed under scalar multiplication. Let $m \in K$, and
   let $r \in R$. Since $\phi(rm) = r\phi(m) = r \cdot 0 = 0$, $rm \in K$, and we are done.
  \end{proof}
\end{homeworkProblem}

\newpage

\begin{homeworkProblem}
  Let $M$ be an $R$-module. Prove that the intersection of any set of submodules is a submodule.

  \begin{proof}
    Let $S$ be a set of submodules of $M$, and let $N = \bigcap_{A \in S} A$. It suffices to check
    that $N$ is nonempty, closed under addition and scalar multiplication. Let $m, n \in N$, and let
    $r \in R$. For all $A \in S$, $0, m + n, rm \in A$, and thus $0, m + n, rm \in N$. 
  \end{proof}
\end{homeworkProblem}

\newpage

\begin{homeworkProblem}
  Let $M$ be an $R$-module and let $X$ be any subset of $M$. Prove the existence of the submodule
  generated by $X$.

  \begin{proof}
    Let $N$ be the intersection of all submodules of $M$ that contains $X$. $N$ contains $X$ and any
    other submodules that contains $X$ also contains $N$. The result now follows from $N$.
  \end{proof}
\end{homeworkProblem}

\newpage

\begin{homeworkProblem}
  Let $M$ be an $R$-module and let $X$ be any set. Show how the set of all maps from $X$ to $M$
  becomes an $R$-module.

  \begin{proof}
    Let $S$ be the set of all maps $X \to M$. Since the map $e: X \to M$ which maps every element to
    0 is in $S$, $S$ is nonempty. Let $f, g \in S$. Since $M$ is associative, commutative, and
    closed under addition, $f + g$ is still a mapping from $X$ to $M$, and thus $S$ is associative,
    commutative, closed under addition. Since $f + e = e + f = f$, $e$ acts as the identity element
    in $S$. Let $-f$ be the map which sends $x$ to $-f(x)$. Since $(f + (-f))(x) = ((-f) + f)(x) =
    0$, $f + (-f) = (-f) + f = e$, so $S$ is closed under taking additive inverses. Therefore, $S$
    is an abelian group under addition. Let $r, s \in R$. Since $r \cdot f(x) \in M$, there exists
    $X \to M$ that maps $x$ to $r \cdot f(x)$, and thus $S$ is closed under scalar multiplication.
    Since $M$ is an $R$-module, $1 \cdot f(x) = f(x)$, $(rs) \cdot f(x) = r \cdot (s \cdot f(x)), (r
    + s) \cdot f(x) = r \cdot f(x) + s \cdot f(x)$, and $r \cdot (f + g)(x) = r \cdot (f(x) + g(x))
    = r \cdot f(x) + r \cdot g(x)$. It follows that $S$ meets all the rules to be a module over $R$.
  \end{proof}
\end{homeworkProblem}

\newpage

\begin{homeworkProblem}
  Let $M$ and $N$ be any two $R$-modules. Denote by $\text{Hom}_R(M, N)$ the set of all $R$-linear
  maps from $M$ to $N$. Show that this set is naturally an $R$-module.

  \begin{proof}
    Let $H = \text{Hom}_R(M, N)$. Since $H$ is a subset of $S$, the set of all maps $M \to N$, it
    suffices to show that $H$ nonempty, closed under addition, and closed under scalar
    multiplication, by Problem 6. Since $H$ contains the maps $M \to N$ that sends $m$ to $rm$ for
    some $r \in R$, $H$ is non empty. Let $f, g \in H$, $x, y \in M$, and $r \in R$. Since
    \[
      (f + g)(x + y) = f(x + y) + g(x + y) = f(x) + g(x) + f(y) + g(y) = (f + g)(x) + (f + g)(y),
    \]
    and
    \[
      (f + g)(rx) = f(rx) + g(rx) = r \cdot f(x) + r \cdot g(x) = r \cdot (f + g)(x),
    \]
    $f + g$ is a linear map, and so $H$ is closed under addition. Define $r \cdot f$ to be the
    mapping $M \to N$ that sends $m$ to $r \cdot f(m)$. Since 
    \[
      (r \cdot f)(x + y) = r \cdot f(x + y) = r \cdot f(x) + r \cdot f(y) = (r \cdot f)(x) + (r \cdot f)(y),
    \]
    and
    \[
      (r \cdot f)(sx) = r \cdot sf(x) = s \cdot (r \cdot f(x)) = s \cdot (r \cdot f)(x),
    \]
    for some $s \in R$, $H$ is closed under scalar multiplication, and the result follows.
  \end{proof}
\end{homeworkProblem}

\newpage

\begin{homeworkProblem}
  Let $M$ be an $R$-module and let $X$ be a subset of $M$. The annihilator $I$ of $X$, is the subset
  of all elements $r$ of $R$, such that $rm = 0$, for all elements $m$ of $X$. Show that $I$ is an
  ideal of $R$. Prove also that the annihilator of $X$ is equal to the annihilator of the submodule
  generated by $X$.

  \begin{proof}
    We first note that $I$ is nonempty, as $0 \in I$. Let $r, s \in I$, and let $m \in M$. Since $(r
    + s)m = rm + sm = 0$ and $(-r)m = -1 \cdot (rm) = 0$, $I$ is closed under addition and taking
    additive inverse, and thus $I$ is an additive subgroup. Let $k \in R$. Since $(kr)m = k(rm) =
    0$, $k \in I$, and thus $I$ is an ideal.

    Let $N$ be the submodule generated by $X$, and let $n \in N$. Since $n = r_1x_1 + r_2x_2 + \dots
    + r_kx_k$, for some $r_1, r_2, \dots, r_k \in R$ and $x_1, x_2, \dots, x_k \in X$, we get $rn =
    r(r_1x_1 + r_2x_2 + \dots + r_kx_k) = r_1(rx_1) + r_2(rx_2) + \dots + r_k(rx_k) = 0$, and the
    result follows.
  \end{proof}
\end{homeworkProblem}

\newpage

\begin{homeworkProblem}
  The next few results refer to the power series ring which is defined as follows. Let $R$ be a
  commutative ring and let $x$ be an indeterminate. The power series ring in $R$, denoted
  $R\llbracket x\rrbracket $, consists of all (possibly infinite) formal sums,
  \[
    \sum_{n\geq 0} a_nx^n,
  \]
  where $a_n \in R$. Thus if $R = \mathbb{Q}$, then both
  \[
    x - \frac{x^3}{3!} + \frac{x^5}{5!} + \ldots,
  \]
  and
  \[
    1 + 2!x + 3!x^2 + 4!x^3 + \ldots,
  \]
  are elements of $\Q\llbracket x\rrbracket $, even though the second, considered as a power series
  in the sense of analysis, does not converge for any $x \neq 0$. Addition and multiplication of
  elements of $R\llbracket x\rrbracket $ are defined as for polynomials.

  The degree of a power series is equal to the smallest $n$, so that the coefficient of $a_n$ is
  non-zero. Even for a polynomial, in what follows the degree always refers to the degree as a power
  series.

  \begin{enumerate}[(i)]
    \item Show that $R\llbracket x\rrbracket $ is a ring.
    \begin{proof}
      We first note that $0, 1 \in R \subset R\llbracket x\rrbracket $ are obviously the zero and
      unit of $R\llbracket x\rrbracket $. Let $f(x), g(x), h(x) \in R\llbracket x\rrbracket $, say
      $f(x) = \sum_{n\geq 0} f_nx^n$, $g(x) = \sum_{n\geq 0} g_nx^n$, and $h(x) = \sum_{n\geq 0}
      h_nx^n$. Then, $(f + g)(x) = \sum_{n\geq 0} (f_n + g_n)x^n$ and $(fg)(x) = \sum_{n\geq 0}
      k_nx^n$, where $k_n = \sum_{i \geq 0}^n f_ig_{n - i}$, and so $R\llbracket x\rrbracket $ is
      closed under addition and multiplication. Since $R$ is associative under addition,
      $R\llbracket x\rrbracket $ is associative under addition and multiplication. Since $-f(x) =
      \sum_{n\geq 0} -f_nx^n \in R\llbracket x\rrbracket $ such that $f(x) + (-f(x)) = (-f(x)) +
      f(x) = 0$, $R\llbracket x\rrbracket $ is closed under taking additive inverse. Since $f(g +
      h)(x) = \sum_{n\geq 0} l_nx^n = (fg)(x) + (gh)(x)$, where $l_n = \sum_{i \geq 0} f_i(g_{n - i}
      + h_{n - i}) =\sum_{i \geq 0} f_ig_{n - i} + f_ih_{n - i}$, $R\llbracket x\rrbracket $ is
      distributive. Hence, $R\llbracket x\rrbracket $ is a ring.
    \end{proof}
    \item Show that $f(x) \in R\llbracket x\rrbracket$ is invertible if and only if the degree of
    $f(x)$ is zero and the constant term is invertible. What is the inverse of $1 - x$?
    \begin{proof}
      Suppose that $f(x) = \sum_{n\geq 0} f_nx^n$ is invertible, with $g(x) = \sum_{n\geq 0} g_nx^n$
      as its inverse. We know $fg(x) = gf(x) = \sum_{n\geq 0} k_nx^n = 1$, where $k_n = \sum_{i \geq
      0} f_ig_{n - i}$. But then $f_0g_0 = g_0f_0 = 1$, so $f_0$ is nonzero and invertible. 
      
      We now assume the converse. Since $f_0$ is invertible, we may assume that $f(x) = 1 + a_1x +
      a_2x^2 + \dots$. Let $y = 1 - f(x)$. We show that $g(x) = 1 + y + y^2 + \dots$ is in
      $R\llbracket x\rrbracket$ and act as the inverse of $f(x)$. Notice that $1 - f(x)$ is of
      degree at least 1, so
      \[
        g(x) = 1 + y + y^2 + \dots = 1 + (1 - f(x)) + (1 - f(x))^2 + \dots = 1 + xf_{1}(x) + x^2f_{2}(x) + \dots,
      \]
      where $f_i(x) = a_{i, 0} + a_{i, 1}x + a_{i, 2}x^2 + \dots$, and $a_{i, k}$ is the $k$th
      coefficient of $(1 - f(x))^{i}$. In particular, $a_{i, k} \in R$ as $1 - f(x) \in R\llbracket
      x\rrbracket$, and thus the $k$th coefficients of $g(x)$ is $\sum_{i = 1}^{k} a_{i, k - i} \in
      R$. Then,
      \begin{align*}
        (fg)(x) = (gf)(x)
        &= (1 - y)(1 + y + y^2 + \dots) \\ 
        &= (1 + y + y^2 + \dots) - (y + y^2 + y^3 + \dots) \\
        &= 1,
      \end{align*}
      and the result follows.

      The inverse of $1 - x$ is obviously $1 + x + x^2 + \dots \in R\llbracket x\rrbracket$, as 
      \begin{align*}
        (1 + x + x^2 + \dots)(1 - x)
        &=  (1 - x)(1 + x + x^2 + \dots)\\
        &= (1 + x + x^2 + \dots) - (x + x^2 + x^3 + \dots) \\
        &= 1.
      \end{align*}
    \end{proof}
    \item Show that if $R$ is an integral domain then the degree of a product is the sum of the
    degrees.
    \begin{proof}
      Suppose that $f(x)$ has degree $m$ and $g(x)$ has degree $n$. If $a$ is the leading
      coefficient of $f(x)$ and $b$ is the leading coefficient of $g(x)$, then $f(x) = ax^m +
      \dots$, $g(x) = bx^n + \dots$, where $\dots$ indicate higher degree terms. Then, $(fg)(x) =
      (ax^m + \dots)(bx^n + \dots) = abx^{m + n} + \dots$. However, $R$ is an integral domain, so
      $ac \neq 0$, which means $(fg)(x) \neq 0$ and is of degree $m + n$.
    \end{proof}
    \item Show that if $R$ is an integral domain then so is $R\llbracket x\rrbracket $.
    \begin{proof}
      Let $f(x), g(x) \in R\llbracket x\rrbracket $ such that $f(x)g(x) = 0$. Then $\deg (fg)(x) =
      0$, by (iii). This means that $\deg f(x) = \deg g(x) = 0$, so $f(x) = a, g(x) = b$, for some
      $a, b \in R$. But then $ab = 0$, so either $a$ or $b$ is 0. The result then follows from
      either $f(x)$ or $g(x)$ is $0$.
    \end{proof}
    \item If $F$ is a field then prove that $F\llbracket x\rrbracket $ is a Euclidean domain.
    \begin{proof}
      Define $d: F\llbracket x\rrbracket   - \{0\} \to \N \cup \{0\}$ by sending $f(x)$ to its
      degree. Suppose that we are given $f(x), g(x) \in R\llbracket x\rrbracket $. By (iii),
      $d(f(x)) \leq d(fg(x))$. It remains to show that we can find $q(x), r(x)$ such that $g(x) =
      q(x)f(x) + r(x)$, where $d(r(x))$ is either $0$ or less than $d(f(x))$. We attempt to divide
      $f(x)$ into $g(x)$. If $\deg g(x) < \deg f(x)$, we take $q(x) = 0, r(x) = g(x)$ are we are
      done. Hence, we may assume $\deg g(x) \geq \deg f(x)$, say $f(x) = ax^m + \dots$, $g(x) = bx^n
      + \dots$, where $n > m$, $a, b \neq 0$, and $\dots$ indicate the higher degree terms. Notice
      that $f(x) = (a + \dots)x^m$ and $g(x) = (b + \dots)x^n$. By (ii), $(a + \dots)$ and $(b +
      \dots)$ are invertible as $F$ is a field, so there exists $h = (a + \dots)^{-1}, k = (b +
      \dots)^{-1} \in F\llbracket x\rrbracket $. Take $q(x) = khx^{n - m}$ and $r(x) = 0$. It
      follows that $g(x) = q(x)f(x) + r(x)$, and this completes the proof.
    \end{proof}
    \item Show that if $F'$ is a field then $F'\llbracket x\rrbracket $ is a UFD.
    \begin{proof}
      It follows from (vi) and Lemma 7.7 that $F\llbracket x\rrbracket $ is an Euclidean domain and
      thus a UFD.
    \end{proof}
  \end{enumerate}
\end{homeworkProblem}

\newpage

\begin{homeworkProblem}
  \begin{enumerate}[(i)]
    \item Prove that if $R$ is Noetherian then so is $R\llbracket x\rrbracket $
    \begin{proof}
      idk bro.
    \end{proof}
    \item Prove that if $R$ is Noetherian then so is $R\llbracket x_1, x_2, \ldots, x_n\rrbracket $,
    where the last term is defined appropriately.
    \begin{proof}
      We proceed by induction on $n$. By (i), $R\llbracket x_1\rrbracket $ is Noetherian. Suppose $n
      > 1$. We treat $R\llbracket x_1, x_2, \dots , x_n\rrbracket$ like polynomial rings. Then, by
      the universal property of polynomial rings, $$R\llbracket x_1, x_2, \dots , x_n\rrbracket
      \simeq R\llbracket x_1, x_2, \dots , x_{n - 1}\rrbracket \llbracket x_n\rrbracket.$$ By
      induction, $R\llbracket x_1, x_2, \dots , x_{n - 1}\rrbracket $ is Noetherian, and the result
      now follows from (i).
    \end{proof}
  \end{enumerate}
\end{homeworkProblem}
\end{document}