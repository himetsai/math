\documentclass{article}

\usepackage{fancyhdr}
\usepackage{extramarks}
\usepackage{amsmath}
\usepackage{amsthm}
\usepackage{amsfonts}
\usepackage{tikz}
\usepackage[plain]{algorithm}
\usepackage{algpseudocode}
\usepackage{enumerate}
\usepackage{amssymb}

\usetikzlibrary{automata,positioning}

%
% Basic Document Settings
%

\topmargin=-0.45in
\evensidemargin=0in
\oddsidemargin=0in
\textwidth=6.5in
\textheight=9.0in
\headsep=0.25in

\linespread{1.1}

\pagestyle{fancy}
\lhead{\hmwkAuthorName}
\chead{\hmwkClass:\ \hmwkTitle}
\rhead{\firstxmark}
\lfoot{\lastxmark}
\cfoot{\thepage}

\renewcommand\headrulewidth{0.4pt}
\renewcommand\footrulewidth{0.4pt}

\setlength\parindent{0pt}
\setlength{\parskip}{5pt}

%
% Create Problem Sections
%

\newcommand{\enterProblemHeader}[1]{
    \nobreak\extramarks{}{Problem \arabic{#1} continued on next page\ldots}\nobreak{}
    \nobreak\extramarks{Problem \arabic{#1} (continued)}{Problem \arabic{#1} continued on next page\ldots}\nobreak{}
}

\newcommand{\exitProblemHeader}[1]{
    \nobreak\extramarks{Problem \arabic{#1} (continued)}{Problem \arabic{#1} continued on next page\ldots}\nobreak{}
    \stepcounter{#1}
    \nobreak\extramarks{Problem \arabic{#1}}{}\nobreak{}
}

\setcounter{secnumdepth}{0}
\newcounter{partCounter}
\newcounter{homeworkProblemCounter}
\setcounter{homeworkProblemCounter}{1}
\nobreak\extramarks{Problem \arabic{homeworkProblemCounter}}{}\nobreak{}

%
% Homework Problem Environment
%
% This environment takes an optional argument. When given, it will adjust the
% problem counter. This is useful for when the problems given for your
% assignment aren't sequential. See the last 3 problems of this template for an
% example.
%
\newenvironment{homeworkProblem}[1][-1]{
    \ifnum#1>0
        \setcounter{homeworkProblemCounter}{#1}
    \fi
    \section{Problem \arabic{homeworkProblemCounter}}
    \setcounter{partCounter}{1}
    \enterProblemHeader{homeworkProblemCounter}
}{
    \exitProblemHeader{homeworkProblemCounter}
}

%
% Homework Details
%   - Title
%   - Due date
%   - Class
%   - Section/Time
%   - Instructor
%   - Author
%

\newcommand{\hmwkTitle}{Homework\ \#3}
\newcommand{\hmwkDueDate}{Feb 1, 2024}
\newcommand{\hmwkClass}{MATH 100B}
\newcommand{\hmwkClassTime}{Section A02 6:00PM - 6:50PM}
\newcommand{\hmwkSectionLeader}{Castellano-Macías}
\newcommand{\hmwkClassInstructor}{Professor McKernan}
\newcommand{\hmwkSource}{Source Consulted: Textbook, Lecture, Discussion, Office Hour}
\newcommand{\hmwkAuthorName}{\textbf{Ray Tsai}}
\newcommand{\hmwkPID}{A16848188}

%
% Title Page
%

\title{
    \vspace{2in}
    \textmd{\textbf{\hmwkClass:\ \hmwkTitle}}\\
    \normalsize\vspace{0.1in}\small{Due\ on\ \hmwkDueDate\ at 12:00pm}\\
    \vspace{0.1in}\large{\textit{\hmwkClassInstructor}} \\
    \vspace{0.1in}\small\hmwkClassTime \\
    \small Section Leader: \hmwkSectionLeader \\
    \vspace{0.1in}\small\hmwkSource \\
    \vspace{3in}
}

\author{
  \hmwkAuthorName \\
  \vspace{0.1in}\small\hmwkPID
}
\date{}

\renewcommand{\part}[1]{\textbf{\large Part \Alph{partCounter}}\stepcounter{partCounter}\\}

%
% Various Helper Commands
%

% Useful for algorithms
\newcommand{\alg}[1]{\textsc{\bfseries \footnotesize #1}}

% For derivatives
\newcommand{\deriv}[1]{\frac{\mathrm{d}}{\mathrm{d}x} (#1)}

% For partial derivatives
\newcommand{\pderiv}[2]{\frac{\partial}{\partial #1} (#2)}

% Integral dx
\newcommand{\dx}{\mathrm{d}x}

% Probability commands: Expectation, Variance, Covariance, Bias
\newcommand{\Var}{\mathrm{Var}}
\newcommand{\Cov}{\mathrm{Cov}}
\newcommand{\Bias}{\mathrm{Bias}}
\newcommand*{\Z}{\mathbb{Z}}
\newcommand*{\Q}{\mathbb{Q}}
\newcommand*{\R}{\mathbb{R}}
\newcommand*{\C}{\mathbb{C}}
\newcommand*{\N}{\mathbb{N}}
\newcommand*{\prob}{\mathds{P}}
\newcommand*{\E}{\mathds{E}}

\begin{document}

\maketitle

\pagebreak

\begin{homeworkProblem}
  If $a, b$ are integers and $3 \nmid a$ or $3 \nmid b$, show that $3 \nmid (a^2 + b^2)$.

  \begin{proof}
    Suppose that $3 \nmid a$. Then, $a \mod 3$ is either 1 or 2, so $a^2 \equiv 1 \mod 3$.
    Similarly, if $3 \nmid b$ then $b^2 \equiv 1 \mod 3$. Otherwise, $3 \nmid b$ and so $b^2 \equiv
    0 \mod 3$. This implies that $a^2 + b^2 \mod 3$ is either $2$ or $1$, so 3 does not divide $a^2
    + b^2$.
  \end{proof}
\end{homeworkProblem}

\newpage

\begin{homeworkProblem}
  Show that in Example 2, $R/M$ is a field having nine elements.

  \begin{proof}
    Since $M$ is a maximum ideal, $R/M$ is a field. Note that $M = \langle 3 \rangle$. Let $r = a +
    bi + M \in R/M$. Suppose that $a \equiv m \mod 3$ and $b \equiv n \mod 3$. Then, $r = m + 3l +
    (n + 3k)i + M = m + ni + 3(l + ki) + M = m + ni + M$. This implies that $r$ can only have $3
    \cdot 3 = 9$ possibilities, and thus $|R/M| = 9$.
  \end{proof}
\end{homeworkProblem}

\newpage

\begin{homeworkProblem}
  In Example 4, show that $R/M$ is a field having 25 elements.

  \begin{proof}
    Since $M = \langle 5 \rangle$, $M$ is an ideal. We show that $M$ is a maximal ideal. Suppose
    that $N \supset M$, $N \neq M$ is an ideal. There exists $a + b\sqrt{2} \in N \backslash M$.
    Consider $t = (a + b\sqrt{2})(a - b\sqrt{2}) = a^2 - 2b^2$. Since $N$ is an ideal in $R$, $t$ is
    an integer in $N$. Note that for all $x \nmid 5$, $x^2 \mod 5$ is either $1$ or $4$. Since $a +
    b\sqrt{2} \notin M$, we may assume $a \nmid 5$. Then, we know $a^2 \mod 5$ can be either 1 or 4,
    and $b^2 \mod 5$ can be either 0, 1, or 4. Notice that there are no possible combinations of $a,
    b$ that allow $t = a^2 - 2b^2 \equiv 0 \mod 5$, so $t \nmid 5$. Since $\gcd(t, 5) = 1$, we know
    $ut + 5w = 1$, for some $u, w \in \Z$. However, since $N, M$ are both ideals, $ut \in N$ and $5w
    \in M \subset N$, and thus $ut + 5w = 1 \in N$. This immediately follows that $N = R$, so $M$ is
    a maximal ideal, and thus $R/M$ is a field.

    Let $r = a + b\sqrt{2} + M \in R/M$. Suppose that $a \equiv m \mod 5$ and $b \equiv n \mod 5$.
    Then, $r = m + 5l + (n + 5k)\sqrt{2} + M = m + ni + 5(l + k\sqrt{2}) + M = m + n\sqrt{2} + M$.
    This implies that $r$ can only have $5 \cdot 5 = 25$ possibilities, and thus $|R/M| = 25$.
  \end{proof}
\end{homeworkProblem}

\newpage

\begin{homeworkProblem}
  Using Example 2 as a model, construct a field having 49 elements.

  \begin{proof}
    Consider $M = \langle 7 \rangle$. $M$ is an ideal. We show that $M$ is a maximal ideal. Suppose
    that $N \supset M$, $N \neq M$ is an ideal. There exists $a + bi \in N \backslash M$. Consider
    $t = (a + bi)(a - bi) = a^2 + b^2$. Since $N$ is an ideal in $R$, $t$ is an integer in $N$. Note
    that for all $x \nmid 7$, $x^2 \mod 7$ is either 1, 2, or 4. Since $a + bi \notin M$, we may
    assume $a \nmid 7$. Then, we know $a^2 \mod 7$ can be either 1, 2, or 4, and $b^2 \mod 7$ can be
    either 0, 1, 2, or 4. Notice that there are no possible combinations of $a, b$ that allow $t =
    a^2 + b^2 \equiv 0 \mod 7$, so $t \nmid 7$. Since $\gcd(t, 7) = 1$, we know $ut + 7w = 1$, for
    some $u, w \in \Z$. However, since $N, M$ are both ideals, $ut \in N$ and $7w \in M \subset N$,
    and thus $ut + 7w = 1 \in N$. This immediately follows that $N = R$, so $M$ is a maximal ideal,
    and thus $R/M$ is a field.

    Let $r = a + bi + M \in R/M$. Suppose that $a \equiv m \mod 7$ and $b \equiv n \mod 7$. Then, $r
    = m + 7l + (n + 7k)i + M = m + ni + 7(l + ki) + M = m + ni + M$. This implies that $r$ can only
    have $7 \cdot 7 = 49$ possibilities, and thus $|R/M| = 49$.
  \end{proof}
\end{homeworkProblem}

\newpage

\begin{homeworkProblem}
  Let $R$ be a ring and let $I$ be an ideal of $R$, not equal to the whole of $R$. Suppose that
  every element not in $I$ is a unit. Prove that $I$ is the unique maximal ideal in $R$.

  \begin{proof}
    Suppose that $N \supset I$, $N \neq I$ is an ideal. Then, there exists $n \in N$ such that $n$
    is invertible, which makes $N = R$. Thus, $I$ is a maximal ideal. We now show that $I$ is the
    unique maximal ideal. Let $I'$ be a maximal ideal. Since $I' \neq R$, all elements in $I'$ are
    not invertible. However, since every element not in $I$ is a unit and $I \neq R$, $I$ contains
    all the non-invertible elements, and thus $I \supseteq I'$. This immediately follows that $I'$
    is a maximal ideal, so $I = I'$.
  \end{proof}
\end{homeworkProblem}

\newpage

\begin{homeworkProblem}
  Let $\varphi: R \rightarrow S$ be a ring homomorphism and suppose that $J$ is a prime ideal of
  $S$.
  \begin{enumerate}[(i)]
      \item Prove that $I = \varphi^{-1}(J)$ is a prime ideal of $R$.
      \begin{proof}
        Define $\psi: S \rightarrow S/J$ as the natural projection. We know $\text{Ker }\psi \circ
        \varphi = \varphi^{-1}(\psi^{-1}(0)) = \varphi^{-1}(J) = I$. Suppose $a, b \in R$ such that
        $ab \in I$. Since $S/J$ is an integral domain, $\psi \circ \varphi(ab) = \psi \circ
        \varphi(a)\psi \circ \varphi(b) = 0$ implies that $\psi \circ \varphi(a)$ or $\psi \circ
        \varphi(b)$ is 0. This immediately follows that $a$ or $b$ is in $I$, so $I$ is a prime
        ideal.
      \end{proof}
      \item Give an example of an ideal $J$ that is maximal such that $I$ is not maximal.
      \begin{proof}
        Consider homomorphism $\Z \hookrightarrow \Q$. Since $\Q$ is a field, $\{0\}$ is a maximal
        ideal. However, the zero ideal is not a maximal ideal of $\Z$.
      \end{proof}
  \end{enumerate}
\end{homeworkProblem}

\newpage

\begin{homeworkProblem}
  Prove that every prime element of an integral domain is irreducible. Let $R$ be a commutative
  ring. 

  \begin{proof}
    Suppose that $p = ab$ is a prime. Hence, we may assume that $a \in \langle p \rangle$, or, $a =
    kp$ for some $k \in \R$. We then get $p = ab = kbp$, Since $R$ is an integral domain, $kb = 1$
    and the result now follows.
  \end{proof}
\end{homeworkProblem}

\newpage

\begin{homeworkProblem}
  Our aim is to prove a very strong form of the Chinese Remainder Theorem. First we need some
  definitions. Let $I$ and $J$ be two ideals. We say that $I$ and $J$ are coprime if $I + J = R$.

  \begin{enumerate}[(a)]
    \item Show that $I$ and $J$ are coprime if and only if there is an $i \in I$ and a $j \in J$
    such that $i + j = 1$.
    \begin{proof}
      Suppose that $I$ and $J$ are coprime. Then, $1 \in I + J = R$, so there exists $i \in I$ and
      $j \in J$ such that $i + j = 1$. Conversely, suppose that there is an $i \in I$ and a $j \in
      J$ such that $i + j = 1$. Consider $(i + j)r$ for some $r \in R$. Since $I, J$ are ideals, $ir
      \in I$ and $jr \in J$. However, this implies that $r = (i + j)r = ir + jr \in I + J$, so $I +
      J = R$. 
    \end{proof}
    \item Show that if $I$ and $J$ are coprime then $IJ = I \cap J$.
    \begin{proof}
      Let $ij \in IJ$, for $i \in I$ and $j \in J$. Since $I$ and $J$ are both ideals, $ij \in I$
      and $ij \in J$, so $ij \in I \cap J$. Conversely, suppose that $r = i + j \in I \cap J$. Since
      $I, J$ are groups under addition, $i + j \in I$ and $i + j \in J$ implies that $i, j \in I
      \cap J$. From part (a), we know that there exists $i' \in I, j' \in J$, such that $i' + j' =
      1$. Then we get $r = (i + j)(i' + j') = i'(i + j) + (i + j)j'$. However, since $i + j \in I
      \cap J$, we get that $i'(i + j) + (i + j)j' = i'j_r + i_rj' \in IJ$, for $i_r = j_r = i + j$.
      Therefore, $IJ = I \cap J$.
    \end{proof}
  \end{enumerate}
\end{homeworkProblem}

\newpage

\begin{homeworkProblem}
  Suppose that $I_1, I_2, \ldots, I_k$ are ideals of $R$. We say these ideals are pairwise coprime,
  if for all $i \neq j$, $I_i$ and $I_j$ are coprime. If $I_1, I_2, \ldots, I_k$ are pairwise
  coprime, show that the product $I$ of the ideals $I_1, I_2, \ldots, I_k$ is equal to the
  intersection, that is

  \[ \prod_{i=1}^{k} I_i = \bigcap_{i=1}^{k} I_i. \]

  \begin{proof}
    We proceed by induction on $k$. The base case is trivial. Suppose $k > 1$. By induction, we know
    $\prod_{i=1}^{k - 1} I_i = \bigcap_{i=1}^{k - 1} I_i$. Note that $\bigcap_{i=1}^{k - 1} I_i$ is
    an additive subgroup. Let $r \in R$ and $i \in \bigcap_{i=1}^{k - 1} I_i$. Since $I_1, I_2,
    \ldots, I_k$ are all ideals, $ri \in \bigcap_{i=1}^{k - 1} I_i$, and so the intersection is also
    an ideal. Thus, it suffices to show that $I_k$ and $\prod_{i=1}^{k - 1} I_i$ are coprime, by the
    result we obtained from the previous problem. Since $I_1, I_2, \ldots, I_k$ are pairwise
    coprime, for each pair of ideals, say $I_m, I_n$, there exists $i_m \in I_m$ and $i_n \in I_n$
    such that $i_m + i_n = 1$, which we denote as $1_{m, n}$. We know
    \[
      \prod_{m < n \leq k} 1_{m, n} = \prod_{m < n \leq k} (i_m + i_n) = 1.
    \]
    Note that in the expansion of $\prod_{m < n \leq k} (i_m + i_n)$, each term is the product of
    elements from at least $k - 1$ distinct ideals $I_i$, for $1 \leq i \leq k$. Since $I_k$ is an
    ideal, each term that is divided by an element from $I_k$ is in $I_k$, and thus there exists
    $i'_k \in I_k$ that is the sum of those terms. On the other hand, the rest of the terms are
    products of elements from each of $I_i$, for $1 \leq i < k$, and thus the sum of those terms is
    in $\prod_{i=1}^{k - 1} I_i$, we denote as $i_{\Pi}$. Therefore, we get $1 = i_k' + i_{\Pi} \in
    I_k + \prod_{i=1}^{k} I_i$. This immediately follows that $I_k + \prod_{i=1}^{k} I_i = R$, and
    we are done.
  \end{proof}
\end{homeworkProblem}

\newpage

\begin{homeworkProblem}
  Let $R_i$ denote the quotient $R/I_i$. Define a map,

  \[ \phi: R \longrightarrow \bigoplus_{i=1}^{k} R_i, \]  

  by $\phi(a) = (a + I_1, a + I_2, \ldots, a + I_k)$

\begin{enumerate}[(a)]
  \item Show that $\phi$ is a ring homomorphism.
  \begin{proof}
    $\phi$ is obviously well defined. Suppose that $a, b \in R$. Then,
    \begin{align*}
      \phi(a + b)
      &= (a + b + I_1, a + b + I_2, \ldots, a + b + I_k) \\
      &= (a + I_1, a + I_2, \ldots, a + I_k) + (b + I_1, b + I_2, \ldots, b + I_k) \\
      &= \phi(a) + \phi(b),
    \end{align*}
    \begin{align*}
      \phi(ab)
      &= (ab + I_1, ab + I_2, \ldots, ab + I_k) \\
      &= (a + I_1, a + I_2, \ldots, a + I_k)(b + I_1, b + I_2, \ldots, b + I_k) \\
      &= \phi(a)\phi(b),
    \end{align*}
    and most importantly, $\phi(1) = (1 + I_1, 1 + I_2, \ldots, 1 + I_k)$. Thus, $\phi$ is a
    homomorphism.
  \end{proof}
  \item  Show that $\phi$ is surjective if and only if the ideals $I_1, I_2,\ldots , I_k$ are
  pairwise coprime.
  \begin{proof}
    idk bro.
  \end{proof}
  \item Show that $\phi$ is injective if and only if $I$, the intersection of the ideals $I_1, I_2,
  \ldots, I_k$, is equal to the zero ideal.
  \begin{proof}
    Suppose that $\phi$ is injective. Then, we know for all $a, b \in R$, $a \neq b$ implies $(a +
    I_1, a + I_2, \ldots, a + I_k) \neq (b + I_1, b + I_2, \ldots, b + I_k)$. That is, there exists
    $I_i$ that does not contain $a - b$. However, for all nonzero $r \in R$, $r = m - n \neq 0$, for
    some $m, n \in R$, so $r \notin \bigcap_{i=1}^{k} I_i$. Since the intersection of groups is
    still a group, $I = \bigcap_{i=1}^{k} I_i$ can only be $\{0\}$.

    We now assume the converse. Suppose that $(a + I_1, a + I_2, \ldots, a + I_k) = (b + I_1, b +
    I_2, \ldots, b + I_k)$, for some $a, b \in R$. Then, $a - b \in \bigcap_{i=1}^{k} I_i = \{0\}$,
    so $a = b$. The result then follows.
  \end{proof}
\end{enumerate}
\end{homeworkProblem}

\newpage

\begin{homeworkProblem}
  Deduce the Chinese Remainder Theorem, which states that if $I_1, I_2, \ldots, I_k$ are pairwise
  coprime and the product $I$ is the zero ideal, then $R$ is isomorphic to $\bigoplus_{i=1}^k R_i$.
  Show how to deduce the other versions of the Chinese Remainder Theorem, which are stated as
  exercises in the book.

  \begin{proof}
    By the previous problem, the conditions given here makes the natural mapping $\phi: R
    \longrightarrow \bigoplus_{i=1}^{k} R_i$ an isomorphism, and thus $R \simeq \bigoplus_{i=1}^{k}
    R_i$. 

    We now deduce the other version of the Chinese Remainder Theorem. Let $m, n \in \Z$ such that
    $\gcd(m, n) = 1$. Consider $I_m = \langle m \rangle, I_n = \langle n \rangle$, for some coprime
    $m, n \in \Z$. That is, let $I_m$ be the set of multiples of $m$ and $I_n$ be the set of
    multiples of $n$. We know that $I_m \cap I_n = I_{mn}$, the set of multiples of $mn$. Since $m,
    n$ are coprime, $um + vn = 1$, for some $u, v \in \Z$. However, since $1 = um + vn \in I_m +
    I_n$, we know $I_m$ and $I_n$ are coprime. Since $I_m \cap I_n = I_{mn}$, we get $\Z/(I_m \cap
    I_n) = \Z_{mn} \simeq \Z_m \oplus \Z_n$.
  \end{proof}
\end{homeworkProblem}
\end{document}