\documentclass{article}

\usepackage{fancyhdr}
\usepackage{extramarks}
\usepackage{amsmath}
\usepackage{amsthm}
\usepackage{amsfonts}
\usepackage{tikz}
\usepackage[plain]{algorithm}
\usepackage{algpseudocode}
\usepackage{enumerate}
\usepackage{amssymb}

\usetikzlibrary{automata,positioning}

%
% Basic Document Settings
%

\topmargin=-0.45in
\evensidemargin=0in
\oddsidemargin=0in
\textwidth=6.5in
\textheight=9.0in
\headsep=0.25in

\linespread{1.1}

\pagestyle{fancy}
\lhead{\hmwkAuthorName}
\chead{\hmwkClass:\ \hmwkTitle}
\rhead{\firstxmark}
\lfoot{\lastxmark}
\cfoot{\thepage}

\renewcommand\headrulewidth{0.4pt}
\renewcommand\footrulewidth{0.4pt}

\setlength\parindent{0pt}
\setlength{\parskip}{5pt}

%
% Create Problem Sections
%

\newcommand{\enterProblemHeader}[1]{
    \nobreak\extramarks{}{Problem \arabic{#1} continued on next page\ldots}\nobreak{}
    \nobreak\extramarks{Problem \arabic{#1} (continued)}{Problem \arabic{#1} continued on next page\ldots}\nobreak{}
}

\newcommand{\exitProblemHeader}[1]{
    \nobreak\extramarks{Problem \arabic{#1} (continued)}{Problem \arabic{#1} continued on next page\ldots}\nobreak{}
    \stepcounter{#1}
    \nobreak\extramarks{Problem \arabic{#1}}{}\nobreak{}
}

\setcounter{secnumdepth}{0}
\newcounter{partCounter}
\newcounter{homeworkProblemCounter}
\setcounter{homeworkProblemCounter}{1}
\nobreak\extramarks{Problem \arabic{homeworkProblemCounter}}{}\nobreak{}

%
% Homework Problem Environment
%
% This environment takes an optional argument. When given, it will adjust the
% problem counter. This is useful for when the problems given for your
% assignment aren't sequential. See the last 3 problems of this template for an
% example.
%
\newenvironment{homeworkProblem}[1][-1]{
    \ifnum#1>0
        \setcounter{homeworkProblemCounter}{#1}
    \fi
    \section{Problem \arabic{homeworkProblemCounter}}
    \setcounter{partCounter}{1}
    \enterProblemHeader{homeworkProblemCounter}
}{
    \exitProblemHeader{homeworkProblemCounter}
}

%
% Homework Details
%   - Title
%   - Due date
%   - Class
%   - Section/Time
%   - Instructor
%   - Author
%

\newcommand{\hmwkTitle}{Homework\ \#4}
\newcommand{\hmwkDueDate}{Feb 8, 2024}
\newcommand{\hmwkClass}{MATH 100B}
\newcommand{\hmwkClassTime}{Section A02 6:00PM - 6:50PM}
\newcommand{\hmwkSectionLeader}{Castellano-Macías}
\newcommand{\hmwkClassInstructor}{Professor McKernan}
\newcommand{\hmwkSource}{Source Consulted: Textbook, Lecture, Discussion, Office Hour}
\newcommand{\hmwkAuthorName}{\textbf{Ray Tsai}}
\newcommand{\hmwkPID}{A16848188}

%
% Title Page
%

\title{
    \vspace{2in}
    \textmd{\textbf{\hmwkClass:\ \hmwkTitle}}\\
    \normalsize\vspace{0.1in}\small{Due\ on\ \hmwkDueDate\ at 12:00pm}\\
    \vspace{0.1in}\large{\textit{\hmwkClassInstructor}} \\
    \vspace{0.1in}\small\hmwkClassTime \\
    \small Section Leader: \hmwkSectionLeader \\
    \vspace{0.1in}\small\hmwkSource \\
    \vspace{3in}
}

\author{
  \hmwkAuthorName \\
  \vspace{0.1in}\small\hmwkPID
}
\date{}

\renewcommand{\part}[1]{\textbf{\large Part \Alph{partCounter}}\stepcounter{partCounter}\\}

%
% Various Helper Commands
%

% Useful for algorithms
\newcommand{\alg}[1]{\textsc{\bfseries \footnotesize #1}}

% For derivatives
\newcommand{\deriv}[1]{\frac{\mathrm{d}}{\mathrm{d}x} (#1)}

% For partial derivatives
\newcommand{\pderiv}[2]{\frac{\partial}{\partial #1} (#2)}

% Integral dx
\newcommand{\dx}{\mathrm{d}x}

% Probability commands: Expectation, Variance, Covariance, Bias
\newcommand{\Var}{\mathrm{Var}}
\newcommand{\Cov}{\mathrm{Cov}}
\newcommand{\Bias}{\mathrm{Bias}}
\newcommand*{\Z}{\mathbb{Z}}
\newcommand*{\Q}{\mathbb{Q}}
\newcommand*{\R}{\mathbb{R}}
\newcommand*{\C}{\mathbb{C}}
\newcommand*{\N}{\mathbb{N}}
\newcommand*{\prob}{\mathds{P}}
\newcommand*{\E}{\mathds{E}}

\begin{document}

\maketitle

\pagebreak

\begin{homeworkProblem}
  Let $R$ be an integral domain. Let $a$ and $b$ be two elements of $R$. Show that if $d$ and $d'$
  are both a gcd for the pair $a$ and $b$, then $d$ and $d'$ are associates.

  \begin{proof}
    Since $d$ and $d'$ both divides $a$ and $b$ and $d$ is a gcd, $d' | d$. However, $d'$ is also a
    gcd, so $d | d'$. The result then follows.
  \end{proof}
\end{homeworkProblem}

\newpage

\begin{homeworkProblem}
  Let $R$ be a UFD.
  \begin{enumerate}[(a)]
    \item Prove that for every pair of elements $a$ and $b$ of $R$, we may find an element $m =
    [a,b]$ that is a least common multiple, that is
    \begin{enumerate}[(i)]
        \item $a | m$ and $b | m$,
        \item and if $a | m'$ and $b | m'$ then $m | m'$.
    \end{enumerate}
    Show that any two lcm's are associates.
    \begin{proof}
      Let $a, b \in R$. If either $a$ or $b$ is $0$, then $0$ is the only possible common multiple
      of $a$ and $b$, and thus 0 is their lcm. Since $R$ is a UFD, we may put $a$ and $b$ into a
      standard form of prime factorizations
      \[
        a = up_1^{m_1}p_2^{m_2}\ldots p_k^{m_k} \quad \text{and} \quad b = vp_1^{n_1}p_2^{n_2}\ldots p_k^{n_k},
      \]
      where $u, v$ are invertible and $p_i$ and $p_j$ are associates if and only if $i = j$. Let $m
      = p_1^{l_1}p_2^{l_2}\ldots p_k^{l_k}$ such that $l_i = \max(m_i, n_i)$. It is obvious that $a
      | m$ and $b | m$. Suppose that $a | m'$ and $b | m'$. Then, $m' = op_1^{h_1}p_2^{h_2}\ldots
      p_k^{h_k}$, where $h_i \geq m_i$ and $h_i \geq n_i$, for all $i$. However, this means that
      $h_i \geq \max(m_i, n_i)$, so $m | m'$. Hence, $m$ is a least common multiple of $a$ and $b$.
      Suppose that $m'$ is also a least common multiple of $a, b$. Then, we have $m' | m$, which
      makes $m$ and $m'$ associates.
    \end{proof}
    \item Show that if $(a, b)$ denotes the gcd then $(a, b)[a, b]$ is an associate of $ab$.
    \begin{proof}
      Again, we put $a$ and $b$ into a standard form of prime factorizations
      \[
        a = up_1^{m_1}p_2^{m_2}\ldots p_k^{m_k} \quad \text{and} \quad b = vp_1^{n_1}p_2^{n_2}\ldots p_k^{n_k}.
      \]
      Let $d = \alpha p_1^{s_1}p_2^{s_2}\ldots p_k^{s_k}$, $m = \beta p_1^{l_1}p_2^{l_2}\ldots
      p_k^{l_k}$, where $d = (a, b)$, $m = [a, b]$, and $\alpha, \beta$ are invertible. Then, we
      know $l_i = \max(m_i, n_i)$ and $s_i = \min(m_i, n_i)$, for all $i$. However, this means that
      $l_i + s_i = m_i + n_i$, and thus
      \[
        dm = \alpha\beta p_1^{m_1 + n_1}p_2^{m_2 + n_2}\ldots p_k^{m_k + n_k} = \alpha\beta(uv)^{-1}ab.
      \]
      Since $\alpha\beta(uv)^{-1}$ is invertible, $dm$ and $ab$ are associates, and this completes
      the proof.
    \end{proof}
  \end{enumerate}
\end{homeworkProblem}

\newpage

\begin{homeworkProblem}
  Find the greatest common divisor of the following polynomials over $\Q$,
  \begin{enumerate}[(a)]
      \item $x^3 - 6x + 7$ and $x + 4$.
      \begin{proof}
        $x + 4$ is prime as it is degree 1, so either $x + 4 | x^3 - 6x + 7$ or they are coprime.
        However, $x^3 - 6x + 7 = (x + 4)(x^2 - 4x + 10) - 33$, so the greatest common divisor of the
        two polynomials is $1$.
      \end{proof}
      \item $x^3 - 1$ and $x^7 - x^4 + x^3 - 1$.
      \begin{proof}
        Note that $x^7 - x^4 + x^3 - 1 = (x^3 - 1)(x^4 + 1)$, so $x^3 - 1$ is their common divisor.
      \end{proof}
  \end{enumerate}
\end{homeworkProblem}

\newpage

\begin{homeworkProblem}
  Find the greatest common divisor of $135 - 14i$ and $155 + 34i$ in the ring of Gaussian integers
  $\Z[i]$.

  \begin{proof}
    We apply the Euclidean Algorithm. Since
    \[
      \frac{155 + 34i}{134 - 14i} = \frac{(134 + 14i)(155 + 34i)}{135^2 + 14^2} = \frac{20294 + 6726i}{18421} \approx 1.1 + 0.37i,
    \]
    we may pick $q = 1$ and the remainder is $r = (155 + 34i) - (135 - 14i)q = 20 + 48i$.

    Since
    \[
      \frac{135 - 14i}{20 + 48i} = 0.75 - 2.5i,
    \]
    we may pick $q = 1 - 2i$ and the remainder $r = (135 - 14i) - (20 + 48i)q = 19 - 22i$.

    Since
    \[
      \frac{20 + 48i}{19 - 22i} = -0.8 + 1.6i,
    \]
    we may pick $q = -1 + 2i$ and the remainder $r = (20 + 48i) - (19 - 22i)q = -5 - 12i$.

    Since
    \[
      \frac{19 - 22i}{-5 - 12i} = 1 + 2i,
    \]
    $-5 - 12i | 19 - 22i$ so there are no remainders left. Hence, the gcd of $135 - 14i$ and $155 +
    34i$ is $-5 - 12i$.
  \end{proof}
\end{homeworkProblem}

\newpage

\begin{homeworkProblem}
  \begin{enumerate}[(a)]
    \item Show that the elements $2$, $3$ and $1 \pm \sqrt{-5}$ are irreducible elements of
    $R = \Z[\sqrt{-5}]$.
    \begin{proof}
      Define $f: \Z[\sqrt{-5}] \rightarrow \Z_{\geq 0}$ as $f(a + b\sqrt{-5}) = a^2 + 5b^2$. For $a
      + b\sqrt{-5} \in R$, we know
      \begin{align*}
        f((a + b\sqrt{-5})(c + d\sqrt{-5})) 
        &= f(ac - 5bd + (ad + bc)\sqrt{-5})\\
        &= a^2c^2 + 5a^2d^2 + 5b^2c^2 + 25b^2d^2 \\
        &= (a^2 + 5b^2)(c^2 + 5d^2) \\
        &= f(a + b\sqrt{-5})f(c + d\sqrt{-5}),
      \end{align*}
      and $f(a + b\sqrt{-5}) \geq 0$. Notice that $f(a + b\sqrt{-5}) \geq 5$ if $b$ is positive, so
      $f(a + b\sqrt{-5}) = a^2 + 5b^2 = 1$ if and only if $a + b\sqrt{-5} = 1$, and thus $f(a +
      b\sqrt{-5}) \geq 2$ when $a + b\sqrt{-5}$ is not 0 or 1.

      Let $m = a + b\sqrt{-5}, n = c + d\sqrt{-5} \in R$. Suppose that $mn = 2$. Then, $f(2) = 4 =
      f(m)f(n)$, so $f(m)$ or $f(n)$ is a multiple of 2. Suppose that $f(m)$ is a multiple of 2. We
      know $f(m) = a^2 + 5b^2$ cannot be $2$, as $f(m) > 2$ if $b$ is positive but there are no
      integers such that $a^2 = 2$, so $f(m) = 4$. But then $f(n) = 1$, so $n = 1$, which is
      invertible. Hence, $2$ is irreducible.

      Suppose that $mn = 3$. Similarly, $f(3) = 9 = f(m)f(n)$, so $f(m)$ or $f(n)$ is a multiple of
      3. Suppose that $f(m)$ is a multiple of 3. We know $f(m) = a^2 + 5b^2$ cannot be $3$, as $f(m)
      > 3$ if $b$ is positive but there are no integers such that $a^2 = 3$, so $f(m) = 9$. But then
      $f(n) = 1$, so $n = 1$, which is invertible. Hence, $3$ is irreducible.

      Suppose that $mn = 1 \pm \sqrt{-5}$. Then, $f(1 \pm \sqrt{-5}) = f(m)f(n) = 6$. Suppose for
      the sake of contradiction that $m, n \neq 1$. Then, either $f(m)$ or $f(n)$ must be 2.
      However, we already know $f(k) \neq 2$ for all $k \in R$, contradiction. Hence, either $m$ or
      $n$ is 1, so $1 \pm \sqrt{-5}$ is irreducible.
    \end{proof}
    \item Show that every element of $R$ can be factored into irreducibles.
    \begin{proof}
      By Proposition 6.11, it suffices to show that the set of principal ideals of $R$ satisfies
      ACC. Suppose that we have an increasing sequence of principal ideals of $R$
      \[
        \langle a_1 \rangle \subset \langle a_2 \rangle \subset \dots \subset \langle a_n \rangle \subset \cdots,
      \]
      for $a_1 \neq 0$ and $a_i = a_j$ if and only if $i = j$. Suppose for the sake of contradiction
      that the increasing sequence does not stabilize. Since for $i \in \Z^+$, $a_{i + 1} = ka_i$
      for some $k \in R$, we know $f(a_{i + 1}) = f(k)f(a_i)$. Hence, $f(a_{i + 1}) \geq 2f(a_i)$,
      as $k \neq 1$. Since the sequence does not stabilize and $f(a_{i})$ is finite, $f(a_{n}) < 1$
      for large enough $n$. But then $a_{n} = 0$, which forces $\langle a_{n} \rangle = \{0\}$, and
      this contradiction completes the proof.
    \end{proof}
    \item Show that $R$ is not a UFD.
    \begin{proof}
      Consider 2. We already know 2 is irreducible. Notice that, $2 | (1 + \sqrt{-5})(1 -
      \sqrt{-5})$. Suppose for the sake of contradiction that $2 = x(1 + \sqrt{-5})$, for some $x
      \in R$. Then, $x = \frac{1 - \sqrt{-5}}{3} \notin R$, contradiction. Similarly, we also know
      $2 \nmid 1 - \sqrt{-5}$. Hence, $2$ does not divide $1 \pm \sqrt{-5}$, so $2$ is not prime.
      The result now follows from Proposition 6.17.
    \end{proof}
  \end{enumerate}
\end{homeworkProblem}
\end{document}