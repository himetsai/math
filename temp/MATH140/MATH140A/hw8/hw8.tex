\documentclass{article}

\usepackage{fancyhdr}
\usepackage{extramarks}
\usepackage{amsmath}
\usepackage{amsthm}
\usepackage{amsfonts}
\usepackage{tikz}
\usepackage[plain]{algorithm}
\usepackage{algpseudocode}
\usepackage{enumerate}
\usepackage{amssymb}

\usetikzlibrary{automata,positioning}

%
% Basic Document Settings
%

\topmargin=-0.45in
\evensidemargin=0in
\oddsidemargin=0in
\textwidth=6.5in
\textheight=9.0in
\headsep=0.25in

\linespread{1.1}

\pagestyle{fancy}
\lhead{\hmwkAuthorName}
\chead{\hmwkClass:\ \hmwkTitle}
\rhead{\firstxmark}
\lfoot{\lastxmark}
\cfoot{\thepage}

\renewcommand\headrulewidth{0.4pt}
\renewcommand\footrulewidth{0.4pt}

\setlength\parindent{0pt}
\setlength{\parskip}{5pt}

%
% Create Problem Sections
%

\newcommand{\enterProblemHeader}[1]{
    \nobreak\extramarks{}{Problem \arabic{#1} continued on next page\ldots}\nobreak{}
    \nobreak\extramarks{Problem \arabic{#1} (continued)}{Problem \arabic{#1} continued on next page\ldots}\nobreak{}
}

\newcommand{\exitProblemHeader}[1]{
    \nobreak\extramarks{Problem \arabic{#1} (continued)}{Problem \arabic{#1} continued on next page\ldots}\nobreak{}
    \stepcounter{#1}
    \nobreak\extramarks{Problem \arabic{#1}}{}\nobreak{}
}

\setcounter{secnumdepth}{0}
\newcounter{partCounter}
\newcounter{homeworkProblemCounter}
\setcounter{homeworkProblemCounter}{1}
\nobreak\extramarks{Problem \arabic{homeworkProblemCounter}}{}\nobreak{}

%
% Homework Problem Environment
%
% This environment takes an optional argument. When given, it will adjust the
% problem counter. This is useful for when the problems given for your
% assignment aren't sequential. See the last 3 problems of this template for an
% example.
%
\newenvironment{homeworkProblem}[1][-1]{
    \ifnum#1>0
        \setcounter{homeworkProblemCounter}{#1}
    \fi
    \section{Problem \arabic{homeworkProblemCounter}}
    \setcounter{partCounter}{1}
    \enterProblemHeader{homeworkProblemCounter}
}{
    \exitProblemHeader{homeworkProblemCounter}
}

%
% Homework Details
%   - Title
%   - Due date
%   - Class
%   - Section/Time
%   - Instructor
%   - Author
%

\newcommand{\hmwkTitle}{Homework\ \#8}
\newcommand{\hmwkDueDate}{Mar 8, 2024}
\newcommand{\hmwkClass}{MATH 140A}
\newcommand{\hmwkClassInstructor}{Professor Seward}
\newcommand{\hmwkAuthorName}{\textbf{Ray Tsai}}
\newcommand{\hmwkPID}{A16848188}

%
% Title Page
%

\title{
    \vspace{2in}
    \textmd{\textbf{\hmwkClass:\ \hmwkTitle}}\\
    \normalsize\vspace{0.1in}\small{Due\ on\ \hmwkDueDate\ at 23:59pm}\\
    \vspace{0.1in}\large{\textit{\hmwkClassInstructor}} \\
    \vspace{3in}
}

\author{
  \hmwkAuthorName \\
  \vspace{0.1in}\small\hmwkPID
}
\date{}

\renewcommand{\part}[1]{\textbf{\large Part \Alph{partCounter}}\stepcounter{partCounter}\\}

%
% Various Helper Commands
%

% Useful for algorithms
\newcommand{\alg}[1]{\textsc{\bfseries \footnotesize #1}}

% For derivatives
\newcommand{\deriv}[1]{\frac{\mathrm{d}}{\mathrm{d}x} (#1)}

% For partial derivatives
\newcommand{\pderiv}[2]{\frac{\partial}{\partial #1} (#2)}

% Integral dx
\newcommand{\dx}{\mathrm{d}x}

% Probability commands: Expectation, Variance, Covariance, Bias
\newcommand{\Var}{\mathrm{Var}}
\newcommand{\Cov}{\mathrm{Cov}}
\newcommand{\Bias}{\mathrm{Bias}}
\newcommand*{\Z}{\mathbb{Z}}
\newcommand*{\Q}{\mathbb{Q}}
\newcommand*{\R}{\mathbb{R}}
\newcommand*{\C}{\mathbb{C}}
\newcommand*{\N}{\mathbb{N}}
\newcommand*{\prob}{\mathds{P}}
\newcommand*{\E}{\mathds{E}}

\begin{document}

\maketitle

\pagebreak

\begin{homeworkProblem}
  Suppose $a_n > 0$, $s_n = a_1 + \cdots + a_n$, and $\sum a_n$ diverges.
  \begin{enumerate}[(a)]
      \item Prove that $\sum \frac{a_n}{(1 + a_n)}$ diverges.
      \begin{proof}
        Note that if $a_n > 1$, then $\frac{a_n}{a_n + 1} = 1 - \frac{1}{a_n + 1} > \frac{1}{2}$. On
        the other hand, if $a_n \leq 1$, we have $\frac{a_n}{a_n + 1} \geq \frac{a_n}{2}$. If there
        are infinitely many $n$ such that $a_n > 1$, then the series obviously diverges, as it would
        be greater than the sum of infinitely many $\frac{1}{2}$. Hence, we may assume there exists
        $N \geq 0$ such that $a_n \leq 1$ for all $n \geq N$. But then
        \[
          \sum \frac{a_n}{(1 + a_n)} \geq \sum_{n = 1}^{N - 1} \frac{a_n}{(1 + a_n)} + \frac{1}{2}\sum_{n = N}^{\infty} a_n.
        \]
        Since $\sum a_n$ diverges, $\frac{1}{2}\sum_{n = N}^{\infty} a_n$ diverges, by comparison
        test. The result now follows.
      \end{proof}
      \item Prove that 
      \[
        \frac{a_{N+1}}{s_{N+1}} + \cdots + \frac{a_{N+k}}{s_{N+k}} \geq 1 - \frac{s_N}{s_{N+k}}
      \]
      and deduce that $\sum \frac{a_n}{s_n}$ diverges.
      \begin{proof}
        We first note that 
        \[
          \frac{a_{N+1}}{s_{N+1}} + \cdots + \frac{a_{N+k}}{s_{N+k}} \geq \frac{a_{N + 1} + \dots + a_{N + k}}{s_{N + k}} =  1 - \frac{s_N}{s_{N+k}}.
        \]
        Fix $\epsilon \in (0, 1)$. Since $S_n$ is increasing and unbounded, $\frac{s_N}{s_{N+k}} \to
        0$. Hence, we may find large enough $k$ such that $\frac{s_N}{s_{N+k}} < 1 - \epsilon$. But
        then $\sum_{n = N + 1}^{N + k} \frac{a_n}{s_n} \geq \epsilon$, which fails to meet the
        Cauchy criterion.
      \end{proof}
      \item Prove that 
      \[
        \frac{a_n}{s_n^2} \leq \frac{1}{s_{n-1}} - \frac{1}{s_n}
      \]
      and deduce that $\sum \frac{a_n}{s_n^2}$ converges.
      \begin{proof}
        Since
        \[
          \frac{a_n}{s_n^2} \leq \frac{a_n}{s_{n - 1}s_n} = \frac{1}{s_{n-1}} - \frac{1}{s_n},
        \]
        the consecutive terms cancel out, and we get $\sum_{n = 1}^N \frac{a_n}{s_n^2} \leq \sum_{n
        = 2}^N \frac{1}{s_{n-1}} - \frac{1}{s_n} = \frac{1}{a_1} - \frac{1}{s_N}$. But then $s_n$ is
        increasing and unbounded, and thus
        \[
          \frac{1}{a_1} \leq \lim_{N \to \infty} \sum \frac{a_n}{s_n^2} \leq \lim_{N \to \infty} \frac{1}{a_1} - \frac{1}{s_N} = \frac{1}{a_1}.
        \]
        Hence, the series converges to $\frac{1}{a_1}$.
      \end{proof}
  \end{enumerate}
\end{homeworkProblem}

\newpage

\begin{homeworkProblem}
  Suppose $a_n > 0$ and $\sum a_n$ converges. Put 
    \[
    r_n = \sum_{m=n}^{\infty} a_m.
    \]
  \begin{enumerate}[(a)]
    \item Prove that 
    \[
      \frac{a_m}{r_m} + \dots + \frac{a_n}{r_n} > 1 - \frac{r_n}{r_m}
    \]
    if $m < n$, and deduce that $\sum \frac{a_n}{r_n}$ diverges.
    \begin{proof}
      Let $A = \sum_{n = 1}^{\infty} a_n$. We know $r_n = A - s_n$, where $s_n$ is the sum of the
      first $n - 1$ terms of $a_n$. Note that $r_n < r_m$, as $s_{n} > s_m$. Hence,
      \[
        \frac{a_m}{r_m} + \dots + \frac{a_n}{r_n} > \frac{a_m + \dots + a_{n - 1}}{r_m} = \frac{r_m - r_n}{r_m} = 1 - \frac{r_n}{r_m}.
      \]
      Let $\epsilon \in (0, 1)$. Since $r_n \to 0$, for any integer $N$, we may find large enough $n
      \geq N$, such that
      \[
        \sum_{m = N}^{n} > 1 - \frac{r_n}{r_N} > \epsilon.
      \]
      The result now follows from the Cauchy criterion. 
      
      $\lim_{n \to \infty} \sum_{k=m}^{n} \frac{a_n}{r_n}$
    \end{proof}
    \item Prove that
    \[
      \frac{a_n}{\sqrt{r_n}} < 2\left( \sqrt{r_n} - \sqrt{r_{n+1}} \right)
    \]
    and deduce that $\sum \frac{a_n}{\sqrt{r_n}}$ converges.
    \begin{proof}
      Since $a_n > 0$, 
      \[
        0 < \frac{a_n}{\sqrt{r_n}} = \frac{2(r_n - r_{n + 1})}{2\sqrt{r_n}} < \frac{2(r_n - r_{n + 1})}{\sqrt{r_n} + \sqrt{r_{n + 1}}} = 2\left( \sqrt{r_n} - \sqrt{r_{n+1}} \right).
      \]
      Note that $\sum_{n = 1}^N 2(\sqrt{r_n} - \sqrt{r_{n+1}}) = 2(\sqrt{r_1} - \sqrt{r_{N + 1}})$.
      But then $r_n \to 0$, so
      \[
        0 \leq \sum_{n = 1}^{\infty} \frac{a_n}{\sqrt{r_n}} \leq \sum_{n = 1}^{\infty} 2\left( \sqrt{r_n} - \sqrt{r_{n+1}} \right) = 2\sqrt{r_1}.
      \]
      Therefore, the series converges to $2 \sqrt{r_1}$, by the comparison test.
    \end{proof}
  \end{enumerate}
\end{homeworkProblem}

\newpage

\begin{homeworkProblem}
  Prove that the Cauchy product of two absolutely convergent series converges absolutely.

  \begin{proof}
    Let $\sum a_n$ and $\sum b_n$ be two absolutely convergent series. Let $A_N = \sum^N_{n = 1}
    |a_n|$ and $B_N = \sum^N_{n = 1} |b_n|$, and $C_N = \sum_{n = 1}^{N} |c_n| = \sum_{n = 1}^{N}
    \left|\sum_{k = 1}^{n} a_kb_{n - k}\right|$. Since $|c_n|$ is nonnegative, it suffices to show
    that that $C_n$ is bounded. Hence,
    \begin{align*}
      C_N
      &= \sum_{n = 1}^{N} \left|\sum_{k = 1}^{n} a_kb_{n - k}\right| \\
      &\leq \sum_{n = 1}^{N} \sum_{k = 1}^{n} |a_k||b_{n - k}| \\
      &= \sum_{k = 1}^{N} |a_k| \sum^{N - k}_{j = 1} |b_j| \\
      &= \sum_{k = 1}^{N} |a_k| B_{N - k} \\
      &\leq \sum_{k = 1}^{N} |a_k|B_{N} \\
      &= A_NB_N,
    \end{align*}
    and the result follows.
  \end{proof}
\end{homeworkProblem}

\newpage

\begin{homeworkProblem}
  Associate to each sequence $a = (\alpha_n)$ in which $\alpha_n$ is 0 or 2, the real number
  \[
    \chi(a) = \sum_{n=1}^{\infty} \frac{\alpha_n}{3^n}.
  \]
  Prove that the set of all $\chi(a)$ is precisely the Cantor set described in Theorem 2.44.

  \begin{proof}
    We continue to use the notations $E_1, E_2, \dots$ and $P$ defined in Theorem 2.44. Given some
    $a$, we first show that $\sum_{k=1}^{n} \frac{\alpha_k}{3^k} = \inf I_n$ for some interval $I_n$
    of $E_n$ by induction on $n$. Since $a_1$ is either $0$ or $2$, $\frac{\alpha_1}{3}$ is
    obviously the lower end point of some interval in $E_1$. Suppose $n > 1$. By induction, we know
    $\sum_{k=1}^{n - 1} \frac{\alpha_k}{3^k} = \sup I_{n - 1}$ for some interval $I_{n - 1} \subset
    E_n$. Since $I_{n - 1} \cap E_n$ is a union of 2 intervals, put $I_{n_1}$ to be the lower
    interval of $I_{n - 1} \cap E_n$ and let $I_{n_2}$ be the upper one. Note that $\inf I_{n_1} =
    \inf I_{n - 1}$ and $\sup I_{n_2} = \sup I_{n - 1}$. If $a_n = 0$, then $\sum_{k=1}^{n}
    \frac{\alpha_k}{3^k} = \inf I_{n - 1} = \inf I_{n_1}$ and we are done. Suppose $a_n = 2$. Note
    that the width of $I_{n - 1}$ is $3^{n - 1}$, and the width of $I_{n_2}$ is $3^{-n}$. Since
    $\sup I_{n - 1} = \sup I_{n_2}$, we get $\inf I_{n_2} = \sup I_{n - 1} - 3^{-n} = \inf I_{n - 1}
    + \frac{2}{3} \cdot 3^{-n}$. But then $\sum_{k=1}^{n} \frac{\alpha_k}{3^k} = \inf I_{n - 1} +
    \frac{2}{3} \cdot 3^{-n} = \inf I_{n_2}$, and this completes the induction. Since all $E_n$ are
    closed and $E_1 \supset E_2 \supset \dots$, we have $\sum_{k=1}^{n} \frac{\alpha_k}{3^k} \in
    E_m$, for all positive integer $m \leq n$. Hence, we have $\chi(a) \in E_n$, for all $n$, and
    thus $\chi(a) \in P$. 

    We now show the converse. Let $x \in P$. We construct a sequence $a = (\alpha_n)$ by putting
    $a_n = 0$ if $x$ is in the lower interval of $I_{n - 1} \cap E_n$, where $I_{n - 1} \subset E_{n
    - 1}$ is the interval which contains $x$. Otherwise, if $x$ is in the upper interval of $I_{n -
    1} \cap E_n$, put $a_n = 2$. From the first part, we already know $\chi(a) \in P$. We show that
    $\sum_{k = 1}^n \frac{\alpha_k}{3^k}$ is in the same interval $I_n \subset E_n$ that contains
    $x$ by induction on $n$. The base case is trivial. Suppose $n > 1$. By induction, $\sum_{k =
    1}^{n - 1} \frac{\alpha_k}{3^k} \in I_{n - 1}$. Note that $\sum_{k = 1}^{n}
    \frac{\alpha_k}{3^k}$ will be in either the upper or lower interval of $I_{n - 1} \cap E_{n}$,
    by the first part of the proof. But then by the construction of $\alpha_n$, $\sum_{k = 1}^{n}
    \frac{\alpha_k}{3^k}$ will be in the upper interval if $x$ is in the upper one and vice versa,
    and this completes the induction. It follows that $\chi(a)$ shares the same interval $I_n$ with
    $x$, for all $n$. Fix $\epsilon > 0$. Since $P$ contains no segments, $I_n \subset
    B_{\epsilon}(x)$ for large enough $n$, where $I_n \subset E_n$ is the interval that contains
    $x$. But then $\chi(a) \in I_n$, and thus $|\chi(a) - x| < \epsilon$. The result now follows.
  \end{proof}
\end{homeworkProblem}

\newpage

\begin{homeworkProblem}
  Suppose $(p_n)$ and $(q_n)$ are Cauchy sequences in a metric space $X$. Show that the sequence
  $(d(p_n, q_n))$ converges. \textit{Hint}: For any $m, n$,
  \[
    d(p_n, q_n) \leq d(p_n, p_m) + d(p_m, q_m) + d(q_m, q_n);
  \]
  it follows that
  \[
    |d(p_n, q_n) - d(p_m, q_m)|
  \]
  is small if $m$ and $n$ are large.

  \begin{proof}
    Fix $\epsilon > 0$. Since $(p_n)$ and  $(q_n)$ are Cauchy sequences, there exists integer $N$
    such that $d(p_n, p_m) < \frac{\epsilon}{2}$ and $d(q_n, q_m) < \frac{\epsilon}{2}$, for $m, n
    \geq N$. But then
    \[
      d(p_n, q_n) \leq d(p_n, p_m) + d(p_m, q_m) + d(q_m, q_n) < d(p_m, q_m) + \epsilon.
    \]
    Since the inequality still holds if we swap $m, n$, we get 
    \[
      |d(p_n, q_n) - d(p_m, q_m)| < \epsilon.
    \]
    Hence, $(d(p_n, q_n))$ is also a Cauchy sequence. Since $(d(p_n, q_n))$ is in $\R$, the result
    now follows from Theorem 3.11.
  \end{proof}
\end{homeworkProblem}
\end{document}