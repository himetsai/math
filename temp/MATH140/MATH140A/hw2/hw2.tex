\documentclass{article}

\usepackage{fancyhdr}
\usepackage{extramarks}
\usepackage{amsmath}
\usepackage{amsthm}
\usepackage{amsfonts}
\usepackage{tikz}
\usepackage[plain]{algorithm}
\usepackage{algpseudocode}
\usepackage{enumerate}
\usepackage{amssymb}

\usetikzlibrary{automata,positioning}

%
% Basic Document Settings
%

\topmargin=-0.45in
\evensidemargin=0in
\oddsidemargin=0in
\textwidth=6.5in
\textheight=9.0in
\headsep=0.25in

\linespread{1.1}

\pagestyle{fancy}
\lhead{\hmwkAuthorName}
\chead{\hmwkClass:\ \hmwkTitle}
\rhead{\firstxmark}
\lfoot{\lastxmark}
\cfoot{\thepage}

\renewcommand\headrulewidth{0.4pt}
\renewcommand\footrulewidth{0.4pt}

\setlength\parindent{0pt}
\setlength{\parskip}{5pt}

%
% Create Problem Sections
%

\newcommand{\enterProblemHeader}[1]{
    \nobreak\extramarks{}{Problem \arabic{#1} continued on next page\ldots}\nobreak{}
    \nobreak\extramarks{Problem \arabic{#1} (continued)}{Problem \arabic{#1} continued on next page\ldots}\nobreak{}
}

\newcommand{\exitProblemHeader}[1]{
    \nobreak\extramarks{Problem \arabic{#1} (continued)}{Problem \arabic{#1} continued on next page\ldots}\nobreak{}
    \stepcounter{#1}
    \nobreak\extramarks{Problem \arabic{#1}}{}\nobreak{}
}

\setcounter{secnumdepth}{0}
\newcounter{partCounter}
\newcounter{homeworkProblemCounter}
\setcounter{homeworkProblemCounter}{1}
\nobreak\extramarks{Problem \arabic{homeworkProblemCounter}}{}\nobreak{}

%
% Homework Problem Environment
%
% This environment takes an optional argument. When given, it will adjust the
% problem counter. This is useful for when the problems given for your
% assignment aren't sequential. See the last 3 problems of this template for an
% example.
%
\newenvironment{homeworkProblem}[1][-1]{
    \ifnum#1>0
        \setcounter{homeworkProblemCounter}{#1}
    \fi
    \section{Problem \arabic{homeworkProblemCounter}}
    \setcounter{partCounter}{1}
    \enterProblemHeader{homeworkProblemCounter}
}{
    \exitProblemHeader{homeworkProblemCounter}
}

%
% Homework Details
%   - Title
%   - Due date
%   - Class
%   - Section/Time
%   - Instructor
%   - Author
%

\newcommand{\hmwkTitle}{Homework\ \#2}
\newcommand{\hmwkDueDate}{Jan 26, 2023}
\newcommand{\hmwkClass}{MATH 140A}
\newcommand{\hmwkClassInstructor}{Professor Seward}
\newcommand{\hmwkAuthorName}{\textbf{Ray Tsai}}
\newcommand{\hmwkPID}{A16848188}

%
% Title Page
%

\title{
    \vspace{2in}
    \textmd{\textbf{\hmwkClass:\ \hmwkTitle}}\\
    \normalsize\vspace{0.1in}\small{Due\ on\ \hmwkDueDate\ at 23:59pm}\\
    \vspace{0.1in}\large{\textit{\hmwkClassInstructor}} \\
    \vspace{3in}
}

\author{
  \hmwkAuthorName \\
  \vspace{0.1in}\small\hmwkPID
}
\date{}

\renewcommand{\part}[1]{\textbf{\large Part \Alph{partCounter}}\stepcounter{partCounter}\\}

%
% Various Helper Commands
%

% Useful for algorithms
\newcommand{\alg}[1]{\textsc{\bfseries \footnotesize #1}}

% For derivatives
\newcommand{\deriv}[1]{\frac{\mathrm{d}}{\mathrm{d}x} (#1)}

% For partial derivatives
\newcommand{\pderiv}[2]{\frac{\partial}{\partial #1} (#2)}

% Integral dx
\newcommand{\dx}{\mathrm{d}x}

% Probability commands: Expectation, Variance, Covariance, Bias
\newcommand{\Var}{\mathrm{Var}}
\newcommand{\Cov}{\mathrm{Cov}}
\newcommand{\Bias}{\mathrm{Bias}}
\newcommand*{\Z}{\mathbb{Z}}
\newcommand*{\Q}{\mathbb{Q}}
\newcommand*{\R}{\mathbb{R}}
\newcommand*{\C}{\mathbb{C}}
\newcommand*{\N}{\mathbb{N}}
\newcommand*{\prob}{\mathds{P}}
\newcommand*{\E}{\mathds{E}}

\begin{document}

\maketitle

\pagebreak

\begin{homeworkProblem}
  Prove that no order can be defined in the complex field that turns it into an ordered field.

  \begin{proof}
    Consider $i^2$. Since $i^2$ is a square of nonzero element, $-1 = i^2 > 0$, contradiction.
    Hence, no order can be defined in the complex field.
  \end{proof}
\end{homeworkProblem}

\newpage

\begin{homeworkProblem}
  Suppose $z = a + bi, w = c + di$. Define $z < w$ if $a < c$, and also if $a = c$ but $b < d$.
  Prove that this turns the set of complex numbers into an ordered set. Does this ordered set have
  the least-upper-bound property?

  \begin{proof}
    If $a > c$, then $z > w$. Suppose that $a = c$. If $b > d$, then $z > w$. If $b = d$, then $z =
    w$. Thus, the order follows the law of trichotomy. 
    
    We now show that the order is transitive. Let $x = g + hi$. Suppose that $z > w$ and $w > x$.
    Since $z > w$, either $a > c$, or $a = c$ and $b > d$. Similarly, since $w > x$, either $c > g$,
    or $c = g$ and $d > h$. We may assume that $a = c = g$, otherwise $a > g$ and we are done. Then,
    $b > d > h$, so $z > x$, so the order is indeed transitive.

    Note that this ordered set has the least-upper-bound property. Let $B \subset \C$ be non-empty.
    Since $\R$ has the least-upper-bound property, we know there exists $\alpha = \sup \{k \in \R
    \mid k + mi \in B\}$ and $\beta = \sup \{m \in \R \mid \alpha + mi \in B\}$. We show that
    $\alpha + \beta i = \sup B$. Let $a + bi \in \C$. We know $\alpha \geq a$. We may assume that
    $\alpha = a$. Then, since $\beta \geq b$, we know $\alpha + \beta i \geq a + bi$, so $\alpha +
    \beta i$ is the upper bound of $B$. Let $w = c + di$, such that $w < \alpha + \beta i$. If $c <
    \alpha$, then we may find $p \in \{k \in \R \mid k + mi \in B\}$ such that $p > c$, and thus
    ther exists $p + qi \in B$ such that $p + qi > w$. If $c = \alpha$ and $d < \beta$, then we may
    find $t \in \{m \in \R \mid \alpha + mi \in B\}$ such that $t > d$, and thus there exists $s +
    ti \in B$, such that $s + ti > y$. Therefore, $\alpha + \beta i = \sup B$, so the ordered set
    does have the least-upper-bound property.
  \end{proof}
\end{homeworkProblem}

\newpage

\begin{homeworkProblem}
  Suppose $z = a + bi$, $w = u + iv$, and
  \[
  a = \left( \frac{|w| + u}{2} \right)^{1/2}, \quad b = \left( \frac{|w| - u}{2} \right)^{1/2}.
  \]
  Prove that $z^2 = w$ if $v \geq 0$ and that $(\bar{z})^2 = w$ if $v \leq 0$. Conclude that every
  complex number (with one exception!) has two complex square roots.

  \begin{proof}
    If $v \geq 0$, then 
    \begin{align*}
      z^2
      &= (a + bi)(a + bi) \\
      &= a^2 - b^2 + 2abi \\
      &= \frac{|w| + u}{2} - \frac{|w| - u}{2} + 2\left( \frac{|w| + u}{2} \right)^{1/2}\left( \frac{|w| - u}{2} \right)^{1/2} i \\
      &= u + 2\left( \frac{|w|^2 - u^2}{4} \right)^{1/2}i \\
      &= u + i|v| = u + iv = w.
    \end{align*}
    If $v \leq 0$, then
    \begin{align*}
      (\bar{z})^2
      &= (a - bi)(a - bi) \\
      &= a^2 - b^2 - 2abi \\
      &= \frac{|w| + u}{2} - \frac{|w| - u}{2} - 2\left( \frac{|w| + u}{2} \right)^{1/2}\left( \frac{|w| - u}{2} \right)^{1/2} i \\
      &= u - 2\left( \frac{|w|^2 - u^2}{4} \right)^{1/2}i \\
      &= u - i|v| = u + iv = w.
    \end{align*}
    Suppose that $w = 0$. Then, $a, b = 0$, so $z = \bar{z} = 0$, which means that $w = 0$ only has
    one complex root. However, when $w \neq 0$, $w$ has $z$ and $\bar{z}$ as its complex roots.
    Therefore, every nonzero complex number has two complex roots.
  \end{proof}
\end{homeworkProblem}

\newpage

\begin{homeworkProblem}
  If x, y are complex, prove that
  \[
    ||x| - |y|| \leq |x - y|.
  \]
  
  \begin{proof}
    On the LHS
    \begin{align*}
      (|x| - |y|)^2
      &= |x|^2 + |y|^2 - 2|x||y| \\
      &= |x|^2 + |y|^2 - 2|x||\bar{y}| \\
      &= |x|^2 + |y|^2 - 2|x\bar{y}|.
    \end{align*}
    On the RHS,
    \begin{align*}
      |x - y|^2 
      &= (x - y)(\bar{x} - \bar{y}) \\
      &= x\bar{x} + y\bar{y} - y\bar{x} - x\bar{y} \\
      &= |x|^2 + |y|^2 - (y\bar{x} + x\bar{y}) \\
      &= |x|^2 + |y|^2 - (x\bar{y} + \overline{x\bar{y}}) \\
      &= |x|^2 + |y|^2 - 2\text{Re }x\bar{y}.
    \end{align*}
    Since $|x\bar{y}| \geq \text{Re }x\bar{y}$, we have $||x| - |y||^2 \leq |x - y|^2$. Since $||x|
    - |y||, |x - y| \geq 0$, the results follows.
  \end{proof}
\end{homeworkProblem}

\newpage

\begin{homeworkProblem}
  If $z$ is a complex number such that $|z| = 1$, that is, such that $z\bar{z} = 1$, compute
  \[
  |1 + z|^2 + |1 - z|^2.
  \]

  \begin{proof}
    \begin{align*}
      |1 + z|^2 + |1 - z|^2
      &= (1 + z)(1 + \bar{z}) + (1 - z)(1 - \bar{z}) \\
      &= z\bar{z} + 1 + z + \bar{z} + z\bar{z} + 1 - z - \bar{z} \\
      &= 2|z|^2 + 2 \\
      &= 4.
    \end{align*}
  \end{proof}
\end{homeworkProblem}

\newpage

\begin{homeworkProblem}
  Under what conditions does equality hold in the Schwarz inequality?

  \begin{proof}
    From the proof of Theorem 1.35, the equality holds when $C^2 = AB$, which is equivalent to
    $|Ba_j - Cb_j| = 0$, for all $j$. Hence, the equality holds when $a_j\sum_i^n |b_i|^2 =
    b_j\sum_i^n a_i\bar{b_i}$, for all $j$.
  \end{proof}
\end{homeworkProblem}

\newpage

\begin{homeworkProblem}
  Prove that
  \[
    |x + y|^2 + |x - y|^2 = 2|x|^2 + 2|y|^2
  \]
  if $x \in \R^k$ and $y \in \R^k$. Interpret this geometrically, as a statement about
  parallelograms.

  \begin{proof}
    \begin{align*}
      |x + y|^2 + |x - y|^2
      &= x \cdot x + 2x \cdot y + y \cdot y + x \cdot x - 2x \cdot y + y \cdot y \\
      &= 2(x \cdot x + y \cdot y) \\
      &= 2|x|^2 + 2|y|^2.
    \end{align*}
    This implies that in a parallelogram, the square sum of the length of the diagonals equals the
    the square sum of the length each side.
  \end{proof}
\end{homeworkProblem}
\end{document}