\documentclass{article}
\usepackage{amsfonts, amsmath, amssymb, amsthm} % Math notations imported
\usepackage{enumitem}

\newtheorem{thm}{Theorem}
\newtheorem{prop}[thm]{Proposition}
\newtheorem{cor}[thm]{Corollary}

% title information
\title{Math 109 HW 5}
\author{Ray Tsai}
\date{10/31/2022}

% main content
\begin{document} 

% placing title information; comment out if using fancyhdr
\maketitle 

\begin{enumerate}
% Q1
\item 
\begin{prop}
    If $A \subseteq B$ then $B^{c} \subseteq A^{c}$.
\end{prop}
\begin{proof}
Suppose that $A \subseteq B$. We will prove by the contrapositive. If $B^{c} = \emptyset$, then $B^{c} \subseteq A^{c}$, as $\emptyset$ is a subset of any set. Suppose that $B$ is not empty. Let $x \notin B$. We will show that $x \notin A$. \\

Since $A \subseteq B$, we have $(\forall y \in U)[(y \in A) \rightarrow (y \in B)]$. The contrapositive of this statement is that $(\forall y \in U)[(y \notin B) \rightarrow (y \notin A)]$. Thus, since $x \notin B$, we have $x \notin A$. \\

Therefore $B^{c} \subseteq A^{c}$ by definition.
\end{proof}

% Q2
\item 
\begin{prop}
    If $A, B$ are disjoint and $C \subseteq B$, then $A, C$ are disjoint.
\end{prop}
\begin{proof}
Suppose that $(\forall y \in A)(y \notin B)$ and $C \subseteq B$. Let $x \in A$. We will show that $x \notin C$. \\

Since $A, B$ are disjoint, $x \notin B$ because $x \in A$. Since $C \subseteq B$, we have $(\forall z \in C)(z \in B)$, which is equivalent to $(\forall z \notin B)(z \notin C)$. Thus, since $x \notin B$, we have $x \notin C$. \\

Therefore, $A, C$ are disjoint by definition.
\end{proof}

% Q3
\item \begin{enumerate}
    
    \item 
    \begin{prop}
        If $A \subseteq B$ and $A \neq \emptyset$, then $A, B$ are not disjoint.
    \end{prop}
    \begin{proof}
        Let $x \in A$. We will show that $x \in B$. \\
        
        Since $A \subseteq B$, $x \in B$ because $x \in A$. Therefore, $A, B$ are not disjoint by definition.
    \end{proof}
        
    \item 
    \begin{prop}
        If $A \subseteq B$, then $A, B$ can be disjoint.
    \end{prop}
    \begin{proof}
        Consider $A = \emptyset$. Since an empty set is a subset of all sets, we have $A \subseteq B$. Since $A = \emptyset$, $A, B$ are disjoint, as they do not share any common elements. 
    \end{proof}
\end{enumerate}

% Q4
\item 
\begin{prop}
    $A, B, C \subseteq U$ and are not empty. If $(A \cap B)^{c} \subseteq C$, then $A \subseteq B \cup C$.
\end{prop}
\begin{proof}
    Suppose that $(A \cap B)^{c} \subseteq C$. Let $x \in A$. We will show that $x \in B \cup C$. 
    
    We can separate the situation into two cases, $x \in B$ and $x \notin B$. 
    
    If $x \in B$, then $x \in B \cup C$.
    
    If $x \notin B$, then $x \notin (A \cap B)$ beacause $x \notin A \vee x \notin B$, which means that $x \in (A \cap B)^{c}$. Since $(A \cap B)^{c} \subseteq C$, we have $x \in C$. Thus, $x \in B \cup C$.
    
    Therefore, $x \in B \cup C$.
\end{proof}

% Q5
\item \begin{enumerate}
    \item 
    \begin{prop}
      $f$ is not injective.
    \end{prop}
    \begin{proof}
    Consider $f(1)$ and $f(3)$, $1, 3 \in \mathbb{Z}$. $f(1) = f(3) = 1$, but $1 \neq 3$. Therefore, $f$ is not injective by definition.
    \end{proof}
    
    \begin{prop}
      $f$ is surjective.
    \end{prop}
    \begin{proof}
    Let $b \in \{0, 1\}$. We will prove that there exist $a \in \mathbb{Z}$ such that $f(a) = b$. \\
    
    We can separate it into 2 cases, $b = 0$ and $b = 1$. 
    
    If $b = 0$, There exist $a = 0$ such that $f(a) = f(0) = 0 = b$. 
    
    If $b = 1$, There exist $a = 1$ such that $f(a) = f(1) = 1 = b$. \\
    
    Therefore, we have exhausted all possibilities of $b$ and shown that $f$ is surjective by definition.
    \end{proof}
    
    \item 
    \begin{prop}
      $g$ is injective.
    \end{prop}
    \begin{proof}
    We will prove by the contrapositive of the definition of an injective function. Let $a_1, a_2 \in \{0, 1\}$. We will show that if $a_1 \neq a_2$, then $g(a_1) \neq g(a_2)$. \\
    
    Let $a_1 = 0, a_2 = 1$. $a_1 = 0 \neq 1 = a_2$ and $g(a_1) = 1 \neq -1 = g(a_2)$. \\
    
    Therefore, since there are no elements other than $0,1$ in $\{0, 1\}$, we have exhausted all the possibilities and proved that $g$ is injective.
    \end{proof}
    
    \begin{prop}
      $g$ is surjective.
    \end{prop}
    \begin{proof}
    Let $b \in \{1, -1\}$. We will prove that there exist $a \in \{0, 1\}$ such that $g(a) = b$. \\
    
    We can separate it into 2 cases, $b = 1$ and $b = -1$. 
    
    If $b = 1$, There exist $a = 0$ such that $f(a) = f(0) = 1 = b$. 
    
    If $b = -1$, There exist $a = 1$ such that $f(a) = f(1) = -1 = b$. \\
    
    Therefore, $g$ is surjective by definition.
    \end{proof}
    
    \item 
    \begin{prop}
      $h$ is injective.
    \end{prop}
    \begin{proof}
    Let $(a_1, b_1), (a_2, b_2) \in \mathbb{R}^2$, $a_1b_1 = a_2b_2 = 1$. We will show that if $h(a_1, b_1) = h(a_2, b_2)$, then $a_1 = a_2$ and $b_1 = b_2$. \\
    
    For all $(x, y) \in \mathbb{R}^2$ and $xy = 1$, $x,y \neq 0$ because if $x = 0$ or $y = 0$ then $xy = 0 \neq 1$. \\
    
    Suppose that $h(a_1, b_1) = h(a_2, b_2)$. Since $a_1b_1 = a_2b_2 = 1$ and $a_1, a_2 \neq 0$, we can assume that $b_1 = \frac{1}{a_1}, b_2 = \frac{1}{a_2}$. Since $h(a_1,b_1) = h(a_2,b_2)$, we know that $a_1 = a_2$. Since $a_1 = a_2$, we have $b_1 = \frac{1}{a_1} = \frac{1}{a_2} = b_2$. \\
    
    Therefore, $h$ is injective by definition.
    \end{proof}
    
    \begin{prop}
      $h$ is surjective.
    \end{prop}
    \begin{proof}
    Let $c \in \mathbb{R}$. We will prove that there exist $(a, b) \in \mathbb{R}^2$ such that $h(a, b) = c$. \\
    
    Let $c = a$. Since $h(a, b) = a$, we know that $h(a, b) = c$ \\
    
    Therefore, $h$ is surjective by definition.
    \end{proof}
    
    \item 
    \begin{prop}
      $k$ is injective.
    \end{prop}
    \begin{proof}
    First, let $a$ be some non-negative even number $2m$, $b$ be some non-negative odd number $2n + 1$, $a,b,m,n \in \mathbb{Z}$, $a, b, m, n \geq 0$. We will use contradiction to prove that if $a,b$ are not both even or both odd, then $k(a) \neq k(b)$. Suppose for the sake of contradiction that $k(a) = k(b)$ and 
    \begin{gather}
        k(a) = \frac{2m}{2} = m \\
        k(b) = -\frac{(2n+1)+1}{2} = -n - 1 \\
    \end{gather}
    Since $k(a) = k(b)$, we know that
    \begin{gather}
        m = -n - 1 \\
        m + n = -1
    \end{gather}
    
    This contradicts our original assumption that $m \geq 0, n \geq 0$. Therefore, if $k(a) = k(b)$, then $a, b$ are both even or both odd. \\
    
    Let $a_1, a_2 \in \mathbb{Z}$, $a_1, a_2 \geq 0$. Now we will show that if $k(a_1) = k(a_2)$, then $a_1 = a_2$. \\
    
    Suppose that $k(a_1) = k(a_2)$. We can separate it into 2 cases, $a_1, a_2$ are both even, $a_1, a_2$ are both odd. \\
    
    If $a_1, a_2$ are both even, since $k(a_1) = k(a_2)$, we know
    \begin{gather}
        \frac{a_1}{2} = \frac{a_2}{2} \\
        a_1 = a_2
    \end{gather}
    If $a_1, a_2$ are both odd, since $k(a_1) = k(a_2)$, we know
    \begin{gather}
        -\frac{a_1 + 1}{2} = -\frac{a_2 + 1}{2} \\
        a_1 + 1 = a_2 + 1 \\
        a_1 = a_2
    \end{gather}
    Therefore, $k$ is injective by definition.
    \end{proof}
    
    \begin{prop}
      $k$ is surjective.
    \end{prop}
    \begin{proof}
    Let $b \in \mathbb{Z}$. We will prove that there exist some non-negative integer $a$ such that $k(a) = b$. We can separate it into 2 cases, $b \geq 0$ and $b < 0$. \\
    
    If $b \geq 0$, let $b = \frac{a}{2}$, then $k(a) = \frac{a}{2} = b$. \\
    
    If $b < 0$, let $b = -\frac{a + 1}{2}$, then $k(a) = -\frac{a + 1}{2} = b$. \\
    
    Therefore, $k$ is surjective by definition.
    \end{proof}
\end{enumerate}

% Q6
\item \begin{enumerate}
    \item \begin{prop}
    $\alpha$ is not a well defined function.
    \end{prop}
    \begin{proof}
    Let $x \in \mathbb{R}$. We will show that $\alpha(x) \notin \mathbb{Z}$. Consider the case $x = \frac{1}{2}$. $\alpha(x) = \frac{1}{2}$, which is not an integer. Therefore, $\alpha$ is not a well defined function.
    \end{proof}
    
    \item \begin{prop}
    $\beta$ is a not well defined function.
    \end{prop}
    \begin{proof}
    Let $x \in \mathbb{Z}$. We will show that there exists $y_1, y_2 \in \{-1, 0, 1\}$, such that $\beta(x) = y_1$, $\beta(x) = y_2$, and $y_1 \neq y_2$. Consider the case $x = 2, y_1 = 1, y_2 = -1$. $\beta(x) = 1 = y_1$, and $\beta(x) = -1 = y_2$. This shows that there exists $x \in \mathbb{Z}$ such that $\beta(x)$ has multiple possible values.
    Therefore, $\beta$ is not a well defined function.
    \end{proof}
    
    \item \begin{prop}
    $|-|$ is a well defined function.
    \end{prop}
    \begin{proof}
    For existence: let $x \in \mathbb{R}$. We will show that there exists $y \in \mathbb{R}_{\geq 0}$ such that $|x| = y$. We can separate it into 2 cases, $x \geq 0$ and $x \leq 0$. \\
    
    If $x \geq 0$, let $y = x$. Since $y = x \geq 0$, we have $y \in \mathbb{R}_{\geq 0}$. We then have $|x| = x = y$. \\
    If $x \leq 0$, let $y = -x$. Since $x \leq 0$, we have $-x = y \geq 0$. Thus, $y \in \mathbb{R}_{\geq 0}$. We then have $|x| = -x = y$. \\
    
    Therefore, for each $x \in \mathbb{R}$, there exists $y \in \mathbb{R}_{\geq 0}$ such that $|x| = y$. \\
    
    For uniqueness: let $a \in \mathbb{R}, b_1 = |a|, b_2 = |a|, b_1, b_2 \in \mathbb{R}_{\geq 0}$. We will show that $b_1 = b_2$. We can separate it into 2 cases, $a \geq 0$ and $a \leq 0$. \\
    
    If $a \geq 0$, $b_1 = |a| = a$ and $b_2 = |a| = a$. Thus, $y_1 = y_2$. \\
    If $a \leq 0$, $b_1 = |a| = -a$ and $b_2 = |a| = -a$. Thus, $b_1 = b_2$. \\
    
    Therefore, for each $a \in \mathbb{R}$, $|a|$ is unique. \\
    
    Therefore, $|-|$ is a well defined function.
    \end{proof}
    
    \item \begin{prop}
    $\gamma$ is a not well defined function.
    \end{prop}
    \begin{proof}
    Let $x \in \mathbb{R}$. Consider the case $x = 0$. We will show that $\gamma(x) \notin \mathbb{R}$. $\gamma(x) = \frac{0}{|0|} = \frac{0}{0}$, which is undefined. Therfore, $\gamma$ is not a well defined function.
    \end{proof}
\end{enumerate}


% Q7
\item \begin{enumerate}
    \item \begin{prop}
        If $g \circ f$ is injective, then $f$ is injective.
    \end{prop}
    \begin{proof}
        We will prove by contradiction. Suppose for the sake of contradiction that $c = f(a) = f(b)$, $a \neq b$, $a, b \in A$, $c \in B$. \\ 
        
        $g(f(a)) = g(f(b)) = g(c)$. Since $g \circ f$ is injective, we know that $g(f(a)) = g(f(b))$ implies $a = b$. However, it contradicts our original assumption that $a \neq b$. \\ 
        
        Therefore, $f$ is injective.
    \end{proof}
    
    \item \begin{prop}
        If $g \circ f$ is injective, then $g$ does not have to be injective.
    \end{prop}
    \begin{proof}
        % We will prove by contradiction. Suppose for the sake of contradiction that $m = f(a)$, $n = f(b)$, $m \neq n$, $a, b \in A$, $m, n \in B$, such that $g(m) = g(n)$. \\ 
        
        % We know that $g(m) = g(f(a))$ and $g(n) = g(f(b))$. Since $g \circ f$ is injective and $g(f(a)) = g(f(b))$, we know that $a = b$. Since $a = b$, we know that $m = f(a) = f(b) = n$. However, this contradicts our assumption that $m \neq n$. \\ 
        
        % Therefore, $g$ is injective.
        
        Consider the case  
        \begin{gather}
            f: \mathbb{R}_{\geq0} \rightarrow \mathbb{R} \\
            f(x) = x \\
            g: \mathbb{R} \rightarrow \mathbb{R}_{\geq0} \\
            g(x) = x^2
        \end{gather}
        Combining $f$ and $g$, we get
        \begin{gather}
            g \circ f: \mathbb{R}_{\geq0} \rightarrow \mathbb{R}_{\geq0} \\
            g \circ f(x) = x^2
        \end{gather}
        We will first show that $g \circ f$ is injective. \\
        Let $a, b$ be some non-negative real numbers. Suppose that  $g \circ f(a) = g \circ f(b)$. We will show that $a = b$.  \\
        
        Since $g \circ f(a) = a^2$, $g \circ f(b) = b^2$,
        \begin{gather}
            a, b \geq 0 \\
            a^2 = b^2 \\
            a = b 
        \end{gather}
        Therefore, $g \circ f(b)$ is injective. 
        However, $g$ is not injective, as both $g(-1)$ and $g(1)$ equals to 1, and $-1, 1 \in \mathbb{R}$. \\
        
        Therefore, $g$ does not have to be injective.
    \end{proof}
    
    \item \begin{prop}
        If $g \circ f$ is surjective, then $f$ does not have to be surjective.
    \end{prop}
    \begin{proof}
        Consider the case  
        \begin{gather}
            f: \{0\} \rightarrow \mathbb{R} \\
            f(x) = 0 \\
            g: \mathbb{R} \rightarrow \{0\} \\
            g(x) = 0
        \end{gather}
        Combining $f$ and $g$, we get
        \begin{gather}
            g \circ f: \{0\} \rightarrow \{0\} \\
            g \circ f(x) = 0
        \end{gather}
        $g \circ f$ in surjective because $g \circ f(0) = 0$ and there are no elements other than 0 in $\{0\}$. However, $f$ is not surjective, since there are no $a \in \{0\}$ such that $f(a) = 1$.
        
        Therefore, $f$ does not have to be surjective.
    \end{proof}
    
    \item \begin{prop}
        If $g \circ f$ is surjective, then $g$ is surjective.
    \end{prop}
    \begin{proof}
        We will prove by contradiction. Suppose for the sake of contradiction that there exists some $z \in C$ such that for all $k \in B$, $g(k) \neq z$. \\
        
        Since $g \circ f$ is surjective, we know that there exist some $x \in A$ such that $g \circ f(x) = z$. Let $y = f(x)$, $y \in B$. We then have
        \begin{align}
            g \circ f(x) &= g(f(x)) \\
                         &= g(y) \\
                         &= z
        \end{align}
        This shows that there exists some $y \in B$ such that $g(y) = z$. However, this contradicts our assumption that for all $k \in B$, $g(k) \neq z$. \\
        
        Therefore, $g$ is surjective.
    \end{proof}
\end{enumerate}

% Q8
\item 
\begin{prop}
  For all $n \in \mathbb{Z}^{+}$, we have 
  \begin{gather}
      1^2 + 2^2 + 3^2 + \ldots + n^2 = \frac{n(n+1)(2n+1)}{6}.
  \end{gather}
\end{prop}
\begin{proof}
    We proceed by induction on $n$.\\
    
    If $n = 1$, then $1^2 = 1$, and $\frac{1(1+1)(2+1)}{6} = \frac{6}{6} = 1$. Thus, the equation is correct when $n = 1$. \\
    If $n = 2$, then $1^2 + 2^2 = 5$, and $\frac{2(2+1)(2\cdot2+1)}{6} = \frac{2(3)(5)}{6} = 5$. Thus, the equation is correct when $n = 2$. \\
    
    Suppose $1^2 + 2^2 + 3^2 + \ldots + n^2 = \frac{n(n+1)(2n+1)}{6}$ for some $n \in \mathbb{Z}^{+}$. \\
    We then have 
    \begin{align}
        \frac{(n+1)((n+1)+1)(2(n+1)+1)}{6} &= \frac{(n+1)(n+2)(2n+3)}{6} \\
                                           &= \frac{n(n+1)(2n+3)}{6} + \frac{2(n+1)(2n+3)}{6} \\
                                           &= \frac{n(n+1)(2n+1)}{6} + \frac{2n(n+1)}{6} + \frac{2(n+1)(2n+3)}{6} \\
                                           &= \frac{n(n+1)(2n+1)}{6} + \frac{6n^2+12n+6}{6} \\
                                           &= \frac{n(n+1)(2n+1)}{6} + n^2 + 2n + 1 \\
                                           &= \frac{n(n+1)(2n+1)}{6} + (n+1)^2 \\
                                           &= 1^2 + 2^2 + 3^2 + \ldots + n^2 + (n+1)^2 
    \end{align}
    Thus, the equation also work for $n + 1$ when $n$ works. \\
    Therefore, for all $n \in \mathbb{Z}^{+}$, 
    \begin{gather}
      1^2 + 2^2 + 3^2 + \ldots + n^2 = \frac{n(n+1)(2n+1)}{6}.
  \end{gather}
\end{proof}
    
% Q9
\item 
\begin{prop}
    Define a sequence $\{a_n\}$ by $a_1 = 1$, $a_2 = 3$, and $a_{n+2} = a_{n+1} + a_n$. For all $n \geq 1$, we have 
    \begin{gather}
        a_n = \left(\frac{1 + \sqrt{5}}{2}\right)^n + \left(\frac{1 - \sqrt{5}}{2}\right)^n
    \end{gather}
\end{prop}
\begin{proof}
    We proceed by induction on $n$.\\
    
    If $n = 1$, then 
    \begin{align}
        1 &= a_1 \\
          &= \left(\frac{1 + \sqrt{5}}{2}\right) + \left(\frac{1 - \sqrt{5}}{2}\right) \\
          &= \frac{1}{2} + \frac{1}{2} \\
          &= 1.
    \end{align}
    Thus, the equation is correct when $n = 1$. \\
    If $n = 2$, then 
    \begin{align}
        3 &= a_2 \\ 
          &= \left(\frac{1 + \sqrt{5}}{2}\right)^2 + \left(\frac{1 - \sqrt{5}}{2}\right)^2 \\
          &= \left(\frac{6 + 2\sqrt{5}}{4}\right) + \left(\frac{6 - 2\sqrt{5}}{4}\right) \\
          &= \frac{6}{4} + \frac{6}{4} \\
          &= 3.
    \end{align}
    Thus, the equation is correct when $n = 2$. \\
    Suppose that for some $n \geq 1$, we have
    \begin{align}
        a_n &= \left(\frac{1 + \sqrt{5}}{2}\right)^n + \left(\frac{1 - \sqrt{5}}{2}\right)^n \\
        a_{n+1} &= \left(\frac{1 + \sqrt{5}}{2}\right)^{n + 1} + \left(\frac{1 - \sqrt{5}}{2}\right)^{n + 1}
    \end{align}
    We then have
    \begin{align}
        a_{n+2} &= \left(\frac{1 + \sqrt{5}}{2}\right)^{n + 2} + \left(\frac{1 - \sqrt{5}}{2}\right)^{n + 2} \\
                &= \left(\frac{1 + \sqrt{5}}{2}\right)^2\left(\frac{1 + \sqrt{5}}{2}\right)^n + \left(\frac{1 - \sqrt{5}}{2}\right)^2\left(\frac{1 - \sqrt{5}}{2}\right)^n \\
                &= \left(\frac{3 + \sqrt{5}}{2}\right)\left(\frac{1 + \sqrt{5}}{2}\right)^n + \left(\frac{3 - \sqrt{5}}{2}\right)\left(\frac{1 - \sqrt{5}}{2}\right)^n \\
                &= \left(1 + \frac{1 + \sqrt{5}}{2}\right)\left(\frac{1 + \sqrt{5}}{2}\right)^n + \left(1 + \frac{1 - \sqrt{5}}{2}\right)\left(\frac{1 - \sqrt{5}}{2}\right)^n \\
                &= \left(\frac{1 + \sqrt{5}}{2}\right)^n + \left(\frac{1 - \sqrt{5}}{2}\right)^n + \left(\frac{1 + \sqrt{5}}{2}\right)\left(\frac{1 + \sqrt{5}}{2}\right)^n + \left(\frac{1 - \sqrt{5}}{2}\right)\left(\frac{1 - \sqrt{5}}{2}\right)^n \\
                &= \left(\frac{1 + \sqrt{5}}{2}\right)^n + \left(\frac{1 - \sqrt{5}}{2}\right)^n + \left(\frac{1 + \sqrt{5}}{2}\right)^{n + 1} + \left(\frac{1 - \sqrt{5}}{2}\right)^{n + 1} \\
                &= a_n + a_{n+1}
    \end{align}
    Thus, the equation is correct for $n+2$ if the equation is correct for $n+1$ and $n$. Since the equation is correct when $n = 1$ and $n = 2$, the equation is correct for all $n \geq 1$.
    
    Therefore, for all $n \geq 1$,
    \begin{gather}
        a_n = \left(\frac{1 + \sqrt{5}}{2}\right)^n + \left(\frac{1 - \sqrt{5}}{2}\right)^n
    \end{gather}
\end{proof}

\end{enumerate}
\end{document}