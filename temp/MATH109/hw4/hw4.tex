\documentclass{article}
\usepackage{amsfonts, amsmath, amssymb, amsthm} % Math notations imported
\usepackage{enumitem}

% If you want to use another style of headers, uncomment (hotkey: ctrl + /) these 7 lines, and comment out "\maketitle" below
% \usepackage{fancyhdr} % Import package
% \pagestyle{fancy} % using the "fancy" pagestyle
% \fancyhf{} % clear out original headers and footers
% \lhead{Math 109 HW 1} % left header, rest is self-explanatory
% \rhead{(your name)}
% \lfoot{(date when hw is due)}
% \rfoot{Page \thepage}

% Some basic theorem environments set up
\newtheorem{thm}{Theorem}
\newtheorem{prop}[thm]{Proposition}
\newtheorem{cor}[thm]{Corollary}

% title information
\title{Math 109 HW 4}
\author{Ray Tsai}
\date{10/24/2022}

% main content
\begin{document} 

% placing title information; comment out if using fancyhdr
\maketitle 

\begin{enumerate}
% Q1
\item 
\begin{prop}
    1 is not even.
\end{prop}
\begin{proof}
We will show that $1$ is not even by contradiction. For the sake of contradiction, by HW4 fact 2, assume $1$ is an even integer $2k$ for some integer $k$. By HW4 fact 1, we know that $1$ is the smallest positive integer, so $k > 0$ is equivalent to $k \geq 1$ if $k$ is an integer. Thus, we can split the situation into two cases, $k \leq 0$ and $k \geq 1$.  \\\\
If $k \leq 0$, then $2k \leq 0$. This suggests that there does not exist integer $k \leq 0$ such that $2k = 1$. \\\\
If $k \geq 1$, then $2k \geq 2$. This suggests that there does not exist integer $k \geq 1$ such that $2k = 1$. \\\\
Hence, it is shown that there does not exist integer $k$ such that $2k = 1$, which contradicts our assumption. \\\\
Therefore, $1$ is not even.
\end{proof}

% Q2
\item 
\begin{prop}
If $n$ is an odd integer, then $n$ is not even.
\end{prop}
\begin{proof}
We will show that if $n$ is an odd integer, then $n$ is not even by contradiction. For the sake of contradiction, by HW4 fact 2 and 3, assume that $n$ is both an odd integer $2k + 1$ and even integer $2l$, for some integers $k, l$. 
\begin{gather}
    n = 2k + 1 = 2l \\
    2k + 1 + (-2k) = 2l + (-2k) \\ 
    1 = 2(l - k)
\end{gather}
Let $l - k$ be some integer $m$.
\begin{gather}
    1 = 2(l - k) = 2m
\end{gather}
By HW4 fact 2, $1$ is even. However, it contradicts the fact that $1$ is not even, which we proved in HW4 Q1. \\
Therefore, if $n$ is an odd integer, then $n$ is not even. \\
In addition, "if n is an even integer, then n is not odd" is also true because it is the contrapositive of the proposition we just proved.
\end{proof}

% Q3
\item \begin{prop}
If 5n is odd, then n is odd.
\end{prop}
\begin{proof}
We will prove by using the contrapositive. \\\\
By HW4 fact 2, let $n$ be some even integer $2k$ for some integer $k$. We will show that if $n$ is not odd, then $5n$ is not odd, which means that if $n$ is even, then $5n$ is even by HW4 Q2. 
\begin{align}
    5n &= 5(2k) \\
    &= 2(5k)
\end{align}
Let $5k$ be some integer $l$.
\begin{align}
    2(5k) &= 2l
\end{align}
Hence, if $n$ is even, then $5n$ is even if $n$ is even by HW4 fact 4.
Therefore, if 5n is odd, then n is odd.
\end{proof}

% Q4
\item \begin{prop}
     if a, b are positive real numbers with $a \neq b$, then
     \begin{gather}
         \frac{1}{a} + \frac{1}{b} \neq \frac{4}{a + b}.
     \end{gather}
\end{prop}

\begin{proof}
    We will prove by contradiction. Suppose for the sake of contradiction that, for some positive integers $a, b$, $a \neq b$ and
    \begin{gather}
         \frac{1}{a} + \frac{1}{b} = \frac{4}{a + b}.
    \end{gather}
     We can do some arithmetic operations to the equation.
    \begin{gather}
        \frac{a + b}{ab} = \frac{4}{a + b} \\
        (a + b)^2 = 4ab \\
        a^2 + 2ab + b^2 = 4ab \\
        a^2 - 2ab + b^2 = 0 \\
        (a - b)^2 = 0 \\
    \end{gather}
    This suggests that $a - b = 0$, which implies that $a = b$. However, this contradicts our assumption $a \neq b$. \\\\
    Therefore, if a, b are positive real numbers with $a \neq b$, then
     \begin{gather}
         \frac{1}{a} + \frac{1}{b} \neq \frac{4}{a + b}.
     \end{gather}
\end{proof}

% Q5
\item \begin{prop}
    $\sqrt[4]{2}$ is irrational.
\end{prop}

\begin{proof}
We will prove by contradiction. Suppose for the sake of contradiction that $\sqrt[4]{2}$ is rational. \\\\
By HW4 fact 7, let $\sqrt[4]{2}$ be some rational number $\frac{m}{n}$, such that the greatest common divisor of $m,n$ is $1$. 
\begin{align}
    \sqrt{2} &= (\sqrt[4]{2})^2 \\
    &= (\frac{m}{n})^2 \\
    &= \frac{m^2}{n^2}
\end{align}
Let $m^2$ and $n^2$ be some integers $k, l$.
\begin{align}
    \frac{m^2}{n^2} &= \frac{k}{l}
\end{align}
This shows that if $\sqrt[4]{2}$ is rational, then $\sqrt{2}$ is rational by HW4 fact 7. However, this contradicts the fact that $\sqrt{2}$ is irrational. \\\\
Therefore, $\sqrt[4]{2}$ is irrational.
\end{proof}

% Q6
\item \begin{prop}
    There does not exist the smallest positive real number $x$ such that for all positive real number $y$, we have $x \leq y$.
\end{prop}
\begin{proof}
We will prove by contradiction. Suppose for the sake of contradiction that there exists the smallest positive real number $x$ such that for all positive real number $y$, we have $x \leq y$. \\\\
Let $y = \frac{x}{2}$. $\frac{x}{2}$ is a positive number, since $x > 0$ so $\frac{x}{2} > \frac{1}{2} \cdot 0 = 0$, which is positive by HW4 fact 5. This shows that $x > \frac{x}{2} = y$, which contradicts our assumption that for all positive real number $y$, we have $x \leq y$. \\\\
Therefore, there does not exist the smallest positive real number.
\end{proof}

% Q7
\item \begin{enumerate}
    \item \begin{prop}
    $A \subseteq A \cup B$.
    \end{prop}
    \begin{proof}
    Let $x \in A$. We will show that $x \in A \cup B$.
    $A \cup B$ implies that $(\forall y)[(y \in A \vee y \in B) \rightarrow (y \in A \cup B)]$. This shows that $(\forall x \in A)(x \in A \cup B)$. Therefore, $A \subseteq A \cup B$.
    \end{proof}
    \item \begin{prop}
    $A \cap B \subseteq A$.
    \end{prop}
    \begin{proof}
    Let $x \in A \cap B$. We will show that $x \in A$. $A \cap B$ implies that $(\forall y)[(y \in A \cup B) \rightarrow (y \in A \wedge y \in B)]$. This shows that $(\forall x \in A \cap B)(x \in A)$. Therefore, $A \cap B \subseteq A$.
    \end{proof}
    \item \begin{prop}
    If $A \subseteq B$ and $B \subseteq C$, then $A \subseteq C$.
    \end{prop}
    \begin{proof}
    Let $x \in A$. We will show that $x \in C$. $A \subseteq B$ means that $(\forall x \in A)(x \in B)$. $B \subseteq C$ means that $(\forall x \in B)(x \in C)$. This shows that $(\forall x \in A)[(x \in B) \rightarrow (x \in C)]$. Since $(\forall x \in A)(x \in B)$, $(\forall x \in A)(x \in C)$. Therefore, $A \subseteq C$.
    \end{proof}
\end{enumerate}

% Q8
\item \begin{prop}
    If $A \cap B^{c} = \emptyset$, then $A \subseteq B$.
\end{prop}

\begin{proof}
Let $x \in A$. We will show that $(\forall x \in A)[(A \cap B^{c} = \emptyset) \rightarrow (x \in B)]$. $(x \in A \cap B^{c})$ is equivalent to $(x \in A \wedge x \notin B)$. Hence, $A \cap B^{c} = \emptyset$ means that $(\nexists x \in A)(x \notin B)$, which is equivalent to $(\forall x \in A)(x \in B)$. Therefore, by definition, $A \subseteq B$.
\end{proof}
    
\end{enumerate}
\end{document}