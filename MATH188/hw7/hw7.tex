\documentclass{article}

\usepackage{fancyhdr}
\usepackage{extramarks}
\usepackage{amsmath}
\usepackage{amsthm}
\usepackage{amsfonts}
\usepackage{tikz}
\usepackage[plain]{algorithm}
\usepackage{algpseudocode}
\usepackage{enumerate}
\usepackage{amssymb}
\usepackage{dsfont}

\usetikzlibrary{automata,positioning}

%
% Basic Document Settings
%

\topmargin=-0.45in
\evensidemargin=0in
\oddsidemargin=0in
\textwidth=6.5in
\textheight=9.0in
\headsep=0.25in

\linespread{1.1}

\pagestyle{fancy}
\lhead{\hmwkAuthorName}
\chead{\hmwkClass:\ \hmwkTitle}
\rhead{\firstxmark}
\lfoot{\lastxmark}
\cfoot{\thepage}

\renewcommand\headrulewidth{0.4pt}
\renewcommand\footrulewidth{0.4pt}

\setlength\parindent{0pt}
\setlength{\parskip}{5pt}

%
% Create Problem Sections
%

\newcommand{\enterProblemHeader}[1]{ \nobreak\extramarks{}{Problem \arabic{#1} continued on next
    page\ldots}\nobreak{} \nobreak\extramarks{Problem \arabic{#1} (continued)}{Problem \arabic{#1}
    continued on next page\ldots}\nobreak{} }

\newcommand{\exitProblemHeader}[1]{ \nobreak\extramarks{Problem \arabic{#1} (continued)}{Problem
    \arabic{#1} continued on next page\ldots}\nobreak{}
    \stepcounter{#1}
    \nobreak\extramarks{Problem \arabic{#1}}{}\nobreak{}
}

\setcounter{secnumdepth}{0}
\newcounter{partCounter}
\newcounter{homeworkProblemCounter}
\setcounter{homeworkProblemCounter}{1}
\nobreak\extramarks{Problem \arabic{homeworkProblemCounter}}{}\nobreak{}

%
% Homework Problem Environment
%
% This environment takes an optional argument. When given, it will adjust the
% problem counter. This is useful for when the problems given for your
% assignment aren't sequential. See the last 3 problems of this template for an
% example.
%
\newenvironment{homeworkProblem}[1][-1]{
    \ifnum#1>0
        \setcounter{homeworkProblemCounter}{#1}
    \fi
    \section{Problem \arabic{homeworkProblemCounter}}
    \setcounter{partCounter}{1}
    \enterProblemHeader{homeworkProblemCounter}
}{
    \exitProblemHeader{homeworkProblemCounter}
}

%
% Homework Details
%   - Title
%   - Due date
%   - Class
%   - Section/Time
%   - Instructor
%   - Author
%

\newcommand{\hmwkTitle}{Homework\ \#7}
\newcommand{\hmwkDueDate}{Jun 14, 2024}
\newcommand{\hmwkClass}{MATH 188}
\newcommand{\hmwkClassInstructor}{Professor Kunnawalkam Elayavalli}
\newcommand{\hmwkAuthorName}{\textbf{Ray Tsai}}
\newcommand{\hmwkPID}{A16848188}

%
% Title Page
%

\title{
    \vspace{2in}
    \textmd{\textbf{\hmwkClass:\ \hmwkTitle}}\\
    \normalsize\vspace{0.1in}\small{Due\ on\ \hmwkDueDate\ at 23:59pm}\\
    \vspace{0.1in}\large{\textit{\hmwkClassInstructor}} \\
    \vspace{3in}
}

\author{
  \hmwkAuthorName \\
  \vspace{0.1in}\small\hmwkPID
}
\date{}

\renewcommand{\part}[1]{\textbf{\large Part \Alph{partCounter}}\stepcounter{partCounter}\\}

%
% Various Helper Commands
%

% Useful for algorithms
\newcommand{\alg}[1]{\textsc{\bfseries \footnotesize #1}}

% For derivatives
\newcommand{\deriv}[1]{\frac{\mathrm{d}}{\mathrm{d}x} (#1)}

% For partial derivatives
\newcommand{\pderiv}[2]{\frac{\partial}{\partial #1} (#2)}

% Integral dx
\newcommand{\dx}{\mathrm{d}x}

% Probability commands: Expectation, Variance, Covariance, Bias
\newcommand{\Var}{\mathrm{Var}}
\newcommand{\Cov}{\mathrm{Cov}}
\newcommand{\Bias}{\mathrm{Bias}}
\newcommand*{\Z}{\mathbb{Z}}
\newcommand*{\Q}{\mathbb{Q}}
\newcommand*{\R}{\mathbb{R}}
\newcommand*{\C}{\mathbb{C}}
\newcommand*{\N}{\mathbb{N}}
\newcommand*{\p}{\mathds{P}}
\newcommand*{\E}{\mathds{E}}

\begin{document}

\maketitle

\pagebreak

\begin{homeworkProblem}
  Do the case of general $n$ of Example 7.11, i.e., give a formula for the number of necklaces (considered equivalent up to reflection) of length $n$ using an alphabet of size $k$.

  \begin{proof}
    Note that $D_n$ consists of $n$ rotations and $n$ reflections. By Example 7.10, each rotations of order $i$ has $\gcd(n, i)$ cycles. Note that each reflection is of order 2. When $n$ is odd, each reflection fixes only 1 point, and thus each reflection consists of one $1$-cycle and $\frac{n - 1}{2}$ $2$-cycles. On the other hand, for even $n$, half of the reflections fixes 2 points and the other half fixes no point. That is, when $n$ is even, there are $\frac{n}{2}$ reflections with $\frac{n - 2}{2} + 2 = \frac{n}{2} + 1$ cycles and $\frac{n}{2}$ reflections with $\frac{n}{2}$ cycles. In total, there are 
    
    It now follows from Theorem 7.9 that there are
    \[
      \begin{cases}
        \frac{1}{2n}\sum_{i = 1}^n k^{\gcd(n, i)} + \frac{1}{2}\left(k^{\frac{n + 1}{2}}\right) & n \text{ is odd} \\
        \frac{1}{2n}\sum_{i = 1}^n k^{\gcd(n, i)} + \frac{1}{4}\left(k^{\frac{n}{2} + 1} + k^{\frac{n}{2}}\right) & n \text{ is even}
      \end{cases}
    \]
    necklaces.
  \end{proof}
\end{homeworkProblem}

\newpage

\begin{homeworkProblem}

Consider assigning one of $k$ colors to each of the entries of a $3 \times 3$ matrix.

\begin{enumerate}[(a)]
    \item How many ways are there to do this if we consider two colorings the same if they differ by rotation? To be explicit, one rotation clockwise means:
    \[
    \begin{bmatrix}
    a & b & c \\
    d & e & f \\
    g & h & i
    \end{bmatrix}
    \mapsto
    \begin{bmatrix}
    g & d & a \\
    h & e & b \\
    i & f & c
    \end{bmatrix}
    \]
    \begin{proof}
      Note that we may interpret the outer 8 elements of a $3 \times 3$ matrix in clockwise order as a word of length 8, up to ``even" cyclic shift. In particular, let $G = 2\Z/8$ be the group of even integers mod 8, let $X = \Z/8$ be the set of $8$ outer positions, and let $Y$ be the set of colors. Then a function $X \to Y$ is a word of length $8$, and a $G$-orbit represents a word up to ``even" cyclic shift. So the words up to ``even" cyclic shift are in bijection with $G$-orbits of $Y^X$. Each element of $G$ gives a permutation of some even power of $(01 \cdots 7)^{g}$. Specifically, the permutations are
      \begin{gather}
        (0246)(1357), (04)(15)(26)(37), (0642)(1753), (0)(1)(2)(3)(4)(5)(6)(7).
      \end{gather}
      It now follows from Theorem 7.9 that the number of orderings of the outer 8 elements of a $3 \times 3$, up to rotation, is $\frac{1}{4}(k^8 + k^4 + 2k^2)$. In addition to the 8 outer elements, we also have to determine the center element of the $3 \times 3$ matrix. Note that the choice of the center element is independent of the choice of the outer 8 elements. Hence, there are 
      \[
        \frac{1}{4}(k^9 + k^5 + 2k^3)
      \]
      ways to color the $9$ entries, up to rotations.
    \end{proof}

    \item How many colorings (up to rotation) are there that use exactly 3 different colors from the $k$, each used to color 3 entries?

    \begin{proof}
      We again interpret the outer 8 elements of a $3 \times 3$ matrix in clockwise order as a word of length 8, up to ``even" cyclic shift, and continue using $G, X, Y$ defined in (a). We need to use exactly 3 different colors, each used to color 3 entries. Let $W \subset Y^X$ be the set of a word of length 8 with exactly 3 colors, 3 entries being the first color, 3 being the second color, and the rest 2 entries be the last color. Since there are $\binom{k}{3}$ ways to pick $3$ colors from $Y$, $3$ way to pick the color which only appears twice in the word, and $\frac{8!}{3!3!2!}$ ways to arrange the colors, we have $|W| = 3\binom{k}{3}\frac{8!}{3!3!2!} = 1680\binom{k}{3}$. Notice in (1) that the trivial permutation $I = (0)(1)(2)(3)(4)(5)(6)(7)$ is the only permutations given by $G$ whose cycles all have lengths that divide 3. That is, $I$ is the only permutation which fixes any word $w \in W$. It now follows by the Burnside Lemma that the number of ways to color the outer 8 elements
      given our rule is
      \[
        |W/G| = \frac{1}{|G|}\sum_{g \in G} |W^g| = \frac{1}{|G|}|W^{I}| = \frac{1}{|G|}|W| = \frac{1680\binom{k}{3}}{4} = 420\binom{k}{3}.
      \]
      But then according to our rule, the center entry of the matrix is determined by the outer 8 entries, so this is also the total number of ways to color the whole matrix with our rule, up to rotation.
    \end{proof}
\end{enumerate}
\end{homeworkProblem}

\newpage

\begin{homeworkProblem}
  In Theorem 7.9, take $X = [n]$, $Y = [d]$, and $G = \mathfrak{S}_n$ with the natural action on $X$.
  \begin{enumerate}[(a)]
      \item Find a bijection between $G$-orbits on $Y^X$ and weak compositions; give a closed formula for their number using this interpretation.
      \begin{proof}
        Note that the each $G$-orbit on $Y^X$ represents a word up to the ordering of the characters. Let $O$ be a $G$-orbit. Suppose that a word in $O$ consists of $a_i$ number of $i$'s, for each $i \in [d]$. Note that $a_1 + \cdots + a_d = n$ and $0 \leq a_i \leq n$ for all $i$, which makes $(a_1, \ldots, a_d)$ a weak compositions of $n$ with $d$ parts. Since each word in $O$ contains the same number of each $i$, it is well-defined to map $O$ to the weak composition $(a_1, \ldots, a_d)$. 
        
        On the other hand, given $(a_1, \ldots, a_d)$ a weak compositions of $n$ with $d$ parts, we may map it to a $G$-orbit $O$ such that each word $w \in O$ contains $a_i$ number of $i's$, for all $i \in [d]$. This mapping is well-defined because words which contain the same number of each characters are in the same orbit, and hence the bijection.

        It now follows that
        \[
          |[d]^{[n]}/\mathfrak{S}_n| = \binom{n + d - 1}{n}.
        \]
      \end{proof}
      \item By varying $d$, explain how the equality between the expression in Theorem 7.9 and your answer to (a) gives a new proof for Corollary 3.30.

      \begin{proof}
        Given a perumtation $\sigma$, let $c(\sigma)$ denote the number of cycles in $\sigma$. By Theorem 7.9 and (a),
        \[
          \binom{n + d - 1}{n} = |[d]^{[n]}/\mathfrak{S}_n| = \frac{1}{n!}\sum_{\sigma \in \mathfrak{S}_n} d^{c(\sigma)} = \frac{1}{n!}\sum_{k = 1}^n c(n, k) d^{k}.
        \]
        It now follows that
        \[
          \frac{(n + d - 1)!}{(d - 1)!} = \sum_{k = 0}^n c(n, k) d^{k}.
        \]
      \end{proof}
  \end{enumerate}
\end{homeworkProblem}

\newpage

\begin{homeworkProblem}
  Let $p$ be a prime and $n \geq p$. Use the method of \S7.4 for the following:
  \begin{enumerate}[(a)]
    \item Show that
    \[
    S(n, k) \equiv S(n-p, k-p) + S(n-p+1, k) \pmod{p}.
    \]
    \begin{proof}
     Let $X$ be the set of partitions of $[n]$ into $k$ blocks. Let $\sigma$ be the permutation which is the $p$-cycle $(12 \cdots p)$. Given a set $S = \{s_1, \ldots, s_m\} \subseteq [n]$, define $g \in \mathfrak{S}_n$ such that $g(S) = \{\sigma(s_1), \ldots, \sigma(s_m)\}$. Hence, given partition $P = \{B_1, \ldots, B_k\} \in X$, we may also define $g(P)$ to be $\{\sigma(B_1), \ldots, \sigma(B_k)\}$. Note that $g$ generates a cyclic group of order $p$.
     
     Now consider $X^g$. Suppose $P \in X^g$. Then, $P = g(P)$. That is, $\sigma: P \to P$ is also a permutation of $P$. But then note that $\sigma^p(B_j) = B_j$ for all $j$, so the lengths of cycles of $\sigma$ as a permutation of $P$ divide $p$, which can either be $1$ or $p$. 
     
     Suppose that $\sigma$ acts as a trivial permutation on $P$. Consider some $B_j \in P$ which contains 1. Since $\sigma(B_j) = B_j$, we know $2 = \sigma(1) \in B_j$. It now follows from induction that $\{1, \ldots, p\} \subseteq B_j$, and there are $S(n-p+1, k)$ such partitions in $X$. 
     
     On the other hand, suppose $\sigma$ contains a $p$ cycle when acting on $P$. Since $\sigma(B_j) = B_j$ if $B_j \cap \{1, \ldots, p\} = \emptyset$, we know every block $B_l$ in the $p$ cycle contains some $i \in \{1, \ldots, p\}$, and thus each $B_l$ in the $p$ cycle contains exactly one element in $\{1, \ldots, p\}$. Observe that if $B_l$ contains an element not in $\{1, \ldots, p\}$, then $\sigma(B_l)$ is different from any block in the $p$ cycle. Hence, each $B_l = \{i\}$, for some $1 \leq i \leq p$, and there are $S(n-p, k-p)$ such partitions in $X$. 
     
     It now follows that $|X^g| = S(n-p, k-p) + S(n-p+1, k)$ and Lemma 7.15 that 
     \[
        S(n, k) \equiv S(n-p, k-p) + S(n-p+1, k) \pmod{p}.
     \]
    \end{proof}
    \item Show that
    \[
    c(n, k) \equiv c(n-p, k-p) - c(n-p, k-1) \pmod{p}.
    \]
    \begin{proof}
      Let $X$ be the set of permutations in $\mathfrak{S}_n$ with exactly $k$ different cycles, and we let $\mathfrak{S}_n$ act on $X$ by conjugation. Let $\sigma \in X$. Let $g = (12 \cdots p) \in \mathfrak{S}_n$. Note that $g$ generates a cyclic group of order $p$.
        
      Now consider $X^g$. Suppose $\sigma \in X$. Since $g \cdot \sigma = g\sigma g^{-1} = \sigma$, we have $g = \sigma g \sigma^{-1}$, and thus $(12 \cdots p) = (\sigma(1)\sigma(2) \cdots \sigma(p))$. Hence, $\sigma$ cyclic shifts each element in $\Z/p$ by some constant $r \in \Z/p$. 
      
      If $r = 0$, then $\sigma$ consists of trivial cycles $(1)(2) \cdots (p)$ and $k - p$ cycles using the remaining $n - p$ elements. Hence, there are $c(n - p, k - p)$ such $\sigma$ in this case. 

      On the other hand, if $1 \leq r \leq p - 1$, then $\sigma$ consists of a cycle $(1 + r, \, 2 + r, \, \cdots \, p + r)$ and $k - 1$ cycles using the remaining $n - p$ elements. Since there are $p - 1$ choices for $r$, there are $(p - 1)c(n - p, k - 1)$ such $\sigma$ in this case.
      
      Hence, we have $|X^g| = c(n-p, k-p) + (p - 1)c(n-p, k-1)$. It now follows from Lemma 7.15 that
      \[
        c(n, k) \equiv c(n-p, k-p) - c(n-p, k-1) \pmod{p}.
      \]
    \end{proof}
  \end{enumerate}
\end{homeworkProblem}
\end{document}