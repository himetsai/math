\documentclass{article}

\usepackage{fancyhdr}
\usepackage{extramarks}
\usepackage{amsmath}
\usepackage{amsthm}
\usepackage{amsfonts}
\usepackage{tikz}
\usepackage[plain]{algorithm}
\usepackage{algpseudocode}
\usepackage{enumerate}
\usepackage{amssymb}
\usepackage{dsfont}

\usetikzlibrary{automata,positioning}

%
% Basic Document Settings
%

\topmargin=-0.45in
\evensidemargin=0in
\oddsidemargin=0in
\textwidth=6.5in
\textheight=9.0in
\headsep=0.25in

\linespread{1.1}

\pagestyle{fancy}
\lhead{\hmwkAuthorName}
\chead{\hmwkClass:\ \hmwkTitle}
\rhead{\firstxmark}
\lfoot{\lastxmark}
\cfoot{\thepage}

\renewcommand\headrulewidth{0.4pt}
\renewcommand\footrulewidth{0.4pt}

\setlength\parindent{0pt}
\setlength{\parskip}{5pt}

%
% Create Problem Sections
%

\newcommand{\enterProblemHeader}[1]{
    \nobreak\extramarks{}{Problem \arabic{#1} continued on next page\ldots}\nobreak{}
    \nobreak\extramarks{Problem \arabic{#1} (continued)}{Problem \arabic{#1} continued on next page\ldots}\nobreak{}
}

\newcommand{\exitProblemHeader}[1]{
    \nobreak\extramarks{Problem \arabic{#1} (continued)}{Problem \arabic{#1} continued on next page\ldots}\nobreak{}
    \stepcounter{#1}
    \nobreak\extramarks{Problem \arabic{#1}}{}\nobreak{}
}

\setcounter{secnumdepth}{0}
\newcounter{partCounter}
\newcounter{homeworkProblemCounter}
\setcounter{homeworkProblemCounter}{1}
\nobreak\extramarks{Problem \arabic{homeworkProblemCounter}}{}\nobreak{}

%
% Homework Problem Environment
%
% This environment takes an optional argument. When given, it will adjust the
% problem counter. This is useful for when the problems given for your
% assignment aren't sequential. See the last 3 problems of this template for an
% example.
%
\newenvironment{homeworkProblem}[1][-1]{
    \ifnum#1>0
        \setcounter{homeworkProblemCounter}{#1}
    \fi
    \section{Problem \arabic{homeworkProblemCounter}}
    \setcounter{partCounter}{1}
    \enterProblemHeader{homeworkProblemCounter}
}{
    \exitProblemHeader{homeworkProblemCounter}
}

%
% Homework Details
%   - Title
%   - Due date
%   - Class
%   - Section/Time
%   - Instructor
%   - Author
%

\newcommand{\hmwkTitle}{Homework\ \#2}
\newcommand{\hmwkDueDate}{Apr 24, 2024}
\newcommand{\hmwkClass}{MATH 188}
\newcommand{\hmwkClassInstructor}{Professor Kunnawalkam Elayavalli}
\newcommand{\hmwkAuthorName}{\textbf{Ray Tsai}}
\newcommand{\hmwkPID}{A16848188}

%
% Title Page
%

\title{
    \vspace{2in}
    \textmd{\textbf{\hmwkClass:\ \hmwkTitle}}\\
    \normalsize\vspace{0.1in}\small{Due\ on\ \hmwkDueDate\ at 23:59pm}\\
    \vspace{0.1in}\large{\textit{\hmwkClassInstructor}} \\
    \vspace{3in}
}

\author{
  \hmwkAuthorName \\
  \vspace{0.1in}\small\hmwkPID
}
\date{}

\renewcommand{\part}[1]{\textbf{\large Part \Alph{partCounter}}\stepcounter{partCounter}\\}

%
% Various Helper Commands
%

% Useful for algorithms
\newcommand{\alg}[1]{\textsc{\bfseries \footnotesize #1}}

% For derivatives
\newcommand{\deriv}[1]{\frac{\mathrm{d}}{\mathrm{d}x} (#1)}

% For partial derivatives
\newcommand{\pderiv}[2]{\frac{\partial}{\partial #1} (#2)}

% Integral dx
\newcommand{\dx}{\mathrm{d}x}

% Probability commands: Expectation, Variance, Covariance, Bias
\newcommand{\Var}{\mathrm{Var}}
\newcommand{\Cov}{\mathrm{Cov}}
\newcommand{\Bias}{\mathrm{Bias}}
\newcommand*{\Z}{\mathbb{Z}}
\newcommand*{\Q}{\mathbb{Q}}
\newcommand*{\R}{\mathbb{R}}
\newcommand*{\C}{\mathbb{C}}
\newcommand*{\N}{\mathbb{N}}
\newcommand*{\p}{\mathds{P}}
\newcommand*{\E}{\mathds{E}}

\begin{document}

\maketitle

\pagebreak

\begin{homeworkProblem}
  Let $F(x)$ be a formal power series with $F(0) = 0$.
  \begin{enumerate}[(a)]
    \item Show that there exists a formal power series $G(x)$ with $G(0) = 0$ such that $F(G(x)) =
    x$ if and only if $[x^1]F(x) \neq 0$.
    \begin{proof}
      Let $F(x) = \sum_{n = 0}^{\infty} a_nx^n$, for some nonzero $a_1$ and $a_0 = 0$. We look for a
      formal power series $G(x) = \sum_{n = 0}^{\infty} b_nx^n$ such that $F(G(x)) = x$ and $b_0 =
      0$. That is,
      \begin{align*}
        F(G(x))
        &= \sum_{i = 1}^{\infty} a_iG(x)^i \\
        &= \sum_{n = 1}^{\infty} x^n \sum_{i = 1}^{n} a_i \sum_{m_1 + m_2 + \dots + m_i = n} b_{m_1}b_{m_2} \cdots b_{m_i} = x.
      \end{align*}
      Note that the inner summation terminates at $n$, as we are enumerating through compositions of
      $n$, which could not exceed $n$ terms. By comparing coefficients, we have
      \[
        b_0 = 0, \quad b_1 = \frac{1}{a_1},
      \]
      and for $n \geq 2$,
      \begin{gather}
        \sum_{i = 1}^{n} a_i \sum_{m_1 + m_2 + \dots + m_i = n} b_{m_1}b_{m_2} \cdots b_{m_i} = 0.
      \end{gather}
      Here, we already know that $G(x)$ exists only if $[x^1]F(x) \neq 0$, it remains to show the
      converse. Suppose $[x^1]F(x) \neq 0$. We already determined the unique existence of $b_1$. For
      $n \geq 2$, rearranging (1) gives an expression of $b_n$ uniquely determined by $a_{1}, \dots
      a_{n}, b_1, \dots b_{n - 1}$. But then the existence of $b_1, \dots b_{n - 1}$ are shown by
      induction, and this ensures the unique existence of $b_n$.
    \end{proof}
    \item Assuming $[x^1]F(x) \neq 0$, show that $G(x)$ is unique and also satisfies $G(F(x)) = x$.
    You may use without proof that composition of formal power series is associative.
    \begin{proof}
      Uniqueness of $G(x)$ is shown in (a). We know $[x^1]G(x) \neq 0$. By (a), there exists a
      formal power series $H(x)$ with $H(0) = 0$ such that $G(H(x)) = x$. But then $F(x) =
      F(G(H(x))) = H(x)$. 
    \end{proof}
  \end{enumerate}
\end{homeworkProblem}

\newpage

\begin{homeworkProblem}
  Evaluate the following sums:
  \begin{enumerate}[(a)]
    \item 
    \[
      \sum_{i=0}^{n} \binom{n}{i} \frac{1}{2^i}
    \]
    \begin{proof}
      By the binomial theorem,
      \[
        \sum_{i=0}^{n} \binom{n}{i} \frac{1}{2^i} = \left(1 + \frac{1}{2}\right)^n = \frac{3^n}{2^n}.
      \]
    \end{proof}
    \item 
    \[
      \sum_{i=0}^{n} i^2 \binom{n}{i} 3^i
    \]
    \begin{proof}
      By the binomial theorem,
      \begin{gather*}
        \sum_{n \geq 1} i\binom{n}{i}x^{i - 1} = \left(\sum_{n \geq 0} \binom{n}{i}x^{i}\right)' = ((1 + x)^n)' = n(1 + x)^{n - 1}, \\
        \sum_{n \geq 2} i(i - 1)\binom{n}{i}x^{i - 2} = \left(\sum_{n \geq 0} \binom{n}{i}x^{i}\right)'' = ((1 + x)^n)'' = n(n - 1)(1 + x)^{n - 2}.
      \end{gather*}
      Hence,
      \begin{align*}
        \sum_{i=0}^{n} i^2 \binom{n}{i} 3^i
        &= \sum_{i=0}^{n} i(i - 1) \binom{n}{i} 3^i + \sum_{i=0}^{n} i \binom{n}{i} 3^i \\
        &= 9\sum_{i=2}^{n} i(i - 1) \binom{n}{i} 3^{i - 2} + 3\sum_{i=1}^{n} i \binom{n}{i} 3^{i - 1} \\
        &= 9\left(\sum_{i=0}^{n} \binom{n}{i} 3^{i}\right)'' + 3\left(\sum_{i=0}^{n} \binom{n}{i} 3^{i}\right)' \\
        &= 9n(n - 1)(1 + 3)^{n - 2} + 3n(1 + 3)^{n - 1} \\
        &= \frac{9}{16}n(n - 1)4^n + \frac{3}{4}n4^n = 3n(3n + 1)4^{n - 2}.
      \end{align*}
    \end{proof}
  \end{enumerate}
\end{homeworkProblem}

\newpage

\begin{homeworkProblem}
  Let $a, b$ be non-negative integers.
  \begin{enumerate}[(a)]
    \item By comparing coefficients in $(1 + x)^{a+b} = (1 + x)^a(1 + x)^b$, prove that for any
    non-negative integer $n$, we have
    \[
      \binom{a + b}{n} = \sum_{i=0}^{n} \binom{a}{i} \binom{b}{n - i}.
    \]
    \begin{proof}
      By the binomial theorem,
      \begin{align*}
        \binom{a + b}{n}
        &= [x^n](1 + x)^{a + b} \\
        &= [x^n](1 + x)^a(1 + x)^b \\
        &= \sum_{i = 0}^n \left([x^i](1 + x)^a\right)\left([x^{n - i}](1 + x)^b\right) \\
        &= \sum_{i = 0}^n \binom{a}{i}\binom{b}{n - i}.
      \end{align*}
    \end{proof}
    \item Now prove this identity using a counting argument.
    \begin{proof}
      Consider choosing $n$ animals from $a$ dogs and $b$ cats. Suppose that we picked $i$ dogs.
      There are $\binom{a}{i}$ ways of choosing them. In order to have $n$ animals in total, we then
      have to pick $n - i$ cats, which has $\binom{b}{n - i}$ ways. The possible values for $i$ are
      between $0$ and $n$, and thus we get the identity
      \[
        \binom{a + b}{n} = \sum_{i=0}^{n} \binom{a}{i} \binom{b}{n - i}.
      \]
    \end{proof}
  \end{enumerate}
\end{homeworkProblem}

\newpage

\begin{homeworkProblem}
  How many ways can we arrange the letters of: MISSISSIPPI?

  \begin{proof}
    There are one M, four I's, two P's, and four S', and we have 11 slots in total. We first choose
    a slot for the M, which has $\binom{11}{1}$ ways. Then, we choose 4 slots from the remaining 10
    slots for the I's, which has $\binom{10}{4}$ ways. Then, we choose 2 slots from the remaining 6
    slots for the P's, which has $\binom{6}{2}$ ways. Finally, we choose 4 slots from the remaining
    4 slots for the S's, which has $\binom{4}{4}$ ways. In total, there are
    \[
      \binom{11}{1}\binom{10}{4}\binom{6}{2}\binom{4}{4} = \frac{11!}{4!2!4!}
    \]
    ways of arranging the letters of MISSISSIPPI.
  \end{proof}
\end{homeworkProblem}

\newpage

\begin{homeworkProblem}
  Let $f(t) = \sum_{k=0}^{d} f_kt^k$ be a degree $d$ polynomial with rational coefficients. From
  lecture, we know that there exist unique rational numbers $g_0, \ldots, g_d$ such that
  \begin{gather}
    \sum_{n \geq 0} f(n)x^n = \frac{g_0 + g_1x + \ldots + g_dx^d}{(1 - x)^{d+1}}.
  \end{gather}
  Now assume that $f(a)$ is an integer for $a = 0, \ldots, d$. (The $f_k$ don’t have to be integers
  for this to be true, for example $f(n) = n(n - 1)/2$ has this property.) Prove that this implies
  that the $g_k$ are all integers and that $f(a)$ is an integer whenever $a$ is an integer.

  \begin{proof}
    From (2), for $k = 0, 1, \dots, d$,
    \begin{align*}
      g_k
      &= [x^k](1 - x)^{d+1}\sum_{n \geq 0} f(n)x^n \\
      &= \sum_{i = 0}^k (-1)^{k - i}\binom{d + 1}{k - i}f(i),
    \end{align*}
    which is an integer as $f(i)$ and $\binom{d + 1}{k - i}$ are both integers, for $i = 0, \dots,
    d$. But then for $n \in \Z_{\geq 0}$,
    \begin{align*}
      f(n) 
      &= [x^n](1 - x)^{-(d + 1)}(g_0 + g_1x + \ldots + g_dx^d) \\
      &= \sum_{k = 0}^d \binom{d + n - k}{n - k}g_k = \sum_{k = 0}^d \binom{d + n - k}{d}g_k.
    \end{align*}
    Note that $h(n) = \sum_{k = 0}^d \binom{d + n - k}{d}g_k$ is a polynomial of degree $d$. Since
    $f(n) - h(n) = 0$ for all $n \in \Z_{\geq 0}$, it follows from the Fundamental Theorem of
    Algebra that $f(n) = h(n)$. Since $g_k \in \Z$ and $\binom{d + n - k}{n - k} \in \Z$ whenever $n
    \in \Z$, we know $f(n)$ is an integer whenever $n \in \Z$. 
  \end{proof}
\end{homeworkProblem}

\newpage

\begin{homeworkProblem}
  Let $n \geq 2$ be an integer.
  \begin{enumerate}[(a)]
    \item Prove that
    \[
    \sum_{i=0}^{n} i \binom{n}{i} (-1)^{i-1} = 0.
    \]
    \begin{proof}
      By the binomial theorem,
      \[
        \sum_{n \geq 1} i\binom{n}{i}x^{i - 1} = \left(\sum_{n \geq 0} \binom{n}{i}x^{i}\right)' = ((1 + x)^n)' = n(1 + x)^{n - 1}, \\
      \]
      and thus 
      \[
        \sum_{i=0}^{n} i \binom{n}{i} (-1)^{i-1} = n(1 + (-1))^{n - 1} = 0
      \]
    \end{proof}
    \item Compute
    \[
      \sum_{\substack{0 \leq i \leq n \\ i \text{ even}}} i \binom{n}{i}.
    \]
    \begin{proof}
      \begin{align*}
        \sum_{\substack{0 \leq i \leq n \\ i \text{ even}}} i \binom{n}{i}
        &= \frac{1}{2}\left(\sum_{i=0}^{n} i \binom{n}{i} - \sum_{i=0}^{n} i \binom{n}{i} (-1)^{i-1}\right) \\
        &= \frac{1}{2}(n(1 + 1)^{n - 1}) = n2^{n - 2}.
      \end{align*}
    \end{proof}
  \end{enumerate}
\end{homeworkProblem}

\newpage

\begin{homeworkProblem}
  \begin{enumerate}[(a)]
    \item Let $a, b$ be rational numbers. Show that for any formal power series $A(x)$ with $A(0) =
    1$, we have
    \[
    A(x)^a A(x)^b = A(x)^{a+b}.
    \]
    \textit{[Remember that we defined rational powers in a very specific way, so your proof needs to use this definition.]}
    
    \begin{proof}
      By definition, $A(x)^{m/n} = (A(x)^{1/n})^m = (A(x)^m)^{1/n}$. Let $a = m/n$, $b = p/q$, for
      some $m, n, p, q \in \Z$. Then,
      \begin{align*}
        A(x)^a A(x)^b
        &= (A(x)^{1/{nq}})^{mq}(A(x)^{1/{nq}})^{np} \\
        &= (A(x)^{1/{nq}})^{mq + np} \\
        &= A(x)^{a + b}.
      \end{align*}
    \end{proof}

    \item Deduce from (a) that
    \[
    \binom{a + b}{n} = \sum_{i=0}^{n} \binom{a}{i} \binom{b}{n - i}
    \]
    for all non-negative integers $n$.
    \begin{proof}
      Put $A(x) = (1 + x)$. Since $(1 + x)^a(1 + x)^b = (1 + x)^{a + b}$, 
      \begin{align*}
        \binom{a + b}{n}
        &= [x^n](1 + x)^{a + b} \\
        &= [x^n](1 + x)^a(1 + x)^b \\
        &= \sum_{i = 0}^n \left([x^i](1 + x)^a\right)\left([x^{n - i}](1 + x)^b\right) \\
        &= \sum_{i = 0}^n \binom{a}{i}\binom{b}{n - i}.
      \end{align*}
    \end{proof}
  \end{enumerate}  
\end{homeworkProblem}

\newpage

\begin{homeworkProblem}
  Assume now that we deal with complex-coefficient formal power series. Define the following sets of
  formal power series:
  \[
    V = \{ F(x) \mid F(0) = 0 \}, \quad W = \{ G(x) \mid G(0) = 1 \}.
  \]
  \begin{enumerate}[(a)]
      \item Given $F \in V$, show that $\textbf{E}(F) = \sum_{n \geq 0} \frac{F^{n}(x)}{n!}$ is the
      \textit{unique} formal power series $G \in W$ such that $DG = DF \cdot G$. This defines a
      function $\textbf{E}\colon V \to W$. [Convention: $F^{0}(x) = 1$ even if $F(x) = 0$.]

      \begin{proof}
        It is easy to see that
        \[
          DG = \sum_{n \geq 0} \frac{D(F^{n}(x))}{n!} = \sum_{n \geq 1} DF \cdot \frac{F^{n - 1}(x)}{(n - 1)!} = DF \sum_{n \geq 0} \frac{F^{n}(x)}{n!} = DF \cdot G,
        \]
        and $G(0) = F^0(0) = 1$. It remains to show that $G$ is unique. Suppose there exists $G =
        \sum_{n \geq 0} b_nx_n, G' = \sum_{n \geq 0} b'_nx_n \in W$ such that $\textbf{E}(F) = G$
        and $\textbf{E}(F) = G'$. Suppose $DF = \sum_{n \geq 0} a_nx_n$ We know $DG = DF \cdot G$
        and $DG' = DF \cdot G'$. By comparing coefficients, for $k \geq 1$,
        \begin{gather*}
          \frac{b_k}{k + 1} = [x^k]DG = [x^k](DF \cdot G) = \sum_{i = 0}^k a_{i}b_{k - i}, \\
          \frac{b'_k}{k + 1} = [x^k]DG = [x^k](DF \cdot G) = \sum_{i = 0}^k a_{i}b'_{k - i}.
        \end{gather*}
        In particular, for $k \geq 1$,
        \[
          b_k = \frac{k + 1}{-a_0k - a_0 + 1}\sum_{i = 1}^k a_ib_{k - i},  \quad b_k = \frac{k + 1}{-a_0k - a_0 + 1}\sum_{i = 1}^k a_ib_{k - i},
        \]
        so $b_k, b'_k$ are uniquely determined by the corrresponding previous coefficients, and thus
        $G = G'$ if and only if they agree with the constant term. But then $G(0) = G'(0) = 1$, and
        the result follows. 
      \end{proof}
      
      \item Given $G \in W$, show that there is a \textit{unique} formal power series $F \in V$ such
      that $DF(x) = DG(x)/G(x)$. We define the function $\textbf{L}\colon W \to V$ by $\textbf{L}(G)
      = F$. [For the rest, it is unnecessary to use explicit formulas for $\textbf{L}$ and
      $\textbf{E}$ and in fact it may be easier to only use the uniqueness properties above.]
      \begin{proof}
        Since $G(0) = 1$, there exists $G^{-1}(x)$ such that $G(x)G^{-1}(x) = G(x)^{-1}G(x) = 1$, so
        $DG(x)/G(x)$ is unique given $G$. Suppose $DG(x)/G(x) = \sum_{n \geq 0} a_nx^n$. There
        exists $F = \sum_{n \geq 1} \frac{1}{n}a_{n - 1}x^n \in V$ such that
        \[
          DF = \sum_{n \geq 1} a_{n - 1}x^{n - 1} = \sum_{n \geq 0} a_nx^n = DG(x)/G(x).
        \]
        That is, all coefficients $a_n$ of $DF$ are uniquely determined by $DG(x)/G(x)$. But then
        all coefficients of $F$ are uniquely determined, as $F$ has no constant term.
      \end{proof}

      \break
      
      \item Show that $\textbf{E}$ and $\textbf{L}$ are inverses of each other.
      \begin{proof}
        Let $F \in V$. $\textbf{E}$ maps $F$ to some unique $G' \in W$ such that $DG' = DF \cdot
        G'$, that is, $DF = DG'/G'$. Then, $\textbf{L}$ maps $G'$ back to some unique $F'$ such that
        $DF' = DG'/G' = DF$. But then both $F$ and $F'$ have no constant terms, so $F$ and $F'$
        actually agree with all coefficients. Hence, $\textbf{L}(\textbf{E}(F)) = F$. 

        Let $G \in W$. $\textbf{L}$ maps $G$ to some unique $F'' \in V$ such that $DG/G = DF''$, and
        $\textbf{E}$ maps $F''$ back to some unique $G''$ such that $DG'' = DF'' \cdot G'' = DG/G
        \cdot G''$. But then $DG''/G'' = DG/G$. By comparing coefficients, for all $k \geq 0$ we get
        \[
          \sum_{i = 0}^k b''_{k - i}(i + 1)b_{i + 1} = [x^k]DG'' \cdot G = [x^k]DG \cdot G'' = \sum_{i = 0}^k b_{k - i}(i + 1)b''_{i + 1}.
        \]
        Since $b_0 = b''_0 = 1$, it follows from induction that $b_k = b''_k$ for all $k \in
        \Z_{\geq 0}$, and so $G = G''$. Hence, $\textbf{E}(\textbf{L}(G)) = G$. 
      \end{proof}
      
      \item Show that $\textbf{E}(F_1 + F_2) = \textbf{E}(F_1)\textbf{E}(F_2)$ for all $F_1, F_2 \in
      V$.
      \begin{proof}
        Let $G_1 = \textbf{E}(F_1)$, $G_2 = \textbf{E}(F_2)$, and $G = \textbf{E}(F_1 + F_2)$. Since
        \begin{align*}
          D(G_1G_2)
          &= DG_1 \cdot G_2 + DG_2 \cdot G_1 \\
          &= (DF_1 \cdot G_1)G_2 + (DF_2 \cdot G_2)G_1 \\
          &= (DF_1 + DF_2)(G_1G_2) \\
          &= D(F_1 + F_2)(G_1G_2).
        \end{align*}
        Note that $G_1G_2 \in W$. But then $G$ is the unique element in $W$ such that $DG = D(F_1 +
        F_2)G$, and so $\textbf{E}(F_1 + F_2) = G = G_1G_2 = \textbf{E}(F_1)\textbf{E}(F_2)$.
      \end{proof}
      
      \item Show that $\textbf{L}(G_1G_2) = \textbf{L}(G_1) + \textbf{L}(G_2)$ for all $G_1, G_2 \in
      W$.
      \begin{proof}
        Let $F_1 = \textbf{L}(G_1)$, $F_2 = \textbf{L}(G_2)$. Since
        \begin{align*}
          D(F_1 + F_2)
          &= DF_1 + DF_2 \\
          &= DG_1/G_1 + DG_2/G_2,
        \end{align*}
        \begin{gather}
          G_1G_2D(F_1F_2) = DG_1 \cdot G_2 + DG_2 \cdot G_1 = D(G_1G_2),
        \end{gather}
        that is, $D(F_1F_2) = D(G_1G_2)/G_1G_2$. But then $F_1F_2 \in F$, so $F_1F_2$ is the unique
        element in $W$ that satisfies (3), and thus $\textbf{L}(G_1G_2) = F_1F_2 = \textbf{L}(G_1) +
        \textbf{L}(G_2)$.
      \end{proof}
      
      \item If $m$ is a positive integer and $G \in W$, show that
      $\textbf{E}(\frac{\textbf{L}(G)}{m})$ is an $m$th root of $G$. [This gives an alternative
      proof for the existence of $m$th roots and in fact we can now define powers for any complex
      number $m$: $F^m = \textbf{E}(m\textbf{L}(F))$.]

      \begin{proof}
        By (e),
        \[
          \textbf{L}\left[\left(\textbf{E}\left(\frac{\textbf{L}(G)}{m}\right)\right)^m\right] = m\textbf{L}\left[\textbf{E}\left(\frac{\textbf{L}(G)}{m}\right)\right] = m \cdot \frac{\textbf{L}(G)}{m} = \textbf{L}(G).
        \]
        But then $\textbf{L}$ is bijective, and the result follows.
      \end{proof}

      \break

      \item Show that if $\sum_{i \geq 0} F_i(x)$ converges to $F(x)$, then $\prod_{i \geq 0}
      \textbf{E}(F_i)$ converges to $\textbf{E}(F)$.
      \begin{proof}
        By (d),
        \[
          \textbf{E}(F(x)) = \textbf{E}\left(\sum_{i \geq 0} F_i(x)\right) = \prod_{i \geq 0} \textbf{E}(F_i).
        \]
      \end{proof}
      
      \item Show that if $\prod_{i \geq 0} G_i(x)$ converges to $G(x)$, then $\sum_{i \geq 0}
      \textbf{L}(G_i)$ converges to $\textbf{L}(G)$.
      \begin{proof}
        By (e),
        \[
          \textbf{L}(G(x)) = \textbf{L}\left(\prod_{i \geq 0} G_i(x)\right) = \sum_{i \geq 0} \textbf{L}(G_i).
        \]
      \end{proof}
  \end{enumerate}

\end{homeworkProblem}
\end{document}