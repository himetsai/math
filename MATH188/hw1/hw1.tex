\documentclass{article}

\usepackage{fancyhdr}
\usepackage{extramarks}
\usepackage{amsmath}
\usepackage{amsthm}
\usepackage{amsfonts}
\usepackage{tikz}
\usepackage[plain]{algorithm}
\usepackage{algpseudocode}
\usepackage{enumerate}
\usepackage{amssymb}
\usepackage{dsfont}

\usetikzlibrary{automata,positioning}

%
% Basic Document Settings
%

\topmargin=-0.45in
\evensidemargin=0in
\oddsidemargin=0in
\textwidth=6.5in
\textheight=9.0in
\headsep=0.25in

\linespread{1.1}

\pagestyle{fancy}
\lhead{\hmwkAuthorName}
\chead{\hmwkClass:\ \hmwkTitle}
\rhead{\firstxmark}
\lfoot{\lastxmark}
\cfoot{\thepage}

\renewcommand\headrulewidth{0.4pt}
\renewcommand\footrulewidth{0.4pt}

\setlength\parindent{0pt}
\setlength{\parskip}{5pt}

%
% Create Problem Sections
%

\newcommand{\enterProblemHeader}[1]{
    \nobreak\extramarks{}{Problem \arabic{#1} continued on next page\ldots}\nobreak{}
    \nobreak\extramarks{Problem \arabic{#1} (continued)}{Problem \arabic{#1} continued on next page\ldots}\nobreak{}
}

\newcommand{\exitProblemHeader}[1]{
    \nobreak\extramarks{Problem \arabic{#1} (continued)}{Problem \arabic{#1} continued on next page\ldots}\nobreak{}
    \stepcounter{#1}
    \nobreak\extramarks{Problem \arabic{#1}}{}\nobreak{}
}

\setcounter{secnumdepth}{0}
\newcounter{partCounter}
\newcounter{homeworkProblemCounter}
\setcounter{homeworkProblemCounter}{1}
\nobreak\extramarks{Problem \arabic{homeworkProblemCounter}}{}\nobreak{}

%
% Homework Problem Environment
%
% This environment takes an optional argument. When given, it will adjust the
% problem counter. This is useful for when the problems given for your
% assignment aren't sequential. See the last 3 problems of this template for an
% example.
%
\newenvironment{homeworkProblem}[1][-1]{
    \ifnum#1>0
        \setcounter{homeworkProblemCounter}{#1}
    \fi
    \section{Problem \arabic{homeworkProblemCounter}}
    \setcounter{partCounter}{1}
    \enterProblemHeader{homeworkProblemCounter}
}{
    \exitProblemHeader{homeworkProblemCounter}
}

%
% Homework Details
%   - Title
%   - Due date
%   - Class
%   - Section/Time
%   - Instructor
%   - Author
%

\newcommand{\hmwkTitle}{Homework\ \#1}
\newcommand{\hmwkDueDate}{Apr 13, 2024}
\newcommand{\hmwkClass}{MATH 188}
\newcommand{\hmwkClassInstructor}{Professor Kunnawalkam Elayavalli}
\newcommand{\hmwkAuthorName}{\textbf{Ray Tsai}}
\newcommand{\hmwkPID}{A16848188}

%
% Title Page
%

\title{
    \vspace{2in}
    \textmd{\textbf{\hmwkClass:\ \hmwkTitle}}\\
    \normalsize\vspace{0.1in}\small{Due\ on\ \hmwkDueDate\ at 23:59pm}\\
    \vspace{0.1in}\large{\textit{\hmwkClassInstructor}} \\
    \vspace{3in}
}

\author{
  \hmwkAuthorName \\
  \vspace{0.1in}\small\hmwkPID
}
\date{}

\renewcommand{\part}[1]{\textbf{\large Part \Alph{partCounter}}\stepcounter{partCounter}\\}

%
% Various Helper Commands
%

% Useful for algorithms
\newcommand{\alg}[1]{\textsc{\bfseries \footnotesize #1}}

% For derivatives
\newcommand{\deriv}[1]{\frac{\mathrm{d}}{\mathrm{d}x} (#1)}

% For partial derivatives
\newcommand{\pderiv}[2]{\frac{\partial}{\partial #1} (#2)}

% Integral dx
\newcommand{\dx}{\mathrm{d}x}

% Probability commands: Expectation, Variance, Covariance, Bias
\newcommand{\Var}{\mathrm{Var}}
\newcommand{\Cov}{\mathrm{Cov}}
\newcommand{\Bias}{\mathrm{Bias}}
\newcommand*{\Z}{\mathbb{Z}}
\newcommand*{\Q}{\mathbb{Q}}
\newcommand*{\R}{\mathbb{R}}
\newcommand*{\C}{\mathbb{C}}
\newcommand*{\N}{\mathbb{N}}
\newcommand*{\p}{\mathds{P}}
\newcommand*{\E}{\mathds{E}}

\begin{document}

\maketitle

\pagebreak

\begin{homeworkProblem}
  Find a closed formula for the following recurrence relation:
  \begin{gather*}
    a_0 = 1, \quad a_1 = 1, \quad a_2 = 2, \\
    a_n = 5a_{n-1} - 8a_{n-2} + 4a_{n-3} \quad (n \geq 3).
  \end{gather*}

  \begin{proof}
    The characteristic polynomial of this recurrence relation is defined to be
    \[
      t^3 - 5t^2 + 8t - 4 = (t - 1)(t - 2)^2,
    \]
    which has roots $t = 1, 2$. Note that $2$ is a repeated root, and thus 
    \[
      a_n = \alpha_1 + \alpha_22^n + \alpha_3n2^n.
    \]
    Solving the system of equations
    \[
      \begin{cases}
        1 = \alpha_1 + \alpha_2 \\
        1 = \alpha_1 + 2\alpha_2 + 2\alpha_3 \\
        2 = \alpha_1 + 4\alpha_2 + 8\alpha_3
      \end{cases},
    \]
    we get
    \[
      a_n = 2 - 2^n + n2^{n - 1}.
    \]
  \end{proof}
\end{homeworkProblem}

\newpage

\begin{homeworkProblem}
  Let $r_1, \ldots, r_d$ be distinct numbers. Show that the determinant of the $d \times d$ matrix
  $(r_i^{j-1})_{i,j=1,\ldots,d}$ is nonzero (interpret $0^0 = 1$). Explain why this implies that the
  sequences $(r_1^n)_{n\geq0}, \ldots, (r_d^n)_{n\geq0}$ are linearly independent.

  \begin{proof}
    Given numbers $x_1, x_2, \dots, x_d$, define
    \[
      M(x_1, x_2, \dots, x_d) = \begin{bmatrix}
        1 & x_1 & x_1^2 & \cdots & x_1^{d - 1} \\
        1 & x_2 & x_2^2 & \cdots & x_2^{d - 1} \\
        \vdots & \vdots & \vdots & \ddots & \vdots \\
        1 & x_{d} & x_{d}^2 & \cdots & x_{d}^{d - 1} \\
      \end{bmatrix}.
    \]
    We first show by induction on $d$ that, $$\det M(x_1, x_2, \dots, x_d) = \prod_{1 \leq i < j
    \leq d} (x_j - x_i),$$ for all $d \geq 2$. We already know.
    \[
      \det M(x_1, x_2) = \det \begin{bmatrix}
        1 & x_1 \\
        1 & x_2
      \end{bmatrix} = x_2 - x_1.
    \]
    Suppose $d > 2$. Note that the determinant remains the same after subtracting to each column the
    preceding column scaled by $x_1$. Hence,
    \begin{align*}
      \det M(x_1, x_2, \dots, x_d) &= \det \begin{bmatrix}
        1 & x_1 & x_1^2 & \cdots & x_1^{d - 1} \\
        1 & x_2 & x_2^2 & \cdots & x_2^{d - 1} \\
        \vdots & \vdots & \vdots & \ddots & \vdots \\
        1 & x_{d} & x_{d}^2 & \cdots & x_{d}^{d - 1} \\
      \end{bmatrix} \\
      &= \det \begin{bmatrix}
        1 & 0 & 0 & \cdots & 0 \\
        1 & x_2 - x_1 & x_2(x_2 - x_1) & \cdots & x_2^{d - 2}(x_2 - x_1) \\
        \vdots & \vdots & \vdots & \ddots & \vdots \\
        1 & x_d - x_1 & x_d(x_d - x_1) & \cdots & x_d^{d - 2}(x_d - x_1) \\
      \end{bmatrix} \\
      &= \det \begin{bmatrix}
        x_2 - x_1 & x_2(x_2 - x_1) & \cdots & x_2^{d - 2}(x_2 - x_1) \\
        \vdots & \vdots & \ddots & \vdots \\
        x_d - x_1 & x_d(x_d - x_1) & \cdots & x_d^{d - 2}(x_d - x_1) \\
      \end{bmatrix}.
    \end{align*}
    Since the entries of $i$th row share a common factor $(x_{i + 1} - x_1)$, we may extract them
    from the determinant and get
    \begin{align*}
      \det M(x_1, x_2, \dots, x_d) 
      &= \left(\prod_{1 \leq i \leq d - 1} (x_{i + 1} - x_1)\right)\det M(x_2, x_2, \dots, x_d) \\
      &= \left(\prod_{1 \leq i \leq d - 1} (x_{i + 1} - x_1)\right)\left(\prod_{2 \leq i < j \leq d} (x_j - x_i)\right) = \prod_{1 \leq i < j \leq d} (x_j - x_i),
    \end{align*}
    by induction. Since all $r_i$'s are distinct,
    \[
      \det M(r_1, r_2, \dots, r_d) = \prod_{1 \leq i < j \leq d} (r_j - r_i) \neq 0.
    \]
    But then $(r_1^n)_{0 \leq n < d}, \ldots, (r_d^n)_{0 \leq n < d}$ are linearly independent, so
    $(r_1^n)_{n\geq0}, \ldots, (r_d^n)_{n\geq0}$ are also linearly independent. (\textit{Source
    cited: en.wikipedia.org/wiki/Vandermonde\underline{\hspace{0.5em}}matrix})
  \end{proof}
\end{homeworkProblem}

\newpage

\begin{homeworkProblem}
  Let $(a_n)_{n\geq0}$ be a sequence satisfying a linear recurrence relation whose characteristic
  polynomial is $(t^2 - 1)^d$. Show that there exist polynomials $p(n)$ and $q(n)$ of degree $\leq d
  - 1$ such that
  \[
    a_n = \begin{cases}
      p(n) & \text{if } n \text{ is even} \\
      q(n) & \text{if } n \text{ is odd}
    \end{cases}.
  \]

  \begin{proof}
    Since $(t^2 - 1)^d = (t - 1)^d(t + 1)^d$,
    \begin{align*}
      a_n 
      &= \alpha_0 + \alpha_1n + \dots + \alpha_{d - 1}n^{d - 1} + (-1)^{n}(\beta_0 + \beta_1n + \dots + \beta_{d - 1}n^{d - 1}) \\
      &= \begin{cases}
        \sum_{0 \leq k \leq d - 1} (\alpha_k + \beta_k)n^k & \text{if } n \text{ is even} \\
        \sum_{0 \leq k \leq d - 1} (\alpha_k - \beta_k)n^k & \text{if } n \text{ is odd} \\
      \end{cases}.
    \end{align*}
    The result follows by taking $p(n) = \sum_{0 \leq k \leq d - 1} (\alpha_k + \beta_k)n^k$ and
    $q(n) = \sum_{0 \leq k \leq d - 1} (\alpha_k - \beta_k)n^k$.
  \end{proof}
\end{homeworkProblem}

\newpage

\begin{homeworkProblem}
  \begin{enumerate}[(a)]
    \item Suppose that $(a_n)_{n\geq0}$ and $(a'_n)_{n\geq0}$ both satisfy the same linear
    recurrence relation of order $d$ and that they agree in $d$ consecutive places, i.e., there
    exists $k$ such that $a_k = a'_k$, $a_{k+1} = a'_{k+1}$, $\ldots$, $a_{k+d-1} = a'_{k+d-1}$.
    Show that these sequences are the same.

    \begin{proof}
      By assumption,
      \begin{gather*}
        a_n = c_1a_{n - 1} + c_2a_{n - 2} + \dots + c_da_{n - d} \\
        a'_n = c_1a'_{n - 1} + c_2a'_{n - 2} + \dots + c_da'_{n - d},
      \end{gather*}
      for some $c_1, \ldots, c_d$, with $c_d \neq 0$. By induction, $a_n = a'_n$ for all $n \geq k$,
      so it remains to show the equality also holds true for $n < k$. Rearranging the equations, we
      get
      \begin{gather*}
        a_{n} = \frac{1}{c_d}(a_{n + d} - c_1a_{n + d - 1} - \dots - c_{d - 1}a_{n + 1}) \\
        a'_{n} = \frac{1}{c_d}(a'_{n + d} - c_1a'_{n + d - 1} - \dots - c_{d - 1}a'_{n + 1}),
      \end{gather*}
      so by induction based on the $k$ consecutive terms that both sequences agree we get $a_n =
      a'_n$ for all $n < k$, and this completes the proof.
    \end{proof}

    \item Suppose that $(a_n)_{n\geq0}$ satisfies the linear recurrence relation of order $d$
    \[
      a_n = c_1a_{n-1} + \ldots + c_da_{n-d} \quad \text{for all } n \geq d
    \]
    with $c_d \neq 0$. Show that there is a unique sequence $(b_n)_{n\in\mathbb{Z}}$ (indexed by
    \textit{all} integers) such that $b_n = a_n$ for $n \geq 0$ and such that
    \begin{gather}
      b_n = c_1b_{n-1} + \ldots + c_db_{n-d} \quad \text{for all } n \in \mathbb{Z}.
    \end{gather}
    \begin{proof}
      Given $b_n = a_n$ for $n \geq 0$, define 
      \begin{gather}
        b_{n} = \frac{1}{c_d}(b_{n + d} - c_1b_{n + d - 1} - \dots - c_{d - 1}b_{n + 1}),
      \end{gather}
      for $n < 0$. Rearranging (2), we know $b_n$ follows (1) for $n \in \Z$. Hence, it remains to
      show the uniqueness of $(b_n)$. Suppose there exists $(b'_n)$ such that $b'_n = a_n$ for $n
      \geq 0$ and satisfies the recurrence relation for all $n \in \Z$. We already know $(b_n)$ and
      $(b'_n)$ agree for all nonnegative terms. But then by (2), $(b_n)$ and $(b'_n)$ agree with
      each negative term by backwards induction on negative $n$ based on the first $d$ nonnegative
      terms, so both sequences also agree on the negative terms. Hence, $(b_n) = (b'_n)$ and we are
      done.
    \end{proof}

    \item Consider the Fibonacci sequence $f_0 = 0$, $f_1 = 1$, and $f_n = f_{n-1} + f_{n-2}$. How
    does the negatively indexed Fibonacci sequence relate to the usual one?
    \begin{proof}
      For $n < 0$, $f_n$ is defined as
      \[
        f_{n} = -f_{n + 1} + f_{n + 2}.
      \]
      Define a new sequence $(g_n)_{n \geq 0}$ as $g_n = f_{-n}$. The characteristic polynomial of
      $(g_n)$ is $t^2 + t - 1$, which has roots $r'_1 = \frac{-1 + \sqrt{5}}{2}$ and $r'_2 =
      \frac{-1 - \sqrt{5}}{2}$. Notice that $r'_1 = -r_1$ and $r'_2 = -r_2$, where $r_1, r_2$ are
      the roots of the characteristic polynomial of the Fibonacci sequence. Since $g_0 = 0$ and
      $g_1 = 1$,
      \[
        g_n = \frac{1}{\sqrt{5}}((r'_1)^n + (r'_2)^n) = \frac{(-1)^n}{\sqrt{5}}(r_1^n + r_2^n) = (-1)^nf_n,
      \]
      so $(g_n)$ is just the alternating Fibonacci sequence.
    \end{proof}
  \end{enumerate}
\end{homeworkProblem}

\newpage

\begin{homeworkProblem}
  Let $A_0(x), A_1(x), \ldots$ and $B_0(x), B_1(x), \ldots$ be sequences of formal power series.
  Assume that $\lim\limits_{i \to \infty} A_i(x) = A(x)$ and $\lim\limits_{i \to \infty} B_i(x) =
  B(x)$.

  \begin{enumerate}[(a)]
    \item Prove that $\lim\limits_{i \to \infty} (A_i(x) + B_i(x)) = A(x) + B(x)$. 
    \begin{proof}
      Note that for any $n$, there exists $N_{a_n}, N_{b_n}$ such that $[x^n]A_i(x) = [x^n]A(x)$ and
      $[x^n]B_i(x) = [x^n]B(x)$, for all $i \geq N_n = \max(N_{a_n}, N_{b_n})$. Hence, 
      \[
        [x^n](A_i(x) + B_i(x)) = [x^n]A_i(x) + [x^n]B_i(x) = [x^n]A(x) + [x^n]B(x) = [x^n](A(x) + B(x)),
      \]
      for $i \geq N_n$, and the result follows.
    \end{proof}
    \item Prove that $\lim\limits_{i \to \infty} (A_i(x)B_i(x)) = A(x)B(x)$.
    \begin{proof}
      Note that for any $n$, there exists $N_{a_n}, N_{b_n}$ such that $[x^n]A_i(x) = [x^n]A(x)$ and
      $[x^n]B_i(x) = [x^n]B(x)$, for all $i \geq N_n = \max(N_{a_n}, N_{b_n})$. Given $m \geq 0$,
      take $N = \max(N_0, N_1, \ldots, N_m)$. Then, 
      \[
        [x^m](A_i(x)B_i(x)) = \sum_{k = 0}^m [x^k]A_i(x)[x^{m - k}]B_i(x) = \sum_{k = 0}^m [x^k]A(x)[x^{m - k}]B(x) = [x^m](A(x)B(x)),
      \]
      for $i \geq N$, and the result follows.
    \end{proof}
  \end{enumerate}
\end{homeworkProblem}

\newpage

\begin{homeworkProblem}
  Continuing from Problem 3, how does the statement generalize if the characteristic polynomial is
  $(t^k - 1)^d$?

  \begin{proof}
    Notice $t^k - 1 = (t - 1)(t - \omega)(t - \omega^2)\dots(t - \omega^k)$, where $\omega =
    e^{\frac{2\pi}{k}}$. Hence, for $m = 0, 1, \dots k - 1$, take $p_m(n) = \sum_{i = 1}^k
    \omega^{im}\sum_{j = 0}^{d - 1} \alpha_{i, j}n^j$, which are polynomials of degree at most $d -
    1$. Then,
    \begin{align*}
      a_n 
      &= \sum_{i = 1}^k \omega^{in}\sum_{j = 0}^{d - 1} \alpha_{i, j}n^j \\
      &= \begin{cases}
        p_0(n) & \text{if } n \equiv 0 \pmod k \\
        p_1(n) & \text{if } n \equiv 1 \pmod k \\
        &\vdots \\
        p_{k - 1}(n) & \text{if } n \equiv k - 1 \pmod k \\
      \end{cases}.
    \end{align*}
  \end{proof}
\end{homeworkProblem}

\newpage

\begin{homeworkProblem}
  Let $p$ be a prime number and let $(a_n)_{n\geq0}$ be a sequence such that $a_n \in \mathbb{Z}/p$
  and which satisfies a homogeneous linear recurrence relation. Show that the sequence is in fact
  periodic.

  \begin{proof}
    By assumption,
    \[
      a_n = c_1a_{n - 1} + c_2a_{n - 1} + \dots + c_da_{n - d},
    \]
    for some $c_1, c_2, \dots, c_d \in \Z/p$, $c_d \neq 0$. Since there are only $p^d$ possible
    strings of length $d$, it is guaranteed that some length $d$ string $s_d$ repeats in the first
    $dp^d$ terms. Suppose that $s_d$ initially appeared at $a_k$ and repeated at $a_{k + l}$, that
    is, $a_k = a_{k + l}, a_{k + 1} = a_{k + 1 + l}, \ldots, a_{k + d - 1} = a_{k + d - 1 + l}$.
    Note that $\Z/p$ is closed under taking multiplicative inverse. Hence, by problem 4(a), we have
    $(a_n)_{n \geq 0} = (a_{n + l})_{n \geq 0}$, and thus $(a_n)_{n \geq 0}$ is periodic.
  \end{proof}
\end{homeworkProblem}

\newpage

\begin{homeworkProblem}
  Let $r_1, \ldots, r_{d-1}$ be distinct numbers. Prove that the sequences $\alpha_1 = (r_1^n),
  \ldots, \alpha_{d-1} = (r_{d-1}^n), \alpha_d = (nr_{d-1}^{n - 1})$ are linearly independent by
  showing that the determinant of $(\alpha_{i,j-1})_{i,j=1,\ldots,d}$ is nonzero (interpret $0^0 =
  1$ and if $r_{d-1} = 0$, interpret $\alpha_{d,0} = 0$).

  \begin{proof}
    Given distinct $d$ numbers $r_1, r_2, \dots, r_{d - 1}$, define
    \[
      M(r_1, r_2, \dots, r_{d - 1}) = \begin{bmatrix}
        1 & r_1 & r_1^2 & \cdots & r_1^{d - 1} \\
        1 & r_2 & r_2^2 & \cdots & r_2^{d - 1} \\
        \vdots & \vdots & \vdots & \ddots & \vdots \\
        1 & r_{d-1} & r_{d-1}^2 & \cdots & r_{d-1}^{d - 1} \\
        0 & 1 & 2r_{d-1} & \cdots & dr_{d-1}^{d - 1} \\
      \end{bmatrix}.
    \]
    We first show by induction on $d$ that, $$\det M(r_1, r_2, \dots, r_{d - 1}) \neq 0,$$ for any
    $d$ distinct numbers, $d \geq 2$. We already know.
    \[
      \det M(r_1) = \det \begin{bmatrix}
        1 & r_1 \\
        0 & 1
      \end{bmatrix} = 1.
    \]
    Suppose we are given distinct $r_1, \ldots, r_{d-1}$, for $d > 2$. Note that the determinant
    remains the same after subtracting to each column the preceding column scaled by $r_1$. Hence,
    \begin{align*}
      \det M(r_1, r_2, \dots, r_{d - 1}) &= \det \begin{bmatrix}
        1 & r_1 & r_1^2 & \cdots & r_1^{d - 1} \\
        1 & r_2 & r_2^2 & \cdots & r_2^{d - 1} \\
        \vdots & \vdots & \vdots & \ddots & \vdots \\
        1 & r_{d-1} & r_{d-1}^2 & \cdots & r_{d-1}^{d - 1} \\
        0 & 1 & 2r_{d-1} & \cdots & dr_{d-1}^{d - 1} \\
      \end{bmatrix} \\
      &= \det \begin{bmatrix}
        1 & 0 & 0 & \cdots & 0 \\
        1 & r_2 - r_1 & r_2(r_2 - r_1) & \cdots & r_2^{d - 2}(r_2 - r_1) \\
        \vdots & \vdots & \vdots & \ddots & \vdots \\
        1 & r_{d - 1} - r_1 & r_{d - 1}(r_{d - 1} - r_1) & \cdots & r_{d - 1}^{d - 2}(r_{d - 1} - r_1) \\
        0 & 1 & 2r_{d-1} - r_1 & \cdots & dr_{d-1}^{d - 1} - (d - 1)r_1r^{d - 2}_{d - 1} \\
      \end{bmatrix} \\
      &= \left(\prod_{1 \leq i \leq d - 1} (r_{i + 1} - r_1)\right) \det \begin{bmatrix}
        1 & r_2 & \cdots & r_2^{d - 2} \\
        \vdots & \vdots & \ddots & \vdots \\
        1 & r_{d - 1} & \cdots & r_{d - 1}^{d - 2} \\
        1 & 2r_{d-1} - r_1 & \cdots & dr_{d-1}^{d - 1} - (d - 1)r_1r^{d - 2}_{d - 1} \\
      \end{bmatrix} \\
      &= \left(\prod_{1 \leq i \leq d - 1} (r_{i + 1} - r_1)\right)\det \begin{bmatrix}
        1 & r_2 & r_2^2 & \cdots & r_2^{d - 2} \\
        \vdots & \vdots & \ddots & \vdots \\
        1 & r_{d - 1} & r^2_{d - 1} & \cdots & r_{d - 1}^{d - 2} \\
        0 & (r_{d-1} - r_1) & 2(r_{d-1} - r_1)r_{d - 1} & \cdots & (d - 1)(r_{d-1} - r_1)r_{d - 1}^{d - 2} \\
      \end{bmatrix} \\
      &= (r_{d-1} - r_1)\left(\prod_{1 \leq i \leq d - 1} (r_{i + 1} - r_1)\right) \det M(r_2, r_3, \dots, r_{d - 1}).
    \end{align*}
    But then all $r_i$'s are distinct, so $\det M(r_1, r_2, \dots, r_{d - 1}) \neq 0$, by induction.
    The induction result implies that $(r_1^n)_{0 \leq n < d}, \ldots, (r_{d - 1}^n)_{0 \leq n < d},
    (nr_{d - 1}^{n - 1})_{0 \leq n < d}$ are linearly independent, so $(r_1^n)_{n\geq0}, \ldots,
    (r_{d - 1}^n)_{n\geq0}, (nr_{d - 1}^{n - 1})_{n\geq0}$ are also linearly independent.
  \end{proof}
\end{homeworkProblem}
\end{document}