\documentclass{article}

\usepackage{fancyhdr}
\usepackage{extramarks}
\usepackage{amsmath}
\usepackage{amsthm}
\usepackage{amsfonts}
\usepackage{tikz}
\usepackage[plain]{algorithm}
\usepackage{algpseudocode}
\usepackage{enumerate}
\usepackage{amssymb}
\usepackage{mathtools}

\usetikzlibrary{automata,positioning}

%
% Basic Document Settings
%

\topmargin=-0.45in
\evensidemargin=0in
\oddsidemargin=0in
\textwidth=6.5in
\textheight=9.0in
\headsep=0.25in

\linespread{1.1}

\pagestyle{fancy}
\lhead{\hmwkAuthorName}
\chead{\hmwkClass:\ \hmwkTitle}
\rhead{\firstxmark}
\lfoot{\lastxmark}
\cfoot{\thepage}

\renewcommand\headrulewidth{0.4pt}
\renewcommand\footrulewidth{0.4pt}

\setlength\parindent{0pt}
\setlength{\parskip}{5pt}

%
% Create Problem Sections
%

\newcommand{\enterProblemHeader}[1]{
    \nobreak\extramarks{}{Problem \arabic{#1} continued on next page\ldots}\nobreak{}
    \nobreak\extramarks{Problem \arabic{#1} (continued)}{Problem \arabic{#1} continued on next page\ldots}\nobreak{}
}

\newcommand{\exitProblemHeader}[1]{
    \nobreak\extramarks{Problem \arabic{#1} (continued)}{Problem \arabic{#1} continued on next page\ldots}\nobreak{}
    \stepcounter{#1}
    \nobreak\extramarks{Problem \arabic{#1}}{}\nobreak{}
}

\setcounter{secnumdepth}{0}
\newcounter{partCounter}
\newcounter{homeworkProblemCounter}
\setcounter{homeworkProblemCounter}{1}
\nobreak\extramarks{Problem \arabic{homeworkProblemCounter}}{}\nobreak{}

%
% Homework Problem Environment
%
% This environment takes an optional argument. When given, it will adjust the
% problem counter. This is useful for when the problems given for your
% assignment aren't sequential. See the last 3 problems of this template for an
% example.
%
\newenvironment{homeworkProblem}[1][-1]{
    \ifnum#1>0
        \setcounter{homeworkProblemCounter}{#1}
    \fi
    \section{Problem \arabic{homeworkProblemCounter}}
    \setcounter{partCounter}{1}
    \enterProblemHeader{homeworkProblemCounter}
}{
    \exitProblemHeader{homeworkProblemCounter}
}

%
% Homework Details
%   - Title
%   - Due date
%   - Class
%   - Section/Time
%   - Instructor
%   - Author
%

\newcommand{\hmwkTitle}{Homework\ \#7}
\newcommand{\hmwkDueDate}{Dec 3, 2024}
\newcommand{\hmwkClass}{MATH 173A}
\newcommand{\hmwkClassInstructor}{Professor Cloninger}
\newcommand{\hmwkAuthorName}{\textbf{Ray Tsai}}
\newcommand{\hmwkPID}{A16848188}

%
% Title Page
%

\title{
    \vspace{2in}
    \textmd{\textbf{\hmwkClass:\ \hmwkTitle}}\\
    \normalsize\vspace{0.1in}\small{Due\ on\ \hmwkDueDate\ at 23:59pm}\\
    \vspace{0.1in}\large{\textit{\hmwkClassInstructor}} \\
    \vspace{3in}
}

\author{
  \hmwkAuthorName \\
  \vspace{0.1in}\small\hmwkPID
}
\date{}

\renewcommand{\part}[1]{\textbf{\large Part \Alph{partCounter}}\stepcounter{partCounter}\\}

%
% Various Helper Commands
%

% Useful for algorithms
\newcommand{\alg}[1]{\textsc{\bfseries \footnotesize #1}}

% For derivatives
\newcommand{\deriv}[1]{\frac{\mathrm{d}}{\mathrm{d}x} (#1)}

% For partial derivatives
\newcommand{\pderiv}[2]{\frac{\partial}{\partial #1} (#2)}

% Integral dx
\newcommand{\dx}{\mathrm{d}x}

% Probability commands: Expectation, Variance, Covariance, Bias
\newcommand{\Var}{\mathrm{Var}}
\newcommand{\Cov}{\mathrm{Cov}}
\newcommand{\Bias}{\mathrm{Bias}}
\newcommand*{\Z}{\mathbb{Z}}
\newcommand*{\Q}{\mathbb{Q}}
\newcommand*{\R}{\mathbb{R}}
\newcommand*{\C}{\mathbb{C}}
\newcommand*{\N}{\mathbb{N}}
\newcommand*{\prob}{\mathds{P}}
\newcommand*{\E}{\mathds{E}}

\begin{document}

\maketitle

\pagebreak

\begin{homeworkProblem}
	Suppose a function $f : \mathbb{R}^d \to \mathbb{R}$ is $L$-smooth with $L = 4$ and satisfies the
	PL-property with parameter $\mu = 2$, i.e., 
	\[
	\frac{1}{2} \|\nabla f(x)\|^2 \geq \mu (f - f^*).
	\]
	Consider the gradient descent method for minimizing $f$. Let $x^*$ be the global minimum and
	suppose $x^{(0)}$ is the initialization such that 
	\[
	\|x^* - x^{(0)}\| \leq 5.
	\]
	Determine the step size $\eta$ and the number of steps needed to satisfy 
	\[
	|f(x^{(t)}) - f(x^*)| \leq 10^{-4}.
	\]

	\begin{proof}
		The step size is $\eta = \frac{1}{L} = \frac{1}{4}$. The convergence rate is
		\begin{align*}
			f(x^{(t)}) - f(x^*) 
			&\leq \left(1 - \frac{\mu}{L}\right)^t [f(x^{(0)}) - f(x^*)] \\
			&= (0.5)^t [f(x^{(0)}) - f(x^*)].
		\end{align*}
		Since $f$ is $L$-smooth and $\|x^* - x^{(0)}\| \leq 5$,
		\[
			\|\nabla f(x^{(0)})\| = \|\nabla f(x^{(0)}) - f(x^{*})\| \leq L\|x^* - x^{(0)}\| \leq 4 \cdot 5 = 20.
		\]
		By the PL-condition,
		\[
			f(x^{(0)}) - f(x^{*}) \leq \frac{1}{2\mu}\|\nabla f(x^{(0)})\|^2 \leq \frac{1}{4} \times 400 = 100.
		\]
		Hence,
		\[
			f(x^{(t)}) - f(x^*) \leq (0.5)^t \times 100 \leq 10^{-4} \implies t \geq 6\log_2 10 \approx 20.
		\]
	\end{proof}
\end{homeworkProblem}

\newpage

\begin{homeworkProblem}
	Consider the following set in $\mathbb{R}^n$ for an integer $s > 0$:

	\[
	B = \{x \in \mathbb{R}^n \mid x_i \geq 0, \text{ for } i = 1, \ldots, n \text{ and } x \text{ has at most } s \text{ nonzeros.}\}.
	\]

	\begin{enumerate}[(a)]
		\item Find an expression for the orthogonal projection of a point $x \in \mathbb{R}^n$ onto $B$
		(No need for justification).

		\begin{proof}
			Let $x^+_i = \max(x_i, 0)$, and let $I_s(x)$ be the index set of the $s$ largest components of
			$x$. Note that $|I_s(x)| = s$. Define projection $\Pi_B(x)$ by sending
			\[
				x_i \mapsto \begin{cases}
					x_i & \text{if } i \in I_s(x^+) \\
					0 & \text{otherwise}.
				\end{cases}
			\]
		\end{proof}

		\item For the function 
		\[
		f(x) = \frac{1}{2}\|Ax - b\|^2,
		\]
		write a projected gradient descent algorithm to solve 
		\[
		\min_{x \in \Omega} f(x)
		\]
		for $\Omega = B$, with $B$ from part (a). You need to specify the gradient formula and the
		projection formula. You do not need to specify the step size for this problem.

		\begin{proof}
			Let $x^{(0)} \in B$, and let $\mu$ be the step size. For $t = 1, \ldots$, 

			\begin{enumerate}[1.]
				\item Set $y^{(t + 1)} = x^{(t)} - \mu \nabla f(x^{(t)}) = x^{(t)} - \mu A^T(Ax^{(t)} - b) = (I - \mu A^TA)x^{(t)} + \mu A^Tb.$
				\item Set $y^{(t + 1)}_i = \max(0, y^{(t + 1)}_i)$ for all $i$.
				\item Calculate $I_s(y^{(t + 1)})$.
				\item Set $x_i^{(t + 1)} = \begin{cases}
					y^{(t + 1)}_i & \text{if } i \in I_s(y^{(t + 1)}) \\
					0 & \text{otherwise}
				\end{cases}.$
			\end{enumerate}
			
		\end{proof}

		\item Consider the function in (b) and suppose 
		\[
		A = \begin{bmatrix} 2 & 0 \\ 0 & 1 \end{bmatrix}, \quad b = \begin{bmatrix} 2 \\ 1 \end{bmatrix}, \quad s = 1
		\]
		for the set $B$ in (a). Does the projected gradient method converge to the global minimizer for
		any initialization $x^{(0)}$ if the step size $\mu \leq \frac{1}{8}$? Justify your answer.

		\begin{proof}
			No. Consider initializations $x^{(0)} = \begin{bmatrix} 1 \\ 0 \end{bmatrix}$ and $x^{(0)} =
			\begin{bmatrix} 0 \\ 1 \end{bmatrix}$. 

			\textbf{Case 1: $x^{(0)} = \begin{bmatrix} 1 \\ 0 \end{bmatrix}$.}

			Following the steps in (b),
			\[
				y^{(1)} = \begin{bmatrix} 1 \\ \mu \end{bmatrix}.
			\]
			Since $\mu \leq 1$, $x^{(1)} = \begin{bmatrix} 1 \\ 0 \end{bmatrix}$, and thus the algorithm
			converges to $\begin{bmatrix} 1 \\ 0 \end{bmatrix}$.

			\textbf{Case 2: $x^{(0)} = \begin{bmatrix} 0 \\ 1 \end{bmatrix}$.}

			Following the steps in (b),
			\[
				y^{(1)} = \begin{bmatrix} 4\mu \\ 1 \end{bmatrix}.
			\]
			Since $4\mu \leq 0.5 \leq 1$, $x^{(1)} = \begin{bmatrix} 0 \\ 1 \end{bmatrix}$, and thus the algorithm
			converges to $\begin{bmatrix} 0 \\ 1 \end{bmatrix}$.
		\end{proof}

		But then $f(1, 0) = 0.5$ and $f(0, 1) = 2$, so the algorithm does converge to the global
		minimum for all initializations.
	\end{enumerate}
\end{homeworkProblem}
\end{document}