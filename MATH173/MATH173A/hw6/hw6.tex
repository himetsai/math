\documentclass{article}

\usepackage{fancyhdr}
\usepackage{extramarks}
\usepackage{amsmath}
\usepackage{amsthm}
\usepackage{amsfonts}
\usepackage{tikz}
\usepackage[plain]{algorithm}
\usepackage{algpseudocode}
\usepackage{enumerate}
\usepackage{amssymb}
\usepackage{mathtools}

\usetikzlibrary{automata,positioning}

%
% Basic Document Settings
%

\topmargin=-0.45in
\evensidemargin=0in
\oddsidemargin=0in
\textwidth=6.5in
\textheight=9.0in
\headsep=0.25in

\linespread{1.1}

\pagestyle{fancy}
\lhead{\hmwkAuthorName}
\chead{\hmwkClass:\ \hmwkTitle}
\rhead{\firstxmark}
\lfoot{\lastxmark}
\cfoot{\thepage}

\renewcommand\headrulewidth{0.4pt}
\renewcommand\footrulewidth{0.4pt}

\setlength\parindent{0pt}
\setlength{\parskip}{5pt}

%
% Create Problem Sections
%

\newcommand{\enterProblemHeader}[1]{
    \nobreak\extramarks{}{Problem \arabic{#1} continued on next page\ldots}\nobreak{}
    \nobreak\extramarks{Problem \arabic{#1} (continued)}{Problem \arabic{#1} continued on next page\ldots}\nobreak{}
}

\newcommand{\exitProblemHeader}[1]{
    \nobreak\extramarks{Problem \arabic{#1} (continued)}{Problem \arabic{#1} continued on next page\ldots}\nobreak{}
    \stepcounter{#1}
    \nobreak\extramarks{Problem \arabic{#1}}{}\nobreak{}
}

\setcounter{secnumdepth}{0}
\newcounter{partCounter}
\newcounter{homeworkProblemCounter}
\setcounter{homeworkProblemCounter}{1}
\nobreak\extramarks{Problem \arabic{homeworkProblemCounter}}{}\nobreak{}

%
% Homework Problem Environment
%
% This environment takes an optional argument. When given, it will adjust the
% problem counter. This is useful for when the problems given for your
% assignment aren't sequential. See the last 3 problems of this template for an
% example.
%
\newenvironment{homeworkProblem}[1][-1]{
    \ifnum#1>0
        \setcounter{homeworkProblemCounter}{#1}
    \fi
    \section{Problem \arabic{homeworkProblemCounter}}
    \setcounter{partCounter}{1}
    \enterProblemHeader{homeworkProblemCounter}
}{
    \exitProblemHeader{homeworkProblemCounter}
}

%
% Homework Details
%   - Title
%   - Due date
%   - Class
%   - Section/Time
%   - Instructor
%   - Author
%

\newcommand{\hmwkTitle}{Homework\ \#6}
\newcommand{\hmwkDueDate}{Nov 26, 2024}
\newcommand{\hmwkClass}{MATH 173A}
\newcommand{\hmwkClassInstructor}{Professor Cloninger}
\newcommand{\hmwkAuthorName}{\textbf{Ray Tsai}}
\newcommand{\hmwkPID}{A16848188}

%
% Title Page
%

\title{
    \vspace{2in}
    \textmd{\textbf{\hmwkClass:\ \hmwkTitle}}\\
    \normalsize\vspace{0.1in}\small{Due\ on\ \hmwkDueDate\ at 23:59pm}\\
    \vspace{0.1in}\large{\textit{\hmwkClassInstructor}} \\
    \vspace{3in}
}

\author{
  \hmwkAuthorName \\
  \vspace{0.1in}\small\hmwkPID
}
\date{}

\renewcommand{\part}[1]{\textbf{\large Part \Alph{partCounter}}\stepcounter{partCounter}\\}

%
% Various Helper Commands
%

% Useful for algorithms
\newcommand{\alg}[1]{\textsc{\bfseries \footnotesize #1}}

% For derivatives
\newcommand{\deriv}[1]{\frac{\mathrm{d}}{\mathrm{d}x} (#1)}

% For partial derivatives
\newcommand{\pderiv}[2]{\frac{\partial}{\partial #1} (#2)}

% Integral dx
\newcommand{\dx}{\mathrm{d}x}

% Probability commands: Expectation, Variance, Covariance, Bias
\newcommand{\Var}{\mathrm{Var}}
\newcommand{\Cov}{\mathrm{Cov}}
\newcommand{\Bias}{\mathrm{Bias}}
\newcommand*{\Z}{\mathbb{Z}}
\newcommand*{\Q}{\mathbb{Q}}
\newcommand*{\R}{\mathbb{R}}
\newcommand*{\C}{\mathbb{C}}
\newcommand*{\N}{\mathbb{N}}
\newcommand*{\prob}{\mathds{P}}
\newcommand*{\E}{\mathds{E}}

\begin{document}

\maketitle

\pagebreak

\begin{homeworkProblem}
	Perform the conjugate gradient method by hand on the problem
	\begin{align*}
	\Phi(x) = \frac{1}{2} x^T \begin{bmatrix} 2 & 0 \\0 & 1\end{bmatrix}x \quad - \quad
	\sum_{i=1}^2 x_i,
	\end{align*}
	where $x\in\mathbb{R}^2$. Perform the algorithm either using version 0 or 1, where the conjugate
	directions are initialized and chosen algorithmically.

	\begin{proof}
		Let $A = \begin{bmatrix} 2 & 0 \\0 & 1\end{bmatrix}, b = \begin{bmatrix}
			1 \\ 1
		\end{bmatrix}$ and we have
		\[
			\Phi(x) = \frac{1}{2} x^TAx - b^Tx,
		\]
		\textbf{Initialization}:
		\[
			x^{(0)} = \begin{bmatrix}
				0 \\ 0
			\end{bmatrix}, \quad r_0 = Ax^{(0)} - b = \begin{bmatrix}
				-1 \\ -1
			\end{bmatrix}, \quad p_0 = -r_0 = \begin{bmatrix}
				1 \\ 1
			\end{bmatrix}.
		\]
		\textbf{Iteration 1}:
		\begin{gather*}
			\alpha_0 = \frac{r_0^Tr_0}{p_0^TAp_0} = \frac{2}{3}, \\
			x^{(1)} = x^{(0)} + \alpha_0p_0 = \begin{bmatrix}
				\frac{2}{3} \\ \frac{2}{3}
			\end{bmatrix}, \\
			r_1 = r_0 + \alpha_0Ap_0 = \begin{bmatrix}
				-1 \\ -1
			\end{bmatrix} +\frac{2}{3} \begin{bmatrix}
				2 \\ 1
			\end{bmatrix} = \begin{bmatrix}
				\frac{1}{3} \\ -\frac{1}{3}
			\end{bmatrix}, \\
			\beta_{1} = \frac{r_1^Tr_1}{r_0^Tr_0} = \frac{1}{9}, \\
			p_1 = -r_1 + \beta_1p_0 = \begin{bmatrix}
				-\frac{2}{9} \\ \frac{4}{9}.
			\end{bmatrix}
		\end{gather*}
		\textbf{Iteration 2}:
		\begin{gather*}
			\alpha_1 = \frac{r_1^Tr_1}{p_1^TAp_1} = \frac{3}{4}, \\
			x^{(2)} = x^{(1)} + \alpha_1p_1 = \begin{bmatrix}
				\frac{1}{2} \\ 1
			\end{bmatrix}, \\
			r_2 = r_1 + \alpha_1Ap_1 = \begin{bmatrix}
				\frac{1}{3} \\ -\frac{1}{3}
			\end{bmatrix} + \frac{3}{4} \begin{bmatrix}
				-\frac{4}{9} \\ \frac{4}{9}
			\end{bmatrix} = \begin{bmatrix}
				0 \\ 0
			\end{bmatrix}, \\
			\beta_{2} = 0, \\
			p_1 = 0
		\end{gather*}
		Thus, the conjugate gradient method converges to the solution $x^* = \begin{bmatrix}
			\frac{1}{2} \\ 1
		\end{bmatrix}$ in 2 iterations.
	\end{proof}
\end{homeworkProblem}

\newpage

\begin{homeworkProblem}
	Here, we will prove the inequality used in class to prove fast convergence for strongly convex
	functions. Let $F(x)$ be a strongly convex function with constant $c$. Our goal is to show
	\begin{align}\label{eq:stcon}
	F(x) - F(x^*) \le \frac{1}{2c} \|\nabla F(x)\|^2 \qquad \textnormal{for all } x\in\mathbb{R}^d.
	\end{align}
	\begin{enumerate}[(a)]
	\item Fix $x\in\mathbb{R}^d$ and define the quadratic function
	\begin{align*}
	q(y) = F(x) + \nabla F(x)^T(y-x) + \frac{c}{2} \|x-y\|^2.
	\end{align*}
	Find the $y^*$ that minimizes $q(y)$.
	\begin{proof}
		\begin{align*}
			\nabla q(y) = \nabla F(x) - c(x - y) = 0 \implies y^* = x - \frac{1}{c}\nabla F(x).
		\end{align*}
	\end{proof}
	\item Show that $q(y^*) = F(x) - \frac{1}{2c}\|\nabla F(x)\|^2$
	\begin{proof}
		\begin{align*}
			q(y^*) = F(x) - \frac{1}{c}\|\nabla F(x)\|^2 + \frac{c}{2}\left\|\frac{1}{c}\nabla F(x)\right\|^2 = F(x) - \frac{1}{2c}\|\nabla F(x)\|^2.
		\end{align*}
	\end{proof}
	\item Use the above to deduce \eqref{eq:stcon}.
	\begin{proof}
		Since $F(x)$ is strongly convex, $F(y) \geq q(y)$ for all $y \in \R^d$, and thus
		\[
			F(x^*) \geq q(x^*) \geq q(y^*) \geq F(x) - \frac{1}{2c}\|\nabla F(x)\|^2 \implies F(x) - F(x^*) \leq \frac{1}{2c}\|\nabla F(x)\|^2.
		\]
	\end{proof}
	\item Explain the proof technique in your own words to demonstrate understanding of what we did.
	\begin{proof}
		The strong convexity property of $F$ yields $F \geq q$. Hence by minimizing $q$ we can obtain a
		lower bound on $F$, and rearranging the equation yields the result.
	\end{proof}
	\end{enumerate}
\end{homeworkProblem}

\newpage

\begin{homeworkProblem}
	Indicate whether the following functions are strongly convex. Justify your answer.
	\begin{enumerate}[(a)]
	\item $f(x) = x$
	\begin{proof}
		Since $\nabla^2 f(x) = 0$, $f$ is not strongly convex, as the Hessian is not positive definite.
	\end{proof}
	\item $f(x) = x^2$
	\begin{proof}
		Since $\nabla^2 f(x) = 2$, $f$ is strongly convex with constant $c = 2$.
	\end{proof}
	\item $f(x) = \log(1+e^x)$
	\begin{proof}
		\begin{align*}
			f'(x) &= \frac{e^x}{1+e^x} = \frac{1}{1+e^{-x}}, \\
			f''(x) &= \frac{e^x}{(1+e^x)^2}.
		\end{align*}
		But then $\inf f''(x) = 0$, so $f$ is not strongly convex.
	\end{proof}
	\end{enumerate}
\end{homeworkProblem}
\end{document}