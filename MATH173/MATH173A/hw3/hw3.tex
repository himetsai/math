\documentclass{article}

\usepackage{fancyhdr}
\usepackage{extramarks}
\usepackage{amsmath}
\usepackage{amsthm}
\usepackage{amsfonts}
\usepackage{tikz}
\usepackage[plain]{algorithm}
\usepackage{algpseudocode}
\usepackage{enumerate}
\usepackage{amssymb}
\usepackage{mathtools}

\usetikzlibrary{automata,positioning}

%
% Basic Document Settings
%

\topmargin=-0.45in
\evensidemargin=0in
\oddsidemargin=0in
\textwidth=6.5in
\textheight=9.0in
\headsep=0.25in

\linespread{1.1}

\pagestyle{fancy}
\lhead{\hmwkAuthorName}
\chead{\hmwkClass:\ \hmwkTitle}
\rhead{\firstxmark}
\lfoot{\lastxmark}
\cfoot{\thepage}

\renewcommand\headrulewidth{0.4pt}
\renewcommand\footrulewidth{0.4pt}

\setlength\parindent{0pt}
\setlength{\parskip}{5pt}

%
% Create Problem Sections
%

\newcommand{\enterProblemHeader}[1]{
    \nobreak\extramarks{}{Problem \arabic{#1} continued on next page\ldots}\nobreak{}
    \nobreak\extramarks{Problem \arabic{#1} (continued)}{Problem \arabic{#1} continued on next page\ldots}\nobreak{}
}

\newcommand{\exitProblemHeader}[1]{
    \nobreak\extramarks{Problem \arabic{#1} (continued)}{Problem \arabic{#1} continued on next page\ldots}\nobreak{}
    \stepcounter{#1}
    \nobreak\extramarks{Problem \arabic{#1}}{}\nobreak{}
}

\setcounter{secnumdepth}{0}
\newcounter{partCounter}
\newcounter{homeworkProblemCounter}
\setcounter{homeworkProblemCounter}{1}
\nobreak\extramarks{Problem \arabic{homeworkProblemCounter}}{}\nobreak{}

%
% Homework Problem Environment
%
% This environment takes an optional argument. When given, it will adjust the
% problem counter. This is useful for when the problems given for your
% assignment aren't sequential. See the last 3 problems of this template for an
% example.
%
\newenvironment{homeworkProblem}[1][-1]{
    \ifnum#1>0
        \setcounter{homeworkProblemCounter}{#1}
    \fi
    \section{Problem \arabic{homeworkProblemCounter}}
    \setcounter{partCounter}{1}
    \enterProblemHeader{homeworkProblemCounter}
}{
    \exitProblemHeader{homeworkProblemCounter}
}

%
% Homework Details
%   - Title
%   - Due date
%   - Class
%   - Section/Time
%   - Instructor
%   - Author
%

\newcommand{\hmwkTitle}{Homework\ \#3}
\newcommand{\hmwkDueDate}{Oct 29, 2024}
\newcommand{\hmwkClass}{MATH 173A}
\newcommand{\hmwkClassInstructor}{Professor Cloninger}
\newcommand{\hmwkAuthorName}{\textbf{Ray Tsai}}
\newcommand{\hmwkPID}{A16848188}

%
% Title Page
%

\title{
    \vspace{2in}
    \textmd{\textbf{\hmwkClass:\ \hmwkTitle}}\\
    \normalsize\vspace{0.1in}\small{Due\ on\ \hmwkDueDate\ at 23:59pm}\\
    \vspace{0.1in}\large{\textit{\hmwkClassInstructor}} \\
    \vspace{3in}
}

\author{
  \hmwkAuthorName \\
  \vspace{0.1in}\small\hmwkPID
}
\date{}

\renewcommand{\part}[1]{\textbf{\large Part \Alph{partCounter}}\stepcounter{partCounter}\\}

%
% Various Helper Commands
%

% Useful for algorithms
\newcommand{\alg}[1]{\textsc{\bfseries \footnotesize #1}}

% For derivatives
\newcommand{\deriv}[1]{\frac{\mathrm{d}}{\mathrm{d}x} (#1)}

% For partial derivatives
\newcommand{\pderiv}[2]{\frac{\partial}{\partial #1} (#2)}

% Integral dx
\newcommand{\dx}{\mathrm{d}x}

% Probability commands: Expectation, Variance, Covariance, Bias
\newcommand{\Var}{\mathrm{Var}}
\newcommand{\Cov}{\mathrm{Cov}}
\newcommand{\Bias}{\mathrm{Bias}}
\newcommand*{\Z}{\mathbb{Z}}
\newcommand*{\Q}{\mathbb{Q}}
\newcommand*{\R}{\mathbb{R}}
\newcommand*{\C}{\mathbb{C}}
\newcommand*{\N}{\mathbb{N}}
\newcommand*{\prob}{\mathds{P}}
\newcommand*{\E}{\mathds{E}}

\begin{document}

\maketitle

\pagebreak

\begin{homeworkProblem}
	Determine whether each function is Lipschitz, and if so find the smallest possible Lipschitz
	constant for the function. For all problems, $\|\cdot \|$ represents the Euclidean norm (2-norm).
	\begin{enumerate}[(a)]
		\item $f:\mathbb{R}^n \rightarrow\mathbb{R}$ for $f(x) = \|x\|$
		\begin{proof}
			By the triangle inequality, 
			\[
				|f(x) - f(y)| = \left| \|x\| - \|y\| \right| \le \|x - y\|.
			\]
			Since the equality holds when $y = 2x \neq 0$, $f$ is Lipschitz, with smallest possible
			Lipschitz constant being $L = 1$.
		\end{proof}
		\item $f:\mathbb{R}^n \rightarrow\mathbb{R}$ for $f(x) = \|x\|^2$
		\begin{proof}
			Note that $f$ is convex and differentiable. Since $\|\nabla f(x)\| = 2\|x\|$ is unbounded, $f$
			is not Lipschitz.
		\end{proof}
		\item $\rho:\mathbb{R}\rightarrow \mathbb{R}$ for $\rho(x) = \frac{1}{1 + e^{-x}}$.
		\begin{proof}
			Since $0 < \rho(x) < 1$ for all $x \in \R$, 
			\[
				|\rho'(x)| = \left|\frac{e^{-x}}{(1 + e^{-x})^2}\right| = |\rho(x)||(1 - \rho(x))| \leq \frac{1}{4},
			\]
			where the equality holds when $x = 0$. Thus, $\rho$ is Lipschitz with the smallest Lipschitz
			constant being $L = \frac{1}{4}$.
		\end{proof}
		\item $f:\mathbb{R}^n\rightarrow \mathbb{R}$ for $f(x) = \rho(w^Tx + b)$ for some weight vector
		$w\in\mathbb{R}^n$, $b\in\mathbb{R}$, and $\rho$ from part (c).
		\begin{proof}
			Since $0 < \rho(x) < 1$ for all $x \in \R$, 
			\[
				\|\nabla f(x)\| = \rho'(x)\|w\| \leq \frac{1}{4}\|w\|,
			\]
			where the equality holds when $x = 0$. Thus, $f$ is Lipschitz with the smallest Lipschitz
			constant being $L = \frac{1}{4}\|w\|$.
		\end{proof}
	\end{enumerate}
\end{homeworkProblem}

\newpage

\begin{homeworkProblem}
	Let $f$ be a convex and differentiable. Let $x^*$ be the global minimum and suppose $x^{(0)}$ is
	the initialization such that $\|x^* - x^{(0)}\| \le 5$.
	\begin{enumerate}[(a)]
	\item Let $f$ be $L-$Lipschitz function where $L=3$. Determine the step size $\mu$ and number of
	steps needed to satisfy
	\begin{align*}
		\left\| f\left(\frac{1}{t}\sum_{s=0}^{t-1} x^{(s)}\right) - f\left( x^*\right) \right\| \le 10^{-4}.
	\end{align*}

	\begin{proof}
		By the rate of convergence theorem, putting $\mu = \frac{5}{3\sqrt{t}}$ yields
		\[
			\left\| f\left(\frac{1}{t}\sum_{s=0}^{t-1} x^{(s)}\right) - f\left( x^*\right) \right\| \le \frac{15}{\sqrt{t}},
		\]
		and thus $t \geq 2.25 \times 10^{10}$ to satisfy the requirement. This makes the step size $\mu
		\leq \frac{1}{90000}$.
	\end{proof}
	\item Let $f$ be $L-$smooth where $L=3$. Determine the step size $\mu$ and number of steps needed
	to satisfy
	\begin{align*}
	\left\| f\left(x^{(t)}\right) - f\left( x^*\right) \right\| \le 10^{-4}.
	\end{align*}

	\begin{proof}
		Pick $\mu = \frac{1}{3}$. Then the gradient descent equation satisfies
		\[
			\left\| f\left(x^{(t)}\right) - f\left( x^*\right) \right\| \le \frac{5^2}{2t\mu} = \frac{75}{2t}.
		\]
		Thus, $t \geq 3.7 5 \times 10^5$ to satisfy the requirement.
	\end{proof}
	\end{enumerate}
\end{homeworkProblem}

\newpage

\begin{homeworkProblem}
	Consider the function $f(x_1, x_2) = (2x_1 - 1)^4 + (x_1 + x_2 - 1)^2$.
	\begin{enumerate}[(a)]
		\item Find the global minimum of $f$, and justify your answer.
		\begin{proof}
			Note that $f(x_1, x_2) \geq 0$ as it is a sum of squares. Hence $(x_1, x_2) = (\frac{1}{2},
			\frac{1}{2})$ is the global minimum of $f$ as it achieves the minimum value of $0$.
		\end{proof}
		\item Starting at $x^{(0)} = (0,0)$, perform gradient descent with backtracking line-search.
		\begin{enumerate}[i.]
			\item Starting at $x^{(0)} = (0,0)$ with learning rate $\mu^{(0)}$, write down the gradient
			descent equation for $x^{(1)}$.
			\begin{proof}
				Since $\nabla f(x) = \begin{bmatrix}
					8(2x_1 - 1)^3 + 2(x_1 + x_2 - 1) \\
					2(x_1 + x_2 - 1)
				\end{bmatrix}$, we have
				\[
					x^{(1)} = x^{(0)} - \mu^{(0)}\nabla f(x^{(0)}) = \begin{bmatrix}
						0 \\ 0
					\end{bmatrix} - \mu^{(0)}\begin{bmatrix}
						-10 \\ -2
					\end{bmatrix} = \mu^{(0)}\begin{bmatrix}
						10 \\ 2
					\end{bmatrix}.
				\]
			\end{proof}
			\item Suppose we want to set $\mu^{(0)}$ using backtracking line search with $\gamma=0.2$ and
			Armijo's condition $f(x^{(1)}) \le f(x^{(0)}) - \mu^{(0)}\gamma \| \nabla f(x^{(0)})\|_2^2$.
			Find a value of $\mu^{(0)}$ that satisfies this.
			\begin{proof}
				We already know $f(x^{(0)}) = 2$ and $\|\nabla f(x^{(0)})\|^2_2 = 104$. Computing
				$f(x^{(1)})$, we have
				\[
					f(x^{(1)}) = \left(4\mu^{(0)} - 1\right)^4 + \left(2\mu^{(0)} + 10\mu^{(0)} - 1\right)^2.
				\]
				By the Armijo's condition, we get
				\[
					\left(4\mu^{(0)} - 1\right)^4 + \left(2\mu^{(0)} + 10\mu^{(0)} - 1\right)^2 \leq 2 - 20.8\mu^{(0)}. 
				\]
				Putting $\mu^{(0)} = 0.01$ satisfies the condition.
			\end{proof}
			\item Suppose instead you started with $\mu^{(0)} = 1$ and an update of $\mu^{(0)} \leftarrow
			\frac{1}{2}\mu^{(0)}$ (i.e. $\beta = \frac{1}{2}$). In the worst case, how many steps of
			back-tracking would you have to take before accepting $x^{(1)}$?

			\begin{proof}
				Define $g(\mu) = 2 - 20.8\mu - \left(4\mu - 1\right)^4 - \left(2\mu + 10\mu - 1\right)^2$.
				We then have
				\[
					g(1) = -220.8, \quad g\left(\frac{1}{2}\right) = -34.4, \quad g\left(\frac{1}{4}\right) = -7.2, \quad g\left(\frac{1}{8}\right) = -0.9125, g\left(\frac{1}{16}\right) = 0.32109375.
				\]
				Thus, we would have to take at most $5$ steps of backtracking before accepting $x^{(1)}$.
			\end{proof}
		\end{enumerate}
	\end{enumerate}
\end{homeworkProblem}
\end{document}